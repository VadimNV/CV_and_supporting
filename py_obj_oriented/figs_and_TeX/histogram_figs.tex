\documentclass[11pt, a4paper]{report}

\usepackage[titletoc]{appendix}
\usepackage{amsmath, amssymb}
\usepackage[normalem]{ulem} % either use this (simple) or
\usepackage{xcolor}
\usepackage{arydshln}
\usepackage[makeroom]{cancel}
\usepackage{lipsum}
\usepackage{graphicx}
\usepackage[english]{babel}
\usepackage[top=2cm, bottom=2.5cm]{geometry}
\usepackage{booktabs} %to make tables look nice
\renewcommand{\arraystretch}{1.2} % to make vertical spacing wider in the tables
\usepackage{multirow} %to make \multirow possible (otherwise only \multicolumn)
\usepackage{cancel} %to strike out part of the equation setting it to 0 or 1, etc.
\usepackage{mathrsfs}%for curly letters use e.g. \mathscr{E}

\listfiles %to see LaTeX version in  .log file

\usepackage{listings} %to incldue Mathematica code
%\usepackage{physics} %includes bracket notation
\lstloadlanguages{Mathematica}

\usepackage{geometry} %to use gnuplot figure as figures 
\usepackage{color}    %to use gnuplot output as figures
\usepackage{anyfontsize} %to specify font of a single word (e.g. in gnuplot)
\usepackage{pgf} %allows Python->Tex figures look native. http://bkanuka.com/articles/native-latex-plots/


\usepackage{caption}
\usepackage{subcaption} %to have side-by-side subfigures
\usepackage{wrapfig} %to wrap text around figures

\usepackage[sort&compress]{natbib} % to have [1-3] instead of [1,2,3]

\makeatletter
\newcommand{\rmnum}[1]{\romannumeral #1}
\newcommand{\Rmnum}[1]{\expandafter\@slowromancap\romannumeral #1@}
\makeatother %to have Roman numerals

\DeclareMathOperator\Erfc{erfc} %to type in erfc function
\DeclareMathOperator\Erf{erf} %to type in Erf function
\DeclareMathOperator\IP{IP} %to type in Ionization Potential function
\DeclareMathOperator\EA{EA} %to type in Electron affinity
\DeclareMathOperator\Min{Min} %to type in Min[... , ...]

%\newcommand*{\MyPath}{../figs_in_TeX} %path to folder where my TeX file is

\begin{document}



\title{\textbf{Understanding QEq in BaTiO$_3$}}
\author{Vadim Nemytov}
%\date{\textit{Submitted on:} May 30, 2015 \\[1cm]
%\textit{Early stage assessment report submitted in partial fulfilment of the requirements for the degree of Doctor of Philosophy of Imperial College London.}}

\maketitle

%\chapter*{Abstract}

\tableofcontents
%\listoftables
\chapter*{Introduction}
\section{Change in charge on an ion vs. local forces, properties}


1) Figure \ref{on_site_Ferr_vs_dq} shows how a change in charge on a Ti ion relates to force on that same ion (recall, charges vary only on Ti species). The exact quantity represented by the y-axis is
\begin{align*}
y_{i,I} &\equiv  \frac{\sqrt{\sum_{\alpha = x,y,z}\left(F_{i,I}^{\alpha}(\{q_l\})-F_{i,I}^{\alpha}(\bar{q}_{\text{Ba}},\bar{q}_{\text{Ti}},\bar{q}_{\text{O}})\right)^2}}{\sum_{J}\sum_{j}\sqrt{\sum_{\alpha = x,y,z}\left(F_{j,J}^{\alpha}(\{q_l\})\right)^2}}
\end{align*}
where $i,j\in \{\text{Ti}_1,...,\text{Ti}_{27}\}$ and $I,J\in\{\text{MD}_1,...,\text{MD}_{10}\}$. 

\begin{figure}[h!]
\centering
	\begin{subfigure}[b]{0.45\textwidth}
	\hspace*{-0.4cm}
	%% Creator: Matplotlib, PGF backend
%%
%% To include the figure in your LaTeX document, write
%%   \input{<filename>.pgf}
%%
%% Make sure the required packages are loaded in your preamble
%%   \usepackage{pgf}
%%
%% Figures using additional raster images can only be included by \input if
%% they are in the same directory as the main LaTeX file. For loading figures
%% from other directories you can use the `import` package
%%   \usepackage{import}
%% and then include the figures with
%%   \import{<path to file>}{<filename>.pgf}
%%
%% Matplotlib used the following preamble
%%   \usepackage[utf8x]{inputenc}
%%   \usepackage[T1]{fontenc}
%%
\begingroup%
\makeatletter%
\begin{pgfpicture}%
\pgfpathrectangle{\pgfpointorigin}{\pgfqpoint{2.538459in}{3.060408in}}%
\pgfusepath{use as bounding box, clip}%
\begin{pgfscope}%
\pgfsetbuttcap%
\pgfsetmiterjoin%
\definecolor{currentfill}{rgb}{1.000000,1.000000,1.000000}%
\pgfsetfillcolor{currentfill}%
\pgfsetlinewidth{0.000000pt}%
\definecolor{currentstroke}{rgb}{1.000000,1.000000,1.000000}%
\pgfsetstrokecolor{currentstroke}%
\pgfsetdash{}{0pt}%
\pgfpathmoveto{\pgfqpoint{0.000000in}{0.000000in}}%
\pgfpathlineto{\pgfqpoint{2.538459in}{0.000000in}}%
\pgfpathlineto{\pgfqpoint{2.538459in}{3.060408in}}%
\pgfpathlineto{\pgfqpoint{0.000000in}{3.060408in}}%
\pgfpathclose%
\pgfusepath{fill}%
\end{pgfscope}%
\begin{pgfscope}%
\pgfsetbuttcap%
\pgfsetmiterjoin%
\definecolor{currentfill}{rgb}{1.000000,1.000000,1.000000}%
\pgfsetfillcolor{currentfill}%
\pgfsetlinewidth{0.000000pt}%
\definecolor{currentstroke}{rgb}{0.000000,0.000000,0.000000}%
\pgfsetstrokecolor{currentstroke}%
\pgfsetstrokeopacity{0.000000}%
\pgfsetdash{}{0pt}%
\pgfpathmoveto{\pgfqpoint{0.444137in}{0.319877in}}%
\pgfpathlineto{\pgfqpoint{1.988155in}{0.319877in}}%
\pgfpathlineto{\pgfqpoint{1.988155in}{2.925408in}}%
\pgfpathlineto{\pgfqpoint{0.444137in}{2.925408in}}%
\pgfpathclose%
\pgfusepath{fill}%
\end{pgfscope}%
\begin{pgfscope}%
\pgfpathrectangle{\pgfqpoint{0.444137in}{0.319877in}}{\pgfqpoint{1.544018in}{2.605531in}} %
\pgfusepath{clip}%
\pgfsys@transformshift{0.444137in}{0.319877in}%
\pgftext[left,bottom]{\pgfimage[interpolate=true,width=1.550000in,height=2.610000in]{Ferr_vs_dq_Ti_100K-img0.png}}%
\end{pgfscope}%
\begin{pgfscope}%
\pgfpathrectangle{\pgfqpoint{0.444137in}{0.319877in}}{\pgfqpoint{1.544018in}{2.605531in}} %
\pgfusepath{clip}%
\pgfsetbuttcap%
\pgfsetroundjoin%
\definecolor{currentfill}{rgb}{1.000000,0.752941,0.796078}%
\pgfsetfillcolor{currentfill}%
\pgfsetlinewidth{1.003750pt}%
\definecolor{currentstroke}{rgb}{1.000000,0.752941,0.796078}%
\pgfsetstrokecolor{currentstroke}%
\pgfsetdash{}{0pt}%
\pgfpathmoveto{\pgfqpoint{1.010276in}{1.059870in}}%
\pgfpathcurveto{\pgfqpoint{1.021327in}{1.059870in}}{\pgfqpoint{1.031926in}{1.064260in}}{\pgfqpoint{1.039739in}{1.072074in}}%
\pgfpathcurveto{\pgfqpoint{1.047553in}{1.079887in}}{\pgfqpoint{1.051943in}{1.090486in}}{\pgfqpoint{1.051943in}{1.101537in}}%
\pgfpathcurveto{\pgfqpoint{1.051943in}{1.112587in}}{\pgfqpoint{1.047553in}{1.123186in}}{\pgfqpoint{1.039739in}{1.130999in}}%
\pgfpathcurveto{\pgfqpoint{1.031926in}{1.138813in}}{\pgfqpoint{1.021327in}{1.143203in}}{\pgfqpoint{1.010276in}{1.143203in}}%
\pgfpathcurveto{\pgfqpoint{0.999226in}{1.143203in}}{\pgfqpoint{0.988627in}{1.138813in}}{\pgfqpoint{0.980814in}{1.130999in}}%
\pgfpathcurveto{\pgfqpoint{0.973000in}{1.123186in}}{\pgfqpoint{0.968610in}{1.112587in}}{\pgfqpoint{0.968610in}{1.101537in}}%
\pgfpathcurveto{\pgfqpoint{0.968610in}{1.090486in}}{\pgfqpoint{0.973000in}{1.079887in}}{\pgfqpoint{0.980814in}{1.072074in}}%
\pgfpathcurveto{\pgfqpoint{0.988627in}{1.064260in}}{\pgfqpoint{0.999226in}{1.059870in}}{\pgfqpoint{1.010276in}{1.059870in}}%
\pgfpathclose%
\pgfusepath{stroke,fill}%
\end{pgfscope}%
\begin{pgfscope}%
\pgfpathrectangle{\pgfqpoint{0.444137in}{0.319877in}}{\pgfqpoint{1.544018in}{2.605531in}} %
\pgfusepath{clip}%
\pgfsetbuttcap%
\pgfsetroundjoin%
\definecolor{currentfill}{rgb}{1.000000,0.752941,0.796078}%
\pgfsetfillcolor{currentfill}%
\pgfsetlinewidth{1.003750pt}%
\definecolor{currentstroke}{rgb}{1.000000,0.752941,0.796078}%
\pgfsetstrokecolor{currentstroke}%
\pgfsetdash{}{0pt}%
\pgfpathmoveto{\pgfqpoint{1.113211in}{0.904886in}}%
\pgfpathcurveto{\pgfqpoint{1.124261in}{0.904886in}}{\pgfqpoint{1.134860in}{0.909276in}}{\pgfqpoint{1.142674in}{0.917090in}}%
\pgfpathcurveto{\pgfqpoint{1.150487in}{0.924903in}}{\pgfqpoint{1.154878in}{0.935502in}}{\pgfqpoint{1.154878in}{0.946552in}}%
\pgfpathcurveto{\pgfqpoint{1.154878in}{0.957602in}}{\pgfqpoint{1.150487in}{0.968202in}}{\pgfqpoint{1.142674in}{0.976015in}}%
\pgfpathcurveto{\pgfqpoint{1.134860in}{0.983829in}}{\pgfqpoint{1.124261in}{0.988219in}}{\pgfqpoint{1.113211in}{0.988219in}}%
\pgfpathcurveto{\pgfqpoint{1.102161in}{0.988219in}}{\pgfqpoint{1.091562in}{0.983829in}}{\pgfqpoint{1.083748in}{0.976015in}}%
\pgfpathcurveto{\pgfqpoint{1.075935in}{0.968202in}}{\pgfqpoint{1.071544in}{0.957602in}}{\pgfqpoint{1.071544in}{0.946552in}}%
\pgfpathcurveto{\pgfqpoint{1.071544in}{0.935502in}}{\pgfqpoint{1.075935in}{0.924903in}}{\pgfqpoint{1.083748in}{0.917090in}}%
\pgfpathcurveto{\pgfqpoint{1.091562in}{0.909276in}}{\pgfqpoint{1.102161in}{0.904886in}}{\pgfqpoint{1.113211in}{0.904886in}}%
\pgfpathclose%
\pgfusepath{stroke,fill}%
\end{pgfscope}%
\begin{pgfscope}%
\pgfpathrectangle{\pgfqpoint{0.444137in}{0.319877in}}{\pgfqpoint{1.544018in}{2.605531in}} %
\pgfusepath{clip}%
\pgfsetbuttcap%
\pgfsetroundjoin%
\definecolor{currentfill}{rgb}{1.000000,0.752941,0.796078}%
\pgfsetfillcolor{currentfill}%
\pgfsetlinewidth{1.003750pt}%
\definecolor{currentstroke}{rgb}{1.000000,0.752941,0.796078}%
\pgfsetstrokecolor{currentstroke}%
\pgfsetdash{}{0pt}%
\pgfpathmoveto{\pgfqpoint{1.216146in}{0.879560in}}%
\pgfpathcurveto{\pgfqpoint{1.227196in}{0.879560in}}{\pgfqpoint{1.237795in}{0.883950in}}{\pgfqpoint{1.245608in}{0.891763in}}%
\pgfpathcurveto{\pgfqpoint{1.253422in}{0.899577in}}{\pgfqpoint{1.257812in}{0.910176in}}{\pgfqpoint{1.257812in}{0.921226in}}%
\pgfpathcurveto{\pgfqpoint{1.257812in}{0.932276in}}{\pgfqpoint{1.253422in}{0.942875in}}{\pgfqpoint{1.245608in}{0.950689in}}%
\pgfpathcurveto{\pgfqpoint{1.237795in}{0.958503in}}{\pgfqpoint{1.227196in}{0.962893in}}{\pgfqpoint{1.216146in}{0.962893in}}%
\pgfpathcurveto{\pgfqpoint{1.205095in}{0.962893in}}{\pgfqpoint{1.194496in}{0.958503in}}{\pgfqpoint{1.186683in}{0.950689in}}%
\pgfpathcurveto{\pgfqpoint{1.178869in}{0.942875in}}{\pgfqpoint{1.174479in}{0.932276in}}{\pgfqpoint{1.174479in}{0.921226in}}%
\pgfpathcurveto{\pgfqpoint{1.174479in}{0.910176in}}{\pgfqpoint{1.178869in}{0.899577in}}{\pgfqpoint{1.186683in}{0.891763in}}%
\pgfpathcurveto{\pgfqpoint{1.194496in}{0.883950in}}{\pgfqpoint{1.205095in}{0.879560in}}{\pgfqpoint{1.216146in}{0.879560in}}%
\pgfpathclose%
\pgfusepath{stroke,fill}%
\end{pgfscope}%
\begin{pgfscope}%
\pgfpathrectangle{\pgfqpoint{0.444137in}{0.319877in}}{\pgfqpoint{1.544018in}{2.605531in}} %
\pgfusepath{clip}%
\pgfsetbuttcap%
\pgfsetroundjoin%
\definecolor{currentfill}{rgb}{1.000000,0.752941,0.796078}%
\pgfsetfillcolor{currentfill}%
\pgfsetlinewidth{1.003750pt}%
\definecolor{currentstroke}{rgb}{1.000000,0.752941,0.796078}%
\pgfsetstrokecolor{currentstroke}%
\pgfsetdash{}{0pt}%
\pgfpathmoveto{\pgfqpoint{1.319080in}{0.965900in}}%
\pgfpathcurveto{\pgfqpoint{1.330130in}{0.965900in}}{\pgfqpoint{1.340729in}{0.970290in}}{\pgfqpoint{1.348543in}{0.978104in}}%
\pgfpathcurveto{\pgfqpoint{1.356357in}{0.985917in}}{\pgfqpoint{1.360747in}{0.996516in}}{\pgfqpoint{1.360747in}{1.007567in}}%
\pgfpathcurveto{\pgfqpoint{1.360747in}{1.018617in}}{\pgfqpoint{1.356357in}{1.029216in}}{\pgfqpoint{1.348543in}{1.037029in}}%
\pgfpathcurveto{\pgfqpoint{1.340729in}{1.044843in}}{\pgfqpoint{1.330130in}{1.049233in}}{\pgfqpoint{1.319080in}{1.049233in}}%
\pgfpathcurveto{\pgfqpoint{1.308030in}{1.049233in}}{\pgfqpoint{1.297431in}{1.044843in}}{\pgfqpoint{1.289617in}{1.037029in}}%
\pgfpathcurveto{\pgfqpoint{1.281804in}{1.029216in}}{\pgfqpoint{1.277413in}{1.018617in}}{\pgfqpoint{1.277413in}{1.007567in}}%
\pgfpathcurveto{\pgfqpoint{1.277413in}{0.996516in}}{\pgfqpoint{1.281804in}{0.985917in}}{\pgfqpoint{1.289617in}{0.978104in}}%
\pgfpathcurveto{\pgfqpoint{1.297431in}{0.970290in}}{\pgfqpoint{1.308030in}{0.965900in}}{\pgfqpoint{1.319080in}{0.965900in}}%
\pgfpathclose%
\pgfusepath{stroke,fill}%
\end{pgfscope}%
\begin{pgfscope}%
\pgfpathrectangle{\pgfqpoint{0.444137in}{0.319877in}}{\pgfqpoint{1.544018in}{2.605531in}} %
\pgfusepath{clip}%
\pgfsetbuttcap%
\pgfsetroundjoin%
\definecolor{currentfill}{rgb}{1.000000,0.752941,0.796078}%
\pgfsetfillcolor{currentfill}%
\pgfsetlinewidth{1.003750pt}%
\definecolor{currentstroke}{rgb}{1.000000,0.752941,0.796078}%
\pgfsetstrokecolor{currentstroke}%
\pgfsetdash{}{0pt}%
\pgfpathmoveto{\pgfqpoint{1.422015in}{1.841529in}}%
\pgfpathcurveto{\pgfqpoint{1.433065in}{1.841529in}}{\pgfqpoint{1.443664in}{1.845919in}}{\pgfqpoint{1.451477in}{1.853733in}}%
\pgfpathcurveto{\pgfqpoint{1.459291in}{1.861547in}}{\pgfqpoint{1.463681in}{1.872146in}}{\pgfqpoint{1.463681in}{1.883196in}}%
\pgfpathcurveto{\pgfqpoint{1.463681in}{1.894246in}}{\pgfqpoint{1.459291in}{1.904845in}}{\pgfqpoint{1.451477in}{1.912658in}}%
\pgfpathcurveto{\pgfqpoint{1.443664in}{1.920472in}}{\pgfqpoint{1.433065in}{1.924862in}}{\pgfqpoint{1.422015in}{1.924862in}}%
\pgfpathcurveto{\pgfqpoint{1.410965in}{1.924862in}}{\pgfqpoint{1.400365in}{1.920472in}}{\pgfqpoint{1.392552in}{1.912658in}}%
\pgfpathcurveto{\pgfqpoint{1.384738in}{1.904845in}}{\pgfqpoint{1.380348in}{1.894246in}}{\pgfqpoint{1.380348in}{1.883196in}}%
\pgfpathcurveto{\pgfqpoint{1.380348in}{1.872146in}}{\pgfqpoint{1.384738in}{1.861547in}}{\pgfqpoint{1.392552in}{1.853733in}}%
\pgfpathcurveto{\pgfqpoint{1.400365in}{1.845919in}}{\pgfqpoint{1.410965in}{1.841529in}}{\pgfqpoint{1.422015in}{1.841529in}}%
\pgfpathclose%
\pgfusepath{stroke,fill}%
\end{pgfscope}%
\begin{pgfscope}%
\pgfsetbuttcap%
\pgfsetroundjoin%
\definecolor{currentfill}{rgb}{0.000000,0.000000,0.000000}%
\pgfsetfillcolor{currentfill}%
\pgfsetlinewidth{0.803000pt}%
\definecolor{currentstroke}{rgb}{0.000000,0.000000,0.000000}%
\pgfsetstrokecolor{currentstroke}%
\pgfsetdash{}{0pt}%
\pgfsys@defobject{currentmarker}{\pgfqpoint{0.000000in}{-0.048611in}}{\pgfqpoint{0.000000in}{0.000000in}}{%
\pgfpathmoveto{\pgfqpoint{0.000000in}{0.000000in}}%
\pgfpathlineto{\pgfqpoint{0.000000in}{-0.048611in}}%
\pgfusepath{stroke,fill}%
}%
\begin{pgfscope}%
\pgfsys@transformshift{0.733640in}{0.319877in}%
\pgfsys@useobject{currentmarker}{}%
\end{pgfscope}%
\end{pgfscope}%
\begin{pgfscope}%
\pgftext[x=0.733640in,y=0.222655in,,top]{\rmfamily\fontsize{10.000000}{12.000000}\selectfont \(\displaystyle -0.05\)}%
\end{pgfscope}%
\begin{pgfscope}%
\pgfsetbuttcap%
\pgfsetroundjoin%
\definecolor{currentfill}{rgb}{0.000000,0.000000,0.000000}%
\pgfsetfillcolor{currentfill}%
\pgfsetlinewidth{0.803000pt}%
\definecolor{currentstroke}{rgb}{0.000000,0.000000,0.000000}%
\pgfsetstrokecolor{currentstroke}%
\pgfsetdash{}{0pt}%
\pgfsys@defobject{currentmarker}{\pgfqpoint{0.000000in}{-0.048611in}}{\pgfqpoint{0.000000in}{0.000000in}}{%
\pgfpathmoveto{\pgfqpoint{0.000000in}{0.000000in}}%
\pgfpathlineto{\pgfqpoint{0.000000in}{-0.048611in}}%
\pgfusepath{stroke,fill}%
}%
\begin{pgfscope}%
\pgfsys@transformshift{1.216146in}{0.319877in}%
\pgfsys@useobject{currentmarker}{}%
\end{pgfscope}%
\end{pgfscope}%
\begin{pgfscope}%
\pgftext[x=1.216146in,y=0.222655in,,top]{\rmfamily\fontsize{10.000000}{12.000000}\selectfont \(\displaystyle 0.00\)}%
\end{pgfscope}%
\begin{pgfscope}%
\pgfsetbuttcap%
\pgfsetroundjoin%
\definecolor{currentfill}{rgb}{0.000000,0.000000,0.000000}%
\pgfsetfillcolor{currentfill}%
\pgfsetlinewidth{0.803000pt}%
\definecolor{currentstroke}{rgb}{0.000000,0.000000,0.000000}%
\pgfsetstrokecolor{currentstroke}%
\pgfsetdash{}{0pt}%
\pgfsys@defobject{currentmarker}{\pgfqpoint{0.000000in}{-0.048611in}}{\pgfqpoint{0.000000in}{0.000000in}}{%
\pgfpathmoveto{\pgfqpoint{0.000000in}{0.000000in}}%
\pgfpathlineto{\pgfqpoint{0.000000in}{-0.048611in}}%
\pgfusepath{stroke,fill}%
}%
\begin{pgfscope}%
\pgfsys@transformshift{1.698651in}{0.319877in}%
\pgfsys@useobject{currentmarker}{}%
\end{pgfscope}%
\end{pgfscope}%
\begin{pgfscope}%
\pgftext[x=1.698651in,y=0.222655in,,top]{\rmfamily\fontsize{10.000000}{12.000000}\selectfont \(\displaystyle 0.05\)}%
\end{pgfscope}%
\begin{pgfscope}%
\pgfsetbuttcap%
\pgfsetroundjoin%
\definecolor{currentfill}{rgb}{0.000000,0.000000,0.000000}%
\pgfsetfillcolor{currentfill}%
\pgfsetlinewidth{0.803000pt}%
\definecolor{currentstroke}{rgb}{0.000000,0.000000,0.000000}%
\pgfsetstrokecolor{currentstroke}%
\pgfsetdash{}{0pt}%
\pgfsys@defobject{currentmarker}{\pgfqpoint{-0.048611in}{0.000000in}}{\pgfqpoint{0.000000in}{0.000000in}}{%
\pgfpathmoveto{\pgfqpoint{0.000000in}{0.000000in}}%
\pgfpathlineto{\pgfqpoint{-0.048611in}{0.000000in}}%
\pgfusepath{stroke,fill}%
}%
\begin{pgfscope}%
\pgfsys@transformshift{0.444137in}{0.319877in}%
\pgfsys@useobject{currentmarker}{}%
\end{pgfscope}%
\end{pgfscope}%
\begin{pgfscope}%
\pgftext[x=0.100000in,y=0.272050in,left,base]{\rmfamily\fontsize{10.000000}{12.000000}\selectfont \(\displaystyle 0.00\)}%
\end{pgfscope}%
\begin{pgfscope}%
\pgfsetbuttcap%
\pgfsetroundjoin%
\definecolor{currentfill}{rgb}{0.000000,0.000000,0.000000}%
\pgfsetfillcolor{currentfill}%
\pgfsetlinewidth{0.803000pt}%
\definecolor{currentstroke}{rgb}{0.000000,0.000000,0.000000}%
\pgfsetstrokecolor{currentstroke}%
\pgfsetdash{}{0pt}%
\pgfsys@defobject{currentmarker}{\pgfqpoint{-0.048611in}{0.000000in}}{\pgfqpoint{0.000000in}{0.000000in}}{%
\pgfpathmoveto{\pgfqpoint{0.000000in}{0.000000in}}%
\pgfpathlineto{\pgfqpoint{-0.048611in}{0.000000in}}%
\pgfusepath{stroke,fill}%
}%
\begin{pgfscope}%
\pgfsys@transformshift{0.444137in}{0.802383in}%
\pgfsys@useobject{currentmarker}{}%
\end{pgfscope}%
\end{pgfscope}%
\begin{pgfscope}%
\pgftext[x=0.100000in,y=0.754555in,left,base]{\rmfamily\fontsize{10.000000}{12.000000}\selectfont \(\displaystyle 0.05\)}%
\end{pgfscope}%
\begin{pgfscope}%
\pgfsetbuttcap%
\pgfsetroundjoin%
\definecolor{currentfill}{rgb}{0.000000,0.000000,0.000000}%
\pgfsetfillcolor{currentfill}%
\pgfsetlinewidth{0.803000pt}%
\definecolor{currentstroke}{rgb}{0.000000,0.000000,0.000000}%
\pgfsetstrokecolor{currentstroke}%
\pgfsetdash{}{0pt}%
\pgfsys@defobject{currentmarker}{\pgfqpoint{-0.048611in}{0.000000in}}{\pgfqpoint{0.000000in}{0.000000in}}{%
\pgfpathmoveto{\pgfqpoint{0.000000in}{0.000000in}}%
\pgfpathlineto{\pgfqpoint{-0.048611in}{0.000000in}}%
\pgfusepath{stroke,fill}%
}%
\begin{pgfscope}%
\pgfsys@transformshift{0.444137in}{1.284889in}%
\pgfsys@useobject{currentmarker}{}%
\end{pgfscope}%
\end{pgfscope}%
\begin{pgfscope}%
\pgftext[x=0.100000in,y=1.237061in,left,base]{\rmfamily\fontsize{10.000000}{12.000000}\selectfont \(\displaystyle 0.10\)}%
\end{pgfscope}%
\begin{pgfscope}%
\pgfsetbuttcap%
\pgfsetroundjoin%
\definecolor{currentfill}{rgb}{0.000000,0.000000,0.000000}%
\pgfsetfillcolor{currentfill}%
\pgfsetlinewidth{0.803000pt}%
\definecolor{currentstroke}{rgb}{0.000000,0.000000,0.000000}%
\pgfsetstrokecolor{currentstroke}%
\pgfsetdash{}{0pt}%
\pgfsys@defobject{currentmarker}{\pgfqpoint{-0.048611in}{0.000000in}}{\pgfqpoint{0.000000in}{0.000000in}}{%
\pgfpathmoveto{\pgfqpoint{0.000000in}{0.000000in}}%
\pgfpathlineto{\pgfqpoint{-0.048611in}{0.000000in}}%
\pgfusepath{stroke,fill}%
}%
\begin{pgfscope}%
\pgfsys@transformshift{0.444137in}{1.767394in}%
\pgfsys@useobject{currentmarker}{}%
\end{pgfscope}%
\end{pgfscope}%
\begin{pgfscope}%
\pgftext[x=0.100000in,y=1.719567in,left,base]{\rmfamily\fontsize{10.000000}{12.000000}\selectfont \(\displaystyle 0.15\)}%
\end{pgfscope}%
\begin{pgfscope}%
\pgfsetbuttcap%
\pgfsetroundjoin%
\definecolor{currentfill}{rgb}{0.000000,0.000000,0.000000}%
\pgfsetfillcolor{currentfill}%
\pgfsetlinewidth{0.803000pt}%
\definecolor{currentstroke}{rgb}{0.000000,0.000000,0.000000}%
\pgfsetstrokecolor{currentstroke}%
\pgfsetdash{}{0pt}%
\pgfsys@defobject{currentmarker}{\pgfqpoint{-0.048611in}{0.000000in}}{\pgfqpoint{0.000000in}{0.000000in}}{%
\pgfpathmoveto{\pgfqpoint{0.000000in}{0.000000in}}%
\pgfpathlineto{\pgfqpoint{-0.048611in}{0.000000in}}%
\pgfusepath{stroke,fill}%
}%
\begin{pgfscope}%
\pgfsys@transformshift{0.444137in}{2.249900in}%
\pgfsys@useobject{currentmarker}{}%
\end{pgfscope}%
\end{pgfscope}%
\begin{pgfscope}%
\pgftext[x=0.100000in,y=2.202072in,left,base]{\rmfamily\fontsize{10.000000}{12.000000}\selectfont \(\displaystyle 0.20\)}%
\end{pgfscope}%
\begin{pgfscope}%
\pgfsetbuttcap%
\pgfsetroundjoin%
\definecolor{currentfill}{rgb}{0.000000,0.000000,0.000000}%
\pgfsetfillcolor{currentfill}%
\pgfsetlinewidth{0.803000pt}%
\definecolor{currentstroke}{rgb}{0.000000,0.000000,0.000000}%
\pgfsetstrokecolor{currentstroke}%
\pgfsetdash{}{0pt}%
\pgfsys@defobject{currentmarker}{\pgfqpoint{-0.048611in}{0.000000in}}{\pgfqpoint{0.000000in}{0.000000in}}{%
\pgfpathmoveto{\pgfqpoint{0.000000in}{0.000000in}}%
\pgfpathlineto{\pgfqpoint{-0.048611in}{0.000000in}}%
\pgfusepath{stroke,fill}%
}%
\begin{pgfscope}%
\pgfsys@transformshift{0.444137in}{2.732406in}%
\pgfsys@useobject{currentmarker}{}%
\end{pgfscope}%
\end{pgfscope}%
\begin{pgfscope}%
\pgftext[x=0.100000in,y=2.684578in,left,base]{\rmfamily\fontsize{10.000000}{12.000000}\selectfont \(\displaystyle 0.25\)}%
\end{pgfscope}%
\begin{pgfscope}%
\pgfsetrectcap%
\pgfsetmiterjoin%
\pgfsetlinewidth{0.803000pt}%
\definecolor{currentstroke}{rgb}{0.000000,0.000000,0.000000}%
\pgfsetstrokecolor{currentstroke}%
\pgfsetdash{}{0pt}%
\pgfpathmoveto{\pgfqpoint{0.444137in}{0.319877in}}%
\pgfpathlineto{\pgfqpoint{0.444137in}{2.925408in}}%
\pgfusepath{stroke}%
\end{pgfscope}%
\begin{pgfscope}%
\pgfsetrectcap%
\pgfsetmiterjoin%
\pgfsetlinewidth{0.803000pt}%
\definecolor{currentstroke}{rgb}{0.000000,0.000000,0.000000}%
\pgfsetstrokecolor{currentstroke}%
\pgfsetdash{}{0pt}%
\pgfpathmoveto{\pgfqpoint{1.988155in}{0.319877in}}%
\pgfpathlineto{\pgfqpoint{1.988155in}{2.925408in}}%
\pgfusepath{stroke}%
\end{pgfscope}%
\begin{pgfscope}%
\pgfsetrectcap%
\pgfsetmiterjoin%
\pgfsetlinewidth{0.803000pt}%
\definecolor{currentstroke}{rgb}{0.000000,0.000000,0.000000}%
\pgfsetstrokecolor{currentstroke}%
\pgfsetdash{}{0pt}%
\pgfpathmoveto{\pgfqpoint{0.444137in}{0.319877in}}%
\pgfpathlineto{\pgfqpoint{1.988155in}{0.319877in}}%
\pgfusepath{stroke}%
\end{pgfscope}%
\begin{pgfscope}%
\pgfsetrectcap%
\pgfsetmiterjoin%
\pgfsetlinewidth{0.803000pt}%
\definecolor{currentstroke}{rgb}{0.000000,0.000000,0.000000}%
\pgfsetstrokecolor{currentstroke}%
\pgfsetdash{}{0pt}%
\pgfpathmoveto{\pgfqpoint{0.444137in}{2.925408in}}%
\pgfpathlineto{\pgfqpoint{1.988155in}{2.925408in}}%
\pgfusepath{stroke}%
\end{pgfscope}%
\begin{pgfscope}%
\pgfpathrectangle{\pgfqpoint{2.072071in}{0.319877in}}{\pgfqpoint{0.130277in}{2.605531in}} %
\pgfusepath{clip}%
\pgfsetbuttcap%
\pgfsetmiterjoin%
\definecolor{currentfill}{rgb}{1.000000,1.000000,1.000000}%
\pgfsetfillcolor{currentfill}%
\pgfsetlinewidth{0.010037pt}%
\definecolor{currentstroke}{rgb}{1.000000,1.000000,1.000000}%
\pgfsetstrokecolor{currentstroke}%
\pgfsetdash{}{0pt}%
\pgfpathmoveto{\pgfqpoint{2.072071in}{0.319877in}}%
\pgfpathlineto{\pgfqpoint{2.072071in}{0.330055in}}%
\pgfpathlineto{\pgfqpoint{2.072071in}{2.915230in}}%
\pgfpathlineto{\pgfqpoint{2.072071in}{2.925408in}}%
\pgfpathlineto{\pgfqpoint{2.202347in}{2.925408in}}%
\pgfpathlineto{\pgfqpoint{2.202347in}{2.915230in}}%
\pgfpathlineto{\pgfqpoint{2.202347in}{0.330055in}}%
\pgfpathlineto{\pgfqpoint{2.202347in}{0.319877in}}%
\pgfpathclose%
\pgfusepath{stroke,fill}%
\end{pgfscope}%
\begin{pgfscope}%
\pgfsys@transformshift{2.070000in}{0.320408in}%
\pgftext[left,bottom]{\pgfimage[interpolate=true,width=0.130000in,height=2.610000in]{Ferr_vs_dq_Ti_100K-img1.png}}%
\end{pgfscope}%
\begin{pgfscope}%
\pgfsetbuttcap%
\pgfsetroundjoin%
\definecolor{currentfill}{rgb}{0.000000,0.000000,0.000000}%
\pgfsetfillcolor{currentfill}%
\pgfsetlinewidth{0.803000pt}%
\definecolor{currentstroke}{rgb}{0.000000,0.000000,0.000000}%
\pgfsetstrokecolor{currentstroke}%
\pgfsetdash{}{0pt}%
\pgfsys@defobject{currentmarker}{\pgfqpoint{0.000000in}{0.000000in}}{\pgfqpoint{0.048611in}{0.000000in}}{%
\pgfpathmoveto{\pgfqpoint{0.000000in}{0.000000in}}%
\pgfpathlineto{\pgfqpoint{0.048611in}{0.000000in}}%
\pgfusepath{stroke,fill}%
}%
\begin{pgfscope}%
\pgfsys@transformshift{2.202347in}{0.319877in}%
\pgfsys@useobject{currentmarker}{}%
\end{pgfscope}%
\end{pgfscope}%
\begin{pgfscope}%
\pgftext[x=2.299570in,y=0.272050in,left,base]{\rmfamily\fontsize{10.000000}{12.000000}\selectfont \(\displaystyle 0\)}%
\end{pgfscope}%
\begin{pgfscope}%
\pgfsetbuttcap%
\pgfsetroundjoin%
\definecolor{currentfill}{rgb}{0.000000,0.000000,0.000000}%
\pgfsetfillcolor{currentfill}%
\pgfsetlinewidth{0.803000pt}%
\definecolor{currentstroke}{rgb}{0.000000,0.000000,0.000000}%
\pgfsetstrokecolor{currentstroke}%
\pgfsetdash{}{0pt}%
\pgfsys@defobject{currentmarker}{\pgfqpoint{0.000000in}{0.000000in}}{\pgfqpoint{0.048611in}{0.000000in}}{%
\pgfpathmoveto{\pgfqpoint{0.000000in}{0.000000in}}%
\pgfpathlineto{\pgfqpoint{0.048611in}{0.000000in}}%
\pgfusepath{stroke,fill}%
}%
\begin{pgfscope}%
\pgfsys@transformshift{2.202347in}{0.862696in}%
\pgfsys@useobject{currentmarker}{}%
\end{pgfscope}%
\end{pgfscope}%
\begin{pgfscope}%
\pgftext[x=2.299570in,y=0.814868in,left,base]{\rmfamily\fontsize{10.000000}{12.000000}\selectfont \(\displaystyle 5\)}%
\end{pgfscope}%
\begin{pgfscope}%
\pgfsetbuttcap%
\pgfsetroundjoin%
\definecolor{currentfill}{rgb}{0.000000,0.000000,0.000000}%
\pgfsetfillcolor{currentfill}%
\pgfsetlinewidth{0.803000pt}%
\definecolor{currentstroke}{rgb}{0.000000,0.000000,0.000000}%
\pgfsetstrokecolor{currentstroke}%
\pgfsetdash{}{0pt}%
\pgfsys@defobject{currentmarker}{\pgfqpoint{0.000000in}{0.000000in}}{\pgfqpoint{0.048611in}{0.000000in}}{%
\pgfpathmoveto{\pgfqpoint{0.000000in}{0.000000in}}%
\pgfpathlineto{\pgfqpoint{0.048611in}{0.000000in}}%
\pgfusepath{stroke,fill}%
}%
\begin{pgfscope}%
\pgfsys@transformshift{2.202347in}{1.405515in}%
\pgfsys@useobject{currentmarker}{}%
\end{pgfscope}%
\end{pgfscope}%
\begin{pgfscope}%
\pgftext[x=2.299570in,y=1.357687in,left,base]{\rmfamily\fontsize{10.000000}{12.000000}\selectfont \(\displaystyle 10\)}%
\end{pgfscope}%
\begin{pgfscope}%
\pgfsetbuttcap%
\pgfsetroundjoin%
\definecolor{currentfill}{rgb}{0.000000,0.000000,0.000000}%
\pgfsetfillcolor{currentfill}%
\pgfsetlinewidth{0.803000pt}%
\definecolor{currentstroke}{rgb}{0.000000,0.000000,0.000000}%
\pgfsetstrokecolor{currentstroke}%
\pgfsetdash{}{0pt}%
\pgfsys@defobject{currentmarker}{\pgfqpoint{0.000000in}{0.000000in}}{\pgfqpoint{0.048611in}{0.000000in}}{%
\pgfpathmoveto{\pgfqpoint{0.000000in}{0.000000in}}%
\pgfpathlineto{\pgfqpoint{0.048611in}{0.000000in}}%
\pgfusepath{stroke,fill}%
}%
\begin{pgfscope}%
\pgfsys@transformshift{2.202347in}{1.948334in}%
\pgfsys@useobject{currentmarker}{}%
\end{pgfscope}%
\end{pgfscope}%
\begin{pgfscope}%
\pgftext[x=2.299570in,y=1.900506in,left,base]{\rmfamily\fontsize{10.000000}{12.000000}\selectfont \(\displaystyle 15\)}%
\end{pgfscope}%
\begin{pgfscope}%
\pgfsetbuttcap%
\pgfsetroundjoin%
\definecolor{currentfill}{rgb}{0.000000,0.000000,0.000000}%
\pgfsetfillcolor{currentfill}%
\pgfsetlinewidth{0.803000pt}%
\definecolor{currentstroke}{rgb}{0.000000,0.000000,0.000000}%
\pgfsetstrokecolor{currentstroke}%
\pgfsetdash{}{0pt}%
\pgfsys@defobject{currentmarker}{\pgfqpoint{0.000000in}{0.000000in}}{\pgfqpoint{0.048611in}{0.000000in}}{%
\pgfpathmoveto{\pgfqpoint{0.000000in}{0.000000in}}%
\pgfpathlineto{\pgfqpoint{0.048611in}{0.000000in}}%
\pgfusepath{stroke,fill}%
}%
\begin{pgfscope}%
\pgfsys@transformshift{2.202347in}{2.491153in}%
\pgfsys@useobject{currentmarker}{}%
\end{pgfscope}%
\end{pgfscope}%
\begin{pgfscope}%
\pgftext[x=2.299570in,y=2.443325in,left,base]{\rmfamily\fontsize{10.000000}{12.000000}\selectfont \(\displaystyle 20\)}%
\end{pgfscope}%
\begin{pgfscope}%
\pgfsetbuttcap%
\pgfsetmiterjoin%
\pgfsetlinewidth{0.803000pt}%
\definecolor{currentstroke}{rgb}{0.000000,0.000000,0.000000}%
\pgfsetstrokecolor{currentstroke}%
\pgfsetdash{}{0pt}%
\pgfpathmoveto{\pgfqpoint{2.072071in}{0.319877in}}%
\pgfpathlineto{\pgfqpoint{2.072071in}{0.330055in}}%
\pgfpathlineto{\pgfqpoint{2.072071in}{2.915230in}}%
\pgfpathlineto{\pgfqpoint{2.072071in}{2.925408in}}%
\pgfpathlineto{\pgfqpoint{2.202347in}{2.925408in}}%
\pgfpathlineto{\pgfqpoint{2.202347in}{2.915230in}}%
\pgfpathlineto{\pgfqpoint{2.202347in}{0.330055in}}%
\pgfpathlineto{\pgfqpoint{2.202347in}{0.319877in}}%
\pgfpathclose%
\pgfusepath{stroke}%
\end{pgfscope}%
\end{pgfpicture}%
\makeatother%
\endgroup%

	\vspace*{-0.4cm}
	\caption{100 K. Bin size $0.0105e$}
	\end{subfigure}
	\hspace{0.6cm}
	\begin{subfigure}[b]{0.45\textwidth}
	\hspace*{-0.4cm}
	%% Creator: Matplotlib, PGF backend
%%
%% To include the figure in your LaTeX document, write
%%   \input{<filename>.pgf}
%%
%% Make sure the required packages are loaded in your preamble
%%   \usepackage{pgf}
%%
%% Figures using additional raster images can only be included by \input if
%% they are in the same directory as the main LaTeX file. For loading figures
%% from other directories you can use the `import` package
%%   \usepackage{import}
%% and then include the figures with
%%   \import{<path to file>}{<filename>.pgf}
%%
%% Matplotlib used the following preamble
%%   \usepackage[utf8x]{inputenc}
%%   \usepackage[T1]{fontenc}
%%
\begingroup%
\makeatletter%
\begin{pgfpicture}%
\pgfpathrectangle{\pgfpointorigin}{\pgfqpoint{2.538459in}{3.060408in}}%
\pgfusepath{use as bounding box, clip}%
\begin{pgfscope}%
\pgfsetbuttcap%
\pgfsetmiterjoin%
\definecolor{currentfill}{rgb}{1.000000,1.000000,1.000000}%
\pgfsetfillcolor{currentfill}%
\pgfsetlinewidth{0.000000pt}%
\definecolor{currentstroke}{rgb}{1.000000,1.000000,1.000000}%
\pgfsetstrokecolor{currentstroke}%
\pgfsetdash{}{0pt}%
\pgfpathmoveto{\pgfqpoint{0.000000in}{0.000000in}}%
\pgfpathlineto{\pgfqpoint{2.538459in}{0.000000in}}%
\pgfpathlineto{\pgfqpoint{2.538459in}{3.060408in}}%
\pgfpathlineto{\pgfqpoint{0.000000in}{3.060408in}}%
\pgfpathclose%
\pgfusepath{fill}%
\end{pgfscope}%
\begin{pgfscope}%
\pgfsetbuttcap%
\pgfsetmiterjoin%
\definecolor{currentfill}{rgb}{1.000000,1.000000,1.000000}%
\pgfsetfillcolor{currentfill}%
\pgfsetlinewidth{0.000000pt}%
\definecolor{currentstroke}{rgb}{0.000000,0.000000,0.000000}%
\pgfsetstrokecolor{currentstroke}%
\pgfsetstrokeopacity{0.000000}%
\pgfsetdash{}{0pt}%
\pgfpathmoveto{\pgfqpoint{0.444137in}{0.319877in}}%
\pgfpathlineto{\pgfqpoint{1.988155in}{0.319877in}}%
\pgfpathlineto{\pgfqpoint{1.988155in}{2.925408in}}%
\pgfpathlineto{\pgfqpoint{0.444137in}{2.925408in}}%
\pgfpathclose%
\pgfusepath{fill}%
\end{pgfscope}%
\begin{pgfscope}%
\pgfpathrectangle{\pgfqpoint{0.444137in}{0.319877in}}{\pgfqpoint{1.544018in}{2.605531in}} %
\pgfusepath{clip}%
\pgfsys@transformshift{0.444137in}{0.319877in}%
\pgftext[left,bottom]{\pgfimage[interpolate=true,width=1.550000in,height=2.610000in]{Ferr_vs_dq_Ti_200K-img0.png}}%
\end{pgfscope}%
\begin{pgfscope}%
\pgfpathrectangle{\pgfqpoint{0.444137in}{0.319877in}}{\pgfqpoint{1.544018in}{2.605531in}} %
\pgfusepath{clip}%
\pgfsetbuttcap%
\pgfsetroundjoin%
\definecolor{currentfill}{rgb}{1.000000,0.752941,0.796078}%
\pgfsetfillcolor{currentfill}%
\pgfsetlinewidth{1.003750pt}%
\definecolor{currentstroke}{rgb}{1.000000,0.752941,0.796078}%
\pgfsetstrokecolor{currentstroke}%
\pgfsetdash{}{0pt}%
\pgfpathmoveto{\pgfqpoint{0.940428in}{1.211859in}}%
\pgfpathcurveto{\pgfqpoint{0.951478in}{1.211859in}}{\pgfqpoint{0.962077in}{1.216249in}}{\pgfqpoint{0.969891in}{1.224063in}}%
\pgfpathcurveto{\pgfqpoint{0.977704in}{1.231877in}}{\pgfqpoint{0.982095in}{1.242476in}}{\pgfqpoint{0.982095in}{1.253526in}}%
\pgfpathcurveto{\pgfqpoint{0.982095in}{1.264576in}}{\pgfqpoint{0.977704in}{1.275175in}}{\pgfqpoint{0.969891in}{1.282989in}}%
\pgfpathcurveto{\pgfqpoint{0.962077in}{1.290802in}}{\pgfqpoint{0.951478in}{1.295192in}}{\pgfqpoint{0.940428in}{1.295192in}}%
\pgfpathcurveto{\pgfqpoint{0.929378in}{1.295192in}}{\pgfqpoint{0.918779in}{1.290802in}}{\pgfqpoint{0.910965in}{1.282989in}}%
\pgfpathcurveto{\pgfqpoint{0.903152in}{1.275175in}}{\pgfqpoint{0.898761in}{1.264576in}}{\pgfqpoint{0.898761in}{1.253526in}}%
\pgfpathcurveto{\pgfqpoint{0.898761in}{1.242476in}}{\pgfqpoint{0.903152in}{1.231877in}}{\pgfqpoint{0.910965in}{1.224063in}}%
\pgfpathcurveto{\pgfqpoint{0.918779in}{1.216249in}}{\pgfqpoint{0.929378in}{1.211859in}}{\pgfqpoint{0.940428in}{1.211859in}}%
\pgfpathclose%
\pgfusepath{stroke,fill}%
\end{pgfscope}%
\begin{pgfscope}%
\pgfpathrectangle{\pgfqpoint{0.444137in}{0.319877in}}{\pgfqpoint{1.544018in}{2.605531in}} %
\pgfusepath{clip}%
\pgfsetbuttcap%
\pgfsetroundjoin%
\definecolor{currentfill}{rgb}{1.000000,0.752941,0.796078}%
\pgfsetfillcolor{currentfill}%
\pgfsetlinewidth{1.003750pt}%
\definecolor{currentstroke}{rgb}{1.000000,0.752941,0.796078}%
\pgfsetstrokecolor{currentstroke}%
\pgfsetdash{}{0pt}%
\pgfpathmoveto{\pgfqpoint{1.050715in}{0.932252in}}%
\pgfpathcurveto{\pgfqpoint{1.061765in}{0.932252in}}{\pgfqpoint{1.072364in}{0.936642in}}{\pgfqpoint{1.080178in}{0.944456in}}%
\pgfpathcurveto{\pgfqpoint{1.087991in}{0.952270in}}{\pgfqpoint{1.092382in}{0.962869in}}{\pgfqpoint{1.092382in}{0.973919in}}%
\pgfpathcurveto{\pgfqpoint{1.092382in}{0.984969in}}{\pgfqpoint{1.087991in}{0.995568in}}{\pgfqpoint{1.080178in}{1.003381in}}%
\pgfpathcurveto{\pgfqpoint{1.072364in}{1.011195in}}{\pgfqpoint{1.061765in}{1.015585in}}{\pgfqpoint{1.050715in}{1.015585in}}%
\pgfpathcurveto{\pgfqpoint{1.039665in}{1.015585in}}{\pgfqpoint{1.029066in}{1.011195in}}{\pgfqpoint{1.021252in}{1.003381in}}%
\pgfpathcurveto{\pgfqpoint{1.013439in}{0.995568in}}{\pgfqpoint{1.009048in}{0.984969in}}{\pgfqpoint{1.009048in}{0.973919in}}%
\pgfpathcurveto{\pgfqpoint{1.009048in}{0.962869in}}{\pgfqpoint{1.013439in}{0.952270in}}{\pgfqpoint{1.021252in}{0.944456in}}%
\pgfpathcurveto{\pgfqpoint{1.029066in}{0.936642in}}{\pgfqpoint{1.039665in}{0.932252in}}{\pgfqpoint{1.050715in}{0.932252in}}%
\pgfpathclose%
\pgfusepath{stroke,fill}%
\end{pgfscope}%
\begin{pgfscope}%
\pgfpathrectangle{\pgfqpoint{0.444137in}{0.319877in}}{\pgfqpoint{1.544018in}{2.605531in}} %
\pgfusepath{clip}%
\pgfsetbuttcap%
\pgfsetroundjoin%
\definecolor{currentfill}{rgb}{1.000000,0.752941,0.796078}%
\pgfsetfillcolor{currentfill}%
\pgfsetlinewidth{1.003750pt}%
\definecolor{currentstroke}{rgb}{1.000000,0.752941,0.796078}%
\pgfsetstrokecolor{currentstroke}%
\pgfsetdash{}{0pt}%
\pgfpathmoveto{\pgfqpoint{1.161002in}{0.914619in}}%
\pgfpathcurveto{\pgfqpoint{1.172052in}{0.914619in}}{\pgfqpoint{1.182651in}{0.919009in}}{\pgfqpoint{1.190465in}{0.926823in}}%
\pgfpathcurveto{\pgfqpoint{1.198278in}{0.934637in}}{\pgfqpoint{1.202669in}{0.945236in}}{\pgfqpoint{1.202669in}{0.956286in}}%
\pgfpathcurveto{\pgfqpoint{1.202669in}{0.967336in}}{\pgfqpoint{1.198278in}{0.977935in}}{\pgfqpoint{1.190465in}{0.985748in}}%
\pgfpathcurveto{\pgfqpoint{1.182651in}{0.993562in}}{\pgfqpoint{1.172052in}{0.997952in}}{\pgfqpoint{1.161002in}{0.997952in}}%
\pgfpathcurveto{\pgfqpoint{1.149952in}{0.997952in}}{\pgfqpoint{1.139353in}{0.993562in}}{\pgfqpoint{1.131539in}{0.985748in}}%
\pgfpathcurveto{\pgfqpoint{1.123726in}{0.977935in}}{\pgfqpoint{1.119335in}{0.967336in}}{\pgfqpoint{1.119335in}{0.956286in}}%
\pgfpathcurveto{\pgfqpoint{1.119335in}{0.945236in}}{\pgfqpoint{1.123726in}{0.934637in}}{\pgfqpoint{1.131539in}{0.926823in}}%
\pgfpathcurveto{\pgfqpoint{1.139353in}{0.919009in}}{\pgfqpoint{1.149952in}{0.914619in}}{\pgfqpoint{1.161002in}{0.914619in}}%
\pgfpathclose%
\pgfusepath{stroke,fill}%
\end{pgfscope}%
\begin{pgfscope}%
\pgfpathrectangle{\pgfqpoint{0.444137in}{0.319877in}}{\pgfqpoint{1.544018in}{2.605531in}} %
\pgfusepath{clip}%
\pgfsetbuttcap%
\pgfsetroundjoin%
\definecolor{currentfill}{rgb}{1.000000,0.752941,0.796078}%
\pgfsetfillcolor{currentfill}%
\pgfsetlinewidth{1.003750pt}%
\definecolor{currentstroke}{rgb}{1.000000,0.752941,0.796078}%
\pgfsetstrokecolor{currentstroke}%
\pgfsetdash{}{0pt}%
\pgfpathmoveto{\pgfqpoint{1.271289in}{0.933022in}}%
\pgfpathcurveto{\pgfqpoint{1.282339in}{0.933022in}}{\pgfqpoint{1.292938in}{0.937412in}}{\pgfqpoint{1.300752in}{0.945226in}}%
\pgfpathcurveto{\pgfqpoint{1.308565in}{0.953039in}}{\pgfqpoint{1.312956in}{0.963638in}}{\pgfqpoint{1.312956in}{0.974688in}}%
\pgfpathcurveto{\pgfqpoint{1.312956in}{0.985738in}}{\pgfqpoint{1.308565in}{0.996337in}}{\pgfqpoint{1.300752in}{1.004151in}}%
\pgfpathcurveto{\pgfqpoint{1.292938in}{1.011965in}}{\pgfqpoint{1.282339in}{1.016355in}}{\pgfqpoint{1.271289in}{1.016355in}}%
\pgfpathcurveto{\pgfqpoint{1.260239in}{1.016355in}}{\pgfqpoint{1.249640in}{1.011965in}}{\pgfqpoint{1.241826in}{1.004151in}}%
\pgfpathcurveto{\pgfqpoint{1.234013in}{0.996337in}}{\pgfqpoint{1.229622in}{0.985738in}}{\pgfqpoint{1.229622in}{0.974688in}}%
\pgfpathcurveto{\pgfqpoint{1.229622in}{0.963638in}}{\pgfqpoint{1.234013in}{0.953039in}}{\pgfqpoint{1.241826in}{0.945226in}}%
\pgfpathcurveto{\pgfqpoint{1.249640in}{0.937412in}}{\pgfqpoint{1.260239in}{0.933022in}}{\pgfqpoint{1.271289in}{0.933022in}}%
\pgfpathclose%
\pgfusepath{stroke,fill}%
\end{pgfscope}%
\begin{pgfscope}%
\pgfpathrectangle{\pgfqpoint{0.444137in}{0.319877in}}{\pgfqpoint{1.544018in}{2.605531in}} %
\pgfusepath{clip}%
\pgfsetbuttcap%
\pgfsetroundjoin%
\definecolor{currentfill}{rgb}{1.000000,0.752941,0.796078}%
\pgfsetfillcolor{currentfill}%
\pgfsetlinewidth{1.003750pt}%
\definecolor{currentstroke}{rgb}{1.000000,0.752941,0.796078}%
\pgfsetstrokecolor{currentstroke}%
\pgfsetdash{}{0pt}%
\pgfpathmoveto{\pgfqpoint{1.381576in}{1.042105in}}%
\pgfpathcurveto{\pgfqpoint{1.392626in}{1.042105in}}{\pgfqpoint{1.403225in}{1.046495in}}{\pgfqpoint{1.411039in}{1.054309in}}%
\pgfpathcurveto{\pgfqpoint{1.418852in}{1.062122in}}{\pgfqpoint{1.423243in}{1.072721in}}{\pgfqpoint{1.423243in}{1.083772in}}%
\pgfpathcurveto{\pgfqpoint{1.423243in}{1.094822in}}{\pgfqpoint{1.418852in}{1.105421in}}{\pgfqpoint{1.411039in}{1.113234in}}%
\pgfpathcurveto{\pgfqpoint{1.403225in}{1.121048in}}{\pgfqpoint{1.392626in}{1.125438in}}{\pgfqpoint{1.381576in}{1.125438in}}%
\pgfpathcurveto{\pgfqpoint{1.370526in}{1.125438in}}{\pgfqpoint{1.359927in}{1.121048in}}{\pgfqpoint{1.352113in}{1.113234in}}%
\pgfpathcurveto{\pgfqpoint{1.344300in}{1.105421in}}{\pgfqpoint{1.339909in}{1.094822in}}{\pgfqpoint{1.339909in}{1.083772in}}%
\pgfpathcurveto{\pgfqpoint{1.339909in}{1.072721in}}{\pgfqpoint{1.344300in}{1.062122in}}{\pgfqpoint{1.352113in}{1.054309in}}%
\pgfpathcurveto{\pgfqpoint{1.359927in}{1.046495in}}{\pgfqpoint{1.370526in}{1.042105in}}{\pgfqpoint{1.381576in}{1.042105in}}%
\pgfpathclose%
\pgfusepath{stroke,fill}%
\end{pgfscope}%
\begin{pgfscope}%
\pgfpathrectangle{\pgfqpoint{0.444137in}{0.319877in}}{\pgfqpoint{1.544018in}{2.605531in}} %
\pgfusepath{clip}%
\pgfsetbuttcap%
\pgfsetroundjoin%
\definecolor{currentfill}{rgb}{1.000000,0.752941,0.796078}%
\pgfsetfillcolor{currentfill}%
\pgfsetlinewidth{1.003750pt}%
\definecolor{currentstroke}{rgb}{1.000000,0.752941,0.796078}%
\pgfsetstrokecolor{currentstroke}%
\pgfsetdash{}{0pt}%
\pgfpathmoveto{\pgfqpoint{1.491863in}{1.168434in}}%
\pgfpathcurveto{\pgfqpoint{1.502913in}{1.168434in}}{\pgfqpoint{1.513512in}{1.172824in}}{\pgfqpoint{1.521326in}{1.180637in}}%
\pgfpathcurveto{\pgfqpoint{1.529139in}{1.188451in}}{\pgfqpoint{1.533530in}{1.199050in}}{\pgfqpoint{1.533530in}{1.210100in}}%
\pgfpathcurveto{\pgfqpoint{1.533530in}{1.221150in}}{\pgfqpoint{1.529139in}{1.231749in}}{\pgfqpoint{1.521326in}{1.239563in}}%
\pgfpathcurveto{\pgfqpoint{1.513512in}{1.247377in}}{\pgfqpoint{1.502913in}{1.251767in}}{\pgfqpoint{1.491863in}{1.251767in}}%
\pgfpathcurveto{\pgfqpoint{1.480813in}{1.251767in}}{\pgfqpoint{1.470214in}{1.247377in}}{\pgfqpoint{1.462400in}{1.239563in}}%
\pgfpathcurveto{\pgfqpoint{1.454587in}{1.231749in}}{\pgfqpoint{1.450196in}{1.221150in}}{\pgfqpoint{1.450196in}{1.210100in}}%
\pgfpathcurveto{\pgfqpoint{1.450196in}{1.199050in}}{\pgfqpoint{1.454587in}{1.188451in}}{\pgfqpoint{1.462400in}{1.180637in}}%
\pgfpathcurveto{\pgfqpoint{1.470214in}{1.172824in}}{\pgfqpoint{1.480813in}{1.168434in}}{\pgfqpoint{1.491863in}{1.168434in}}%
\pgfpathclose%
\pgfusepath{stroke,fill}%
\end{pgfscope}%
\begin{pgfscope}%
\pgfpathrectangle{\pgfqpoint{0.444137in}{0.319877in}}{\pgfqpoint{1.544018in}{2.605531in}} %
\pgfusepath{clip}%
\pgfsetbuttcap%
\pgfsetroundjoin%
\definecolor{currentfill}{rgb}{1.000000,0.752941,0.796078}%
\pgfsetfillcolor{currentfill}%
\pgfsetlinewidth{1.003750pt}%
\definecolor{currentstroke}{rgb}{1.000000,0.752941,0.796078}%
\pgfsetstrokecolor{currentstroke}%
\pgfsetdash{}{0pt}%
\pgfpathmoveto{\pgfqpoint{1.602150in}{1.450699in}}%
\pgfpathcurveto{\pgfqpoint{1.613200in}{1.450699in}}{\pgfqpoint{1.623799in}{1.455090in}}{\pgfqpoint{1.631613in}{1.462903in}}%
\pgfpathcurveto{\pgfqpoint{1.639426in}{1.470717in}}{\pgfqpoint{1.643817in}{1.481316in}}{\pgfqpoint{1.643817in}{1.492366in}}%
\pgfpathcurveto{\pgfqpoint{1.643817in}{1.503416in}}{\pgfqpoint{1.639426in}{1.514015in}}{\pgfqpoint{1.631613in}{1.521829in}}%
\pgfpathcurveto{\pgfqpoint{1.623799in}{1.529642in}}{\pgfqpoint{1.613200in}{1.534033in}}{\pgfqpoint{1.602150in}{1.534033in}}%
\pgfpathcurveto{\pgfqpoint{1.591100in}{1.534033in}}{\pgfqpoint{1.580501in}{1.529642in}}{\pgfqpoint{1.572687in}{1.521829in}}%
\pgfpathcurveto{\pgfqpoint{1.564874in}{1.514015in}}{\pgfqpoint{1.560483in}{1.503416in}}{\pgfqpoint{1.560483in}{1.492366in}}%
\pgfpathcurveto{\pgfqpoint{1.560483in}{1.481316in}}{\pgfqpoint{1.564874in}{1.470717in}}{\pgfqpoint{1.572687in}{1.462903in}}%
\pgfpathcurveto{\pgfqpoint{1.580501in}{1.455090in}}{\pgfqpoint{1.591100in}{1.450699in}}{\pgfqpoint{1.602150in}{1.450699in}}%
\pgfpathclose%
\pgfusepath{stroke,fill}%
\end{pgfscope}%
\begin{pgfscope}%
\pgfsetbuttcap%
\pgfsetroundjoin%
\definecolor{currentfill}{rgb}{0.000000,0.000000,0.000000}%
\pgfsetfillcolor{currentfill}%
\pgfsetlinewidth{0.803000pt}%
\definecolor{currentstroke}{rgb}{0.000000,0.000000,0.000000}%
\pgfsetstrokecolor{currentstroke}%
\pgfsetdash{}{0pt}%
\pgfsys@defobject{currentmarker}{\pgfqpoint{0.000000in}{-0.048611in}}{\pgfqpoint{0.000000in}{0.000000in}}{%
\pgfpathmoveto{\pgfqpoint{0.000000in}{0.000000in}}%
\pgfpathlineto{\pgfqpoint{0.000000in}{-0.048611in}}%
\pgfusepath{stroke,fill}%
}%
\begin{pgfscope}%
\pgfsys@transformshift{0.733640in}{0.319877in}%
\pgfsys@useobject{currentmarker}{}%
\end{pgfscope}%
\end{pgfscope}%
\begin{pgfscope}%
\pgftext[x=0.733640in,y=0.222655in,,top]{\rmfamily\fontsize{10.000000}{12.000000}\selectfont \(\displaystyle -0.05\)}%
\end{pgfscope}%
\begin{pgfscope}%
\pgfsetbuttcap%
\pgfsetroundjoin%
\definecolor{currentfill}{rgb}{0.000000,0.000000,0.000000}%
\pgfsetfillcolor{currentfill}%
\pgfsetlinewidth{0.803000pt}%
\definecolor{currentstroke}{rgb}{0.000000,0.000000,0.000000}%
\pgfsetstrokecolor{currentstroke}%
\pgfsetdash{}{0pt}%
\pgfsys@defobject{currentmarker}{\pgfqpoint{0.000000in}{-0.048611in}}{\pgfqpoint{0.000000in}{0.000000in}}{%
\pgfpathmoveto{\pgfqpoint{0.000000in}{0.000000in}}%
\pgfpathlineto{\pgfqpoint{0.000000in}{-0.048611in}}%
\pgfusepath{stroke,fill}%
}%
\begin{pgfscope}%
\pgfsys@transformshift{1.216146in}{0.319877in}%
\pgfsys@useobject{currentmarker}{}%
\end{pgfscope}%
\end{pgfscope}%
\begin{pgfscope}%
\pgftext[x=1.216146in,y=0.222655in,,top]{\rmfamily\fontsize{10.000000}{12.000000}\selectfont \(\displaystyle 0.00\)}%
\end{pgfscope}%
\begin{pgfscope}%
\pgfsetbuttcap%
\pgfsetroundjoin%
\definecolor{currentfill}{rgb}{0.000000,0.000000,0.000000}%
\pgfsetfillcolor{currentfill}%
\pgfsetlinewidth{0.803000pt}%
\definecolor{currentstroke}{rgb}{0.000000,0.000000,0.000000}%
\pgfsetstrokecolor{currentstroke}%
\pgfsetdash{}{0pt}%
\pgfsys@defobject{currentmarker}{\pgfqpoint{0.000000in}{-0.048611in}}{\pgfqpoint{0.000000in}{0.000000in}}{%
\pgfpathmoveto{\pgfqpoint{0.000000in}{0.000000in}}%
\pgfpathlineto{\pgfqpoint{0.000000in}{-0.048611in}}%
\pgfusepath{stroke,fill}%
}%
\begin{pgfscope}%
\pgfsys@transformshift{1.698651in}{0.319877in}%
\pgfsys@useobject{currentmarker}{}%
\end{pgfscope}%
\end{pgfscope}%
\begin{pgfscope}%
\pgftext[x=1.698651in,y=0.222655in,,top]{\rmfamily\fontsize{10.000000}{12.000000}\selectfont \(\displaystyle 0.05\)}%
\end{pgfscope}%
\begin{pgfscope}%
\pgfsetbuttcap%
\pgfsetroundjoin%
\definecolor{currentfill}{rgb}{0.000000,0.000000,0.000000}%
\pgfsetfillcolor{currentfill}%
\pgfsetlinewidth{0.803000pt}%
\definecolor{currentstroke}{rgb}{0.000000,0.000000,0.000000}%
\pgfsetstrokecolor{currentstroke}%
\pgfsetdash{}{0pt}%
\pgfsys@defobject{currentmarker}{\pgfqpoint{-0.048611in}{0.000000in}}{\pgfqpoint{0.000000in}{0.000000in}}{%
\pgfpathmoveto{\pgfqpoint{0.000000in}{0.000000in}}%
\pgfpathlineto{\pgfqpoint{-0.048611in}{0.000000in}}%
\pgfusepath{stroke,fill}%
}%
\begin{pgfscope}%
\pgfsys@transformshift{0.444137in}{0.319877in}%
\pgfsys@useobject{currentmarker}{}%
\end{pgfscope}%
\end{pgfscope}%
\begin{pgfscope}%
\pgftext[x=0.100000in,y=0.272050in,left,base]{\rmfamily\fontsize{10.000000}{12.000000}\selectfont \(\displaystyle 0.00\)}%
\end{pgfscope}%
\begin{pgfscope}%
\pgfsetbuttcap%
\pgfsetroundjoin%
\definecolor{currentfill}{rgb}{0.000000,0.000000,0.000000}%
\pgfsetfillcolor{currentfill}%
\pgfsetlinewidth{0.803000pt}%
\definecolor{currentstroke}{rgb}{0.000000,0.000000,0.000000}%
\pgfsetstrokecolor{currentstroke}%
\pgfsetdash{}{0pt}%
\pgfsys@defobject{currentmarker}{\pgfqpoint{-0.048611in}{0.000000in}}{\pgfqpoint{0.000000in}{0.000000in}}{%
\pgfpathmoveto{\pgfqpoint{0.000000in}{0.000000in}}%
\pgfpathlineto{\pgfqpoint{-0.048611in}{0.000000in}}%
\pgfusepath{stroke,fill}%
}%
\begin{pgfscope}%
\pgfsys@transformshift{0.444137in}{0.802383in}%
\pgfsys@useobject{currentmarker}{}%
\end{pgfscope}%
\end{pgfscope}%
\begin{pgfscope}%
\pgftext[x=0.100000in,y=0.754555in,left,base]{\rmfamily\fontsize{10.000000}{12.000000}\selectfont \(\displaystyle 0.05\)}%
\end{pgfscope}%
\begin{pgfscope}%
\pgfsetbuttcap%
\pgfsetroundjoin%
\definecolor{currentfill}{rgb}{0.000000,0.000000,0.000000}%
\pgfsetfillcolor{currentfill}%
\pgfsetlinewidth{0.803000pt}%
\definecolor{currentstroke}{rgb}{0.000000,0.000000,0.000000}%
\pgfsetstrokecolor{currentstroke}%
\pgfsetdash{}{0pt}%
\pgfsys@defobject{currentmarker}{\pgfqpoint{-0.048611in}{0.000000in}}{\pgfqpoint{0.000000in}{0.000000in}}{%
\pgfpathmoveto{\pgfqpoint{0.000000in}{0.000000in}}%
\pgfpathlineto{\pgfqpoint{-0.048611in}{0.000000in}}%
\pgfusepath{stroke,fill}%
}%
\begin{pgfscope}%
\pgfsys@transformshift{0.444137in}{1.284889in}%
\pgfsys@useobject{currentmarker}{}%
\end{pgfscope}%
\end{pgfscope}%
\begin{pgfscope}%
\pgftext[x=0.100000in,y=1.237061in,left,base]{\rmfamily\fontsize{10.000000}{12.000000}\selectfont \(\displaystyle 0.10\)}%
\end{pgfscope}%
\begin{pgfscope}%
\pgfsetbuttcap%
\pgfsetroundjoin%
\definecolor{currentfill}{rgb}{0.000000,0.000000,0.000000}%
\pgfsetfillcolor{currentfill}%
\pgfsetlinewidth{0.803000pt}%
\definecolor{currentstroke}{rgb}{0.000000,0.000000,0.000000}%
\pgfsetstrokecolor{currentstroke}%
\pgfsetdash{}{0pt}%
\pgfsys@defobject{currentmarker}{\pgfqpoint{-0.048611in}{0.000000in}}{\pgfqpoint{0.000000in}{0.000000in}}{%
\pgfpathmoveto{\pgfqpoint{0.000000in}{0.000000in}}%
\pgfpathlineto{\pgfqpoint{-0.048611in}{0.000000in}}%
\pgfusepath{stroke,fill}%
}%
\begin{pgfscope}%
\pgfsys@transformshift{0.444137in}{1.767394in}%
\pgfsys@useobject{currentmarker}{}%
\end{pgfscope}%
\end{pgfscope}%
\begin{pgfscope}%
\pgftext[x=0.100000in,y=1.719567in,left,base]{\rmfamily\fontsize{10.000000}{12.000000}\selectfont \(\displaystyle 0.15\)}%
\end{pgfscope}%
\begin{pgfscope}%
\pgfsetbuttcap%
\pgfsetroundjoin%
\definecolor{currentfill}{rgb}{0.000000,0.000000,0.000000}%
\pgfsetfillcolor{currentfill}%
\pgfsetlinewidth{0.803000pt}%
\definecolor{currentstroke}{rgb}{0.000000,0.000000,0.000000}%
\pgfsetstrokecolor{currentstroke}%
\pgfsetdash{}{0pt}%
\pgfsys@defobject{currentmarker}{\pgfqpoint{-0.048611in}{0.000000in}}{\pgfqpoint{0.000000in}{0.000000in}}{%
\pgfpathmoveto{\pgfqpoint{0.000000in}{0.000000in}}%
\pgfpathlineto{\pgfqpoint{-0.048611in}{0.000000in}}%
\pgfusepath{stroke,fill}%
}%
\begin{pgfscope}%
\pgfsys@transformshift{0.444137in}{2.249900in}%
\pgfsys@useobject{currentmarker}{}%
\end{pgfscope}%
\end{pgfscope}%
\begin{pgfscope}%
\pgftext[x=0.100000in,y=2.202072in,left,base]{\rmfamily\fontsize{10.000000}{12.000000}\selectfont \(\displaystyle 0.20\)}%
\end{pgfscope}%
\begin{pgfscope}%
\pgfsetbuttcap%
\pgfsetroundjoin%
\definecolor{currentfill}{rgb}{0.000000,0.000000,0.000000}%
\pgfsetfillcolor{currentfill}%
\pgfsetlinewidth{0.803000pt}%
\definecolor{currentstroke}{rgb}{0.000000,0.000000,0.000000}%
\pgfsetstrokecolor{currentstroke}%
\pgfsetdash{}{0pt}%
\pgfsys@defobject{currentmarker}{\pgfqpoint{-0.048611in}{0.000000in}}{\pgfqpoint{0.000000in}{0.000000in}}{%
\pgfpathmoveto{\pgfqpoint{0.000000in}{0.000000in}}%
\pgfpathlineto{\pgfqpoint{-0.048611in}{0.000000in}}%
\pgfusepath{stroke,fill}%
}%
\begin{pgfscope}%
\pgfsys@transformshift{0.444137in}{2.732406in}%
\pgfsys@useobject{currentmarker}{}%
\end{pgfscope}%
\end{pgfscope}%
\begin{pgfscope}%
\pgftext[x=0.100000in,y=2.684578in,left,base]{\rmfamily\fontsize{10.000000}{12.000000}\selectfont \(\displaystyle 0.25\)}%
\end{pgfscope}%
\begin{pgfscope}%
\pgfsetrectcap%
\pgfsetmiterjoin%
\pgfsetlinewidth{0.803000pt}%
\definecolor{currentstroke}{rgb}{0.000000,0.000000,0.000000}%
\pgfsetstrokecolor{currentstroke}%
\pgfsetdash{}{0pt}%
\pgfpathmoveto{\pgfqpoint{0.444137in}{0.319877in}}%
\pgfpathlineto{\pgfqpoint{0.444137in}{2.925408in}}%
\pgfusepath{stroke}%
\end{pgfscope}%
\begin{pgfscope}%
\pgfsetrectcap%
\pgfsetmiterjoin%
\pgfsetlinewidth{0.803000pt}%
\definecolor{currentstroke}{rgb}{0.000000,0.000000,0.000000}%
\pgfsetstrokecolor{currentstroke}%
\pgfsetdash{}{0pt}%
\pgfpathmoveto{\pgfqpoint{1.988155in}{0.319877in}}%
\pgfpathlineto{\pgfqpoint{1.988155in}{2.925408in}}%
\pgfusepath{stroke}%
\end{pgfscope}%
\begin{pgfscope}%
\pgfsetrectcap%
\pgfsetmiterjoin%
\pgfsetlinewidth{0.803000pt}%
\definecolor{currentstroke}{rgb}{0.000000,0.000000,0.000000}%
\pgfsetstrokecolor{currentstroke}%
\pgfsetdash{}{0pt}%
\pgfpathmoveto{\pgfqpoint{0.444137in}{0.319877in}}%
\pgfpathlineto{\pgfqpoint{1.988155in}{0.319877in}}%
\pgfusepath{stroke}%
\end{pgfscope}%
\begin{pgfscope}%
\pgfsetrectcap%
\pgfsetmiterjoin%
\pgfsetlinewidth{0.803000pt}%
\definecolor{currentstroke}{rgb}{0.000000,0.000000,0.000000}%
\pgfsetstrokecolor{currentstroke}%
\pgfsetdash{}{0pt}%
\pgfpathmoveto{\pgfqpoint{0.444137in}{2.925408in}}%
\pgfpathlineto{\pgfqpoint{1.988155in}{2.925408in}}%
\pgfusepath{stroke}%
\end{pgfscope}%
\begin{pgfscope}%
\pgfpathrectangle{\pgfqpoint{2.072071in}{0.319877in}}{\pgfqpoint{0.130277in}{2.605531in}} %
\pgfusepath{clip}%
\pgfsetbuttcap%
\pgfsetmiterjoin%
\definecolor{currentfill}{rgb}{1.000000,1.000000,1.000000}%
\pgfsetfillcolor{currentfill}%
\pgfsetlinewidth{0.010037pt}%
\definecolor{currentstroke}{rgb}{1.000000,1.000000,1.000000}%
\pgfsetstrokecolor{currentstroke}%
\pgfsetdash{}{0pt}%
\pgfpathmoveto{\pgfqpoint{2.072071in}{0.319877in}}%
\pgfpathlineto{\pgfqpoint{2.072071in}{0.330055in}}%
\pgfpathlineto{\pgfqpoint{2.072071in}{2.915230in}}%
\pgfpathlineto{\pgfqpoint{2.072071in}{2.925408in}}%
\pgfpathlineto{\pgfqpoint{2.202347in}{2.925408in}}%
\pgfpathlineto{\pgfqpoint{2.202347in}{2.915230in}}%
\pgfpathlineto{\pgfqpoint{2.202347in}{0.330055in}}%
\pgfpathlineto{\pgfqpoint{2.202347in}{0.319877in}}%
\pgfpathclose%
\pgfusepath{stroke,fill}%
\end{pgfscope}%
\begin{pgfscope}%
\pgfsys@transformshift{2.070000in}{0.320408in}%
\pgftext[left,bottom]{\pgfimage[interpolate=true,width=0.130000in,height=2.610000in]{Ferr_vs_dq_Ti_200K-img1.png}}%
\end{pgfscope}%
\begin{pgfscope}%
\pgfsetbuttcap%
\pgfsetroundjoin%
\definecolor{currentfill}{rgb}{0.000000,0.000000,0.000000}%
\pgfsetfillcolor{currentfill}%
\pgfsetlinewidth{0.803000pt}%
\definecolor{currentstroke}{rgb}{0.000000,0.000000,0.000000}%
\pgfsetstrokecolor{currentstroke}%
\pgfsetdash{}{0pt}%
\pgfsys@defobject{currentmarker}{\pgfqpoint{0.000000in}{0.000000in}}{\pgfqpoint{0.048611in}{0.000000in}}{%
\pgfpathmoveto{\pgfqpoint{0.000000in}{0.000000in}}%
\pgfpathlineto{\pgfqpoint{0.048611in}{0.000000in}}%
\pgfusepath{stroke,fill}%
}%
\begin{pgfscope}%
\pgfsys@transformshift{2.202347in}{0.319877in}%
\pgfsys@useobject{currentmarker}{}%
\end{pgfscope}%
\end{pgfscope}%
\begin{pgfscope}%
\pgftext[x=2.299570in,y=0.272050in,left,base]{\rmfamily\fontsize{10.000000}{12.000000}\selectfont \(\displaystyle 0\)}%
\end{pgfscope}%
\begin{pgfscope}%
\pgfsetbuttcap%
\pgfsetroundjoin%
\definecolor{currentfill}{rgb}{0.000000,0.000000,0.000000}%
\pgfsetfillcolor{currentfill}%
\pgfsetlinewidth{0.803000pt}%
\definecolor{currentstroke}{rgb}{0.000000,0.000000,0.000000}%
\pgfsetstrokecolor{currentstroke}%
\pgfsetdash{}{0pt}%
\pgfsys@defobject{currentmarker}{\pgfqpoint{0.000000in}{0.000000in}}{\pgfqpoint{0.048611in}{0.000000in}}{%
\pgfpathmoveto{\pgfqpoint{0.000000in}{0.000000in}}%
\pgfpathlineto{\pgfqpoint{0.048611in}{0.000000in}}%
\pgfusepath{stroke,fill}%
}%
\begin{pgfscope}%
\pgfsys@transformshift{2.202347in}{0.862696in}%
\pgfsys@useobject{currentmarker}{}%
\end{pgfscope}%
\end{pgfscope}%
\begin{pgfscope}%
\pgftext[x=2.299570in,y=0.814868in,left,base]{\rmfamily\fontsize{10.000000}{12.000000}\selectfont \(\displaystyle 5\)}%
\end{pgfscope}%
\begin{pgfscope}%
\pgfsetbuttcap%
\pgfsetroundjoin%
\definecolor{currentfill}{rgb}{0.000000,0.000000,0.000000}%
\pgfsetfillcolor{currentfill}%
\pgfsetlinewidth{0.803000pt}%
\definecolor{currentstroke}{rgb}{0.000000,0.000000,0.000000}%
\pgfsetstrokecolor{currentstroke}%
\pgfsetdash{}{0pt}%
\pgfsys@defobject{currentmarker}{\pgfqpoint{0.000000in}{0.000000in}}{\pgfqpoint{0.048611in}{0.000000in}}{%
\pgfpathmoveto{\pgfqpoint{0.000000in}{0.000000in}}%
\pgfpathlineto{\pgfqpoint{0.048611in}{0.000000in}}%
\pgfusepath{stroke,fill}%
}%
\begin{pgfscope}%
\pgfsys@transformshift{2.202347in}{1.405515in}%
\pgfsys@useobject{currentmarker}{}%
\end{pgfscope}%
\end{pgfscope}%
\begin{pgfscope}%
\pgftext[x=2.299570in,y=1.357687in,left,base]{\rmfamily\fontsize{10.000000}{12.000000}\selectfont \(\displaystyle 10\)}%
\end{pgfscope}%
\begin{pgfscope}%
\pgfsetbuttcap%
\pgfsetroundjoin%
\definecolor{currentfill}{rgb}{0.000000,0.000000,0.000000}%
\pgfsetfillcolor{currentfill}%
\pgfsetlinewidth{0.803000pt}%
\definecolor{currentstroke}{rgb}{0.000000,0.000000,0.000000}%
\pgfsetstrokecolor{currentstroke}%
\pgfsetdash{}{0pt}%
\pgfsys@defobject{currentmarker}{\pgfqpoint{0.000000in}{0.000000in}}{\pgfqpoint{0.048611in}{0.000000in}}{%
\pgfpathmoveto{\pgfqpoint{0.000000in}{0.000000in}}%
\pgfpathlineto{\pgfqpoint{0.048611in}{0.000000in}}%
\pgfusepath{stroke,fill}%
}%
\begin{pgfscope}%
\pgfsys@transformshift{2.202347in}{1.948334in}%
\pgfsys@useobject{currentmarker}{}%
\end{pgfscope}%
\end{pgfscope}%
\begin{pgfscope}%
\pgftext[x=2.299570in,y=1.900506in,left,base]{\rmfamily\fontsize{10.000000}{12.000000}\selectfont \(\displaystyle 15\)}%
\end{pgfscope}%
\begin{pgfscope}%
\pgfsetbuttcap%
\pgfsetroundjoin%
\definecolor{currentfill}{rgb}{0.000000,0.000000,0.000000}%
\pgfsetfillcolor{currentfill}%
\pgfsetlinewidth{0.803000pt}%
\definecolor{currentstroke}{rgb}{0.000000,0.000000,0.000000}%
\pgfsetstrokecolor{currentstroke}%
\pgfsetdash{}{0pt}%
\pgfsys@defobject{currentmarker}{\pgfqpoint{0.000000in}{0.000000in}}{\pgfqpoint{0.048611in}{0.000000in}}{%
\pgfpathmoveto{\pgfqpoint{0.000000in}{0.000000in}}%
\pgfpathlineto{\pgfqpoint{0.048611in}{0.000000in}}%
\pgfusepath{stroke,fill}%
}%
\begin{pgfscope}%
\pgfsys@transformshift{2.202347in}{2.491153in}%
\pgfsys@useobject{currentmarker}{}%
\end{pgfscope}%
\end{pgfscope}%
\begin{pgfscope}%
\pgftext[x=2.299570in,y=2.443325in,left,base]{\rmfamily\fontsize{10.000000}{12.000000}\selectfont \(\displaystyle 20\)}%
\end{pgfscope}%
\begin{pgfscope}%
\pgfsetbuttcap%
\pgfsetmiterjoin%
\pgfsetlinewidth{0.803000pt}%
\definecolor{currentstroke}{rgb}{0.000000,0.000000,0.000000}%
\pgfsetstrokecolor{currentstroke}%
\pgfsetdash{}{0pt}%
\pgfpathmoveto{\pgfqpoint{2.072071in}{0.319877in}}%
\pgfpathlineto{\pgfqpoint{2.072071in}{0.330055in}}%
\pgfpathlineto{\pgfqpoint{2.072071in}{2.915230in}}%
\pgfpathlineto{\pgfqpoint{2.072071in}{2.925408in}}%
\pgfpathlineto{\pgfqpoint{2.202347in}{2.925408in}}%
\pgfpathlineto{\pgfqpoint{2.202347in}{2.915230in}}%
\pgfpathlineto{\pgfqpoint{2.202347in}{0.330055in}}%
\pgfpathlineto{\pgfqpoint{2.202347in}{0.319877in}}%
\pgfpathclose%
\pgfusepath{stroke}%
\end{pgfscope}%
\end{pgfpicture}%
\makeatother%
\endgroup%

	\vspace*{-0.4cm}
	\caption{200 K. Bin size $0.011e$}
	\end{subfigure}
	\quad
	\begin{subfigure}[b]{0.45\textwidth}
	\hspace*{-0.4cm}
	%% Creator: Matplotlib, PGF backend
%%
%% To include the figure in your LaTeX document, write
%%   \input{<filename>.pgf}
%%
%% Make sure the required packages are loaded in your preamble
%%   \usepackage{pgf}
%%
%% Figures using additional raster images can only be included by \input if
%% they are in the same directory as the main LaTeX file. For loading figures
%% from other directories you can use the `import` package
%%   \usepackage{import}
%% and then include the figures with
%%   \import{<path to file>}{<filename>.pgf}
%%
%% Matplotlib used the following preamble
%%   \usepackage[utf8x]{inputenc}
%%   \usepackage[T1]{fontenc}
%%
\begingroup%
\makeatletter%
\begin{pgfpicture}%
\pgfpathrectangle{\pgfpointorigin}{\pgfqpoint{2.538459in}{3.060408in}}%
\pgfusepath{use as bounding box, clip}%
\begin{pgfscope}%
\pgfsetbuttcap%
\pgfsetmiterjoin%
\definecolor{currentfill}{rgb}{1.000000,1.000000,1.000000}%
\pgfsetfillcolor{currentfill}%
\pgfsetlinewidth{0.000000pt}%
\definecolor{currentstroke}{rgb}{1.000000,1.000000,1.000000}%
\pgfsetstrokecolor{currentstroke}%
\pgfsetdash{}{0pt}%
\pgfpathmoveto{\pgfqpoint{0.000000in}{0.000000in}}%
\pgfpathlineto{\pgfqpoint{2.538459in}{0.000000in}}%
\pgfpathlineto{\pgfqpoint{2.538459in}{3.060408in}}%
\pgfpathlineto{\pgfqpoint{0.000000in}{3.060408in}}%
\pgfpathclose%
\pgfusepath{fill}%
\end{pgfscope}%
\begin{pgfscope}%
\pgfsetbuttcap%
\pgfsetmiterjoin%
\definecolor{currentfill}{rgb}{1.000000,1.000000,1.000000}%
\pgfsetfillcolor{currentfill}%
\pgfsetlinewidth{0.000000pt}%
\definecolor{currentstroke}{rgb}{0.000000,0.000000,0.000000}%
\pgfsetstrokecolor{currentstroke}%
\pgfsetstrokeopacity{0.000000}%
\pgfsetdash{}{0pt}%
\pgfpathmoveto{\pgfqpoint{0.444137in}{0.319877in}}%
\pgfpathlineto{\pgfqpoint{1.988155in}{0.319877in}}%
\pgfpathlineto{\pgfqpoint{1.988155in}{2.925408in}}%
\pgfpathlineto{\pgfqpoint{0.444137in}{2.925408in}}%
\pgfpathclose%
\pgfusepath{fill}%
\end{pgfscope}%
\begin{pgfscope}%
\pgfpathrectangle{\pgfqpoint{0.444137in}{0.319877in}}{\pgfqpoint{1.544018in}{2.605531in}} %
\pgfusepath{clip}%
\pgfsys@transformshift{0.444137in}{0.319877in}%
\pgftext[left,bottom]{\pgfimage[interpolate=true,width=1.550000in,height=2.610000in]{Ferr_vs_dq_Ti_300K-img0.png}}%
\end{pgfscope}%
\begin{pgfscope}%
\pgfpathrectangle{\pgfqpoint{0.444137in}{0.319877in}}{\pgfqpoint{1.544018in}{2.605531in}} %
\pgfusepath{clip}%
\pgfsetbuttcap%
\pgfsetroundjoin%
\definecolor{currentfill}{rgb}{1.000000,0.752941,0.796078}%
\pgfsetfillcolor{currentfill}%
\pgfsetlinewidth{1.003750pt}%
\definecolor{currentstroke}{rgb}{1.000000,0.752941,0.796078}%
\pgfsetstrokecolor{currentstroke}%
\pgfsetdash{}{0pt}%
\pgfpathmoveto{\pgfqpoint{0.795050in}{0.669040in}}%
\pgfpathcurveto{\pgfqpoint{0.806100in}{0.669040in}}{\pgfqpoint{0.816699in}{0.673431in}}{\pgfqpoint{0.824513in}{0.681244in}}%
\pgfpathcurveto{\pgfqpoint{0.832326in}{0.689058in}}{\pgfqpoint{0.836716in}{0.699657in}}{\pgfqpoint{0.836716in}{0.710707in}}%
\pgfpathcurveto{\pgfqpoint{0.836716in}{0.721757in}}{\pgfqpoint{0.832326in}{0.732356in}}{\pgfqpoint{0.824513in}{0.740170in}}%
\pgfpathcurveto{\pgfqpoint{0.816699in}{0.747983in}}{\pgfqpoint{0.806100in}{0.752374in}}{\pgfqpoint{0.795050in}{0.752374in}}%
\pgfpathcurveto{\pgfqpoint{0.784000in}{0.752374in}}{\pgfqpoint{0.773401in}{0.747983in}}{\pgfqpoint{0.765587in}{0.740170in}}%
\pgfpathcurveto{\pgfqpoint{0.757773in}{0.732356in}}{\pgfqpoint{0.753383in}{0.721757in}}{\pgfqpoint{0.753383in}{0.710707in}}%
\pgfpathcurveto{\pgfqpoint{0.753383in}{0.699657in}}{\pgfqpoint{0.757773in}{0.689058in}}{\pgfqpoint{0.765587in}{0.681244in}}%
\pgfpathcurveto{\pgfqpoint{0.773401in}{0.673431in}}{\pgfqpoint{0.784000in}{0.669040in}}{\pgfqpoint{0.795050in}{0.669040in}}%
\pgfpathclose%
\pgfusepath{stroke,fill}%
\end{pgfscope}%
\begin{pgfscope}%
\pgfpathrectangle{\pgfqpoint{0.444137in}{0.319877in}}{\pgfqpoint{1.544018in}{2.605531in}} %
\pgfusepath{clip}%
\pgfsetbuttcap%
\pgfsetroundjoin%
\definecolor{currentfill}{rgb}{1.000000,0.752941,0.796078}%
\pgfsetfillcolor{currentfill}%
\pgfsetlinewidth{1.003750pt}%
\definecolor{currentstroke}{rgb}{1.000000,0.752941,0.796078}%
\pgfsetstrokecolor{currentstroke}%
\pgfsetdash{}{0pt}%
\pgfpathmoveto{\pgfqpoint{0.935415in}{0.951306in}}%
\pgfpathcurveto{\pgfqpoint{0.946465in}{0.951306in}}{\pgfqpoint{0.957064in}{0.955696in}}{\pgfqpoint{0.964878in}{0.963510in}}%
\pgfpathcurveto{\pgfqpoint{0.972691in}{0.971324in}}{\pgfqpoint{0.977082in}{0.981923in}}{\pgfqpoint{0.977082in}{0.992973in}}%
\pgfpathcurveto{\pgfqpoint{0.977082in}{1.004023in}}{\pgfqpoint{0.972691in}{1.014622in}}{\pgfqpoint{0.964878in}{1.022436in}}%
\pgfpathcurveto{\pgfqpoint{0.957064in}{1.030249in}}{\pgfqpoint{0.946465in}{1.034639in}}{\pgfqpoint{0.935415in}{1.034639in}}%
\pgfpathcurveto{\pgfqpoint{0.924365in}{1.034639in}}{\pgfqpoint{0.913766in}{1.030249in}}{\pgfqpoint{0.905952in}{1.022436in}}%
\pgfpathcurveto{\pgfqpoint{0.898139in}{1.014622in}}{\pgfqpoint{0.893748in}{1.004023in}}{\pgfqpoint{0.893748in}{0.992973in}}%
\pgfpathcurveto{\pgfqpoint{0.893748in}{0.981923in}}{\pgfqpoint{0.898139in}{0.971324in}}{\pgfqpoint{0.905952in}{0.963510in}}%
\pgfpathcurveto{\pgfqpoint{0.913766in}{0.955696in}}{\pgfqpoint{0.924365in}{0.951306in}}{\pgfqpoint{0.935415in}{0.951306in}}%
\pgfpathclose%
\pgfusepath{stroke,fill}%
\end{pgfscope}%
\begin{pgfscope}%
\pgfpathrectangle{\pgfqpoint{0.444137in}{0.319877in}}{\pgfqpoint{1.544018in}{2.605531in}} %
\pgfusepath{clip}%
\pgfsetbuttcap%
\pgfsetroundjoin%
\definecolor{currentfill}{rgb}{1.000000,0.752941,0.796078}%
\pgfsetfillcolor{currentfill}%
\pgfsetlinewidth{1.003750pt}%
\definecolor{currentstroke}{rgb}{1.000000,0.752941,0.796078}%
\pgfsetstrokecolor{currentstroke}%
\pgfsetdash{}{0pt}%
\pgfpathmoveto{\pgfqpoint{1.075780in}{0.872860in}}%
\pgfpathcurveto{\pgfqpoint{1.086830in}{0.872860in}}{\pgfqpoint{1.097429in}{0.877250in}}{\pgfqpoint{1.105243in}{0.885064in}}%
\pgfpathcurveto{\pgfqpoint{1.113057in}{0.892877in}}{\pgfqpoint{1.117447in}{0.903477in}}{\pgfqpoint{1.117447in}{0.914527in}}%
\pgfpathcurveto{\pgfqpoint{1.117447in}{0.925577in}}{\pgfqpoint{1.113057in}{0.936176in}}{\pgfqpoint{1.105243in}{0.943989in}}%
\pgfpathcurveto{\pgfqpoint{1.097429in}{0.951803in}}{\pgfqpoint{1.086830in}{0.956193in}}{\pgfqpoint{1.075780in}{0.956193in}}%
\pgfpathcurveto{\pgfqpoint{1.064730in}{0.956193in}}{\pgfqpoint{1.054131in}{0.951803in}}{\pgfqpoint{1.046318in}{0.943989in}}%
\pgfpathcurveto{\pgfqpoint{1.038504in}{0.936176in}}{\pgfqpoint{1.034114in}{0.925577in}}{\pgfqpoint{1.034114in}{0.914527in}}%
\pgfpathcurveto{\pgfqpoint{1.034114in}{0.903477in}}{\pgfqpoint{1.038504in}{0.892877in}}{\pgfqpoint{1.046318in}{0.885064in}}%
\pgfpathcurveto{\pgfqpoint{1.054131in}{0.877250in}}{\pgfqpoint{1.064730in}{0.872860in}}{\pgfqpoint{1.075780in}{0.872860in}}%
\pgfpathclose%
\pgfusepath{stroke,fill}%
\end{pgfscope}%
\begin{pgfscope}%
\pgfpathrectangle{\pgfqpoint{0.444137in}{0.319877in}}{\pgfqpoint{1.544018in}{2.605531in}} %
\pgfusepath{clip}%
\pgfsetbuttcap%
\pgfsetroundjoin%
\definecolor{currentfill}{rgb}{1.000000,0.752941,0.796078}%
\pgfsetfillcolor{currentfill}%
\pgfsetlinewidth{1.003750pt}%
\definecolor{currentstroke}{rgb}{1.000000,0.752941,0.796078}%
\pgfsetstrokecolor{currentstroke}%
\pgfsetdash{}{0pt}%
\pgfpathmoveto{\pgfqpoint{1.216146in}{0.849920in}}%
\pgfpathcurveto{\pgfqpoint{1.227196in}{0.849920in}}{\pgfqpoint{1.237795in}{0.854310in}}{\pgfqpoint{1.245608in}{0.862124in}}%
\pgfpathcurveto{\pgfqpoint{1.253422in}{0.869938in}}{\pgfqpoint{1.257812in}{0.880537in}}{\pgfqpoint{1.257812in}{0.891587in}}%
\pgfpathcurveto{\pgfqpoint{1.257812in}{0.902637in}}{\pgfqpoint{1.253422in}{0.913236in}}{\pgfqpoint{1.245608in}{0.921050in}}%
\pgfpathcurveto{\pgfqpoint{1.237795in}{0.928863in}}{\pgfqpoint{1.227196in}{0.933253in}}{\pgfqpoint{1.216146in}{0.933253in}}%
\pgfpathcurveto{\pgfqpoint{1.205095in}{0.933253in}}{\pgfqpoint{1.194496in}{0.928863in}}{\pgfqpoint{1.186683in}{0.921050in}}%
\pgfpathcurveto{\pgfqpoint{1.178869in}{0.913236in}}{\pgfqpoint{1.174479in}{0.902637in}}{\pgfqpoint{1.174479in}{0.891587in}}%
\pgfpathcurveto{\pgfqpoint{1.174479in}{0.880537in}}{\pgfqpoint{1.178869in}{0.869938in}}{\pgfqpoint{1.186683in}{0.862124in}}%
\pgfpathcurveto{\pgfqpoint{1.194496in}{0.854310in}}{\pgfqpoint{1.205095in}{0.849920in}}{\pgfqpoint{1.216146in}{0.849920in}}%
\pgfpathclose%
\pgfusepath{stroke,fill}%
\end{pgfscope}%
\begin{pgfscope}%
\pgfpathrectangle{\pgfqpoint{0.444137in}{0.319877in}}{\pgfqpoint{1.544018in}{2.605531in}} %
\pgfusepath{clip}%
\pgfsetbuttcap%
\pgfsetroundjoin%
\definecolor{currentfill}{rgb}{1.000000,0.752941,0.796078}%
\pgfsetfillcolor{currentfill}%
\pgfsetlinewidth{1.003750pt}%
\definecolor{currentstroke}{rgb}{1.000000,0.752941,0.796078}%
\pgfsetstrokecolor{currentstroke}%
\pgfsetdash{}{0pt}%
\pgfpathmoveto{\pgfqpoint{1.356511in}{0.918737in}}%
\pgfpathcurveto{\pgfqpoint{1.367561in}{0.918737in}}{\pgfqpoint{1.378160in}{0.923127in}}{\pgfqpoint{1.385974in}{0.930941in}}%
\pgfpathcurveto{\pgfqpoint{1.393787in}{0.938754in}}{\pgfqpoint{1.398178in}{0.949353in}}{\pgfqpoint{1.398178in}{0.960404in}}%
\pgfpathcurveto{\pgfqpoint{1.398178in}{0.971454in}}{\pgfqpoint{1.393787in}{0.982053in}}{\pgfqpoint{1.385974in}{0.989866in}}%
\pgfpathcurveto{\pgfqpoint{1.378160in}{0.997680in}}{\pgfqpoint{1.367561in}{1.002070in}}{\pgfqpoint{1.356511in}{1.002070in}}%
\pgfpathcurveto{\pgfqpoint{1.345461in}{1.002070in}}{\pgfqpoint{1.334862in}{0.997680in}}{\pgfqpoint{1.327048in}{0.989866in}}%
\pgfpathcurveto{\pgfqpoint{1.319234in}{0.982053in}}{\pgfqpoint{1.314844in}{0.971454in}}{\pgfqpoint{1.314844in}{0.960404in}}%
\pgfpathcurveto{\pgfqpoint{1.314844in}{0.949353in}}{\pgfqpoint{1.319234in}{0.938754in}}{\pgfqpoint{1.327048in}{0.930941in}}%
\pgfpathcurveto{\pgfqpoint{1.334862in}{0.923127in}}{\pgfqpoint{1.345461in}{0.918737in}}{\pgfqpoint{1.356511in}{0.918737in}}%
\pgfpathclose%
\pgfusepath{stroke,fill}%
\end{pgfscope}%
\begin{pgfscope}%
\pgfpathrectangle{\pgfqpoint{0.444137in}{0.319877in}}{\pgfqpoint{1.544018in}{2.605531in}} %
\pgfusepath{clip}%
\pgfsetbuttcap%
\pgfsetroundjoin%
\definecolor{currentfill}{rgb}{1.000000,0.752941,0.796078}%
\pgfsetfillcolor{currentfill}%
\pgfsetlinewidth{1.003750pt}%
\definecolor{currentstroke}{rgb}{1.000000,0.752941,0.796078}%
\pgfsetstrokecolor{currentstroke}%
\pgfsetdash{}{0pt}%
\pgfpathmoveto{\pgfqpoint{1.496876in}{1.162230in}}%
\pgfpathcurveto{\pgfqpoint{1.507926in}{1.162230in}}{\pgfqpoint{1.518525in}{1.166620in}}{\pgfqpoint{1.526339in}{1.174434in}}%
\pgfpathcurveto{\pgfqpoint{1.534153in}{1.182247in}}{\pgfqpoint{1.538543in}{1.192847in}}{\pgfqpoint{1.538543in}{1.203897in}}%
\pgfpathcurveto{\pgfqpoint{1.538543in}{1.214947in}}{\pgfqpoint{1.534153in}{1.225546in}}{\pgfqpoint{1.526339in}{1.233359in}}%
\pgfpathcurveto{\pgfqpoint{1.518525in}{1.241173in}}{\pgfqpoint{1.507926in}{1.245563in}}{\pgfqpoint{1.496876in}{1.245563in}}%
\pgfpathcurveto{\pgfqpoint{1.485826in}{1.245563in}}{\pgfqpoint{1.475227in}{1.241173in}}{\pgfqpoint{1.467413in}{1.233359in}}%
\pgfpathcurveto{\pgfqpoint{1.459600in}{1.225546in}}{\pgfqpoint{1.455209in}{1.214947in}}{\pgfqpoint{1.455209in}{1.203897in}}%
\pgfpathcurveto{\pgfqpoint{1.455209in}{1.192847in}}{\pgfqpoint{1.459600in}{1.182247in}}{\pgfqpoint{1.467413in}{1.174434in}}%
\pgfpathcurveto{\pgfqpoint{1.475227in}{1.166620in}}{\pgfqpoint{1.485826in}{1.162230in}}{\pgfqpoint{1.496876in}{1.162230in}}%
\pgfpathclose%
\pgfusepath{stroke,fill}%
\end{pgfscope}%
\begin{pgfscope}%
\pgfsetbuttcap%
\pgfsetroundjoin%
\definecolor{currentfill}{rgb}{0.000000,0.000000,0.000000}%
\pgfsetfillcolor{currentfill}%
\pgfsetlinewidth{0.803000pt}%
\definecolor{currentstroke}{rgb}{0.000000,0.000000,0.000000}%
\pgfsetstrokecolor{currentstroke}%
\pgfsetdash{}{0pt}%
\pgfsys@defobject{currentmarker}{\pgfqpoint{0.000000in}{-0.048611in}}{\pgfqpoint{0.000000in}{0.000000in}}{%
\pgfpathmoveto{\pgfqpoint{0.000000in}{0.000000in}}%
\pgfpathlineto{\pgfqpoint{0.000000in}{-0.048611in}}%
\pgfusepath{stroke,fill}%
}%
\begin{pgfscope}%
\pgfsys@transformshift{0.733640in}{0.319877in}%
\pgfsys@useobject{currentmarker}{}%
\end{pgfscope}%
\end{pgfscope}%
\begin{pgfscope}%
\pgftext[x=0.733640in,y=0.222655in,,top]{\rmfamily\fontsize{10.000000}{12.000000}\selectfont \(\displaystyle -0.05\)}%
\end{pgfscope}%
\begin{pgfscope}%
\pgfsetbuttcap%
\pgfsetroundjoin%
\definecolor{currentfill}{rgb}{0.000000,0.000000,0.000000}%
\pgfsetfillcolor{currentfill}%
\pgfsetlinewidth{0.803000pt}%
\definecolor{currentstroke}{rgb}{0.000000,0.000000,0.000000}%
\pgfsetstrokecolor{currentstroke}%
\pgfsetdash{}{0pt}%
\pgfsys@defobject{currentmarker}{\pgfqpoint{0.000000in}{-0.048611in}}{\pgfqpoint{0.000000in}{0.000000in}}{%
\pgfpathmoveto{\pgfqpoint{0.000000in}{0.000000in}}%
\pgfpathlineto{\pgfqpoint{0.000000in}{-0.048611in}}%
\pgfusepath{stroke,fill}%
}%
\begin{pgfscope}%
\pgfsys@transformshift{1.216146in}{0.319877in}%
\pgfsys@useobject{currentmarker}{}%
\end{pgfscope}%
\end{pgfscope}%
\begin{pgfscope}%
\pgftext[x=1.216146in,y=0.222655in,,top]{\rmfamily\fontsize{10.000000}{12.000000}\selectfont \(\displaystyle 0.00\)}%
\end{pgfscope}%
\begin{pgfscope}%
\pgfsetbuttcap%
\pgfsetroundjoin%
\definecolor{currentfill}{rgb}{0.000000,0.000000,0.000000}%
\pgfsetfillcolor{currentfill}%
\pgfsetlinewidth{0.803000pt}%
\definecolor{currentstroke}{rgb}{0.000000,0.000000,0.000000}%
\pgfsetstrokecolor{currentstroke}%
\pgfsetdash{}{0pt}%
\pgfsys@defobject{currentmarker}{\pgfqpoint{0.000000in}{-0.048611in}}{\pgfqpoint{0.000000in}{0.000000in}}{%
\pgfpathmoveto{\pgfqpoint{0.000000in}{0.000000in}}%
\pgfpathlineto{\pgfqpoint{0.000000in}{-0.048611in}}%
\pgfusepath{stroke,fill}%
}%
\begin{pgfscope}%
\pgfsys@transformshift{1.698651in}{0.319877in}%
\pgfsys@useobject{currentmarker}{}%
\end{pgfscope}%
\end{pgfscope}%
\begin{pgfscope}%
\pgftext[x=1.698651in,y=0.222655in,,top]{\rmfamily\fontsize{10.000000}{12.000000}\selectfont \(\displaystyle 0.05\)}%
\end{pgfscope}%
\begin{pgfscope}%
\pgfsetbuttcap%
\pgfsetroundjoin%
\definecolor{currentfill}{rgb}{0.000000,0.000000,0.000000}%
\pgfsetfillcolor{currentfill}%
\pgfsetlinewidth{0.803000pt}%
\definecolor{currentstroke}{rgb}{0.000000,0.000000,0.000000}%
\pgfsetstrokecolor{currentstroke}%
\pgfsetdash{}{0pt}%
\pgfsys@defobject{currentmarker}{\pgfqpoint{-0.048611in}{0.000000in}}{\pgfqpoint{0.000000in}{0.000000in}}{%
\pgfpathmoveto{\pgfqpoint{0.000000in}{0.000000in}}%
\pgfpathlineto{\pgfqpoint{-0.048611in}{0.000000in}}%
\pgfusepath{stroke,fill}%
}%
\begin{pgfscope}%
\pgfsys@transformshift{0.444137in}{0.319877in}%
\pgfsys@useobject{currentmarker}{}%
\end{pgfscope}%
\end{pgfscope}%
\begin{pgfscope}%
\pgftext[x=0.100000in,y=0.272050in,left,base]{\rmfamily\fontsize{10.000000}{12.000000}\selectfont \(\displaystyle 0.00\)}%
\end{pgfscope}%
\begin{pgfscope}%
\pgfsetbuttcap%
\pgfsetroundjoin%
\definecolor{currentfill}{rgb}{0.000000,0.000000,0.000000}%
\pgfsetfillcolor{currentfill}%
\pgfsetlinewidth{0.803000pt}%
\definecolor{currentstroke}{rgb}{0.000000,0.000000,0.000000}%
\pgfsetstrokecolor{currentstroke}%
\pgfsetdash{}{0pt}%
\pgfsys@defobject{currentmarker}{\pgfqpoint{-0.048611in}{0.000000in}}{\pgfqpoint{0.000000in}{0.000000in}}{%
\pgfpathmoveto{\pgfqpoint{0.000000in}{0.000000in}}%
\pgfpathlineto{\pgfqpoint{-0.048611in}{0.000000in}}%
\pgfusepath{stroke,fill}%
}%
\begin{pgfscope}%
\pgfsys@transformshift{0.444137in}{0.802383in}%
\pgfsys@useobject{currentmarker}{}%
\end{pgfscope}%
\end{pgfscope}%
\begin{pgfscope}%
\pgftext[x=0.100000in,y=0.754555in,left,base]{\rmfamily\fontsize{10.000000}{12.000000}\selectfont \(\displaystyle 0.05\)}%
\end{pgfscope}%
\begin{pgfscope}%
\pgfsetbuttcap%
\pgfsetroundjoin%
\definecolor{currentfill}{rgb}{0.000000,0.000000,0.000000}%
\pgfsetfillcolor{currentfill}%
\pgfsetlinewidth{0.803000pt}%
\definecolor{currentstroke}{rgb}{0.000000,0.000000,0.000000}%
\pgfsetstrokecolor{currentstroke}%
\pgfsetdash{}{0pt}%
\pgfsys@defobject{currentmarker}{\pgfqpoint{-0.048611in}{0.000000in}}{\pgfqpoint{0.000000in}{0.000000in}}{%
\pgfpathmoveto{\pgfqpoint{0.000000in}{0.000000in}}%
\pgfpathlineto{\pgfqpoint{-0.048611in}{0.000000in}}%
\pgfusepath{stroke,fill}%
}%
\begin{pgfscope}%
\pgfsys@transformshift{0.444137in}{1.284889in}%
\pgfsys@useobject{currentmarker}{}%
\end{pgfscope}%
\end{pgfscope}%
\begin{pgfscope}%
\pgftext[x=0.100000in,y=1.237061in,left,base]{\rmfamily\fontsize{10.000000}{12.000000}\selectfont \(\displaystyle 0.10\)}%
\end{pgfscope}%
\begin{pgfscope}%
\pgfsetbuttcap%
\pgfsetroundjoin%
\definecolor{currentfill}{rgb}{0.000000,0.000000,0.000000}%
\pgfsetfillcolor{currentfill}%
\pgfsetlinewidth{0.803000pt}%
\definecolor{currentstroke}{rgb}{0.000000,0.000000,0.000000}%
\pgfsetstrokecolor{currentstroke}%
\pgfsetdash{}{0pt}%
\pgfsys@defobject{currentmarker}{\pgfqpoint{-0.048611in}{0.000000in}}{\pgfqpoint{0.000000in}{0.000000in}}{%
\pgfpathmoveto{\pgfqpoint{0.000000in}{0.000000in}}%
\pgfpathlineto{\pgfqpoint{-0.048611in}{0.000000in}}%
\pgfusepath{stroke,fill}%
}%
\begin{pgfscope}%
\pgfsys@transformshift{0.444137in}{1.767394in}%
\pgfsys@useobject{currentmarker}{}%
\end{pgfscope}%
\end{pgfscope}%
\begin{pgfscope}%
\pgftext[x=0.100000in,y=1.719567in,left,base]{\rmfamily\fontsize{10.000000}{12.000000}\selectfont \(\displaystyle 0.15\)}%
\end{pgfscope}%
\begin{pgfscope}%
\pgfsetbuttcap%
\pgfsetroundjoin%
\definecolor{currentfill}{rgb}{0.000000,0.000000,0.000000}%
\pgfsetfillcolor{currentfill}%
\pgfsetlinewidth{0.803000pt}%
\definecolor{currentstroke}{rgb}{0.000000,0.000000,0.000000}%
\pgfsetstrokecolor{currentstroke}%
\pgfsetdash{}{0pt}%
\pgfsys@defobject{currentmarker}{\pgfqpoint{-0.048611in}{0.000000in}}{\pgfqpoint{0.000000in}{0.000000in}}{%
\pgfpathmoveto{\pgfqpoint{0.000000in}{0.000000in}}%
\pgfpathlineto{\pgfqpoint{-0.048611in}{0.000000in}}%
\pgfusepath{stroke,fill}%
}%
\begin{pgfscope}%
\pgfsys@transformshift{0.444137in}{2.249900in}%
\pgfsys@useobject{currentmarker}{}%
\end{pgfscope}%
\end{pgfscope}%
\begin{pgfscope}%
\pgftext[x=0.100000in,y=2.202072in,left,base]{\rmfamily\fontsize{10.000000}{12.000000}\selectfont \(\displaystyle 0.20\)}%
\end{pgfscope}%
\begin{pgfscope}%
\pgfsetbuttcap%
\pgfsetroundjoin%
\definecolor{currentfill}{rgb}{0.000000,0.000000,0.000000}%
\pgfsetfillcolor{currentfill}%
\pgfsetlinewidth{0.803000pt}%
\definecolor{currentstroke}{rgb}{0.000000,0.000000,0.000000}%
\pgfsetstrokecolor{currentstroke}%
\pgfsetdash{}{0pt}%
\pgfsys@defobject{currentmarker}{\pgfqpoint{-0.048611in}{0.000000in}}{\pgfqpoint{0.000000in}{0.000000in}}{%
\pgfpathmoveto{\pgfqpoint{0.000000in}{0.000000in}}%
\pgfpathlineto{\pgfqpoint{-0.048611in}{0.000000in}}%
\pgfusepath{stroke,fill}%
}%
\begin{pgfscope}%
\pgfsys@transformshift{0.444137in}{2.732406in}%
\pgfsys@useobject{currentmarker}{}%
\end{pgfscope}%
\end{pgfscope}%
\begin{pgfscope}%
\pgftext[x=0.100000in,y=2.684578in,left,base]{\rmfamily\fontsize{10.000000}{12.000000}\selectfont \(\displaystyle 0.25\)}%
\end{pgfscope}%
\begin{pgfscope}%
\pgfsetrectcap%
\pgfsetmiterjoin%
\pgfsetlinewidth{0.803000pt}%
\definecolor{currentstroke}{rgb}{0.000000,0.000000,0.000000}%
\pgfsetstrokecolor{currentstroke}%
\pgfsetdash{}{0pt}%
\pgfpathmoveto{\pgfqpoint{0.444137in}{0.319877in}}%
\pgfpathlineto{\pgfqpoint{0.444137in}{2.925408in}}%
\pgfusepath{stroke}%
\end{pgfscope}%
\begin{pgfscope}%
\pgfsetrectcap%
\pgfsetmiterjoin%
\pgfsetlinewidth{0.803000pt}%
\definecolor{currentstroke}{rgb}{0.000000,0.000000,0.000000}%
\pgfsetstrokecolor{currentstroke}%
\pgfsetdash{}{0pt}%
\pgfpathmoveto{\pgfqpoint{1.988155in}{0.319877in}}%
\pgfpathlineto{\pgfqpoint{1.988155in}{2.925408in}}%
\pgfusepath{stroke}%
\end{pgfscope}%
\begin{pgfscope}%
\pgfsetrectcap%
\pgfsetmiterjoin%
\pgfsetlinewidth{0.803000pt}%
\definecolor{currentstroke}{rgb}{0.000000,0.000000,0.000000}%
\pgfsetstrokecolor{currentstroke}%
\pgfsetdash{}{0pt}%
\pgfpathmoveto{\pgfqpoint{0.444137in}{0.319877in}}%
\pgfpathlineto{\pgfqpoint{1.988155in}{0.319877in}}%
\pgfusepath{stroke}%
\end{pgfscope}%
\begin{pgfscope}%
\pgfsetrectcap%
\pgfsetmiterjoin%
\pgfsetlinewidth{0.803000pt}%
\definecolor{currentstroke}{rgb}{0.000000,0.000000,0.000000}%
\pgfsetstrokecolor{currentstroke}%
\pgfsetdash{}{0pt}%
\pgfpathmoveto{\pgfqpoint{0.444137in}{2.925408in}}%
\pgfpathlineto{\pgfqpoint{1.988155in}{2.925408in}}%
\pgfusepath{stroke}%
\end{pgfscope}%
\begin{pgfscope}%
\pgfpathrectangle{\pgfqpoint{2.072071in}{0.319877in}}{\pgfqpoint{0.130277in}{2.605531in}} %
\pgfusepath{clip}%
\pgfsetbuttcap%
\pgfsetmiterjoin%
\definecolor{currentfill}{rgb}{1.000000,1.000000,1.000000}%
\pgfsetfillcolor{currentfill}%
\pgfsetlinewidth{0.010037pt}%
\definecolor{currentstroke}{rgb}{1.000000,1.000000,1.000000}%
\pgfsetstrokecolor{currentstroke}%
\pgfsetdash{}{0pt}%
\pgfpathmoveto{\pgfqpoint{2.072071in}{0.319877in}}%
\pgfpathlineto{\pgfqpoint{2.072071in}{0.330055in}}%
\pgfpathlineto{\pgfqpoint{2.072071in}{2.915230in}}%
\pgfpathlineto{\pgfqpoint{2.072071in}{2.925408in}}%
\pgfpathlineto{\pgfqpoint{2.202347in}{2.925408in}}%
\pgfpathlineto{\pgfqpoint{2.202347in}{2.915230in}}%
\pgfpathlineto{\pgfqpoint{2.202347in}{0.330055in}}%
\pgfpathlineto{\pgfqpoint{2.202347in}{0.319877in}}%
\pgfpathclose%
\pgfusepath{stroke,fill}%
\end{pgfscope}%
\begin{pgfscope}%
\pgfsys@transformshift{2.070000in}{0.320408in}%
\pgftext[left,bottom]{\pgfimage[interpolate=true,width=0.130000in,height=2.610000in]{Ferr_vs_dq_Ti_300K-img1.png}}%
\end{pgfscope}%
\begin{pgfscope}%
\pgfsetbuttcap%
\pgfsetroundjoin%
\definecolor{currentfill}{rgb}{0.000000,0.000000,0.000000}%
\pgfsetfillcolor{currentfill}%
\pgfsetlinewidth{0.803000pt}%
\definecolor{currentstroke}{rgb}{0.000000,0.000000,0.000000}%
\pgfsetstrokecolor{currentstroke}%
\pgfsetdash{}{0pt}%
\pgfsys@defobject{currentmarker}{\pgfqpoint{0.000000in}{0.000000in}}{\pgfqpoint{0.048611in}{0.000000in}}{%
\pgfpathmoveto{\pgfqpoint{0.000000in}{0.000000in}}%
\pgfpathlineto{\pgfqpoint{0.048611in}{0.000000in}}%
\pgfusepath{stroke,fill}%
}%
\begin{pgfscope}%
\pgfsys@transformshift{2.202347in}{0.319877in}%
\pgfsys@useobject{currentmarker}{}%
\end{pgfscope}%
\end{pgfscope}%
\begin{pgfscope}%
\pgftext[x=2.299570in,y=0.272050in,left,base]{\rmfamily\fontsize{10.000000}{12.000000}\selectfont \(\displaystyle 0\)}%
\end{pgfscope}%
\begin{pgfscope}%
\pgfsetbuttcap%
\pgfsetroundjoin%
\definecolor{currentfill}{rgb}{0.000000,0.000000,0.000000}%
\pgfsetfillcolor{currentfill}%
\pgfsetlinewidth{0.803000pt}%
\definecolor{currentstroke}{rgb}{0.000000,0.000000,0.000000}%
\pgfsetstrokecolor{currentstroke}%
\pgfsetdash{}{0pt}%
\pgfsys@defobject{currentmarker}{\pgfqpoint{0.000000in}{0.000000in}}{\pgfqpoint{0.048611in}{0.000000in}}{%
\pgfpathmoveto{\pgfqpoint{0.000000in}{0.000000in}}%
\pgfpathlineto{\pgfqpoint{0.048611in}{0.000000in}}%
\pgfusepath{stroke,fill}%
}%
\begin{pgfscope}%
\pgfsys@transformshift{2.202347in}{0.862696in}%
\pgfsys@useobject{currentmarker}{}%
\end{pgfscope}%
\end{pgfscope}%
\begin{pgfscope}%
\pgftext[x=2.299570in,y=0.814868in,left,base]{\rmfamily\fontsize{10.000000}{12.000000}\selectfont \(\displaystyle 5\)}%
\end{pgfscope}%
\begin{pgfscope}%
\pgfsetbuttcap%
\pgfsetroundjoin%
\definecolor{currentfill}{rgb}{0.000000,0.000000,0.000000}%
\pgfsetfillcolor{currentfill}%
\pgfsetlinewidth{0.803000pt}%
\definecolor{currentstroke}{rgb}{0.000000,0.000000,0.000000}%
\pgfsetstrokecolor{currentstroke}%
\pgfsetdash{}{0pt}%
\pgfsys@defobject{currentmarker}{\pgfqpoint{0.000000in}{0.000000in}}{\pgfqpoint{0.048611in}{0.000000in}}{%
\pgfpathmoveto{\pgfqpoint{0.000000in}{0.000000in}}%
\pgfpathlineto{\pgfqpoint{0.048611in}{0.000000in}}%
\pgfusepath{stroke,fill}%
}%
\begin{pgfscope}%
\pgfsys@transformshift{2.202347in}{1.405515in}%
\pgfsys@useobject{currentmarker}{}%
\end{pgfscope}%
\end{pgfscope}%
\begin{pgfscope}%
\pgftext[x=2.299570in,y=1.357687in,left,base]{\rmfamily\fontsize{10.000000}{12.000000}\selectfont \(\displaystyle 10\)}%
\end{pgfscope}%
\begin{pgfscope}%
\pgfsetbuttcap%
\pgfsetroundjoin%
\definecolor{currentfill}{rgb}{0.000000,0.000000,0.000000}%
\pgfsetfillcolor{currentfill}%
\pgfsetlinewidth{0.803000pt}%
\definecolor{currentstroke}{rgb}{0.000000,0.000000,0.000000}%
\pgfsetstrokecolor{currentstroke}%
\pgfsetdash{}{0pt}%
\pgfsys@defobject{currentmarker}{\pgfqpoint{0.000000in}{0.000000in}}{\pgfqpoint{0.048611in}{0.000000in}}{%
\pgfpathmoveto{\pgfqpoint{0.000000in}{0.000000in}}%
\pgfpathlineto{\pgfqpoint{0.048611in}{0.000000in}}%
\pgfusepath{stroke,fill}%
}%
\begin{pgfscope}%
\pgfsys@transformshift{2.202347in}{1.948334in}%
\pgfsys@useobject{currentmarker}{}%
\end{pgfscope}%
\end{pgfscope}%
\begin{pgfscope}%
\pgftext[x=2.299570in,y=1.900506in,left,base]{\rmfamily\fontsize{10.000000}{12.000000}\selectfont \(\displaystyle 15\)}%
\end{pgfscope}%
\begin{pgfscope}%
\pgfsetbuttcap%
\pgfsetroundjoin%
\definecolor{currentfill}{rgb}{0.000000,0.000000,0.000000}%
\pgfsetfillcolor{currentfill}%
\pgfsetlinewidth{0.803000pt}%
\definecolor{currentstroke}{rgb}{0.000000,0.000000,0.000000}%
\pgfsetstrokecolor{currentstroke}%
\pgfsetdash{}{0pt}%
\pgfsys@defobject{currentmarker}{\pgfqpoint{0.000000in}{0.000000in}}{\pgfqpoint{0.048611in}{0.000000in}}{%
\pgfpathmoveto{\pgfqpoint{0.000000in}{0.000000in}}%
\pgfpathlineto{\pgfqpoint{0.048611in}{0.000000in}}%
\pgfusepath{stroke,fill}%
}%
\begin{pgfscope}%
\pgfsys@transformshift{2.202347in}{2.491153in}%
\pgfsys@useobject{currentmarker}{}%
\end{pgfscope}%
\end{pgfscope}%
\begin{pgfscope}%
\pgftext[x=2.299570in,y=2.443325in,left,base]{\rmfamily\fontsize{10.000000}{12.000000}\selectfont \(\displaystyle 20\)}%
\end{pgfscope}%
\begin{pgfscope}%
\pgfsetbuttcap%
\pgfsetmiterjoin%
\pgfsetlinewidth{0.803000pt}%
\definecolor{currentstroke}{rgb}{0.000000,0.000000,0.000000}%
\pgfsetstrokecolor{currentstroke}%
\pgfsetdash{}{0pt}%
\pgfpathmoveto{\pgfqpoint{2.072071in}{0.319877in}}%
\pgfpathlineto{\pgfqpoint{2.072071in}{0.330055in}}%
\pgfpathlineto{\pgfqpoint{2.072071in}{2.915230in}}%
\pgfpathlineto{\pgfqpoint{2.072071in}{2.925408in}}%
\pgfpathlineto{\pgfqpoint{2.202347in}{2.925408in}}%
\pgfpathlineto{\pgfqpoint{2.202347in}{2.915230in}}%
\pgfpathlineto{\pgfqpoint{2.202347in}{0.330055in}}%
\pgfpathlineto{\pgfqpoint{2.202347in}{0.319877in}}%
\pgfpathclose%
\pgfusepath{stroke}%
\end{pgfscope}%
\end{pgfpicture}%
\makeatother%
\endgroup%

    \vspace*{-0.4cm}
	\caption{300 K. Bin size $0.014e$}
	\end{subfigure}
	\hspace{0.6cm}
	\begin{subfigure}[b]{0.45\textwidth}
	\hspace*{-0.4cm}
	%% Creator: Matplotlib, PGF backend
%%
%% To include the figure in your LaTeX document, write
%%   \input{<filename>.pgf}
%%
%% Make sure the required packages are loaded in your preamble
%%   \usepackage{pgf}
%%
%% Figures using additional raster images can only be included by \input if
%% they are in the same directory as the main LaTeX file. For loading figures
%% from other directories you can use the `import` package
%%   \usepackage{import}
%% and then include the figures with
%%   \import{<path to file>}{<filename>.pgf}
%%
%% Matplotlib used the following preamble
%%   \usepackage[utf8x]{inputenc}
%%   \usepackage[T1]{fontenc}
%%
\begingroup%
\makeatletter%
\begin{pgfpicture}%
\pgfpathrectangle{\pgfpointorigin}{\pgfqpoint{2.538459in}{3.060408in}}%
\pgfusepath{use as bounding box, clip}%
\begin{pgfscope}%
\pgfsetbuttcap%
\pgfsetmiterjoin%
\definecolor{currentfill}{rgb}{1.000000,1.000000,1.000000}%
\pgfsetfillcolor{currentfill}%
\pgfsetlinewidth{0.000000pt}%
\definecolor{currentstroke}{rgb}{1.000000,1.000000,1.000000}%
\pgfsetstrokecolor{currentstroke}%
\pgfsetdash{}{0pt}%
\pgfpathmoveto{\pgfqpoint{0.000000in}{0.000000in}}%
\pgfpathlineto{\pgfqpoint{2.538459in}{0.000000in}}%
\pgfpathlineto{\pgfqpoint{2.538459in}{3.060408in}}%
\pgfpathlineto{\pgfqpoint{0.000000in}{3.060408in}}%
\pgfpathclose%
\pgfusepath{fill}%
\end{pgfscope}%
\begin{pgfscope}%
\pgfsetbuttcap%
\pgfsetmiterjoin%
\definecolor{currentfill}{rgb}{1.000000,1.000000,1.000000}%
\pgfsetfillcolor{currentfill}%
\pgfsetlinewidth{0.000000pt}%
\definecolor{currentstroke}{rgb}{0.000000,0.000000,0.000000}%
\pgfsetstrokecolor{currentstroke}%
\pgfsetstrokeopacity{0.000000}%
\pgfsetdash{}{0pt}%
\pgfpathmoveto{\pgfqpoint{0.444137in}{0.319877in}}%
\pgfpathlineto{\pgfqpoint{1.988155in}{0.319877in}}%
\pgfpathlineto{\pgfqpoint{1.988155in}{2.925408in}}%
\pgfpathlineto{\pgfqpoint{0.444137in}{2.925408in}}%
\pgfpathclose%
\pgfusepath{fill}%
\end{pgfscope}%
\begin{pgfscope}%
\pgfpathrectangle{\pgfqpoint{0.444137in}{0.319877in}}{\pgfqpoint{1.544018in}{2.605531in}} %
\pgfusepath{clip}%
\pgfsys@transformshift{0.444137in}{0.319877in}%
\pgftext[left,bottom]{\pgfimage[interpolate=true,width=1.550000in,height=2.610000in]{Ferr_vs_dq_Ti_500K-img0.png}}%
\end{pgfscope}%
\begin{pgfscope}%
\pgfpathrectangle{\pgfqpoint{0.444137in}{0.319877in}}{\pgfqpoint{1.544018in}{2.605531in}} %
\pgfusepath{clip}%
\pgfsetbuttcap%
\pgfsetroundjoin%
\definecolor{currentfill}{rgb}{1.000000,0.752941,0.796078}%
\pgfsetfillcolor{currentfill}%
\pgfsetlinewidth{1.003750pt}%
\definecolor{currentstroke}{rgb}{1.000000,0.752941,0.796078}%
\pgfsetstrokecolor{currentstroke}%
\pgfsetdash{}{0pt}%
\pgfpathmoveto{\pgfqpoint{0.733640in}{1.242257in}}%
\pgfpathcurveto{\pgfqpoint{0.744690in}{1.242257in}}{\pgfqpoint{0.755289in}{1.246647in}}{\pgfqpoint{0.763103in}{1.254461in}}%
\pgfpathcurveto{\pgfqpoint{0.770916in}{1.262274in}}{\pgfqpoint{0.775307in}{1.272874in}}{\pgfqpoint{0.775307in}{1.283924in}}%
\pgfpathcurveto{\pgfqpoint{0.775307in}{1.294974in}}{\pgfqpoint{0.770916in}{1.305573in}}{\pgfqpoint{0.763103in}{1.313386in}}%
\pgfpathcurveto{\pgfqpoint{0.755289in}{1.321200in}}{\pgfqpoint{0.744690in}{1.325590in}}{\pgfqpoint{0.733640in}{1.325590in}}%
\pgfpathcurveto{\pgfqpoint{0.722590in}{1.325590in}}{\pgfqpoint{0.711991in}{1.321200in}}{\pgfqpoint{0.704177in}{1.313386in}}%
\pgfpathcurveto{\pgfqpoint{0.696364in}{1.305573in}}{\pgfqpoint{0.691973in}{1.294974in}}{\pgfqpoint{0.691973in}{1.283924in}}%
\pgfpathcurveto{\pgfqpoint{0.691973in}{1.272874in}}{\pgfqpoint{0.696364in}{1.262274in}}{\pgfqpoint{0.704177in}{1.254461in}}%
\pgfpathcurveto{\pgfqpoint{0.711991in}{1.246647in}}{\pgfqpoint{0.722590in}{1.242257in}}{\pgfqpoint{0.733640in}{1.242257in}}%
\pgfpathclose%
\pgfusepath{stroke,fill}%
\end{pgfscope}%
\begin{pgfscope}%
\pgfpathrectangle{\pgfqpoint{0.444137in}{0.319877in}}{\pgfqpoint{1.544018in}{2.605531in}} %
\pgfusepath{clip}%
\pgfsetbuttcap%
\pgfsetroundjoin%
\definecolor{currentfill}{rgb}{1.000000,0.752941,0.796078}%
\pgfsetfillcolor{currentfill}%
\pgfsetlinewidth{1.003750pt}%
\definecolor{currentstroke}{rgb}{1.000000,0.752941,0.796078}%
\pgfsetstrokecolor{currentstroke}%
\pgfsetdash{}{0pt}%
\pgfpathmoveto{\pgfqpoint{0.926642in}{1.090523in}}%
\pgfpathcurveto{\pgfqpoint{0.937692in}{1.090523in}}{\pgfqpoint{0.948291in}{1.094913in}}{\pgfqpoint{0.956105in}{1.102727in}}%
\pgfpathcurveto{\pgfqpoint{0.963919in}{1.110541in}}{\pgfqpoint{0.968309in}{1.121140in}}{\pgfqpoint{0.968309in}{1.132190in}}%
\pgfpathcurveto{\pgfqpoint{0.968309in}{1.143240in}}{\pgfqpoint{0.963919in}{1.153839in}}{\pgfqpoint{0.956105in}{1.161653in}}%
\pgfpathcurveto{\pgfqpoint{0.948291in}{1.169466in}}{\pgfqpoint{0.937692in}{1.173856in}}{\pgfqpoint{0.926642in}{1.173856in}}%
\pgfpathcurveto{\pgfqpoint{0.915592in}{1.173856in}}{\pgfqpoint{0.904993in}{1.169466in}}{\pgfqpoint{0.897179in}{1.161653in}}%
\pgfpathcurveto{\pgfqpoint{0.889366in}{1.153839in}}{\pgfqpoint{0.884976in}{1.143240in}}{\pgfqpoint{0.884976in}{1.132190in}}%
\pgfpathcurveto{\pgfqpoint{0.884976in}{1.121140in}}{\pgfqpoint{0.889366in}{1.110541in}}{\pgfqpoint{0.897179in}{1.102727in}}%
\pgfpathcurveto{\pgfqpoint{0.904993in}{1.094913in}}{\pgfqpoint{0.915592in}{1.090523in}}{\pgfqpoint{0.926642in}{1.090523in}}%
\pgfpathclose%
\pgfusepath{stroke,fill}%
\end{pgfscope}%
\begin{pgfscope}%
\pgfpathrectangle{\pgfqpoint{0.444137in}{0.319877in}}{\pgfqpoint{1.544018in}{2.605531in}} %
\pgfusepath{clip}%
\pgfsetbuttcap%
\pgfsetroundjoin%
\definecolor{currentfill}{rgb}{1.000000,0.752941,0.796078}%
\pgfsetfillcolor{currentfill}%
\pgfsetlinewidth{1.003750pt}%
\definecolor{currentstroke}{rgb}{1.000000,0.752941,0.796078}%
\pgfsetstrokecolor{currentstroke}%
\pgfsetdash{}{0pt}%
\pgfpathmoveto{\pgfqpoint{1.119644in}{0.930979in}}%
\pgfpathcurveto{\pgfqpoint{1.130695in}{0.930979in}}{\pgfqpoint{1.141294in}{0.935370in}}{\pgfqpoint{1.149107in}{0.943183in}}%
\pgfpathcurveto{\pgfqpoint{1.156921in}{0.950997in}}{\pgfqpoint{1.161311in}{0.961596in}}{\pgfqpoint{1.161311in}{0.972646in}}%
\pgfpathcurveto{\pgfqpoint{1.161311in}{0.983696in}}{\pgfqpoint{1.156921in}{0.994295in}}{\pgfqpoint{1.149107in}{1.002109in}}%
\pgfpathcurveto{\pgfqpoint{1.141294in}{1.009922in}}{\pgfqpoint{1.130695in}{1.014313in}}{\pgfqpoint{1.119644in}{1.014313in}}%
\pgfpathcurveto{\pgfqpoint{1.108594in}{1.014313in}}{\pgfqpoint{1.097995in}{1.009922in}}{\pgfqpoint{1.090182in}{1.002109in}}%
\pgfpathcurveto{\pgfqpoint{1.082368in}{0.994295in}}{\pgfqpoint{1.077978in}{0.983696in}}{\pgfqpoint{1.077978in}{0.972646in}}%
\pgfpathcurveto{\pgfqpoint{1.077978in}{0.961596in}}{\pgfqpoint{1.082368in}{0.950997in}}{\pgfqpoint{1.090182in}{0.943183in}}%
\pgfpathcurveto{\pgfqpoint{1.097995in}{0.935370in}}{\pgfqpoint{1.108594in}{0.930979in}}{\pgfqpoint{1.119644in}{0.930979in}}%
\pgfpathclose%
\pgfusepath{stroke,fill}%
\end{pgfscope}%
\begin{pgfscope}%
\pgfpathrectangle{\pgfqpoint{0.444137in}{0.319877in}}{\pgfqpoint{1.544018in}{2.605531in}} %
\pgfusepath{clip}%
\pgfsetbuttcap%
\pgfsetroundjoin%
\definecolor{currentfill}{rgb}{1.000000,0.752941,0.796078}%
\pgfsetfillcolor{currentfill}%
\pgfsetlinewidth{1.003750pt}%
\definecolor{currentstroke}{rgb}{1.000000,0.752941,0.796078}%
\pgfsetstrokecolor{currentstroke}%
\pgfsetdash{}{0pt}%
\pgfpathmoveto{\pgfqpoint{1.312647in}{0.840952in}}%
\pgfpathcurveto{\pgfqpoint{1.323697in}{0.840952in}}{\pgfqpoint{1.334296in}{0.845342in}}{\pgfqpoint{1.342109in}{0.853155in}}%
\pgfpathcurveto{\pgfqpoint{1.349923in}{0.860969in}}{\pgfqpoint{1.354313in}{0.871568in}}{\pgfqpoint{1.354313in}{0.882618in}}%
\pgfpathcurveto{\pgfqpoint{1.354313in}{0.893668in}}{\pgfqpoint{1.349923in}{0.904267in}}{\pgfqpoint{1.342109in}{0.912081in}}%
\pgfpathcurveto{\pgfqpoint{1.334296in}{0.919895in}}{\pgfqpoint{1.323697in}{0.924285in}}{\pgfqpoint{1.312647in}{0.924285in}}%
\pgfpathcurveto{\pgfqpoint{1.301597in}{0.924285in}}{\pgfqpoint{1.290998in}{0.919895in}}{\pgfqpoint{1.283184in}{0.912081in}}%
\pgfpathcurveto{\pgfqpoint{1.275370in}{0.904267in}}{\pgfqpoint{1.270980in}{0.893668in}}{\pgfqpoint{1.270980in}{0.882618in}}%
\pgfpathcurveto{\pgfqpoint{1.270980in}{0.871568in}}{\pgfqpoint{1.275370in}{0.860969in}}{\pgfqpoint{1.283184in}{0.853155in}}%
\pgfpathcurveto{\pgfqpoint{1.290998in}{0.845342in}}{\pgfqpoint{1.301597in}{0.840952in}}{\pgfqpoint{1.312647in}{0.840952in}}%
\pgfpathclose%
\pgfusepath{stroke,fill}%
\end{pgfscope}%
\begin{pgfscope}%
\pgfpathrectangle{\pgfqpoint{0.444137in}{0.319877in}}{\pgfqpoint{1.544018in}{2.605531in}} %
\pgfusepath{clip}%
\pgfsetbuttcap%
\pgfsetroundjoin%
\definecolor{currentfill}{rgb}{1.000000,0.752941,0.796078}%
\pgfsetfillcolor{currentfill}%
\pgfsetlinewidth{1.003750pt}%
\definecolor{currentstroke}{rgb}{1.000000,0.752941,0.796078}%
\pgfsetstrokecolor{currentstroke}%
\pgfsetdash{}{0pt}%
\pgfpathmoveto{\pgfqpoint{1.505649in}{1.115703in}}%
\pgfpathcurveto{\pgfqpoint{1.516699in}{1.115703in}}{\pgfqpoint{1.527298in}{1.120093in}}{\pgfqpoint{1.535112in}{1.127907in}}%
\pgfpathcurveto{\pgfqpoint{1.542925in}{1.135720in}}{\pgfqpoint{1.547316in}{1.146319in}}{\pgfqpoint{1.547316in}{1.157369in}}%
\pgfpathcurveto{\pgfqpoint{1.547316in}{1.168419in}}{\pgfqpoint{1.542925in}{1.179018in}}{\pgfqpoint{1.535112in}{1.186832in}}%
\pgfpathcurveto{\pgfqpoint{1.527298in}{1.194646in}}{\pgfqpoint{1.516699in}{1.199036in}}{\pgfqpoint{1.505649in}{1.199036in}}%
\pgfpathcurveto{\pgfqpoint{1.494599in}{1.199036in}}{\pgfqpoint{1.484000in}{1.194646in}}{\pgfqpoint{1.476186in}{1.186832in}}%
\pgfpathcurveto{\pgfqpoint{1.468373in}{1.179018in}}{\pgfqpoint{1.463982in}{1.168419in}}{\pgfqpoint{1.463982in}{1.157369in}}%
\pgfpathcurveto{\pgfqpoint{1.463982in}{1.146319in}}{\pgfqpoint{1.468373in}{1.135720in}}{\pgfqpoint{1.476186in}{1.127907in}}%
\pgfpathcurveto{\pgfqpoint{1.484000in}{1.120093in}}{\pgfqpoint{1.494599in}{1.115703in}}{\pgfqpoint{1.505649in}{1.115703in}}%
\pgfpathclose%
\pgfusepath{stroke,fill}%
\end{pgfscope}%
\begin{pgfscope}%
\pgfpathrectangle{\pgfqpoint{0.444137in}{0.319877in}}{\pgfqpoint{1.544018in}{2.605531in}} %
\pgfusepath{clip}%
\pgfsetbuttcap%
\pgfsetroundjoin%
\definecolor{currentfill}{rgb}{1.000000,0.752941,0.796078}%
\pgfsetfillcolor{currentfill}%
\pgfsetlinewidth{1.003750pt}%
\definecolor{currentstroke}{rgb}{1.000000,0.752941,0.796078}%
\pgfsetstrokecolor{currentstroke}%
\pgfsetdash{}{0pt}%
\pgfpathmoveto{\pgfqpoint{1.698651in}{1.548407in}}%
\pgfpathcurveto{\pgfqpoint{1.709701in}{1.548407in}}{\pgfqpoint{1.720300in}{1.552797in}}{\pgfqpoint{1.728114in}{1.560611in}}%
\pgfpathcurveto{\pgfqpoint{1.735928in}{1.568424in}}{\pgfqpoint{1.740318in}{1.579023in}}{\pgfqpoint{1.740318in}{1.590073in}}%
\pgfpathcurveto{\pgfqpoint{1.740318in}{1.601124in}}{\pgfqpoint{1.735928in}{1.611723in}}{\pgfqpoint{1.728114in}{1.619536in}}%
\pgfpathcurveto{\pgfqpoint{1.720300in}{1.627350in}}{\pgfqpoint{1.709701in}{1.631740in}}{\pgfqpoint{1.698651in}{1.631740in}}%
\pgfpathcurveto{\pgfqpoint{1.687601in}{1.631740in}}{\pgfqpoint{1.677002in}{1.627350in}}{\pgfqpoint{1.669188in}{1.619536in}}%
\pgfpathcurveto{\pgfqpoint{1.661375in}{1.611723in}}{\pgfqpoint{1.656985in}{1.601124in}}{\pgfqpoint{1.656985in}{1.590073in}}%
\pgfpathcurveto{\pgfqpoint{1.656985in}{1.579023in}}{\pgfqpoint{1.661375in}{1.568424in}}{\pgfqpoint{1.669188in}{1.560611in}}%
\pgfpathcurveto{\pgfqpoint{1.677002in}{1.552797in}}{\pgfqpoint{1.687601in}{1.548407in}}{\pgfqpoint{1.698651in}{1.548407in}}%
\pgfpathclose%
\pgfusepath{stroke,fill}%
\end{pgfscope}%
\begin{pgfscope}%
\pgfpathrectangle{\pgfqpoint{0.444137in}{0.319877in}}{\pgfqpoint{1.544018in}{2.605531in}} %
\pgfusepath{clip}%
\pgfsetbuttcap%
\pgfsetroundjoin%
\definecolor{currentfill}{rgb}{1.000000,0.752941,0.796078}%
\pgfsetfillcolor{currentfill}%
\pgfsetlinewidth{1.003750pt}%
\definecolor{currentstroke}{rgb}{1.000000,0.752941,0.796078}%
\pgfsetstrokecolor{currentstroke}%
\pgfsetdash{}{0pt}%
\pgfpathmoveto{\pgfqpoint{1.891653in}{2.102082in}}%
\pgfpathcurveto{\pgfqpoint{1.902704in}{2.102082in}}{\pgfqpoint{1.913303in}{2.106472in}}{\pgfqpoint{1.921116in}{2.114286in}}%
\pgfpathcurveto{\pgfqpoint{1.928930in}{2.122100in}}{\pgfqpoint{1.933320in}{2.132699in}}{\pgfqpoint{1.933320in}{2.143749in}}%
\pgfpathcurveto{\pgfqpoint{1.933320in}{2.154799in}}{\pgfqpoint{1.928930in}{2.165398in}}{\pgfqpoint{1.921116in}{2.173211in}}%
\pgfpathcurveto{\pgfqpoint{1.913303in}{2.181025in}}{\pgfqpoint{1.902704in}{2.185415in}}{\pgfqpoint{1.891653in}{2.185415in}}%
\pgfpathcurveto{\pgfqpoint{1.880603in}{2.185415in}}{\pgfqpoint{1.870004in}{2.181025in}}{\pgfqpoint{1.862191in}{2.173211in}}%
\pgfpathcurveto{\pgfqpoint{1.854377in}{2.165398in}}{\pgfqpoint{1.849987in}{2.154799in}}{\pgfqpoint{1.849987in}{2.143749in}}%
\pgfpathcurveto{\pgfqpoint{1.849987in}{2.132699in}}{\pgfqpoint{1.854377in}{2.122100in}}{\pgfqpoint{1.862191in}{2.114286in}}%
\pgfpathcurveto{\pgfqpoint{1.870004in}{2.106472in}}{\pgfqpoint{1.880603in}{2.102082in}}{\pgfqpoint{1.891653in}{2.102082in}}%
\pgfpathclose%
\pgfusepath{stroke,fill}%
\end{pgfscope}%
\begin{pgfscope}%
\pgfsetbuttcap%
\pgfsetroundjoin%
\definecolor{currentfill}{rgb}{0.000000,0.000000,0.000000}%
\pgfsetfillcolor{currentfill}%
\pgfsetlinewidth{0.803000pt}%
\definecolor{currentstroke}{rgb}{0.000000,0.000000,0.000000}%
\pgfsetstrokecolor{currentstroke}%
\pgfsetdash{}{0pt}%
\pgfsys@defobject{currentmarker}{\pgfqpoint{0.000000in}{-0.048611in}}{\pgfqpoint{0.000000in}{0.000000in}}{%
\pgfpathmoveto{\pgfqpoint{0.000000in}{0.000000in}}%
\pgfpathlineto{\pgfqpoint{0.000000in}{-0.048611in}}%
\pgfusepath{stroke,fill}%
}%
\begin{pgfscope}%
\pgfsys@transformshift{0.733640in}{0.319877in}%
\pgfsys@useobject{currentmarker}{}%
\end{pgfscope}%
\end{pgfscope}%
\begin{pgfscope}%
\pgftext[x=0.733640in,y=0.222655in,,top]{\rmfamily\fontsize{10.000000}{12.000000}\selectfont \(\displaystyle -0.05\)}%
\end{pgfscope}%
\begin{pgfscope}%
\pgfsetbuttcap%
\pgfsetroundjoin%
\definecolor{currentfill}{rgb}{0.000000,0.000000,0.000000}%
\pgfsetfillcolor{currentfill}%
\pgfsetlinewidth{0.803000pt}%
\definecolor{currentstroke}{rgb}{0.000000,0.000000,0.000000}%
\pgfsetstrokecolor{currentstroke}%
\pgfsetdash{}{0pt}%
\pgfsys@defobject{currentmarker}{\pgfqpoint{0.000000in}{-0.048611in}}{\pgfqpoint{0.000000in}{0.000000in}}{%
\pgfpathmoveto{\pgfqpoint{0.000000in}{0.000000in}}%
\pgfpathlineto{\pgfqpoint{0.000000in}{-0.048611in}}%
\pgfusepath{stroke,fill}%
}%
\begin{pgfscope}%
\pgfsys@transformshift{1.216146in}{0.319877in}%
\pgfsys@useobject{currentmarker}{}%
\end{pgfscope}%
\end{pgfscope}%
\begin{pgfscope}%
\pgftext[x=1.216146in,y=0.222655in,,top]{\rmfamily\fontsize{10.000000}{12.000000}\selectfont \(\displaystyle 0.00\)}%
\end{pgfscope}%
\begin{pgfscope}%
\pgfsetbuttcap%
\pgfsetroundjoin%
\definecolor{currentfill}{rgb}{0.000000,0.000000,0.000000}%
\pgfsetfillcolor{currentfill}%
\pgfsetlinewidth{0.803000pt}%
\definecolor{currentstroke}{rgb}{0.000000,0.000000,0.000000}%
\pgfsetstrokecolor{currentstroke}%
\pgfsetdash{}{0pt}%
\pgfsys@defobject{currentmarker}{\pgfqpoint{0.000000in}{-0.048611in}}{\pgfqpoint{0.000000in}{0.000000in}}{%
\pgfpathmoveto{\pgfqpoint{0.000000in}{0.000000in}}%
\pgfpathlineto{\pgfqpoint{0.000000in}{-0.048611in}}%
\pgfusepath{stroke,fill}%
}%
\begin{pgfscope}%
\pgfsys@transformshift{1.698651in}{0.319877in}%
\pgfsys@useobject{currentmarker}{}%
\end{pgfscope}%
\end{pgfscope}%
\begin{pgfscope}%
\pgftext[x=1.698651in,y=0.222655in,,top]{\rmfamily\fontsize{10.000000}{12.000000}\selectfont \(\displaystyle 0.05\)}%
\end{pgfscope}%
\begin{pgfscope}%
\pgfsetbuttcap%
\pgfsetroundjoin%
\definecolor{currentfill}{rgb}{0.000000,0.000000,0.000000}%
\pgfsetfillcolor{currentfill}%
\pgfsetlinewidth{0.803000pt}%
\definecolor{currentstroke}{rgb}{0.000000,0.000000,0.000000}%
\pgfsetstrokecolor{currentstroke}%
\pgfsetdash{}{0pt}%
\pgfsys@defobject{currentmarker}{\pgfqpoint{-0.048611in}{0.000000in}}{\pgfqpoint{0.000000in}{0.000000in}}{%
\pgfpathmoveto{\pgfqpoint{0.000000in}{0.000000in}}%
\pgfpathlineto{\pgfqpoint{-0.048611in}{0.000000in}}%
\pgfusepath{stroke,fill}%
}%
\begin{pgfscope}%
\pgfsys@transformshift{0.444137in}{0.319877in}%
\pgfsys@useobject{currentmarker}{}%
\end{pgfscope}%
\end{pgfscope}%
\begin{pgfscope}%
\pgftext[x=0.100000in,y=0.272050in,left,base]{\rmfamily\fontsize{10.000000}{12.000000}\selectfont \(\displaystyle 0.00\)}%
\end{pgfscope}%
\begin{pgfscope}%
\pgfsetbuttcap%
\pgfsetroundjoin%
\definecolor{currentfill}{rgb}{0.000000,0.000000,0.000000}%
\pgfsetfillcolor{currentfill}%
\pgfsetlinewidth{0.803000pt}%
\definecolor{currentstroke}{rgb}{0.000000,0.000000,0.000000}%
\pgfsetstrokecolor{currentstroke}%
\pgfsetdash{}{0pt}%
\pgfsys@defobject{currentmarker}{\pgfqpoint{-0.048611in}{0.000000in}}{\pgfqpoint{0.000000in}{0.000000in}}{%
\pgfpathmoveto{\pgfqpoint{0.000000in}{0.000000in}}%
\pgfpathlineto{\pgfqpoint{-0.048611in}{0.000000in}}%
\pgfusepath{stroke,fill}%
}%
\begin{pgfscope}%
\pgfsys@transformshift{0.444137in}{0.802383in}%
\pgfsys@useobject{currentmarker}{}%
\end{pgfscope}%
\end{pgfscope}%
\begin{pgfscope}%
\pgftext[x=0.100000in,y=0.754555in,left,base]{\rmfamily\fontsize{10.000000}{12.000000}\selectfont \(\displaystyle 0.05\)}%
\end{pgfscope}%
\begin{pgfscope}%
\pgfsetbuttcap%
\pgfsetroundjoin%
\definecolor{currentfill}{rgb}{0.000000,0.000000,0.000000}%
\pgfsetfillcolor{currentfill}%
\pgfsetlinewidth{0.803000pt}%
\definecolor{currentstroke}{rgb}{0.000000,0.000000,0.000000}%
\pgfsetstrokecolor{currentstroke}%
\pgfsetdash{}{0pt}%
\pgfsys@defobject{currentmarker}{\pgfqpoint{-0.048611in}{0.000000in}}{\pgfqpoint{0.000000in}{0.000000in}}{%
\pgfpathmoveto{\pgfqpoint{0.000000in}{0.000000in}}%
\pgfpathlineto{\pgfqpoint{-0.048611in}{0.000000in}}%
\pgfusepath{stroke,fill}%
}%
\begin{pgfscope}%
\pgfsys@transformshift{0.444137in}{1.284889in}%
\pgfsys@useobject{currentmarker}{}%
\end{pgfscope}%
\end{pgfscope}%
\begin{pgfscope}%
\pgftext[x=0.100000in,y=1.237061in,left,base]{\rmfamily\fontsize{10.000000}{12.000000}\selectfont \(\displaystyle 0.10\)}%
\end{pgfscope}%
\begin{pgfscope}%
\pgfsetbuttcap%
\pgfsetroundjoin%
\definecolor{currentfill}{rgb}{0.000000,0.000000,0.000000}%
\pgfsetfillcolor{currentfill}%
\pgfsetlinewidth{0.803000pt}%
\definecolor{currentstroke}{rgb}{0.000000,0.000000,0.000000}%
\pgfsetstrokecolor{currentstroke}%
\pgfsetdash{}{0pt}%
\pgfsys@defobject{currentmarker}{\pgfqpoint{-0.048611in}{0.000000in}}{\pgfqpoint{0.000000in}{0.000000in}}{%
\pgfpathmoveto{\pgfqpoint{0.000000in}{0.000000in}}%
\pgfpathlineto{\pgfqpoint{-0.048611in}{0.000000in}}%
\pgfusepath{stroke,fill}%
}%
\begin{pgfscope}%
\pgfsys@transformshift{0.444137in}{1.767394in}%
\pgfsys@useobject{currentmarker}{}%
\end{pgfscope}%
\end{pgfscope}%
\begin{pgfscope}%
\pgftext[x=0.100000in,y=1.719567in,left,base]{\rmfamily\fontsize{10.000000}{12.000000}\selectfont \(\displaystyle 0.15\)}%
\end{pgfscope}%
\begin{pgfscope}%
\pgfsetbuttcap%
\pgfsetroundjoin%
\definecolor{currentfill}{rgb}{0.000000,0.000000,0.000000}%
\pgfsetfillcolor{currentfill}%
\pgfsetlinewidth{0.803000pt}%
\definecolor{currentstroke}{rgb}{0.000000,0.000000,0.000000}%
\pgfsetstrokecolor{currentstroke}%
\pgfsetdash{}{0pt}%
\pgfsys@defobject{currentmarker}{\pgfqpoint{-0.048611in}{0.000000in}}{\pgfqpoint{0.000000in}{0.000000in}}{%
\pgfpathmoveto{\pgfqpoint{0.000000in}{0.000000in}}%
\pgfpathlineto{\pgfqpoint{-0.048611in}{0.000000in}}%
\pgfusepath{stroke,fill}%
}%
\begin{pgfscope}%
\pgfsys@transformshift{0.444137in}{2.249900in}%
\pgfsys@useobject{currentmarker}{}%
\end{pgfscope}%
\end{pgfscope}%
\begin{pgfscope}%
\pgftext[x=0.100000in,y=2.202072in,left,base]{\rmfamily\fontsize{10.000000}{12.000000}\selectfont \(\displaystyle 0.20\)}%
\end{pgfscope}%
\begin{pgfscope}%
\pgfsetbuttcap%
\pgfsetroundjoin%
\definecolor{currentfill}{rgb}{0.000000,0.000000,0.000000}%
\pgfsetfillcolor{currentfill}%
\pgfsetlinewidth{0.803000pt}%
\definecolor{currentstroke}{rgb}{0.000000,0.000000,0.000000}%
\pgfsetstrokecolor{currentstroke}%
\pgfsetdash{}{0pt}%
\pgfsys@defobject{currentmarker}{\pgfqpoint{-0.048611in}{0.000000in}}{\pgfqpoint{0.000000in}{0.000000in}}{%
\pgfpathmoveto{\pgfqpoint{0.000000in}{0.000000in}}%
\pgfpathlineto{\pgfqpoint{-0.048611in}{0.000000in}}%
\pgfusepath{stroke,fill}%
}%
\begin{pgfscope}%
\pgfsys@transformshift{0.444137in}{2.732406in}%
\pgfsys@useobject{currentmarker}{}%
\end{pgfscope}%
\end{pgfscope}%
\begin{pgfscope}%
\pgftext[x=0.100000in,y=2.684578in,left,base]{\rmfamily\fontsize{10.000000}{12.000000}\selectfont \(\displaystyle 0.25\)}%
\end{pgfscope}%
\begin{pgfscope}%
\pgfsetrectcap%
\pgfsetmiterjoin%
\pgfsetlinewidth{0.803000pt}%
\definecolor{currentstroke}{rgb}{0.000000,0.000000,0.000000}%
\pgfsetstrokecolor{currentstroke}%
\pgfsetdash{}{0pt}%
\pgfpathmoveto{\pgfqpoint{0.444137in}{0.319877in}}%
\pgfpathlineto{\pgfqpoint{0.444137in}{2.925408in}}%
\pgfusepath{stroke}%
\end{pgfscope}%
\begin{pgfscope}%
\pgfsetrectcap%
\pgfsetmiterjoin%
\pgfsetlinewidth{0.803000pt}%
\definecolor{currentstroke}{rgb}{0.000000,0.000000,0.000000}%
\pgfsetstrokecolor{currentstroke}%
\pgfsetdash{}{0pt}%
\pgfpathmoveto{\pgfqpoint{1.988155in}{0.319877in}}%
\pgfpathlineto{\pgfqpoint{1.988155in}{2.925408in}}%
\pgfusepath{stroke}%
\end{pgfscope}%
\begin{pgfscope}%
\pgfsetrectcap%
\pgfsetmiterjoin%
\pgfsetlinewidth{0.803000pt}%
\definecolor{currentstroke}{rgb}{0.000000,0.000000,0.000000}%
\pgfsetstrokecolor{currentstroke}%
\pgfsetdash{}{0pt}%
\pgfpathmoveto{\pgfqpoint{0.444137in}{0.319877in}}%
\pgfpathlineto{\pgfqpoint{1.988155in}{0.319877in}}%
\pgfusepath{stroke}%
\end{pgfscope}%
\begin{pgfscope}%
\pgfsetrectcap%
\pgfsetmiterjoin%
\pgfsetlinewidth{0.803000pt}%
\definecolor{currentstroke}{rgb}{0.000000,0.000000,0.000000}%
\pgfsetstrokecolor{currentstroke}%
\pgfsetdash{}{0pt}%
\pgfpathmoveto{\pgfqpoint{0.444137in}{2.925408in}}%
\pgfpathlineto{\pgfqpoint{1.988155in}{2.925408in}}%
\pgfusepath{stroke}%
\end{pgfscope}%
\begin{pgfscope}%
\pgfpathrectangle{\pgfqpoint{2.072071in}{0.319877in}}{\pgfqpoint{0.130277in}{2.605531in}} %
\pgfusepath{clip}%
\pgfsetbuttcap%
\pgfsetmiterjoin%
\definecolor{currentfill}{rgb}{1.000000,1.000000,1.000000}%
\pgfsetfillcolor{currentfill}%
\pgfsetlinewidth{0.010037pt}%
\definecolor{currentstroke}{rgb}{1.000000,1.000000,1.000000}%
\pgfsetstrokecolor{currentstroke}%
\pgfsetdash{}{0pt}%
\pgfpathmoveto{\pgfqpoint{2.072071in}{0.319877in}}%
\pgfpathlineto{\pgfqpoint{2.072071in}{0.330055in}}%
\pgfpathlineto{\pgfqpoint{2.072071in}{2.915230in}}%
\pgfpathlineto{\pgfqpoint{2.072071in}{2.925408in}}%
\pgfpathlineto{\pgfqpoint{2.202347in}{2.925408in}}%
\pgfpathlineto{\pgfqpoint{2.202347in}{2.915230in}}%
\pgfpathlineto{\pgfqpoint{2.202347in}{0.330055in}}%
\pgfpathlineto{\pgfqpoint{2.202347in}{0.319877in}}%
\pgfpathclose%
\pgfusepath{stroke,fill}%
\end{pgfscope}%
\begin{pgfscope}%
\pgfsys@transformshift{2.070000in}{0.320408in}%
\pgftext[left,bottom]{\pgfimage[interpolate=true,width=0.130000in,height=2.610000in]{Ferr_vs_dq_Ti_500K-img1.png}}%
\end{pgfscope}%
\begin{pgfscope}%
\pgfsetbuttcap%
\pgfsetroundjoin%
\definecolor{currentfill}{rgb}{0.000000,0.000000,0.000000}%
\pgfsetfillcolor{currentfill}%
\pgfsetlinewidth{0.803000pt}%
\definecolor{currentstroke}{rgb}{0.000000,0.000000,0.000000}%
\pgfsetstrokecolor{currentstroke}%
\pgfsetdash{}{0pt}%
\pgfsys@defobject{currentmarker}{\pgfqpoint{0.000000in}{0.000000in}}{\pgfqpoint{0.048611in}{0.000000in}}{%
\pgfpathmoveto{\pgfqpoint{0.000000in}{0.000000in}}%
\pgfpathlineto{\pgfqpoint{0.048611in}{0.000000in}}%
\pgfusepath{stroke,fill}%
}%
\begin{pgfscope}%
\pgfsys@transformshift{2.202347in}{0.319877in}%
\pgfsys@useobject{currentmarker}{}%
\end{pgfscope}%
\end{pgfscope}%
\begin{pgfscope}%
\pgftext[x=2.299570in,y=0.272050in,left,base]{\rmfamily\fontsize{10.000000}{12.000000}\selectfont \(\displaystyle 0\)}%
\end{pgfscope}%
\begin{pgfscope}%
\pgfsetbuttcap%
\pgfsetroundjoin%
\definecolor{currentfill}{rgb}{0.000000,0.000000,0.000000}%
\pgfsetfillcolor{currentfill}%
\pgfsetlinewidth{0.803000pt}%
\definecolor{currentstroke}{rgb}{0.000000,0.000000,0.000000}%
\pgfsetstrokecolor{currentstroke}%
\pgfsetdash{}{0pt}%
\pgfsys@defobject{currentmarker}{\pgfqpoint{0.000000in}{0.000000in}}{\pgfqpoint{0.048611in}{0.000000in}}{%
\pgfpathmoveto{\pgfqpoint{0.000000in}{0.000000in}}%
\pgfpathlineto{\pgfqpoint{0.048611in}{0.000000in}}%
\pgfusepath{stroke,fill}%
}%
\begin{pgfscope}%
\pgfsys@transformshift{2.202347in}{0.862696in}%
\pgfsys@useobject{currentmarker}{}%
\end{pgfscope}%
\end{pgfscope}%
\begin{pgfscope}%
\pgftext[x=2.299570in,y=0.814868in,left,base]{\rmfamily\fontsize{10.000000}{12.000000}\selectfont \(\displaystyle 5\)}%
\end{pgfscope}%
\begin{pgfscope}%
\pgfsetbuttcap%
\pgfsetroundjoin%
\definecolor{currentfill}{rgb}{0.000000,0.000000,0.000000}%
\pgfsetfillcolor{currentfill}%
\pgfsetlinewidth{0.803000pt}%
\definecolor{currentstroke}{rgb}{0.000000,0.000000,0.000000}%
\pgfsetstrokecolor{currentstroke}%
\pgfsetdash{}{0pt}%
\pgfsys@defobject{currentmarker}{\pgfqpoint{0.000000in}{0.000000in}}{\pgfqpoint{0.048611in}{0.000000in}}{%
\pgfpathmoveto{\pgfqpoint{0.000000in}{0.000000in}}%
\pgfpathlineto{\pgfqpoint{0.048611in}{0.000000in}}%
\pgfusepath{stroke,fill}%
}%
\begin{pgfscope}%
\pgfsys@transformshift{2.202347in}{1.405515in}%
\pgfsys@useobject{currentmarker}{}%
\end{pgfscope}%
\end{pgfscope}%
\begin{pgfscope}%
\pgftext[x=2.299570in,y=1.357687in,left,base]{\rmfamily\fontsize{10.000000}{12.000000}\selectfont \(\displaystyle 10\)}%
\end{pgfscope}%
\begin{pgfscope}%
\pgfsetbuttcap%
\pgfsetroundjoin%
\definecolor{currentfill}{rgb}{0.000000,0.000000,0.000000}%
\pgfsetfillcolor{currentfill}%
\pgfsetlinewidth{0.803000pt}%
\definecolor{currentstroke}{rgb}{0.000000,0.000000,0.000000}%
\pgfsetstrokecolor{currentstroke}%
\pgfsetdash{}{0pt}%
\pgfsys@defobject{currentmarker}{\pgfqpoint{0.000000in}{0.000000in}}{\pgfqpoint{0.048611in}{0.000000in}}{%
\pgfpathmoveto{\pgfqpoint{0.000000in}{0.000000in}}%
\pgfpathlineto{\pgfqpoint{0.048611in}{0.000000in}}%
\pgfusepath{stroke,fill}%
}%
\begin{pgfscope}%
\pgfsys@transformshift{2.202347in}{1.948334in}%
\pgfsys@useobject{currentmarker}{}%
\end{pgfscope}%
\end{pgfscope}%
\begin{pgfscope}%
\pgftext[x=2.299570in,y=1.900506in,left,base]{\rmfamily\fontsize{10.000000}{12.000000}\selectfont \(\displaystyle 15\)}%
\end{pgfscope}%
\begin{pgfscope}%
\pgfsetbuttcap%
\pgfsetroundjoin%
\definecolor{currentfill}{rgb}{0.000000,0.000000,0.000000}%
\pgfsetfillcolor{currentfill}%
\pgfsetlinewidth{0.803000pt}%
\definecolor{currentstroke}{rgb}{0.000000,0.000000,0.000000}%
\pgfsetstrokecolor{currentstroke}%
\pgfsetdash{}{0pt}%
\pgfsys@defobject{currentmarker}{\pgfqpoint{0.000000in}{0.000000in}}{\pgfqpoint{0.048611in}{0.000000in}}{%
\pgfpathmoveto{\pgfqpoint{0.000000in}{0.000000in}}%
\pgfpathlineto{\pgfqpoint{0.048611in}{0.000000in}}%
\pgfusepath{stroke,fill}%
}%
\begin{pgfscope}%
\pgfsys@transformshift{2.202347in}{2.491153in}%
\pgfsys@useobject{currentmarker}{}%
\end{pgfscope}%
\end{pgfscope}%
\begin{pgfscope}%
\pgftext[x=2.299570in,y=2.443325in,left,base]{\rmfamily\fontsize{10.000000}{12.000000}\selectfont \(\displaystyle 20\)}%
\end{pgfscope}%
\begin{pgfscope}%
\pgfsetbuttcap%
\pgfsetmiterjoin%
\pgfsetlinewidth{0.803000pt}%
\definecolor{currentstroke}{rgb}{0.000000,0.000000,0.000000}%
\pgfsetstrokecolor{currentstroke}%
\pgfsetdash{}{0pt}%
\pgfpathmoveto{\pgfqpoint{2.072071in}{0.319877in}}%
\pgfpathlineto{\pgfqpoint{2.072071in}{0.330055in}}%
\pgfpathlineto{\pgfqpoint{2.072071in}{2.915230in}}%
\pgfpathlineto{\pgfqpoint{2.072071in}{2.925408in}}%
\pgfpathlineto{\pgfqpoint{2.202347in}{2.925408in}}%
\pgfpathlineto{\pgfqpoint{2.202347in}{2.915230in}}%
\pgfpathlineto{\pgfqpoint{2.202347in}{0.330055in}}%
\pgfpathlineto{\pgfqpoint{2.202347in}{0.319877in}}%
\pgfpathclose%
\pgfusepath{stroke}%
\end{pgfscope}%
\end{pgfpicture}%
\makeatother%
\endgroup%

    \vspace*{-0.4cm}
	\caption{500 K. Bin size $0.018e$}
	\end{subfigure}
	\quad
	\begin{subfigure}[b]{0.45\textwidth}
	\hspace*{-0.4cm}
	%% Creator: Matplotlib, PGF backend
%%
%% To include the figure in your LaTeX document, write
%%   \input{<filename>.pgf}
%%
%% Make sure the required packages are loaded in your preamble
%%   \usepackage{pgf}
%%
%% Figures using additional raster images can only be included by \input if
%% they are in the same directory as the main LaTeX file. For loading figures
%% from other directories you can use the `import` package
%%   \usepackage{import}
%% and then include the figures with
%%   \import{<path to file>}{<filename>.pgf}
%%
%% Matplotlib used the following preamble
%%   \usepackage[utf8x]{inputenc}
%%   \usepackage[T1]{fontenc}
%%
\begingroup%
\makeatletter%
\begin{pgfpicture}%
\pgfpathrectangle{\pgfpointorigin}{\pgfqpoint{2.538459in}{3.060408in}}%
\pgfusepath{use as bounding box, clip}%
\begin{pgfscope}%
\pgfsetbuttcap%
\pgfsetmiterjoin%
\definecolor{currentfill}{rgb}{1.000000,1.000000,1.000000}%
\pgfsetfillcolor{currentfill}%
\pgfsetlinewidth{0.000000pt}%
\definecolor{currentstroke}{rgb}{1.000000,1.000000,1.000000}%
\pgfsetstrokecolor{currentstroke}%
\pgfsetdash{}{0pt}%
\pgfpathmoveto{\pgfqpoint{0.000000in}{0.000000in}}%
\pgfpathlineto{\pgfqpoint{2.538459in}{0.000000in}}%
\pgfpathlineto{\pgfqpoint{2.538459in}{3.060408in}}%
\pgfpathlineto{\pgfqpoint{0.000000in}{3.060408in}}%
\pgfpathclose%
\pgfusepath{fill}%
\end{pgfscope}%
\begin{pgfscope}%
\pgfsetbuttcap%
\pgfsetmiterjoin%
\definecolor{currentfill}{rgb}{1.000000,1.000000,1.000000}%
\pgfsetfillcolor{currentfill}%
\pgfsetlinewidth{0.000000pt}%
\definecolor{currentstroke}{rgb}{0.000000,0.000000,0.000000}%
\pgfsetstrokecolor{currentstroke}%
\pgfsetstrokeopacity{0.000000}%
\pgfsetdash{}{0pt}%
\pgfpathmoveto{\pgfqpoint{0.444137in}{0.319877in}}%
\pgfpathlineto{\pgfqpoint{1.988155in}{0.319877in}}%
\pgfpathlineto{\pgfqpoint{1.988155in}{2.925408in}}%
\pgfpathlineto{\pgfqpoint{0.444137in}{2.925408in}}%
\pgfpathclose%
\pgfusepath{fill}%
\end{pgfscope}%
\begin{pgfscope}%
\pgfpathrectangle{\pgfqpoint{0.444137in}{0.319877in}}{\pgfqpoint{1.544018in}{2.605531in}} %
\pgfusepath{clip}%
\pgfsys@transformshift{0.444137in}{0.319877in}%
\pgftext[left,bottom]{\pgfimage[interpolate=true,width=1.550000in,height=2.610000in]{Ferr_vs_dq_Ti_1000K-img0.png}}%
\end{pgfscope}%
\begin{pgfscope}%
\pgfpathrectangle{\pgfqpoint{0.444137in}{0.319877in}}{\pgfqpoint{1.544018in}{2.605531in}} %
\pgfusepath{clip}%
\pgfsetbuttcap%
\pgfsetroundjoin%
\definecolor{currentfill}{rgb}{1.000000,0.752941,0.796078}%
\pgfsetfillcolor{currentfill}%
\pgfsetlinewidth{1.003750pt}%
\definecolor{currentstroke}{rgb}{1.000000,0.752941,0.796078}%
\pgfsetstrokecolor{currentstroke}%
\pgfsetdash{}{0pt}%
\pgfpathmoveto{\pgfqpoint{0.540638in}{1.502810in}}%
\pgfpathcurveto{\pgfqpoint{0.551688in}{1.502810in}}{\pgfqpoint{0.562287in}{1.507200in}}{\pgfqpoint{0.570100in}{1.515014in}}%
\pgfpathcurveto{\pgfqpoint{0.577914in}{1.522828in}}{\pgfqpoint{0.582304in}{1.533427in}}{\pgfqpoint{0.582304in}{1.544477in}}%
\pgfpathcurveto{\pgfqpoint{0.582304in}{1.555527in}}{\pgfqpoint{0.577914in}{1.566126in}}{\pgfqpoint{0.570100in}{1.573939in}}%
\pgfpathcurveto{\pgfqpoint{0.562287in}{1.581753in}}{\pgfqpoint{0.551688in}{1.586143in}}{\pgfqpoint{0.540638in}{1.586143in}}%
\pgfpathcurveto{\pgfqpoint{0.529588in}{1.586143in}}{\pgfqpoint{0.518988in}{1.581753in}}{\pgfqpoint{0.511175in}{1.573939in}}%
\pgfpathcurveto{\pgfqpoint{0.503361in}{1.566126in}}{\pgfqpoint{0.498971in}{1.555527in}}{\pgfqpoint{0.498971in}{1.544477in}}%
\pgfpathcurveto{\pgfqpoint{0.498971in}{1.533427in}}{\pgfqpoint{0.503361in}{1.522828in}}{\pgfqpoint{0.511175in}{1.515014in}}%
\pgfpathcurveto{\pgfqpoint{0.518988in}{1.507200in}}{\pgfqpoint{0.529588in}{1.502810in}}{\pgfqpoint{0.540638in}{1.502810in}}%
\pgfpathclose%
\pgfusepath{stroke,fill}%
\end{pgfscope}%
\begin{pgfscope}%
\pgfpathrectangle{\pgfqpoint{0.444137in}{0.319877in}}{\pgfqpoint{1.544018in}{2.605531in}} %
\pgfusepath{clip}%
\pgfsetbuttcap%
\pgfsetroundjoin%
\definecolor{currentfill}{rgb}{1.000000,0.752941,0.796078}%
\pgfsetfillcolor{currentfill}%
\pgfsetlinewidth{1.003750pt}%
\definecolor{currentstroke}{rgb}{1.000000,0.752941,0.796078}%
\pgfsetstrokecolor{currentstroke}%
\pgfsetdash{}{0pt}%
\pgfpathmoveto{\pgfqpoint{0.733640in}{1.245979in}}%
\pgfpathcurveto{\pgfqpoint{0.744690in}{1.245979in}}{\pgfqpoint{0.755289in}{1.250369in}}{\pgfqpoint{0.763103in}{1.258183in}}%
\pgfpathcurveto{\pgfqpoint{0.770916in}{1.265997in}}{\pgfqpoint{0.775307in}{1.276596in}}{\pgfqpoint{0.775307in}{1.287646in}}%
\pgfpathcurveto{\pgfqpoint{0.775307in}{1.298696in}}{\pgfqpoint{0.770916in}{1.309295in}}{\pgfqpoint{0.763103in}{1.317109in}}%
\pgfpathcurveto{\pgfqpoint{0.755289in}{1.324922in}}{\pgfqpoint{0.744690in}{1.329312in}}{\pgfqpoint{0.733640in}{1.329312in}}%
\pgfpathcurveto{\pgfqpoint{0.722590in}{1.329312in}}{\pgfqpoint{0.711991in}{1.324922in}}{\pgfqpoint{0.704177in}{1.317109in}}%
\pgfpathcurveto{\pgfqpoint{0.696364in}{1.309295in}}{\pgfqpoint{0.691973in}{1.298696in}}{\pgfqpoint{0.691973in}{1.287646in}}%
\pgfpathcurveto{\pgfqpoint{0.691973in}{1.276596in}}{\pgfqpoint{0.696364in}{1.265997in}}{\pgfqpoint{0.704177in}{1.258183in}}%
\pgfpathcurveto{\pgfqpoint{0.711991in}{1.250369in}}{\pgfqpoint{0.722590in}{1.245979in}}{\pgfqpoint{0.733640in}{1.245979in}}%
\pgfpathclose%
\pgfusepath{stroke,fill}%
\end{pgfscope}%
\begin{pgfscope}%
\pgfpathrectangle{\pgfqpoint{0.444137in}{0.319877in}}{\pgfqpoint{1.544018in}{2.605531in}} %
\pgfusepath{clip}%
\pgfsetbuttcap%
\pgfsetroundjoin%
\definecolor{currentfill}{rgb}{1.000000,0.752941,0.796078}%
\pgfsetfillcolor{currentfill}%
\pgfsetlinewidth{1.003750pt}%
\definecolor{currentstroke}{rgb}{1.000000,0.752941,0.796078}%
\pgfsetstrokecolor{currentstroke}%
\pgfsetdash{}{0pt}%
\pgfpathmoveto{\pgfqpoint{0.926642in}{1.017331in}}%
\pgfpathcurveto{\pgfqpoint{0.937692in}{1.017331in}}{\pgfqpoint{0.948291in}{1.021721in}}{\pgfqpoint{0.956105in}{1.029534in}}%
\pgfpathcurveto{\pgfqpoint{0.963919in}{1.037348in}}{\pgfqpoint{0.968309in}{1.047947in}}{\pgfqpoint{0.968309in}{1.058997in}}%
\pgfpathcurveto{\pgfqpoint{0.968309in}{1.070047in}}{\pgfqpoint{0.963919in}{1.080646in}}{\pgfqpoint{0.956105in}{1.088460in}}%
\pgfpathcurveto{\pgfqpoint{0.948291in}{1.096274in}}{\pgfqpoint{0.937692in}{1.100664in}}{\pgfqpoint{0.926642in}{1.100664in}}%
\pgfpathcurveto{\pgfqpoint{0.915592in}{1.100664in}}{\pgfqpoint{0.904993in}{1.096274in}}{\pgfqpoint{0.897179in}{1.088460in}}%
\pgfpathcurveto{\pgfqpoint{0.889366in}{1.080646in}}{\pgfqpoint{0.884976in}{1.070047in}}{\pgfqpoint{0.884976in}{1.058997in}}%
\pgfpathcurveto{\pgfqpoint{0.884976in}{1.047947in}}{\pgfqpoint{0.889366in}{1.037348in}}{\pgfqpoint{0.897179in}{1.029534in}}%
\pgfpathcurveto{\pgfqpoint{0.904993in}{1.021721in}}{\pgfqpoint{0.915592in}{1.017331in}}{\pgfqpoint{0.926642in}{1.017331in}}%
\pgfpathclose%
\pgfusepath{stroke,fill}%
\end{pgfscope}%
\begin{pgfscope}%
\pgfpathrectangle{\pgfqpoint{0.444137in}{0.319877in}}{\pgfqpoint{1.544018in}{2.605531in}} %
\pgfusepath{clip}%
\pgfsetbuttcap%
\pgfsetroundjoin%
\definecolor{currentfill}{rgb}{1.000000,0.752941,0.796078}%
\pgfsetfillcolor{currentfill}%
\pgfsetlinewidth{1.003750pt}%
\definecolor{currentstroke}{rgb}{1.000000,0.752941,0.796078}%
\pgfsetstrokecolor{currentstroke}%
\pgfsetdash{}{0pt}%
\pgfpathmoveto{\pgfqpoint{1.119644in}{0.919087in}}%
\pgfpathcurveto{\pgfqpoint{1.130695in}{0.919087in}}{\pgfqpoint{1.141294in}{0.923477in}}{\pgfqpoint{1.149107in}{0.931291in}}%
\pgfpathcurveto{\pgfqpoint{1.156921in}{0.939105in}}{\pgfqpoint{1.161311in}{0.949704in}}{\pgfqpoint{1.161311in}{0.960754in}}%
\pgfpathcurveto{\pgfqpoint{1.161311in}{0.971804in}}{\pgfqpoint{1.156921in}{0.982403in}}{\pgfqpoint{1.149107in}{0.990217in}}%
\pgfpathcurveto{\pgfqpoint{1.141294in}{0.998030in}}{\pgfqpoint{1.130695in}{1.002420in}}{\pgfqpoint{1.119644in}{1.002420in}}%
\pgfpathcurveto{\pgfqpoint{1.108594in}{1.002420in}}{\pgfqpoint{1.097995in}{0.998030in}}{\pgfqpoint{1.090182in}{0.990217in}}%
\pgfpathcurveto{\pgfqpoint{1.082368in}{0.982403in}}{\pgfqpoint{1.077978in}{0.971804in}}{\pgfqpoint{1.077978in}{0.960754in}}%
\pgfpathcurveto{\pgfqpoint{1.077978in}{0.949704in}}{\pgfqpoint{1.082368in}{0.939105in}}{\pgfqpoint{1.090182in}{0.931291in}}%
\pgfpathcurveto{\pgfqpoint{1.097995in}{0.923477in}}{\pgfqpoint{1.108594in}{0.919087in}}{\pgfqpoint{1.119644in}{0.919087in}}%
\pgfpathclose%
\pgfusepath{stroke,fill}%
\end{pgfscope}%
\begin{pgfscope}%
\pgfpathrectangle{\pgfqpoint{0.444137in}{0.319877in}}{\pgfqpoint{1.544018in}{2.605531in}} %
\pgfusepath{clip}%
\pgfsetbuttcap%
\pgfsetroundjoin%
\definecolor{currentfill}{rgb}{1.000000,0.752941,0.796078}%
\pgfsetfillcolor{currentfill}%
\pgfsetlinewidth{1.003750pt}%
\definecolor{currentstroke}{rgb}{1.000000,0.752941,0.796078}%
\pgfsetstrokecolor{currentstroke}%
\pgfsetdash{}{0pt}%
\pgfpathmoveto{\pgfqpoint{1.312647in}{0.873026in}}%
\pgfpathcurveto{\pgfqpoint{1.323697in}{0.873026in}}{\pgfqpoint{1.334296in}{0.877416in}}{\pgfqpoint{1.342109in}{0.885230in}}%
\pgfpathcurveto{\pgfqpoint{1.349923in}{0.893043in}}{\pgfqpoint{1.354313in}{0.903642in}}{\pgfqpoint{1.354313in}{0.914693in}}%
\pgfpathcurveto{\pgfqpoint{1.354313in}{0.925743in}}{\pgfqpoint{1.349923in}{0.936342in}}{\pgfqpoint{1.342109in}{0.944155in}}%
\pgfpathcurveto{\pgfqpoint{1.334296in}{0.951969in}}{\pgfqpoint{1.323697in}{0.956359in}}{\pgfqpoint{1.312647in}{0.956359in}}%
\pgfpathcurveto{\pgfqpoint{1.301597in}{0.956359in}}{\pgfqpoint{1.290998in}{0.951969in}}{\pgfqpoint{1.283184in}{0.944155in}}%
\pgfpathcurveto{\pgfqpoint{1.275370in}{0.936342in}}{\pgfqpoint{1.270980in}{0.925743in}}{\pgfqpoint{1.270980in}{0.914693in}}%
\pgfpathcurveto{\pgfqpoint{1.270980in}{0.903642in}}{\pgfqpoint{1.275370in}{0.893043in}}{\pgfqpoint{1.283184in}{0.885230in}}%
\pgfpathcurveto{\pgfqpoint{1.290998in}{0.877416in}}{\pgfqpoint{1.301597in}{0.873026in}}{\pgfqpoint{1.312647in}{0.873026in}}%
\pgfpathclose%
\pgfusepath{stroke,fill}%
\end{pgfscope}%
\begin{pgfscope}%
\pgfpathrectangle{\pgfqpoint{0.444137in}{0.319877in}}{\pgfqpoint{1.544018in}{2.605531in}} %
\pgfusepath{clip}%
\pgfsetbuttcap%
\pgfsetroundjoin%
\definecolor{currentfill}{rgb}{1.000000,0.752941,0.796078}%
\pgfsetfillcolor{currentfill}%
\pgfsetlinewidth{1.003750pt}%
\definecolor{currentstroke}{rgb}{1.000000,0.752941,0.796078}%
\pgfsetstrokecolor{currentstroke}%
\pgfsetdash{}{0pt}%
\pgfpathmoveto{\pgfqpoint{1.505649in}{0.956241in}}%
\pgfpathcurveto{\pgfqpoint{1.516699in}{0.956241in}}{\pgfqpoint{1.527298in}{0.960631in}}{\pgfqpoint{1.535112in}{0.968445in}}%
\pgfpathcurveto{\pgfqpoint{1.542925in}{0.976258in}}{\pgfqpoint{1.547316in}{0.986857in}}{\pgfqpoint{1.547316in}{0.997907in}}%
\pgfpathcurveto{\pgfqpoint{1.547316in}{1.008958in}}{\pgfqpoint{1.542925in}{1.019557in}}{\pgfqpoint{1.535112in}{1.027370in}}%
\pgfpathcurveto{\pgfqpoint{1.527298in}{1.035184in}}{\pgfqpoint{1.516699in}{1.039574in}}{\pgfqpoint{1.505649in}{1.039574in}}%
\pgfpathcurveto{\pgfqpoint{1.494599in}{1.039574in}}{\pgfqpoint{1.484000in}{1.035184in}}{\pgfqpoint{1.476186in}{1.027370in}}%
\pgfpathcurveto{\pgfqpoint{1.468373in}{1.019557in}}{\pgfqpoint{1.463982in}{1.008958in}}{\pgfqpoint{1.463982in}{0.997907in}}%
\pgfpathcurveto{\pgfqpoint{1.463982in}{0.986857in}}{\pgfqpoint{1.468373in}{0.976258in}}{\pgfqpoint{1.476186in}{0.968445in}}%
\pgfpathcurveto{\pgfqpoint{1.484000in}{0.960631in}}{\pgfqpoint{1.494599in}{0.956241in}}{\pgfqpoint{1.505649in}{0.956241in}}%
\pgfpathclose%
\pgfusepath{stroke,fill}%
\end{pgfscope}%
\begin{pgfscope}%
\pgfpathrectangle{\pgfqpoint{0.444137in}{0.319877in}}{\pgfqpoint{1.544018in}{2.605531in}} %
\pgfusepath{clip}%
\pgfsetbuttcap%
\pgfsetroundjoin%
\definecolor{currentfill}{rgb}{1.000000,0.752941,0.796078}%
\pgfsetfillcolor{currentfill}%
\pgfsetlinewidth{1.003750pt}%
\definecolor{currentstroke}{rgb}{1.000000,0.752941,0.796078}%
\pgfsetstrokecolor{currentstroke}%
\pgfsetdash{}{0pt}%
\pgfpathmoveto{\pgfqpoint{1.698651in}{1.255285in}}%
\pgfpathcurveto{\pgfqpoint{1.709701in}{1.255285in}}{\pgfqpoint{1.720300in}{1.259675in}}{\pgfqpoint{1.728114in}{1.267489in}}%
\pgfpathcurveto{\pgfqpoint{1.735928in}{1.275302in}}{\pgfqpoint{1.740318in}{1.285901in}}{\pgfqpoint{1.740318in}{1.296951in}}%
\pgfpathcurveto{\pgfqpoint{1.740318in}{1.308001in}}{\pgfqpoint{1.735928in}{1.318600in}}{\pgfqpoint{1.728114in}{1.326414in}}%
\pgfpathcurveto{\pgfqpoint{1.720300in}{1.334228in}}{\pgfqpoint{1.709701in}{1.338618in}}{\pgfqpoint{1.698651in}{1.338618in}}%
\pgfpathcurveto{\pgfqpoint{1.687601in}{1.338618in}}{\pgfqpoint{1.677002in}{1.334228in}}{\pgfqpoint{1.669188in}{1.326414in}}%
\pgfpathcurveto{\pgfqpoint{1.661375in}{1.318600in}}{\pgfqpoint{1.656985in}{1.308001in}}{\pgfqpoint{1.656985in}{1.296951in}}%
\pgfpathcurveto{\pgfqpoint{1.656985in}{1.285901in}}{\pgfqpoint{1.661375in}{1.275302in}}{\pgfqpoint{1.669188in}{1.267489in}}%
\pgfpathcurveto{\pgfqpoint{1.677002in}{1.259675in}}{\pgfqpoint{1.687601in}{1.255285in}}{\pgfqpoint{1.698651in}{1.255285in}}%
\pgfpathclose%
\pgfusepath{stroke,fill}%
\end{pgfscope}%
\begin{pgfscope}%
\pgfpathrectangle{\pgfqpoint{0.444137in}{0.319877in}}{\pgfqpoint{1.544018in}{2.605531in}} %
\pgfusepath{clip}%
\pgfsetbuttcap%
\pgfsetroundjoin%
\definecolor{currentfill}{rgb}{1.000000,0.752941,0.796078}%
\pgfsetfillcolor{currentfill}%
\pgfsetlinewidth{1.003750pt}%
\definecolor{currentstroke}{rgb}{1.000000,0.752941,0.796078}%
\pgfsetstrokecolor{currentstroke}%
\pgfsetdash{}{0pt}%
\pgfpathmoveto{\pgfqpoint{1.891653in}{1.580976in}}%
\pgfpathcurveto{\pgfqpoint{1.902704in}{1.580976in}}{\pgfqpoint{1.913303in}{1.585366in}}{\pgfqpoint{1.921116in}{1.593180in}}%
\pgfpathcurveto{\pgfqpoint{1.928930in}{1.600993in}}{\pgfqpoint{1.933320in}{1.611592in}}{\pgfqpoint{1.933320in}{1.622643in}}%
\pgfpathcurveto{\pgfqpoint{1.933320in}{1.633693in}}{\pgfqpoint{1.928930in}{1.644292in}}{\pgfqpoint{1.921116in}{1.652105in}}%
\pgfpathcurveto{\pgfqpoint{1.913303in}{1.659919in}}{\pgfqpoint{1.902704in}{1.664309in}}{\pgfqpoint{1.891653in}{1.664309in}}%
\pgfpathcurveto{\pgfqpoint{1.880603in}{1.664309in}}{\pgfqpoint{1.870004in}{1.659919in}}{\pgfqpoint{1.862191in}{1.652105in}}%
\pgfpathcurveto{\pgfqpoint{1.854377in}{1.644292in}}{\pgfqpoint{1.849987in}{1.633693in}}{\pgfqpoint{1.849987in}{1.622643in}}%
\pgfpathcurveto{\pgfqpoint{1.849987in}{1.611592in}}{\pgfqpoint{1.854377in}{1.600993in}}{\pgfqpoint{1.862191in}{1.593180in}}%
\pgfpathcurveto{\pgfqpoint{1.870004in}{1.585366in}}{\pgfqpoint{1.880603in}{1.580976in}}{\pgfqpoint{1.891653in}{1.580976in}}%
\pgfpathclose%
\pgfusepath{stroke,fill}%
\end{pgfscope}%
\begin{pgfscope}%
\pgfsetbuttcap%
\pgfsetroundjoin%
\definecolor{currentfill}{rgb}{0.000000,0.000000,0.000000}%
\pgfsetfillcolor{currentfill}%
\pgfsetlinewidth{0.803000pt}%
\definecolor{currentstroke}{rgb}{0.000000,0.000000,0.000000}%
\pgfsetstrokecolor{currentstroke}%
\pgfsetdash{}{0pt}%
\pgfsys@defobject{currentmarker}{\pgfqpoint{0.000000in}{-0.048611in}}{\pgfqpoint{0.000000in}{0.000000in}}{%
\pgfpathmoveto{\pgfqpoint{0.000000in}{0.000000in}}%
\pgfpathlineto{\pgfqpoint{0.000000in}{-0.048611in}}%
\pgfusepath{stroke,fill}%
}%
\begin{pgfscope}%
\pgfsys@transformshift{0.733640in}{0.319877in}%
\pgfsys@useobject{currentmarker}{}%
\end{pgfscope}%
\end{pgfscope}%
\begin{pgfscope}%
\pgftext[x=0.733640in,y=0.222655in,,top]{\rmfamily\fontsize{10.000000}{12.000000}\selectfont \(\displaystyle -0.05\)}%
\end{pgfscope}%
\begin{pgfscope}%
\pgfsetbuttcap%
\pgfsetroundjoin%
\definecolor{currentfill}{rgb}{0.000000,0.000000,0.000000}%
\pgfsetfillcolor{currentfill}%
\pgfsetlinewidth{0.803000pt}%
\definecolor{currentstroke}{rgb}{0.000000,0.000000,0.000000}%
\pgfsetstrokecolor{currentstroke}%
\pgfsetdash{}{0pt}%
\pgfsys@defobject{currentmarker}{\pgfqpoint{0.000000in}{-0.048611in}}{\pgfqpoint{0.000000in}{0.000000in}}{%
\pgfpathmoveto{\pgfqpoint{0.000000in}{0.000000in}}%
\pgfpathlineto{\pgfqpoint{0.000000in}{-0.048611in}}%
\pgfusepath{stroke,fill}%
}%
\begin{pgfscope}%
\pgfsys@transformshift{1.216146in}{0.319877in}%
\pgfsys@useobject{currentmarker}{}%
\end{pgfscope}%
\end{pgfscope}%
\begin{pgfscope}%
\pgftext[x=1.216146in,y=0.222655in,,top]{\rmfamily\fontsize{10.000000}{12.000000}\selectfont \(\displaystyle 0.00\)}%
\end{pgfscope}%
\begin{pgfscope}%
\pgfsetbuttcap%
\pgfsetroundjoin%
\definecolor{currentfill}{rgb}{0.000000,0.000000,0.000000}%
\pgfsetfillcolor{currentfill}%
\pgfsetlinewidth{0.803000pt}%
\definecolor{currentstroke}{rgb}{0.000000,0.000000,0.000000}%
\pgfsetstrokecolor{currentstroke}%
\pgfsetdash{}{0pt}%
\pgfsys@defobject{currentmarker}{\pgfqpoint{0.000000in}{-0.048611in}}{\pgfqpoint{0.000000in}{0.000000in}}{%
\pgfpathmoveto{\pgfqpoint{0.000000in}{0.000000in}}%
\pgfpathlineto{\pgfqpoint{0.000000in}{-0.048611in}}%
\pgfusepath{stroke,fill}%
}%
\begin{pgfscope}%
\pgfsys@transformshift{1.698651in}{0.319877in}%
\pgfsys@useobject{currentmarker}{}%
\end{pgfscope}%
\end{pgfscope}%
\begin{pgfscope}%
\pgftext[x=1.698651in,y=0.222655in,,top]{\rmfamily\fontsize{10.000000}{12.000000}\selectfont \(\displaystyle 0.05\)}%
\end{pgfscope}%
\begin{pgfscope}%
\pgfsetbuttcap%
\pgfsetroundjoin%
\definecolor{currentfill}{rgb}{0.000000,0.000000,0.000000}%
\pgfsetfillcolor{currentfill}%
\pgfsetlinewidth{0.803000pt}%
\definecolor{currentstroke}{rgb}{0.000000,0.000000,0.000000}%
\pgfsetstrokecolor{currentstroke}%
\pgfsetdash{}{0pt}%
\pgfsys@defobject{currentmarker}{\pgfqpoint{-0.048611in}{0.000000in}}{\pgfqpoint{0.000000in}{0.000000in}}{%
\pgfpathmoveto{\pgfqpoint{0.000000in}{0.000000in}}%
\pgfpathlineto{\pgfqpoint{-0.048611in}{0.000000in}}%
\pgfusepath{stroke,fill}%
}%
\begin{pgfscope}%
\pgfsys@transformshift{0.444137in}{0.319877in}%
\pgfsys@useobject{currentmarker}{}%
\end{pgfscope}%
\end{pgfscope}%
\begin{pgfscope}%
\pgftext[x=0.100000in,y=0.272050in,left,base]{\rmfamily\fontsize{10.000000}{12.000000}\selectfont \(\displaystyle 0.00\)}%
\end{pgfscope}%
\begin{pgfscope}%
\pgfsetbuttcap%
\pgfsetroundjoin%
\definecolor{currentfill}{rgb}{0.000000,0.000000,0.000000}%
\pgfsetfillcolor{currentfill}%
\pgfsetlinewidth{0.803000pt}%
\definecolor{currentstroke}{rgb}{0.000000,0.000000,0.000000}%
\pgfsetstrokecolor{currentstroke}%
\pgfsetdash{}{0pt}%
\pgfsys@defobject{currentmarker}{\pgfqpoint{-0.048611in}{0.000000in}}{\pgfqpoint{0.000000in}{0.000000in}}{%
\pgfpathmoveto{\pgfqpoint{0.000000in}{0.000000in}}%
\pgfpathlineto{\pgfqpoint{-0.048611in}{0.000000in}}%
\pgfusepath{stroke,fill}%
}%
\begin{pgfscope}%
\pgfsys@transformshift{0.444137in}{0.802383in}%
\pgfsys@useobject{currentmarker}{}%
\end{pgfscope}%
\end{pgfscope}%
\begin{pgfscope}%
\pgftext[x=0.100000in,y=0.754555in,left,base]{\rmfamily\fontsize{10.000000}{12.000000}\selectfont \(\displaystyle 0.05\)}%
\end{pgfscope}%
\begin{pgfscope}%
\pgfsetbuttcap%
\pgfsetroundjoin%
\definecolor{currentfill}{rgb}{0.000000,0.000000,0.000000}%
\pgfsetfillcolor{currentfill}%
\pgfsetlinewidth{0.803000pt}%
\definecolor{currentstroke}{rgb}{0.000000,0.000000,0.000000}%
\pgfsetstrokecolor{currentstroke}%
\pgfsetdash{}{0pt}%
\pgfsys@defobject{currentmarker}{\pgfqpoint{-0.048611in}{0.000000in}}{\pgfqpoint{0.000000in}{0.000000in}}{%
\pgfpathmoveto{\pgfqpoint{0.000000in}{0.000000in}}%
\pgfpathlineto{\pgfqpoint{-0.048611in}{0.000000in}}%
\pgfusepath{stroke,fill}%
}%
\begin{pgfscope}%
\pgfsys@transformshift{0.444137in}{1.284889in}%
\pgfsys@useobject{currentmarker}{}%
\end{pgfscope}%
\end{pgfscope}%
\begin{pgfscope}%
\pgftext[x=0.100000in,y=1.237061in,left,base]{\rmfamily\fontsize{10.000000}{12.000000}\selectfont \(\displaystyle 0.10\)}%
\end{pgfscope}%
\begin{pgfscope}%
\pgfsetbuttcap%
\pgfsetroundjoin%
\definecolor{currentfill}{rgb}{0.000000,0.000000,0.000000}%
\pgfsetfillcolor{currentfill}%
\pgfsetlinewidth{0.803000pt}%
\definecolor{currentstroke}{rgb}{0.000000,0.000000,0.000000}%
\pgfsetstrokecolor{currentstroke}%
\pgfsetdash{}{0pt}%
\pgfsys@defobject{currentmarker}{\pgfqpoint{-0.048611in}{0.000000in}}{\pgfqpoint{0.000000in}{0.000000in}}{%
\pgfpathmoveto{\pgfqpoint{0.000000in}{0.000000in}}%
\pgfpathlineto{\pgfqpoint{-0.048611in}{0.000000in}}%
\pgfusepath{stroke,fill}%
}%
\begin{pgfscope}%
\pgfsys@transformshift{0.444137in}{1.767394in}%
\pgfsys@useobject{currentmarker}{}%
\end{pgfscope}%
\end{pgfscope}%
\begin{pgfscope}%
\pgftext[x=0.100000in,y=1.719567in,left,base]{\rmfamily\fontsize{10.000000}{12.000000}\selectfont \(\displaystyle 0.15\)}%
\end{pgfscope}%
\begin{pgfscope}%
\pgfsetbuttcap%
\pgfsetroundjoin%
\definecolor{currentfill}{rgb}{0.000000,0.000000,0.000000}%
\pgfsetfillcolor{currentfill}%
\pgfsetlinewidth{0.803000pt}%
\definecolor{currentstroke}{rgb}{0.000000,0.000000,0.000000}%
\pgfsetstrokecolor{currentstroke}%
\pgfsetdash{}{0pt}%
\pgfsys@defobject{currentmarker}{\pgfqpoint{-0.048611in}{0.000000in}}{\pgfqpoint{0.000000in}{0.000000in}}{%
\pgfpathmoveto{\pgfqpoint{0.000000in}{0.000000in}}%
\pgfpathlineto{\pgfqpoint{-0.048611in}{0.000000in}}%
\pgfusepath{stroke,fill}%
}%
\begin{pgfscope}%
\pgfsys@transformshift{0.444137in}{2.249900in}%
\pgfsys@useobject{currentmarker}{}%
\end{pgfscope}%
\end{pgfscope}%
\begin{pgfscope}%
\pgftext[x=0.100000in,y=2.202072in,left,base]{\rmfamily\fontsize{10.000000}{12.000000}\selectfont \(\displaystyle 0.20\)}%
\end{pgfscope}%
\begin{pgfscope}%
\pgfsetbuttcap%
\pgfsetroundjoin%
\definecolor{currentfill}{rgb}{0.000000,0.000000,0.000000}%
\pgfsetfillcolor{currentfill}%
\pgfsetlinewidth{0.803000pt}%
\definecolor{currentstroke}{rgb}{0.000000,0.000000,0.000000}%
\pgfsetstrokecolor{currentstroke}%
\pgfsetdash{}{0pt}%
\pgfsys@defobject{currentmarker}{\pgfqpoint{-0.048611in}{0.000000in}}{\pgfqpoint{0.000000in}{0.000000in}}{%
\pgfpathmoveto{\pgfqpoint{0.000000in}{0.000000in}}%
\pgfpathlineto{\pgfqpoint{-0.048611in}{0.000000in}}%
\pgfusepath{stroke,fill}%
}%
\begin{pgfscope}%
\pgfsys@transformshift{0.444137in}{2.732406in}%
\pgfsys@useobject{currentmarker}{}%
\end{pgfscope}%
\end{pgfscope}%
\begin{pgfscope}%
\pgftext[x=0.100000in,y=2.684578in,left,base]{\rmfamily\fontsize{10.000000}{12.000000}\selectfont \(\displaystyle 0.25\)}%
\end{pgfscope}%
\begin{pgfscope}%
\pgfsetrectcap%
\pgfsetmiterjoin%
\pgfsetlinewidth{0.803000pt}%
\definecolor{currentstroke}{rgb}{0.000000,0.000000,0.000000}%
\pgfsetstrokecolor{currentstroke}%
\pgfsetdash{}{0pt}%
\pgfpathmoveto{\pgfqpoint{0.444137in}{0.319877in}}%
\pgfpathlineto{\pgfqpoint{0.444137in}{2.925408in}}%
\pgfusepath{stroke}%
\end{pgfscope}%
\begin{pgfscope}%
\pgfsetrectcap%
\pgfsetmiterjoin%
\pgfsetlinewidth{0.803000pt}%
\definecolor{currentstroke}{rgb}{0.000000,0.000000,0.000000}%
\pgfsetstrokecolor{currentstroke}%
\pgfsetdash{}{0pt}%
\pgfpathmoveto{\pgfqpoint{1.988155in}{0.319877in}}%
\pgfpathlineto{\pgfqpoint{1.988155in}{2.925408in}}%
\pgfusepath{stroke}%
\end{pgfscope}%
\begin{pgfscope}%
\pgfsetrectcap%
\pgfsetmiterjoin%
\pgfsetlinewidth{0.803000pt}%
\definecolor{currentstroke}{rgb}{0.000000,0.000000,0.000000}%
\pgfsetstrokecolor{currentstroke}%
\pgfsetdash{}{0pt}%
\pgfpathmoveto{\pgfqpoint{0.444137in}{0.319877in}}%
\pgfpathlineto{\pgfqpoint{1.988155in}{0.319877in}}%
\pgfusepath{stroke}%
\end{pgfscope}%
\begin{pgfscope}%
\pgfsetrectcap%
\pgfsetmiterjoin%
\pgfsetlinewidth{0.803000pt}%
\definecolor{currentstroke}{rgb}{0.000000,0.000000,0.000000}%
\pgfsetstrokecolor{currentstroke}%
\pgfsetdash{}{0pt}%
\pgfpathmoveto{\pgfqpoint{0.444137in}{2.925408in}}%
\pgfpathlineto{\pgfqpoint{1.988155in}{2.925408in}}%
\pgfusepath{stroke}%
\end{pgfscope}%
\begin{pgfscope}%
\pgfpathrectangle{\pgfqpoint{2.072071in}{0.319877in}}{\pgfqpoint{0.130277in}{2.605531in}} %
\pgfusepath{clip}%
\pgfsetbuttcap%
\pgfsetmiterjoin%
\definecolor{currentfill}{rgb}{1.000000,1.000000,1.000000}%
\pgfsetfillcolor{currentfill}%
\pgfsetlinewidth{0.010037pt}%
\definecolor{currentstroke}{rgb}{1.000000,1.000000,1.000000}%
\pgfsetstrokecolor{currentstroke}%
\pgfsetdash{}{0pt}%
\pgfpathmoveto{\pgfqpoint{2.072071in}{0.319877in}}%
\pgfpathlineto{\pgfqpoint{2.072071in}{0.330055in}}%
\pgfpathlineto{\pgfqpoint{2.072071in}{2.915230in}}%
\pgfpathlineto{\pgfqpoint{2.072071in}{2.925408in}}%
\pgfpathlineto{\pgfqpoint{2.202347in}{2.925408in}}%
\pgfpathlineto{\pgfqpoint{2.202347in}{2.915230in}}%
\pgfpathlineto{\pgfqpoint{2.202347in}{0.330055in}}%
\pgfpathlineto{\pgfqpoint{2.202347in}{0.319877in}}%
\pgfpathclose%
\pgfusepath{stroke,fill}%
\end{pgfscope}%
\begin{pgfscope}%
\pgfsys@transformshift{2.070000in}{0.320408in}%
\pgftext[left,bottom]{\pgfimage[interpolate=true,width=0.130000in,height=2.610000in]{Ferr_vs_dq_Ti_1000K-img1.png}}%
\end{pgfscope}%
\begin{pgfscope}%
\pgfsetbuttcap%
\pgfsetroundjoin%
\definecolor{currentfill}{rgb}{0.000000,0.000000,0.000000}%
\pgfsetfillcolor{currentfill}%
\pgfsetlinewidth{0.803000pt}%
\definecolor{currentstroke}{rgb}{0.000000,0.000000,0.000000}%
\pgfsetstrokecolor{currentstroke}%
\pgfsetdash{}{0pt}%
\pgfsys@defobject{currentmarker}{\pgfqpoint{0.000000in}{0.000000in}}{\pgfqpoint{0.048611in}{0.000000in}}{%
\pgfpathmoveto{\pgfqpoint{0.000000in}{0.000000in}}%
\pgfpathlineto{\pgfqpoint{0.048611in}{0.000000in}}%
\pgfusepath{stroke,fill}%
}%
\begin{pgfscope}%
\pgfsys@transformshift{2.202347in}{0.319877in}%
\pgfsys@useobject{currentmarker}{}%
\end{pgfscope}%
\end{pgfscope}%
\begin{pgfscope}%
\pgftext[x=2.299570in,y=0.272050in,left,base]{\rmfamily\fontsize{10.000000}{12.000000}\selectfont \(\displaystyle 0\)}%
\end{pgfscope}%
\begin{pgfscope}%
\pgfsetbuttcap%
\pgfsetroundjoin%
\definecolor{currentfill}{rgb}{0.000000,0.000000,0.000000}%
\pgfsetfillcolor{currentfill}%
\pgfsetlinewidth{0.803000pt}%
\definecolor{currentstroke}{rgb}{0.000000,0.000000,0.000000}%
\pgfsetstrokecolor{currentstroke}%
\pgfsetdash{}{0pt}%
\pgfsys@defobject{currentmarker}{\pgfqpoint{0.000000in}{0.000000in}}{\pgfqpoint{0.048611in}{0.000000in}}{%
\pgfpathmoveto{\pgfqpoint{0.000000in}{0.000000in}}%
\pgfpathlineto{\pgfqpoint{0.048611in}{0.000000in}}%
\pgfusepath{stroke,fill}%
}%
\begin{pgfscope}%
\pgfsys@transformshift{2.202347in}{0.862696in}%
\pgfsys@useobject{currentmarker}{}%
\end{pgfscope}%
\end{pgfscope}%
\begin{pgfscope}%
\pgftext[x=2.299570in,y=0.814868in,left,base]{\rmfamily\fontsize{10.000000}{12.000000}\selectfont \(\displaystyle 5\)}%
\end{pgfscope}%
\begin{pgfscope}%
\pgfsetbuttcap%
\pgfsetroundjoin%
\definecolor{currentfill}{rgb}{0.000000,0.000000,0.000000}%
\pgfsetfillcolor{currentfill}%
\pgfsetlinewidth{0.803000pt}%
\definecolor{currentstroke}{rgb}{0.000000,0.000000,0.000000}%
\pgfsetstrokecolor{currentstroke}%
\pgfsetdash{}{0pt}%
\pgfsys@defobject{currentmarker}{\pgfqpoint{0.000000in}{0.000000in}}{\pgfqpoint{0.048611in}{0.000000in}}{%
\pgfpathmoveto{\pgfqpoint{0.000000in}{0.000000in}}%
\pgfpathlineto{\pgfqpoint{0.048611in}{0.000000in}}%
\pgfusepath{stroke,fill}%
}%
\begin{pgfscope}%
\pgfsys@transformshift{2.202347in}{1.405515in}%
\pgfsys@useobject{currentmarker}{}%
\end{pgfscope}%
\end{pgfscope}%
\begin{pgfscope}%
\pgftext[x=2.299570in,y=1.357687in,left,base]{\rmfamily\fontsize{10.000000}{12.000000}\selectfont \(\displaystyle 10\)}%
\end{pgfscope}%
\begin{pgfscope}%
\pgfsetbuttcap%
\pgfsetroundjoin%
\definecolor{currentfill}{rgb}{0.000000,0.000000,0.000000}%
\pgfsetfillcolor{currentfill}%
\pgfsetlinewidth{0.803000pt}%
\definecolor{currentstroke}{rgb}{0.000000,0.000000,0.000000}%
\pgfsetstrokecolor{currentstroke}%
\pgfsetdash{}{0pt}%
\pgfsys@defobject{currentmarker}{\pgfqpoint{0.000000in}{0.000000in}}{\pgfqpoint{0.048611in}{0.000000in}}{%
\pgfpathmoveto{\pgfqpoint{0.000000in}{0.000000in}}%
\pgfpathlineto{\pgfqpoint{0.048611in}{0.000000in}}%
\pgfusepath{stroke,fill}%
}%
\begin{pgfscope}%
\pgfsys@transformshift{2.202347in}{1.948334in}%
\pgfsys@useobject{currentmarker}{}%
\end{pgfscope}%
\end{pgfscope}%
\begin{pgfscope}%
\pgftext[x=2.299570in,y=1.900506in,left,base]{\rmfamily\fontsize{10.000000}{12.000000}\selectfont \(\displaystyle 15\)}%
\end{pgfscope}%
\begin{pgfscope}%
\pgfsetbuttcap%
\pgfsetroundjoin%
\definecolor{currentfill}{rgb}{0.000000,0.000000,0.000000}%
\pgfsetfillcolor{currentfill}%
\pgfsetlinewidth{0.803000pt}%
\definecolor{currentstroke}{rgb}{0.000000,0.000000,0.000000}%
\pgfsetstrokecolor{currentstroke}%
\pgfsetdash{}{0pt}%
\pgfsys@defobject{currentmarker}{\pgfqpoint{0.000000in}{0.000000in}}{\pgfqpoint{0.048611in}{0.000000in}}{%
\pgfpathmoveto{\pgfqpoint{0.000000in}{0.000000in}}%
\pgfpathlineto{\pgfqpoint{0.048611in}{0.000000in}}%
\pgfusepath{stroke,fill}%
}%
\begin{pgfscope}%
\pgfsys@transformshift{2.202347in}{2.491153in}%
\pgfsys@useobject{currentmarker}{}%
\end{pgfscope}%
\end{pgfscope}%
\begin{pgfscope}%
\pgftext[x=2.299570in,y=2.443325in,left,base]{\rmfamily\fontsize{10.000000}{12.000000}\selectfont \(\displaystyle 20\)}%
\end{pgfscope}%
\begin{pgfscope}%
\pgfsetbuttcap%
\pgfsetmiterjoin%
\pgfsetlinewidth{0.803000pt}%
\definecolor{currentstroke}{rgb}{0.000000,0.000000,0.000000}%
\pgfsetstrokecolor{currentstroke}%
\pgfsetdash{}{0pt}%
\pgfpathmoveto{\pgfqpoint{2.072071in}{0.319877in}}%
\pgfpathlineto{\pgfqpoint{2.072071in}{0.330055in}}%
\pgfpathlineto{\pgfqpoint{2.072071in}{2.915230in}}%
\pgfpathlineto{\pgfqpoint{2.072071in}{2.925408in}}%
\pgfpathlineto{\pgfqpoint{2.202347in}{2.925408in}}%
\pgfpathlineto{\pgfqpoint{2.202347in}{2.915230in}}%
\pgfpathlineto{\pgfqpoint{2.202347in}{0.330055in}}%
\pgfpathlineto{\pgfqpoint{2.202347in}{0.319877in}}%
\pgfpathclose%
\pgfusepath{stroke}%
\end{pgfscope}%
\end{pgfpicture}%
\makeatother%
\endgroup%

    \vspace*{-0.4cm}
	\caption{1000 K. Bin size $0.020e$}
	\end{subfigure}
\caption{On-site force on ion, vs its change in charge}
\label{on_site_Ferr_vs_dq}
\end{figure}

2) Figure \ref{on_site_FerrNN_vs_dq} shows how a large change in charge on a Ti ion relates to forces on that same ion's $6$ Oxygen nearest neighbours. The exact quantity represented by the y-axis is:
\begin{align*}
y_{i,I} &\equiv  \frac{(1/6)\sum_{s\in \text{NN}_i}\sqrt{\sum_{\alpha = x,y,z}\left(F_{s,I}^{\alpha}(\{q_l\})-F_{s,I}^{\alpha}(\bar{q}_{\text{Ba}},\bar{q}_{\text{Ti}},\bar{q}_{\text{O}})\right)^2}}{\sum_{J}\sum_{j}\sqrt{\sum_{\alpha = x,y,z}\left(F_{j,J}^{\alpha}(\{q_l\})\right)^2}}
\end{align*}
where $6$ stands for the number of Ti nearest neighbours, $s\in\text{NN}_i=\{s=1,...,6 : \text{R}_{is} \text{ is nearest neighbour}\}$, $i\in \{\text{Ti}_1,...,\text{Ti}_{27}\}$, $j\in \{\text{O}_1,...,\text{O}_{81}\}$ and, as before, $I,J\in\{\text{MD}_1,...,\text{MD}_{10}\}$. y-axis measures forces on Oxygen ions surrounding a given Ti ion. 

\begin{figure}[h!]
\centering
	\begin{subfigure}[b]{0.45\textwidth}
	\hspace*{-0.4cm}
	%% Creator: Matplotlib, PGF backend
%%
%% To include the figure in your LaTeX document, write
%%   \input{<filename>.pgf}
%%
%% Make sure the required packages are loaded in your preamble
%%   \usepackage{pgf}
%%
%% Figures using additional raster images can only be included by \input if
%% they are in the same directory as the main LaTeX file. For loading figures
%% from other directories you can use the `import` package
%%   \usepackage{import}
%% and then include the figures with
%%   \import{<path to file>}{<filename>.pgf}
%%
%% Matplotlib used the following preamble
%%   \usepackage[utf8x]{inputenc}
%%   \usepackage[T1]{fontenc}
%%
\begingroup%
\makeatletter%
\begin{pgfpicture}%
\pgfpathrectangle{\pgfpointorigin}{\pgfqpoint{2.529900in}{3.060408in}}%
\pgfusepath{use as bounding box, clip}%
\begin{pgfscope}%
\pgfsetbuttcap%
\pgfsetmiterjoin%
\definecolor{currentfill}{rgb}{1.000000,1.000000,1.000000}%
\pgfsetfillcolor{currentfill}%
\pgfsetlinewidth{0.000000pt}%
\definecolor{currentstroke}{rgb}{1.000000,1.000000,1.000000}%
\pgfsetstrokecolor{currentstroke}%
\pgfsetdash{}{0pt}%
\pgfpathmoveto{\pgfqpoint{0.000000in}{0.000000in}}%
\pgfpathlineto{\pgfqpoint{2.529900in}{0.000000in}}%
\pgfpathlineto{\pgfqpoint{2.529900in}{3.060408in}}%
\pgfpathlineto{\pgfqpoint{0.000000in}{3.060408in}}%
\pgfpathclose%
\pgfusepath{fill}%
\end{pgfscope}%
\begin{pgfscope}%
\pgfsetbuttcap%
\pgfsetmiterjoin%
\definecolor{currentfill}{rgb}{1.000000,1.000000,1.000000}%
\pgfsetfillcolor{currentfill}%
\pgfsetlinewidth{0.000000pt}%
\definecolor{currentstroke}{rgb}{0.000000,0.000000,0.000000}%
\pgfsetstrokecolor{currentstroke}%
\pgfsetstrokeopacity{0.000000}%
\pgfsetdash{}{0pt}%
\pgfpathmoveto{\pgfqpoint{0.444137in}{0.319877in}}%
\pgfpathlineto{\pgfqpoint{1.968255in}{0.319877in}}%
\pgfpathlineto{\pgfqpoint{1.968255in}{2.925408in}}%
\pgfpathlineto{\pgfqpoint{0.444137in}{2.925408in}}%
\pgfpathclose%
\pgfusepath{fill}%
\end{pgfscope}%
\begin{pgfscope}%
\pgfpathrectangle{\pgfqpoint{0.444137in}{0.319877in}}{\pgfqpoint{1.524118in}{2.605531in}} %
\pgfusepath{clip}%
\pgfsys@transformshift{0.444137in}{0.319877in}%
\pgftext[left,bottom]{\pgfimage[interpolate=true,width=1.530000in,height=2.610000in]{FerrNN_vs_dq_Ti_100K-img0.png}}%
\end{pgfscope}%
\begin{pgfscope}%
\pgfpathrectangle{\pgfqpoint{0.444137in}{0.319877in}}{\pgfqpoint{1.524118in}{2.605531in}} %
\pgfusepath{clip}%
\pgfsetbuttcap%
\pgfsetroundjoin%
\definecolor{currentfill}{rgb}{1.000000,0.752941,0.796078}%
\pgfsetfillcolor{currentfill}%
\pgfsetlinewidth{1.003750pt}%
\definecolor{currentstroke}{rgb}{1.000000,0.752941,0.796078}%
\pgfsetstrokecolor{currentstroke}%
\pgfsetdash{}{0pt}%
\pgfpathmoveto{\pgfqpoint{1.002980in}{2.157029in}}%
\pgfpathcurveto{\pgfqpoint{1.014030in}{2.157029in}}{\pgfqpoint{1.024629in}{2.161419in}}{\pgfqpoint{1.032443in}{2.169232in}}%
\pgfpathcurveto{\pgfqpoint{1.040256in}{2.177046in}}{\pgfqpoint{1.044647in}{2.187645in}}{\pgfqpoint{1.044647in}{2.198695in}}%
\pgfpathcurveto{\pgfqpoint{1.044647in}{2.209745in}}{\pgfqpoint{1.040256in}{2.220344in}}{\pgfqpoint{1.032443in}{2.228158in}}%
\pgfpathcurveto{\pgfqpoint{1.024629in}{2.235972in}}{\pgfqpoint{1.014030in}{2.240362in}}{\pgfqpoint{1.002980in}{2.240362in}}%
\pgfpathcurveto{\pgfqpoint{0.991930in}{2.240362in}}{\pgfqpoint{0.981331in}{2.235972in}}{\pgfqpoint{0.973517in}{2.228158in}}%
\pgfpathcurveto{\pgfqpoint{0.965703in}{2.220344in}}{\pgfqpoint{0.961313in}{2.209745in}}{\pgfqpoint{0.961313in}{2.198695in}}%
\pgfpathcurveto{\pgfqpoint{0.961313in}{2.187645in}}{\pgfqpoint{0.965703in}{2.177046in}}{\pgfqpoint{0.973517in}{2.169232in}}%
\pgfpathcurveto{\pgfqpoint{0.981331in}{2.161419in}}{\pgfqpoint{0.991930in}{2.157029in}}{\pgfqpoint{1.002980in}{2.157029in}}%
\pgfpathclose%
\pgfusepath{stroke,fill}%
\end{pgfscope}%
\begin{pgfscope}%
\pgfpathrectangle{\pgfqpoint{0.444137in}{0.319877in}}{\pgfqpoint{1.524118in}{2.605531in}} %
\pgfusepath{clip}%
\pgfsetbuttcap%
\pgfsetroundjoin%
\definecolor{currentfill}{rgb}{1.000000,0.752941,0.796078}%
\pgfsetfillcolor{currentfill}%
\pgfsetlinewidth{1.003750pt}%
\definecolor{currentstroke}{rgb}{1.000000,0.752941,0.796078}%
\pgfsetstrokecolor{currentstroke}%
\pgfsetdash{}{0pt}%
\pgfpathmoveto{\pgfqpoint{1.104588in}{1.272933in}}%
\pgfpathcurveto{\pgfqpoint{1.115638in}{1.272933in}}{\pgfqpoint{1.126237in}{1.277323in}}{\pgfqpoint{1.134051in}{1.285137in}}%
\pgfpathcurveto{\pgfqpoint{1.141864in}{1.292950in}}{\pgfqpoint{1.146254in}{1.303549in}}{\pgfqpoint{1.146254in}{1.314600in}}%
\pgfpathcurveto{\pgfqpoint{1.146254in}{1.325650in}}{\pgfqpoint{1.141864in}{1.336249in}}{\pgfqpoint{1.134051in}{1.344062in}}%
\pgfpathcurveto{\pgfqpoint{1.126237in}{1.351876in}}{\pgfqpoint{1.115638in}{1.356266in}}{\pgfqpoint{1.104588in}{1.356266in}}%
\pgfpathcurveto{\pgfqpoint{1.093538in}{1.356266in}}{\pgfqpoint{1.082939in}{1.351876in}}{\pgfqpoint{1.075125in}{1.344062in}}%
\pgfpathcurveto{\pgfqpoint{1.067311in}{1.336249in}}{\pgfqpoint{1.062921in}{1.325650in}}{\pgfqpoint{1.062921in}{1.314600in}}%
\pgfpathcurveto{\pgfqpoint{1.062921in}{1.303549in}}{\pgfqpoint{1.067311in}{1.292950in}}{\pgfqpoint{1.075125in}{1.285137in}}%
\pgfpathcurveto{\pgfqpoint{1.082939in}{1.277323in}}{\pgfqpoint{1.093538in}{1.272933in}}{\pgfqpoint{1.104588in}{1.272933in}}%
\pgfpathclose%
\pgfusepath{stroke,fill}%
\end{pgfscope}%
\begin{pgfscope}%
\pgfpathrectangle{\pgfqpoint{0.444137in}{0.319877in}}{\pgfqpoint{1.524118in}{2.605531in}} %
\pgfusepath{clip}%
\pgfsetbuttcap%
\pgfsetroundjoin%
\definecolor{currentfill}{rgb}{1.000000,0.752941,0.796078}%
\pgfsetfillcolor{currentfill}%
\pgfsetlinewidth{1.003750pt}%
\definecolor{currentstroke}{rgb}{1.000000,0.752941,0.796078}%
\pgfsetstrokecolor{currentstroke}%
\pgfsetdash{}{0pt}%
\pgfpathmoveto{\pgfqpoint{1.206196in}{1.000171in}}%
\pgfpathcurveto{\pgfqpoint{1.217246in}{1.000171in}}{\pgfqpoint{1.227845in}{1.004561in}}{\pgfqpoint{1.235658in}{1.012375in}}%
\pgfpathcurveto{\pgfqpoint{1.243472in}{1.020189in}}{\pgfqpoint{1.247862in}{1.030788in}}{\pgfqpoint{1.247862in}{1.041838in}}%
\pgfpathcurveto{\pgfqpoint{1.247862in}{1.052888in}}{\pgfqpoint{1.243472in}{1.063487in}}{\pgfqpoint{1.235658in}{1.071300in}}%
\pgfpathcurveto{\pgfqpoint{1.227845in}{1.079114in}}{\pgfqpoint{1.217246in}{1.083504in}}{\pgfqpoint{1.206196in}{1.083504in}}%
\pgfpathcurveto{\pgfqpoint{1.195145in}{1.083504in}}{\pgfqpoint{1.184546in}{1.079114in}}{\pgfqpoint{1.176733in}{1.071300in}}%
\pgfpathcurveto{\pgfqpoint{1.168919in}{1.063487in}}{\pgfqpoint{1.164529in}{1.052888in}}{\pgfqpoint{1.164529in}{1.041838in}}%
\pgfpathcurveto{\pgfqpoint{1.164529in}{1.030788in}}{\pgfqpoint{1.168919in}{1.020189in}}{\pgfqpoint{1.176733in}{1.012375in}}%
\pgfpathcurveto{\pgfqpoint{1.184546in}{1.004561in}}{\pgfqpoint{1.195145in}{1.000171in}}{\pgfqpoint{1.206196in}{1.000171in}}%
\pgfpathclose%
\pgfusepath{stroke,fill}%
\end{pgfscope}%
\begin{pgfscope}%
\pgfpathrectangle{\pgfqpoint{0.444137in}{0.319877in}}{\pgfqpoint{1.524118in}{2.605531in}} %
\pgfusepath{clip}%
\pgfsetbuttcap%
\pgfsetroundjoin%
\definecolor{currentfill}{rgb}{1.000000,0.752941,0.796078}%
\pgfsetfillcolor{currentfill}%
\pgfsetlinewidth{1.003750pt}%
\definecolor{currentstroke}{rgb}{1.000000,0.752941,0.796078}%
\pgfsetstrokecolor{currentstroke}%
\pgfsetdash{}{0pt}%
\pgfpathmoveto{\pgfqpoint{1.307803in}{1.287066in}}%
\pgfpathcurveto{\pgfqpoint{1.318854in}{1.287066in}}{\pgfqpoint{1.329453in}{1.291456in}}{\pgfqpoint{1.337266in}{1.299269in}}%
\pgfpathcurveto{\pgfqpoint{1.345080in}{1.307083in}}{\pgfqpoint{1.349470in}{1.317682in}}{\pgfqpoint{1.349470in}{1.328732in}}%
\pgfpathcurveto{\pgfqpoint{1.349470in}{1.339782in}}{\pgfqpoint{1.345080in}{1.350381in}}{\pgfqpoint{1.337266in}{1.358195in}}%
\pgfpathcurveto{\pgfqpoint{1.329453in}{1.366009in}}{\pgfqpoint{1.318854in}{1.370399in}}{\pgfqpoint{1.307803in}{1.370399in}}%
\pgfpathcurveto{\pgfqpoint{1.296753in}{1.370399in}}{\pgfqpoint{1.286154in}{1.366009in}}{\pgfqpoint{1.278341in}{1.358195in}}%
\pgfpathcurveto{\pgfqpoint{1.270527in}{1.350381in}}{\pgfqpoint{1.266137in}{1.339782in}}{\pgfqpoint{1.266137in}{1.328732in}}%
\pgfpathcurveto{\pgfqpoint{1.266137in}{1.317682in}}{\pgfqpoint{1.270527in}{1.307083in}}{\pgfqpoint{1.278341in}{1.299269in}}%
\pgfpathcurveto{\pgfqpoint{1.286154in}{1.291456in}}{\pgfqpoint{1.296753in}{1.287066in}}{\pgfqpoint{1.307803in}{1.287066in}}%
\pgfpathclose%
\pgfusepath{stroke,fill}%
\end{pgfscope}%
\begin{pgfscope}%
\pgfpathrectangle{\pgfqpoint{0.444137in}{0.319877in}}{\pgfqpoint{1.524118in}{2.605531in}} %
\pgfusepath{clip}%
\pgfsetbuttcap%
\pgfsetroundjoin%
\definecolor{currentfill}{rgb}{1.000000,0.752941,0.796078}%
\pgfsetfillcolor{currentfill}%
\pgfsetlinewidth{1.003750pt}%
\definecolor{currentstroke}{rgb}{1.000000,0.752941,0.796078}%
\pgfsetstrokecolor{currentstroke}%
\pgfsetdash{}{0pt}%
\pgfpathmoveto{\pgfqpoint{1.409411in}{2.387450in}}%
\pgfpathcurveto{\pgfqpoint{1.420461in}{2.387450in}}{\pgfqpoint{1.431061in}{2.391840in}}{\pgfqpoint{1.438874in}{2.399654in}}%
\pgfpathcurveto{\pgfqpoint{1.446688in}{2.407467in}}{\pgfqpoint{1.451078in}{2.418066in}}{\pgfqpoint{1.451078in}{2.429116in}}%
\pgfpathcurveto{\pgfqpoint{1.451078in}{2.440166in}}{\pgfqpoint{1.446688in}{2.450765in}}{\pgfqpoint{1.438874in}{2.458579in}}%
\pgfpathcurveto{\pgfqpoint{1.431061in}{2.466393in}}{\pgfqpoint{1.420461in}{2.470783in}}{\pgfqpoint{1.409411in}{2.470783in}}%
\pgfpathcurveto{\pgfqpoint{1.398361in}{2.470783in}}{\pgfqpoint{1.387762in}{2.466393in}}{\pgfqpoint{1.379949in}{2.458579in}}%
\pgfpathcurveto{\pgfqpoint{1.372135in}{2.450765in}}{\pgfqpoint{1.367745in}{2.440166in}}{\pgfqpoint{1.367745in}{2.429116in}}%
\pgfpathcurveto{\pgfqpoint{1.367745in}{2.418066in}}{\pgfqpoint{1.372135in}{2.407467in}}{\pgfqpoint{1.379949in}{2.399654in}}%
\pgfpathcurveto{\pgfqpoint{1.387762in}{2.391840in}}{\pgfqpoint{1.398361in}{2.387450in}}{\pgfqpoint{1.409411in}{2.387450in}}%
\pgfpathclose%
\pgfusepath{stroke,fill}%
\end{pgfscope}%
\begin{pgfscope}%
\pgfsetbuttcap%
\pgfsetroundjoin%
\definecolor{currentfill}{rgb}{0.000000,0.000000,0.000000}%
\pgfsetfillcolor{currentfill}%
\pgfsetlinewidth{0.803000pt}%
\definecolor{currentstroke}{rgb}{0.000000,0.000000,0.000000}%
\pgfsetstrokecolor{currentstroke}%
\pgfsetdash{}{0pt}%
\pgfsys@defobject{currentmarker}{\pgfqpoint{0.000000in}{-0.048611in}}{\pgfqpoint{0.000000in}{0.000000in}}{%
\pgfpathmoveto{\pgfqpoint{0.000000in}{0.000000in}}%
\pgfpathlineto{\pgfqpoint{0.000000in}{-0.048611in}}%
\pgfusepath{stroke,fill}%
}%
\begin{pgfscope}%
\pgfsys@transformshift{0.729909in}{0.319877in}%
\pgfsys@useobject{currentmarker}{}%
\end{pgfscope}%
\end{pgfscope}%
\begin{pgfscope}%
\pgftext[x=0.729909in,y=0.222655in,,top]{\rmfamily\fontsize{10.000000}{12.000000}\selectfont \(\displaystyle -0.05\)}%
\end{pgfscope}%
\begin{pgfscope}%
\pgfsetbuttcap%
\pgfsetroundjoin%
\definecolor{currentfill}{rgb}{0.000000,0.000000,0.000000}%
\pgfsetfillcolor{currentfill}%
\pgfsetlinewidth{0.803000pt}%
\definecolor{currentstroke}{rgb}{0.000000,0.000000,0.000000}%
\pgfsetstrokecolor{currentstroke}%
\pgfsetdash{}{0pt}%
\pgfsys@defobject{currentmarker}{\pgfqpoint{0.000000in}{-0.048611in}}{\pgfqpoint{0.000000in}{0.000000in}}{%
\pgfpathmoveto{\pgfqpoint{0.000000in}{0.000000in}}%
\pgfpathlineto{\pgfqpoint{0.000000in}{-0.048611in}}%
\pgfusepath{stroke,fill}%
}%
\begin{pgfscope}%
\pgfsys@transformshift{1.206196in}{0.319877in}%
\pgfsys@useobject{currentmarker}{}%
\end{pgfscope}%
\end{pgfscope}%
\begin{pgfscope}%
\pgftext[x=1.206196in,y=0.222655in,,top]{\rmfamily\fontsize{10.000000}{12.000000}\selectfont \(\displaystyle 0.00\)}%
\end{pgfscope}%
\begin{pgfscope}%
\pgfsetbuttcap%
\pgfsetroundjoin%
\definecolor{currentfill}{rgb}{0.000000,0.000000,0.000000}%
\pgfsetfillcolor{currentfill}%
\pgfsetlinewidth{0.803000pt}%
\definecolor{currentstroke}{rgb}{0.000000,0.000000,0.000000}%
\pgfsetstrokecolor{currentstroke}%
\pgfsetdash{}{0pt}%
\pgfsys@defobject{currentmarker}{\pgfqpoint{0.000000in}{-0.048611in}}{\pgfqpoint{0.000000in}{0.000000in}}{%
\pgfpathmoveto{\pgfqpoint{0.000000in}{0.000000in}}%
\pgfpathlineto{\pgfqpoint{0.000000in}{-0.048611in}}%
\pgfusepath{stroke,fill}%
}%
\begin{pgfscope}%
\pgfsys@transformshift{1.682483in}{0.319877in}%
\pgfsys@useobject{currentmarker}{}%
\end{pgfscope}%
\end{pgfscope}%
\begin{pgfscope}%
\pgftext[x=1.682483in,y=0.222655in,,top]{\rmfamily\fontsize{10.000000}{12.000000}\selectfont \(\displaystyle 0.05\)}%
\end{pgfscope}%
\begin{pgfscope}%
\pgfsetbuttcap%
\pgfsetroundjoin%
\definecolor{currentfill}{rgb}{0.000000,0.000000,0.000000}%
\pgfsetfillcolor{currentfill}%
\pgfsetlinewidth{0.803000pt}%
\definecolor{currentstroke}{rgb}{0.000000,0.000000,0.000000}%
\pgfsetstrokecolor{currentstroke}%
\pgfsetdash{}{0pt}%
\pgfsys@defobject{currentmarker}{\pgfqpoint{-0.048611in}{0.000000in}}{\pgfqpoint{0.000000in}{0.000000in}}{%
\pgfpathmoveto{\pgfqpoint{0.000000in}{0.000000in}}%
\pgfpathlineto{\pgfqpoint{-0.048611in}{0.000000in}}%
\pgfusepath{stroke,fill}%
}%
\begin{pgfscope}%
\pgfsys@transformshift{0.444137in}{0.319877in}%
\pgfsys@useobject{currentmarker}{}%
\end{pgfscope}%
\end{pgfscope}%
\begin{pgfscope}%
\pgftext[x=0.100000in,y=0.272050in,left,base]{\rmfamily\fontsize{10.000000}{12.000000}\selectfont \(\displaystyle 0.00\)}%
\end{pgfscope}%
\begin{pgfscope}%
\pgfsetbuttcap%
\pgfsetroundjoin%
\definecolor{currentfill}{rgb}{0.000000,0.000000,0.000000}%
\pgfsetfillcolor{currentfill}%
\pgfsetlinewidth{0.803000pt}%
\definecolor{currentstroke}{rgb}{0.000000,0.000000,0.000000}%
\pgfsetstrokecolor{currentstroke}%
\pgfsetdash{}{0pt}%
\pgfsys@defobject{currentmarker}{\pgfqpoint{-0.048611in}{0.000000in}}{\pgfqpoint{0.000000in}{0.000000in}}{%
\pgfpathmoveto{\pgfqpoint{0.000000in}{0.000000in}}%
\pgfpathlineto{\pgfqpoint{-0.048611in}{0.000000in}}%
\pgfusepath{stroke,fill}%
}%
\begin{pgfscope}%
\pgfsys@transformshift{0.444137in}{0.776472in}%
\pgfsys@useobject{currentmarker}{}%
\end{pgfscope}%
\end{pgfscope}%
\begin{pgfscope}%
\pgftext[x=0.100000in,y=0.728645in,left,base]{\rmfamily\fontsize{10.000000}{12.000000}\selectfont \(\displaystyle 0.05\)}%
\end{pgfscope}%
\begin{pgfscope}%
\pgfsetbuttcap%
\pgfsetroundjoin%
\definecolor{currentfill}{rgb}{0.000000,0.000000,0.000000}%
\pgfsetfillcolor{currentfill}%
\pgfsetlinewidth{0.803000pt}%
\definecolor{currentstroke}{rgb}{0.000000,0.000000,0.000000}%
\pgfsetstrokecolor{currentstroke}%
\pgfsetdash{}{0pt}%
\pgfsys@defobject{currentmarker}{\pgfqpoint{-0.048611in}{0.000000in}}{\pgfqpoint{0.000000in}{0.000000in}}{%
\pgfpathmoveto{\pgfqpoint{0.000000in}{0.000000in}}%
\pgfpathlineto{\pgfqpoint{-0.048611in}{0.000000in}}%
\pgfusepath{stroke,fill}%
}%
\begin{pgfscope}%
\pgfsys@transformshift{0.444137in}{1.233067in}%
\pgfsys@useobject{currentmarker}{}%
\end{pgfscope}%
\end{pgfscope}%
\begin{pgfscope}%
\pgftext[x=0.100000in,y=1.185240in,left,base]{\rmfamily\fontsize{10.000000}{12.000000}\selectfont \(\displaystyle 0.10\)}%
\end{pgfscope}%
\begin{pgfscope}%
\pgfsetbuttcap%
\pgfsetroundjoin%
\definecolor{currentfill}{rgb}{0.000000,0.000000,0.000000}%
\pgfsetfillcolor{currentfill}%
\pgfsetlinewidth{0.803000pt}%
\definecolor{currentstroke}{rgb}{0.000000,0.000000,0.000000}%
\pgfsetstrokecolor{currentstroke}%
\pgfsetdash{}{0pt}%
\pgfsys@defobject{currentmarker}{\pgfqpoint{-0.048611in}{0.000000in}}{\pgfqpoint{0.000000in}{0.000000in}}{%
\pgfpathmoveto{\pgfqpoint{0.000000in}{0.000000in}}%
\pgfpathlineto{\pgfqpoint{-0.048611in}{0.000000in}}%
\pgfusepath{stroke,fill}%
}%
\begin{pgfscope}%
\pgfsys@transformshift{0.444137in}{1.689662in}%
\pgfsys@useobject{currentmarker}{}%
\end{pgfscope}%
\end{pgfscope}%
\begin{pgfscope}%
\pgftext[x=0.100000in,y=1.641835in,left,base]{\rmfamily\fontsize{10.000000}{12.000000}\selectfont \(\displaystyle 0.15\)}%
\end{pgfscope}%
\begin{pgfscope}%
\pgfsetbuttcap%
\pgfsetroundjoin%
\definecolor{currentfill}{rgb}{0.000000,0.000000,0.000000}%
\pgfsetfillcolor{currentfill}%
\pgfsetlinewidth{0.803000pt}%
\definecolor{currentstroke}{rgb}{0.000000,0.000000,0.000000}%
\pgfsetstrokecolor{currentstroke}%
\pgfsetdash{}{0pt}%
\pgfsys@defobject{currentmarker}{\pgfqpoint{-0.048611in}{0.000000in}}{\pgfqpoint{0.000000in}{0.000000in}}{%
\pgfpathmoveto{\pgfqpoint{0.000000in}{0.000000in}}%
\pgfpathlineto{\pgfqpoint{-0.048611in}{0.000000in}}%
\pgfusepath{stroke,fill}%
}%
\begin{pgfscope}%
\pgfsys@transformshift{0.444137in}{2.146257in}%
\pgfsys@useobject{currentmarker}{}%
\end{pgfscope}%
\end{pgfscope}%
\begin{pgfscope}%
\pgftext[x=0.100000in,y=2.098430in,left,base]{\rmfamily\fontsize{10.000000}{12.000000}\selectfont \(\displaystyle 0.20\)}%
\end{pgfscope}%
\begin{pgfscope}%
\pgfsetbuttcap%
\pgfsetroundjoin%
\definecolor{currentfill}{rgb}{0.000000,0.000000,0.000000}%
\pgfsetfillcolor{currentfill}%
\pgfsetlinewidth{0.803000pt}%
\definecolor{currentstroke}{rgb}{0.000000,0.000000,0.000000}%
\pgfsetstrokecolor{currentstroke}%
\pgfsetdash{}{0pt}%
\pgfsys@defobject{currentmarker}{\pgfqpoint{-0.048611in}{0.000000in}}{\pgfqpoint{0.000000in}{0.000000in}}{%
\pgfpathmoveto{\pgfqpoint{0.000000in}{0.000000in}}%
\pgfpathlineto{\pgfqpoint{-0.048611in}{0.000000in}}%
\pgfusepath{stroke,fill}%
}%
\begin{pgfscope}%
\pgfsys@transformshift{0.444137in}{2.602852in}%
\pgfsys@useobject{currentmarker}{}%
\end{pgfscope}%
\end{pgfscope}%
\begin{pgfscope}%
\pgftext[x=0.100000in,y=2.555025in,left,base]{\rmfamily\fontsize{10.000000}{12.000000}\selectfont \(\displaystyle 0.25\)}%
\end{pgfscope}%
\begin{pgfscope}%
\pgfsetrectcap%
\pgfsetmiterjoin%
\pgfsetlinewidth{0.803000pt}%
\definecolor{currentstroke}{rgb}{0.000000,0.000000,0.000000}%
\pgfsetstrokecolor{currentstroke}%
\pgfsetdash{}{0pt}%
\pgfpathmoveto{\pgfqpoint{0.444137in}{0.319877in}}%
\pgfpathlineto{\pgfqpoint{0.444137in}{2.925408in}}%
\pgfusepath{stroke}%
\end{pgfscope}%
\begin{pgfscope}%
\pgfsetrectcap%
\pgfsetmiterjoin%
\pgfsetlinewidth{0.803000pt}%
\definecolor{currentstroke}{rgb}{0.000000,0.000000,0.000000}%
\pgfsetstrokecolor{currentstroke}%
\pgfsetdash{}{0pt}%
\pgfpathmoveto{\pgfqpoint{1.968255in}{0.319877in}}%
\pgfpathlineto{\pgfqpoint{1.968255in}{2.925408in}}%
\pgfusepath{stroke}%
\end{pgfscope}%
\begin{pgfscope}%
\pgfsetrectcap%
\pgfsetmiterjoin%
\pgfsetlinewidth{0.803000pt}%
\definecolor{currentstroke}{rgb}{0.000000,0.000000,0.000000}%
\pgfsetstrokecolor{currentstroke}%
\pgfsetdash{}{0pt}%
\pgfpathmoveto{\pgfqpoint{0.444137in}{0.319877in}}%
\pgfpathlineto{\pgfqpoint{1.968255in}{0.319877in}}%
\pgfusepath{stroke}%
\end{pgfscope}%
\begin{pgfscope}%
\pgfsetrectcap%
\pgfsetmiterjoin%
\pgfsetlinewidth{0.803000pt}%
\definecolor{currentstroke}{rgb}{0.000000,0.000000,0.000000}%
\pgfsetstrokecolor{currentstroke}%
\pgfsetdash{}{0pt}%
\pgfpathmoveto{\pgfqpoint{0.444137in}{2.925408in}}%
\pgfpathlineto{\pgfqpoint{1.968255in}{2.925408in}}%
\pgfusepath{stroke}%
\end{pgfscope}%
\begin{pgfscope}%
\pgfpathrectangle{\pgfqpoint{2.063512in}{0.319877in}}{\pgfqpoint{0.130277in}{2.605531in}} %
\pgfusepath{clip}%
\pgfsetbuttcap%
\pgfsetmiterjoin%
\definecolor{currentfill}{rgb}{1.000000,1.000000,1.000000}%
\pgfsetfillcolor{currentfill}%
\pgfsetlinewidth{0.010037pt}%
\definecolor{currentstroke}{rgb}{1.000000,1.000000,1.000000}%
\pgfsetstrokecolor{currentstroke}%
\pgfsetdash{}{0pt}%
\pgfpathmoveto{\pgfqpoint{2.063512in}{0.319877in}}%
\pgfpathlineto{\pgfqpoint{2.063512in}{0.330055in}}%
\pgfpathlineto{\pgfqpoint{2.063512in}{2.915230in}}%
\pgfpathlineto{\pgfqpoint{2.063512in}{2.925408in}}%
\pgfpathlineto{\pgfqpoint{2.193789in}{2.925408in}}%
\pgfpathlineto{\pgfqpoint{2.193789in}{2.915230in}}%
\pgfpathlineto{\pgfqpoint{2.193789in}{0.330055in}}%
\pgfpathlineto{\pgfqpoint{2.193789in}{0.319877in}}%
\pgfpathclose%
\pgfusepath{stroke,fill}%
\end{pgfscope}%
\begin{pgfscope}%
\pgfsys@transformshift{2.060000in}{0.320408in}%
\pgftext[left,bottom]{\pgfimage[interpolate=true,width=0.130000in,height=2.610000in]{FerrNN_vs_dq_Ti_100K-img1.png}}%
\end{pgfscope}%
\begin{pgfscope}%
\pgfsetbuttcap%
\pgfsetroundjoin%
\definecolor{currentfill}{rgb}{0.000000,0.000000,0.000000}%
\pgfsetfillcolor{currentfill}%
\pgfsetlinewidth{0.803000pt}%
\definecolor{currentstroke}{rgb}{0.000000,0.000000,0.000000}%
\pgfsetstrokecolor{currentstroke}%
\pgfsetdash{}{0pt}%
\pgfsys@defobject{currentmarker}{\pgfqpoint{0.000000in}{0.000000in}}{\pgfqpoint{0.048611in}{0.000000in}}{%
\pgfpathmoveto{\pgfqpoint{0.000000in}{0.000000in}}%
\pgfpathlineto{\pgfqpoint{0.048611in}{0.000000in}}%
\pgfusepath{stroke,fill}%
}%
\begin{pgfscope}%
\pgfsys@transformshift{2.193789in}{0.319877in}%
\pgfsys@useobject{currentmarker}{}%
\end{pgfscope}%
\end{pgfscope}%
\begin{pgfscope}%
\pgftext[x=2.291011in,y=0.272050in,left,base]{\rmfamily\fontsize{10.000000}{12.000000}\selectfont \(\displaystyle 0\)}%
\end{pgfscope}%
\begin{pgfscope}%
\pgfsetbuttcap%
\pgfsetroundjoin%
\definecolor{currentfill}{rgb}{0.000000,0.000000,0.000000}%
\pgfsetfillcolor{currentfill}%
\pgfsetlinewidth{0.803000pt}%
\definecolor{currentstroke}{rgb}{0.000000,0.000000,0.000000}%
\pgfsetstrokecolor{currentstroke}%
\pgfsetdash{}{0pt}%
\pgfsys@defobject{currentmarker}{\pgfqpoint{0.000000in}{0.000000in}}{\pgfqpoint{0.048611in}{0.000000in}}{%
\pgfpathmoveto{\pgfqpoint{0.000000in}{0.000000in}}%
\pgfpathlineto{\pgfqpoint{0.048611in}{0.000000in}}%
\pgfusepath{stroke,fill}%
}%
\begin{pgfscope}%
\pgfsys@transformshift{2.193789in}{0.626410in}%
\pgfsys@useobject{currentmarker}{}%
\end{pgfscope}%
\end{pgfscope}%
\begin{pgfscope}%
\pgftext[x=2.291011in,y=0.578583in,left,base]{\rmfamily\fontsize{10.000000}{12.000000}\selectfont \(\displaystyle 5\)}%
\end{pgfscope}%
\begin{pgfscope}%
\pgfsetbuttcap%
\pgfsetroundjoin%
\definecolor{currentfill}{rgb}{0.000000,0.000000,0.000000}%
\pgfsetfillcolor{currentfill}%
\pgfsetlinewidth{0.803000pt}%
\definecolor{currentstroke}{rgb}{0.000000,0.000000,0.000000}%
\pgfsetstrokecolor{currentstroke}%
\pgfsetdash{}{0pt}%
\pgfsys@defobject{currentmarker}{\pgfqpoint{0.000000in}{0.000000in}}{\pgfqpoint{0.048611in}{0.000000in}}{%
\pgfpathmoveto{\pgfqpoint{0.000000in}{0.000000in}}%
\pgfpathlineto{\pgfqpoint{0.048611in}{0.000000in}}%
\pgfusepath{stroke,fill}%
}%
\begin{pgfscope}%
\pgfsys@transformshift{2.193789in}{0.932943in}%
\pgfsys@useobject{currentmarker}{}%
\end{pgfscope}%
\end{pgfscope}%
\begin{pgfscope}%
\pgftext[x=2.291011in,y=0.885116in,left,base]{\rmfamily\fontsize{10.000000}{12.000000}\selectfont \(\displaystyle 10\)}%
\end{pgfscope}%
\begin{pgfscope}%
\pgfsetbuttcap%
\pgfsetroundjoin%
\definecolor{currentfill}{rgb}{0.000000,0.000000,0.000000}%
\pgfsetfillcolor{currentfill}%
\pgfsetlinewidth{0.803000pt}%
\definecolor{currentstroke}{rgb}{0.000000,0.000000,0.000000}%
\pgfsetstrokecolor{currentstroke}%
\pgfsetdash{}{0pt}%
\pgfsys@defobject{currentmarker}{\pgfqpoint{0.000000in}{0.000000in}}{\pgfqpoint{0.048611in}{0.000000in}}{%
\pgfpathmoveto{\pgfqpoint{0.000000in}{0.000000in}}%
\pgfpathlineto{\pgfqpoint{0.048611in}{0.000000in}}%
\pgfusepath{stroke,fill}%
}%
\begin{pgfscope}%
\pgfsys@transformshift{2.193789in}{1.239476in}%
\pgfsys@useobject{currentmarker}{}%
\end{pgfscope}%
\end{pgfscope}%
\begin{pgfscope}%
\pgftext[x=2.291011in,y=1.191649in,left,base]{\rmfamily\fontsize{10.000000}{12.000000}\selectfont \(\displaystyle 15\)}%
\end{pgfscope}%
\begin{pgfscope}%
\pgfsetbuttcap%
\pgfsetroundjoin%
\definecolor{currentfill}{rgb}{0.000000,0.000000,0.000000}%
\pgfsetfillcolor{currentfill}%
\pgfsetlinewidth{0.803000pt}%
\definecolor{currentstroke}{rgb}{0.000000,0.000000,0.000000}%
\pgfsetstrokecolor{currentstroke}%
\pgfsetdash{}{0pt}%
\pgfsys@defobject{currentmarker}{\pgfqpoint{0.000000in}{0.000000in}}{\pgfqpoint{0.048611in}{0.000000in}}{%
\pgfpathmoveto{\pgfqpoint{0.000000in}{0.000000in}}%
\pgfpathlineto{\pgfqpoint{0.048611in}{0.000000in}}%
\pgfusepath{stroke,fill}%
}%
\begin{pgfscope}%
\pgfsys@transformshift{2.193789in}{1.546009in}%
\pgfsys@useobject{currentmarker}{}%
\end{pgfscope}%
\end{pgfscope}%
\begin{pgfscope}%
\pgftext[x=2.291011in,y=1.498182in,left,base]{\rmfamily\fontsize{10.000000}{12.000000}\selectfont \(\displaystyle 20\)}%
\end{pgfscope}%
\begin{pgfscope}%
\pgfsetbuttcap%
\pgfsetroundjoin%
\definecolor{currentfill}{rgb}{0.000000,0.000000,0.000000}%
\pgfsetfillcolor{currentfill}%
\pgfsetlinewidth{0.803000pt}%
\definecolor{currentstroke}{rgb}{0.000000,0.000000,0.000000}%
\pgfsetstrokecolor{currentstroke}%
\pgfsetdash{}{0pt}%
\pgfsys@defobject{currentmarker}{\pgfqpoint{0.000000in}{0.000000in}}{\pgfqpoint{0.048611in}{0.000000in}}{%
\pgfpathmoveto{\pgfqpoint{0.000000in}{0.000000in}}%
\pgfpathlineto{\pgfqpoint{0.048611in}{0.000000in}}%
\pgfusepath{stroke,fill}%
}%
\begin{pgfscope}%
\pgfsys@transformshift{2.193789in}{1.852542in}%
\pgfsys@useobject{currentmarker}{}%
\end{pgfscope}%
\end{pgfscope}%
\begin{pgfscope}%
\pgftext[x=2.291011in,y=1.804715in,left,base]{\rmfamily\fontsize{10.000000}{12.000000}\selectfont \(\displaystyle 25\)}%
\end{pgfscope}%
\begin{pgfscope}%
\pgfsetbuttcap%
\pgfsetroundjoin%
\definecolor{currentfill}{rgb}{0.000000,0.000000,0.000000}%
\pgfsetfillcolor{currentfill}%
\pgfsetlinewidth{0.803000pt}%
\definecolor{currentstroke}{rgb}{0.000000,0.000000,0.000000}%
\pgfsetstrokecolor{currentstroke}%
\pgfsetdash{}{0pt}%
\pgfsys@defobject{currentmarker}{\pgfqpoint{0.000000in}{0.000000in}}{\pgfqpoint{0.048611in}{0.000000in}}{%
\pgfpathmoveto{\pgfqpoint{0.000000in}{0.000000in}}%
\pgfpathlineto{\pgfqpoint{0.048611in}{0.000000in}}%
\pgfusepath{stroke,fill}%
}%
\begin{pgfscope}%
\pgfsys@transformshift{2.193789in}{2.159075in}%
\pgfsys@useobject{currentmarker}{}%
\end{pgfscope}%
\end{pgfscope}%
\begin{pgfscope}%
\pgftext[x=2.291011in,y=2.111248in,left,base]{\rmfamily\fontsize{10.000000}{12.000000}\selectfont \(\displaystyle 30\)}%
\end{pgfscope}%
\begin{pgfscope}%
\pgfsetbuttcap%
\pgfsetroundjoin%
\definecolor{currentfill}{rgb}{0.000000,0.000000,0.000000}%
\pgfsetfillcolor{currentfill}%
\pgfsetlinewidth{0.803000pt}%
\definecolor{currentstroke}{rgb}{0.000000,0.000000,0.000000}%
\pgfsetstrokecolor{currentstroke}%
\pgfsetdash{}{0pt}%
\pgfsys@defobject{currentmarker}{\pgfqpoint{0.000000in}{0.000000in}}{\pgfqpoint{0.048611in}{0.000000in}}{%
\pgfpathmoveto{\pgfqpoint{0.000000in}{0.000000in}}%
\pgfpathlineto{\pgfqpoint{0.048611in}{0.000000in}}%
\pgfusepath{stroke,fill}%
}%
\begin{pgfscope}%
\pgfsys@transformshift{2.193789in}{2.465608in}%
\pgfsys@useobject{currentmarker}{}%
\end{pgfscope}%
\end{pgfscope}%
\begin{pgfscope}%
\pgftext[x=2.291011in,y=2.417781in,left,base]{\rmfamily\fontsize{10.000000}{12.000000}\selectfont \(\displaystyle 35\)}%
\end{pgfscope}%
\begin{pgfscope}%
\pgfsetbuttcap%
\pgfsetroundjoin%
\definecolor{currentfill}{rgb}{0.000000,0.000000,0.000000}%
\pgfsetfillcolor{currentfill}%
\pgfsetlinewidth{0.803000pt}%
\definecolor{currentstroke}{rgb}{0.000000,0.000000,0.000000}%
\pgfsetstrokecolor{currentstroke}%
\pgfsetdash{}{0pt}%
\pgfsys@defobject{currentmarker}{\pgfqpoint{0.000000in}{0.000000in}}{\pgfqpoint{0.048611in}{0.000000in}}{%
\pgfpathmoveto{\pgfqpoint{0.000000in}{0.000000in}}%
\pgfpathlineto{\pgfqpoint{0.048611in}{0.000000in}}%
\pgfusepath{stroke,fill}%
}%
\begin{pgfscope}%
\pgfsys@transformshift{2.193789in}{2.772141in}%
\pgfsys@useobject{currentmarker}{}%
\end{pgfscope}%
\end{pgfscope}%
\begin{pgfscope}%
\pgftext[x=2.291011in,y=2.724314in,left,base]{\rmfamily\fontsize{10.000000}{12.000000}\selectfont \(\displaystyle 40\)}%
\end{pgfscope}%
\begin{pgfscope}%
\pgfsetbuttcap%
\pgfsetmiterjoin%
\pgfsetlinewidth{0.803000pt}%
\definecolor{currentstroke}{rgb}{0.000000,0.000000,0.000000}%
\pgfsetstrokecolor{currentstroke}%
\pgfsetdash{}{0pt}%
\pgfpathmoveto{\pgfqpoint{2.063512in}{0.319877in}}%
\pgfpathlineto{\pgfqpoint{2.063512in}{0.330055in}}%
\pgfpathlineto{\pgfqpoint{2.063512in}{2.915230in}}%
\pgfpathlineto{\pgfqpoint{2.063512in}{2.925408in}}%
\pgfpathlineto{\pgfqpoint{2.193789in}{2.925408in}}%
\pgfpathlineto{\pgfqpoint{2.193789in}{2.915230in}}%
\pgfpathlineto{\pgfqpoint{2.193789in}{0.330055in}}%
\pgfpathlineto{\pgfqpoint{2.193789in}{0.319877in}}%
\pgfpathclose%
\pgfusepath{stroke}%
\end{pgfscope}%
\end{pgfpicture}%
\makeatother%
\endgroup%

	\vspace*{-0.4cm}
	\caption{100 K. Bin size $0.0105e$}
	\end{subfigure}
	\hspace{0.6cm}
	\begin{subfigure}[b]{0.45\textwidth}
	\hspace*{-0.4cm}
	%% Creator: Matplotlib, PGF backend
%%
%% To include the figure in your LaTeX document, write
%%   \input{<filename>.pgf}
%%
%% Make sure the required packages are loaded in your preamble
%%   \usepackage{pgf}
%%
%% Figures using additional raster images can only be included by \input if
%% they are in the same directory as the main LaTeX file. For loading figures
%% from other directories you can use the `import` package
%%   \usepackage{import}
%% and then include the figures with
%%   \import{<path to file>}{<filename>.pgf}
%%
%% Matplotlib used the following preamble
%%   \usepackage[utf8x]{inputenc}
%%   \usepackage[T1]{fontenc}
%%
\begingroup%
\makeatletter%
\begin{pgfpicture}%
\pgfpathrectangle{\pgfpointorigin}{\pgfqpoint{2.529900in}{3.060408in}}%
\pgfusepath{use as bounding box, clip}%
\begin{pgfscope}%
\pgfsetbuttcap%
\pgfsetmiterjoin%
\definecolor{currentfill}{rgb}{1.000000,1.000000,1.000000}%
\pgfsetfillcolor{currentfill}%
\pgfsetlinewidth{0.000000pt}%
\definecolor{currentstroke}{rgb}{1.000000,1.000000,1.000000}%
\pgfsetstrokecolor{currentstroke}%
\pgfsetdash{}{0pt}%
\pgfpathmoveto{\pgfqpoint{0.000000in}{0.000000in}}%
\pgfpathlineto{\pgfqpoint{2.529900in}{0.000000in}}%
\pgfpathlineto{\pgfqpoint{2.529900in}{3.060408in}}%
\pgfpathlineto{\pgfqpoint{0.000000in}{3.060408in}}%
\pgfpathclose%
\pgfusepath{fill}%
\end{pgfscope}%
\begin{pgfscope}%
\pgfsetbuttcap%
\pgfsetmiterjoin%
\definecolor{currentfill}{rgb}{1.000000,1.000000,1.000000}%
\pgfsetfillcolor{currentfill}%
\pgfsetlinewidth{0.000000pt}%
\definecolor{currentstroke}{rgb}{0.000000,0.000000,0.000000}%
\pgfsetstrokecolor{currentstroke}%
\pgfsetstrokeopacity{0.000000}%
\pgfsetdash{}{0pt}%
\pgfpathmoveto{\pgfqpoint{0.444137in}{0.319877in}}%
\pgfpathlineto{\pgfqpoint{1.968255in}{0.319877in}}%
\pgfpathlineto{\pgfqpoint{1.968255in}{2.925408in}}%
\pgfpathlineto{\pgfqpoint{0.444137in}{2.925408in}}%
\pgfpathclose%
\pgfusepath{fill}%
\end{pgfscope}%
\begin{pgfscope}%
\pgfpathrectangle{\pgfqpoint{0.444137in}{0.319877in}}{\pgfqpoint{1.524118in}{2.605531in}} %
\pgfusepath{clip}%
\pgfsys@transformshift{0.444137in}{0.319877in}%
\pgftext[left,bottom]{\pgfimage[interpolate=true,width=1.530000in,height=2.610000in]{FerrNN_vs_dq_Ti_200K-img0.png}}%
\end{pgfscope}%
\begin{pgfscope}%
\pgfpathrectangle{\pgfqpoint{0.444137in}{0.319877in}}{\pgfqpoint{1.524118in}{2.605531in}} %
\pgfusepath{clip}%
\pgfsetbuttcap%
\pgfsetroundjoin%
\definecolor{currentfill}{rgb}{1.000000,0.752941,0.796078}%
\pgfsetfillcolor{currentfill}%
\pgfsetlinewidth{1.003750pt}%
\definecolor{currentstroke}{rgb}{1.000000,0.752941,0.796078}%
\pgfsetstrokecolor{currentstroke}%
\pgfsetdash{}{0pt}%
\pgfpathmoveto{\pgfqpoint{0.934032in}{1.953195in}}%
\pgfpathcurveto{\pgfqpoint{0.945082in}{1.953195in}}{\pgfqpoint{0.955681in}{1.957585in}}{\pgfqpoint{0.963494in}{1.965398in}}%
\pgfpathcurveto{\pgfqpoint{0.971308in}{1.973212in}}{\pgfqpoint{0.975698in}{1.983811in}}{\pgfqpoint{0.975698in}{1.994861in}}%
\pgfpathcurveto{\pgfqpoint{0.975698in}{2.005911in}}{\pgfqpoint{0.971308in}{2.016510in}}{\pgfqpoint{0.963494in}{2.024324in}}%
\pgfpathcurveto{\pgfqpoint{0.955681in}{2.032138in}}{\pgfqpoint{0.945082in}{2.036528in}}{\pgfqpoint{0.934032in}{2.036528in}}%
\pgfpathcurveto{\pgfqpoint{0.922982in}{2.036528in}}{\pgfqpoint{0.912382in}{2.032138in}}{\pgfqpoint{0.904569in}{2.024324in}}%
\pgfpathcurveto{\pgfqpoint{0.896755in}{2.016510in}}{\pgfqpoint{0.892365in}{2.005911in}}{\pgfqpoint{0.892365in}{1.994861in}}%
\pgfpathcurveto{\pgfqpoint{0.892365in}{1.983811in}}{\pgfqpoint{0.896755in}{1.973212in}}{\pgfqpoint{0.904569in}{1.965398in}}%
\pgfpathcurveto{\pgfqpoint{0.912382in}{1.957585in}}{\pgfqpoint{0.922982in}{1.953195in}}{\pgfqpoint{0.934032in}{1.953195in}}%
\pgfpathclose%
\pgfusepath{stroke,fill}%
\end{pgfscope}%
\begin{pgfscope}%
\pgfpathrectangle{\pgfqpoint{0.444137in}{0.319877in}}{\pgfqpoint{1.524118in}{2.605531in}} %
\pgfusepath{clip}%
\pgfsetbuttcap%
\pgfsetroundjoin%
\definecolor{currentfill}{rgb}{1.000000,0.752941,0.796078}%
\pgfsetfillcolor{currentfill}%
\pgfsetlinewidth{1.003750pt}%
\definecolor{currentstroke}{rgb}{1.000000,0.752941,0.796078}%
\pgfsetstrokecolor{currentstroke}%
\pgfsetdash{}{0pt}%
\pgfpathmoveto{\pgfqpoint{1.042897in}{1.361949in}}%
\pgfpathcurveto{\pgfqpoint{1.053947in}{1.361949in}}{\pgfqpoint{1.064546in}{1.366340in}}{\pgfqpoint{1.072360in}{1.374153in}}%
\pgfpathcurveto{\pgfqpoint{1.080174in}{1.381967in}}{\pgfqpoint{1.084564in}{1.392566in}}{\pgfqpoint{1.084564in}{1.403616in}}%
\pgfpathcurveto{\pgfqpoint{1.084564in}{1.414666in}}{\pgfqpoint{1.080174in}{1.425265in}}{\pgfqpoint{1.072360in}{1.433079in}}%
\pgfpathcurveto{\pgfqpoint{1.064546in}{1.440892in}}{\pgfqpoint{1.053947in}{1.445283in}}{\pgfqpoint{1.042897in}{1.445283in}}%
\pgfpathcurveto{\pgfqpoint{1.031847in}{1.445283in}}{\pgfqpoint{1.021248in}{1.440892in}}{\pgfqpoint{1.013434in}{1.433079in}}%
\pgfpathcurveto{\pgfqpoint{1.005621in}{1.425265in}}{\pgfqpoint{1.001231in}{1.414666in}}{\pgfqpoint{1.001231in}{1.403616in}}%
\pgfpathcurveto{\pgfqpoint{1.001231in}{1.392566in}}{\pgfqpoint{1.005621in}{1.381967in}}{\pgfqpoint{1.013434in}{1.374153in}}%
\pgfpathcurveto{\pgfqpoint{1.021248in}{1.366340in}}{\pgfqpoint{1.031847in}{1.361949in}}{\pgfqpoint{1.042897in}{1.361949in}}%
\pgfpathclose%
\pgfusepath{stroke,fill}%
\end{pgfscope}%
\begin{pgfscope}%
\pgfpathrectangle{\pgfqpoint{0.444137in}{0.319877in}}{\pgfqpoint{1.524118in}{2.605531in}} %
\pgfusepath{clip}%
\pgfsetbuttcap%
\pgfsetroundjoin%
\definecolor{currentfill}{rgb}{1.000000,0.752941,0.796078}%
\pgfsetfillcolor{currentfill}%
\pgfsetlinewidth{1.003750pt}%
\definecolor{currentstroke}{rgb}{1.000000,0.752941,0.796078}%
\pgfsetstrokecolor{currentstroke}%
\pgfsetdash{}{0pt}%
\pgfpathmoveto{\pgfqpoint{1.151763in}{1.059727in}}%
\pgfpathcurveto{\pgfqpoint{1.162813in}{1.059727in}}{\pgfqpoint{1.173412in}{1.064117in}}{\pgfqpoint{1.181226in}{1.071931in}}%
\pgfpathcurveto{\pgfqpoint{1.189039in}{1.079745in}}{\pgfqpoint{1.193429in}{1.090344in}}{\pgfqpoint{1.193429in}{1.101394in}}%
\pgfpathcurveto{\pgfqpoint{1.193429in}{1.112444in}}{\pgfqpoint{1.189039in}{1.123043in}}{\pgfqpoint{1.181226in}{1.130857in}}%
\pgfpathcurveto{\pgfqpoint{1.173412in}{1.138670in}}{\pgfqpoint{1.162813in}{1.143061in}}{\pgfqpoint{1.151763in}{1.143061in}}%
\pgfpathcurveto{\pgfqpoint{1.140713in}{1.143061in}}{\pgfqpoint{1.130114in}{1.138670in}}{\pgfqpoint{1.122300in}{1.130857in}}%
\pgfpathcurveto{\pgfqpoint{1.114486in}{1.123043in}}{\pgfqpoint{1.110096in}{1.112444in}}{\pgfqpoint{1.110096in}{1.101394in}}%
\pgfpathcurveto{\pgfqpoint{1.110096in}{1.090344in}}{\pgfqpoint{1.114486in}{1.079745in}}{\pgfqpoint{1.122300in}{1.071931in}}%
\pgfpathcurveto{\pgfqpoint{1.130114in}{1.064117in}}{\pgfqpoint{1.140713in}{1.059727in}}{\pgfqpoint{1.151763in}{1.059727in}}%
\pgfpathclose%
\pgfusepath{stroke,fill}%
\end{pgfscope}%
\begin{pgfscope}%
\pgfpathrectangle{\pgfqpoint{0.444137in}{0.319877in}}{\pgfqpoint{1.524118in}{2.605531in}} %
\pgfusepath{clip}%
\pgfsetbuttcap%
\pgfsetroundjoin%
\definecolor{currentfill}{rgb}{1.000000,0.752941,0.796078}%
\pgfsetfillcolor{currentfill}%
\pgfsetlinewidth{1.003750pt}%
\definecolor{currentstroke}{rgb}{1.000000,0.752941,0.796078}%
\pgfsetstrokecolor{currentstroke}%
\pgfsetdash{}{0pt}%
\pgfpathmoveto{\pgfqpoint{1.260628in}{1.105907in}}%
\pgfpathcurveto{\pgfqpoint{1.271679in}{1.105907in}}{\pgfqpoint{1.282278in}{1.110298in}}{\pgfqpoint{1.290091in}{1.118111in}}%
\pgfpathcurveto{\pgfqpoint{1.297905in}{1.125925in}}{\pgfqpoint{1.302295in}{1.136524in}}{\pgfqpoint{1.302295in}{1.147574in}}%
\pgfpathcurveto{\pgfqpoint{1.302295in}{1.158624in}}{\pgfqpoint{1.297905in}{1.169223in}}{\pgfqpoint{1.290091in}{1.177037in}}%
\pgfpathcurveto{\pgfqpoint{1.282278in}{1.184850in}}{\pgfqpoint{1.271679in}{1.189241in}}{\pgfqpoint{1.260628in}{1.189241in}}%
\pgfpathcurveto{\pgfqpoint{1.249578in}{1.189241in}}{\pgfqpoint{1.238979in}{1.184850in}}{\pgfqpoint{1.231166in}{1.177037in}}%
\pgfpathcurveto{\pgfqpoint{1.223352in}{1.169223in}}{\pgfqpoint{1.218962in}{1.158624in}}{\pgfqpoint{1.218962in}{1.147574in}}%
\pgfpathcurveto{\pgfqpoint{1.218962in}{1.136524in}}{\pgfqpoint{1.223352in}{1.125925in}}{\pgfqpoint{1.231166in}{1.118111in}}%
\pgfpathcurveto{\pgfqpoint{1.238979in}{1.110298in}}{\pgfqpoint{1.249578in}{1.105907in}}{\pgfqpoint{1.260628in}{1.105907in}}%
\pgfpathclose%
\pgfusepath{stroke,fill}%
\end{pgfscope}%
\begin{pgfscope}%
\pgfpathrectangle{\pgfqpoint{0.444137in}{0.319877in}}{\pgfqpoint{1.524118in}{2.605531in}} %
\pgfusepath{clip}%
\pgfsetbuttcap%
\pgfsetroundjoin%
\definecolor{currentfill}{rgb}{1.000000,0.752941,0.796078}%
\pgfsetfillcolor{currentfill}%
\pgfsetlinewidth{1.003750pt}%
\definecolor{currentstroke}{rgb}{1.000000,0.752941,0.796078}%
\pgfsetstrokecolor{currentstroke}%
\pgfsetdash{}{0pt}%
\pgfpathmoveto{\pgfqpoint{1.369494in}{1.471002in}}%
\pgfpathcurveto{\pgfqpoint{1.380544in}{1.471002in}}{\pgfqpoint{1.391143in}{1.475393in}}{\pgfqpoint{1.398957in}{1.483206in}}%
\pgfpathcurveto{\pgfqpoint{1.406770in}{1.491020in}}{\pgfqpoint{1.411161in}{1.501619in}}{\pgfqpoint{1.411161in}{1.512669in}}%
\pgfpathcurveto{\pgfqpoint{1.411161in}{1.523719in}}{\pgfqpoint{1.406770in}{1.534318in}}{\pgfqpoint{1.398957in}{1.542132in}}%
\pgfpathcurveto{\pgfqpoint{1.391143in}{1.549945in}}{\pgfqpoint{1.380544in}{1.554336in}}{\pgfqpoint{1.369494in}{1.554336in}}%
\pgfpathcurveto{\pgfqpoint{1.358444in}{1.554336in}}{\pgfqpoint{1.347845in}{1.549945in}}{\pgfqpoint{1.340031in}{1.542132in}}%
\pgfpathcurveto{\pgfqpoint{1.332218in}{1.534318in}}{\pgfqpoint{1.327827in}{1.523719in}}{\pgfqpoint{1.327827in}{1.512669in}}%
\pgfpathcurveto{\pgfqpoint{1.327827in}{1.501619in}}{\pgfqpoint{1.332218in}{1.491020in}}{\pgfqpoint{1.340031in}{1.483206in}}%
\pgfpathcurveto{\pgfqpoint{1.347845in}{1.475393in}}{\pgfqpoint{1.358444in}{1.471002in}}{\pgfqpoint{1.369494in}{1.471002in}}%
\pgfpathclose%
\pgfusepath{stroke,fill}%
\end{pgfscope}%
\begin{pgfscope}%
\pgfpathrectangle{\pgfqpoint{0.444137in}{0.319877in}}{\pgfqpoint{1.524118in}{2.605531in}} %
\pgfusepath{clip}%
\pgfsetbuttcap%
\pgfsetroundjoin%
\definecolor{currentfill}{rgb}{1.000000,0.752941,0.796078}%
\pgfsetfillcolor{currentfill}%
\pgfsetlinewidth{1.003750pt}%
\definecolor{currentstroke}{rgb}{1.000000,0.752941,0.796078}%
\pgfsetstrokecolor{currentstroke}%
\pgfsetdash{}{0pt}%
\pgfpathmoveto{\pgfqpoint{1.478360in}{1.994552in}}%
\pgfpathcurveto{\pgfqpoint{1.489410in}{1.994552in}}{\pgfqpoint{1.500009in}{1.998942in}}{\pgfqpoint{1.507822in}{2.006756in}}%
\pgfpathcurveto{\pgfqpoint{1.515636in}{2.014570in}}{\pgfqpoint{1.520026in}{2.025169in}}{\pgfqpoint{1.520026in}{2.036219in}}%
\pgfpathcurveto{\pgfqpoint{1.520026in}{2.047269in}}{\pgfqpoint{1.515636in}{2.057868in}}{\pgfqpoint{1.507822in}{2.065682in}}%
\pgfpathcurveto{\pgfqpoint{1.500009in}{2.073495in}}{\pgfqpoint{1.489410in}{2.077886in}}{\pgfqpoint{1.478360in}{2.077886in}}%
\pgfpathcurveto{\pgfqpoint{1.467309in}{2.077886in}}{\pgfqpoint{1.456710in}{2.073495in}}{\pgfqpoint{1.448897in}{2.065682in}}%
\pgfpathcurveto{\pgfqpoint{1.441083in}{2.057868in}}{\pgfqpoint{1.436693in}{2.047269in}}{\pgfqpoint{1.436693in}{2.036219in}}%
\pgfpathcurveto{\pgfqpoint{1.436693in}{2.025169in}}{\pgfqpoint{1.441083in}{2.014570in}}{\pgfqpoint{1.448897in}{2.006756in}}%
\pgfpathcurveto{\pgfqpoint{1.456710in}{1.998942in}}{\pgfqpoint{1.467309in}{1.994552in}}{\pgfqpoint{1.478360in}{1.994552in}}%
\pgfpathclose%
\pgfusepath{stroke,fill}%
\end{pgfscope}%
\begin{pgfscope}%
\pgfpathrectangle{\pgfqpoint{0.444137in}{0.319877in}}{\pgfqpoint{1.524118in}{2.605531in}} %
\pgfusepath{clip}%
\pgfsetbuttcap%
\pgfsetroundjoin%
\definecolor{currentfill}{rgb}{1.000000,0.752941,0.796078}%
\pgfsetfillcolor{currentfill}%
\pgfsetlinewidth{1.003750pt}%
\definecolor{currentstroke}{rgb}{1.000000,0.752941,0.796078}%
\pgfsetstrokecolor{currentstroke}%
\pgfsetdash{}{0pt}%
\pgfpathmoveto{\pgfqpoint{1.587225in}{2.449486in}}%
\pgfpathcurveto{\pgfqpoint{1.598275in}{2.449486in}}{\pgfqpoint{1.608874in}{2.453876in}}{\pgfqpoint{1.616688in}{2.461690in}}%
\pgfpathcurveto{\pgfqpoint{1.624502in}{2.469504in}}{\pgfqpoint{1.628892in}{2.480103in}}{\pgfqpoint{1.628892in}{2.491153in}}%
\pgfpathcurveto{\pgfqpoint{1.628892in}{2.502203in}}{\pgfqpoint{1.624502in}{2.512802in}}{\pgfqpoint{1.616688in}{2.520616in}}%
\pgfpathcurveto{\pgfqpoint{1.608874in}{2.528429in}}{\pgfqpoint{1.598275in}{2.532819in}}{\pgfqpoint{1.587225in}{2.532819in}}%
\pgfpathcurveto{\pgfqpoint{1.576175in}{2.532819in}}{\pgfqpoint{1.565576in}{2.528429in}}{\pgfqpoint{1.557762in}{2.520616in}}%
\pgfpathcurveto{\pgfqpoint{1.549949in}{2.512802in}}{\pgfqpoint{1.545558in}{2.502203in}}{\pgfqpoint{1.545558in}{2.491153in}}%
\pgfpathcurveto{\pgfqpoint{1.545558in}{2.480103in}}{\pgfqpoint{1.549949in}{2.469504in}}{\pgfqpoint{1.557762in}{2.461690in}}%
\pgfpathcurveto{\pgfqpoint{1.565576in}{2.453876in}}{\pgfqpoint{1.576175in}{2.449486in}}{\pgfqpoint{1.587225in}{2.449486in}}%
\pgfpathclose%
\pgfusepath{stroke,fill}%
\end{pgfscope}%
\begin{pgfscope}%
\pgfsetbuttcap%
\pgfsetroundjoin%
\definecolor{currentfill}{rgb}{0.000000,0.000000,0.000000}%
\pgfsetfillcolor{currentfill}%
\pgfsetlinewidth{0.803000pt}%
\definecolor{currentstroke}{rgb}{0.000000,0.000000,0.000000}%
\pgfsetstrokecolor{currentstroke}%
\pgfsetdash{}{0pt}%
\pgfsys@defobject{currentmarker}{\pgfqpoint{0.000000in}{-0.048611in}}{\pgfqpoint{0.000000in}{0.000000in}}{%
\pgfpathmoveto{\pgfqpoint{0.000000in}{0.000000in}}%
\pgfpathlineto{\pgfqpoint{0.000000in}{-0.048611in}}%
\pgfusepath{stroke,fill}%
}%
\begin{pgfscope}%
\pgfsys@transformshift{0.729909in}{0.319877in}%
\pgfsys@useobject{currentmarker}{}%
\end{pgfscope}%
\end{pgfscope}%
\begin{pgfscope}%
\pgftext[x=0.729909in,y=0.222655in,,top]{\rmfamily\fontsize{10.000000}{12.000000}\selectfont \(\displaystyle -0.05\)}%
\end{pgfscope}%
\begin{pgfscope}%
\pgfsetbuttcap%
\pgfsetroundjoin%
\definecolor{currentfill}{rgb}{0.000000,0.000000,0.000000}%
\pgfsetfillcolor{currentfill}%
\pgfsetlinewidth{0.803000pt}%
\definecolor{currentstroke}{rgb}{0.000000,0.000000,0.000000}%
\pgfsetstrokecolor{currentstroke}%
\pgfsetdash{}{0pt}%
\pgfsys@defobject{currentmarker}{\pgfqpoint{0.000000in}{-0.048611in}}{\pgfqpoint{0.000000in}{0.000000in}}{%
\pgfpathmoveto{\pgfqpoint{0.000000in}{0.000000in}}%
\pgfpathlineto{\pgfqpoint{0.000000in}{-0.048611in}}%
\pgfusepath{stroke,fill}%
}%
\begin{pgfscope}%
\pgfsys@transformshift{1.206196in}{0.319877in}%
\pgfsys@useobject{currentmarker}{}%
\end{pgfscope}%
\end{pgfscope}%
\begin{pgfscope}%
\pgftext[x=1.206196in,y=0.222655in,,top]{\rmfamily\fontsize{10.000000}{12.000000}\selectfont \(\displaystyle 0.00\)}%
\end{pgfscope}%
\begin{pgfscope}%
\pgfsetbuttcap%
\pgfsetroundjoin%
\definecolor{currentfill}{rgb}{0.000000,0.000000,0.000000}%
\pgfsetfillcolor{currentfill}%
\pgfsetlinewidth{0.803000pt}%
\definecolor{currentstroke}{rgb}{0.000000,0.000000,0.000000}%
\pgfsetstrokecolor{currentstroke}%
\pgfsetdash{}{0pt}%
\pgfsys@defobject{currentmarker}{\pgfqpoint{0.000000in}{-0.048611in}}{\pgfqpoint{0.000000in}{0.000000in}}{%
\pgfpathmoveto{\pgfqpoint{0.000000in}{0.000000in}}%
\pgfpathlineto{\pgfqpoint{0.000000in}{-0.048611in}}%
\pgfusepath{stroke,fill}%
}%
\begin{pgfscope}%
\pgfsys@transformshift{1.682483in}{0.319877in}%
\pgfsys@useobject{currentmarker}{}%
\end{pgfscope}%
\end{pgfscope}%
\begin{pgfscope}%
\pgftext[x=1.682483in,y=0.222655in,,top]{\rmfamily\fontsize{10.000000}{12.000000}\selectfont \(\displaystyle 0.05\)}%
\end{pgfscope}%
\begin{pgfscope}%
\pgfsetbuttcap%
\pgfsetroundjoin%
\definecolor{currentfill}{rgb}{0.000000,0.000000,0.000000}%
\pgfsetfillcolor{currentfill}%
\pgfsetlinewidth{0.803000pt}%
\definecolor{currentstroke}{rgb}{0.000000,0.000000,0.000000}%
\pgfsetstrokecolor{currentstroke}%
\pgfsetdash{}{0pt}%
\pgfsys@defobject{currentmarker}{\pgfqpoint{-0.048611in}{0.000000in}}{\pgfqpoint{0.000000in}{0.000000in}}{%
\pgfpathmoveto{\pgfqpoint{0.000000in}{0.000000in}}%
\pgfpathlineto{\pgfqpoint{-0.048611in}{0.000000in}}%
\pgfusepath{stroke,fill}%
}%
\begin{pgfscope}%
\pgfsys@transformshift{0.444137in}{0.319877in}%
\pgfsys@useobject{currentmarker}{}%
\end{pgfscope}%
\end{pgfscope}%
\begin{pgfscope}%
\pgftext[x=0.100000in,y=0.272050in,left,base]{\rmfamily\fontsize{10.000000}{12.000000}\selectfont \(\displaystyle 0.00\)}%
\end{pgfscope}%
\begin{pgfscope}%
\pgfsetbuttcap%
\pgfsetroundjoin%
\definecolor{currentfill}{rgb}{0.000000,0.000000,0.000000}%
\pgfsetfillcolor{currentfill}%
\pgfsetlinewidth{0.803000pt}%
\definecolor{currentstroke}{rgb}{0.000000,0.000000,0.000000}%
\pgfsetstrokecolor{currentstroke}%
\pgfsetdash{}{0pt}%
\pgfsys@defobject{currentmarker}{\pgfqpoint{-0.048611in}{0.000000in}}{\pgfqpoint{0.000000in}{0.000000in}}{%
\pgfpathmoveto{\pgfqpoint{0.000000in}{0.000000in}}%
\pgfpathlineto{\pgfqpoint{-0.048611in}{0.000000in}}%
\pgfusepath{stroke,fill}%
}%
\begin{pgfscope}%
\pgfsys@transformshift{0.444137in}{0.776472in}%
\pgfsys@useobject{currentmarker}{}%
\end{pgfscope}%
\end{pgfscope}%
\begin{pgfscope}%
\pgftext[x=0.100000in,y=0.728645in,left,base]{\rmfamily\fontsize{10.000000}{12.000000}\selectfont \(\displaystyle 0.05\)}%
\end{pgfscope}%
\begin{pgfscope}%
\pgfsetbuttcap%
\pgfsetroundjoin%
\definecolor{currentfill}{rgb}{0.000000,0.000000,0.000000}%
\pgfsetfillcolor{currentfill}%
\pgfsetlinewidth{0.803000pt}%
\definecolor{currentstroke}{rgb}{0.000000,0.000000,0.000000}%
\pgfsetstrokecolor{currentstroke}%
\pgfsetdash{}{0pt}%
\pgfsys@defobject{currentmarker}{\pgfqpoint{-0.048611in}{0.000000in}}{\pgfqpoint{0.000000in}{0.000000in}}{%
\pgfpathmoveto{\pgfqpoint{0.000000in}{0.000000in}}%
\pgfpathlineto{\pgfqpoint{-0.048611in}{0.000000in}}%
\pgfusepath{stroke,fill}%
}%
\begin{pgfscope}%
\pgfsys@transformshift{0.444137in}{1.233067in}%
\pgfsys@useobject{currentmarker}{}%
\end{pgfscope}%
\end{pgfscope}%
\begin{pgfscope}%
\pgftext[x=0.100000in,y=1.185240in,left,base]{\rmfamily\fontsize{10.000000}{12.000000}\selectfont \(\displaystyle 0.10\)}%
\end{pgfscope}%
\begin{pgfscope}%
\pgfsetbuttcap%
\pgfsetroundjoin%
\definecolor{currentfill}{rgb}{0.000000,0.000000,0.000000}%
\pgfsetfillcolor{currentfill}%
\pgfsetlinewidth{0.803000pt}%
\definecolor{currentstroke}{rgb}{0.000000,0.000000,0.000000}%
\pgfsetstrokecolor{currentstroke}%
\pgfsetdash{}{0pt}%
\pgfsys@defobject{currentmarker}{\pgfqpoint{-0.048611in}{0.000000in}}{\pgfqpoint{0.000000in}{0.000000in}}{%
\pgfpathmoveto{\pgfqpoint{0.000000in}{0.000000in}}%
\pgfpathlineto{\pgfqpoint{-0.048611in}{0.000000in}}%
\pgfusepath{stroke,fill}%
}%
\begin{pgfscope}%
\pgfsys@transformshift{0.444137in}{1.689662in}%
\pgfsys@useobject{currentmarker}{}%
\end{pgfscope}%
\end{pgfscope}%
\begin{pgfscope}%
\pgftext[x=0.100000in,y=1.641835in,left,base]{\rmfamily\fontsize{10.000000}{12.000000}\selectfont \(\displaystyle 0.15\)}%
\end{pgfscope}%
\begin{pgfscope}%
\pgfsetbuttcap%
\pgfsetroundjoin%
\definecolor{currentfill}{rgb}{0.000000,0.000000,0.000000}%
\pgfsetfillcolor{currentfill}%
\pgfsetlinewidth{0.803000pt}%
\definecolor{currentstroke}{rgb}{0.000000,0.000000,0.000000}%
\pgfsetstrokecolor{currentstroke}%
\pgfsetdash{}{0pt}%
\pgfsys@defobject{currentmarker}{\pgfqpoint{-0.048611in}{0.000000in}}{\pgfqpoint{0.000000in}{0.000000in}}{%
\pgfpathmoveto{\pgfqpoint{0.000000in}{0.000000in}}%
\pgfpathlineto{\pgfqpoint{-0.048611in}{0.000000in}}%
\pgfusepath{stroke,fill}%
}%
\begin{pgfscope}%
\pgfsys@transformshift{0.444137in}{2.146257in}%
\pgfsys@useobject{currentmarker}{}%
\end{pgfscope}%
\end{pgfscope}%
\begin{pgfscope}%
\pgftext[x=0.100000in,y=2.098430in,left,base]{\rmfamily\fontsize{10.000000}{12.000000}\selectfont \(\displaystyle 0.20\)}%
\end{pgfscope}%
\begin{pgfscope}%
\pgfsetbuttcap%
\pgfsetroundjoin%
\definecolor{currentfill}{rgb}{0.000000,0.000000,0.000000}%
\pgfsetfillcolor{currentfill}%
\pgfsetlinewidth{0.803000pt}%
\definecolor{currentstroke}{rgb}{0.000000,0.000000,0.000000}%
\pgfsetstrokecolor{currentstroke}%
\pgfsetdash{}{0pt}%
\pgfsys@defobject{currentmarker}{\pgfqpoint{-0.048611in}{0.000000in}}{\pgfqpoint{0.000000in}{0.000000in}}{%
\pgfpathmoveto{\pgfqpoint{0.000000in}{0.000000in}}%
\pgfpathlineto{\pgfqpoint{-0.048611in}{0.000000in}}%
\pgfusepath{stroke,fill}%
}%
\begin{pgfscope}%
\pgfsys@transformshift{0.444137in}{2.602852in}%
\pgfsys@useobject{currentmarker}{}%
\end{pgfscope}%
\end{pgfscope}%
\begin{pgfscope}%
\pgftext[x=0.100000in,y=2.555025in,left,base]{\rmfamily\fontsize{10.000000}{12.000000}\selectfont \(\displaystyle 0.25\)}%
\end{pgfscope}%
\begin{pgfscope}%
\pgfsetrectcap%
\pgfsetmiterjoin%
\pgfsetlinewidth{0.803000pt}%
\definecolor{currentstroke}{rgb}{0.000000,0.000000,0.000000}%
\pgfsetstrokecolor{currentstroke}%
\pgfsetdash{}{0pt}%
\pgfpathmoveto{\pgfqpoint{0.444137in}{0.319877in}}%
\pgfpathlineto{\pgfqpoint{0.444137in}{2.925408in}}%
\pgfusepath{stroke}%
\end{pgfscope}%
\begin{pgfscope}%
\pgfsetrectcap%
\pgfsetmiterjoin%
\pgfsetlinewidth{0.803000pt}%
\definecolor{currentstroke}{rgb}{0.000000,0.000000,0.000000}%
\pgfsetstrokecolor{currentstroke}%
\pgfsetdash{}{0pt}%
\pgfpathmoveto{\pgfqpoint{1.968255in}{0.319877in}}%
\pgfpathlineto{\pgfqpoint{1.968255in}{2.925408in}}%
\pgfusepath{stroke}%
\end{pgfscope}%
\begin{pgfscope}%
\pgfsetrectcap%
\pgfsetmiterjoin%
\pgfsetlinewidth{0.803000pt}%
\definecolor{currentstroke}{rgb}{0.000000,0.000000,0.000000}%
\pgfsetstrokecolor{currentstroke}%
\pgfsetdash{}{0pt}%
\pgfpathmoveto{\pgfqpoint{0.444137in}{0.319877in}}%
\pgfpathlineto{\pgfqpoint{1.968255in}{0.319877in}}%
\pgfusepath{stroke}%
\end{pgfscope}%
\begin{pgfscope}%
\pgfsetrectcap%
\pgfsetmiterjoin%
\pgfsetlinewidth{0.803000pt}%
\definecolor{currentstroke}{rgb}{0.000000,0.000000,0.000000}%
\pgfsetstrokecolor{currentstroke}%
\pgfsetdash{}{0pt}%
\pgfpathmoveto{\pgfqpoint{0.444137in}{2.925408in}}%
\pgfpathlineto{\pgfqpoint{1.968255in}{2.925408in}}%
\pgfusepath{stroke}%
\end{pgfscope}%
\begin{pgfscope}%
\pgfpathrectangle{\pgfqpoint{2.063512in}{0.319877in}}{\pgfqpoint{0.130277in}{2.605531in}} %
\pgfusepath{clip}%
\pgfsetbuttcap%
\pgfsetmiterjoin%
\definecolor{currentfill}{rgb}{1.000000,1.000000,1.000000}%
\pgfsetfillcolor{currentfill}%
\pgfsetlinewidth{0.010037pt}%
\definecolor{currentstroke}{rgb}{1.000000,1.000000,1.000000}%
\pgfsetstrokecolor{currentstroke}%
\pgfsetdash{}{0pt}%
\pgfpathmoveto{\pgfqpoint{2.063512in}{0.319877in}}%
\pgfpathlineto{\pgfqpoint{2.063512in}{0.330055in}}%
\pgfpathlineto{\pgfqpoint{2.063512in}{2.915230in}}%
\pgfpathlineto{\pgfqpoint{2.063512in}{2.925408in}}%
\pgfpathlineto{\pgfqpoint{2.193789in}{2.925408in}}%
\pgfpathlineto{\pgfqpoint{2.193789in}{2.915230in}}%
\pgfpathlineto{\pgfqpoint{2.193789in}{0.330055in}}%
\pgfpathlineto{\pgfqpoint{2.193789in}{0.319877in}}%
\pgfpathclose%
\pgfusepath{stroke,fill}%
\end{pgfscope}%
\begin{pgfscope}%
\pgfsys@transformshift{2.060000in}{0.320408in}%
\pgftext[left,bottom]{\pgfimage[interpolate=true,width=0.130000in,height=2.610000in]{FerrNN_vs_dq_Ti_200K-img1.png}}%
\end{pgfscope}%
\begin{pgfscope}%
\pgfsetbuttcap%
\pgfsetroundjoin%
\definecolor{currentfill}{rgb}{0.000000,0.000000,0.000000}%
\pgfsetfillcolor{currentfill}%
\pgfsetlinewidth{0.803000pt}%
\definecolor{currentstroke}{rgb}{0.000000,0.000000,0.000000}%
\pgfsetstrokecolor{currentstroke}%
\pgfsetdash{}{0pt}%
\pgfsys@defobject{currentmarker}{\pgfqpoint{0.000000in}{0.000000in}}{\pgfqpoint{0.048611in}{0.000000in}}{%
\pgfpathmoveto{\pgfqpoint{0.000000in}{0.000000in}}%
\pgfpathlineto{\pgfqpoint{0.048611in}{0.000000in}}%
\pgfusepath{stroke,fill}%
}%
\begin{pgfscope}%
\pgfsys@transformshift{2.193789in}{0.319877in}%
\pgfsys@useobject{currentmarker}{}%
\end{pgfscope}%
\end{pgfscope}%
\begin{pgfscope}%
\pgftext[x=2.291011in,y=0.272050in,left,base]{\rmfamily\fontsize{10.000000}{12.000000}\selectfont \(\displaystyle 0\)}%
\end{pgfscope}%
\begin{pgfscope}%
\pgfsetbuttcap%
\pgfsetroundjoin%
\definecolor{currentfill}{rgb}{0.000000,0.000000,0.000000}%
\pgfsetfillcolor{currentfill}%
\pgfsetlinewidth{0.803000pt}%
\definecolor{currentstroke}{rgb}{0.000000,0.000000,0.000000}%
\pgfsetstrokecolor{currentstroke}%
\pgfsetdash{}{0pt}%
\pgfsys@defobject{currentmarker}{\pgfqpoint{0.000000in}{0.000000in}}{\pgfqpoint{0.048611in}{0.000000in}}{%
\pgfpathmoveto{\pgfqpoint{0.000000in}{0.000000in}}%
\pgfpathlineto{\pgfqpoint{0.048611in}{0.000000in}}%
\pgfusepath{stroke,fill}%
}%
\begin{pgfscope}%
\pgfsys@transformshift{2.193789in}{0.626410in}%
\pgfsys@useobject{currentmarker}{}%
\end{pgfscope}%
\end{pgfscope}%
\begin{pgfscope}%
\pgftext[x=2.291011in,y=0.578583in,left,base]{\rmfamily\fontsize{10.000000}{12.000000}\selectfont \(\displaystyle 5\)}%
\end{pgfscope}%
\begin{pgfscope}%
\pgfsetbuttcap%
\pgfsetroundjoin%
\definecolor{currentfill}{rgb}{0.000000,0.000000,0.000000}%
\pgfsetfillcolor{currentfill}%
\pgfsetlinewidth{0.803000pt}%
\definecolor{currentstroke}{rgb}{0.000000,0.000000,0.000000}%
\pgfsetstrokecolor{currentstroke}%
\pgfsetdash{}{0pt}%
\pgfsys@defobject{currentmarker}{\pgfqpoint{0.000000in}{0.000000in}}{\pgfqpoint{0.048611in}{0.000000in}}{%
\pgfpathmoveto{\pgfqpoint{0.000000in}{0.000000in}}%
\pgfpathlineto{\pgfqpoint{0.048611in}{0.000000in}}%
\pgfusepath{stroke,fill}%
}%
\begin{pgfscope}%
\pgfsys@transformshift{2.193789in}{0.932943in}%
\pgfsys@useobject{currentmarker}{}%
\end{pgfscope}%
\end{pgfscope}%
\begin{pgfscope}%
\pgftext[x=2.291011in,y=0.885116in,left,base]{\rmfamily\fontsize{10.000000}{12.000000}\selectfont \(\displaystyle 10\)}%
\end{pgfscope}%
\begin{pgfscope}%
\pgfsetbuttcap%
\pgfsetroundjoin%
\definecolor{currentfill}{rgb}{0.000000,0.000000,0.000000}%
\pgfsetfillcolor{currentfill}%
\pgfsetlinewidth{0.803000pt}%
\definecolor{currentstroke}{rgb}{0.000000,0.000000,0.000000}%
\pgfsetstrokecolor{currentstroke}%
\pgfsetdash{}{0pt}%
\pgfsys@defobject{currentmarker}{\pgfqpoint{0.000000in}{0.000000in}}{\pgfqpoint{0.048611in}{0.000000in}}{%
\pgfpathmoveto{\pgfqpoint{0.000000in}{0.000000in}}%
\pgfpathlineto{\pgfqpoint{0.048611in}{0.000000in}}%
\pgfusepath{stroke,fill}%
}%
\begin{pgfscope}%
\pgfsys@transformshift{2.193789in}{1.239476in}%
\pgfsys@useobject{currentmarker}{}%
\end{pgfscope}%
\end{pgfscope}%
\begin{pgfscope}%
\pgftext[x=2.291011in,y=1.191649in,left,base]{\rmfamily\fontsize{10.000000}{12.000000}\selectfont \(\displaystyle 15\)}%
\end{pgfscope}%
\begin{pgfscope}%
\pgfsetbuttcap%
\pgfsetroundjoin%
\definecolor{currentfill}{rgb}{0.000000,0.000000,0.000000}%
\pgfsetfillcolor{currentfill}%
\pgfsetlinewidth{0.803000pt}%
\definecolor{currentstroke}{rgb}{0.000000,0.000000,0.000000}%
\pgfsetstrokecolor{currentstroke}%
\pgfsetdash{}{0pt}%
\pgfsys@defobject{currentmarker}{\pgfqpoint{0.000000in}{0.000000in}}{\pgfqpoint{0.048611in}{0.000000in}}{%
\pgfpathmoveto{\pgfqpoint{0.000000in}{0.000000in}}%
\pgfpathlineto{\pgfqpoint{0.048611in}{0.000000in}}%
\pgfusepath{stroke,fill}%
}%
\begin{pgfscope}%
\pgfsys@transformshift{2.193789in}{1.546009in}%
\pgfsys@useobject{currentmarker}{}%
\end{pgfscope}%
\end{pgfscope}%
\begin{pgfscope}%
\pgftext[x=2.291011in,y=1.498182in,left,base]{\rmfamily\fontsize{10.000000}{12.000000}\selectfont \(\displaystyle 20\)}%
\end{pgfscope}%
\begin{pgfscope}%
\pgfsetbuttcap%
\pgfsetroundjoin%
\definecolor{currentfill}{rgb}{0.000000,0.000000,0.000000}%
\pgfsetfillcolor{currentfill}%
\pgfsetlinewidth{0.803000pt}%
\definecolor{currentstroke}{rgb}{0.000000,0.000000,0.000000}%
\pgfsetstrokecolor{currentstroke}%
\pgfsetdash{}{0pt}%
\pgfsys@defobject{currentmarker}{\pgfqpoint{0.000000in}{0.000000in}}{\pgfqpoint{0.048611in}{0.000000in}}{%
\pgfpathmoveto{\pgfqpoint{0.000000in}{0.000000in}}%
\pgfpathlineto{\pgfqpoint{0.048611in}{0.000000in}}%
\pgfusepath{stroke,fill}%
}%
\begin{pgfscope}%
\pgfsys@transformshift{2.193789in}{1.852542in}%
\pgfsys@useobject{currentmarker}{}%
\end{pgfscope}%
\end{pgfscope}%
\begin{pgfscope}%
\pgftext[x=2.291011in,y=1.804715in,left,base]{\rmfamily\fontsize{10.000000}{12.000000}\selectfont \(\displaystyle 25\)}%
\end{pgfscope}%
\begin{pgfscope}%
\pgfsetbuttcap%
\pgfsetroundjoin%
\definecolor{currentfill}{rgb}{0.000000,0.000000,0.000000}%
\pgfsetfillcolor{currentfill}%
\pgfsetlinewidth{0.803000pt}%
\definecolor{currentstroke}{rgb}{0.000000,0.000000,0.000000}%
\pgfsetstrokecolor{currentstroke}%
\pgfsetdash{}{0pt}%
\pgfsys@defobject{currentmarker}{\pgfqpoint{0.000000in}{0.000000in}}{\pgfqpoint{0.048611in}{0.000000in}}{%
\pgfpathmoveto{\pgfqpoint{0.000000in}{0.000000in}}%
\pgfpathlineto{\pgfqpoint{0.048611in}{0.000000in}}%
\pgfusepath{stroke,fill}%
}%
\begin{pgfscope}%
\pgfsys@transformshift{2.193789in}{2.159075in}%
\pgfsys@useobject{currentmarker}{}%
\end{pgfscope}%
\end{pgfscope}%
\begin{pgfscope}%
\pgftext[x=2.291011in,y=2.111248in,left,base]{\rmfamily\fontsize{10.000000}{12.000000}\selectfont \(\displaystyle 30\)}%
\end{pgfscope}%
\begin{pgfscope}%
\pgfsetbuttcap%
\pgfsetroundjoin%
\definecolor{currentfill}{rgb}{0.000000,0.000000,0.000000}%
\pgfsetfillcolor{currentfill}%
\pgfsetlinewidth{0.803000pt}%
\definecolor{currentstroke}{rgb}{0.000000,0.000000,0.000000}%
\pgfsetstrokecolor{currentstroke}%
\pgfsetdash{}{0pt}%
\pgfsys@defobject{currentmarker}{\pgfqpoint{0.000000in}{0.000000in}}{\pgfqpoint{0.048611in}{0.000000in}}{%
\pgfpathmoveto{\pgfqpoint{0.000000in}{0.000000in}}%
\pgfpathlineto{\pgfqpoint{0.048611in}{0.000000in}}%
\pgfusepath{stroke,fill}%
}%
\begin{pgfscope}%
\pgfsys@transformshift{2.193789in}{2.465608in}%
\pgfsys@useobject{currentmarker}{}%
\end{pgfscope}%
\end{pgfscope}%
\begin{pgfscope}%
\pgftext[x=2.291011in,y=2.417781in,left,base]{\rmfamily\fontsize{10.000000}{12.000000}\selectfont \(\displaystyle 35\)}%
\end{pgfscope}%
\begin{pgfscope}%
\pgfsetbuttcap%
\pgfsetroundjoin%
\definecolor{currentfill}{rgb}{0.000000,0.000000,0.000000}%
\pgfsetfillcolor{currentfill}%
\pgfsetlinewidth{0.803000pt}%
\definecolor{currentstroke}{rgb}{0.000000,0.000000,0.000000}%
\pgfsetstrokecolor{currentstroke}%
\pgfsetdash{}{0pt}%
\pgfsys@defobject{currentmarker}{\pgfqpoint{0.000000in}{0.000000in}}{\pgfqpoint{0.048611in}{0.000000in}}{%
\pgfpathmoveto{\pgfqpoint{0.000000in}{0.000000in}}%
\pgfpathlineto{\pgfqpoint{0.048611in}{0.000000in}}%
\pgfusepath{stroke,fill}%
}%
\begin{pgfscope}%
\pgfsys@transformshift{2.193789in}{2.772141in}%
\pgfsys@useobject{currentmarker}{}%
\end{pgfscope}%
\end{pgfscope}%
\begin{pgfscope}%
\pgftext[x=2.291011in,y=2.724314in,left,base]{\rmfamily\fontsize{10.000000}{12.000000}\selectfont \(\displaystyle 40\)}%
\end{pgfscope}%
\begin{pgfscope}%
\pgfsetbuttcap%
\pgfsetmiterjoin%
\pgfsetlinewidth{0.803000pt}%
\definecolor{currentstroke}{rgb}{0.000000,0.000000,0.000000}%
\pgfsetstrokecolor{currentstroke}%
\pgfsetdash{}{0pt}%
\pgfpathmoveto{\pgfqpoint{2.063512in}{0.319877in}}%
\pgfpathlineto{\pgfqpoint{2.063512in}{0.330055in}}%
\pgfpathlineto{\pgfqpoint{2.063512in}{2.915230in}}%
\pgfpathlineto{\pgfqpoint{2.063512in}{2.925408in}}%
\pgfpathlineto{\pgfqpoint{2.193789in}{2.925408in}}%
\pgfpathlineto{\pgfqpoint{2.193789in}{2.915230in}}%
\pgfpathlineto{\pgfqpoint{2.193789in}{0.330055in}}%
\pgfpathlineto{\pgfqpoint{2.193789in}{0.319877in}}%
\pgfpathclose%
\pgfusepath{stroke}%
\end{pgfscope}%
\end{pgfpicture}%
\makeatother%
\endgroup%

	\vspace*{-0.4cm}
	\caption{200 K. Bin size $0.011e$}
	\end{subfigure}
	\quad
	\begin{subfigure}[b]{0.45\textwidth}
	\hspace*{-0.4cm}
	%% Creator: Matplotlib, PGF backend
%%
%% To include the figure in your LaTeX document, write
%%   \input{<filename>.pgf}
%%
%% Make sure the required packages are loaded in your preamble
%%   \usepackage{pgf}
%%
%% Figures using additional raster images can only be included by \input if
%% they are in the same directory as the main LaTeX file. For loading figures
%% from other directories you can use the `import` package
%%   \usepackage{import}
%% and then include the figures with
%%   \import{<path to file>}{<filename>.pgf}
%%
%% Matplotlib used the following preamble
%%   \usepackage[utf8x]{inputenc}
%%   \usepackage[T1]{fontenc}
%%
\begingroup%
\makeatletter%
\begin{pgfpicture}%
\pgfpathrectangle{\pgfpointorigin}{\pgfqpoint{2.529900in}{3.060408in}}%
\pgfusepath{use as bounding box, clip}%
\begin{pgfscope}%
\pgfsetbuttcap%
\pgfsetmiterjoin%
\definecolor{currentfill}{rgb}{1.000000,1.000000,1.000000}%
\pgfsetfillcolor{currentfill}%
\pgfsetlinewidth{0.000000pt}%
\definecolor{currentstroke}{rgb}{1.000000,1.000000,1.000000}%
\pgfsetstrokecolor{currentstroke}%
\pgfsetdash{}{0pt}%
\pgfpathmoveto{\pgfqpoint{0.000000in}{0.000000in}}%
\pgfpathlineto{\pgfqpoint{2.529900in}{0.000000in}}%
\pgfpathlineto{\pgfqpoint{2.529900in}{3.060408in}}%
\pgfpathlineto{\pgfqpoint{0.000000in}{3.060408in}}%
\pgfpathclose%
\pgfusepath{fill}%
\end{pgfscope}%
\begin{pgfscope}%
\pgfsetbuttcap%
\pgfsetmiterjoin%
\definecolor{currentfill}{rgb}{1.000000,1.000000,1.000000}%
\pgfsetfillcolor{currentfill}%
\pgfsetlinewidth{0.000000pt}%
\definecolor{currentstroke}{rgb}{0.000000,0.000000,0.000000}%
\pgfsetstrokecolor{currentstroke}%
\pgfsetstrokeopacity{0.000000}%
\pgfsetdash{}{0pt}%
\pgfpathmoveto{\pgfqpoint{0.444137in}{0.319877in}}%
\pgfpathlineto{\pgfqpoint{1.968255in}{0.319877in}}%
\pgfpathlineto{\pgfqpoint{1.968255in}{2.925408in}}%
\pgfpathlineto{\pgfqpoint{0.444137in}{2.925408in}}%
\pgfpathclose%
\pgfusepath{fill}%
\end{pgfscope}%
\begin{pgfscope}%
\pgfpathrectangle{\pgfqpoint{0.444137in}{0.319877in}}{\pgfqpoint{1.524118in}{2.605531in}} %
\pgfusepath{clip}%
\pgfsys@transformshift{0.444137in}{0.319877in}%
\pgftext[left,bottom]{\pgfimage[interpolate=true,width=1.530000in,height=2.610000in]{FerrNN_vs_dq_Ti_300K-img0.png}}%
\end{pgfscope}%
\begin{pgfscope}%
\pgfpathrectangle{\pgfqpoint{0.444137in}{0.319877in}}{\pgfqpoint{1.524118in}{2.605531in}} %
\pgfusepath{clip}%
\pgfsetbuttcap%
\pgfsetroundjoin%
\definecolor{currentfill}{rgb}{1.000000,0.752941,0.796078}%
\pgfsetfillcolor{currentfill}%
\pgfsetlinewidth{1.003750pt}%
\definecolor{currentstroke}{rgb}{1.000000,0.752941,0.796078}%
\pgfsetstrokecolor{currentstroke}%
\pgfsetdash{}{0pt}%
\pgfpathmoveto{\pgfqpoint{0.790527in}{2.635595in}}%
\pgfpathcurveto{\pgfqpoint{0.801577in}{2.635595in}}{\pgfqpoint{0.812176in}{2.639986in}}{\pgfqpoint{0.819990in}{2.647799in}}%
\pgfpathcurveto{\pgfqpoint{0.827803in}{2.655613in}}{\pgfqpoint{0.832194in}{2.666212in}}{\pgfqpoint{0.832194in}{2.677262in}}%
\pgfpathcurveto{\pgfqpoint{0.832194in}{2.688312in}}{\pgfqpoint{0.827803in}{2.698911in}}{\pgfqpoint{0.819990in}{2.706725in}}%
\pgfpathcurveto{\pgfqpoint{0.812176in}{2.714539in}}{\pgfqpoint{0.801577in}{2.718929in}}{\pgfqpoint{0.790527in}{2.718929in}}%
\pgfpathcurveto{\pgfqpoint{0.779477in}{2.718929in}}{\pgfqpoint{0.768878in}{2.714539in}}{\pgfqpoint{0.761064in}{2.706725in}}%
\pgfpathcurveto{\pgfqpoint{0.753251in}{2.698911in}}{\pgfqpoint{0.748860in}{2.688312in}}{\pgfqpoint{0.748860in}{2.677262in}}%
\pgfpathcurveto{\pgfqpoint{0.748860in}{2.666212in}}{\pgfqpoint{0.753251in}{2.655613in}}{\pgfqpoint{0.761064in}{2.647799in}}%
\pgfpathcurveto{\pgfqpoint{0.768878in}{2.639986in}}{\pgfqpoint{0.779477in}{2.635595in}}{\pgfqpoint{0.790527in}{2.635595in}}%
\pgfpathclose%
\pgfusepath{stroke,fill}%
\end{pgfscope}%
\begin{pgfscope}%
\pgfpathrectangle{\pgfqpoint{0.444137in}{0.319877in}}{\pgfqpoint{1.524118in}{2.605531in}} %
\pgfusepath{clip}%
\pgfsetbuttcap%
\pgfsetroundjoin%
\definecolor{currentfill}{rgb}{1.000000,0.752941,0.796078}%
\pgfsetfillcolor{currentfill}%
\pgfsetlinewidth{1.003750pt}%
\definecolor{currentstroke}{rgb}{1.000000,0.752941,0.796078}%
\pgfsetstrokecolor{currentstroke}%
\pgfsetdash{}{0pt}%
\pgfpathmoveto{\pgfqpoint{0.929083in}{1.725728in}}%
\pgfpathcurveto{\pgfqpoint{0.940133in}{1.725728in}}{\pgfqpoint{0.950732in}{1.730118in}}{\pgfqpoint{0.958546in}{1.737932in}}%
\pgfpathcurveto{\pgfqpoint{0.966360in}{1.745745in}}{\pgfqpoint{0.970750in}{1.756344in}}{\pgfqpoint{0.970750in}{1.767394in}}%
\pgfpathcurveto{\pgfqpoint{0.970750in}{1.778444in}}{\pgfqpoint{0.966360in}{1.789043in}}{\pgfqpoint{0.958546in}{1.796857in}}%
\pgfpathcurveto{\pgfqpoint{0.950732in}{1.804671in}}{\pgfqpoint{0.940133in}{1.809061in}}{\pgfqpoint{0.929083in}{1.809061in}}%
\pgfpathcurveto{\pgfqpoint{0.918033in}{1.809061in}}{\pgfqpoint{0.907434in}{1.804671in}}{\pgfqpoint{0.899620in}{1.796857in}}%
\pgfpathcurveto{\pgfqpoint{0.891807in}{1.789043in}}{\pgfqpoint{0.887417in}{1.778444in}}{\pgfqpoint{0.887417in}{1.767394in}}%
\pgfpathcurveto{\pgfqpoint{0.887417in}{1.756344in}}{\pgfqpoint{0.891807in}{1.745745in}}{\pgfqpoint{0.899620in}{1.737932in}}%
\pgfpathcurveto{\pgfqpoint{0.907434in}{1.730118in}}{\pgfqpoint{0.918033in}{1.725728in}}{\pgfqpoint{0.929083in}{1.725728in}}%
\pgfpathclose%
\pgfusepath{stroke,fill}%
\end{pgfscope}%
\begin{pgfscope}%
\pgfpathrectangle{\pgfqpoint{0.444137in}{0.319877in}}{\pgfqpoint{1.524118in}{2.605531in}} %
\pgfusepath{clip}%
\pgfsetbuttcap%
\pgfsetroundjoin%
\definecolor{currentfill}{rgb}{1.000000,0.752941,0.796078}%
\pgfsetfillcolor{currentfill}%
\pgfsetlinewidth{1.003750pt}%
\definecolor{currentstroke}{rgb}{1.000000,0.752941,0.796078}%
\pgfsetstrokecolor{currentstroke}%
\pgfsetdash{}{0pt}%
\pgfpathmoveto{\pgfqpoint{1.067639in}{1.088687in}}%
\pgfpathcurveto{\pgfqpoint{1.078690in}{1.088687in}}{\pgfqpoint{1.089289in}{1.093077in}}{\pgfqpoint{1.097102in}{1.100891in}}%
\pgfpathcurveto{\pgfqpoint{1.104916in}{1.108704in}}{\pgfqpoint{1.109306in}{1.119303in}}{\pgfqpoint{1.109306in}{1.130353in}}%
\pgfpathcurveto{\pgfqpoint{1.109306in}{1.141404in}}{\pgfqpoint{1.104916in}{1.152003in}}{\pgfqpoint{1.097102in}{1.159816in}}%
\pgfpathcurveto{\pgfqpoint{1.089289in}{1.167630in}}{\pgfqpoint{1.078690in}{1.172020in}}{\pgfqpoint{1.067639in}{1.172020in}}%
\pgfpathcurveto{\pgfqpoint{1.056589in}{1.172020in}}{\pgfqpoint{1.045990in}{1.167630in}}{\pgfqpoint{1.038177in}{1.159816in}}%
\pgfpathcurveto{\pgfqpoint{1.030363in}{1.152003in}}{\pgfqpoint{1.025973in}{1.141404in}}{\pgfqpoint{1.025973in}{1.130353in}}%
\pgfpathcurveto{\pgfqpoint{1.025973in}{1.119303in}}{\pgfqpoint{1.030363in}{1.108704in}}{\pgfqpoint{1.038177in}{1.100891in}}%
\pgfpathcurveto{\pgfqpoint{1.045990in}{1.093077in}}{\pgfqpoint{1.056589in}{1.088687in}}{\pgfqpoint{1.067639in}{1.088687in}}%
\pgfpathclose%
\pgfusepath{stroke,fill}%
\end{pgfscope}%
\begin{pgfscope}%
\pgfpathrectangle{\pgfqpoint{0.444137in}{0.319877in}}{\pgfqpoint{1.524118in}{2.605531in}} %
\pgfusepath{clip}%
\pgfsetbuttcap%
\pgfsetroundjoin%
\definecolor{currentfill}{rgb}{1.000000,0.752941,0.796078}%
\pgfsetfillcolor{currentfill}%
\pgfsetlinewidth{1.003750pt}%
\definecolor{currentstroke}{rgb}{1.000000,0.752941,0.796078}%
\pgfsetstrokecolor{currentstroke}%
\pgfsetdash{}{0pt}%
\pgfpathmoveto{\pgfqpoint{1.206196in}{0.923185in}}%
\pgfpathcurveto{\pgfqpoint{1.217246in}{0.923185in}}{\pgfqpoint{1.227845in}{0.927575in}}{\pgfqpoint{1.235658in}{0.935388in}}%
\pgfpathcurveto{\pgfqpoint{1.243472in}{0.943202in}}{\pgfqpoint{1.247862in}{0.953801in}}{\pgfqpoint{1.247862in}{0.964851in}}%
\pgfpathcurveto{\pgfqpoint{1.247862in}{0.975901in}}{\pgfqpoint{1.243472in}{0.986500in}}{\pgfqpoint{1.235658in}{0.994314in}}%
\pgfpathcurveto{\pgfqpoint{1.227845in}{1.002128in}}{\pgfqpoint{1.217246in}{1.006518in}}{\pgfqpoint{1.206196in}{1.006518in}}%
\pgfpathcurveto{\pgfqpoint{1.195145in}{1.006518in}}{\pgfqpoint{1.184546in}{1.002128in}}{\pgfqpoint{1.176733in}{0.994314in}}%
\pgfpathcurveto{\pgfqpoint{1.168919in}{0.986500in}}{\pgfqpoint{1.164529in}{0.975901in}}{\pgfqpoint{1.164529in}{0.964851in}}%
\pgfpathcurveto{\pgfqpoint{1.164529in}{0.953801in}}{\pgfqpoint{1.168919in}{0.943202in}}{\pgfqpoint{1.176733in}{0.935388in}}%
\pgfpathcurveto{\pgfqpoint{1.184546in}{0.927575in}}{\pgfqpoint{1.195145in}{0.923185in}}{\pgfqpoint{1.206196in}{0.923185in}}%
\pgfpathclose%
\pgfusepath{stroke,fill}%
\end{pgfscope}%
\begin{pgfscope}%
\pgfpathrectangle{\pgfqpoint{0.444137in}{0.319877in}}{\pgfqpoint{1.524118in}{2.605531in}} %
\pgfusepath{clip}%
\pgfsetbuttcap%
\pgfsetroundjoin%
\definecolor{currentfill}{rgb}{1.000000,0.752941,0.796078}%
\pgfsetfillcolor{currentfill}%
\pgfsetlinewidth{1.003750pt}%
\definecolor{currentstroke}{rgb}{1.000000,0.752941,0.796078}%
\pgfsetstrokecolor{currentstroke}%
\pgfsetdash{}{0pt}%
\pgfpathmoveto{\pgfqpoint{1.344752in}{1.177739in}}%
\pgfpathcurveto{\pgfqpoint{1.355802in}{1.177739in}}{\pgfqpoint{1.366401in}{1.182129in}}{\pgfqpoint{1.374215in}{1.189943in}}%
\pgfpathcurveto{\pgfqpoint{1.382028in}{1.197757in}}{\pgfqpoint{1.386418in}{1.208356in}}{\pgfqpoint{1.386418in}{1.219406in}}%
\pgfpathcurveto{\pgfqpoint{1.386418in}{1.230456in}}{\pgfqpoint{1.382028in}{1.241055in}}{\pgfqpoint{1.374215in}{1.248869in}}%
\pgfpathcurveto{\pgfqpoint{1.366401in}{1.256682in}}{\pgfqpoint{1.355802in}{1.261072in}}{\pgfqpoint{1.344752in}{1.261072in}}%
\pgfpathcurveto{\pgfqpoint{1.333702in}{1.261072in}}{\pgfqpoint{1.323103in}{1.256682in}}{\pgfqpoint{1.315289in}{1.248869in}}%
\pgfpathcurveto{\pgfqpoint{1.307475in}{1.241055in}}{\pgfqpoint{1.303085in}{1.230456in}}{\pgfqpoint{1.303085in}{1.219406in}}%
\pgfpathcurveto{\pgfqpoint{1.303085in}{1.208356in}}{\pgfqpoint{1.307475in}{1.197757in}}{\pgfqpoint{1.315289in}{1.189943in}}%
\pgfpathcurveto{\pgfqpoint{1.323103in}{1.182129in}}{\pgfqpoint{1.333702in}{1.177739in}}{\pgfqpoint{1.344752in}{1.177739in}}%
\pgfpathclose%
\pgfusepath{stroke,fill}%
\end{pgfscope}%
\begin{pgfscope}%
\pgfpathrectangle{\pgfqpoint{0.444137in}{0.319877in}}{\pgfqpoint{1.524118in}{2.605531in}} %
\pgfusepath{clip}%
\pgfsetbuttcap%
\pgfsetroundjoin%
\definecolor{currentfill}{rgb}{1.000000,0.752941,0.796078}%
\pgfsetfillcolor{currentfill}%
\pgfsetlinewidth{1.003750pt}%
\definecolor{currentstroke}{rgb}{1.000000,0.752941,0.796078}%
\pgfsetstrokecolor{currentstroke}%
\pgfsetdash{}{0pt}%
\pgfpathmoveto{\pgfqpoint{1.483308in}{1.651875in}}%
\pgfpathcurveto{\pgfqpoint{1.494358in}{1.651875in}}{\pgfqpoint{1.504957in}{1.656265in}}{\pgfqpoint{1.512771in}{1.664079in}}%
\pgfpathcurveto{\pgfqpoint{1.520584in}{1.671892in}}{\pgfqpoint{1.524975in}{1.682491in}}{\pgfqpoint{1.524975in}{1.693541in}}%
\pgfpathcurveto{\pgfqpoint{1.524975in}{1.704592in}}{\pgfqpoint{1.520584in}{1.715191in}}{\pgfqpoint{1.512771in}{1.723004in}}%
\pgfpathcurveto{\pgfqpoint{1.504957in}{1.730818in}}{\pgfqpoint{1.494358in}{1.735208in}}{\pgfqpoint{1.483308in}{1.735208in}}%
\pgfpathcurveto{\pgfqpoint{1.472258in}{1.735208in}}{\pgfqpoint{1.461659in}{1.730818in}}{\pgfqpoint{1.453845in}{1.723004in}}%
\pgfpathcurveto{\pgfqpoint{1.446032in}{1.715191in}}{\pgfqpoint{1.441641in}{1.704592in}}{\pgfqpoint{1.441641in}{1.693541in}}%
\pgfpathcurveto{\pgfqpoint{1.441641in}{1.682491in}}{\pgfqpoint{1.446032in}{1.671892in}}{\pgfqpoint{1.453845in}{1.664079in}}%
\pgfpathcurveto{\pgfqpoint{1.461659in}{1.656265in}}{\pgfqpoint{1.472258in}{1.651875in}}{\pgfqpoint{1.483308in}{1.651875in}}%
\pgfpathclose%
\pgfusepath{stroke,fill}%
\end{pgfscope}%
\begin{pgfscope}%
\pgfsetbuttcap%
\pgfsetroundjoin%
\definecolor{currentfill}{rgb}{0.000000,0.000000,0.000000}%
\pgfsetfillcolor{currentfill}%
\pgfsetlinewidth{0.803000pt}%
\definecolor{currentstroke}{rgb}{0.000000,0.000000,0.000000}%
\pgfsetstrokecolor{currentstroke}%
\pgfsetdash{}{0pt}%
\pgfsys@defobject{currentmarker}{\pgfqpoint{0.000000in}{-0.048611in}}{\pgfqpoint{0.000000in}{0.000000in}}{%
\pgfpathmoveto{\pgfqpoint{0.000000in}{0.000000in}}%
\pgfpathlineto{\pgfqpoint{0.000000in}{-0.048611in}}%
\pgfusepath{stroke,fill}%
}%
\begin{pgfscope}%
\pgfsys@transformshift{0.729909in}{0.319877in}%
\pgfsys@useobject{currentmarker}{}%
\end{pgfscope}%
\end{pgfscope}%
\begin{pgfscope}%
\pgftext[x=0.729909in,y=0.222655in,,top]{\rmfamily\fontsize{10.000000}{12.000000}\selectfont \(\displaystyle -0.05\)}%
\end{pgfscope}%
\begin{pgfscope}%
\pgfsetbuttcap%
\pgfsetroundjoin%
\definecolor{currentfill}{rgb}{0.000000,0.000000,0.000000}%
\pgfsetfillcolor{currentfill}%
\pgfsetlinewidth{0.803000pt}%
\definecolor{currentstroke}{rgb}{0.000000,0.000000,0.000000}%
\pgfsetstrokecolor{currentstroke}%
\pgfsetdash{}{0pt}%
\pgfsys@defobject{currentmarker}{\pgfqpoint{0.000000in}{-0.048611in}}{\pgfqpoint{0.000000in}{0.000000in}}{%
\pgfpathmoveto{\pgfqpoint{0.000000in}{0.000000in}}%
\pgfpathlineto{\pgfqpoint{0.000000in}{-0.048611in}}%
\pgfusepath{stroke,fill}%
}%
\begin{pgfscope}%
\pgfsys@transformshift{1.206196in}{0.319877in}%
\pgfsys@useobject{currentmarker}{}%
\end{pgfscope}%
\end{pgfscope}%
\begin{pgfscope}%
\pgftext[x=1.206196in,y=0.222655in,,top]{\rmfamily\fontsize{10.000000}{12.000000}\selectfont \(\displaystyle 0.00\)}%
\end{pgfscope}%
\begin{pgfscope}%
\pgfsetbuttcap%
\pgfsetroundjoin%
\definecolor{currentfill}{rgb}{0.000000,0.000000,0.000000}%
\pgfsetfillcolor{currentfill}%
\pgfsetlinewidth{0.803000pt}%
\definecolor{currentstroke}{rgb}{0.000000,0.000000,0.000000}%
\pgfsetstrokecolor{currentstroke}%
\pgfsetdash{}{0pt}%
\pgfsys@defobject{currentmarker}{\pgfqpoint{0.000000in}{-0.048611in}}{\pgfqpoint{0.000000in}{0.000000in}}{%
\pgfpathmoveto{\pgfqpoint{0.000000in}{0.000000in}}%
\pgfpathlineto{\pgfqpoint{0.000000in}{-0.048611in}}%
\pgfusepath{stroke,fill}%
}%
\begin{pgfscope}%
\pgfsys@transformshift{1.682483in}{0.319877in}%
\pgfsys@useobject{currentmarker}{}%
\end{pgfscope}%
\end{pgfscope}%
\begin{pgfscope}%
\pgftext[x=1.682483in,y=0.222655in,,top]{\rmfamily\fontsize{10.000000}{12.000000}\selectfont \(\displaystyle 0.05\)}%
\end{pgfscope}%
\begin{pgfscope}%
\pgfsetbuttcap%
\pgfsetroundjoin%
\definecolor{currentfill}{rgb}{0.000000,0.000000,0.000000}%
\pgfsetfillcolor{currentfill}%
\pgfsetlinewidth{0.803000pt}%
\definecolor{currentstroke}{rgb}{0.000000,0.000000,0.000000}%
\pgfsetstrokecolor{currentstroke}%
\pgfsetdash{}{0pt}%
\pgfsys@defobject{currentmarker}{\pgfqpoint{-0.048611in}{0.000000in}}{\pgfqpoint{0.000000in}{0.000000in}}{%
\pgfpathmoveto{\pgfqpoint{0.000000in}{0.000000in}}%
\pgfpathlineto{\pgfqpoint{-0.048611in}{0.000000in}}%
\pgfusepath{stroke,fill}%
}%
\begin{pgfscope}%
\pgfsys@transformshift{0.444137in}{0.319877in}%
\pgfsys@useobject{currentmarker}{}%
\end{pgfscope}%
\end{pgfscope}%
\begin{pgfscope}%
\pgftext[x=0.100000in,y=0.272050in,left,base]{\rmfamily\fontsize{10.000000}{12.000000}\selectfont \(\displaystyle 0.00\)}%
\end{pgfscope}%
\begin{pgfscope}%
\pgfsetbuttcap%
\pgfsetroundjoin%
\definecolor{currentfill}{rgb}{0.000000,0.000000,0.000000}%
\pgfsetfillcolor{currentfill}%
\pgfsetlinewidth{0.803000pt}%
\definecolor{currentstroke}{rgb}{0.000000,0.000000,0.000000}%
\pgfsetstrokecolor{currentstroke}%
\pgfsetdash{}{0pt}%
\pgfsys@defobject{currentmarker}{\pgfqpoint{-0.048611in}{0.000000in}}{\pgfqpoint{0.000000in}{0.000000in}}{%
\pgfpathmoveto{\pgfqpoint{0.000000in}{0.000000in}}%
\pgfpathlineto{\pgfqpoint{-0.048611in}{0.000000in}}%
\pgfusepath{stroke,fill}%
}%
\begin{pgfscope}%
\pgfsys@transformshift{0.444137in}{0.776472in}%
\pgfsys@useobject{currentmarker}{}%
\end{pgfscope}%
\end{pgfscope}%
\begin{pgfscope}%
\pgftext[x=0.100000in,y=0.728645in,left,base]{\rmfamily\fontsize{10.000000}{12.000000}\selectfont \(\displaystyle 0.05\)}%
\end{pgfscope}%
\begin{pgfscope}%
\pgfsetbuttcap%
\pgfsetroundjoin%
\definecolor{currentfill}{rgb}{0.000000,0.000000,0.000000}%
\pgfsetfillcolor{currentfill}%
\pgfsetlinewidth{0.803000pt}%
\definecolor{currentstroke}{rgb}{0.000000,0.000000,0.000000}%
\pgfsetstrokecolor{currentstroke}%
\pgfsetdash{}{0pt}%
\pgfsys@defobject{currentmarker}{\pgfqpoint{-0.048611in}{0.000000in}}{\pgfqpoint{0.000000in}{0.000000in}}{%
\pgfpathmoveto{\pgfqpoint{0.000000in}{0.000000in}}%
\pgfpathlineto{\pgfqpoint{-0.048611in}{0.000000in}}%
\pgfusepath{stroke,fill}%
}%
\begin{pgfscope}%
\pgfsys@transformshift{0.444137in}{1.233067in}%
\pgfsys@useobject{currentmarker}{}%
\end{pgfscope}%
\end{pgfscope}%
\begin{pgfscope}%
\pgftext[x=0.100000in,y=1.185240in,left,base]{\rmfamily\fontsize{10.000000}{12.000000}\selectfont \(\displaystyle 0.10\)}%
\end{pgfscope}%
\begin{pgfscope}%
\pgfsetbuttcap%
\pgfsetroundjoin%
\definecolor{currentfill}{rgb}{0.000000,0.000000,0.000000}%
\pgfsetfillcolor{currentfill}%
\pgfsetlinewidth{0.803000pt}%
\definecolor{currentstroke}{rgb}{0.000000,0.000000,0.000000}%
\pgfsetstrokecolor{currentstroke}%
\pgfsetdash{}{0pt}%
\pgfsys@defobject{currentmarker}{\pgfqpoint{-0.048611in}{0.000000in}}{\pgfqpoint{0.000000in}{0.000000in}}{%
\pgfpathmoveto{\pgfqpoint{0.000000in}{0.000000in}}%
\pgfpathlineto{\pgfqpoint{-0.048611in}{0.000000in}}%
\pgfusepath{stroke,fill}%
}%
\begin{pgfscope}%
\pgfsys@transformshift{0.444137in}{1.689662in}%
\pgfsys@useobject{currentmarker}{}%
\end{pgfscope}%
\end{pgfscope}%
\begin{pgfscope}%
\pgftext[x=0.100000in,y=1.641835in,left,base]{\rmfamily\fontsize{10.000000}{12.000000}\selectfont \(\displaystyle 0.15\)}%
\end{pgfscope}%
\begin{pgfscope}%
\pgfsetbuttcap%
\pgfsetroundjoin%
\definecolor{currentfill}{rgb}{0.000000,0.000000,0.000000}%
\pgfsetfillcolor{currentfill}%
\pgfsetlinewidth{0.803000pt}%
\definecolor{currentstroke}{rgb}{0.000000,0.000000,0.000000}%
\pgfsetstrokecolor{currentstroke}%
\pgfsetdash{}{0pt}%
\pgfsys@defobject{currentmarker}{\pgfqpoint{-0.048611in}{0.000000in}}{\pgfqpoint{0.000000in}{0.000000in}}{%
\pgfpathmoveto{\pgfqpoint{0.000000in}{0.000000in}}%
\pgfpathlineto{\pgfqpoint{-0.048611in}{0.000000in}}%
\pgfusepath{stroke,fill}%
}%
\begin{pgfscope}%
\pgfsys@transformshift{0.444137in}{2.146257in}%
\pgfsys@useobject{currentmarker}{}%
\end{pgfscope}%
\end{pgfscope}%
\begin{pgfscope}%
\pgftext[x=0.100000in,y=2.098430in,left,base]{\rmfamily\fontsize{10.000000}{12.000000}\selectfont \(\displaystyle 0.20\)}%
\end{pgfscope}%
\begin{pgfscope}%
\pgfsetbuttcap%
\pgfsetroundjoin%
\definecolor{currentfill}{rgb}{0.000000,0.000000,0.000000}%
\pgfsetfillcolor{currentfill}%
\pgfsetlinewidth{0.803000pt}%
\definecolor{currentstroke}{rgb}{0.000000,0.000000,0.000000}%
\pgfsetstrokecolor{currentstroke}%
\pgfsetdash{}{0pt}%
\pgfsys@defobject{currentmarker}{\pgfqpoint{-0.048611in}{0.000000in}}{\pgfqpoint{0.000000in}{0.000000in}}{%
\pgfpathmoveto{\pgfqpoint{0.000000in}{0.000000in}}%
\pgfpathlineto{\pgfqpoint{-0.048611in}{0.000000in}}%
\pgfusepath{stroke,fill}%
}%
\begin{pgfscope}%
\pgfsys@transformshift{0.444137in}{2.602852in}%
\pgfsys@useobject{currentmarker}{}%
\end{pgfscope}%
\end{pgfscope}%
\begin{pgfscope}%
\pgftext[x=0.100000in,y=2.555025in,left,base]{\rmfamily\fontsize{10.000000}{12.000000}\selectfont \(\displaystyle 0.25\)}%
\end{pgfscope}%
\begin{pgfscope}%
\pgfsetrectcap%
\pgfsetmiterjoin%
\pgfsetlinewidth{0.803000pt}%
\definecolor{currentstroke}{rgb}{0.000000,0.000000,0.000000}%
\pgfsetstrokecolor{currentstroke}%
\pgfsetdash{}{0pt}%
\pgfpathmoveto{\pgfqpoint{0.444137in}{0.319877in}}%
\pgfpathlineto{\pgfqpoint{0.444137in}{2.925408in}}%
\pgfusepath{stroke}%
\end{pgfscope}%
\begin{pgfscope}%
\pgfsetrectcap%
\pgfsetmiterjoin%
\pgfsetlinewidth{0.803000pt}%
\definecolor{currentstroke}{rgb}{0.000000,0.000000,0.000000}%
\pgfsetstrokecolor{currentstroke}%
\pgfsetdash{}{0pt}%
\pgfpathmoveto{\pgfqpoint{1.968255in}{0.319877in}}%
\pgfpathlineto{\pgfqpoint{1.968255in}{2.925408in}}%
\pgfusepath{stroke}%
\end{pgfscope}%
\begin{pgfscope}%
\pgfsetrectcap%
\pgfsetmiterjoin%
\pgfsetlinewidth{0.803000pt}%
\definecolor{currentstroke}{rgb}{0.000000,0.000000,0.000000}%
\pgfsetstrokecolor{currentstroke}%
\pgfsetdash{}{0pt}%
\pgfpathmoveto{\pgfqpoint{0.444137in}{0.319877in}}%
\pgfpathlineto{\pgfqpoint{1.968255in}{0.319877in}}%
\pgfusepath{stroke}%
\end{pgfscope}%
\begin{pgfscope}%
\pgfsetrectcap%
\pgfsetmiterjoin%
\pgfsetlinewidth{0.803000pt}%
\definecolor{currentstroke}{rgb}{0.000000,0.000000,0.000000}%
\pgfsetstrokecolor{currentstroke}%
\pgfsetdash{}{0pt}%
\pgfpathmoveto{\pgfqpoint{0.444137in}{2.925408in}}%
\pgfpathlineto{\pgfqpoint{1.968255in}{2.925408in}}%
\pgfusepath{stroke}%
\end{pgfscope}%
\begin{pgfscope}%
\pgfpathrectangle{\pgfqpoint{2.063512in}{0.319877in}}{\pgfqpoint{0.130277in}{2.605531in}} %
\pgfusepath{clip}%
\pgfsetbuttcap%
\pgfsetmiterjoin%
\definecolor{currentfill}{rgb}{1.000000,1.000000,1.000000}%
\pgfsetfillcolor{currentfill}%
\pgfsetlinewidth{0.010037pt}%
\definecolor{currentstroke}{rgb}{1.000000,1.000000,1.000000}%
\pgfsetstrokecolor{currentstroke}%
\pgfsetdash{}{0pt}%
\pgfpathmoveto{\pgfqpoint{2.063512in}{0.319877in}}%
\pgfpathlineto{\pgfqpoint{2.063512in}{0.330055in}}%
\pgfpathlineto{\pgfqpoint{2.063512in}{2.915230in}}%
\pgfpathlineto{\pgfqpoint{2.063512in}{2.925408in}}%
\pgfpathlineto{\pgfqpoint{2.193789in}{2.925408in}}%
\pgfpathlineto{\pgfqpoint{2.193789in}{2.915230in}}%
\pgfpathlineto{\pgfqpoint{2.193789in}{0.330055in}}%
\pgfpathlineto{\pgfqpoint{2.193789in}{0.319877in}}%
\pgfpathclose%
\pgfusepath{stroke,fill}%
\end{pgfscope}%
\begin{pgfscope}%
\pgfsys@transformshift{2.060000in}{0.320408in}%
\pgftext[left,bottom]{\pgfimage[interpolate=true,width=0.130000in,height=2.610000in]{FerrNN_vs_dq_Ti_300K-img1.png}}%
\end{pgfscope}%
\begin{pgfscope}%
\pgfsetbuttcap%
\pgfsetroundjoin%
\definecolor{currentfill}{rgb}{0.000000,0.000000,0.000000}%
\pgfsetfillcolor{currentfill}%
\pgfsetlinewidth{0.803000pt}%
\definecolor{currentstroke}{rgb}{0.000000,0.000000,0.000000}%
\pgfsetstrokecolor{currentstroke}%
\pgfsetdash{}{0pt}%
\pgfsys@defobject{currentmarker}{\pgfqpoint{0.000000in}{0.000000in}}{\pgfqpoint{0.048611in}{0.000000in}}{%
\pgfpathmoveto{\pgfqpoint{0.000000in}{0.000000in}}%
\pgfpathlineto{\pgfqpoint{0.048611in}{0.000000in}}%
\pgfusepath{stroke,fill}%
}%
\begin{pgfscope}%
\pgfsys@transformshift{2.193789in}{0.319877in}%
\pgfsys@useobject{currentmarker}{}%
\end{pgfscope}%
\end{pgfscope}%
\begin{pgfscope}%
\pgftext[x=2.291011in,y=0.272050in,left,base]{\rmfamily\fontsize{10.000000}{12.000000}\selectfont \(\displaystyle 0\)}%
\end{pgfscope}%
\begin{pgfscope}%
\pgfsetbuttcap%
\pgfsetroundjoin%
\definecolor{currentfill}{rgb}{0.000000,0.000000,0.000000}%
\pgfsetfillcolor{currentfill}%
\pgfsetlinewidth{0.803000pt}%
\definecolor{currentstroke}{rgb}{0.000000,0.000000,0.000000}%
\pgfsetstrokecolor{currentstroke}%
\pgfsetdash{}{0pt}%
\pgfsys@defobject{currentmarker}{\pgfqpoint{0.000000in}{0.000000in}}{\pgfqpoint{0.048611in}{0.000000in}}{%
\pgfpathmoveto{\pgfqpoint{0.000000in}{0.000000in}}%
\pgfpathlineto{\pgfqpoint{0.048611in}{0.000000in}}%
\pgfusepath{stroke,fill}%
}%
\begin{pgfscope}%
\pgfsys@transformshift{2.193789in}{0.626410in}%
\pgfsys@useobject{currentmarker}{}%
\end{pgfscope}%
\end{pgfscope}%
\begin{pgfscope}%
\pgftext[x=2.291011in,y=0.578583in,left,base]{\rmfamily\fontsize{10.000000}{12.000000}\selectfont \(\displaystyle 5\)}%
\end{pgfscope}%
\begin{pgfscope}%
\pgfsetbuttcap%
\pgfsetroundjoin%
\definecolor{currentfill}{rgb}{0.000000,0.000000,0.000000}%
\pgfsetfillcolor{currentfill}%
\pgfsetlinewidth{0.803000pt}%
\definecolor{currentstroke}{rgb}{0.000000,0.000000,0.000000}%
\pgfsetstrokecolor{currentstroke}%
\pgfsetdash{}{0pt}%
\pgfsys@defobject{currentmarker}{\pgfqpoint{0.000000in}{0.000000in}}{\pgfqpoint{0.048611in}{0.000000in}}{%
\pgfpathmoveto{\pgfqpoint{0.000000in}{0.000000in}}%
\pgfpathlineto{\pgfqpoint{0.048611in}{0.000000in}}%
\pgfusepath{stroke,fill}%
}%
\begin{pgfscope}%
\pgfsys@transformshift{2.193789in}{0.932943in}%
\pgfsys@useobject{currentmarker}{}%
\end{pgfscope}%
\end{pgfscope}%
\begin{pgfscope}%
\pgftext[x=2.291011in,y=0.885116in,left,base]{\rmfamily\fontsize{10.000000}{12.000000}\selectfont \(\displaystyle 10\)}%
\end{pgfscope}%
\begin{pgfscope}%
\pgfsetbuttcap%
\pgfsetroundjoin%
\definecolor{currentfill}{rgb}{0.000000,0.000000,0.000000}%
\pgfsetfillcolor{currentfill}%
\pgfsetlinewidth{0.803000pt}%
\definecolor{currentstroke}{rgb}{0.000000,0.000000,0.000000}%
\pgfsetstrokecolor{currentstroke}%
\pgfsetdash{}{0pt}%
\pgfsys@defobject{currentmarker}{\pgfqpoint{0.000000in}{0.000000in}}{\pgfqpoint{0.048611in}{0.000000in}}{%
\pgfpathmoveto{\pgfqpoint{0.000000in}{0.000000in}}%
\pgfpathlineto{\pgfqpoint{0.048611in}{0.000000in}}%
\pgfusepath{stroke,fill}%
}%
\begin{pgfscope}%
\pgfsys@transformshift{2.193789in}{1.239476in}%
\pgfsys@useobject{currentmarker}{}%
\end{pgfscope}%
\end{pgfscope}%
\begin{pgfscope}%
\pgftext[x=2.291011in,y=1.191649in,left,base]{\rmfamily\fontsize{10.000000}{12.000000}\selectfont \(\displaystyle 15\)}%
\end{pgfscope}%
\begin{pgfscope}%
\pgfsetbuttcap%
\pgfsetroundjoin%
\definecolor{currentfill}{rgb}{0.000000,0.000000,0.000000}%
\pgfsetfillcolor{currentfill}%
\pgfsetlinewidth{0.803000pt}%
\definecolor{currentstroke}{rgb}{0.000000,0.000000,0.000000}%
\pgfsetstrokecolor{currentstroke}%
\pgfsetdash{}{0pt}%
\pgfsys@defobject{currentmarker}{\pgfqpoint{0.000000in}{0.000000in}}{\pgfqpoint{0.048611in}{0.000000in}}{%
\pgfpathmoveto{\pgfqpoint{0.000000in}{0.000000in}}%
\pgfpathlineto{\pgfqpoint{0.048611in}{0.000000in}}%
\pgfusepath{stroke,fill}%
}%
\begin{pgfscope}%
\pgfsys@transformshift{2.193789in}{1.546009in}%
\pgfsys@useobject{currentmarker}{}%
\end{pgfscope}%
\end{pgfscope}%
\begin{pgfscope}%
\pgftext[x=2.291011in,y=1.498182in,left,base]{\rmfamily\fontsize{10.000000}{12.000000}\selectfont \(\displaystyle 20\)}%
\end{pgfscope}%
\begin{pgfscope}%
\pgfsetbuttcap%
\pgfsetroundjoin%
\definecolor{currentfill}{rgb}{0.000000,0.000000,0.000000}%
\pgfsetfillcolor{currentfill}%
\pgfsetlinewidth{0.803000pt}%
\definecolor{currentstroke}{rgb}{0.000000,0.000000,0.000000}%
\pgfsetstrokecolor{currentstroke}%
\pgfsetdash{}{0pt}%
\pgfsys@defobject{currentmarker}{\pgfqpoint{0.000000in}{0.000000in}}{\pgfqpoint{0.048611in}{0.000000in}}{%
\pgfpathmoveto{\pgfqpoint{0.000000in}{0.000000in}}%
\pgfpathlineto{\pgfqpoint{0.048611in}{0.000000in}}%
\pgfusepath{stroke,fill}%
}%
\begin{pgfscope}%
\pgfsys@transformshift{2.193789in}{1.852542in}%
\pgfsys@useobject{currentmarker}{}%
\end{pgfscope}%
\end{pgfscope}%
\begin{pgfscope}%
\pgftext[x=2.291011in,y=1.804715in,left,base]{\rmfamily\fontsize{10.000000}{12.000000}\selectfont \(\displaystyle 25\)}%
\end{pgfscope}%
\begin{pgfscope}%
\pgfsetbuttcap%
\pgfsetroundjoin%
\definecolor{currentfill}{rgb}{0.000000,0.000000,0.000000}%
\pgfsetfillcolor{currentfill}%
\pgfsetlinewidth{0.803000pt}%
\definecolor{currentstroke}{rgb}{0.000000,0.000000,0.000000}%
\pgfsetstrokecolor{currentstroke}%
\pgfsetdash{}{0pt}%
\pgfsys@defobject{currentmarker}{\pgfqpoint{0.000000in}{0.000000in}}{\pgfqpoint{0.048611in}{0.000000in}}{%
\pgfpathmoveto{\pgfqpoint{0.000000in}{0.000000in}}%
\pgfpathlineto{\pgfqpoint{0.048611in}{0.000000in}}%
\pgfusepath{stroke,fill}%
}%
\begin{pgfscope}%
\pgfsys@transformshift{2.193789in}{2.159075in}%
\pgfsys@useobject{currentmarker}{}%
\end{pgfscope}%
\end{pgfscope}%
\begin{pgfscope}%
\pgftext[x=2.291011in,y=2.111248in,left,base]{\rmfamily\fontsize{10.000000}{12.000000}\selectfont \(\displaystyle 30\)}%
\end{pgfscope}%
\begin{pgfscope}%
\pgfsetbuttcap%
\pgfsetroundjoin%
\definecolor{currentfill}{rgb}{0.000000,0.000000,0.000000}%
\pgfsetfillcolor{currentfill}%
\pgfsetlinewidth{0.803000pt}%
\definecolor{currentstroke}{rgb}{0.000000,0.000000,0.000000}%
\pgfsetstrokecolor{currentstroke}%
\pgfsetdash{}{0pt}%
\pgfsys@defobject{currentmarker}{\pgfqpoint{0.000000in}{0.000000in}}{\pgfqpoint{0.048611in}{0.000000in}}{%
\pgfpathmoveto{\pgfqpoint{0.000000in}{0.000000in}}%
\pgfpathlineto{\pgfqpoint{0.048611in}{0.000000in}}%
\pgfusepath{stroke,fill}%
}%
\begin{pgfscope}%
\pgfsys@transformshift{2.193789in}{2.465608in}%
\pgfsys@useobject{currentmarker}{}%
\end{pgfscope}%
\end{pgfscope}%
\begin{pgfscope}%
\pgftext[x=2.291011in,y=2.417781in,left,base]{\rmfamily\fontsize{10.000000}{12.000000}\selectfont \(\displaystyle 35\)}%
\end{pgfscope}%
\begin{pgfscope}%
\pgfsetbuttcap%
\pgfsetroundjoin%
\definecolor{currentfill}{rgb}{0.000000,0.000000,0.000000}%
\pgfsetfillcolor{currentfill}%
\pgfsetlinewidth{0.803000pt}%
\definecolor{currentstroke}{rgb}{0.000000,0.000000,0.000000}%
\pgfsetstrokecolor{currentstroke}%
\pgfsetdash{}{0pt}%
\pgfsys@defobject{currentmarker}{\pgfqpoint{0.000000in}{0.000000in}}{\pgfqpoint{0.048611in}{0.000000in}}{%
\pgfpathmoveto{\pgfqpoint{0.000000in}{0.000000in}}%
\pgfpathlineto{\pgfqpoint{0.048611in}{0.000000in}}%
\pgfusepath{stroke,fill}%
}%
\begin{pgfscope}%
\pgfsys@transformshift{2.193789in}{2.772141in}%
\pgfsys@useobject{currentmarker}{}%
\end{pgfscope}%
\end{pgfscope}%
\begin{pgfscope}%
\pgftext[x=2.291011in,y=2.724314in,left,base]{\rmfamily\fontsize{10.000000}{12.000000}\selectfont \(\displaystyle 40\)}%
\end{pgfscope}%
\begin{pgfscope}%
\pgfsetbuttcap%
\pgfsetmiterjoin%
\pgfsetlinewidth{0.803000pt}%
\definecolor{currentstroke}{rgb}{0.000000,0.000000,0.000000}%
\pgfsetstrokecolor{currentstroke}%
\pgfsetdash{}{0pt}%
\pgfpathmoveto{\pgfqpoint{2.063512in}{0.319877in}}%
\pgfpathlineto{\pgfqpoint{2.063512in}{0.330055in}}%
\pgfpathlineto{\pgfqpoint{2.063512in}{2.915230in}}%
\pgfpathlineto{\pgfqpoint{2.063512in}{2.925408in}}%
\pgfpathlineto{\pgfqpoint{2.193789in}{2.925408in}}%
\pgfpathlineto{\pgfqpoint{2.193789in}{2.915230in}}%
\pgfpathlineto{\pgfqpoint{2.193789in}{0.330055in}}%
\pgfpathlineto{\pgfqpoint{2.193789in}{0.319877in}}%
\pgfpathclose%
\pgfusepath{stroke}%
\end{pgfscope}%
\end{pgfpicture}%
\makeatother%
\endgroup%

    \vspace*{-0.4cm}
	\caption{300 K. Bin size $0.014e$}
	\end{subfigure}
	\hspace{0.6cm}
	\begin{subfigure}[b]{0.45\textwidth}
	\hspace*{-0.4cm}
	%% Creator: Matplotlib, PGF backend
%%
%% To include the figure in your LaTeX document, write
%%   \input{<filename>.pgf}
%%
%% Make sure the required packages are loaded in your preamble
%%   \usepackage{pgf}
%%
%% Figures using additional raster images can only be included by \input if
%% they are in the same directory as the main LaTeX file. For loading figures
%% from other directories you can use the `import` package
%%   \usepackage{import}
%% and then include the figures with
%%   \import{<path to file>}{<filename>.pgf}
%%
%% Matplotlib used the following preamble
%%   \usepackage[utf8x]{inputenc}
%%   \usepackage[T1]{fontenc}
%%
\begingroup%
\makeatletter%
\begin{pgfpicture}%
\pgfpathrectangle{\pgfpointorigin}{\pgfqpoint{2.529900in}{3.060408in}}%
\pgfusepath{use as bounding box, clip}%
\begin{pgfscope}%
\pgfsetbuttcap%
\pgfsetmiterjoin%
\definecolor{currentfill}{rgb}{1.000000,1.000000,1.000000}%
\pgfsetfillcolor{currentfill}%
\pgfsetlinewidth{0.000000pt}%
\definecolor{currentstroke}{rgb}{1.000000,1.000000,1.000000}%
\pgfsetstrokecolor{currentstroke}%
\pgfsetdash{}{0pt}%
\pgfpathmoveto{\pgfqpoint{0.000000in}{0.000000in}}%
\pgfpathlineto{\pgfqpoint{2.529900in}{0.000000in}}%
\pgfpathlineto{\pgfqpoint{2.529900in}{3.060408in}}%
\pgfpathlineto{\pgfqpoint{0.000000in}{3.060408in}}%
\pgfpathclose%
\pgfusepath{fill}%
\end{pgfscope}%
\begin{pgfscope}%
\pgfsetbuttcap%
\pgfsetmiterjoin%
\definecolor{currentfill}{rgb}{1.000000,1.000000,1.000000}%
\pgfsetfillcolor{currentfill}%
\pgfsetlinewidth{0.000000pt}%
\definecolor{currentstroke}{rgb}{0.000000,0.000000,0.000000}%
\pgfsetstrokecolor{currentstroke}%
\pgfsetstrokeopacity{0.000000}%
\pgfsetdash{}{0pt}%
\pgfpathmoveto{\pgfqpoint{0.444137in}{0.319877in}}%
\pgfpathlineto{\pgfqpoint{1.968255in}{0.319877in}}%
\pgfpathlineto{\pgfqpoint{1.968255in}{2.925408in}}%
\pgfpathlineto{\pgfqpoint{0.444137in}{2.925408in}}%
\pgfpathclose%
\pgfusepath{fill}%
\end{pgfscope}%
\begin{pgfscope}%
\pgfpathrectangle{\pgfqpoint{0.444137in}{0.319877in}}{\pgfqpoint{1.524118in}{2.605531in}} %
\pgfusepath{clip}%
\pgfsys@transformshift{0.444137in}{0.319877in}%
\pgftext[left,bottom]{\pgfimage[interpolate=true,width=1.530000in,height=2.610000in]{FerrNN_vs_dq_Ti_500K-img0.png}}%
\end{pgfscope}%
\begin{pgfscope}%
\pgfpathrectangle{\pgfqpoint{0.444137in}{0.319877in}}{\pgfqpoint{1.524118in}{2.605531in}} %
\pgfusepath{clip}%
\pgfsetbuttcap%
\pgfsetroundjoin%
\definecolor{currentfill}{rgb}{1.000000,0.752941,0.796078}%
\pgfsetfillcolor{currentfill}%
\pgfsetlinewidth{1.003750pt}%
\definecolor{currentstroke}{rgb}{1.000000,0.752941,0.796078}%
\pgfsetstrokecolor{currentstroke}%
\pgfsetdash{}{0pt}%
\pgfpathmoveto{\pgfqpoint{0.729909in}{2.139304in}}%
\pgfpathcurveto{\pgfqpoint{0.740959in}{2.139304in}}{\pgfqpoint{0.751558in}{2.143694in}}{\pgfqpoint{0.759371in}{2.151508in}}%
\pgfpathcurveto{\pgfqpoint{0.767185in}{2.159321in}}{\pgfqpoint{0.771575in}{2.169920in}}{\pgfqpoint{0.771575in}{2.180971in}}%
\pgfpathcurveto{\pgfqpoint{0.771575in}{2.192021in}}{\pgfqpoint{0.767185in}{2.202620in}}{\pgfqpoint{0.759371in}{2.210433in}}%
\pgfpathcurveto{\pgfqpoint{0.751558in}{2.218247in}}{\pgfqpoint{0.740959in}{2.222637in}}{\pgfqpoint{0.729909in}{2.222637in}}%
\pgfpathcurveto{\pgfqpoint{0.718859in}{2.222637in}}{\pgfqpoint{0.708260in}{2.218247in}}{\pgfqpoint{0.700446in}{2.210433in}}%
\pgfpathcurveto{\pgfqpoint{0.692632in}{2.202620in}}{\pgfqpoint{0.688242in}{2.192021in}}{\pgfqpoint{0.688242in}{2.180971in}}%
\pgfpathcurveto{\pgfqpoint{0.688242in}{2.169920in}}{\pgfqpoint{0.692632in}{2.159321in}}{\pgfqpoint{0.700446in}{2.151508in}}%
\pgfpathcurveto{\pgfqpoint{0.708260in}{2.143694in}}{\pgfqpoint{0.718859in}{2.139304in}}{\pgfqpoint{0.729909in}{2.139304in}}%
\pgfpathclose%
\pgfusepath{stroke,fill}%
\end{pgfscope}%
\begin{pgfscope}%
\pgfpathrectangle{\pgfqpoint{0.444137in}{0.319877in}}{\pgfqpoint{1.524118in}{2.605531in}} %
\pgfusepath{clip}%
\pgfsetbuttcap%
\pgfsetroundjoin%
\definecolor{currentfill}{rgb}{1.000000,0.752941,0.796078}%
\pgfsetfillcolor{currentfill}%
\pgfsetlinewidth{1.003750pt}%
\definecolor{currentstroke}{rgb}{1.000000,0.752941,0.796078}%
\pgfsetstrokecolor{currentstroke}%
\pgfsetdash{}{0pt}%
\pgfpathmoveto{\pgfqpoint{0.920423in}{1.482447in}}%
\pgfpathcurveto{\pgfqpoint{0.931474in}{1.482447in}}{\pgfqpoint{0.942073in}{1.486838in}}{\pgfqpoint{0.949886in}{1.494651in}}%
\pgfpathcurveto{\pgfqpoint{0.957700in}{1.502465in}}{\pgfqpoint{0.962090in}{1.513064in}}{\pgfqpoint{0.962090in}{1.524114in}}%
\pgfpathcurveto{\pgfqpoint{0.962090in}{1.535164in}}{\pgfqpoint{0.957700in}{1.545763in}}{\pgfqpoint{0.949886in}{1.553577in}}%
\pgfpathcurveto{\pgfqpoint{0.942073in}{1.561391in}}{\pgfqpoint{0.931474in}{1.565781in}}{\pgfqpoint{0.920423in}{1.565781in}}%
\pgfpathcurveto{\pgfqpoint{0.909373in}{1.565781in}}{\pgfqpoint{0.898774in}{1.561391in}}{\pgfqpoint{0.890961in}{1.553577in}}%
\pgfpathcurveto{\pgfqpoint{0.883147in}{1.545763in}}{\pgfqpoint{0.878757in}{1.535164in}}{\pgfqpoint{0.878757in}{1.524114in}}%
\pgfpathcurveto{\pgfqpoint{0.878757in}{1.513064in}}{\pgfqpoint{0.883147in}{1.502465in}}{\pgfqpoint{0.890961in}{1.494651in}}%
\pgfpathcurveto{\pgfqpoint{0.898774in}{1.486838in}}{\pgfqpoint{0.909373in}{1.482447in}}{\pgfqpoint{0.920423in}{1.482447in}}%
\pgfpathclose%
\pgfusepath{stroke,fill}%
\end{pgfscope}%
\begin{pgfscope}%
\pgfpathrectangle{\pgfqpoint{0.444137in}{0.319877in}}{\pgfqpoint{1.524118in}{2.605531in}} %
\pgfusepath{clip}%
\pgfsetbuttcap%
\pgfsetroundjoin%
\definecolor{currentfill}{rgb}{1.000000,0.752941,0.796078}%
\pgfsetfillcolor{currentfill}%
\pgfsetlinewidth{1.003750pt}%
\definecolor{currentstroke}{rgb}{1.000000,0.752941,0.796078}%
\pgfsetstrokecolor{currentstroke}%
\pgfsetdash{}{0pt}%
\pgfpathmoveto{\pgfqpoint{1.110938in}{1.023968in}}%
\pgfpathcurveto{\pgfqpoint{1.121988in}{1.023968in}}{\pgfqpoint{1.132587in}{1.028358in}}{\pgfqpoint{1.140401in}{1.036172in}}%
\pgfpathcurveto{\pgfqpoint{1.148215in}{1.043985in}}{\pgfqpoint{1.152605in}{1.054584in}}{\pgfqpoint{1.152605in}{1.065635in}}%
\pgfpathcurveto{\pgfqpoint{1.152605in}{1.076685in}}{\pgfqpoint{1.148215in}{1.087284in}}{\pgfqpoint{1.140401in}{1.095097in}}%
\pgfpathcurveto{\pgfqpoint{1.132587in}{1.102911in}}{\pgfqpoint{1.121988in}{1.107301in}}{\pgfqpoint{1.110938in}{1.107301in}}%
\pgfpathcurveto{\pgfqpoint{1.099888in}{1.107301in}}{\pgfqpoint{1.089289in}{1.102911in}}{\pgfqpoint{1.081475in}{1.095097in}}%
\pgfpathcurveto{\pgfqpoint{1.073662in}{1.087284in}}{\pgfqpoint{1.069272in}{1.076685in}}{\pgfqpoint{1.069272in}{1.065635in}}%
\pgfpathcurveto{\pgfqpoint{1.069272in}{1.054584in}}{\pgfqpoint{1.073662in}{1.043985in}}{\pgfqpoint{1.081475in}{1.036172in}}%
\pgfpathcurveto{\pgfqpoint{1.089289in}{1.028358in}}{\pgfqpoint{1.099888in}{1.023968in}}{\pgfqpoint{1.110938in}{1.023968in}}%
\pgfpathclose%
\pgfusepath{stroke,fill}%
\end{pgfscope}%
\begin{pgfscope}%
\pgfpathrectangle{\pgfqpoint{0.444137in}{0.319877in}}{\pgfqpoint{1.524118in}{2.605531in}} %
\pgfusepath{clip}%
\pgfsetbuttcap%
\pgfsetroundjoin%
\definecolor{currentfill}{rgb}{1.000000,0.752941,0.796078}%
\pgfsetfillcolor{currentfill}%
\pgfsetlinewidth{1.003750pt}%
\definecolor{currentstroke}{rgb}{1.000000,0.752941,0.796078}%
\pgfsetstrokecolor{currentstroke}%
\pgfsetdash{}{0pt}%
\pgfpathmoveto{\pgfqpoint{1.301453in}{0.961251in}}%
\pgfpathcurveto{\pgfqpoint{1.312503in}{0.961251in}}{\pgfqpoint{1.323102in}{0.965641in}}{\pgfqpoint{1.330916in}{0.973455in}}%
\pgfpathcurveto{\pgfqpoint{1.338729in}{0.981269in}}{\pgfqpoint{1.343120in}{0.991868in}}{\pgfqpoint{1.343120in}{1.002918in}}%
\pgfpathcurveto{\pgfqpoint{1.343120in}{1.013968in}}{\pgfqpoint{1.338729in}{1.024567in}}{\pgfqpoint{1.330916in}{1.032381in}}%
\pgfpathcurveto{\pgfqpoint{1.323102in}{1.040194in}}{\pgfqpoint{1.312503in}{1.044584in}}{\pgfqpoint{1.301453in}{1.044584in}}%
\pgfpathcurveto{\pgfqpoint{1.290403in}{1.044584in}}{\pgfqpoint{1.279804in}{1.040194in}}{\pgfqpoint{1.271990in}{1.032381in}}%
\pgfpathcurveto{\pgfqpoint{1.264177in}{1.024567in}}{\pgfqpoint{1.259786in}{1.013968in}}{\pgfqpoint{1.259786in}{1.002918in}}%
\pgfpathcurveto{\pgfqpoint{1.259786in}{0.991868in}}{\pgfqpoint{1.264177in}{0.981269in}}{\pgfqpoint{1.271990in}{0.973455in}}%
\pgfpathcurveto{\pgfqpoint{1.279804in}{0.965641in}}{\pgfqpoint{1.290403in}{0.961251in}}{\pgfqpoint{1.301453in}{0.961251in}}%
\pgfpathclose%
\pgfusepath{stroke,fill}%
\end{pgfscope}%
\begin{pgfscope}%
\pgfpathrectangle{\pgfqpoint{0.444137in}{0.319877in}}{\pgfqpoint{1.524118in}{2.605531in}} %
\pgfusepath{clip}%
\pgfsetbuttcap%
\pgfsetroundjoin%
\definecolor{currentfill}{rgb}{1.000000,0.752941,0.796078}%
\pgfsetfillcolor{currentfill}%
\pgfsetlinewidth{1.003750pt}%
\definecolor{currentstroke}{rgb}{1.000000,0.752941,0.796078}%
\pgfsetstrokecolor{currentstroke}%
\pgfsetdash{}{0pt}%
\pgfpathmoveto{\pgfqpoint{1.491968in}{1.536664in}}%
\pgfpathcurveto{\pgfqpoint{1.503018in}{1.536664in}}{\pgfqpoint{1.513617in}{1.541054in}}{\pgfqpoint{1.521431in}{1.548868in}}%
\pgfpathcurveto{\pgfqpoint{1.529244in}{1.556682in}}{\pgfqpoint{1.533634in}{1.567281in}}{\pgfqpoint{1.533634in}{1.578331in}}%
\pgfpathcurveto{\pgfqpoint{1.533634in}{1.589381in}}{\pgfqpoint{1.529244in}{1.599980in}}{\pgfqpoint{1.521431in}{1.607794in}}%
\pgfpathcurveto{\pgfqpoint{1.513617in}{1.615607in}}{\pgfqpoint{1.503018in}{1.619998in}}{\pgfqpoint{1.491968in}{1.619998in}}%
\pgfpathcurveto{\pgfqpoint{1.480918in}{1.619998in}}{\pgfqpoint{1.470319in}{1.615607in}}{\pgfqpoint{1.462505in}{1.607794in}}%
\pgfpathcurveto{\pgfqpoint{1.454691in}{1.599980in}}{\pgfqpoint{1.450301in}{1.589381in}}{\pgfqpoint{1.450301in}{1.578331in}}%
\pgfpathcurveto{\pgfqpoint{1.450301in}{1.567281in}}{\pgfqpoint{1.454691in}{1.556682in}}{\pgfqpoint{1.462505in}{1.548868in}}%
\pgfpathcurveto{\pgfqpoint{1.470319in}{1.541054in}}{\pgfqpoint{1.480918in}{1.536664in}}{\pgfqpoint{1.491968in}{1.536664in}}%
\pgfpathclose%
\pgfusepath{stroke,fill}%
\end{pgfscope}%
\begin{pgfscope}%
\pgfpathrectangle{\pgfqpoint{0.444137in}{0.319877in}}{\pgfqpoint{1.524118in}{2.605531in}} %
\pgfusepath{clip}%
\pgfsetbuttcap%
\pgfsetroundjoin%
\definecolor{currentfill}{rgb}{1.000000,0.752941,0.796078}%
\pgfsetfillcolor{currentfill}%
\pgfsetlinewidth{1.003750pt}%
\definecolor{currentstroke}{rgb}{1.000000,0.752941,0.796078}%
\pgfsetstrokecolor{currentstroke}%
\pgfsetdash{}{0pt}%
\pgfpathmoveto{\pgfqpoint{1.682483in}{2.201340in}}%
\pgfpathcurveto{\pgfqpoint{1.693533in}{2.201340in}}{\pgfqpoint{1.704132in}{2.205731in}}{\pgfqpoint{1.711945in}{2.213544in}}%
\pgfpathcurveto{\pgfqpoint{1.719759in}{2.221358in}}{\pgfqpoint{1.724149in}{2.231957in}}{\pgfqpoint{1.724149in}{2.243007in}}%
\pgfpathcurveto{\pgfqpoint{1.724149in}{2.254057in}}{\pgfqpoint{1.719759in}{2.264656in}}{\pgfqpoint{1.711945in}{2.272470in}}%
\pgfpathcurveto{\pgfqpoint{1.704132in}{2.280283in}}{\pgfqpoint{1.693533in}{2.284674in}}{\pgfqpoint{1.682483in}{2.284674in}}%
\pgfpathcurveto{\pgfqpoint{1.671432in}{2.284674in}}{\pgfqpoint{1.660833in}{2.280283in}}{\pgfqpoint{1.653020in}{2.272470in}}%
\pgfpathcurveto{\pgfqpoint{1.645206in}{2.264656in}}{\pgfqpoint{1.640816in}{2.254057in}}{\pgfqpoint{1.640816in}{2.243007in}}%
\pgfpathcurveto{\pgfqpoint{1.640816in}{2.231957in}}{\pgfqpoint{1.645206in}{2.221358in}}{\pgfqpoint{1.653020in}{2.213544in}}%
\pgfpathcurveto{\pgfqpoint{1.660833in}{2.205731in}}{\pgfqpoint{1.671432in}{2.201340in}}{\pgfqpoint{1.682483in}{2.201340in}}%
\pgfpathclose%
\pgfusepath{stroke,fill}%
\end{pgfscope}%
\begin{pgfscope}%
\pgfpathrectangle{\pgfqpoint{0.444137in}{0.319877in}}{\pgfqpoint{1.524118in}{2.605531in}} %
\pgfusepath{clip}%
\pgfsetbuttcap%
\pgfsetroundjoin%
\definecolor{currentfill}{rgb}{1.000000,0.752941,0.796078}%
\pgfsetfillcolor{currentfill}%
\pgfsetlinewidth{1.003750pt}%
\definecolor{currentstroke}{rgb}{1.000000,0.752941,0.796078}%
\pgfsetstrokecolor{currentstroke}%
\pgfsetdash{}{0pt}%
\pgfpathmoveto{\pgfqpoint{1.872997in}{2.759668in}}%
\pgfpathcurveto{\pgfqpoint{1.884047in}{2.759668in}}{\pgfqpoint{1.894646in}{2.764059in}}{\pgfqpoint{1.902460in}{2.771872in}}%
\pgfpathcurveto{\pgfqpoint{1.910274in}{2.779686in}}{\pgfqpoint{1.914664in}{2.790285in}}{\pgfqpoint{1.914664in}{2.801335in}}%
\pgfpathcurveto{\pgfqpoint{1.914664in}{2.812385in}}{\pgfqpoint{1.910274in}{2.822984in}}{\pgfqpoint{1.902460in}{2.830798in}}%
\pgfpathcurveto{\pgfqpoint{1.894646in}{2.838611in}}{\pgfqpoint{1.884047in}{2.843002in}}{\pgfqpoint{1.872997in}{2.843002in}}%
\pgfpathcurveto{\pgfqpoint{1.861947in}{2.843002in}}{\pgfqpoint{1.851348in}{2.838611in}}{\pgfqpoint{1.843534in}{2.830798in}}%
\pgfpathcurveto{\pgfqpoint{1.835721in}{2.822984in}}{\pgfqpoint{1.831331in}{2.812385in}}{\pgfqpoint{1.831331in}{2.801335in}}%
\pgfpathcurveto{\pgfqpoint{1.831331in}{2.790285in}}{\pgfqpoint{1.835721in}{2.779686in}}{\pgfqpoint{1.843534in}{2.771872in}}%
\pgfpathcurveto{\pgfqpoint{1.851348in}{2.764059in}}{\pgfqpoint{1.861947in}{2.759668in}}{\pgfqpoint{1.872997in}{2.759668in}}%
\pgfpathclose%
\pgfusepath{stroke,fill}%
\end{pgfscope}%
\begin{pgfscope}%
\pgfsetbuttcap%
\pgfsetroundjoin%
\definecolor{currentfill}{rgb}{0.000000,0.000000,0.000000}%
\pgfsetfillcolor{currentfill}%
\pgfsetlinewidth{0.803000pt}%
\definecolor{currentstroke}{rgb}{0.000000,0.000000,0.000000}%
\pgfsetstrokecolor{currentstroke}%
\pgfsetdash{}{0pt}%
\pgfsys@defobject{currentmarker}{\pgfqpoint{0.000000in}{-0.048611in}}{\pgfqpoint{0.000000in}{0.000000in}}{%
\pgfpathmoveto{\pgfqpoint{0.000000in}{0.000000in}}%
\pgfpathlineto{\pgfqpoint{0.000000in}{-0.048611in}}%
\pgfusepath{stroke,fill}%
}%
\begin{pgfscope}%
\pgfsys@transformshift{0.729909in}{0.319877in}%
\pgfsys@useobject{currentmarker}{}%
\end{pgfscope}%
\end{pgfscope}%
\begin{pgfscope}%
\pgftext[x=0.729909in,y=0.222655in,,top]{\rmfamily\fontsize{10.000000}{12.000000}\selectfont \(\displaystyle -0.05\)}%
\end{pgfscope}%
\begin{pgfscope}%
\pgfsetbuttcap%
\pgfsetroundjoin%
\definecolor{currentfill}{rgb}{0.000000,0.000000,0.000000}%
\pgfsetfillcolor{currentfill}%
\pgfsetlinewidth{0.803000pt}%
\definecolor{currentstroke}{rgb}{0.000000,0.000000,0.000000}%
\pgfsetstrokecolor{currentstroke}%
\pgfsetdash{}{0pt}%
\pgfsys@defobject{currentmarker}{\pgfqpoint{0.000000in}{-0.048611in}}{\pgfqpoint{0.000000in}{0.000000in}}{%
\pgfpathmoveto{\pgfqpoint{0.000000in}{0.000000in}}%
\pgfpathlineto{\pgfqpoint{0.000000in}{-0.048611in}}%
\pgfusepath{stroke,fill}%
}%
\begin{pgfscope}%
\pgfsys@transformshift{1.206196in}{0.319877in}%
\pgfsys@useobject{currentmarker}{}%
\end{pgfscope}%
\end{pgfscope}%
\begin{pgfscope}%
\pgftext[x=1.206196in,y=0.222655in,,top]{\rmfamily\fontsize{10.000000}{12.000000}\selectfont \(\displaystyle 0.00\)}%
\end{pgfscope}%
\begin{pgfscope}%
\pgfsetbuttcap%
\pgfsetroundjoin%
\definecolor{currentfill}{rgb}{0.000000,0.000000,0.000000}%
\pgfsetfillcolor{currentfill}%
\pgfsetlinewidth{0.803000pt}%
\definecolor{currentstroke}{rgb}{0.000000,0.000000,0.000000}%
\pgfsetstrokecolor{currentstroke}%
\pgfsetdash{}{0pt}%
\pgfsys@defobject{currentmarker}{\pgfqpoint{0.000000in}{-0.048611in}}{\pgfqpoint{0.000000in}{0.000000in}}{%
\pgfpathmoveto{\pgfqpoint{0.000000in}{0.000000in}}%
\pgfpathlineto{\pgfqpoint{0.000000in}{-0.048611in}}%
\pgfusepath{stroke,fill}%
}%
\begin{pgfscope}%
\pgfsys@transformshift{1.682483in}{0.319877in}%
\pgfsys@useobject{currentmarker}{}%
\end{pgfscope}%
\end{pgfscope}%
\begin{pgfscope}%
\pgftext[x=1.682483in,y=0.222655in,,top]{\rmfamily\fontsize{10.000000}{12.000000}\selectfont \(\displaystyle 0.05\)}%
\end{pgfscope}%
\begin{pgfscope}%
\pgfsetbuttcap%
\pgfsetroundjoin%
\definecolor{currentfill}{rgb}{0.000000,0.000000,0.000000}%
\pgfsetfillcolor{currentfill}%
\pgfsetlinewidth{0.803000pt}%
\definecolor{currentstroke}{rgb}{0.000000,0.000000,0.000000}%
\pgfsetstrokecolor{currentstroke}%
\pgfsetdash{}{0pt}%
\pgfsys@defobject{currentmarker}{\pgfqpoint{-0.048611in}{0.000000in}}{\pgfqpoint{0.000000in}{0.000000in}}{%
\pgfpathmoveto{\pgfqpoint{0.000000in}{0.000000in}}%
\pgfpathlineto{\pgfqpoint{-0.048611in}{0.000000in}}%
\pgfusepath{stroke,fill}%
}%
\begin{pgfscope}%
\pgfsys@transformshift{0.444137in}{0.319877in}%
\pgfsys@useobject{currentmarker}{}%
\end{pgfscope}%
\end{pgfscope}%
\begin{pgfscope}%
\pgftext[x=0.100000in,y=0.272050in,left,base]{\rmfamily\fontsize{10.000000}{12.000000}\selectfont \(\displaystyle 0.00\)}%
\end{pgfscope}%
\begin{pgfscope}%
\pgfsetbuttcap%
\pgfsetroundjoin%
\definecolor{currentfill}{rgb}{0.000000,0.000000,0.000000}%
\pgfsetfillcolor{currentfill}%
\pgfsetlinewidth{0.803000pt}%
\definecolor{currentstroke}{rgb}{0.000000,0.000000,0.000000}%
\pgfsetstrokecolor{currentstroke}%
\pgfsetdash{}{0pt}%
\pgfsys@defobject{currentmarker}{\pgfqpoint{-0.048611in}{0.000000in}}{\pgfqpoint{0.000000in}{0.000000in}}{%
\pgfpathmoveto{\pgfqpoint{0.000000in}{0.000000in}}%
\pgfpathlineto{\pgfqpoint{-0.048611in}{0.000000in}}%
\pgfusepath{stroke,fill}%
}%
\begin{pgfscope}%
\pgfsys@transformshift{0.444137in}{0.776472in}%
\pgfsys@useobject{currentmarker}{}%
\end{pgfscope}%
\end{pgfscope}%
\begin{pgfscope}%
\pgftext[x=0.100000in,y=0.728645in,left,base]{\rmfamily\fontsize{10.000000}{12.000000}\selectfont \(\displaystyle 0.05\)}%
\end{pgfscope}%
\begin{pgfscope}%
\pgfsetbuttcap%
\pgfsetroundjoin%
\definecolor{currentfill}{rgb}{0.000000,0.000000,0.000000}%
\pgfsetfillcolor{currentfill}%
\pgfsetlinewidth{0.803000pt}%
\definecolor{currentstroke}{rgb}{0.000000,0.000000,0.000000}%
\pgfsetstrokecolor{currentstroke}%
\pgfsetdash{}{0pt}%
\pgfsys@defobject{currentmarker}{\pgfqpoint{-0.048611in}{0.000000in}}{\pgfqpoint{0.000000in}{0.000000in}}{%
\pgfpathmoveto{\pgfqpoint{0.000000in}{0.000000in}}%
\pgfpathlineto{\pgfqpoint{-0.048611in}{0.000000in}}%
\pgfusepath{stroke,fill}%
}%
\begin{pgfscope}%
\pgfsys@transformshift{0.444137in}{1.233067in}%
\pgfsys@useobject{currentmarker}{}%
\end{pgfscope}%
\end{pgfscope}%
\begin{pgfscope}%
\pgftext[x=0.100000in,y=1.185240in,left,base]{\rmfamily\fontsize{10.000000}{12.000000}\selectfont \(\displaystyle 0.10\)}%
\end{pgfscope}%
\begin{pgfscope}%
\pgfsetbuttcap%
\pgfsetroundjoin%
\definecolor{currentfill}{rgb}{0.000000,0.000000,0.000000}%
\pgfsetfillcolor{currentfill}%
\pgfsetlinewidth{0.803000pt}%
\definecolor{currentstroke}{rgb}{0.000000,0.000000,0.000000}%
\pgfsetstrokecolor{currentstroke}%
\pgfsetdash{}{0pt}%
\pgfsys@defobject{currentmarker}{\pgfqpoint{-0.048611in}{0.000000in}}{\pgfqpoint{0.000000in}{0.000000in}}{%
\pgfpathmoveto{\pgfqpoint{0.000000in}{0.000000in}}%
\pgfpathlineto{\pgfqpoint{-0.048611in}{0.000000in}}%
\pgfusepath{stroke,fill}%
}%
\begin{pgfscope}%
\pgfsys@transformshift{0.444137in}{1.689662in}%
\pgfsys@useobject{currentmarker}{}%
\end{pgfscope}%
\end{pgfscope}%
\begin{pgfscope}%
\pgftext[x=0.100000in,y=1.641835in,left,base]{\rmfamily\fontsize{10.000000}{12.000000}\selectfont \(\displaystyle 0.15\)}%
\end{pgfscope}%
\begin{pgfscope}%
\pgfsetbuttcap%
\pgfsetroundjoin%
\definecolor{currentfill}{rgb}{0.000000,0.000000,0.000000}%
\pgfsetfillcolor{currentfill}%
\pgfsetlinewidth{0.803000pt}%
\definecolor{currentstroke}{rgb}{0.000000,0.000000,0.000000}%
\pgfsetstrokecolor{currentstroke}%
\pgfsetdash{}{0pt}%
\pgfsys@defobject{currentmarker}{\pgfqpoint{-0.048611in}{0.000000in}}{\pgfqpoint{0.000000in}{0.000000in}}{%
\pgfpathmoveto{\pgfqpoint{0.000000in}{0.000000in}}%
\pgfpathlineto{\pgfqpoint{-0.048611in}{0.000000in}}%
\pgfusepath{stroke,fill}%
}%
\begin{pgfscope}%
\pgfsys@transformshift{0.444137in}{2.146257in}%
\pgfsys@useobject{currentmarker}{}%
\end{pgfscope}%
\end{pgfscope}%
\begin{pgfscope}%
\pgftext[x=0.100000in,y=2.098430in,left,base]{\rmfamily\fontsize{10.000000}{12.000000}\selectfont \(\displaystyle 0.20\)}%
\end{pgfscope}%
\begin{pgfscope}%
\pgfsetbuttcap%
\pgfsetroundjoin%
\definecolor{currentfill}{rgb}{0.000000,0.000000,0.000000}%
\pgfsetfillcolor{currentfill}%
\pgfsetlinewidth{0.803000pt}%
\definecolor{currentstroke}{rgb}{0.000000,0.000000,0.000000}%
\pgfsetstrokecolor{currentstroke}%
\pgfsetdash{}{0pt}%
\pgfsys@defobject{currentmarker}{\pgfqpoint{-0.048611in}{0.000000in}}{\pgfqpoint{0.000000in}{0.000000in}}{%
\pgfpathmoveto{\pgfqpoint{0.000000in}{0.000000in}}%
\pgfpathlineto{\pgfqpoint{-0.048611in}{0.000000in}}%
\pgfusepath{stroke,fill}%
}%
\begin{pgfscope}%
\pgfsys@transformshift{0.444137in}{2.602852in}%
\pgfsys@useobject{currentmarker}{}%
\end{pgfscope}%
\end{pgfscope}%
\begin{pgfscope}%
\pgftext[x=0.100000in,y=2.555025in,left,base]{\rmfamily\fontsize{10.000000}{12.000000}\selectfont \(\displaystyle 0.25\)}%
\end{pgfscope}%
\begin{pgfscope}%
\pgfsetrectcap%
\pgfsetmiterjoin%
\pgfsetlinewidth{0.803000pt}%
\definecolor{currentstroke}{rgb}{0.000000,0.000000,0.000000}%
\pgfsetstrokecolor{currentstroke}%
\pgfsetdash{}{0pt}%
\pgfpathmoveto{\pgfqpoint{0.444137in}{0.319877in}}%
\pgfpathlineto{\pgfqpoint{0.444137in}{2.925408in}}%
\pgfusepath{stroke}%
\end{pgfscope}%
\begin{pgfscope}%
\pgfsetrectcap%
\pgfsetmiterjoin%
\pgfsetlinewidth{0.803000pt}%
\definecolor{currentstroke}{rgb}{0.000000,0.000000,0.000000}%
\pgfsetstrokecolor{currentstroke}%
\pgfsetdash{}{0pt}%
\pgfpathmoveto{\pgfqpoint{1.968255in}{0.319877in}}%
\pgfpathlineto{\pgfqpoint{1.968255in}{2.925408in}}%
\pgfusepath{stroke}%
\end{pgfscope}%
\begin{pgfscope}%
\pgfsetrectcap%
\pgfsetmiterjoin%
\pgfsetlinewidth{0.803000pt}%
\definecolor{currentstroke}{rgb}{0.000000,0.000000,0.000000}%
\pgfsetstrokecolor{currentstroke}%
\pgfsetdash{}{0pt}%
\pgfpathmoveto{\pgfqpoint{0.444137in}{0.319877in}}%
\pgfpathlineto{\pgfqpoint{1.968255in}{0.319877in}}%
\pgfusepath{stroke}%
\end{pgfscope}%
\begin{pgfscope}%
\pgfsetrectcap%
\pgfsetmiterjoin%
\pgfsetlinewidth{0.803000pt}%
\definecolor{currentstroke}{rgb}{0.000000,0.000000,0.000000}%
\pgfsetstrokecolor{currentstroke}%
\pgfsetdash{}{0pt}%
\pgfpathmoveto{\pgfqpoint{0.444137in}{2.925408in}}%
\pgfpathlineto{\pgfqpoint{1.968255in}{2.925408in}}%
\pgfusepath{stroke}%
\end{pgfscope}%
\begin{pgfscope}%
\pgfpathrectangle{\pgfqpoint{2.063512in}{0.319877in}}{\pgfqpoint{0.130277in}{2.605531in}} %
\pgfusepath{clip}%
\pgfsetbuttcap%
\pgfsetmiterjoin%
\definecolor{currentfill}{rgb}{1.000000,1.000000,1.000000}%
\pgfsetfillcolor{currentfill}%
\pgfsetlinewidth{0.010037pt}%
\definecolor{currentstroke}{rgb}{1.000000,1.000000,1.000000}%
\pgfsetstrokecolor{currentstroke}%
\pgfsetdash{}{0pt}%
\pgfpathmoveto{\pgfqpoint{2.063512in}{0.319877in}}%
\pgfpathlineto{\pgfqpoint{2.063512in}{0.330055in}}%
\pgfpathlineto{\pgfqpoint{2.063512in}{2.915230in}}%
\pgfpathlineto{\pgfqpoint{2.063512in}{2.925408in}}%
\pgfpathlineto{\pgfqpoint{2.193789in}{2.925408in}}%
\pgfpathlineto{\pgfqpoint{2.193789in}{2.915230in}}%
\pgfpathlineto{\pgfqpoint{2.193789in}{0.330055in}}%
\pgfpathlineto{\pgfqpoint{2.193789in}{0.319877in}}%
\pgfpathclose%
\pgfusepath{stroke,fill}%
\end{pgfscope}%
\begin{pgfscope}%
\pgfsys@transformshift{2.060000in}{0.320408in}%
\pgftext[left,bottom]{\pgfimage[interpolate=true,width=0.130000in,height=2.610000in]{FerrNN_vs_dq_Ti_500K-img1.png}}%
\end{pgfscope}%
\begin{pgfscope}%
\pgfsetbuttcap%
\pgfsetroundjoin%
\definecolor{currentfill}{rgb}{0.000000,0.000000,0.000000}%
\pgfsetfillcolor{currentfill}%
\pgfsetlinewidth{0.803000pt}%
\definecolor{currentstroke}{rgb}{0.000000,0.000000,0.000000}%
\pgfsetstrokecolor{currentstroke}%
\pgfsetdash{}{0pt}%
\pgfsys@defobject{currentmarker}{\pgfqpoint{0.000000in}{0.000000in}}{\pgfqpoint{0.048611in}{0.000000in}}{%
\pgfpathmoveto{\pgfqpoint{0.000000in}{0.000000in}}%
\pgfpathlineto{\pgfqpoint{0.048611in}{0.000000in}}%
\pgfusepath{stroke,fill}%
}%
\begin{pgfscope}%
\pgfsys@transformshift{2.193789in}{0.319877in}%
\pgfsys@useobject{currentmarker}{}%
\end{pgfscope}%
\end{pgfscope}%
\begin{pgfscope}%
\pgftext[x=2.291011in,y=0.272050in,left,base]{\rmfamily\fontsize{10.000000}{12.000000}\selectfont \(\displaystyle 0\)}%
\end{pgfscope}%
\begin{pgfscope}%
\pgfsetbuttcap%
\pgfsetroundjoin%
\definecolor{currentfill}{rgb}{0.000000,0.000000,0.000000}%
\pgfsetfillcolor{currentfill}%
\pgfsetlinewidth{0.803000pt}%
\definecolor{currentstroke}{rgb}{0.000000,0.000000,0.000000}%
\pgfsetstrokecolor{currentstroke}%
\pgfsetdash{}{0pt}%
\pgfsys@defobject{currentmarker}{\pgfqpoint{0.000000in}{0.000000in}}{\pgfqpoint{0.048611in}{0.000000in}}{%
\pgfpathmoveto{\pgfqpoint{0.000000in}{0.000000in}}%
\pgfpathlineto{\pgfqpoint{0.048611in}{0.000000in}}%
\pgfusepath{stroke,fill}%
}%
\begin{pgfscope}%
\pgfsys@transformshift{2.193789in}{0.626410in}%
\pgfsys@useobject{currentmarker}{}%
\end{pgfscope}%
\end{pgfscope}%
\begin{pgfscope}%
\pgftext[x=2.291011in,y=0.578583in,left,base]{\rmfamily\fontsize{10.000000}{12.000000}\selectfont \(\displaystyle 5\)}%
\end{pgfscope}%
\begin{pgfscope}%
\pgfsetbuttcap%
\pgfsetroundjoin%
\definecolor{currentfill}{rgb}{0.000000,0.000000,0.000000}%
\pgfsetfillcolor{currentfill}%
\pgfsetlinewidth{0.803000pt}%
\definecolor{currentstroke}{rgb}{0.000000,0.000000,0.000000}%
\pgfsetstrokecolor{currentstroke}%
\pgfsetdash{}{0pt}%
\pgfsys@defobject{currentmarker}{\pgfqpoint{0.000000in}{0.000000in}}{\pgfqpoint{0.048611in}{0.000000in}}{%
\pgfpathmoveto{\pgfqpoint{0.000000in}{0.000000in}}%
\pgfpathlineto{\pgfqpoint{0.048611in}{0.000000in}}%
\pgfusepath{stroke,fill}%
}%
\begin{pgfscope}%
\pgfsys@transformshift{2.193789in}{0.932943in}%
\pgfsys@useobject{currentmarker}{}%
\end{pgfscope}%
\end{pgfscope}%
\begin{pgfscope}%
\pgftext[x=2.291011in,y=0.885116in,left,base]{\rmfamily\fontsize{10.000000}{12.000000}\selectfont \(\displaystyle 10\)}%
\end{pgfscope}%
\begin{pgfscope}%
\pgfsetbuttcap%
\pgfsetroundjoin%
\definecolor{currentfill}{rgb}{0.000000,0.000000,0.000000}%
\pgfsetfillcolor{currentfill}%
\pgfsetlinewidth{0.803000pt}%
\definecolor{currentstroke}{rgb}{0.000000,0.000000,0.000000}%
\pgfsetstrokecolor{currentstroke}%
\pgfsetdash{}{0pt}%
\pgfsys@defobject{currentmarker}{\pgfqpoint{0.000000in}{0.000000in}}{\pgfqpoint{0.048611in}{0.000000in}}{%
\pgfpathmoveto{\pgfqpoint{0.000000in}{0.000000in}}%
\pgfpathlineto{\pgfqpoint{0.048611in}{0.000000in}}%
\pgfusepath{stroke,fill}%
}%
\begin{pgfscope}%
\pgfsys@transformshift{2.193789in}{1.239476in}%
\pgfsys@useobject{currentmarker}{}%
\end{pgfscope}%
\end{pgfscope}%
\begin{pgfscope}%
\pgftext[x=2.291011in,y=1.191649in,left,base]{\rmfamily\fontsize{10.000000}{12.000000}\selectfont \(\displaystyle 15\)}%
\end{pgfscope}%
\begin{pgfscope}%
\pgfsetbuttcap%
\pgfsetroundjoin%
\definecolor{currentfill}{rgb}{0.000000,0.000000,0.000000}%
\pgfsetfillcolor{currentfill}%
\pgfsetlinewidth{0.803000pt}%
\definecolor{currentstroke}{rgb}{0.000000,0.000000,0.000000}%
\pgfsetstrokecolor{currentstroke}%
\pgfsetdash{}{0pt}%
\pgfsys@defobject{currentmarker}{\pgfqpoint{0.000000in}{0.000000in}}{\pgfqpoint{0.048611in}{0.000000in}}{%
\pgfpathmoveto{\pgfqpoint{0.000000in}{0.000000in}}%
\pgfpathlineto{\pgfqpoint{0.048611in}{0.000000in}}%
\pgfusepath{stroke,fill}%
}%
\begin{pgfscope}%
\pgfsys@transformshift{2.193789in}{1.546009in}%
\pgfsys@useobject{currentmarker}{}%
\end{pgfscope}%
\end{pgfscope}%
\begin{pgfscope}%
\pgftext[x=2.291011in,y=1.498182in,left,base]{\rmfamily\fontsize{10.000000}{12.000000}\selectfont \(\displaystyle 20\)}%
\end{pgfscope}%
\begin{pgfscope}%
\pgfsetbuttcap%
\pgfsetroundjoin%
\definecolor{currentfill}{rgb}{0.000000,0.000000,0.000000}%
\pgfsetfillcolor{currentfill}%
\pgfsetlinewidth{0.803000pt}%
\definecolor{currentstroke}{rgb}{0.000000,0.000000,0.000000}%
\pgfsetstrokecolor{currentstroke}%
\pgfsetdash{}{0pt}%
\pgfsys@defobject{currentmarker}{\pgfqpoint{0.000000in}{0.000000in}}{\pgfqpoint{0.048611in}{0.000000in}}{%
\pgfpathmoveto{\pgfqpoint{0.000000in}{0.000000in}}%
\pgfpathlineto{\pgfqpoint{0.048611in}{0.000000in}}%
\pgfusepath{stroke,fill}%
}%
\begin{pgfscope}%
\pgfsys@transformshift{2.193789in}{1.852542in}%
\pgfsys@useobject{currentmarker}{}%
\end{pgfscope}%
\end{pgfscope}%
\begin{pgfscope}%
\pgftext[x=2.291011in,y=1.804715in,left,base]{\rmfamily\fontsize{10.000000}{12.000000}\selectfont \(\displaystyle 25\)}%
\end{pgfscope}%
\begin{pgfscope}%
\pgfsetbuttcap%
\pgfsetroundjoin%
\definecolor{currentfill}{rgb}{0.000000,0.000000,0.000000}%
\pgfsetfillcolor{currentfill}%
\pgfsetlinewidth{0.803000pt}%
\definecolor{currentstroke}{rgb}{0.000000,0.000000,0.000000}%
\pgfsetstrokecolor{currentstroke}%
\pgfsetdash{}{0pt}%
\pgfsys@defobject{currentmarker}{\pgfqpoint{0.000000in}{0.000000in}}{\pgfqpoint{0.048611in}{0.000000in}}{%
\pgfpathmoveto{\pgfqpoint{0.000000in}{0.000000in}}%
\pgfpathlineto{\pgfqpoint{0.048611in}{0.000000in}}%
\pgfusepath{stroke,fill}%
}%
\begin{pgfscope}%
\pgfsys@transformshift{2.193789in}{2.159075in}%
\pgfsys@useobject{currentmarker}{}%
\end{pgfscope}%
\end{pgfscope}%
\begin{pgfscope}%
\pgftext[x=2.291011in,y=2.111248in,left,base]{\rmfamily\fontsize{10.000000}{12.000000}\selectfont \(\displaystyle 30\)}%
\end{pgfscope}%
\begin{pgfscope}%
\pgfsetbuttcap%
\pgfsetroundjoin%
\definecolor{currentfill}{rgb}{0.000000,0.000000,0.000000}%
\pgfsetfillcolor{currentfill}%
\pgfsetlinewidth{0.803000pt}%
\definecolor{currentstroke}{rgb}{0.000000,0.000000,0.000000}%
\pgfsetstrokecolor{currentstroke}%
\pgfsetdash{}{0pt}%
\pgfsys@defobject{currentmarker}{\pgfqpoint{0.000000in}{0.000000in}}{\pgfqpoint{0.048611in}{0.000000in}}{%
\pgfpathmoveto{\pgfqpoint{0.000000in}{0.000000in}}%
\pgfpathlineto{\pgfqpoint{0.048611in}{0.000000in}}%
\pgfusepath{stroke,fill}%
}%
\begin{pgfscope}%
\pgfsys@transformshift{2.193789in}{2.465608in}%
\pgfsys@useobject{currentmarker}{}%
\end{pgfscope}%
\end{pgfscope}%
\begin{pgfscope}%
\pgftext[x=2.291011in,y=2.417781in,left,base]{\rmfamily\fontsize{10.000000}{12.000000}\selectfont \(\displaystyle 35\)}%
\end{pgfscope}%
\begin{pgfscope}%
\pgfsetbuttcap%
\pgfsetroundjoin%
\definecolor{currentfill}{rgb}{0.000000,0.000000,0.000000}%
\pgfsetfillcolor{currentfill}%
\pgfsetlinewidth{0.803000pt}%
\definecolor{currentstroke}{rgb}{0.000000,0.000000,0.000000}%
\pgfsetstrokecolor{currentstroke}%
\pgfsetdash{}{0pt}%
\pgfsys@defobject{currentmarker}{\pgfqpoint{0.000000in}{0.000000in}}{\pgfqpoint{0.048611in}{0.000000in}}{%
\pgfpathmoveto{\pgfqpoint{0.000000in}{0.000000in}}%
\pgfpathlineto{\pgfqpoint{0.048611in}{0.000000in}}%
\pgfusepath{stroke,fill}%
}%
\begin{pgfscope}%
\pgfsys@transformshift{2.193789in}{2.772141in}%
\pgfsys@useobject{currentmarker}{}%
\end{pgfscope}%
\end{pgfscope}%
\begin{pgfscope}%
\pgftext[x=2.291011in,y=2.724314in,left,base]{\rmfamily\fontsize{10.000000}{12.000000}\selectfont \(\displaystyle 40\)}%
\end{pgfscope}%
\begin{pgfscope}%
\pgfsetbuttcap%
\pgfsetmiterjoin%
\pgfsetlinewidth{0.803000pt}%
\definecolor{currentstroke}{rgb}{0.000000,0.000000,0.000000}%
\pgfsetstrokecolor{currentstroke}%
\pgfsetdash{}{0pt}%
\pgfpathmoveto{\pgfqpoint{2.063512in}{0.319877in}}%
\pgfpathlineto{\pgfqpoint{2.063512in}{0.330055in}}%
\pgfpathlineto{\pgfqpoint{2.063512in}{2.915230in}}%
\pgfpathlineto{\pgfqpoint{2.063512in}{2.925408in}}%
\pgfpathlineto{\pgfqpoint{2.193789in}{2.925408in}}%
\pgfpathlineto{\pgfqpoint{2.193789in}{2.915230in}}%
\pgfpathlineto{\pgfqpoint{2.193789in}{0.330055in}}%
\pgfpathlineto{\pgfqpoint{2.193789in}{0.319877in}}%
\pgfpathclose%
\pgfusepath{stroke}%
\end{pgfscope}%
\end{pgfpicture}%
\makeatother%
\endgroup%

    \vspace*{-0.4cm}
	\caption{500 K. Bin size $0.018e$}
	\end{subfigure}
	\quad
	\begin{subfigure}[b]{0.45\textwidth}
	\hspace*{-0.4cm}
	%% Creator: Matplotlib, PGF backend
%%
%% To include the figure in your LaTeX document, write
%%   \input{<filename>.pgf}
%%
%% Make sure the required packages are loaded in your preamble
%%   \usepackage{pgf}
%%
%% Figures using additional raster images can only be included by \input if
%% they are in the same directory as the main LaTeX file. For loading figures
%% from other directories you can use the `import` package
%%   \usepackage{import}
%% and then include the figures with
%%   \import{<path to file>}{<filename>.pgf}
%%
%% Matplotlib used the following preamble
%%   \usepackage[utf8x]{inputenc}
%%   \usepackage[T1]{fontenc}
%%
\begingroup%
\makeatletter%
\begin{pgfpicture}%
\pgfpathrectangle{\pgfpointorigin}{\pgfqpoint{2.529900in}{3.060408in}}%
\pgfusepath{use as bounding box, clip}%
\begin{pgfscope}%
\pgfsetbuttcap%
\pgfsetmiterjoin%
\definecolor{currentfill}{rgb}{1.000000,1.000000,1.000000}%
\pgfsetfillcolor{currentfill}%
\pgfsetlinewidth{0.000000pt}%
\definecolor{currentstroke}{rgb}{1.000000,1.000000,1.000000}%
\pgfsetstrokecolor{currentstroke}%
\pgfsetdash{}{0pt}%
\pgfpathmoveto{\pgfqpoint{0.000000in}{0.000000in}}%
\pgfpathlineto{\pgfqpoint{2.529900in}{0.000000in}}%
\pgfpathlineto{\pgfqpoint{2.529900in}{3.060408in}}%
\pgfpathlineto{\pgfqpoint{0.000000in}{3.060408in}}%
\pgfpathclose%
\pgfusepath{fill}%
\end{pgfscope}%
\begin{pgfscope}%
\pgfsetbuttcap%
\pgfsetmiterjoin%
\definecolor{currentfill}{rgb}{1.000000,1.000000,1.000000}%
\pgfsetfillcolor{currentfill}%
\pgfsetlinewidth{0.000000pt}%
\definecolor{currentstroke}{rgb}{0.000000,0.000000,0.000000}%
\pgfsetstrokecolor{currentstroke}%
\pgfsetstrokeopacity{0.000000}%
\pgfsetdash{}{0pt}%
\pgfpathmoveto{\pgfqpoint{0.444137in}{0.319877in}}%
\pgfpathlineto{\pgfqpoint{1.968255in}{0.319877in}}%
\pgfpathlineto{\pgfqpoint{1.968255in}{2.925408in}}%
\pgfpathlineto{\pgfqpoint{0.444137in}{2.925408in}}%
\pgfpathclose%
\pgfusepath{fill}%
\end{pgfscope}%
\begin{pgfscope}%
\pgfpathrectangle{\pgfqpoint{0.444137in}{0.319877in}}{\pgfqpoint{1.524118in}{2.605531in}} %
\pgfusepath{clip}%
\pgfsys@transformshift{0.444137in}{0.319877in}%
\pgftext[left,bottom]{\pgfimage[interpolate=true,width=1.530000in,height=2.610000in]{FerrNN_vs_dq_Ti_1000K-img0.png}}%
\end{pgfscope}%
\begin{pgfscope}%
\pgfpathrectangle{\pgfqpoint{0.444137in}{0.319877in}}{\pgfqpoint{1.524118in}{2.605531in}} %
\pgfusepath{clip}%
\pgfsetbuttcap%
\pgfsetroundjoin%
\definecolor{currentfill}{rgb}{1.000000,0.752941,0.796078}%
\pgfsetfillcolor{currentfill}%
\pgfsetlinewidth{1.003750pt}%
\definecolor{currentstroke}{rgb}{1.000000,0.752941,0.796078}%
\pgfsetstrokecolor{currentstroke}%
\pgfsetdash{}{0pt}%
\pgfpathmoveto{\pgfqpoint{0.539394in}{2.139304in}}%
\pgfpathcurveto{\pgfqpoint{0.550444in}{2.139304in}}{\pgfqpoint{0.561043in}{2.143694in}}{\pgfqpoint{0.568857in}{2.151508in}}%
\pgfpathcurveto{\pgfqpoint{0.576670in}{2.159321in}}{\pgfqpoint{0.581061in}{2.169920in}}{\pgfqpoint{0.581061in}{2.180971in}}%
\pgfpathcurveto{\pgfqpoint{0.581061in}{2.192021in}}{\pgfqpoint{0.576670in}{2.202620in}}{\pgfqpoint{0.568857in}{2.210433in}}%
\pgfpathcurveto{\pgfqpoint{0.561043in}{2.218247in}}{\pgfqpoint{0.550444in}{2.222637in}}{\pgfqpoint{0.539394in}{2.222637in}}%
\pgfpathcurveto{\pgfqpoint{0.528344in}{2.222637in}}{\pgfqpoint{0.517745in}{2.218247in}}{\pgfqpoint{0.509931in}{2.210433in}}%
\pgfpathcurveto{\pgfqpoint{0.502118in}{2.202620in}}{\pgfqpoint{0.497727in}{2.192021in}}{\pgfqpoint{0.497727in}{2.180971in}}%
\pgfpathcurveto{\pgfqpoint{0.497727in}{2.169920in}}{\pgfqpoint{0.502118in}{2.159321in}}{\pgfqpoint{0.509931in}{2.151508in}}%
\pgfpathcurveto{\pgfqpoint{0.517745in}{2.143694in}}{\pgfqpoint{0.528344in}{2.139304in}}{\pgfqpoint{0.539394in}{2.139304in}}%
\pgfpathclose%
\pgfusepath{stroke,fill}%
\end{pgfscope}%
\begin{pgfscope}%
\pgfpathrectangle{\pgfqpoint{0.444137in}{0.319877in}}{\pgfqpoint{1.524118in}{2.605531in}} %
\pgfusepath{clip}%
\pgfsetbuttcap%
\pgfsetroundjoin%
\definecolor{currentfill}{rgb}{1.000000,0.752941,0.796078}%
\pgfsetfillcolor{currentfill}%
\pgfsetlinewidth{1.003750pt}%
\definecolor{currentstroke}{rgb}{1.000000,0.752941,0.796078}%
\pgfsetstrokecolor{currentstroke}%
\pgfsetdash{}{0pt}%
\pgfpathmoveto{\pgfqpoint{0.729909in}{1.731636in}}%
\pgfpathcurveto{\pgfqpoint{0.740959in}{1.731636in}}{\pgfqpoint{0.751558in}{1.736026in}}{\pgfqpoint{0.759371in}{1.743840in}}%
\pgfpathcurveto{\pgfqpoint{0.767185in}{1.751653in}}{\pgfqpoint{0.771575in}{1.762252in}}{\pgfqpoint{0.771575in}{1.773303in}}%
\pgfpathcurveto{\pgfqpoint{0.771575in}{1.784353in}}{\pgfqpoint{0.767185in}{1.794952in}}{\pgfqpoint{0.759371in}{1.802765in}}%
\pgfpathcurveto{\pgfqpoint{0.751558in}{1.810579in}}{\pgfqpoint{0.740959in}{1.814969in}}{\pgfqpoint{0.729909in}{1.814969in}}%
\pgfpathcurveto{\pgfqpoint{0.718859in}{1.814969in}}{\pgfqpoint{0.708260in}{1.810579in}}{\pgfqpoint{0.700446in}{1.802765in}}%
\pgfpathcurveto{\pgfqpoint{0.692632in}{1.794952in}}{\pgfqpoint{0.688242in}{1.784353in}}{\pgfqpoint{0.688242in}{1.773303in}}%
\pgfpathcurveto{\pgfqpoint{0.688242in}{1.762252in}}{\pgfqpoint{0.692632in}{1.751653in}}{\pgfqpoint{0.700446in}{1.743840in}}%
\pgfpathcurveto{\pgfqpoint{0.708260in}{1.736026in}}{\pgfqpoint{0.718859in}{1.731636in}}{\pgfqpoint{0.729909in}{1.731636in}}%
\pgfpathclose%
\pgfusepath{stroke,fill}%
\end{pgfscope}%
\begin{pgfscope}%
\pgfpathrectangle{\pgfqpoint{0.444137in}{0.319877in}}{\pgfqpoint{1.524118in}{2.605531in}} %
\pgfusepath{clip}%
\pgfsetbuttcap%
\pgfsetroundjoin%
\definecolor{currentfill}{rgb}{1.000000,0.752941,0.796078}%
\pgfsetfillcolor{currentfill}%
\pgfsetlinewidth{1.003750pt}%
\definecolor{currentstroke}{rgb}{1.000000,0.752941,0.796078}%
\pgfsetstrokecolor{currentstroke}%
\pgfsetdash{}{0pt}%
\pgfpathmoveto{\pgfqpoint{0.920423in}{1.245473in}}%
\pgfpathcurveto{\pgfqpoint{0.931474in}{1.245473in}}{\pgfqpoint{0.942073in}{1.249863in}}{\pgfqpoint{0.949886in}{1.257677in}}%
\pgfpathcurveto{\pgfqpoint{0.957700in}{1.265490in}}{\pgfqpoint{0.962090in}{1.276089in}}{\pgfqpoint{0.962090in}{1.287139in}}%
\pgfpathcurveto{\pgfqpoint{0.962090in}{1.298190in}}{\pgfqpoint{0.957700in}{1.308789in}}{\pgfqpoint{0.949886in}{1.316602in}}%
\pgfpathcurveto{\pgfqpoint{0.942073in}{1.324416in}}{\pgfqpoint{0.931474in}{1.328806in}}{\pgfqpoint{0.920423in}{1.328806in}}%
\pgfpathcurveto{\pgfqpoint{0.909373in}{1.328806in}}{\pgfqpoint{0.898774in}{1.324416in}}{\pgfqpoint{0.890961in}{1.316602in}}%
\pgfpathcurveto{\pgfqpoint{0.883147in}{1.308789in}}{\pgfqpoint{0.878757in}{1.298190in}}{\pgfqpoint{0.878757in}{1.287139in}}%
\pgfpathcurveto{\pgfqpoint{0.878757in}{1.276089in}}{\pgfqpoint{0.883147in}{1.265490in}}{\pgfqpoint{0.890961in}{1.257677in}}%
\pgfpathcurveto{\pgfqpoint{0.898774in}{1.249863in}}{\pgfqpoint{0.909373in}{1.245473in}}{\pgfqpoint{0.920423in}{1.245473in}}%
\pgfpathclose%
\pgfusepath{stroke,fill}%
\end{pgfscope}%
\begin{pgfscope}%
\pgfpathrectangle{\pgfqpoint{0.444137in}{0.319877in}}{\pgfqpoint{1.524118in}{2.605531in}} %
\pgfusepath{clip}%
\pgfsetbuttcap%
\pgfsetroundjoin%
\definecolor{currentfill}{rgb}{1.000000,0.752941,0.796078}%
\pgfsetfillcolor{currentfill}%
\pgfsetlinewidth{1.003750pt}%
\definecolor{currentstroke}{rgb}{1.000000,0.752941,0.796078}%
\pgfsetstrokecolor{currentstroke}%
\pgfsetdash{}{0pt}%
\pgfpathmoveto{\pgfqpoint{1.110938in}{1.018646in}}%
\pgfpathcurveto{\pgfqpoint{1.121988in}{1.018646in}}{\pgfqpoint{1.132587in}{1.023036in}}{\pgfqpoint{1.140401in}{1.030850in}}%
\pgfpathcurveto{\pgfqpoint{1.148215in}{1.038663in}}{\pgfqpoint{1.152605in}{1.049262in}}{\pgfqpoint{1.152605in}{1.060312in}}%
\pgfpathcurveto{\pgfqpoint{1.152605in}{1.071362in}}{\pgfqpoint{1.148215in}{1.081961in}}{\pgfqpoint{1.140401in}{1.089775in}}%
\pgfpathcurveto{\pgfqpoint{1.132587in}{1.097589in}}{\pgfqpoint{1.121988in}{1.101979in}}{\pgfqpoint{1.110938in}{1.101979in}}%
\pgfpathcurveto{\pgfqpoint{1.099888in}{1.101979in}}{\pgfqpoint{1.089289in}{1.097589in}}{\pgfqpoint{1.081475in}{1.089775in}}%
\pgfpathcurveto{\pgfqpoint{1.073662in}{1.081961in}}{\pgfqpoint{1.069272in}{1.071362in}}{\pgfqpoint{1.069272in}{1.060312in}}%
\pgfpathcurveto{\pgfqpoint{1.069272in}{1.049262in}}{\pgfqpoint{1.073662in}{1.038663in}}{\pgfqpoint{1.081475in}{1.030850in}}%
\pgfpathcurveto{\pgfqpoint{1.089289in}{1.023036in}}{\pgfqpoint{1.099888in}{1.018646in}}{\pgfqpoint{1.110938in}{1.018646in}}%
\pgfpathclose%
\pgfusepath{stroke,fill}%
\end{pgfscope}%
\begin{pgfscope}%
\pgfpathrectangle{\pgfqpoint{0.444137in}{0.319877in}}{\pgfqpoint{1.524118in}{2.605531in}} %
\pgfusepath{clip}%
\pgfsetbuttcap%
\pgfsetroundjoin%
\definecolor{currentfill}{rgb}{1.000000,0.752941,0.796078}%
\pgfsetfillcolor{currentfill}%
\pgfsetlinewidth{1.003750pt}%
\definecolor{currentstroke}{rgb}{1.000000,0.752941,0.796078}%
\pgfsetstrokecolor{currentstroke}%
\pgfsetdash{}{0pt}%
\pgfpathmoveto{\pgfqpoint{1.301453in}{0.929593in}}%
\pgfpathcurveto{\pgfqpoint{1.312503in}{0.929593in}}{\pgfqpoint{1.323102in}{0.933984in}}{\pgfqpoint{1.330916in}{0.941797in}}%
\pgfpathcurveto{\pgfqpoint{1.338729in}{0.949611in}}{\pgfqpoint{1.343120in}{0.960210in}}{\pgfqpoint{1.343120in}{0.971260in}}%
\pgfpathcurveto{\pgfqpoint{1.343120in}{0.982310in}}{\pgfqpoint{1.338729in}{0.992909in}}{\pgfqpoint{1.330916in}{1.000723in}}%
\pgfpathcurveto{\pgfqpoint{1.323102in}{1.008536in}}{\pgfqpoint{1.312503in}{1.012927in}}{\pgfqpoint{1.301453in}{1.012927in}}%
\pgfpathcurveto{\pgfqpoint{1.290403in}{1.012927in}}{\pgfqpoint{1.279804in}{1.008536in}}{\pgfqpoint{1.271990in}{1.000723in}}%
\pgfpathcurveto{\pgfqpoint{1.264177in}{0.992909in}}{\pgfqpoint{1.259786in}{0.982310in}}{\pgfqpoint{1.259786in}{0.971260in}}%
\pgfpathcurveto{\pgfqpoint{1.259786in}{0.960210in}}{\pgfqpoint{1.264177in}{0.949611in}}{\pgfqpoint{1.271990in}{0.941797in}}%
\pgfpathcurveto{\pgfqpoint{1.279804in}{0.933984in}}{\pgfqpoint{1.290403in}{0.929593in}}{\pgfqpoint{1.301453in}{0.929593in}}%
\pgfpathclose%
\pgfusepath{stroke,fill}%
\end{pgfscope}%
\begin{pgfscope}%
\pgfpathrectangle{\pgfqpoint{0.444137in}{0.319877in}}{\pgfqpoint{1.524118in}{2.605531in}} %
\pgfusepath{clip}%
\pgfsetbuttcap%
\pgfsetroundjoin%
\definecolor{currentfill}{rgb}{1.000000,0.752941,0.796078}%
\pgfsetfillcolor{currentfill}%
\pgfsetlinewidth{1.003750pt}%
\definecolor{currentstroke}{rgb}{1.000000,0.752941,0.796078}%
\pgfsetstrokecolor{currentstroke}%
\pgfsetdash{}{0pt}%
\pgfpathmoveto{\pgfqpoint{1.491968in}{1.220037in}}%
\pgfpathcurveto{\pgfqpoint{1.503018in}{1.220037in}}{\pgfqpoint{1.513617in}{1.224427in}}{\pgfqpoint{1.521431in}{1.232241in}}%
\pgfpathcurveto{\pgfqpoint{1.529244in}{1.240054in}}{\pgfqpoint{1.533634in}{1.250653in}}{\pgfqpoint{1.533634in}{1.261703in}}%
\pgfpathcurveto{\pgfqpoint{1.533634in}{1.272753in}}{\pgfqpoint{1.529244in}{1.283352in}}{\pgfqpoint{1.521431in}{1.291166in}}%
\pgfpathcurveto{\pgfqpoint{1.513617in}{1.298980in}}{\pgfqpoint{1.503018in}{1.303370in}}{\pgfqpoint{1.491968in}{1.303370in}}%
\pgfpathcurveto{\pgfqpoint{1.480918in}{1.303370in}}{\pgfqpoint{1.470319in}{1.298980in}}{\pgfqpoint{1.462505in}{1.291166in}}%
\pgfpathcurveto{\pgfqpoint{1.454691in}{1.283352in}}{\pgfqpoint{1.450301in}{1.272753in}}{\pgfqpoint{1.450301in}{1.261703in}}%
\pgfpathcurveto{\pgfqpoint{1.450301in}{1.250653in}}{\pgfqpoint{1.454691in}{1.240054in}}{\pgfqpoint{1.462505in}{1.232241in}}%
\pgfpathcurveto{\pgfqpoint{1.470319in}{1.224427in}}{\pgfqpoint{1.480918in}{1.220037in}}{\pgfqpoint{1.491968in}{1.220037in}}%
\pgfpathclose%
\pgfusepath{stroke,fill}%
\end{pgfscope}%
\begin{pgfscope}%
\pgfpathrectangle{\pgfqpoint{0.444137in}{0.319877in}}{\pgfqpoint{1.524118in}{2.605531in}} %
\pgfusepath{clip}%
\pgfsetbuttcap%
\pgfsetroundjoin%
\definecolor{currentfill}{rgb}{1.000000,0.752941,0.796078}%
\pgfsetfillcolor{currentfill}%
\pgfsetlinewidth{1.003750pt}%
\definecolor{currentstroke}{rgb}{1.000000,0.752941,0.796078}%
\pgfsetstrokecolor{currentstroke}%
\pgfsetdash{}{0pt}%
\pgfpathmoveto{\pgfqpoint{1.682483in}{1.790349in}}%
\pgfpathcurveto{\pgfqpoint{1.693533in}{1.790349in}}{\pgfqpoint{1.704132in}{1.794739in}}{\pgfqpoint{1.711945in}{1.802553in}}%
\pgfpathcurveto{\pgfqpoint{1.719759in}{1.810366in}}{\pgfqpoint{1.724149in}{1.820965in}}{\pgfqpoint{1.724149in}{1.832016in}}%
\pgfpathcurveto{\pgfqpoint{1.724149in}{1.843066in}}{\pgfqpoint{1.719759in}{1.853665in}}{\pgfqpoint{1.711945in}{1.861478in}}%
\pgfpathcurveto{\pgfqpoint{1.704132in}{1.869292in}}{\pgfqpoint{1.693533in}{1.873682in}}{\pgfqpoint{1.682483in}{1.873682in}}%
\pgfpathcurveto{\pgfqpoint{1.671432in}{1.873682in}}{\pgfqpoint{1.660833in}{1.869292in}}{\pgfqpoint{1.653020in}{1.861478in}}%
\pgfpathcurveto{\pgfqpoint{1.645206in}{1.853665in}}{\pgfqpoint{1.640816in}{1.843066in}}{\pgfqpoint{1.640816in}{1.832016in}}%
\pgfpathcurveto{\pgfqpoint{1.640816in}{1.820965in}}{\pgfqpoint{1.645206in}{1.810366in}}{\pgfqpoint{1.653020in}{1.802553in}}%
\pgfpathcurveto{\pgfqpoint{1.660833in}{1.794739in}}{\pgfqpoint{1.671432in}{1.790349in}}{\pgfqpoint{1.682483in}{1.790349in}}%
\pgfpathclose%
\pgfusepath{stroke,fill}%
\end{pgfscope}%
\begin{pgfscope}%
\pgfpathrectangle{\pgfqpoint{0.444137in}{0.319877in}}{\pgfqpoint{1.524118in}{2.605531in}} %
\pgfusepath{clip}%
\pgfsetbuttcap%
\pgfsetroundjoin%
\definecolor{currentfill}{rgb}{1.000000,0.752941,0.796078}%
\pgfsetfillcolor{currentfill}%
\pgfsetlinewidth{1.003750pt}%
\definecolor{currentstroke}{rgb}{1.000000,0.752941,0.796078}%
\pgfsetstrokecolor{currentstroke}%
\pgfsetdash{}{0pt}%
\pgfpathmoveto{\pgfqpoint{1.872997in}{2.201340in}}%
\pgfpathcurveto{\pgfqpoint{1.884047in}{2.201340in}}{\pgfqpoint{1.894646in}{2.205731in}}{\pgfqpoint{1.902460in}{2.213544in}}%
\pgfpathcurveto{\pgfqpoint{1.910274in}{2.221358in}}{\pgfqpoint{1.914664in}{2.231957in}}{\pgfqpoint{1.914664in}{2.243007in}}%
\pgfpathcurveto{\pgfqpoint{1.914664in}{2.254057in}}{\pgfqpoint{1.910274in}{2.264656in}}{\pgfqpoint{1.902460in}{2.272470in}}%
\pgfpathcurveto{\pgfqpoint{1.894646in}{2.280283in}}{\pgfqpoint{1.884047in}{2.284674in}}{\pgfqpoint{1.872997in}{2.284674in}}%
\pgfpathcurveto{\pgfqpoint{1.861947in}{2.284674in}}{\pgfqpoint{1.851348in}{2.280283in}}{\pgfqpoint{1.843534in}{2.272470in}}%
\pgfpathcurveto{\pgfqpoint{1.835721in}{2.264656in}}{\pgfqpoint{1.831331in}{2.254057in}}{\pgfqpoint{1.831331in}{2.243007in}}%
\pgfpathcurveto{\pgfqpoint{1.831331in}{2.231957in}}{\pgfqpoint{1.835721in}{2.221358in}}{\pgfqpoint{1.843534in}{2.213544in}}%
\pgfpathcurveto{\pgfqpoint{1.851348in}{2.205731in}}{\pgfqpoint{1.861947in}{2.201340in}}{\pgfqpoint{1.872997in}{2.201340in}}%
\pgfpathclose%
\pgfusepath{stroke,fill}%
\end{pgfscope}%
\begin{pgfscope}%
\pgfsetbuttcap%
\pgfsetroundjoin%
\definecolor{currentfill}{rgb}{0.000000,0.000000,0.000000}%
\pgfsetfillcolor{currentfill}%
\pgfsetlinewidth{0.803000pt}%
\definecolor{currentstroke}{rgb}{0.000000,0.000000,0.000000}%
\pgfsetstrokecolor{currentstroke}%
\pgfsetdash{}{0pt}%
\pgfsys@defobject{currentmarker}{\pgfqpoint{0.000000in}{-0.048611in}}{\pgfqpoint{0.000000in}{0.000000in}}{%
\pgfpathmoveto{\pgfqpoint{0.000000in}{0.000000in}}%
\pgfpathlineto{\pgfqpoint{0.000000in}{-0.048611in}}%
\pgfusepath{stroke,fill}%
}%
\begin{pgfscope}%
\pgfsys@transformshift{0.729909in}{0.319877in}%
\pgfsys@useobject{currentmarker}{}%
\end{pgfscope}%
\end{pgfscope}%
\begin{pgfscope}%
\pgftext[x=0.729909in,y=0.222655in,,top]{\rmfamily\fontsize{10.000000}{12.000000}\selectfont \(\displaystyle -0.05\)}%
\end{pgfscope}%
\begin{pgfscope}%
\pgfsetbuttcap%
\pgfsetroundjoin%
\definecolor{currentfill}{rgb}{0.000000,0.000000,0.000000}%
\pgfsetfillcolor{currentfill}%
\pgfsetlinewidth{0.803000pt}%
\definecolor{currentstroke}{rgb}{0.000000,0.000000,0.000000}%
\pgfsetstrokecolor{currentstroke}%
\pgfsetdash{}{0pt}%
\pgfsys@defobject{currentmarker}{\pgfqpoint{0.000000in}{-0.048611in}}{\pgfqpoint{0.000000in}{0.000000in}}{%
\pgfpathmoveto{\pgfqpoint{0.000000in}{0.000000in}}%
\pgfpathlineto{\pgfqpoint{0.000000in}{-0.048611in}}%
\pgfusepath{stroke,fill}%
}%
\begin{pgfscope}%
\pgfsys@transformshift{1.206196in}{0.319877in}%
\pgfsys@useobject{currentmarker}{}%
\end{pgfscope}%
\end{pgfscope}%
\begin{pgfscope}%
\pgftext[x=1.206196in,y=0.222655in,,top]{\rmfamily\fontsize{10.000000}{12.000000}\selectfont \(\displaystyle 0.00\)}%
\end{pgfscope}%
\begin{pgfscope}%
\pgfsetbuttcap%
\pgfsetroundjoin%
\definecolor{currentfill}{rgb}{0.000000,0.000000,0.000000}%
\pgfsetfillcolor{currentfill}%
\pgfsetlinewidth{0.803000pt}%
\definecolor{currentstroke}{rgb}{0.000000,0.000000,0.000000}%
\pgfsetstrokecolor{currentstroke}%
\pgfsetdash{}{0pt}%
\pgfsys@defobject{currentmarker}{\pgfqpoint{0.000000in}{-0.048611in}}{\pgfqpoint{0.000000in}{0.000000in}}{%
\pgfpathmoveto{\pgfqpoint{0.000000in}{0.000000in}}%
\pgfpathlineto{\pgfqpoint{0.000000in}{-0.048611in}}%
\pgfusepath{stroke,fill}%
}%
\begin{pgfscope}%
\pgfsys@transformshift{1.682483in}{0.319877in}%
\pgfsys@useobject{currentmarker}{}%
\end{pgfscope}%
\end{pgfscope}%
\begin{pgfscope}%
\pgftext[x=1.682483in,y=0.222655in,,top]{\rmfamily\fontsize{10.000000}{12.000000}\selectfont \(\displaystyle 0.05\)}%
\end{pgfscope}%
\begin{pgfscope}%
\pgfsetbuttcap%
\pgfsetroundjoin%
\definecolor{currentfill}{rgb}{0.000000,0.000000,0.000000}%
\pgfsetfillcolor{currentfill}%
\pgfsetlinewidth{0.803000pt}%
\definecolor{currentstroke}{rgb}{0.000000,0.000000,0.000000}%
\pgfsetstrokecolor{currentstroke}%
\pgfsetdash{}{0pt}%
\pgfsys@defobject{currentmarker}{\pgfqpoint{-0.048611in}{0.000000in}}{\pgfqpoint{0.000000in}{0.000000in}}{%
\pgfpathmoveto{\pgfqpoint{0.000000in}{0.000000in}}%
\pgfpathlineto{\pgfqpoint{-0.048611in}{0.000000in}}%
\pgfusepath{stroke,fill}%
}%
\begin{pgfscope}%
\pgfsys@transformshift{0.444137in}{0.319877in}%
\pgfsys@useobject{currentmarker}{}%
\end{pgfscope}%
\end{pgfscope}%
\begin{pgfscope}%
\pgftext[x=0.100000in,y=0.272050in,left,base]{\rmfamily\fontsize{10.000000}{12.000000}\selectfont \(\displaystyle 0.00\)}%
\end{pgfscope}%
\begin{pgfscope}%
\pgfsetbuttcap%
\pgfsetroundjoin%
\definecolor{currentfill}{rgb}{0.000000,0.000000,0.000000}%
\pgfsetfillcolor{currentfill}%
\pgfsetlinewidth{0.803000pt}%
\definecolor{currentstroke}{rgb}{0.000000,0.000000,0.000000}%
\pgfsetstrokecolor{currentstroke}%
\pgfsetdash{}{0pt}%
\pgfsys@defobject{currentmarker}{\pgfqpoint{-0.048611in}{0.000000in}}{\pgfqpoint{0.000000in}{0.000000in}}{%
\pgfpathmoveto{\pgfqpoint{0.000000in}{0.000000in}}%
\pgfpathlineto{\pgfqpoint{-0.048611in}{0.000000in}}%
\pgfusepath{stroke,fill}%
}%
\begin{pgfscope}%
\pgfsys@transformshift{0.444137in}{0.776472in}%
\pgfsys@useobject{currentmarker}{}%
\end{pgfscope}%
\end{pgfscope}%
\begin{pgfscope}%
\pgftext[x=0.100000in,y=0.728645in,left,base]{\rmfamily\fontsize{10.000000}{12.000000}\selectfont \(\displaystyle 0.05\)}%
\end{pgfscope}%
\begin{pgfscope}%
\pgfsetbuttcap%
\pgfsetroundjoin%
\definecolor{currentfill}{rgb}{0.000000,0.000000,0.000000}%
\pgfsetfillcolor{currentfill}%
\pgfsetlinewidth{0.803000pt}%
\definecolor{currentstroke}{rgb}{0.000000,0.000000,0.000000}%
\pgfsetstrokecolor{currentstroke}%
\pgfsetdash{}{0pt}%
\pgfsys@defobject{currentmarker}{\pgfqpoint{-0.048611in}{0.000000in}}{\pgfqpoint{0.000000in}{0.000000in}}{%
\pgfpathmoveto{\pgfqpoint{0.000000in}{0.000000in}}%
\pgfpathlineto{\pgfqpoint{-0.048611in}{0.000000in}}%
\pgfusepath{stroke,fill}%
}%
\begin{pgfscope}%
\pgfsys@transformshift{0.444137in}{1.233067in}%
\pgfsys@useobject{currentmarker}{}%
\end{pgfscope}%
\end{pgfscope}%
\begin{pgfscope}%
\pgftext[x=0.100000in,y=1.185240in,left,base]{\rmfamily\fontsize{10.000000}{12.000000}\selectfont \(\displaystyle 0.10\)}%
\end{pgfscope}%
\begin{pgfscope}%
\pgfsetbuttcap%
\pgfsetroundjoin%
\definecolor{currentfill}{rgb}{0.000000,0.000000,0.000000}%
\pgfsetfillcolor{currentfill}%
\pgfsetlinewidth{0.803000pt}%
\definecolor{currentstroke}{rgb}{0.000000,0.000000,0.000000}%
\pgfsetstrokecolor{currentstroke}%
\pgfsetdash{}{0pt}%
\pgfsys@defobject{currentmarker}{\pgfqpoint{-0.048611in}{0.000000in}}{\pgfqpoint{0.000000in}{0.000000in}}{%
\pgfpathmoveto{\pgfqpoint{0.000000in}{0.000000in}}%
\pgfpathlineto{\pgfqpoint{-0.048611in}{0.000000in}}%
\pgfusepath{stroke,fill}%
}%
\begin{pgfscope}%
\pgfsys@transformshift{0.444137in}{1.689662in}%
\pgfsys@useobject{currentmarker}{}%
\end{pgfscope}%
\end{pgfscope}%
\begin{pgfscope}%
\pgftext[x=0.100000in,y=1.641835in,left,base]{\rmfamily\fontsize{10.000000}{12.000000}\selectfont \(\displaystyle 0.15\)}%
\end{pgfscope}%
\begin{pgfscope}%
\pgfsetbuttcap%
\pgfsetroundjoin%
\definecolor{currentfill}{rgb}{0.000000,0.000000,0.000000}%
\pgfsetfillcolor{currentfill}%
\pgfsetlinewidth{0.803000pt}%
\definecolor{currentstroke}{rgb}{0.000000,0.000000,0.000000}%
\pgfsetstrokecolor{currentstroke}%
\pgfsetdash{}{0pt}%
\pgfsys@defobject{currentmarker}{\pgfqpoint{-0.048611in}{0.000000in}}{\pgfqpoint{0.000000in}{0.000000in}}{%
\pgfpathmoveto{\pgfqpoint{0.000000in}{0.000000in}}%
\pgfpathlineto{\pgfqpoint{-0.048611in}{0.000000in}}%
\pgfusepath{stroke,fill}%
}%
\begin{pgfscope}%
\pgfsys@transformshift{0.444137in}{2.146257in}%
\pgfsys@useobject{currentmarker}{}%
\end{pgfscope}%
\end{pgfscope}%
\begin{pgfscope}%
\pgftext[x=0.100000in,y=2.098430in,left,base]{\rmfamily\fontsize{10.000000}{12.000000}\selectfont \(\displaystyle 0.20\)}%
\end{pgfscope}%
\begin{pgfscope}%
\pgfsetbuttcap%
\pgfsetroundjoin%
\definecolor{currentfill}{rgb}{0.000000,0.000000,0.000000}%
\pgfsetfillcolor{currentfill}%
\pgfsetlinewidth{0.803000pt}%
\definecolor{currentstroke}{rgb}{0.000000,0.000000,0.000000}%
\pgfsetstrokecolor{currentstroke}%
\pgfsetdash{}{0pt}%
\pgfsys@defobject{currentmarker}{\pgfqpoint{-0.048611in}{0.000000in}}{\pgfqpoint{0.000000in}{0.000000in}}{%
\pgfpathmoveto{\pgfqpoint{0.000000in}{0.000000in}}%
\pgfpathlineto{\pgfqpoint{-0.048611in}{0.000000in}}%
\pgfusepath{stroke,fill}%
}%
\begin{pgfscope}%
\pgfsys@transformshift{0.444137in}{2.602852in}%
\pgfsys@useobject{currentmarker}{}%
\end{pgfscope}%
\end{pgfscope}%
\begin{pgfscope}%
\pgftext[x=0.100000in,y=2.555025in,left,base]{\rmfamily\fontsize{10.000000}{12.000000}\selectfont \(\displaystyle 0.25\)}%
\end{pgfscope}%
\begin{pgfscope}%
\pgfsetrectcap%
\pgfsetmiterjoin%
\pgfsetlinewidth{0.803000pt}%
\definecolor{currentstroke}{rgb}{0.000000,0.000000,0.000000}%
\pgfsetstrokecolor{currentstroke}%
\pgfsetdash{}{0pt}%
\pgfpathmoveto{\pgfqpoint{0.444137in}{0.319877in}}%
\pgfpathlineto{\pgfqpoint{0.444137in}{2.925408in}}%
\pgfusepath{stroke}%
\end{pgfscope}%
\begin{pgfscope}%
\pgfsetrectcap%
\pgfsetmiterjoin%
\pgfsetlinewidth{0.803000pt}%
\definecolor{currentstroke}{rgb}{0.000000,0.000000,0.000000}%
\pgfsetstrokecolor{currentstroke}%
\pgfsetdash{}{0pt}%
\pgfpathmoveto{\pgfqpoint{1.968255in}{0.319877in}}%
\pgfpathlineto{\pgfqpoint{1.968255in}{2.925408in}}%
\pgfusepath{stroke}%
\end{pgfscope}%
\begin{pgfscope}%
\pgfsetrectcap%
\pgfsetmiterjoin%
\pgfsetlinewidth{0.803000pt}%
\definecolor{currentstroke}{rgb}{0.000000,0.000000,0.000000}%
\pgfsetstrokecolor{currentstroke}%
\pgfsetdash{}{0pt}%
\pgfpathmoveto{\pgfqpoint{0.444137in}{0.319877in}}%
\pgfpathlineto{\pgfqpoint{1.968255in}{0.319877in}}%
\pgfusepath{stroke}%
\end{pgfscope}%
\begin{pgfscope}%
\pgfsetrectcap%
\pgfsetmiterjoin%
\pgfsetlinewidth{0.803000pt}%
\definecolor{currentstroke}{rgb}{0.000000,0.000000,0.000000}%
\pgfsetstrokecolor{currentstroke}%
\pgfsetdash{}{0pt}%
\pgfpathmoveto{\pgfqpoint{0.444137in}{2.925408in}}%
\pgfpathlineto{\pgfqpoint{1.968255in}{2.925408in}}%
\pgfusepath{stroke}%
\end{pgfscope}%
\begin{pgfscope}%
\pgfpathrectangle{\pgfqpoint{2.063512in}{0.319877in}}{\pgfqpoint{0.130277in}{2.605531in}} %
\pgfusepath{clip}%
\pgfsetbuttcap%
\pgfsetmiterjoin%
\definecolor{currentfill}{rgb}{1.000000,1.000000,1.000000}%
\pgfsetfillcolor{currentfill}%
\pgfsetlinewidth{0.010037pt}%
\definecolor{currentstroke}{rgb}{1.000000,1.000000,1.000000}%
\pgfsetstrokecolor{currentstroke}%
\pgfsetdash{}{0pt}%
\pgfpathmoveto{\pgfqpoint{2.063512in}{0.319877in}}%
\pgfpathlineto{\pgfqpoint{2.063512in}{0.330055in}}%
\pgfpathlineto{\pgfqpoint{2.063512in}{2.915230in}}%
\pgfpathlineto{\pgfqpoint{2.063512in}{2.925408in}}%
\pgfpathlineto{\pgfqpoint{2.193789in}{2.925408in}}%
\pgfpathlineto{\pgfqpoint{2.193789in}{2.915230in}}%
\pgfpathlineto{\pgfqpoint{2.193789in}{0.330055in}}%
\pgfpathlineto{\pgfqpoint{2.193789in}{0.319877in}}%
\pgfpathclose%
\pgfusepath{stroke,fill}%
\end{pgfscope}%
\begin{pgfscope}%
\pgfsys@transformshift{2.060000in}{0.320408in}%
\pgftext[left,bottom]{\pgfimage[interpolate=true,width=0.130000in,height=2.610000in]{FerrNN_vs_dq_Ti_1000K-img1.png}}%
\end{pgfscope}%
\begin{pgfscope}%
\pgfsetbuttcap%
\pgfsetroundjoin%
\definecolor{currentfill}{rgb}{0.000000,0.000000,0.000000}%
\pgfsetfillcolor{currentfill}%
\pgfsetlinewidth{0.803000pt}%
\definecolor{currentstroke}{rgb}{0.000000,0.000000,0.000000}%
\pgfsetstrokecolor{currentstroke}%
\pgfsetdash{}{0pt}%
\pgfsys@defobject{currentmarker}{\pgfqpoint{0.000000in}{0.000000in}}{\pgfqpoint{0.048611in}{0.000000in}}{%
\pgfpathmoveto{\pgfqpoint{0.000000in}{0.000000in}}%
\pgfpathlineto{\pgfqpoint{0.048611in}{0.000000in}}%
\pgfusepath{stroke,fill}%
}%
\begin{pgfscope}%
\pgfsys@transformshift{2.193789in}{0.319877in}%
\pgfsys@useobject{currentmarker}{}%
\end{pgfscope}%
\end{pgfscope}%
\begin{pgfscope}%
\pgftext[x=2.291011in,y=0.272050in,left,base]{\rmfamily\fontsize{10.000000}{12.000000}\selectfont \(\displaystyle 0\)}%
\end{pgfscope}%
\begin{pgfscope}%
\pgfsetbuttcap%
\pgfsetroundjoin%
\definecolor{currentfill}{rgb}{0.000000,0.000000,0.000000}%
\pgfsetfillcolor{currentfill}%
\pgfsetlinewidth{0.803000pt}%
\definecolor{currentstroke}{rgb}{0.000000,0.000000,0.000000}%
\pgfsetstrokecolor{currentstroke}%
\pgfsetdash{}{0pt}%
\pgfsys@defobject{currentmarker}{\pgfqpoint{0.000000in}{0.000000in}}{\pgfqpoint{0.048611in}{0.000000in}}{%
\pgfpathmoveto{\pgfqpoint{0.000000in}{0.000000in}}%
\pgfpathlineto{\pgfqpoint{0.048611in}{0.000000in}}%
\pgfusepath{stroke,fill}%
}%
\begin{pgfscope}%
\pgfsys@transformshift{2.193789in}{0.626410in}%
\pgfsys@useobject{currentmarker}{}%
\end{pgfscope}%
\end{pgfscope}%
\begin{pgfscope}%
\pgftext[x=2.291011in,y=0.578583in,left,base]{\rmfamily\fontsize{10.000000}{12.000000}\selectfont \(\displaystyle 5\)}%
\end{pgfscope}%
\begin{pgfscope}%
\pgfsetbuttcap%
\pgfsetroundjoin%
\definecolor{currentfill}{rgb}{0.000000,0.000000,0.000000}%
\pgfsetfillcolor{currentfill}%
\pgfsetlinewidth{0.803000pt}%
\definecolor{currentstroke}{rgb}{0.000000,0.000000,0.000000}%
\pgfsetstrokecolor{currentstroke}%
\pgfsetdash{}{0pt}%
\pgfsys@defobject{currentmarker}{\pgfqpoint{0.000000in}{0.000000in}}{\pgfqpoint{0.048611in}{0.000000in}}{%
\pgfpathmoveto{\pgfqpoint{0.000000in}{0.000000in}}%
\pgfpathlineto{\pgfqpoint{0.048611in}{0.000000in}}%
\pgfusepath{stroke,fill}%
}%
\begin{pgfscope}%
\pgfsys@transformshift{2.193789in}{0.932943in}%
\pgfsys@useobject{currentmarker}{}%
\end{pgfscope}%
\end{pgfscope}%
\begin{pgfscope}%
\pgftext[x=2.291011in,y=0.885116in,left,base]{\rmfamily\fontsize{10.000000}{12.000000}\selectfont \(\displaystyle 10\)}%
\end{pgfscope}%
\begin{pgfscope}%
\pgfsetbuttcap%
\pgfsetroundjoin%
\definecolor{currentfill}{rgb}{0.000000,0.000000,0.000000}%
\pgfsetfillcolor{currentfill}%
\pgfsetlinewidth{0.803000pt}%
\definecolor{currentstroke}{rgb}{0.000000,0.000000,0.000000}%
\pgfsetstrokecolor{currentstroke}%
\pgfsetdash{}{0pt}%
\pgfsys@defobject{currentmarker}{\pgfqpoint{0.000000in}{0.000000in}}{\pgfqpoint{0.048611in}{0.000000in}}{%
\pgfpathmoveto{\pgfqpoint{0.000000in}{0.000000in}}%
\pgfpathlineto{\pgfqpoint{0.048611in}{0.000000in}}%
\pgfusepath{stroke,fill}%
}%
\begin{pgfscope}%
\pgfsys@transformshift{2.193789in}{1.239476in}%
\pgfsys@useobject{currentmarker}{}%
\end{pgfscope}%
\end{pgfscope}%
\begin{pgfscope}%
\pgftext[x=2.291011in,y=1.191649in,left,base]{\rmfamily\fontsize{10.000000}{12.000000}\selectfont \(\displaystyle 15\)}%
\end{pgfscope}%
\begin{pgfscope}%
\pgfsetbuttcap%
\pgfsetroundjoin%
\definecolor{currentfill}{rgb}{0.000000,0.000000,0.000000}%
\pgfsetfillcolor{currentfill}%
\pgfsetlinewidth{0.803000pt}%
\definecolor{currentstroke}{rgb}{0.000000,0.000000,0.000000}%
\pgfsetstrokecolor{currentstroke}%
\pgfsetdash{}{0pt}%
\pgfsys@defobject{currentmarker}{\pgfqpoint{0.000000in}{0.000000in}}{\pgfqpoint{0.048611in}{0.000000in}}{%
\pgfpathmoveto{\pgfqpoint{0.000000in}{0.000000in}}%
\pgfpathlineto{\pgfqpoint{0.048611in}{0.000000in}}%
\pgfusepath{stroke,fill}%
}%
\begin{pgfscope}%
\pgfsys@transformshift{2.193789in}{1.546009in}%
\pgfsys@useobject{currentmarker}{}%
\end{pgfscope}%
\end{pgfscope}%
\begin{pgfscope}%
\pgftext[x=2.291011in,y=1.498182in,left,base]{\rmfamily\fontsize{10.000000}{12.000000}\selectfont \(\displaystyle 20\)}%
\end{pgfscope}%
\begin{pgfscope}%
\pgfsetbuttcap%
\pgfsetroundjoin%
\definecolor{currentfill}{rgb}{0.000000,0.000000,0.000000}%
\pgfsetfillcolor{currentfill}%
\pgfsetlinewidth{0.803000pt}%
\definecolor{currentstroke}{rgb}{0.000000,0.000000,0.000000}%
\pgfsetstrokecolor{currentstroke}%
\pgfsetdash{}{0pt}%
\pgfsys@defobject{currentmarker}{\pgfqpoint{0.000000in}{0.000000in}}{\pgfqpoint{0.048611in}{0.000000in}}{%
\pgfpathmoveto{\pgfqpoint{0.000000in}{0.000000in}}%
\pgfpathlineto{\pgfqpoint{0.048611in}{0.000000in}}%
\pgfusepath{stroke,fill}%
}%
\begin{pgfscope}%
\pgfsys@transformshift{2.193789in}{1.852542in}%
\pgfsys@useobject{currentmarker}{}%
\end{pgfscope}%
\end{pgfscope}%
\begin{pgfscope}%
\pgftext[x=2.291011in,y=1.804715in,left,base]{\rmfamily\fontsize{10.000000}{12.000000}\selectfont \(\displaystyle 25\)}%
\end{pgfscope}%
\begin{pgfscope}%
\pgfsetbuttcap%
\pgfsetroundjoin%
\definecolor{currentfill}{rgb}{0.000000,0.000000,0.000000}%
\pgfsetfillcolor{currentfill}%
\pgfsetlinewidth{0.803000pt}%
\definecolor{currentstroke}{rgb}{0.000000,0.000000,0.000000}%
\pgfsetstrokecolor{currentstroke}%
\pgfsetdash{}{0pt}%
\pgfsys@defobject{currentmarker}{\pgfqpoint{0.000000in}{0.000000in}}{\pgfqpoint{0.048611in}{0.000000in}}{%
\pgfpathmoveto{\pgfqpoint{0.000000in}{0.000000in}}%
\pgfpathlineto{\pgfqpoint{0.048611in}{0.000000in}}%
\pgfusepath{stroke,fill}%
}%
\begin{pgfscope}%
\pgfsys@transformshift{2.193789in}{2.159075in}%
\pgfsys@useobject{currentmarker}{}%
\end{pgfscope}%
\end{pgfscope}%
\begin{pgfscope}%
\pgftext[x=2.291011in,y=2.111248in,left,base]{\rmfamily\fontsize{10.000000}{12.000000}\selectfont \(\displaystyle 30\)}%
\end{pgfscope}%
\begin{pgfscope}%
\pgfsetbuttcap%
\pgfsetroundjoin%
\definecolor{currentfill}{rgb}{0.000000,0.000000,0.000000}%
\pgfsetfillcolor{currentfill}%
\pgfsetlinewidth{0.803000pt}%
\definecolor{currentstroke}{rgb}{0.000000,0.000000,0.000000}%
\pgfsetstrokecolor{currentstroke}%
\pgfsetdash{}{0pt}%
\pgfsys@defobject{currentmarker}{\pgfqpoint{0.000000in}{0.000000in}}{\pgfqpoint{0.048611in}{0.000000in}}{%
\pgfpathmoveto{\pgfqpoint{0.000000in}{0.000000in}}%
\pgfpathlineto{\pgfqpoint{0.048611in}{0.000000in}}%
\pgfusepath{stroke,fill}%
}%
\begin{pgfscope}%
\pgfsys@transformshift{2.193789in}{2.465608in}%
\pgfsys@useobject{currentmarker}{}%
\end{pgfscope}%
\end{pgfscope}%
\begin{pgfscope}%
\pgftext[x=2.291011in,y=2.417781in,left,base]{\rmfamily\fontsize{10.000000}{12.000000}\selectfont \(\displaystyle 35\)}%
\end{pgfscope}%
\begin{pgfscope}%
\pgfsetbuttcap%
\pgfsetroundjoin%
\definecolor{currentfill}{rgb}{0.000000,0.000000,0.000000}%
\pgfsetfillcolor{currentfill}%
\pgfsetlinewidth{0.803000pt}%
\definecolor{currentstroke}{rgb}{0.000000,0.000000,0.000000}%
\pgfsetstrokecolor{currentstroke}%
\pgfsetdash{}{0pt}%
\pgfsys@defobject{currentmarker}{\pgfqpoint{0.000000in}{0.000000in}}{\pgfqpoint{0.048611in}{0.000000in}}{%
\pgfpathmoveto{\pgfqpoint{0.000000in}{0.000000in}}%
\pgfpathlineto{\pgfqpoint{0.048611in}{0.000000in}}%
\pgfusepath{stroke,fill}%
}%
\begin{pgfscope}%
\pgfsys@transformshift{2.193789in}{2.772141in}%
\pgfsys@useobject{currentmarker}{}%
\end{pgfscope}%
\end{pgfscope}%
\begin{pgfscope}%
\pgftext[x=2.291011in,y=2.724314in,left,base]{\rmfamily\fontsize{10.000000}{12.000000}\selectfont \(\displaystyle 40\)}%
\end{pgfscope}%
\begin{pgfscope}%
\pgfsetbuttcap%
\pgfsetmiterjoin%
\pgfsetlinewidth{0.803000pt}%
\definecolor{currentstroke}{rgb}{0.000000,0.000000,0.000000}%
\pgfsetstrokecolor{currentstroke}%
\pgfsetdash{}{0pt}%
\pgfpathmoveto{\pgfqpoint{2.063512in}{0.319877in}}%
\pgfpathlineto{\pgfqpoint{2.063512in}{0.330055in}}%
\pgfpathlineto{\pgfqpoint{2.063512in}{2.915230in}}%
\pgfpathlineto{\pgfqpoint{2.063512in}{2.925408in}}%
\pgfpathlineto{\pgfqpoint{2.193789in}{2.925408in}}%
\pgfpathlineto{\pgfqpoint{2.193789in}{2.915230in}}%
\pgfpathlineto{\pgfqpoint{2.193789in}{0.330055in}}%
\pgfpathlineto{\pgfqpoint{2.193789in}{0.319877in}}%
\pgfpathclose%
\pgfusepath{stroke}%
\end{pgfscope}%
\end{pgfpicture}%
\makeatother%
\endgroup%

    \vspace*{-0.4cm}
	\caption{1000 K. Bin size $0.020e$}
	\end{subfigure}
\caption{Forces on nearest neighbours of Ti (Oxygens) vs change in Ti charge}
\label{on_site_FerrNN_vs_dq}
\end{figure}

3) Figure \ref{on_site_RnnNorm_vs_dq} then examines whether the change in charge on a Ti ion takes place in response to the change in its nearest neighbour structure; namely it looks at the average distance to its neighbours as a measure of how small or large the Oxygen shell around a Ti ions is. The exact value represented by the y-axis is
\begin{align}
y_{i,I} \equiv\underbrace{ (1/6)\sum_{s\in \text{NN}_i}\sqrt{\sum_{\alpha = x,y,z}\left(R_{s,I}^{\alpha}-R_{i,I}^{\alpha}\right)^2}}_{\equiv\bar{\text{R}}_{i,I}^{\text{NN}}}
\end{align}\label{RnnAve}
which is simply the average distance to the nearest neighbours of Ti ion $i$. 

\begin{figure}[h!]
\centering
	\begin{subfigure}[b]{0.45\textwidth}
	\hspace*{-0.4cm}
	%% Creator: Matplotlib, PGF backend
%%
%% To include the figure in your LaTeX document, write
%%   \input{<filename>.pgf}
%%
%% Make sure the required packages are loaded in your preamble
%%   \usepackage{pgf}
%%
%% Figures using additional raster images can only be included by \input if
%% they are in the same directory as the main LaTeX file. For loading figures
%% from other directories you can use the `import` package
%%   \usepackage{import}
%% and then include the figures with
%%   \import{<path to file>}{<filename>.pgf}
%%
%% Matplotlib used the following preamble
%%   \usepackage[utf8x]{inputenc}
%%   \usepackage[T1]{fontenc}
%%
\begingroup%
\makeatletter%
\begin{pgfpicture}%
\pgfpathrectangle{\pgfpointorigin}{\pgfqpoint{2.519483in}{3.060408in}}%
\pgfusepath{use as bounding box, clip}%
\begin{pgfscope}%
\pgfsetbuttcap%
\pgfsetmiterjoin%
\definecolor{currentfill}{rgb}{1.000000,1.000000,1.000000}%
\pgfsetfillcolor{currentfill}%
\pgfsetlinewidth{0.000000pt}%
\definecolor{currentstroke}{rgb}{1.000000,1.000000,1.000000}%
\pgfsetstrokecolor{currentstroke}%
\pgfsetdash{}{0pt}%
\pgfpathmoveto{\pgfqpoint{0.000000in}{0.000000in}}%
\pgfpathlineto{\pgfqpoint{2.519483in}{0.000000in}}%
\pgfpathlineto{\pgfqpoint{2.519483in}{3.060408in}}%
\pgfpathlineto{\pgfqpoint{0.000000in}{3.060408in}}%
\pgfpathclose%
\pgfusepath{fill}%
\end{pgfscope}%
\begin{pgfscope}%
\pgfsetbuttcap%
\pgfsetmiterjoin%
\definecolor{currentfill}{rgb}{1.000000,1.000000,1.000000}%
\pgfsetfillcolor{currentfill}%
\pgfsetlinewidth{0.000000pt}%
\definecolor{currentstroke}{rgb}{0.000000,0.000000,0.000000}%
\pgfsetstrokecolor{currentstroke}%
\pgfsetstrokeopacity{0.000000}%
\pgfsetdash{}{0pt}%
\pgfpathmoveto{\pgfqpoint{0.374692in}{0.319877in}}%
\pgfpathlineto{\pgfqpoint{1.954366in}{0.319877in}}%
\pgfpathlineto{\pgfqpoint{1.954366in}{2.925408in}}%
\pgfpathlineto{\pgfqpoint{0.374692in}{2.925408in}}%
\pgfpathclose%
\pgfusepath{fill}%
\end{pgfscope}%
\begin{pgfscope}%
\pgfpathrectangle{\pgfqpoint{0.374692in}{0.319877in}}{\pgfqpoint{1.579674in}{2.605531in}} %
\pgfusepath{clip}%
\pgfsys@transformshift{0.374692in}{0.319877in}%
\pgftext[left,bottom]{\pgfimage[interpolate=true,width=1.580000in,height=2.610000in]{RnnNorm_vs_dq_Ti_100K-img0.png}}%
\end{pgfscope}%
\begin{pgfscope}%
\pgfpathrectangle{\pgfqpoint{0.374692in}{0.319877in}}{\pgfqpoint{1.579674in}{2.605531in}} %
\pgfusepath{clip}%
\pgfsetbuttcap%
\pgfsetroundjoin%
\definecolor{currentfill}{rgb}{1.000000,0.752941,0.796078}%
\pgfsetfillcolor{currentfill}%
\pgfsetlinewidth{1.003750pt}%
\definecolor{currentstroke}{rgb}{1.000000,0.752941,0.796078}%
\pgfsetstrokecolor{currentstroke}%
\pgfsetdash{}{0pt}%
\pgfpathmoveto{\pgfqpoint{0.953906in}{1.480763in}}%
\pgfpathcurveto{\pgfqpoint{0.964956in}{1.480763in}}{\pgfqpoint{0.975555in}{1.485153in}}{\pgfqpoint{0.983368in}{1.492967in}}%
\pgfpathcurveto{\pgfqpoint{0.991182in}{1.500781in}}{\pgfqpoint{0.995572in}{1.511380in}}{\pgfqpoint{0.995572in}{1.522430in}}%
\pgfpathcurveto{\pgfqpoint{0.995572in}{1.533480in}}{\pgfqpoint{0.991182in}{1.544079in}}{\pgfqpoint{0.983368in}{1.551893in}}%
\pgfpathcurveto{\pgfqpoint{0.975555in}{1.559706in}}{\pgfqpoint{0.964956in}{1.564097in}}{\pgfqpoint{0.953906in}{1.564097in}}%
\pgfpathcurveto{\pgfqpoint{0.942856in}{1.564097in}}{\pgfqpoint{0.932257in}{1.559706in}}{\pgfqpoint{0.924443in}{1.551893in}}%
\pgfpathcurveto{\pgfqpoint{0.916629in}{1.544079in}}{\pgfqpoint{0.912239in}{1.533480in}}{\pgfqpoint{0.912239in}{1.522430in}}%
\pgfpathcurveto{\pgfqpoint{0.912239in}{1.511380in}}{\pgfqpoint{0.916629in}{1.500781in}}{\pgfqpoint{0.924443in}{1.492967in}}%
\pgfpathcurveto{\pgfqpoint{0.932257in}{1.485153in}}{\pgfqpoint{0.942856in}{1.480763in}}{\pgfqpoint{0.953906in}{1.480763in}}%
\pgfpathclose%
\pgfusepath{stroke,fill}%
\end{pgfscope}%
\begin{pgfscope}%
\pgfpathrectangle{\pgfqpoint{0.374692in}{0.319877in}}{\pgfqpoint{1.579674in}{2.605531in}} %
\pgfusepath{clip}%
\pgfsetbuttcap%
\pgfsetroundjoin%
\definecolor{currentfill}{rgb}{1.000000,0.752941,0.796078}%
\pgfsetfillcolor{currentfill}%
\pgfsetlinewidth{1.003750pt}%
\definecolor{currentstroke}{rgb}{1.000000,0.752941,0.796078}%
\pgfsetstrokecolor{currentstroke}%
\pgfsetdash{}{0pt}%
\pgfpathmoveto{\pgfqpoint{1.059217in}{1.422018in}}%
\pgfpathcurveto{\pgfqpoint{1.070267in}{1.422018in}}{\pgfqpoint{1.080866in}{1.426408in}}{\pgfqpoint{1.088680in}{1.434222in}}%
\pgfpathcurveto{\pgfqpoint{1.096494in}{1.442035in}}{\pgfqpoint{1.100884in}{1.452634in}}{\pgfqpoint{1.100884in}{1.463685in}}%
\pgfpathcurveto{\pgfqpoint{1.100884in}{1.474735in}}{\pgfqpoint{1.096494in}{1.485334in}}{\pgfqpoint{1.088680in}{1.493147in}}%
\pgfpathcurveto{\pgfqpoint{1.080866in}{1.500961in}}{\pgfqpoint{1.070267in}{1.505351in}}{\pgfqpoint{1.059217in}{1.505351in}}%
\pgfpathcurveto{\pgfqpoint{1.048167in}{1.505351in}}{\pgfqpoint{1.037568in}{1.500961in}}{\pgfqpoint{1.029754in}{1.493147in}}%
\pgfpathcurveto{\pgfqpoint{1.021941in}{1.485334in}}{\pgfqpoint{1.017551in}{1.474735in}}{\pgfqpoint{1.017551in}{1.463685in}}%
\pgfpathcurveto{\pgfqpoint{1.017551in}{1.452634in}}{\pgfqpoint{1.021941in}{1.442035in}}{\pgfqpoint{1.029754in}{1.434222in}}%
\pgfpathcurveto{\pgfqpoint{1.037568in}{1.426408in}}{\pgfqpoint{1.048167in}{1.422018in}}{\pgfqpoint{1.059217in}{1.422018in}}%
\pgfpathclose%
\pgfusepath{stroke,fill}%
\end{pgfscope}%
\begin{pgfscope}%
\pgfpathrectangle{\pgfqpoint{0.374692in}{0.319877in}}{\pgfqpoint{1.579674in}{2.605531in}} %
\pgfusepath{clip}%
\pgfsetbuttcap%
\pgfsetroundjoin%
\definecolor{currentfill}{rgb}{1.000000,0.752941,0.796078}%
\pgfsetfillcolor{currentfill}%
\pgfsetlinewidth{1.003750pt}%
\definecolor{currentstroke}{rgb}{1.000000,0.752941,0.796078}%
\pgfsetstrokecolor{currentstroke}%
\pgfsetdash{}{0pt}%
\pgfpathmoveto{\pgfqpoint{1.164529in}{1.379582in}}%
\pgfpathcurveto{\pgfqpoint{1.175579in}{1.379582in}}{\pgfqpoint{1.186178in}{1.383973in}}{\pgfqpoint{1.193992in}{1.391786in}}%
\pgfpathcurveto{\pgfqpoint{1.201805in}{1.399600in}}{\pgfqpoint{1.206196in}{1.410199in}}{\pgfqpoint{1.206196in}{1.421249in}}%
\pgfpathcurveto{\pgfqpoint{1.206196in}{1.432299in}}{\pgfqpoint{1.201805in}{1.442898in}}{\pgfqpoint{1.193992in}{1.450712in}}%
\pgfpathcurveto{\pgfqpoint{1.186178in}{1.458525in}}{\pgfqpoint{1.175579in}{1.462916in}}{\pgfqpoint{1.164529in}{1.462916in}}%
\pgfpathcurveto{\pgfqpoint{1.153479in}{1.462916in}}{\pgfqpoint{1.142880in}{1.458525in}}{\pgfqpoint{1.135066in}{1.450712in}}%
\pgfpathcurveto{\pgfqpoint{1.127252in}{1.442898in}}{\pgfqpoint{1.122862in}{1.432299in}}{\pgfqpoint{1.122862in}{1.421249in}}%
\pgfpathcurveto{\pgfqpoint{1.122862in}{1.410199in}}{\pgfqpoint{1.127252in}{1.399600in}}{\pgfqpoint{1.135066in}{1.391786in}}%
\pgfpathcurveto{\pgfqpoint{1.142880in}{1.383973in}}{\pgfqpoint{1.153479in}{1.379582in}}{\pgfqpoint{1.164529in}{1.379582in}}%
\pgfpathclose%
\pgfusepath{stroke,fill}%
\end{pgfscope}%
\begin{pgfscope}%
\pgfpathrectangle{\pgfqpoint{0.374692in}{0.319877in}}{\pgfqpoint{1.579674in}{2.605531in}} %
\pgfusepath{clip}%
\pgfsetbuttcap%
\pgfsetroundjoin%
\definecolor{currentfill}{rgb}{1.000000,0.752941,0.796078}%
\pgfsetfillcolor{currentfill}%
\pgfsetlinewidth{1.003750pt}%
\definecolor{currentstroke}{rgb}{1.000000,0.752941,0.796078}%
\pgfsetstrokecolor{currentstroke}%
\pgfsetdash{}{0pt}%
\pgfpathmoveto{\pgfqpoint{1.269840in}{1.319766in}}%
\pgfpathcurveto{\pgfqpoint{1.280891in}{1.319766in}}{\pgfqpoint{1.291490in}{1.324156in}}{\pgfqpoint{1.299303in}{1.331970in}}%
\pgfpathcurveto{\pgfqpoint{1.307117in}{1.339783in}}{\pgfqpoint{1.311507in}{1.350382in}}{\pgfqpoint{1.311507in}{1.361432in}}%
\pgfpathcurveto{\pgfqpoint{1.311507in}{1.372483in}}{\pgfqpoint{1.307117in}{1.383082in}}{\pgfqpoint{1.299303in}{1.390895in}}%
\pgfpathcurveto{\pgfqpoint{1.291490in}{1.398709in}}{\pgfqpoint{1.280891in}{1.403099in}}{\pgfqpoint{1.269840in}{1.403099in}}%
\pgfpathcurveto{\pgfqpoint{1.258790in}{1.403099in}}{\pgfqpoint{1.248191in}{1.398709in}}{\pgfqpoint{1.240378in}{1.390895in}}%
\pgfpathcurveto{\pgfqpoint{1.232564in}{1.383082in}}{\pgfqpoint{1.228174in}{1.372483in}}{\pgfqpoint{1.228174in}{1.361432in}}%
\pgfpathcurveto{\pgfqpoint{1.228174in}{1.350382in}}{\pgfqpoint{1.232564in}{1.339783in}}{\pgfqpoint{1.240378in}{1.331970in}}%
\pgfpathcurveto{\pgfqpoint{1.248191in}{1.324156in}}{\pgfqpoint{1.258790in}{1.319766in}}{\pgfqpoint{1.269840in}{1.319766in}}%
\pgfpathclose%
\pgfusepath{stroke,fill}%
\end{pgfscope}%
\begin{pgfscope}%
\pgfpathrectangle{\pgfqpoint{0.374692in}{0.319877in}}{\pgfqpoint{1.579674in}{2.605531in}} %
\pgfusepath{clip}%
\pgfsetbuttcap%
\pgfsetroundjoin%
\definecolor{currentfill}{rgb}{1.000000,0.752941,0.796078}%
\pgfsetfillcolor{currentfill}%
\pgfsetlinewidth{1.003750pt}%
\definecolor{currentstroke}{rgb}{1.000000,0.752941,0.796078}%
\pgfsetstrokecolor{currentstroke}%
\pgfsetdash{}{0pt}%
\pgfpathmoveto{\pgfqpoint{1.375152in}{1.280338in}}%
\pgfpathcurveto{\pgfqpoint{1.386202in}{1.280338in}}{\pgfqpoint{1.396801in}{1.284728in}}{\pgfqpoint{1.404615in}{1.292542in}}%
\pgfpathcurveto{\pgfqpoint{1.412428in}{1.300355in}}{\pgfqpoint{1.416819in}{1.310954in}}{\pgfqpoint{1.416819in}{1.322004in}}%
\pgfpathcurveto{\pgfqpoint{1.416819in}{1.333055in}}{\pgfqpoint{1.412428in}{1.343654in}}{\pgfqpoint{1.404615in}{1.351467in}}%
\pgfpathcurveto{\pgfqpoint{1.396801in}{1.359281in}}{\pgfqpoint{1.386202in}{1.363671in}}{\pgfqpoint{1.375152in}{1.363671in}}%
\pgfpathcurveto{\pgfqpoint{1.364102in}{1.363671in}}{\pgfqpoint{1.353503in}{1.359281in}}{\pgfqpoint{1.345689in}{1.351467in}}%
\pgfpathcurveto{\pgfqpoint{1.337876in}{1.343654in}}{\pgfqpoint{1.333485in}{1.333055in}}{\pgfqpoint{1.333485in}{1.322004in}}%
\pgfpathcurveto{\pgfqpoint{1.333485in}{1.310954in}}{\pgfqpoint{1.337876in}{1.300355in}}{\pgfqpoint{1.345689in}{1.292542in}}%
\pgfpathcurveto{\pgfqpoint{1.353503in}{1.284728in}}{\pgfqpoint{1.364102in}{1.280338in}}{\pgfqpoint{1.375152in}{1.280338in}}%
\pgfpathclose%
\pgfusepath{stroke,fill}%
\end{pgfscope}%
\begin{pgfscope}%
\pgfsetbuttcap%
\pgfsetroundjoin%
\definecolor{currentfill}{rgb}{0.000000,0.000000,0.000000}%
\pgfsetfillcolor{currentfill}%
\pgfsetlinewidth{0.803000pt}%
\definecolor{currentstroke}{rgb}{0.000000,0.000000,0.000000}%
\pgfsetstrokecolor{currentstroke}%
\pgfsetdash{}{0pt}%
\pgfsys@defobject{currentmarker}{\pgfqpoint{0.000000in}{-0.048611in}}{\pgfqpoint{0.000000in}{0.000000in}}{%
\pgfpathmoveto{\pgfqpoint{0.000000in}{0.000000in}}%
\pgfpathlineto{\pgfqpoint{0.000000in}{-0.048611in}}%
\pgfusepath{stroke,fill}%
}%
\begin{pgfscope}%
\pgfsys@transformshift{0.670881in}{0.319877in}%
\pgfsys@useobject{currentmarker}{}%
\end{pgfscope}%
\end{pgfscope}%
\begin{pgfscope}%
\pgftext[x=0.670881in,y=0.222655in,,top]{\rmfamily\fontsize{10.000000}{12.000000}\selectfont \(\displaystyle -0.05\)}%
\end{pgfscope}%
\begin{pgfscope}%
\pgfsetbuttcap%
\pgfsetroundjoin%
\definecolor{currentfill}{rgb}{0.000000,0.000000,0.000000}%
\pgfsetfillcolor{currentfill}%
\pgfsetlinewidth{0.803000pt}%
\definecolor{currentstroke}{rgb}{0.000000,0.000000,0.000000}%
\pgfsetstrokecolor{currentstroke}%
\pgfsetdash{}{0pt}%
\pgfsys@defobject{currentmarker}{\pgfqpoint{0.000000in}{-0.048611in}}{\pgfqpoint{0.000000in}{0.000000in}}{%
\pgfpathmoveto{\pgfqpoint{0.000000in}{0.000000in}}%
\pgfpathlineto{\pgfqpoint{0.000000in}{-0.048611in}}%
\pgfusepath{stroke,fill}%
}%
\begin{pgfscope}%
\pgfsys@transformshift{1.164529in}{0.319877in}%
\pgfsys@useobject{currentmarker}{}%
\end{pgfscope}%
\end{pgfscope}%
\begin{pgfscope}%
\pgftext[x=1.164529in,y=0.222655in,,top]{\rmfamily\fontsize{10.000000}{12.000000}\selectfont \(\displaystyle 0.00\)}%
\end{pgfscope}%
\begin{pgfscope}%
\pgfsetbuttcap%
\pgfsetroundjoin%
\definecolor{currentfill}{rgb}{0.000000,0.000000,0.000000}%
\pgfsetfillcolor{currentfill}%
\pgfsetlinewidth{0.803000pt}%
\definecolor{currentstroke}{rgb}{0.000000,0.000000,0.000000}%
\pgfsetstrokecolor{currentstroke}%
\pgfsetdash{}{0pt}%
\pgfsys@defobject{currentmarker}{\pgfqpoint{0.000000in}{-0.048611in}}{\pgfqpoint{0.000000in}{0.000000in}}{%
\pgfpathmoveto{\pgfqpoint{0.000000in}{0.000000in}}%
\pgfpathlineto{\pgfqpoint{0.000000in}{-0.048611in}}%
\pgfusepath{stroke,fill}%
}%
\begin{pgfscope}%
\pgfsys@transformshift{1.658177in}{0.319877in}%
\pgfsys@useobject{currentmarker}{}%
\end{pgfscope}%
\end{pgfscope}%
\begin{pgfscope}%
\pgftext[x=1.658177in,y=0.222655in,,top]{\rmfamily\fontsize{10.000000}{12.000000}\selectfont \(\displaystyle 0.05\)}%
\end{pgfscope}%
\begin{pgfscope}%
\pgfsetbuttcap%
\pgfsetroundjoin%
\definecolor{currentfill}{rgb}{0.000000,0.000000,0.000000}%
\pgfsetfillcolor{currentfill}%
\pgfsetlinewidth{0.803000pt}%
\definecolor{currentstroke}{rgb}{0.000000,0.000000,0.000000}%
\pgfsetstrokecolor{currentstroke}%
\pgfsetdash{}{0pt}%
\pgfsys@defobject{currentmarker}{\pgfqpoint{-0.048611in}{0.000000in}}{\pgfqpoint{0.000000in}{0.000000in}}{%
\pgfpathmoveto{\pgfqpoint{0.000000in}{0.000000in}}%
\pgfpathlineto{\pgfqpoint{-0.048611in}{0.000000in}}%
\pgfusepath{stroke,fill}%
}%
\begin{pgfscope}%
\pgfsys@transformshift{0.374692in}{0.622908in}%
\pgfsys@useobject{currentmarker}{}%
\end{pgfscope}%
\end{pgfscope}%
\begin{pgfscope}%
\pgftext[x=0.100000in,y=0.575080in,left,base]{\rmfamily\fontsize{10.000000}{12.000000}\selectfont \(\displaystyle 3.6\)}%
\end{pgfscope}%
\begin{pgfscope}%
\pgfsetbuttcap%
\pgfsetroundjoin%
\definecolor{currentfill}{rgb}{0.000000,0.000000,0.000000}%
\pgfsetfillcolor{currentfill}%
\pgfsetlinewidth{0.803000pt}%
\definecolor{currentstroke}{rgb}{0.000000,0.000000,0.000000}%
\pgfsetstrokecolor{currentstroke}%
\pgfsetdash{}{0pt}%
\pgfsys@defobject{currentmarker}{\pgfqpoint{-0.048611in}{0.000000in}}{\pgfqpoint{0.000000in}{0.000000in}}{%
\pgfpathmoveto{\pgfqpoint{0.000000in}{0.000000in}}%
\pgfpathlineto{\pgfqpoint{-0.048611in}{0.000000in}}%
\pgfusepath{stroke,fill}%
}%
\begin{pgfscope}%
\pgfsys@transformshift{0.374692in}{1.067280in}%
\pgfsys@useobject{currentmarker}{}%
\end{pgfscope}%
\end{pgfscope}%
\begin{pgfscope}%
\pgftext[x=0.100000in,y=1.019453in,left,base]{\rmfamily\fontsize{10.000000}{12.000000}\selectfont \(\displaystyle 3.7\)}%
\end{pgfscope}%
\begin{pgfscope}%
\pgfsetbuttcap%
\pgfsetroundjoin%
\definecolor{currentfill}{rgb}{0.000000,0.000000,0.000000}%
\pgfsetfillcolor{currentfill}%
\pgfsetlinewidth{0.803000pt}%
\definecolor{currentstroke}{rgb}{0.000000,0.000000,0.000000}%
\pgfsetstrokecolor{currentstroke}%
\pgfsetdash{}{0pt}%
\pgfsys@defobject{currentmarker}{\pgfqpoint{-0.048611in}{0.000000in}}{\pgfqpoint{0.000000in}{0.000000in}}{%
\pgfpathmoveto{\pgfqpoint{0.000000in}{0.000000in}}%
\pgfpathlineto{\pgfqpoint{-0.048611in}{0.000000in}}%
\pgfusepath{stroke,fill}%
}%
\begin{pgfscope}%
\pgfsys@transformshift{0.374692in}{1.511652in}%
\pgfsys@useobject{currentmarker}{}%
\end{pgfscope}%
\end{pgfscope}%
\begin{pgfscope}%
\pgftext[x=0.100000in,y=1.463825in,left,base]{\rmfamily\fontsize{10.000000}{12.000000}\selectfont \(\displaystyle 3.8\)}%
\end{pgfscope}%
\begin{pgfscope}%
\pgfsetbuttcap%
\pgfsetroundjoin%
\definecolor{currentfill}{rgb}{0.000000,0.000000,0.000000}%
\pgfsetfillcolor{currentfill}%
\pgfsetlinewidth{0.803000pt}%
\definecolor{currentstroke}{rgb}{0.000000,0.000000,0.000000}%
\pgfsetstrokecolor{currentstroke}%
\pgfsetdash{}{0pt}%
\pgfsys@defobject{currentmarker}{\pgfqpoint{-0.048611in}{0.000000in}}{\pgfqpoint{0.000000in}{0.000000in}}{%
\pgfpathmoveto{\pgfqpoint{0.000000in}{0.000000in}}%
\pgfpathlineto{\pgfqpoint{-0.048611in}{0.000000in}}%
\pgfusepath{stroke,fill}%
}%
\begin{pgfscope}%
\pgfsys@transformshift{0.374692in}{1.956024in}%
\pgfsys@useobject{currentmarker}{}%
\end{pgfscope}%
\end{pgfscope}%
\begin{pgfscope}%
\pgftext[x=0.100000in,y=1.908197in,left,base]{\rmfamily\fontsize{10.000000}{12.000000}\selectfont \(\displaystyle 3.9\)}%
\end{pgfscope}%
\begin{pgfscope}%
\pgfsetbuttcap%
\pgfsetroundjoin%
\definecolor{currentfill}{rgb}{0.000000,0.000000,0.000000}%
\pgfsetfillcolor{currentfill}%
\pgfsetlinewidth{0.803000pt}%
\definecolor{currentstroke}{rgb}{0.000000,0.000000,0.000000}%
\pgfsetstrokecolor{currentstroke}%
\pgfsetdash{}{0pt}%
\pgfsys@defobject{currentmarker}{\pgfqpoint{-0.048611in}{0.000000in}}{\pgfqpoint{0.000000in}{0.000000in}}{%
\pgfpathmoveto{\pgfqpoint{0.000000in}{0.000000in}}%
\pgfpathlineto{\pgfqpoint{-0.048611in}{0.000000in}}%
\pgfusepath{stroke,fill}%
}%
\begin{pgfscope}%
\pgfsys@transformshift{0.374692in}{2.400396in}%
\pgfsys@useobject{currentmarker}{}%
\end{pgfscope}%
\end{pgfscope}%
\begin{pgfscope}%
\pgftext[x=0.100000in,y=2.352569in,left,base]{\rmfamily\fontsize{10.000000}{12.000000}\selectfont \(\displaystyle 4.0\)}%
\end{pgfscope}%
\begin{pgfscope}%
\pgfsetbuttcap%
\pgfsetroundjoin%
\definecolor{currentfill}{rgb}{0.000000,0.000000,0.000000}%
\pgfsetfillcolor{currentfill}%
\pgfsetlinewidth{0.803000pt}%
\definecolor{currentstroke}{rgb}{0.000000,0.000000,0.000000}%
\pgfsetstrokecolor{currentstroke}%
\pgfsetdash{}{0pt}%
\pgfsys@defobject{currentmarker}{\pgfqpoint{-0.048611in}{0.000000in}}{\pgfqpoint{0.000000in}{0.000000in}}{%
\pgfpathmoveto{\pgfqpoint{0.000000in}{0.000000in}}%
\pgfpathlineto{\pgfqpoint{-0.048611in}{0.000000in}}%
\pgfusepath{stroke,fill}%
}%
\begin{pgfscope}%
\pgfsys@transformshift{0.374692in}{2.844769in}%
\pgfsys@useobject{currentmarker}{}%
\end{pgfscope}%
\end{pgfscope}%
\begin{pgfscope}%
\pgftext[x=0.100000in,y=2.796941in,left,base]{\rmfamily\fontsize{10.000000}{12.000000}\selectfont \(\displaystyle 4.1\)}%
\end{pgfscope}%
\begin{pgfscope}%
\pgfsetrectcap%
\pgfsetmiterjoin%
\pgfsetlinewidth{0.803000pt}%
\definecolor{currentstroke}{rgb}{0.000000,0.000000,0.000000}%
\pgfsetstrokecolor{currentstroke}%
\pgfsetdash{}{0pt}%
\pgfpathmoveto{\pgfqpoint{0.374692in}{0.319877in}}%
\pgfpathlineto{\pgfqpoint{0.374692in}{2.925408in}}%
\pgfusepath{stroke}%
\end{pgfscope}%
\begin{pgfscope}%
\pgfsetrectcap%
\pgfsetmiterjoin%
\pgfsetlinewidth{0.803000pt}%
\definecolor{currentstroke}{rgb}{0.000000,0.000000,0.000000}%
\pgfsetstrokecolor{currentstroke}%
\pgfsetdash{}{0pt}%
\pgfpathmoveto{\pgfqpoint{1.954366in}{0.319877in}}%
\pgfpathlineto{\pgfqpoint{1.954366in}{2.925408in}}%
\pgfusepath{stroke}%
\end{pgfscope}%
\begin{pgfscope}%
\pgfsetrectcap%
\pgfsetmiterjoin%
\pgfsetlinewidth{0.803000pt}%
\definecolor{currentstroke}{rgb}{0.000000,0.000000,0.000000}%
\pgfsetstrokecolor{currentstroke}%
\pgfsetdash{}{0pt}%
\pgfpathmoveto{\pgfqpoint{0.374692in}{0.319877in}}%
\pgfpathlineto{\pgfqpoint{1.954366in}{0.319877in}}%
\pgfusepath{stroke}%
\end{pgfscope}%
\begin{pgfscope}%
\pgfsetrectcap%
\pgfsetmiterjoin%
\pgfsetlinewidth{0.803000pt}%
\definecolor{currentstroke}{rgb}{0.000000,0.000000,0.000000}%
\pgfsetstrokecolor{currentstroke}%
\pgfsetdash{}{0pt}%
\pgfpathmoveto{\pgfqpoint{0.374692in}{2.925408in}}%
\pgfpathlineto{\pgfqpoint{1.954366in}{2.925408in}}%
\pgfusepath{stroke}%
\end{pgfscope}%
\begin{pgfscope}%
\pgfpathrectangle{\pgfqpoint{2.053095in}{0.319877in}}{\pgfqpoint{0.130277in}{2.605531in}} %
\pgfusepath{clip}%
\pgfsetbuttcap%
\pgfsetmiterjoin%
\definecolor{currentfill}{rgb}{1.000000,1.000000,1.000000}%
\pgfsetfillcolor{currentfill}%
\pgfsetlinewidth{0.010037pt}%
\definecolor{currentstroke}{rgb}{1.000000,1.000000,1.000000}%
\pgfsetstrokecolor{currentstroke}%
\pgfsetdash{}{0pt}%
\pgfpathmoveto{\pgfqpoint{2.053095in}{0.319877in}}%
\pgfpathlineto{\pgfqpoint{2.053095in}{0.330055in}}%
\pgfpathlineto{\pgfqpoint{2.053095in}{2.915230in}}%
\pgfpathlineto{\pgfqpoint{2.053095in}{2.925408in}}%
\pgfpathlineto{\pgfqpoint{2.183372in}{2.925408in}}%
\pgfpathlineto{\pgfqpoint{2.183372in}{2.915230in}}%
\pgfpathlineto{\pgfqpoint{2.183372in}{0.330055in}}%
\pgfpathlineto{\pgfqpoint{2.183372in}{0.319877in}}%
\pgfpathclose%
\pgfusepath{stroke,fill}%
\end{pgfscope}%
\begin{pgfscope}%
\pgfsys@transformshift{2.050000in}{0.320408in}%
\pgftext[left,bottom]{\pgfimage[interpolate=true,width=0.130000in,height=2.610000in]{RnnNorm_vs_dq_Ti_100K-img1.png}}%
\end{pgfscope}%
\begin{pgfscope}%
\pgfsetbuttcap%
\pgfsetroundjoin%
\definecolor{currentfill}{rgb}{0.000000,0.000000,0.000000}%
\pgfsetfillcolor{currentfill}%
\pgfsetlinewidth{0.803000pt}%
\definecolor{currentstroke}{rgb}{0.000000,0.000000,0.000000}%
\pgfsetstrokecolor{currentstroke}%
\pgfsetdash{}{0pt}%
\pgfsys@defobject{currentmarker}{\pgfqpoint{0.000000in}{0.000000in}}{\pgfqpoint{0.048611in}{0.000000in}}{%
\pgfpathmoveto{\pgfqpoint{0.000000in}{0.000000in}}%
\pgfpathlineto{\pgfqpoint{0.048611in}{0.000000in}}%
\pgfusepath{stroke,fill}%
}%
\begin{pgfscope}%
\pgfsys@transformshift{2.183372in}{0.319877in}%
\pgfsys@useobject{currentmarker}{}%
\end{pgfscope}%
\end{pgfscope}%
\begin{pgfscope}%
\pgftext[x=2.280594in,y=0.272050in,left,base]{\rmfamily\fontsize{10.000000}{12.000000}\selectfont \(\displaystyle 0\)}%
\end{pgfscope}%
\begin{pgfscope}%
\pgfsetbuttcap%
\pgfsetroundjoin%
\definecolor{currentfill}{rgb}{0.000000,0.000000,0.000000}%
\pgfsetfillcolor{currentfill}%
\pgfsetlinewidth{0.803000pt}%
\definecolor{currentstroke}{rgb}{0.000000,0.000000,0.000000}%
\pgfsetstrokecolor{currentstroke}%
\pgfsetdash{}{0pt}%
\pgfsys@defobject{currentmarker}{\pgfqpoint{0.000000in}{0.000000in}}{\pgfqpoint{0.048611in}{0.000000in}}{%
\pgfpathmoveto{\pgfqpoint{0.000000in}{0.000000in}}%
\pgfpathlineto{\pgfqpoint{0.048611in}{0.000000in}}%
\pgfusepath{stroke,fill}%
}%
\begin{pgfscope}%
\pgfsys@transformshift{2.183372in}{0.793610in}%
\pgfsys@useobject{currentmarker}{}%
\end{pgfscope}%
\end{pgfscope}%
\begin{pgfscope}%
\pgftext[x=2.280594in,y=0.745782in,left,base]{\rmfamily\fontsize{10.000000}{12.000000}\selectfont \(\displaystyle 10\)}%
\end{pgfscope}%
\begin{pgfscope}%
\pgfsetbuttcap%
\pgfsetroundjoin%
\definecolor{currentfill}{rgb}{0.000000,0.000000,0.000000}%
\pgfsetfillcolor{currentfill}%
\pgfsetlinewidth{0.803000pt}%
\definecolor{currentstroke}{rgb}{0.000000,0.000000,0.000000}%
\pgfsetstrokecolor{currentstroke}%
\pgfsetdash{}{0pt}%
\pgfsys@defobject{currentmarker}{\pgfqpoint{0.000000in}{0.000000in}}{\pgfqpoint{0.048611in}{0.000000in}}{%
\pgfpathmoveto{\pgfqpoint{0.000000in}{0.000000in}}%
\pgfpathlineto{\pgfqpoint{0.048611in}{0.000000in}}%
\pgfusepath{stroke,fill}%
}%
\begin{pgfscope}%
\pgfsys@transformshift{2.183372in}{1.267343in}%
\pgfsys@useobject{currentmarker}{}%
\end{pgfscope}%
\end{pgfscope}%
\begin{pgfscope}%
\pgftext[x=2.280594in,y=1.219515in,left,base]{\rmfamily\fontsize{10.000000}{12.000000}\selectfont \(\displaystyle 20\)}%
\end{pgfscope}%
\begin{pgfscope}%
\pgfsetbuttcap%
\pgfsetroundjoin%
\definecolor{currentfill}{rgb}{0.000000,0.000000,0.000000}%
\pgfsetfillcolor{currentfill}%
\pgfsetlinewidth{0.803000pt}%
\definecolor{currentstroke}{rgb}{0.000000,0.000000,0.000000}%
\pgfsetstrokecolor{currentstroke}%
\pgfsetdash{}{0pt}%
\pgfsys@defobject{currentmarker}{\pgfqpoint{0.000000in}{0.000000in}}{\pgfqpoint{0.048611in}{0.000000in}}{%
\pgfpathmoveto{\pgfqpoint{0.000000in}{0.000000in}}%
\pgfpathlineto{\pgfqpoint{0.048611in}{0.000000in}}%
\pgfusepath{stroke,fill}%
}%
\begin{pgfscope}%
\pgfsys@transformshift{2.183372in}{1.741076in}%
\pgfsys@useobject{currentmarker}{}%
\end{pgfscope}%
\end{pgfscope}%
\begin{pgfscope}%
\pgftext[x=2.280594in,y=1.693248in,left,base]{\rmfamily\fontsize{10.000000}{12.000000}\selectfont \(\displaystyle 30\)}%
\end{pgfscope}%
\begin{pgfscope}%
\pgfsetbuttcap%
\pgfsetroundjoin%
\definecolor{currentfill}{rgb}{0.000000,0.000000,0.000000}%
\pgfsetfillcolor{currentfill}%
\pgfsetlinewidth{0.803000pt}%
\definecolor{currentstroke}{rgb}{0.000000,0.000000,0.000000}%
\pgfsetstrokecolor{currentstroke}%
\pgfsetdash{}{0pt}%
\pgfsys@defobject{currentmarker}{\pgfqpoint{0.000000in}{0.000000in}}{\pgfqpoint{0.048611in}{0.000000in}}{%
\pgfpathmoveto{\pgfqpoint{0.000000in}{0.000000in}}%
\pgfpathlineto{\pgfqpoint{0.048611in}{0.000000in}}%
\pgfusepath{stroke,fill}%
}%
\begin{pgfscope}%
\pgfsys@transformshift{2.183372in}{2.214809in}%
\pgfsys@useobject{currentmarker}{}%
\end{pgfscope}%
\end{pgfscope}%
\begin{pgfscope}%
\pgftext[x=2.280594in,y=2.166981in,left,base]{\rmfamily\fontsize{10.000000}{12.000000}\selectfont \(\displaystyle 40\)}%
\end{pgfscope}%
\begin{pgfscope}%
\pgfsetbuttcap%
\pgfsetroundjoin%
\definecolor{currentfill}{rgb}{0.000000,0.000000,0.000000}%
\pgfsetfillcolor{currentfill}%
\pgfsetlinewidth{0.803000pt}%
\definecolor{currentstroke}{rgb}{0.000000,0.000000,0.000000}%
\pgfsetstrokecolor{currentstroke}%
\pgfsetdash{}{0pt}%
\pgfsys@defobject{currentmarker}{\pgfqpoint{0.000000in}{0.000000in}}{\pgfqpoint{0.048611in}{0.000000in}}{%
\pgfpathmoveto{\pgfqpoint{0.000000in}{0.000000in}}%
\pgfpathlineto{\pgfqpoint{0.048611in}{0.000000in}}%
\pgfusepath{stroke,fill}%
}%
\begin{pgfscope}%
\pgfsys@transformshift{2.183372in}{2.688541in}%
\pgfsys@useobject{currentmarker}{}%
\end{pgfscope}%
\end{pgfscope}%
\begin{pgfscope}%
\pgftext[x=2.280594in,y=2.640714in,left,base]{\rmfamily\fontsize{10.000000}{12.000000}\selectfont \(\displaystyle 50\)}%
\end{pgfscope}%
\begin{pgfscope}%
\pgfsetbuttcap%
\pgfsetmiterjoin%
\pgfsetlinewidth{0.803000pt}%
\definecolor{currentstroke}{rgb}{0.000000,0.000000,0.000000}%
\pgfsetstrokecolor{currentstroke}%
\pgfsetdash{}{0pt}%
\pgfpathmoveto{\pgfqpoint{2.053095in}{0.319877in}}%
\pgfpathlineto{\pgfqpoint{2.053095in}{0.330055in}}%
\pgfpathlineto{\pgfqpoint{2.053095in}{2.915230in}}%
\pgfpathlineto{\pgfqpoint{2.053095in}{2.925408in}}%
\pgfpathlineto{\pgfqpoint{2.183372in}{2.925408in}}%
\pgfpathlineto{\pgfqpoint{2.183372in}{2.915230in}}%
\pgfpathlineto{\pgfqpoint{2.183372in}{0.330055in}}%
\pgfpathlineto{\pgfqpoint{2.183372in}{0.319877in}}%
\pgfpathclose%
\pgfusepath{stroke}%
\end{pgfscope}%
\end{pgfpicture}%
\makeatother%
\endgroup%

	\vspace*{-0.4cm}
	\caption{100 K. Bin size $0.0105e$}
	\end{subfigure}
	\hspace{0.6cm}
	\begin{subfigure}[b]{0.45\textwidth}
	\hspace*{-0.4cm}
	%% Creator: Matplotlib, PGF backend
%%
%% To include the figure in your LaTeX document, write
%%   \input{<filename>.pgf}
%%
%% Make sure the required packages are loaded in your preamble
%%   \usepackage{pgf}
%%
%% Figures using additional raster images can only be included by \input if
%% they are in the same directory as the main LaTeX file. For loading figures
%% from other directories you can use the `import` package
%%   \usepackage{import}
%% and then include the figures with
%%   \import{<path to file>}{<filename>.pgf}
%%
%% Matplotlib used the following preamble
%%   \usepackage[utf8x]{inputenc}
%%   \usepackage[T1]{fontenc}
%%
\begingroup%
\makeatletter%
\begin{pgfpicture}%
\pgfpathrectangle{\pgfpointorigin}{\pgfqpoint{2.519483in}{3.060408in}}%
\pgfusepath{use as bounding box, clip}%
\begin{pgfscope}%
\pgfsetbuttcap%
\pgfsetmiterjoin%
\definecolor{currentfill}{rgb}{1.000000,1.000000,1.000000}%
\pgfsetfillcolor{currentfill}%
\pgfsetlinewidth{0.000000pt}%
\definecolor{currentstroke}{rgb}{1.000000,1.000000,1.000000}%
\pgfsetstrokecolor{currentstroke}%
\pgfsetdash{}{0pt}%
\pgfpathmoveto{\pgfqpoint{0.000000in}{0.000000in}}%
\pgfpathlineto{\pgfqpoint{2.519483in}{0.000000in}}%
\pgfpathlineto{\pgfqpoint{2.519483in}{3.060408in}}%
\pgfpathlineto{\pgfqpoint{0.000000in}{3.060408in}}%
\pgfpathclose%
\pgfusepath{fill}%
\end{pgfscope}%
\begin{pgfscope}%
\pgfsetbuttcap%
\pgfsetmiterjoin%
\definecolor{currentfill}{rgb}{1.000000,1.000000,1.000000}%
\pgfsetfillcolor{currentfill}%
\pgfsetlinewidth{0.000000pt}%
\definecolor{currentstroke}{rgb}{0.000000,0.000000,0.000000}%
\pgfsetstrokecolor{currentstroke}%
\pgfsetstrokeopacity{0.000000}%
\pgfsetdash{}{0pt}%
\pgfpathmoveto{\pgfqpoint{0.374692in}{0.319877in}}%
\pgfpathlineto{\pgfqpoint{1.954366in}{0.319877in}}%
\pgfpathlineto{\pgfqpoint{1.954366in}{2.925408in}}%
\pgfpathlineto{\pgfqpoint{0.374692in}{2.925408in}}%
\pgfpathclose%
\pgfusepath{fill}%
\end{pgfscope}%
\begin{pgfscope}%
\pgfpathrectangle{\pgfqpoint{0.374692in}{0.319877in}}{\pgfqpoint{1.579674in}{2.605531in}} %
\pgfusepath{clip}%
\pgfsys@transformshift{0.374692in}{0.319877in}%
\pgftext[left,bottom]{\pgfimage[interpolate=true,width=1.580000in,height=2.610000in]{RnnNorm_vs_dq_Ti_200K-img0.png}}%
\end{pgfscope}%
\begin{pgfscope}%
\pgfpathrectangle{\pgfqpoint{0.374692in}{0.319877in}}{\pgfqpoint{1.579674in}{2.605531in}} %
\pgfusepath{clip}%
\pgfsetbuttcap%
\pgfsetroundjoin%
\definecolor{currentfill}{rgb}{1.000000,0.752941,0.796078}%
\pgfsetfillcolor{currentfill}%
\pgfsetlinewidth{1.003750pt}%
\definecolor{currentstroke}{rgb}{1.000000,0.752941,0.796078}%
\pgfsetstrokecolor{currentstroke}%
\pgfsetdash{}{0pt}%
\pgfpathmoveto{\pgfqpoint{0.882444in}{1.580976in}}%
\pgfpathcurveto{\pgfqpoint{0.893494in}{1.580976in}}{\pgfqpoint{0.904093in}{1.585366in}}{\pgfqpoint{0.911907in}{1.593180in}}%
\pgfpathcurveto{\pgfqpoint{0.919721in}{1.600993in}}{\pgfqpoint{0.924111in}{1.611592in}}{\pgfqpoint{0.924111in}{1.622643in}}%
\pgfpathcurveto{\pgfqpoint{0.924111in}{1.633693in}}{\pgfqpoint{0.919721in}{1.644292in}}{\pgfqpoint{0.911907in}{1.652105in}}%
\pgfpathcurveto{\pgfqpoint{0.904093in}{1.659919in}}{\pgfqpoint{0.893494in}{1.664309in}}{\pgfqpoint{0.882444in}{1.664309in}}%
\pgfpathcurveto{\pgfqpoint{0.871394in}{1.664309in}}{\pgfqpoint{0.860795in}{1.659919in}}{\pgfqpoint{0.852981in}{1.652105in}}%
\pgfpathcurveto{\pgfqpoint{0.845168in}{1.644292in}}{\pgfqpoint{0.840778in}{1.633693in}}{\pgfqpoint{0.840778in}{1.622643in}}%
\pgfpathcurveto{\pgfqpoint{0.840778in}{1.611592in}}{\pgfqpoint{0.845168in}{1.600993in}}{\pgfqpoint{0.852981in}{1.593180in}}%
\pgfpathcurveto{\pgfqpoint{0.860795in}{1.585366in}}{\pgfqpoint{0.871394in}{1.580976in}}{\pgfqpoint{0.882444in}{1.580976in}}%
\pgfpathclose%
\pgfusepath{stroke,fill}%
\end{pgfscope}%
\begin{pgfscope}%
\pgfpathrectangle{\pgfqpoint{0.374692in}{0.319877in}}{\pgfqpoint{1.579674in}{2.605531in}} %
\pgfusepath{clip}%
\pgfsetbuttcap%
\pgfsetroundjoin%
\definecolor{currentfill}{rgb}{1.000000,0.752941,0.796078}%
\pgfsetfillcolor{currentfill}%
\pgfsetlinewidth{1.003750pt}%
\definecolor{currentstroke}{rgb}{1.000000,0.752941,0.796078}%
\pgfsetstrokecolor{currentstroke}%
\pgfsetdash{}{0pt}%
\pgfpathmoveto{\pgfqpoint{0.995278in}{1.480763in}}%
\pgfpathcurveto{\pgfqpoint{1.006328in}{1.480763in}}{\pgfqpoint{1.016927in}{1.485153in}}{\pgfqpoint{1.024741in}{1.492967in}}%
\pgfpathcurveto{\pgfqpoint{1.032554in}{1.500781in}}{\pgfqpoint{1.036945in}{1.511380in}}{\pgfqpoint{1.036945in}{1.522430in}}%
\pgfpathcurveto{\pgfqpoint{1.036945in}{1.533480in}}{\pgfqpoint{1.032554in}{1.544079in}}{\pgfqpoint{1.024741in}{1.551893in}}%
\pgfpathcurveto{\pgfqpoint{1.016927in}{1.559706in}}{\pgfqpoint{1.006328in}{1.564097in}}{\pgfqpoint{0.995278in}{1.564097in}}%
\pgfpathcurveto{\pgfqpoint{0.984228in}{1.564097in}}{\pgfqpoint{0.973629in}{1.559706in}}{\pgfqpoint{0.965815in}{1.551893in}}%
\pgfpathcurveto{\pgfqpoint{0.958002in}{1.544079in}}{\pgfqpoint{0.953611in}{1.533480in}}{\pgfqpoint{0.953611in}{1.522430in}}%
\pgfpathcurveto{\pgfqpoint{0.953611in}{1.511380in}}{\pgfqpoint{0.958002in}{1.500781in}}{\pgfqpoint{0.965815in}{1.492967in}}%
\pgfpathcurveto{\pgfqpoint{0.973629in}{1.485153in}}{\pgfqpoint{0.984228in}{1.480763in}}{\pgfqpoint{0.995278in}{1.480763in}}%
\pgfpathclose%
\pgfusepath{stroke,fill}%
\end{pgfscope}%
\begin{pgfscope}%
\pgfpathrectangle{\pgfqpoint{0.374692in}{0.319877in}}{\pgfqpoint{1.579674in}{2.605531in}} %
\pgfusepath{clip}%
\pgfsetbuttcap%
\pgfsetroundjoin%
\definecolor{currentfill}{rgb}{1.000000,0.752941,0.796078}%
\pgfsetfillcolor{currentfill}%
\pgfsetlinewidth{1.003750pt}%
\definecolor{currentstroke}{rgb}{1.000000,0.752941,0.796078}%
\pgfsetstrokecolor{currentstroke}%
\pgfsetdash{}{0pt}%
\pgfpathmoveto{\pgfqpoint{1.108112in}{1.413187in}}%
\pgfpathcurveto{\pgfqpoint{1.119162in}{1.413187in}}{\pgfqpoint{1.129761in}{1.417577in}}{\pgfqpoint{1.137575in}{1.425391in}}%
\pgfpathcurveto{\pgfqpoint{1.145388in}{1.433204in}}{\pgfqpoint{1.149779in}{1.443803in}}{\pgfqpoint{1.149779in}{1.454854in}}%
\pgfpathcurveto{\pgfqpoint{1.149779in}{1.465904in}}{\pgfqpoint{1.145388in}{1.476503in}}{\pgfqpoint{1.137575in}{1.484316in}}%
\pgfpathcurveto{\pgfqpoint{1.129761in}{1.492130in}}{\pgfqpoint{1.119162in}{1.496520in}}{\pgfqpoint{1.108112in}{1.496520in}}%
\pgfpathcurveto{\pgfqpoint{1.097062in}{1.496520in}}{\pgfqpoint{1.086463in}{1.492130in}}{\pgfqpoint{1.078649in}{1.484316in}}%
\pgfpathcurveto{\pgfqpoint{1.070836in}{1.476503in}}{\pgfqpoint{1.066445in}{1.465904in}}{\pgfqpoint{1.066445in}{1.454854in}}%
\pgfpathcurveto{\pgfqpoint{1.066445in}{1.443803in}}{\pgfqpoint{1.070836in}{1.433204in}}{\pgfqpoint{1.078649in}{1.425391in}}%
\pgfpathcurveto{\pgfqpoint{1.086463in}{1.417577in}}{\pgfqpoint{1.097062in}{1.413187in}}{\pgfqpoint{1.108112in}{1.413187in}}%
\pgfpathclose%
\pgfusepath{stroke,fill}%
\end{pgfscope}%
\begin{pgfscope}%
\pgfpathrectangle{\pgfqpoint{0.374692in}{0.319877in}}{\pgfqpoint{1.579674in}{2.605531in}} %
\pgfusepath{clip}%
\pgfsetbuttcap%
\pgfsetroundjoin%
\definecolor{currentfill}{rgb}{1.000000,0.752941,0.796078}%
\pgfsetfillcolor{currentfill}%
\pgfsetlinewidth{1.003750pt}%
\definecolor{currentstroke}{rgb}{1.000000,0.752941,0.796078}%
\pgfsetstrokecolor{currentstroke}%
\pgfsetdash{}{0pt}%
\pgfpathmoveto{\pgfqpoint{1.220946in}{1.372639in}}%
\pgfpathcurveto{\pgfqpoint{1.231996in}{1.372639in}}{\pgfqpoint{1.242595in}{1.377029in}}{\pgfqpoint{1.250409in}{1.384843in}}%
\pgfpathcurveto{\pgfqpoint{1.258222in}{1.392656in}}{\pgfqpoint{1.262612in}{1.403256in}}{\pgfqpoint{1.262612in}{1.414306in}}%
\pgfpathcurveto{\pgfqpoint{1.262612in}{1.425356in}}{\pgfqpoint{1.258222in}{1.435955in}}{\pgfqpoint{1.250409in}{1.443768in}}%
\pgfpathcurveto{\pgfqpoint{1.242595in}{1.451582in}}{\pgfqpoint{1.231996in}{1.455972in}}{\pgfqpoint{1.220946in}{1.455972in}}%
\pgfpathcurveto{\pgfqpoint{1.209896in}{1.455972in}}{\pgfqpoint{1.199297in}{1.451582in}}{\pgfqpoint{1.191483in}{1.443768in}}%
\pgfpathcurveto{\pgfqpoint{1.183669in}{1.435955in}}{\pgfqpoint{1.179279in}{1.425356in}}{\pgfqpoint{1.179279in}{1.414306in}}%
\pgfpathcurveto{\pgfqpoint{1.179279in}{1.403256in}}{\pgfqpoint{1.183669in}{1.392656in}}{\pgfqpoint{1.191483in}{1.384843in}}%
\pgfpathcurveto{\pgfqpoint{1.199297in}{1.377029in}}{\pgfqpoint{1.209896in}{1.372639in}}{\pgfqpoint{1.220946in}{1.372639in}}%
\pgfpathclose%
\pgfusepath{stroke,fill}%
\end{pgfscope}%
\begin{pgfscope}%
\pgfpathrectangle{\pgfqpoint{0.374692in}{0.319877in}}{\pgfqpoint{1.579674in}{2.605531in}} %
\pgfusepath{clip}%
\pgfsetbuttcap%
\pgfsetroundjoin%
\definecolor{currentfill}{rgb}{1.000000,0.752941,0.796078}%
\pgfsetfillcolor{currentfill}%
\pgfsetlinewidth{1.003750pt}%
\definecolor{currentstroke}{rgb}{1.000000,0.752941,0.796078}%
\pgfsetstrokecolor{currentstroke}%
\pgfsetdash{}{0pt}%
\pgfpathmoveto{\pgfqpoint{1.333780in}{1.313742in}}%
\pgfpathcurveto{\pgfqpoint{1.344830in}{1.313742in}}{\pgfqpoint{1.355429in}{1.318132in}}{\pgfqpoint{1.363242in}{1.325946in}}%
\pgfpathcurveto{\pgfqpoint{1.371056in}{1.333760in}}{\pgfqpoint{1.375446in}{1.344359in}}{\pgfqpoint{1.375446in}{1.355409in}}%
\pgfpathcurveto{\pgfqpoint{1.375446in}{1.366459in}}{\pgfqpoint{1.371056in}{1.377058in}}{\pgfqpoint{1.363242in}{1.384871in}}%
\pgfpathcurveto{\pgfqpoint{1.355429in}{1.392685in}}{\pgfqpoint{1.344830in}{1.397075in}}{\pgfqpoint{1.333780in}{1.397075in}}%
\pgfpathcurveto{\pgfqpoint{1.322729in}{1.397075in}}{\pgfqpoint{1.312130in}{1.392685in}}{\pgfqpoint{1.304317in}{1.384871in}}%
\pgfpathcurveto{\pgfqpoint{1.296503in}{1.377058in}}{\pgfqpoint{1.292113in}{1.366459in}}{\pgfqpoint{1.292113in}{1.355409in}}%
\pgfpathcurveto{\pgfqpoint{1.292113in}{1.344359in}}{\pgfqpoint{1.296503in}{1.333760in}}{\pgfqpoint{1.304317in}{1.325946in}}%
\pgfpathcurveto{\pgfqpoint{1.312130in}{1.318132in}}{\pgfqpoint{1.322729in}{1.313742in}}{\pgfqpoint{1.333780in}{1.313742in}}%
\pgfpathclose%
\pgfusepath{stroke,fill}%
\end{pgfscope}%
\begin{pgfscope}%
\pgfpathrectangle{\pgfqpoint{0.374692in}{0.319877in}}{\pgfqpoint{1.579674in}{2.605531in}} %
\pgfusepath{clip}%
\pgfsetbuttcap%
\pgfsetroundjoin%
\definecolor{currentfill}{rgb}{1.000000,0.752941,0.796078}%
\pgfsetfillcolor{currentfill}%
\pgfsetlinewidth{1.003750pt}%
\definecolor{currentstroke}{rgb}{1.000000,0.752941,0.796078}%
\pgfsetstrokecolor{currentstroke}%
\pgfsetdash{}{0pt}%
\pgfpathmoveto{\pgfqpoint{1.446613in}{1.280338in}}%
\pgfpathcurveto{\pgfqpoint{1.457664in}{1.280338in}}{\pgfqpoint{1.468263in}{1.284728in}}{\pgfqpoint{1.476076in}{1.292542in}}%
\pgfpathcurveto{\pgfqpoint{1.483890in}{1.300355in}}{\pgfqpoint{1.488280in}{1.310954in}}{\pgfqpoint{1.488280in}{1.322004in}}%
\pgfpathcurveto{\pgfqpoint{1.488280in}{1.333055in}}{\pgfqpoint{1.483890in}{1.343654in}}{\pgfqpoint{1.476076in}{1.351467in}}%
\pgfpathcurveto{\pgfqpoint{1.468263in}{1.359281in}}{\pgfqpoint{1.457664in}{1.363671in}}{\pgfqpoint{1.446613in}{1.363671in}}%
\pgfpathcurveto{\pgfqpoint{1.435563in}{1.363671in}}{\pgfqpoint{1.424964in}{1.359281in}}{\pgfqpoint{1.417151in}{1.351467in}}%
\pgfpathcurveto{\pgfqpoint{1.409337in}{1.343654in}}{\pgfqpoint{1.404947in}{1.333055in}}{\pgfqpoint{1.404947in}{1.322004in}}%
\pgfpathcurveto{\pgfqpoint{1.404947in}{1.310954in}}{\pgfqpoint{1.409337in}{1.300355in}}{\pgfqpoint{1.417151in}{1.292542in}}%
\pgfpathcurveto{\pgfqpoint{1.424964in}{1.284728in}}{\pgfqpoint{1.435563in}{1.280338in}}{\pgfqpoint{1.446613in}{1.280338in}}%
\pgfpathclose%
\pgfusepath{stroke,fill}%
\end{pgfscope}%
\begin{pgfscope}%
\pgfpathrectangle{\pgfqpoint{0.374692in}{0.319877in}}{\pgfqpoint{1.579674in}{2.605531in}} %
\pgfusepath{clip}%
\pgfsetbuttcap%
\pgfsetroundjoin%
\definecolor{currentfill}{rgb}{1.000000,0.752941,0.796078}%
\pgfsetfillcolor{currentfill}%
\pgfsetlinewidth{1.003750pt}%
\definecolor{currentstroke}{rgb}{1.000000,0.752941,0.796078}%
\pgfsetstrokecolor{currentstroke}%
\pgfsetdash{}{0pt}%
\pgfpathmoveto{\pgfqpoint{1.559447in}{1.180125in}}%
\pgfpathcurveto{\pgfqpoint{1.570497in}{1.180125in}}{\pgfqpoint{1.581096in}{1.184515in}}{\pgfqpoint{1.588910in}{1.192329in}}%
\pgfpathcurveto{\pgfqpoint{1.596724in}{1.200143in}}{\pgfqpoint{1.601114in}{1.210742in}}{\pgfqpoint{1.601114in}{1.221792in}}%
\pgfpathcurveto{\pgfqpoint{1.601114in}{1.232842in}}{\pgfqpoint{1.596724in}{1.243441in}}{\pgfqpoint{1.588910in}{1.251255in}}%
\pgfpathcurveto{\pgfqpoint{1.581096in}{1.259068in}}{\pgfqpoint{1.570497in}{1.263458in}}{\pgfqpoint{1.559447in}{1.263458in}}%
\pgfpathcurveto{\pgfqpoint{1.548397in}{1.263458in}}{\pgfqpoint{1.537798in}{1.259068in}}{\pgfqpoint{1.529985in}{1.251255in}}%
\pgfpathcurveto{\pgfqpoint{1.522171in}{1.243441in}}{\pgfqpoint{1.517781in}{1.232842in}}{\pgfqpoint{1.517781in}{1.221792in}}%
\pgfpathcurveto{\pgfqpoint{1.517781in}{1.210742in}}{\pgfqpoint{1.522171in}{1.200143in}}{\pgfqpoint{1.529985in}{1.192329in}}%
\pgfpathcurveto{\pgfqpoint{1.537798in}{1.184515in}}{\pgfqpoint{1.548397in}{1.180125in}}{\pgfqpoint{1.559447in}{1.180125in}}%
\pgfpathclose%
\pgfusepath{stroke,fill}%
\end{pgfscope}%
\begin{pgfscope}%
\pgfsetbuttcap%
\pgfsetroundjoin%
\definecolor{currentfill}{rgb}{0.000000,0.000000,0.000000}%
\pgfsetfillcolor{currentfill}%
\pgfsetlinewidth{0.803000pt}%
\definecolor{currentstroke}{rgb}{0.000000,0.000000,0.000000}%
\pgfsetstrokecolor{currentstroke}%
\pgfsetdash{}{0pt}%
\pgfsys@defobject{currentmarker}{\pgfqpoint{0.000000in}{-0.048611in}}{\pgfqpoint{0.000000in}{0.000000in}}{%
\pgfpathmoveto{\pgfqpoint{0.000000in}{0.000000in}}%
\pgfpathlineto{\pgfqpoint{0.000000in}{-0.048611in}}%
\pgfusepath{stroke,fill}%
}%
\begin{pgfscope}%
\pgfsys@transformshift{0.670881in}{0.319877in}%
\pgfsys@useobject{currentmarker}{}%
\end{pgfscope}%
\end{pgfscope}%
\begin{pgfscope}%
\pgftext[x=0.670881in,y=0.222655in,,top]{\rmfamily\fontsize{10.000000}{12.000000}\selectfont \(\displaystyle -0.05\)}%
\end{pgfscope}%
\begin{pgfscope}%
\pgfsetbuttcap%
\pgfsetroundjoin%
\definecolor{currentfill}{rgb}{0.000000,0.000000,0.000000}%
\pgfsetfillcolor{currentfill}%
\pgfsetlinewidth{0.803000pt}%
\definecolor{currentstroke}{rgb}{0.000000,0.000000,0.000000}%
\pgfsetstrokecolor{currentstroke}%
\pgfsetdash{}{0pt}%
\pgfsys@defobject{currentmarker}{\pgfqpoint{0.000000in}{-0.048611in}}{\pgfqpoint{0.000000in}{0.000000in}}{%
\pgfpathmoveto{\pgfqpoint{0.000000in}{0.000000in}}%
\pgfpathlineto{\pgfqpoint{0.000000in}{-0.048611in}}%
\pgfusepath{stroke,fill}%
}%
\begin{pgfscope}%
\pgfsys@transformshift{1.164529in}{0.319877in}%
\pgfsys@useobject{currentmarker}{}%
\end{pgfscope}%
\end{pgfscope}%
\begin{pgfscope}%
\pgftext[x=1.164529in,y=0.222655in,,top]{\rmfamily\fontsize{10.000000}{12.000000}\selectfont \(\displaystyle 0.00\)}%
\end{pgfscope}%
\begin{pgfscope}%
\pgfsetbuttcap%
\pgfsetroundjoin%
\definecolor{currentfill}{rgb}{0.000000,0.000000,0.000000}%
\pgfsetfillcolor{currentfill}%
\pgfsetlinewidth{0.803000pt}%
\definecolor{currentstroke}{rgb}{0.000000,0.000000,0.000000}%
\pgfsetstrokecolor{currentstroke}%
\pgfsetdash{}{0pt}%
\pgfsys@defobject{currentmarker}{\pgfqpoint{0.000000in}{-0.048611in}}{\pgfqpoint{0.000000in}{0.000000in}}{%
\pgfpathmoveto{\pgfqpoint{0.000000in}{0.000000in}}%
\pgfpathlineto{\pgfqpoint{0.000000in}{-0.048611in}}%
\pgfusepath{stroke,fill}%
}%
\begin{pgfscope}%
\pgfsys@transformshift{1.658177in}{0.319877in}%
\pgfsys@useobject{currentmarker}{}%
\end{pgfscope}%
\end{pgfscope}%
\begin{pgfscope}%
\pgftext[x=1.658177in,y=0.222655in,,top]{\rmfamily\fontsize{10.000000}{12.000000}\selectfont \(\displaystyle 0.05\)}%
\end{pgfscope}%
\begin{pgfscope}%
\pgfsetbuttcap%
\pgfsetroundjoin%
\definecolor{currentfill}{rgb}{0.000000,0.000000,0.000000}%
\pgfsetfillcolor{currentfill}%
\pgfsetlinewidth{0.803000pt}%
\definecolor{currentstroke}{rgb}{0.000000,0.000000,0.000000}%
\pgfsetstrokecolor{currentstroke}%
\pgfsetdash{}{0pt}%
\pgfsys@defobject{currentmarker}{\pgfqpoint{-0.048611in}{0.000000in}}{\pgfqpoint{0.000000in}{0.000000in}}{%
\pgfpathmoveto{\pgfqpoint{0.000000in}{0.000000in}}%
\pgfpathlineto{\pgfqpoint{-0.048611in}{0.000000in}}%
\pgfusepath{stroke,fill}%
}%
\begin{pgfscope}%
\pgfsys@transformshift{0.374692in}{0.622908in}%
\pgfsys@useobject{currentmarker}{}%
\end{pgfscope}%
\end{pgfscope}%
\begin{pgfscope}%
\pgftext[x=0.100000in,y=0.575080in,left,base]{\rmfamily\fontsize{10.000000}{12.000000}\selectfont \(\displaystyle 3.6\)}%
\end{pgfscope}%
\begin{pgfscope}%
\pgfsetbuttcap%
\pgfsetroundjoin%
\definecolor{currentfill}{rgb}{0.000000,0.000000,0.000000}%
\pgfsetfillcolor{currentfill}%
\pgfsetlinewidth{0.803000pt}%
\definecolor{currentstroke}{rgb}{0.000000,0.000000,0.000000}%
\pgfsetstrokecolor{currentstroke}%
\pgfsetdash{}{0pt}%
\pgfsys@defobject{currentmarker}{\pgfqpoint{-0.048611in}{0.000000in}}{\pgfqpoint{0.000000in}{0.000000in}}{%
\pgfpathmoveto{\pgfqpoint{0.000000in}{0.000000in}}%
\pgfpathlineto{\pgfqpoint{-0.048611in}{0.000000in}}%
\pgfusepath{stroke,fill}%
}%
\begin{pgfscope}%
\pgfsys@transformshift{0.374692in}{1.067280in}%
\pgfsys@useobject{currentmarker}{}%
\end{pgfscope}%
\end{pgfscope}%
\begin{pgfscope}%
\pgftext[x=0.100000in,y=1.019453in,left,base]{\rmfamily\fontsize{10.000000}{12.000000}\selectfont \(\displaystyle 3.7\)}%
\end{pgfscope}%
\begin{pgfscope}%
\pgfsetbuttcap%
\pgfsetroundjoin%
\definecolor{currentfill}{rgb}{0.000000,0.000000,0.000000}%
\pgfsetfillcolor{currentfill}%
\pgfsetlinewidth{0.803000pt}%
\definecolor{currentstroke}{rgb}{0.000000,0.000000,0.000000}%
\pgfsetstrokecolor{currentstroke}%
\pgfsetdash{}{0pt}%
\pgfsys@defobject{currentmarker}{\pgfqpoint{-0.048611in}{0.000000in}}{\pgfqpoint{0.000000in}{0.000000in}}{%
\pgfpathmoveto{\pgfqpoint{0.000000in}{0.000000in}}%
\pgfpathlineto{\pgfqpoint{-0.048611in}{0.000000in}}%
\pgfusepath{stroke,fill}%
}%
\begin{pgfscope}%
\pgfsys@transformshift{0.374692in}{1.511652in}%
\pgfsys@useobject{currentmarker}{}%
\end{pgfscope}%
\end{pgfscope}%
\begin{pgfscope}%
\pgftext[x=0.100000in,y=1.463825in,left,base]{\rmfamily\fontsize{10.000000}{12.000000}\selectfont \(\displaystyle 3.8\)}%
\end{pgfscope}%
\begin{pgfscope}%
\pgfsetbuttcap%
\pgfsetroundjoin%
\definecolor{currentfill}{rgb}{0.000000,0.000000,0.000000}%
\pgfsetfillcolor{currentfill}%
\pgfsetlinewidth{0.803000pt}%
\definecolor{currentstroke}{rgb}{0.000000,0.000000,0.000000}%
\pgfsetstrokecolor{currentstroke}%
\pgfsetdash{}{0pt}%
\pgfsys@defobject{currentmarker}{\pgfqpoint{-0.048611in}{0.000000in}}{\pgfqpoint{0.000000in}{0.000000in}}{%
\pgfpathmoveto{\pgfqpoint{0.000000in}{0.000000in}}%
\pgfpathlineto{\pgfqpoint{-0.048611in}{0.000000in}}%
\pgfusepath{stroke,fill}%
}%
\begin{pgfscope}%
\pgfsys@transformshift{0.374692in}{1.956024in}%
\pgfsys@useobject{currentmarker}{}%
\end{pgfscope}%
\end{pgfscope}%
\begin{pgfscope}%
\pgftext[x=0.100000in,y=1.908197in,left,base]{\rmfamily\fontsize{10.000000}{12.000000}\selectfont \(\displaystyle 3.9\)}%
\end{pgfscope}%
\begin{pgfscope}%
\pgfsetbuttcap%
\pgfsetroundjoin%
\definecolor{currentfill}{rgb}{0.000000,0.000000,0.000000}%
\pgfsetfillcolor{currentfill}%
\pgfsetlinewidth{0.803000pt}%
\definecolor{currentstroke}{rgb}{0.000000,0.000000,0.000000}%
\pgfsetstrokecolor{currentstroke}%
\pgfsetdash{}{0pt}%
\pgfsys@defobject{currentmarker}{\pgfqpoint{-0.048611in}{0.000000in}}{\pgfqpoint{0.000000in}{0.000000in}}{%
\pgfpathmoveto{\pgfqpoint{0.000000in}{0.000000in}}%
\pgfpathlineto{\pgfqpoint{-0.048611in}{0.000000in}}%
\pgfusepath{stroke,fill}%
}%
\begin{pgfscope}%
\pgfsys@transformshift{0.374692in}{2.400396in}%
\pgfsys@useobject{currentmarker}{}%
\end{pgfscope}%
\end{pgfscope}%
\begin{pgfscope}%
\pgftext[x=0.100000in,y=2.352569in,left,base]{\rmfamily\fontsize{10.000000}{12.000000}\selectfont \(\displaystyle 4.0\)}%
\end{pgfscope}%
\begin{pgfscope}%
\pgfsetbuttcap%
\pgfsetroundjoin%
\definecolor{currentfill}{rgb}{0.000000,0.000000,0.000000}%
\pgfsetfillcolor{currentfill}%
\pgfsetlinewidth{0.803000pt}%
\definecolor{currentstroke}{rgb}{0.000000,0.000000,0.000000}%
\pgfsetstrokecolor{currentstroke}%
\pgfsetdash{}{0pt}%
\pgfsys@defobject{currentmarker}{\pgfqpoint{-0.048611in}{0.000000in}}{\pgfqpoint{0.000000in}{0.000000in}}{%
\pgfpathmoveto{\pgfqpoint{0.000000in}{0.000000in}}%
\pgfpathlineto{\pgfqpoint{-0.048611in}{0.000000in}}%
\pgfusepath{stroke,fill}%
}%
\begin{pgfscope}%
\pgfsys@transformshift{0.374692in}{2.844769in}%
\pgfsys@useobject{currentmarker}{}%
\end{pgfscope}%
\end{pgfscope}%
\begin{pgfscope}%
\pgftext[x=0.100000in,y=2.796941in,left,base]{\rmfamily\fontsize{10.000000}{12.000000}\selectfont \(\displaystyle 4.1\)}%
\end{pgfscope}%
\begin{pgfscope}%
\pgfsetrectcap%
\pgfsetmiterjoin%
\pgfsetlinewidth{0.803000pt}%
\definecolor{currentstroke}{rgb}{0.000000,0.000000,0.000000}%
\pgfsetstrokecolor{currentstroke}%
\pgfsetdash{}{0pt}%
\pgfpathmoveto{\pgfqpoint{0.374692in}{0.319877in}}%
\pgfpathlineto{\pgfqpoint{0.374692in}{2.925408in}}%
\pgfusepath{stroke}%
\end{pgfscope}%
\begin{pgfscope}%
\pgfsetrectcap%
\pgfsetmiterjoin%
\pgfsetlinewidth{0.803000pt}%
\definecolor{currentstroke}{rgb}{0.000000,0.000000,0.000000}%
\pgfsetstrokecolor{currentstroke}%
\pgfsetdash{}{0pt}%
\pgfpathmoveto{\pgfqpoint{1.954366in}{0.319877in}}%
\pgfpathlineto{\pgfqpoint{1.954366in}{2.925408in}}%
\pgfusepath{stroke}%
\end{pgfscope}%
\begin{pgfscope}%
\pgfsetrectcap%
\pgfsetmiterjoin%
\pgfsetlinewidth{0.803000pt}%
\definecolor{currentstroke}{rgb}{0.000000,0.000000,0.000000}%
\pgfsetstrokecolor{currentstroke}%
\pgfsetdash{}{0pt}%
\pgfpathmoveto{\pgfqpoint{0.374692in}{0.319877in}}%
\pgfpathlineto{\pgfqpoint{1.954366in}{0.319877in}}%
\pgfusepath{stroke}%
\end{pgfscope}%
\begin{pgfscope}%
\pgfsetrectcap%
\pgfsetmiterjoin%
\pgfsetlinewidth{0.803000pt}%
\definecolor{currentstroke}{rgb}{0.000000,0.000000,0.000000}%
\pgfsetstrokecolor{currentstroke}%
\pgfsetdash{}{0pt}%
\pgfpathmoveto{\pgfqpoint{0.374692in}{2.925408in}}%
\pgfpathlineto{\pgfqpoint{1.954366in}{2.925408in}}%
\pgfusepath{stroke}%
\end{pgfscope}%
\begin{pgfscope}%
\pgfpathrectangle{\pgfqpoint{2.053095in}{0.319877in}}{\pgfqpoint{0.130277in}{2.605531in}} %
\pgfusepath{clip}%
\pgfsetbuttcap%
\pgfsetmiterjoin%
\definecolor{currentfill}{rgb}{1.000000,1.000000,1.000000}%
\pgfsetfillcolor{currentfill}%
\pgfsetlinewidth{0.010037pt}%
\definecolor{currentstroke}{rgb}{1.000000,1.000000,1.000000}%
\pgfsetstrokecolor{currentstroke}%
\pgfsetdash{}{0pt}%
\pgfpathmoveto{\pgfqpoint{2.053095in}{0.319877in}}%
\pgfpathlineto{\pgfqpoint{2.053095in}{0.330055in}}%
\pgfpathlineto{\pgfqpoint{2.053095in}{2.915230in}}%
\pgfpathlineto{\pgfqpoint{2.053095in}{2.925408in}}%
\pgfpathlineto{\pgfqpoint{2.183372in}{2.925408in}}%
\pgfpathlineto{\pgfqpoint{2.183372in}{2.915230in}}%
\pgfpathlineto{\pgfqpoint{2.183372in}{0.330055in}}%
\pgfpathlineto{\pgfqpoint{2.183372in}{0.319877in}}%
\pgfpathclose%
\pgfusepath{stroke,fill}%
\end{pgfscope}%
\begin{pgfscope}%
\pgfsys@transformshift{2.050000in}{0.320408in}%
\pgftext[left,bottom]{\pgfimage[interpolate=true,width=0.130000in,height=2.610000in]{RnnNorm_vs_dq_Ti_200K-img1.png}}%
\end{pgfscope}%
\begin{pgfscope}%
\pgfsetbuttcap%
\pgfsetroundjoin%
\definecolor{currentfill}{rgb}{0.000000,0.000000,0.000000}%
\pgfsetfillcolor{currentfill}%
\pgfsetlinewidth{0.803000pt}%
\definecolor{currentstroke}{rgb}{0.000000,0.000000,0.000000}%
\pgfsetstrokecolor{currentstroke}%
\pgfsetdash{}{0pt}%
\pgfsys@defobject{currentmarker}{\pgfqpoint{0.000000in}{0.000000in}}{\pgfqpoint{0.048611in}{0.000000in}}{%
\pgfpathmoveto{\pgfqpoint{0.000000in}{0.000000in}}%
\pgfpathlineto{\pgfqpoint{0.048611in}{0.000000in}}%
\pgfusepath{stroke,fill}%
}%
\begin{pgfscope}%
\pgfsys@transformshift{2.183372in}{0.319877in}%
\pgfsys@useobject{currentmarker}{}%
\end{pgfscope}%
\end{pgfscope}%
\begin{pgfscope}%
\pgftext[x=2.280594in,y=0.272050in,left,base]{\rmfamily\fontsize{10.000000}{12.000000}\selectfont \(\displaystyle 0\)}%
\end{pgfscope}%
\begin{pgfscope}%
\pgfsetbuttcap%
\pgfsetroundjoin%
\definecolor{currentfill}{rgb}{0.000000,0.000000,0.000000}%
\pgfsetfillcolor{currentfill}%
\pgfsetlinewidth{0.803000pt}%
\definecolor{currentstroke}{rgb}{0.000000,0.000000,0.000000}%
\pgfsetstrokecolor{currentstroke}%
\pgfsetdash{}{0pt}%
\pgfsys@defobject{currentmarker}{\pgfqpoint{0.000000in}{0.000000in}}{\pgfqpoint{0.048611in}{0.000000in}}{%
\pgfpathmoveto{\pgfqpoint{0.000000in}{0.000000in}}%
\pgfpathlineto{\pgfqpoint{0.048611in}{0.000000in}}%
\pgfusepath{stroke,fill}%
}%
\begin{pgfscope}%
\pgfsys@transformshift{2.183372in}{0.793610in}%
\pgfsys@useobject{currentmarker}{}%
\end{pgfscope}%
\end{pgfscope}%
\begin{pgfscope}%
\pgftext[x=2.280594in,y=0.745782in,left,base]{\rmfamily\fontsize{10.000000}{12.000000}\selectfont \(\displaystyle 10\)}%
\end{pgfscope}%
\begin{pgfscope}%
\pgfsetbuttcap%
\pgfsetroundjoin%
\definecolor{currentfill}{rgb}{0.000000,0.000000,0.000000}%
\pgfsetfillcolor{currentfill}%
\pgfsetlinewidth{0.803000pt}%
\definecolor{currentstroke}{rgb}{0.000000,0.000000,0.000000}%
\pgfsetstrokecolor{currentstroke}%
\pgfsetdash{}{0pt}%
\pgfsys@defobject{currentmarker}{\pgfqpoint{0.000000in}{0.000000in}}{\pgfqpoint{0.048611in}{0.000000in}}{%
\pgfpathmoveto{\pgfqpoint{0.000000in}{0.000000in}}%
\pgfpathlineto{\pgfqpoint{0.048611in}{0.000000in}}%
\pgfusepath{stroke,fill}%
}%
\begin{pgfscope}%
\pgfsys@transformshift{2.183372in}{1.267343in}%
\pgfsys@useobject{currentmarker}{}%
\end{pgfscope}%
\end{pgfscope}%
\begin{pgfscope}%
\pgftext[x=2.280594in,y=1.219515in,left,base]{\rmfamily\fontsize{10.000000}{12.000000}\selectfont \(\displaystyle 20\)}%
\end{pgfscope}%
\begin{pgfscope}%
\pgfsetbuttcap%
\pgfsetroundjoin%
\definecolor{currentfill}{rgb}{0.000000,0.000000,0.000000}%
\pgfsetfillcolor{currentfill}%
\pgfsetlinewidth{0.803000pt}%
\definecolor{currentstroke}{rgb}{0.000000,0.000000,0.000000}%
\pgfsetstrokecolor{currentstroke}%
\pgfsetdash{}{0pt}%
\pgfsys@defobject{currentmarker}{\pgfqpoint{0.000000in}{0.000000in}}{\pgfqpoint{0.048611in}{0.000000in}}{%
\pgfpathmoveto{\pgfqpoint{0.000000in}{0.000000in}}%
\pgfpathlineto{\pgfqpoint{0.048611in}{0.000000in}}%
\pgfusepath{stroke,fill}%
}%
\begin{pgfscope}%
\pgfsys@transformshift{2.183372in}{1.741076in}%
\pgfsys@useobject{currentmarker}{}%
\end{pgfscope}%
\end{pgfscope}%
\begin{pgfscope}%
\pgftext[x=2.280594in,y=1.693248in,left,base]{\rmfamily\fontsize{10.000000}{12.000000}\selectfont \(\displaystyle 30\)}%
\end{pgfscope}%
\begin{pgfscope}%
\pgfsetbuttcap%
\pgfsetroundjoin%
\definecolor{currentfill}{rgb}{0.000000,0.000000,0.000000}%
\pgfsetfillcolor{currentfill}%
\pgfsetlinewidth{0.803000pt}%
\definecolor{currentstroke}{rgb}{0.000000,0.000000,0.000000}%
\pgfsetstrokecolor{currentstroke}%
\pgfsetdash{}{0pt}%
\pgfsys@defobject{currentmarker}{\pgfqpoint{0.000000in}{0.000000in}}{\pgfqpoint{0.048611in}{0.000000in}}{%
\pgfpathmoveto{\pgfqpoint{0.000000in}{0.000000in}}%
\pgfpathlineto{\pgfqpoint{0.048611in}{0.000000in}}%
\pgfusepath{stroke,fill}%
}%
\begin{pgfscope}%
\pgfsys@transformshift{2.183372in}{2.214809in}%
\pgfsys@useobject{currentmarker}{}%
\end{pgfscope}%
\end{pgfscope}%
\begin{pgfscope}%
\pgftext[x=2.280594in,y=2.166981in,left,base]{\rmfamily\fontsize{10.000000}{12.000000}\selectfont \(\displaystyle 40\)}%
\end{pgfscope}%
\begin{pgfscope}%
\pgfsetbuttcap%
\pgfsetroundjoin%
\definecolor{currentfill}{rgb}{0.000000,0.000000,0.000000}%
\pgfsetfillcolor{currentfill}%
\pgfsetlinewidth{0.803000pt}%
\definecolor{currentstroke}{rgb}{0.000000,0.000000,0.000000}%
\pgfsetstrokecolor{currentstroke}%
\pgfsetdash{}{0pt}%
\pgfsys@defobject{currentmarker}{\pgfqpoint{0.000000in}{0.000000in}}{\pgfqpoint{0.048611in}{0.000000in}}{%
\pgfpathmoveto{\pgfqpoint{0.000000in}{0.000000in}}%
\pgfpathlineto{\pgfqpoint{0.048611in}{0.000000in}}%
\pgfusepath{stroke,fill}%
}%
\begin{pgfscope}%
\pgfsys@transformshift{2.183372in}{2.688541in}%
\pgfsys@useobject{currentmarker}{}%
\end{pgfscope}%
\end{pgfscope}%
\begin{pgfscope}%
\pgftext[x=2.280594in,y=2.640714in,left,base]{\rmfamily\fontsize{10.000000}{12.000000}\selectfont \(\displaystyle 50\)}%
\end{pgfscope}%
\begin{pgfscope}%
\pgfsetbuttcap%
\pgfsetmiterjoin%
\pgfsetlinewidth{0.803000pt}%
\definecolor{currentstroke}{rgb}{0.000000,0.000000,0.000000}%
\pgfsetstrokecolor{currentstroke}%
\pgfsetdash{}{0pt}%
\pgfpathmoveto{\pgfqpoint{2.053095in}{0.319877in}}%
\pgfpathlineto{\pgfqpoint{2.053095in}{0.330055in}}%
\pgfpathlineto{\pgfqpoint{2.053095in}{2.915230in}}%
\pgfpathlineto{\pgfqpoint{2.053095in}{2.925408in}}%
\pgfpathlineto{\pgfqpoint{2.183372in}{2.925408in}}%
\pgfpathlineto{\pgfqpoint{2.183372in}{2.915230in}}%
\pgfpathlineto{\pgfqpoint{2.183372in}{0.330055in}}%
\pgfpathlineto{\pgfqpoint{2.183372in}{0.319877in}}%
\pgfpathclose%
\pgfusepath{stroke}%
\end{pgfscope}%
\end{pgfpicture}%
\makeatother%
\endgroup%

	\vspace*{-0.4cm}
	\caption{200 K. Bin size $0.011e$}
	\end{subfigure}
	\quad
	\begin{subfigure}[b]{0.45\textwidth}
	\hspace*{-0.4cm}
	%% Creator: Matplotlib, PGF backend
%%
%% To include the figure in your LaTeX document, write
%%   \input{<filename>.pgf}
%%
%% Make sure the required packages are loaded in your preamble
%%   \usepackage{pgf}
%%
%% Figures using additional raster images can only be included by \input if
%% they are in the same directory as the main LaTeX file. For loading figures
%% from other directories you can use the `import` package
%%   \usepackage{import}
%% and then include the figures with
%%   \import{<path to file>}{<filename>.pgf}
%%
%% Matplotlib used the following preamble
%%   \usepackage[utf8x]{inputenc}
%%   \usepackage[T1]{fontenc}
%%
\begingroup%
\makeatletter%
\begin{pgfpicture}%
\pgfpathrectangle{\pgfpointorigin}{\pgfqpoint{2.519483in}{3.060408in}}%
\pgfusepath{use as bounding box, clip}%
\begin{pgfscope}%
\pgfsetbuttcap%
\pgfsetmiterjoin%
\definecolor{currentfill}{rgb}{1.000000,1.000000,1.000000}%
\pgfsetfillcolor{currentfill}%
\pgfsetlinewidth{0.000000pt}%
\definecolor{currentstroke}{rgb}{1.000000,1.000000,1.000000}%
\pgfsetstrokecolor{currentstroke}%
\pgfsetdash{}{0pt}%
\pgfpathmoveto{\pgfqpoint{0.000000in}{0.000000in}}%
\pgfpathlineto{\pgfqpoint{2.519483in}{0.000000in}}%
\pgfpathlineto{\pgfqpoint{2.519483in}{3.060408in}}%
\pgfpathlineto{\pgfqpoint{0.000000in}{3.060408in}}%
\pgfpathclose%
\pgfusepath{fill}%
\end{pgfscope}%
\begin{pgfscope}%
\pgfsetbuttcap%
\pgfsetmiterjoin%
\definecolor{currentfill}{rgb}{1.000000,1.000000,1.000000}%
\pgfsetfillcolor{currentfill}%
\pgfsetlinewidth{0.000000pt}%
\definecolor{currentstroke}{rgb}{0.000000,0.000000,0.000000}%
\pgfsetstrokecolor{currentstroke}%
\pgfsetstrokeopacity{0.000000}%
\pgfsetdash{}{0pt}%
\pgfpathmoveto{\pgfqpoint{0.374692in}{0.319877in}}%
\pgfpathlineto{\pgfqpoint{1.954366in}{0.319877in}}%
\pgfpathlineto{\pgfqpoint{1.954366in}{2.925408in}}%
\pgfpathlineto{\pgfqpoint{0.374692in}{2.925408in}}%
\pgfpathclose%
\pgfusepath{fill}%
\end{pgfscope}%
\begin{pgfscope}%
\pgfpathrectangle{\pgfqpoint{0.374692in}{0.319877in}}{\pgfqpoint{1.579674in}{2.605531in}} %
\pgfusepath{clip}%
\pgfsys@transformshift{0.374692in}{0.319877in}%
\pgftext[left,bottom]{\pgfimage[interpolate=true,width=1.580000in,height=2.610000in]{RnnNorm_vs_dq_Ti_300K-img0.png}}%
\end{pgfscope}%
\begin{pgfscope}%
\pgfpathrectangle{\pgfqpoint{0.374692in}{0.319877in}}{\pgfqpoint{1.579674in}{2.605531in}} %
\pgfusepath{clip}%
\pgfsetbuttcap%
\pgfsetroundjoin%
\definecolor{currentfill}{rgb}{1.000000,0.752941,0.796078}%
\pgfsetfillcolor{currentfill}%
\pgfsetlinewidth{1.003750pt}%
\definecolor{currentstroke}{rgb}{1.000000,0.752941,0.796078}%
\pgfsetstrokecolor{currentstroke}%
\pgfsetdash{}{0pt}%
\pgfpathmoveto{\pgfqpoint{0.733709in}{1.881614in}}%
\pgfpathcurveto{\pgfqpoint{0.744759in}{1.881614in}}{\pgfqpoint{0.755358in}{1.886004in}}{\pgfqpoint{0.763171in}{1.893818in}}%
\pgfpathcurveto{\pgfqpoint{0.770985in}{1.901632in}}{\pgfqpoint{0.775375in}{1.912231in}}{\pgfqpoint{0.775375in}{1.923281in}}%
\pgfpathcurveto{\pgfqpoint{0.775375in}{1.934331in}}{\pgfqpoint{0.770985in}{1.944930in}}{\pgfqpoint{0.763171in}{1.952744in}}%
\pgfpathcurveto{\pgfqpoint{0.755358in}{1.960557in}}{\pgfqpoint{0.744759in}{1.964947in}}{\pgfqpoint{0.733709in}{1.964947in}}%
\pgfpathcurveto{\pgfqpoint{0.722659in}{1.964947in}}{\pgfqpoint{0.712060in}{1.960557in}}{\pgfqpoint{0.704246in}{1.952744in}}%
\pgfpathcurveto{\pgfqpoint{0.696432in}{1.944930in}}{\pgfqpoint{0.692042in}{1.934331in}}{\pgfqpoint{0.692042in}{1.923281in}}%
\pgfpathcurveto{\pgfqpoint{0.692042in}{1.912231in}}{\pgfqpoint{0.696432in}{1.901632in}}{\pgfqpoint{0.704246in}{1.893818in}}%
\pgfpathcurveto{\pgfqpoint{0.712060in}{1.886004in}}{\pgfqpoint{0.722659in}{1.881614in}}{\pgfqpoint{0.733709in}{1.881614in}}%
\pgfpathclose%
\pgfusepath{stroke,fill}%
\end{pgfscope}%
\begin{pgfscope}%
\pgfpathrectangle{\pgfqpoint{0.374692in}{0.319877in}}{\pgfqpoint{1.579674in}{2.605531in}} %
\pgfusepath{clip}%
\pgfsetbuttcap%
\pgfsetroundjoin%
\definecolor{currentfill}{rgb}{1.000000,0.752941,0.796078}%
\pgfsetfillcolor{currentfill}%
\pgfsetlinewidth{1.003750pt}%
\definecolor{currentstroke}{rgb}{1.000000,0.752941,0.796078}%
\pgfsetstrokecolor{currentstroke}%
\pgfsetdash{}{0pt}%
\pgfpathmoveto{\pgfqpoint{0.877315in}{1.569841in}}%
\pgfpathcurveto{\pgfqpoint{0.888366in}{1.569841in}}{\pgfqpoint{0.898965in}{1.574231in}}{\pgfqpoint{0.906778in}{1.582045in}}%
\pgfpathcurveto{\pgfqpoint{0.914592in}{1.589859in}}{\pgfqpoint{0.918982in}{1.600458in}}{\pgfqpoint{0.918982in}{1.611508in}}%
\pgfpathcurveto{\pgfqpoint{0.918982in}{1.622558in}}{\pgfqpoint{0.914592in}{1.633157in}}{\pgfqpoint{0.906778in}{1.640971in}}%
\pgfpathcurveto{\pgfqpoint{0.898965in}{1.648784in}}{\pgfqpoint{0.888366in}{1.653175in}}{\pgfqpoint{0.877315in}{1.653175in}}%
\pgfpathcurveto{\pgfqpoint{0.866265in}{1.653175in}}{\pgfqpoint{0.855666in}{1.648784in}}{\pgfqpoint{0.847853in}{1.640971in}}%
\pgfpathcurveto{\pgfqpoint{0.840039in}{1.633157in}}{\pgfqpoint{0.835649in}{1.622558in}}{\pgfqpoint{0.835649in}{1.611508in}}%
\pgfpathcurveto{\pgfqpoint{0.835649in}{1.600458in}}{\pgfqpoint{0.840039in}{1.589859in}}{\pgfqpoint{0.847853in}{1.582045in}}%
\pgfpathcurveto{\pgfqpoint{0.855666in}{1.574231in}}{\pgfqpoint{0.866265in}{1.569841in}}{\pgfqpoint{0.877315in}{1.569841in}}%
\pgfpathclose%
\pgfusepath{stroke,fill}%
\end{pgfscope}%
\begin{pgfscope}%
\pgfpathrectangle{\pgfqpoint{0.374692in}{0.319877in}}{\pgfqpoint{1.579674in}{2.605531in}} %
\pgfusepath{clip}%
\pgfsetbuttcap%
\pgfsetroundjoin%
\definecolor{currentfill}{rgb}{1.000000,0.752941,0.796078}%
\pgfsetfillcolor{currentfill}%
\pgfsetlinewidth{1.003750pt}%
\definecolor{currentstroke}{rgb}{1.000000,0.752941,0.796078}%
\pgfsetstrokecolor{currentstroke}%
\pgfsetdash{}{0pt}%
\pgfpathmoveto{\pgfqpoint{1.020922in}{1.522788in}}%
\pgfpathcurveto{\pgfqpoint{1.031972in}{1.522788in}}{\pgfqpoint{1.042571in}{1.527178in}}{\pgfqpoint{1.050385in}{1.534992in}}%
\pgfpathcurveto{\pgfqpoint{1.058199in}{1.542805in}}{\pgfqpoint{1.062589in}{1.553404in}}{\pgfqpoint{1.062589in}{1.564455in}}%
\pgfpathcurveto{\pgfqpoint{1.062589in}{1.575505in}}{\pgfqpoint{1.058199in}{1.586104in}}{\pgfqpoint{1.050385in}{1.593917in}}%
\pgfpathcurveto{\pgfqpoint{1.042571in}{1.601731in}}{\pgfqpoint{1.031972in}{1.606121in}}{\pgfqpoint{1.020922in}{1.606121in}}%
\pgfpathcurveto{\pgfqpoint{1.009872in}{1.606121in}}{\pgfqpoint{0.999273in}{1.601731in}}{\pgfqpoint{0.991459in}{1.593917in}}%
\pgfpathcurveto{\pgfqpoint{0.983646in}{1.586104in}}{\pgfqpoint{0.979255in}{1.575505in}}{\pgfqpoint{0.979255in}{1.564455in}}%
\pgfpathcurveto{\pgfqpoint{0.979255in}{1.553404in}}{\pgfqpoint{0.983646in}{1.542805in}}{\pgfqpoint{0.991459in}{1.534992in}}%
\pgfpathcurveto{\pgfqpoint{0.999273in}{1.527178in}}{\pgfqpoint{1.009872in}{1.522788in}}{\pgfqpoint{1.020922in}{1.522788in}}%
\pgfpathclose%
\pgfusepath{stroke,fill}%
\end{pgfscope}%
\begin{pgfscope}%
\pgfpathrectangle{\pgfqpoint{0.374692in}{0.319877in}}{\pgfqpoint{1.579674in}{2.605531in}} %
\pgfusepath{clip}%
\pgfsetbuttcap%
\pgfsetroundjoin%
\definecolor{currentfill}{rgb}{1.000000,0.752941,0.796078}%
\pgfsetfillcolor{currentfill}%
\pgfsetlinewidth{1.003750pt}%
\definecolor{currentstroke}{rgb}{1.000000,0.752941,0.796078}%
\pgfsetstrokecolor{currentstroke}%
\pgfsetdash{}{0pt}%
\pgfpathmoveto{\pgfqpoint{1.164529in}{1.421685in}}%
\pgfpathcurveto{\pgfqpoint{1.175579in}{1.421685in}}{\pgfqpoint{1.186178in}{1.426075in}}{\pgfqpoint{1.193992in}{1.433889in}}%
\pgfpathcurveto{\pgfqpoint{1.201805in}{1.441702in}}{\pgfqpoint{1.206196in}{1.452301in}}{\pgfqpoint{1.206196in}{1.463351in}}%
\pgfpathcurveto{\pgfqpoint{1.206196in}{1.474401in}}{\pgfqpoint{1.201805in}{1.485000in}}{\pgfqpoint{1.193992in}{1.492814in}}%
\pgfpathcurveto{\pgfqpoint{1.186178in}{1.500628in}}{\pgfqpoint{1.175579in}{1.505018in}}{\pgfqpoint{1.164529in}{1.505018in}}%
\pgfpathcurveto{\pgfqpoint{1.153479in}{1.505018in}}{\pgfqpoint{1.142880in}{1.500628in}}{\pgfqpoint{1.135066in}{1.492814in}}%
\pgfpathcurveto{\pgfqpoint{1.127252in}{1.485000in}}{\pgfqpoint{1.122862in}{1.474401in}}{\pgfqpoint{1.122862in}{1.463351in}}%
\pgfpathcurveto{\pgfqpoint{1.122862in}{1.452301in}}{\pgfqpoint{1.127252in}{1.441702in}}{\pgfqpoint{1.135066in}{1.433889in}}%
\pgfpathcurveto{\pgfqpoint{1.142880in}{1.426075in}}{\pgfqpoint{1.153479in}{1.421685in}}{\pgfqpoint{1.164529in}{1.421685in}}%
\pgfpathclose%
\pgfusepath{stroke,fill}%
\end{pgfscope}%
\begin{pgfscope}%
\pgfpathrectangle{\pgfqpoint{0.374692in}{0.319877in}}{\pgfqpoint{1.579674in}{2.605531in}} %
\pgfusepath{clip}%
\pgfsetbuttcap%
\pgfsetroundjoin%
\definecolor{currentfill}{rgb}{1.000000,0.752941,0.796078}%
\pgfsetfillcolor{currentfill}%
\pgfsetlinewidth{1.003750pt}%
\definecolor{currentstroke}{rgb}{1.000000,0.752941,0.796078}%
\pgfsetstrokecolor{currentstroke}%
\pgfsetdash{}{0pt}%
\pgfpathmoveto{\pgfqpoint{1.308136in}{1.327104in}}%
\pgfpathcurveto{\pgfqpoint{1.319186in}{1.327104in}}{\pgfqpoint{1.329785in}{1.331494in}}{\pgfqpoint{1.337598in}{1.339308in}}%
\pgfpathcurveto{\pgfqpoint{1.345412in}{1.347121in}}{\pgfqpoint{1.349802in}{1.357720in}}{\pgfqpoint{1.349802in}{1.368770in}}%
\pgfpathcurveto{\pgfqpoint{1.349802in}{1.379821in}}{\pgfqpoint{1.345412in}{1.390420in}}{\pgfqpoint{1.337598in}{1.398233in}}%
\pgfpathcurveto{\pgfqpoint{1.329785in}{1.406047in}}{\pgfqpoint{1.319186in}{1.410437in}}{\pgfqpoint{1.308136in}{1.410437in}}%
\pgfpathcurveto{\pgfqpoint{1.297085in}{1.410437in}}{\pgfqpoint{1.286486in}{1.406047in}}{\pgfqpoint{1.278673in}{1.398233in}}%
\pgfpathcurveto{\pgfqpoint{1.270859in}{1.390420in}}{\pgfqpoint{1.266469in}{1.379821in}}{\pgfqpoint{1.266469in}{1.368770in}}%
\pgfpathcurveto{\pgfqpoint{1.266469in}{1.357720in}}{\pgfqpoint{1.270859in}{1.347121in}}{\pgfqpoint{1.278673in}{1.339308in}}%
\pgfpathcurveto{\pgfqpoint{1.286486in}{1.331494in}}{\pgfqpoint{1.297085in}{1.327104in}}{\pgfqpoint{1.308136in}{1.327104in}}%
\pgfpathclose%
\pgfusepath{stroke,fill}%
\end{pgfscope}%
\begin{pgfscope}%
\pgfpathrectangle{\pgfqpoint{0.374692in}{0.319877in}}{\pgfqpoint{1.579674in}{2.605531in}} %
\pgfusepath{clip}%
\pgfsetbuttcap%
\pgfsetroundjoin%
\definecolor{currentfill}{rgb}{1.000000,0.752941,0.796078}%
\pgfsetfillcolor{currentfill}%
\pgfsetlinewidth{1.003750pt}%
\definecolor{currentstroke}{rgb}{1.000000,0.752941,0.796078}%
\pgfsetstrokecolor{currentstroke}%
\pgfsetdash{}{0pt}%
\pgfpathmoveto{\pgfqpoint{1.451742in}{1.266022in}}%
\pgfpathcurveto{\pgfqpoint{1.462792in}{1.266022in}}{\pgfqpoint{1.473391in}{1.270412in}}{\pgfqpoint{1.481205in}{1.278226in}}%
\pgfpathcurveto{\pgfqpoint{1.489019in}{1.286039in}}{\pgfqpoint{1.493409in}{1.296638in}}{\pgfqpoint{1.493409in}{1.307688in}}%
\pgfpathcurveto{\pgfqpoint{1.493409in}{1.318739in}}{\pgfqpoint{1.489019in}{1.329338in}}{\pgfqpoint{1.481205in}{1.337151in}}%
\pgfpathcurveto{\pgfqpoint{1.473391in}{1.344965in}}{\pgfqpoint{1.462792in}{1.349355in}}{\pgfqpoint{1.451742in}{1.349355in}}%
\pgfpathcurveto{\pgfqpoint{1.440692in}{1.349355in}}{\pgfqpoint{1.430093in}{1.344965in}}{\pgfqpoint{1.422279in}{1.337151in}}%
\pgfpathcurveto{\pgfqpoint{1.414466in}{1.329338in}}{\pgfqpoint{1.410076in}{1.318739in}}{\pgfqpoint{1.410076in}{1.307688in}}%
\pgfpathcurveto{\pgfqpoint{1.410076in}{1.296638in}}{\pgfqpoint{1.414466in}{1.286039in}}{\pgfqpoint{1.422279in}{1.278226in}}%
\pgfpathcurveto{\pgfqpoint{1.430093in}{1.270412in}}{\pgfqpoint{1.440692in}{1.266022in}}{\pgfqpoint{1.451742in}{1.266022in}}%
\pgfpathclose%
\pgfusepath{stroke,fill}%
\end{pgfscope}%
\begin{pgfscope}%
\pgfsetbuttcap%
\pgfsetroundjoin%
\definecolor{currentfill}{rgb}{0.000000,0.000000,0.000000}%
\pgfsetfillcolor{currentfill}%
\pgfsetlinewidth{0.803000pt}%
\definecolor{currentstroke}{rgb}{0.000000,0.000000,0.000000}%
\pgfsetstrokecolor{currentstroke}%
\pgfsetdash{}{0pt}%
\pgfsys@defobject{currentmarker}{\pgfqpoint{0.000000in}{-0.048611in}}{\pgfqpoint{0.000000in}{0.000000in}}{%
\pgfpathmoveto{\pgfqpoint{0.000000in}{0.000000in}}%
\pgfpathlineto{\pgfqpoint{0.000000in}{-0.048611in}}%
\pgfusepath{stroke,fill}%
}%
\begin{pgfscope}%
\pgfsys@transformshift{0.670881in}{0.319877in}%
\pgfsys@useobject{currentmarker}{}%
\end{pgfscope}%
\end{pgfscope}%
\begin{pgfscope}%
\pgftext[x=0.670881in,y=0.222655in,,top]{\rmfamily\fontsize{10.000000}{12.000000}\selectfont \(\displaystyle -0.05\)}%
\end{pgfscope}%
\begin{pgfscope}%
\pgfsetbuttcap%
\pgfsetroundjoin%
\definecolor{currentfill}{rgb}{0.000000,0.000000,0.000000}%
\pgfsetfillcolor{currentfill}%
\pgfsetlinewidth{0.803000pt}%
\definecolor{currentstroke}{rgb}{0.000000,0.000000,0.000000}%
\pgfsetstrokecolor{currentstroke}%
\pgfsetdash{}{0pt}%
\pgfsys@defobject{currentmarker}{\pgfqpoint{0.000000in}{-0.048611in}}{\pgfqpoint{0.000000in}{0.000000in}}{%
\pgfpathmoveto{\pgfqpoint{0.000000in}{0.000000in}}%
\pgfpathlineto{\pgfqpoint{0.000000in}{-0.048611in}}%
\pgfusepath{stroke,fill}%
}%
\begin{pgfscope}%
\pgfsys@transformshift{1.164529in}{0.319877in}%
\pgfsys@useobject{currentmarker}{}%
\end{pgfscope}%
\end{pgfscope}%
\begin{pgfscope}%
\pgftext[x=1.164529in,y=0.222655in,,top]{\rmfamily\fontsize{10.000000}{12.000000}\selectfont \(\displaystyle 0.00\)}%
\end{pgfscope}%
\begin{pgfscope}%
\pgfsetbuttcap%
\pgfsetroundjoin%
\definecolor{currentfill}{rgb}{0.000000,0.000000,0.000000}%
\pgfsetfillcolor{currentfill}%
\pgfsetlinewidth{0.803000pt}%
\definecolor{currentstroke}{rgb}{0.000000,0.000000,0.000000}%
\pgfsetstrokecolor{currentstroke}%
\pgfsetdash{}{0pt}%
\pgfsys@defobject{currentmarker}{\pgfqpoint{0.000000in}{-0.048611in}}{\pgfqpoint{0.000000in}{0.000000in}}{%
\pgfpathmoveto{\pgfqpoint{0.000000in}{0.000000in}}%
\pgfpathlineto{\pgfqpoint{0.000000in}{-0.048611in}}%
\pgfusepath{stroke,fill}%
}%
\begin{pgfscope}%
\pgfsys@transformshift{1.658177in}{0.319877in}%
\pgfsys@useobject{currentmarker}{}%
\end{pgfscope}%
\end{pgfscope}%
\begin{pgfscope}%
\pgftext[x=1.658177in,y=0.222655in,,top]{\rmfamily\fontsize{10.000000}{12.000000}\selectfont \(\displaystyle 0.05\)}%
\end{pgfscope}%
\begin{pgfscope}%
\pgfsetbuttcap%
\pgfsetroundjoin%
\definecolor{currentfill}{rgb}{0.000000,0.000000,0.000000}%
\pgfsetfillcolor{currentfill}%
\pgfsetlinewidth{0.803000pt}%
\definecolor{currentstroke}{rgb}{0.000000,0.000000,0.000000}%
\pgfsetstrokecolor{currentstroke}%
\pgfsetdash{}{0pt}%
\pgfsys@defobject{currentmarker}{\pgfqpoint{-0.048611in}{0.000000in}}{\pgfqpoint{0.000000in}{0.000000in}}{%
\pgfpathmoveto{\pgfqpoint{0.000000in}{0.000000in}}%
\pgfpathlineto{\pgfqpoint{-0.048611in}{0.000000in}}%
\pgfusepath{stroke,fill}%
}%
\begin{pgfscope}%
\pgfsys@transformshift{0.374692in}{0.622908in}%
\pgfsys@useobject{currentmarker}{}%
\end{pgfscope}%
\end{pgfscope}%
\begin{pgfscope}%
\pgftext[x=0.100000in,y=0.575080in,left,base]{\rmfamily\fontsize{10.000000}{12.000000}\selectfont \(\displaystyle 3.6\)}%
\end{pgfscope}%
\begin{pgfscope}%
\pgfsetbuttcap%
\pgfsetroundjoin%
\definecolor{currentfill}{rgb}{0.000000,0.000000,0.000000}%
\pgfsetfillcolor{currentfill}%
\pgfsetlinewidth{0.803000pt}%
\definecolor{currentstroke}{rgb}{0.000000,0.000000,0.000000}%
\pgfsetstrokecolor{currentstroke}%
\pgfsetdash{}{0pt}%
\pgfsys@defobject{currentmarker}{\pgfqpoint{-0.048611in}{0.000000in}}{\pgfqpoint{0.000000in}{0.000000in}}{%
\pgfpathmoveto{\pgfqpoint{0.000000in}{0.000000in}}%
\pgfpathlineto{\pgfqpoint{-0.048611in}{0.000000in}}%
\pgfusepath{stroke,fill}%
}%
\begin{pgfscope}%
\pgfsys@transformshift{0.374692in}{1.067280in}%
\pgfsys@useobject{currentmarker}{}%
\end{pgfscope}%
\end{pgfscope}%
\begin{pgfscope}%
\pgftext[x=0.100000in,y=1.019453in,left,base]{\rmfamily\fontsize{10.000000}{12.000000}\selectfont \(\displaystyle 3.7\)}%
\end{pgfscope}%
\begin{pgfscope}%
\pgfsetbuttcap%
\pgfsetroundjoin%
\definecolor{currentfill}{rgb}{0.000000,0.000000,0.000000}%
\pgfsetfillcolor{currentfill}%
\pgfsetlinewidth{0.803000pt}%
\definecolor{currentstroke}{rgb}{0.000000,0.000000,0.000000}%
\pgfsetstrokecolor{currentstroke}%
\pgfsetdash{}{0pt}%
\pgfsys@defobject{currentmarker}{\pgfqpoint{-0.048611in}{0.000000in}}{\pgfqpoint{0.000000in}{0.000000in}}{%
\pgfpathmoveto{\pgfqpoint{0.000000in}{0.000000in}}%
\pgfpathlineto{\pgfqpoint{-0.048611in}{0.000000in}}%
\pgfusepath{stroke,fill}%
}%
\begin{pgfscope}%
\pgfsys@transformshift{0.374692in}{1.511652in}%
\pgfsys@useobject{currentmarker}{}%
\end{pgfscope}%
\end{pgfscope}%
\begin{pgfscope}%
\pgftext[x=0.100000in,y=1.463825in,left,base]{\rmfamily\fontsize{10.000000}{12.000000}\selectfont \(\displaystyle 3.8\)}%
\end{pgfscope}%
\begin{pgfscope}%
\pgfsetbuttcap%
\pgfsetroundjoin%
\definecolor{currentfill}{rgb}{0.000000,0.000000,0.000000}%
\pgfsetfillcolor{currentfill}%
\pgfsetlinewidth{0.803000pt}%
\definecolor{currentstroke}{rgb}{0.000000,0.000000,0.000000}%
\pgfsetstrokecolor{currentstroke}%
\pgfsetdash{}{0pt}%
\pgfsys@defobject{currentmarker}{\pgfqpoint{-0.048611in}{0.000000in}}{\pgfqpoint{0.000000in}{0.000000in}}{%
\pgfpathmoveto{\pgfqpoint{0.000000in}{0.000000in}}%
\pgfpathlineto{\pgfqpoint{-0.048611in}{0.000000in}}%
\pgfusepath{stroke,fill}%
}%
\begin{pgfscope}%
\pgfsys@transformshift{0.374692in}{1.956024in}%
\pgfsys@useobject{currentmarker}{}%
\end{pgfscope}%
\end{pgfscope}%
\begin{pgfscope}%
\pgftext[x=0.100000in,y=1.908197in,left,base]{\rmfamily\fontsize{10.000000}{12.000000}\selectfont \(\displaystyle 3.9\)}%
\end{pgfscope}%
\begin{pgfscope}%
\pgfsetbuttcap%
\pgfsetroundjoin%
\definecolor{currentfill}{rgb}{0.000000,0.000000,0.000000}%
\pgfsetfillcolor{currentfill}%
\pgfsetlinewidth{0.803000pt}%
\definecolor{currentstroke}{rgb}{0.000000,0.000000,0.000000}%
\pgfsetstrokecolor{currentstroke}%
\pgfsetdash{}{0pt}%
\pgfsys@defobject{currentmarker}{\pgfqpoint{-0.048611in}{0.000000in}}{\pgfqpoint{0.000000in}{0.000000in}}{%
\pgfpathmoveto{\pgfqpoint{0.000000in}{0.000000in}}%
\pgfpathlineto{\pgfqpoint{-0.048611in}{0.000000in}}%
\pgfusepath{stroke,fill}%
}%
\begin{pgfscope}%
\pgfsys@transformshift{0.374692in}{2.400396in}%
\pgfsys@useobject{currentmarker}{}%
\end{pgfscope}%
\end{pgfscope}%
\begin{pgfscope}%
\pgftext[x=0.100000in,y=2.352569in,left,base]{\rmfamily\fontsize{10.000000}{12.000000}\selectfont \(\displaystyle 4.0\)}%
\end{pgfscope}%
\begin{pgfscope}%
\pgfsetbuttcap%
\pgfsetroundjoin%
\definecolor{currentfill}{rgb}{0.000000,0.000000,0.000000}%
\pgfsetfillcolor{currentfill}%
\pgfsetlinewidth{0.803000pt}%
\definecolor{currentstroke}{rgb}{0.000000,0.000000,0.000000}%
\pgfsetstrokecolor{currentstroke}%
\pgfsetdash{}{0pt}%
\pgfsys@defobject{currentmarker}{\pgfqpoint{-0.048611in}{0.000000in}}{\pgfqpoint{0.000000in}{0.000000in}}{%
\pgfpathmoveto{\pgfqpoint{0.000000in}{0.000000in}}%
\pgfpathlineto{\pgfqpoint{-0.048611in}{0.000000in}}%
\pgfusepath{stroke,fill}%
}%
\begin{pgfscope}%
\pgfsys@transformshift{0.374692in}{2.844769in}%
\pgfsys@useobject{currentmarker}{}%
\end{pgfscope}%
\end{pgfscope}%
\begin{pgfscope}%
\pgftext[x=0.100000in,y=2.796941in,left,base]{\rmfamily\fontsize{10.000000}{12.000000}\selectfont \(\displaystyle 4.1\)}%
\end{pgfscope}%
\begin{pgfscope}%
\pgfsetrectcap%
\pgfsetmiterjoin%
\pgfsetlinewidth{0.803000pt}%
\definecolor{currentstroke}{rgb}{0.000000,0.000000,0.000000}%
\pgfsetstrokecolor{currentstroke}%
\pgfsetdash{}{0pt}%
\pgfpathmoveto{\pgfqpoint{0.374692in}{0.319877in}}%
\pgfpathlineto{\pgfqpoint{0.374692in}{2.925408in}}%
\pgfusepath{stroke}%
\end{pgfscope}%
\begin{pgfscope}%
\pgfsetrectcap%
\pgfsetmiterjoin%
\pgfsetlinewidth{0.803000pt}%
\definecolor{currentstroke}{rgb}{0.000000,0.000000,0.000000}%
\pgfsetstrokecolor{currentstroke}%
\pgfsetdash{}{0pt}%
\pgfpathmoveto{\pgfqpoint{1.954366in}{0.319877in}}%
\pgfpathlineto{\pgfqpoint{1.954366in}{2.925408in}}%
\pgfusepath{stroke}%
\end{pgfscope}%
\begin{pgfscope}%
\pgfsetrectcap%
\pgfsetmiterjoin%
\pgfsetlinewidth{0.803000pt}%
\definecolor{currentstroke}{rgb}{0.000000,0.000000,0.000000}%
\pgfsetstrokecolor{currentstroke}%
\pgfsetdash{}{0pt}%
\pgfpathmoveto{\pgfqpoint{0.374692in}{0.319877in}}%
\pgfpathlineto{\pgfqpoint{1.954366in}{0.319877in}}%
\pgfusepath{stroke}%
\end{pgfscope}%
\begin{pgfscope}%
\pgfsetrectcap%
\pgfsetmiterjoin%
\pgfsetlinewidth{0.803000pt}%
\definecolor{currentstroke}{rgb}{0.000000,0.000000,0.000000}%
\pgfsetstrokecolor{currentstroke}%
\pgfsetdash{}{0pt}%
\pgfpathmoveto{\pgfqpoint{0.374692in}{2.925408in}}%
\pgfpathlineto{\pgfqpoint{1.954366in}{2.925408in}}%
\pgfusepath{stroke}%
\end{pgfscope}%
\begin{pgfscope}%
\pgfpathrectangle{\pgfqpoint{2.053095in}{0.319877in}}{\pgfqpoint{0.130277in}{2.605531in}} %
\pgfusepath{clip}%
\pgfsetbuttcap%
\pgfsetmiterjoin%
\definecolor{currentfill}{rgb}{1.000000,1.000000,1.000000}%
\pgfsetfillcolor{currentfill}%
\pgfsetlinewidth{0.010037pt}%
\definecolor{currentstroke}{rgb}{1.000000,1.000000,1.000000}%
\pgfsetstrokecolor{currentstroke}%
\pgfsetdash{}{0pt}%
\pgfpathmoveto{\pgfqpoint{2.053095in}{0.319877in}}%
\pgfpathlineto{\pgfqpoint{2.053095in}{0.330055in}}%
\pgfpathlineto{\pgfqpoint{2.053095in}{2.915230in}}%
\pgfpathlineto{\pgfqpoint{2.053095in}{2.925408in}}%
\pgfpathlineto{\pgfqpoint{2.183372in}{2.925408in}}%
\pgfpathlineto{\pgfqpoint{2.183372in}{2.915230in}}%
\pgfpathlineto{\pgfqpoint{2.183372in}{0.330055in}}%
\pgfpathlineto{\pgfqpoint{2.183372in}{0.319877in}}%
\pgfpathclose%
\pgfusepath{stroke,fill}%
\end{pgfscope}%
\begin{pgfscope}%
\pgfsys@transformshift{2.050000in}{0.320408in}%
\pgftext[left,bottom]{\pgfimage[interpolate=true,width=0.130000in,height=2.610000in]{RnnNorm_vs_dq_Ti_300K-img1.png}}%
\end{pgfscope}%
\begin{pgfscope}%
\pgfsetbuttcap%
\pgfsetroundjoin%
\definecolor{currentfill}{rgb}{0.000000,0.000000,0.000000}%
\pgfsetfillcolor{currentfill}%
\pgfsetlinewidth{0.803000pt}%
\definecolor{currentstroke}{rgb}{0.000000,0.000000,0.000000}%
\pgfsetstrokecolor{currentstroke}%
\pgfsetdash{}{0pt}%
\pgfsys@defobject{currentmarker}{\pgfqpoint{0.000000in}{0.000000in}}{\pgfqpoint{0.048611in}{0.000000in}}{%
\pgfpathmoveto{\pgfqpoint{0.000000in}{0.000000in}}%
\pgfpathlineto{\pgfqpoint{0.048611in}{0.000000in}}%
\pgfusepath{stroke,fill}%
}%
\begin{pgfscope}%
\pgfsys@transformshift{2.183372in}{0.319877in}%
\pgfsys@useobject{currentmarker}{}%
\end{pgfscope}%
\end{pgfscope}%
\begin{pgfscope}%
\pgftext[x=2.280594in,y=0.272050in,left,base]{\rmfamily\fontsize{10.000000}{12.000000}\selectfont \(\displaystyle 0\)}%
\end{pgfscope}%
\begin{pgfscope}%
\pgfsetbuttcap%
\pgfsetroundjoin%
\definecolor{currentfill}{rgb}{0.000000,0.000000,0.000000}%
\pgfsetfillcolor{currentfill}%
\pgfsetlinewidth{0.803000pt}%
\definecolor{currentstroke}{rgb}{0.000000,0.000000,0.000000}%
\pgfsetstrokecolor{currentstroke}%
\pgfsetdash{}{0pt}%
\pgfsys@defobject{currentmarker}{\pgfqpoint{0.000000in}{0.000000in}}{\pgfqpoint{0.048611in}{0.000000in}}{%
\pgfpathmoveto{\pgfqpoint{0.000000in}{0.000000in}}%
\pgfpathlineto{\pgfqpoint{0.048611in}{0.000000in}}%
\pgfusepath{stroke,fill}%
}%
\begin{pgfscope}%
\pgfsys@transformshift{2.183372in}{0.793610in}%
\pgfsys@useobject{currentmarker}{}%
\end{pgfscope}%
\end{pgfscope}%
\begin{pgfscope}%
\pgftext[x=2.280594in,y=0.745782in,left,base]{\rmfamily\fontsize{10.000000}{12.000000}\selectfont \(\displaystyle 10\)}%
\end{pgfscope}%
\begin{pgfscope}%
\pgfsetbuttcap%
\pgfsetroundjoin%
\definecolor{currentfill}{rgb}{0.000000,0.000000,0.000000}%
\pgfsetfillcolor{currentfill}%
\pgfsetlinewidth{0.803000pt}%
\definecolor{currentstroke}{rgb}{0.000000,0.000000,0.000000}%
\pgfsetstrokecolor{currentstroke}%
\pgfsetdash{}{0pt}%
\pgfsys@defobject{currentmarker}{\pgfqpoint{0.000000in}{0.000000in}}{\pgfqpoint{0.048611in}{0.000000in}}{%
\pgfpathmoveto{\pgfqpoint{0.000000in}{0.000000in}}%
\pgfpathlineto{\pgfqpoint{0.048611in}{0.000000in}}%
\pgfusepath{stroke,fill}%
}%
\begin{pgfscope}%
\pgfsys@transformshift{2.183372in}{1.267343in}%
\pgfsys@useobject{currentmarker}{}%
\end{pgfscope}%
\end{pgfscope}%
\begin{pgfscope}%
\pgftext[x=2.280594in,y=1.219515in,left,base]{\rmfamily\fontsize{10.000000}{12.000000}\selectfont \(\displaystyle 20\)}%
\end{pgfscope}%
\begin{pgfscope}%
\pgfsetbuttcap%
\pgfsetroundjoin%
\definecolor{currentfill}{rgb}{0.000000,0.000000,0.000000}%
\pgfsetfillcolor{currentfill}%
\pgfsetlinewidth{0.803000pt}%
\definecolor{currentstroke}{rgb}{0.000000,0.000000,0.000000}%
\pgfsetstrokecolor{currentstroke}%
\pgfsetdash{}{0pt}%
\pgfsys@defobject{currentmarker}{\pgfqpoint{0.000000in}{0.000000in}}{\pgfqpoint{0.048611in}{0.000000in}}{%
\pgfpathmoveto{\pgfqpoint{0.000000in}{0.000000in}}%
\pgfpathlineto{\pgfqpoint{0.048611in}{0.000000in}}%
\pgfusepath{stroke,fill}%
}%
\begin{pgfscope}%
\pgfsys@transformshift{2.183372in}{1.741076in}%
\pgfsys@useobject{currentmarker}{}%
\end{pgfscope}%
\end{pgfscope}%
\begin{pgfscope}%
\pgftext[x=2.280594in,y=1.693248in,left,base]{\rmfamily\fontsize{10.000000}{12.000000}\selectfont \(\displaystyle 30\)}%
\end{pgfscope}%
\begin{pgfscope}%
\pgfsetbuttcap%
\pgfsetroundjoin%
\definecolor{currentfill}{rgb}{0.000000,0.000000,0.000000}%
\pgfsetfillcolor{currentfill}%
\pgfsetlinewidth{0.803000pt}%
\definecolor{currentstroke}{rgb}{0.000000,0.000000,0.000000}%
\pgfsetstrokecolor{currentstroke}%
\pgfsetdash{}{0pt}%
\pgfsys@defobject{currentmarker}{\pgfqpoint{0.000000in}{0.000000in}}{\pgfqpoint{0.048611in}{0.000000in}}{%
\pgfpathmoveto{\pgfqpoint{0.000000in}{0.000000in}}%
\pgfpathlineto{\pgfqpoint{0.048611in}{0.000000in}}%
\pgfusepath{stroke,fill}%
}%
\begin{pgfscope}%
\pgfsys@transformshift{2.183372in}{2.214809in}%
\pgfsys@useobject{currentmarker}{}%
\end{pgfscope}%
\end{pgfscope}%
\begin{pgfscope}%
\pgftext[x=2.280594in,y=2.166981in,left,base]{\rmfamily\fontsize{10.000000}{12.000000}\selectfont \(\displaystyle 40\)}%
\end{pgfscope}%
\begin{pgfscope}%
\pgfsetbuttcap%
\pgfsetroundjoin%
\definecolor{currentfill}{rgb}{0.000000,0.000000,0.000000}%
\pgfsetfillcolor{currentfill}%
\pgfsetlinewidth{0.803000pt}%
\definecolor{currentstroke}{rgb}{0.000000,0.000000,0.000000}%
\pgfsetstrokecolor{currentstroke}%
\pgfsetdash{}{0pt}%
\pgfsys@defobject{currentmarker}{\pgfqpoint{0.000000in}{0.000000in}}{\pgfqpoint{0.048611in}{0.000000in}}{%
\pgfpathmoveto{\pgfqpoint{0.000000in}{0.000000in}}%
\pgfpathlineto{\pgfqpoint{0.048611in}{0.000000in}}%
\pgfusepath{stroke,fill}%
}%
\begin{pgfscope}%
\pgfsys@transformshift{2.183372in}{2.688541in}%
\pgfsys@useobject{currentmarker}{}%
\end{pgfscope}%
\end{pgfscope}%
\begin{pgfscope}%
\pgftext[x=2.280594in,y=2.640714in,left,base]{\rmfamily\fontsize{10.000000}{12.000000}\selectfont \(\displaystyle 50\)}%
\end{pgfscope}%
\begin{pgfscope}%
\pgfsetbuttcap%
\pgfsetmiterjoin%
\pgfsetlinewidth{0.803000pt}%
\definecolor{currentstroke}{rgb}{0.000000,0.000000,0.000000}%
\pgfsetstrokecolor{currentstroke}%
\pgfsetdash{}{0pt}%
\pgfpathmoveto{\pgfqpoint{2.053095in}{0.319877in}}%
\pgfpathlineto{\pgfqpoint{2.053095in}{0.330055in}}%
\pgfpathlineto{\pgfqpoint{2.053095in}{2.915230in}}%
\pgfpathlineto{\pgfqpoint{2.053095in}{2.925408in}}%
\pgfpathlineto{\pgfqpoint{2.183372in}{2.925408in}}%
\pgfpathlineto{\pgfqpoint{2.183372in}{2.915230in}}%
\pgfpathlineto{\pgfqpoint{2.183372in}{0.330055in}}%
\pgfpathlineto{\pgfqpoint{2.183372in}{0.319877in}}%
\pgfpathclose%
\pgfusepath{stroke}%
\end{pgfscope}%
\end{pgfpicture}%
\makeatother%
\endgroup%

    \vspace*{-0.4cm}
	\caption{300 K. Bin size $0.014e$}
	\end{subfigure}
	\hspace{0.6cm}
	\begin{subfigure}[b]{0.45\textwidth}
	\hspace*{-0.4cm}
	%% Creator: Matplotlib, PGF backend
%%
%% To include the figure in your LaTeX document, write
%%   \input{<filename>.pgf}
%%
%% Make sure the required packages are loaded in your preamble
%%   \usepackage{pgf}
%%
%% Figures using additional raster images can only be included by \input if
%% they are in the same directory as the main LaTeX file. For loading figures
%% from other directories you can use the `import` package
%%   \usepackage{import}
%% and then include the figures with
%%   \import{<path to file>}{<filename>.pgf}
%%
%% Matplotlib used the following preamble
%%   \usepackage[utf8x]{inputenc}
%%   \usepackage[T1]{fontenc}
%%
\begingroup%
\makeatletter%
\begin{pgfpicture}%
\pgfpathrectangle{\pgfpointorigin}{\pgfqpoint{2.519483in}{3.060408in}}%
\pgfusepath{use as bounding box, clip}%
\begin{pgfscope}%
\pgfsetbuttcap%
\pgfsetmiterjoin%
\definecolor{currentfill}{rgb}{1.000000,1.000000,1.000000}%
\pgfsetfillcolor{currentfill}%
\pgfsetlinewidth{0.000000pt}%
\definecolor{currentstroke}{rgb}{1.000000,1.000000,1.000000}%
\pgfsetstrokecolor{currentstroke}%
\pgfsetdash{}{0pt}%
\pgfpathmoveto{\pgfqpoint{0.000000in}{0.000000in}}%
\pgfpathlineto{\pgfqpoint{2.519483in}{0.000000in}}%
\pgfpathlineto{\pgfqpoint{2.519483in}{3.060408in}}%
\pgfpathlineto{\pgfqpoint{0.000000in}{3.060408in}}%
\pgfpathclose%
\pgfusepath{fill}%
\end{pgfscope}%
\begin{pgfscope}%
\pgfsetbuttcap%
\pgfsetmiterjoin%
\definecolor{currentfill}{rgb}{1.000000,1.000000,1.000000}%
\pgfsetfillcolor{currentfill}%
\pgfsetlinewidth{0.000000pt}%
\definecolor{currentstroke}{rgb}{0.000000,0.000000,0.000000}%
\pgfsetstrokecolor{currentstroke}%
\pgfsetstrokeopacity{0.000000}%
\pgfsetdash{}{0pt}%
\pgfpathmoveto{\pgfqpoint{0.374692in}{0.319877in}}%
\pgfpathlineto{\pgfqpoint{1.954366in}{0.319877in}}%
\pgfpathlineto{\pgfqpoint{1.954366in}{2.925408in}}%
\pgfpathlineto{\pgfqpoint{0.374692in}{2.925408in}}%
\pgfpathclose%
\pgfusepath{fill}%
\end{pgfscope}%
\begin{pgfscope}%
\pgfpathrectangle{\pgfqpoint{0.374692in}{0.319877in}}{\pgfqpoint{1.579674in}{2.605531in}} %
\pgfusepath{clip}%
\pgfsys@transformshift{0.374692in}{0.319877in}%
\pgftext[left,bottom]{\pgfimage[interpolate=true,width=1.580000in,height=2.610000in]{RnnNorm_vs_dq_Ti_500K-img0.png}}%
\end{pgfscope}%
\begin{pgfscope}%
\pgfpathrectangle{\pgfqpoint{0.374692in}{0.319877in}}{\pgfqpoint{1.579674in}{2.605531in}} %
\pgfusepath{clip}%
\pgfsetbuttcap%
\pgfsetroundjoin%
\definecolor{currentfill}{rgb}{1.000000,0.752941,0.796078}%
\pgfsetfillcolor{currentfill}%
\pgfsetlinewidth{1.003750pt}%
\definecolor{currentstroke}{rgb}{1.000000,0.752941,0.796078}%
\pgfsetstrokecolor{currentstroke}%
\pgfsetdash{}{0pt}%
\pgfpathmoveto{\pgfqpoint{0.670881in}{1.814806in}}%
\pgfpathcurveto{\pgfqpoint{0.681931in}{1.814806in}}{\pgfqpoint{0.692530in}{1.819196in}}{\pgfqpoint{0.700344in}{1.827009in}}%
\pgfpathcurveto{\pgfqpoint{0.708157in}{1.834823in}}{\pgfqpoint{0.712547in}{1.845422in}}{\pgfqpoint{0.712547in}{1.856472in}}%
\pgfpathcurveto{\pgfqpoint{0.712547in}{1.867522in}}{\pgfqpoint{0.708157in}{1.878121in}}{\pgfqpoint{0.700344in}{1.885935in}}%
\pgfpathcurveto{\pgfqpoint{0.692530in}{1.893749in}}{\pgfqpoint{0.681931in}{1.898139in}}{\pgfqpoint{0.670881in}{1.898139in}}%
\pgfpathcurveto{\pgfqpoint{0.659831in}{1.898139in}}{\pgfqpoint{0.649232in}{1.893749in}}{\pgfqpoint{0.641418in}{1.885935in}}%
\pgfpathcurveto{\pgfqpoint{0.633604in}{1.878121in}}{\pgfqpoint{0.629214in}{1.867522in}}{\pgfqpoint{0.629214in}{1.856472in}}%
\pgfpathcurveto{\pgfqpoint{0.629214in}{1.845422in}}{\pgfqpoint{0.633604in}{1.834823in}}{\pgfqpoint{0.641418in}{1.827009in}}%
\pgfpathcurveto{\pgfqpoint{0.649232in}{1.819196in}}{\pgfqpoint{0.659831in}{1.814806in}}{\pgfqpoint{0.670881in}{1.814806in}}%
\pgfpathclose%
\pgfusepath{stroke,fill}%
\end{pgfscope}%
\begin{pgfscope}%
\pgfpathrectangle{\pgfqpoint{0.374692in}{0.319877in}}{\pgfqpoint{1.579674in}{2.605531in}} %
\pgfusepath{clip}%
\pgfsetbuttcap%
\pgfsetroundjoin%
\definecolor{currentfill}{rgb}{1.000000,0.752941,0.796078}%
\pgfsetfillcolor{currentfill}%
\pgfsetlinewidth{1.003750pt}%
\definecolor{currentstroke}{rgb}{1.000000,0.752941,0.796078}%
\pgfsetstrokecolor{currentstroke}%
\pgfsetdash{}{0pt}%
\pgfpathmoveto{\pgfqpoint{0.868340in}{1.669399in}}%
\pgfpathcurveto{\pgfqpoint{0.879390in}{1.669399in}}{\pgfqpoint{0.889989in}{1.673789in}}{\pgfqpoint{0.897803in}{1.681603in}}%
\pgfpathcurveto{\pgfqpoint{0.905616in}{1.689416in}}{\pgfqpoint{0.910007in}{1.700015in}}{\pgfqpoint{0.910007in}{1.711066in}}%
\pgfpathcurveto{\pgfqpoint{0.910007in}{1.722116in}}{\pgfqpoint{0.905616in}{1.732715in}}{\pgfqpoint{0.897803in}{1.740528in}}%
\pgfpathcurveto{\pgfqpoint{0.889989in}{1.748342in}}{\pgfqpoint{0.879390in}{1.752732in}}{\pgfqpoint{0.868340in}{1.752732in}}%
\pgfpathcurveto{\pgfqpoint{0.857290in}{1.752732in}}{\pgfqpoint{0.846691in}{1.748342in}}{\pgfqpoint{0.838877in}{1.740528in}}%
\pgfpathcurveto{\pgfqpoint{0.831064in}{1.732715in}}{\pgfqpoint{0.826673in}{1.722116in}}{\pgfqpoint{0.826673in}{1.711066in}}%
\pgfpathcurveto{\pgfqpoint{0.826673in}{1.700015in}}{\pgfqpoint{0.831064in}{1.689416in}}{\pgfqpoint{0.838877in}{1.681603in}}%
\pgfpathcurveto{\pgfqpoint{0.846691in}{1.673789in}}{\pgfqpoint{0.857290in}{1.669399in}}{\pgfqpoint{0.868340in}{1.669399in}}%
\pgfpathclose%
\pgfusepath{stroke,fill}%
\end{pgfscope}%
\begin{pgfscope}%
\pgfpathrectangle{\pgfqpoint{0.374692in}{0.319877in}}{\pgfqpoint{1.579674in}{2.605531in}} %
\pgfusepath{clip}%
\pgfsetbuttcap%
\pgfsetroundjoin%
\definecolor{currentfill}{rgb}{1.000000,0.752941,0.796078}%
\pgfsetfillcolor{currentfill}%
\pgfsetlinewidth{1.003750pt}%
\definecolor{currentstroke}{rgb}{1.000000,0.752941,0.796078}%
\pgfsetstrokecolor{currentstroke}%
\pgfsetdash{}{0pt}%
\pgfpathmoveto{\pgfqpoint{1.065799in}{1.554679in}}%
\pgfpathcurveto{\pgfqpoint{1.076849in}{1.554679in}}{\pgfqpoint{1.087448in}{1.559069in}}{\pgfqpoint{1.095262in}{1.566883in}}%
\pgfpathcurveto{\pgfqpoint{1.103076in}{1.574696in}}{\pgfqpoint{1.107466in}{1.585296in}}{\pgfqpoint{1.107466in}{1.596346in}}%
\pgfpathcurveto{\pgfqpoint{1.107466in}{1.607396in}}{\pgfqpoint{1.103076in}{1.617995in}}{\pgfqpoint{1.095262in}{1.625808in}}%
\pgfpathcurveto{\pgfqpoint{1.087448in}{1.633622in}}{\pgfqpoint{1.076849in}{1.638012in}}{\pgfqpoint{1.065799in}{1.638012in}}%
\pgfpathcurveto{\pgfqpoint{1.054749in}{1.638012in}}{\pgfqpoint{1.044150in}{1.633622in}}{\pgfqpoint{1.036336in}{1.625808in}}%
\pgfpathcurveto{\pgfqpoint{1.028523in}{1.617995in}}{\pgfqpoint{1.024133in}{1.607396in}}{\pgfqpoint{1.024133in}{1.596346in}}%
\pgfpathcurveto{\pgfqpoint{1.024133in}{1.585296in}}{\pgfqpoint{1.028523in}{1.574696in}}{\pgfqpoint{1.036336in}{1.566883in}}%
\pgfpathcurveto{\pgfqpoint{1.044150in}{1.559069in}}{\pgfqpoint{1.054749in}{1.554679in}}{\pgfqpoint{1.065799in}{1.554679in}}%
\pgfpathclose%
\pgfusepath{stroke,fill}%
\end{pgfscope}%
\begin{pgfscope}%
\pgfpathrectangle{\pgfqpoint{0.374692in}{0.319877in}}{\pgfqpoint{1.579674in}{2.605531in}} %
\pgfusepath{clip}%
\pgfsetbuttcap%
\pgfsetroundjoin%
\definecolor{currentfill}{rgb}{1.000000,0.752941,0.796078}%
\pgfsetfillcolor{currentfill}%
\pgfsetlinewidth{1.003750pt}%
\definecolor{currentstroke}{rgb}{1.000000,0.752941,0.796078}%
\pgfsetstrokecolor{currentstroke}%
\pgfsetdash{}{0pt}%
\pgfpathmoveto{\pgfqpoint{1.263258in}{1.440127in}}%
\pgfpathcurveto{\pgfqpoint{1.274309in}{1.440127in}}{\pgfqpoint{1.284908in}{1.444517in}}{\pgfqpoint{1.292721in}{1.452331in}}%
\pgfpathcurveto{\pgfqpoint{1.300535in}{1.460145in}}{\pgfqpoint{1.304925in}{1.470744in}}{\pgfqpoint{1.304925in}{1.481794in}}%
\pgfpathcurveto{\pgfqpoint{1.304925in}{1.492844in}}{\pgfqpoint{1.300535in}{1.503443in}}{\pgfqpoint{1.292721in}{1.511257in}}%
\pgfpathcurveto{\pgfqpoint{1.284908in}{1.519070in}}{\pgfqpoint{1.274309in}{1.523460in}}{\pgfqpoint{1.263258in}{1.523460in}}%
\pgfpathcurveto{\pgfqpoint{1.252208in}{1.523460in}}{\pgfqpoint{1.241609in}{1.519070in}}{\pgfqpoint{1.233796in}{1.511257in}}%
\pgfpathcurveto{\pgfqpoint{1.225982in}{1.503443in}}{\pgfqpoint{1.221592in}{1.492844in}}{\pgfqpoint{1.221592in}{1.481794in}}%
\pgfpathcurveto{\pgfqpoint{1.221592in}{1.470744in}}{\pgfqpoint{1.225982in}{1.460145in}}{\pgfqpoint{1.233796in}{1.452331in}}%
\pgfpathcurveto{\pgfqpoint{1.241609in}{1.444517in}}{\pgfqpoint{1.252208in}{1.440127in}}{\pgfqpoint{1.263258in}{1.440127in}}%
\pgfpathclose%
\pgfusepath{stroke,fill}%
\end{pgfscope}%
\begin{pgfscope}%
\pgfpathrectangle{\pgfqpoint{0.374692in}{0.319877in}}{\pgfqpoint{1.579674in}{2.605531in}} %
\pgfusepath{clip}%
\pgfsetbuttcap%
\pgfsetroundjoin%
\definecolor{currentfill}{rgb}{1.000000,0.752941,0.796078}%
\pgfsetfillcolor{currentfill}%
\pgfsetlinewidth{1.003750pt}%
\definecolor{currentstroke}{rgb}{1.000000,0.752941,0.796078}%
\pgfsetstrokecolor{currentstroke}%
\pgfsetdash{}{0pt}%
\pgfpathmoveto{\pgfqpoint{1.460718in}{1.286064in}}%
\pgfpathcurveto{\pgfqpoint{1.471768in}{1.286064in}}{\pgfqpoint{1.482367in}{1.290455in}}{\pgfqpoint{1.490180in}{1.298268in}}%
\pgfpathcurveto{\pgfqpoint{1.497994in}{1.306082in}}{\pgfqpoint{1.502384in}{1.316681in}}{\pgfqpoint{1.502384in}{1.327731in}}%
\pgfpathcurveto{\pgfqpoint{1.502384in}{1.338781in}}{\pgfqpoint{1.497994in}{1.349380in}}{\pgfqpoint{1.490180in}{1.357194in}}%
\pgfpathcurveto{\pgfqpoint{1.482367in}{1.365007in}}{\pgfqpoint{1.471768in}{1.369398in}}{\pgfqpoint{1.460718in}{1.369398in}}%
\pgfpathcurveto{\pgfqpoint{1.449668in}{1.369398in}}{\pgfqpoint{1.439069in}{1.365007in}}{\pgfqpoint{1.431255in}{1.357194in}}%
\pgfpathcurveto{\pgfqpoint{1.423441in}{1.349380in}}{\pgfqpoint{1.419051in}{1.338781in}}{\pgfqpoint{1.419051in}{1.327731in}}%
\pgfpathcurveto{\pgfqpoint{1.419051in}{1.316681in}}{\pgfqpoint{1.423441in}{1.306082in}}{\pgfqpoint{1.431255in}{1.298268in}}%
\pgfpathcurveto{\pgfqpoint{1.439069in}{1.290455in}}{\pgfqpoint{1.449668in}{1.286064in}}{\pgfqpoint{1.460718in}{1.286064in}}%
\pgfpathclose%
\pgfusepath{stroke,fill}%
\end{pgfscope}%
\begin{pgfscope}%
\pgfpathrectangle{\pgfqpoint{0.374692in}{0.319877in}}{\pgfqpoint{1.579674in}{2.605531in}} %
\pgfusepath{clip}%
\pgfsetbuttcap%
\pgfsetroundjoin%
\definecolor{currentfill}{rgb}{1.000000,0.752941,0.796078}%
\pgfsetfillcolor{currentfill}%
\pgfsetlinewidth{1.003750pt}%
\definecolor{currentstroke}{rgb}{1.000000,0.752941,0.796078}%
\pgfsetstrokecolor{currentstroke}%
\pgfsetdash{}{0pt}%
\pgfpathmoveto{\pgfqpoint{1.658177in}{1.347146in}}%
\pgfpathcurveto{\pgfqpoint{1.669227in}{1.347146in}}{\pgfqpoint{1.679826in}{1.351537in}}{\pgfqpoint{1.687640in}{1.359350in}}%
\pgfpathcurveto{\pgfqpoint{1.695453in}{1.367164in}}{\pgfqpoint{1.699844in}{1.377763in}}{\pgfqpoint{1.699844in}{1.388813in}}%
\pgfpathcurveto{\pgfqpoint{1.699844in}{1.399863in}}{\pgfqpoint{1.695453in}{1.410462in}}{\pgfqpoint{1.687640in}{1.418276in}}%
\pgfpathcurveto{\pgfqpoint{1.679826in}{1.426089in}}{\pgfqpoint{1.669227in}{1.430480in}}{\pgfqpoint{1.658177in}{1.430480in}}%
\pgfpathcurveto{\pgfqpoint{1.647127in}{1.430480in}}{\pgfqpoint{1.636528in}{1.426089in}}{\pgfqpoint{1.628714in}{1.418276in}}%
\pgfpathcurveto{\pgfqpoint{1.620901in}{1.410462in}}{\pgfqpoint{1.616510in}{1.399863in}}{\pgfqpoint{1.616510in}{1.388813in}}%
\pgfpathcurveto{\pgfqpoint{1.616510in}{1.377763in}}{\pgfqpoint{1.620901in}{1.367164in}}{\pgfqpoint{1.628714in}{1.359350in}}%
\pgfpathcurveto{\pgfqpoint{1.636528in}{1.351537in}}{\pgfqpoint{1.647127in}{1.347146in}}{\pgfqpoint{1.658177in}{1.347146in}}%
\pgfpathclose%
\pgfusepath{stroke,fill}%
\end{pgfscope}%
\begin{pgfscope}%
\pgfpathrectangle{\pgfqpoint{0.374692in}{0.319877in}}{\pgfqpoint{1.579674in}{2.605531in}} %
\pgfusepath{clip}%
\pgfsetbuttcap%
\pgfsetroundjoin%
\definecolor{currentfill}{rgb}{1.000000,0.752941,0.796078}%
\pgfsetfillcolor{currentfill}%
\pgfsetlinewidth{1.003750pt}%
\definecolor{currentstroke}{rgb}{1.000000,0.752941,0.796078}%
\pgfsetstrokecolor{currentstroke}%
\pgfsetdash{}{0pt}%
\pgfpathmoveto{\pgfqpoint{1.855636in}{0.679062in}}%
\pgfpathcurveto{\pgfqpoint{1.866686in}{0.679062in}}{\pgfqpoint{1.877285in}{0.683452in}}{\pgfqpoint{1.885099in}{0.691265in}}%
\pgfpathcurveto{\pgfqpoint{1.892913in}{0.699079in}}{\pgfqpoint{1.897303in}{0.709678in}}{\pgfqpoint{1.897303in}{0.720728in}}%
\pgfpathcurveto{\pgfqpoint{1.897303in}{0.731778in}}{\pgfqpoint{1.892913in}{0.742377in}}{\pgfqpoint{1.885099in}{0.750191in}}%
\pgfpathcurveto{\pgfqpoint{1.877285in}{0.758005in}}{\pgfqpoint{1.866686in}{0.762395in}}{\pgfqpoint{1.855636in}{0.762395in}}%
\pgfpathcurveto{\pgfqpoint{1.844586in}{0.762395in}}{\pgfqpoint{1.833987in}{0.758005in}}{\pgfqpoint{1.826173in}{0.750191in}}%
\pgfpathcurveto{\pgfqpoint{1.818360in}{0.742377in}}{\pgfqpoint{1.813969in}{0.731778in}}{\pgfqpoint{1.813969in}{0.720728in}}%
\pgfpathcurveto{\pgfqpoint{1.813969in}{0.709678in}}{\pgfqpoint{1.818360in}{0.699079in}}{\pgfqpoint{1.826173in}{0.691265in}}%
\pgfpathcurveto{\pgfqpoint{1.833987in}{0.683452in}}{\pgfqpoint{1.844586in}{0.679062in}}{\pgfqpoint{1.855636in}{0.679062in}}%
\pgfpathclose%
\pgfusepath{stroke,fill}%
\end{pgfscope}%
\begin{pgfscope}%
\pgfsetbuttcap%
\pgfsetroundjoin%
\definecolor{currentfill}{rgb}{0.000000,0.000000,0.000000}%
\pgfsetfillcolor{currentfill}%
\pgfsetlinewidth{0.803000pt}%
\definecolor{currentstroke}{rgb}{0.000000,0.000000,0.000000}%
\pgfsetstrokecolor{currentstroke}%
\pgfsetdash{}{0pt}%
\pgfsys@defobject{currentmarker}{\pgfqpoint{0.000000in}{-0.048611in}}{\pgfqpoint{0.000000in}{0.000000in}}{%
\pgfpathmoveto{\pgfqpoint{0.000000in}{0.000000in}}%
\pgfpathlineto{\pgfqpoint{0.000000in}{-0.048611in}}%
\pgfusepath{stroke,fill}%
}%
\begin{pgfscope}%
\pgfsys@transformshift{0.670881in}{0.319877in}%
\pgfsys@useobject{currentmarker}{}%
\end{pgfscope}%
\end{pgfscope}%
\begin{pgfscope}%
\pgftext[x=0.670881in,y=0.222655in,,top]{\rmfamily\fontsize{10.000000}{12.000000}\selectfont \(\displaystyle -0.05\)}%
\end{pgfscope}%
\begin{pgfscope}%
\pgfsetbuttcap%
\pgfsetroundjoin%
\definecolor{currentfill}{rgb}{0.000000,0.000000,0.000000}%
\pgfsetfillcolor{currentfill}%
\pgfsetlinewidth{0.803000pt}%
\definecolor{currentstroke}{rgb}{0.000000,0.000000,0.000000}%
\pgfsetstrokecolor{currentstroke}%
\pgfsetdash{}{0pt}%
\pgfsys@defobject{currentmarker}{\pgfqpoint{0.000000in}{-0.048611in}}{\pgfqpoint{0.000000in}{0.000000in}}{%
\pgfpathmoveto{\pgfqpoint{0.000000in}{0.000000in}}%
\pgfpathlineto{\pgfqpoint{0.000000in}{-0.048611in}}%
\pgfusepath{stroke,fill}%
}%
\begin{pgfscope}%
\pgfsys@transformshift{1.164529in}{0.319877in}%
\pgfsys@useobject{currentmarker}{}%
\end{pgfscope}%
\end{pgfscope}%
\begin{pgfscope}%
\pgftext[x=1.164529in,y=0.222655in,,top]{\rmfamily\fontsize{10.000000}{12.000000}\selectfont \(\displaystyle 0.00\)}%
\end{pgfscope}%
\begin{pgfscope}%
\pgfsetbuttcap%
\pgfsetroundjoin%
\definecolor{currentfill}{rgb}{0.000000,0.000000,0.000000}%
\pgfsetfillcolor{currentfill}%
\pgfsetlinewidth{0.803000pt}%
\definecolor{currentstroke}{rgb}{0.000000,0.000000,0.000000}%
\pgfsetstrokecolor{currentstroke}%
\pgfsetdash{}{0pt}%
\pgfsys@defobject{currentmarker}{\pgfqpoint{0.000000in}{-0.048611in}}{\pgfqpoint{0.000000in}{0.000000in}}{%
\pgfpathmoveto{\pgfqpoint{0.000000in}{0.000000in}}%
\pgfpathlineto{\pgfqpoint{0.000000in}{-0.048611in}}%
\pgfusepath{stroke,fill}%
}%
\begin{pgfscope}%
\pgfsys@transformshift{1.658177in}{0.319877in}%
\pgfsys@useobject{currentmarker}{}%
\end{pgfscope}%
\end{pgfscope}%
\begin{pgfscope}%
\pgftext[x=1.658177in,y=0.222655in,,top]{\rmfamily\fontsize{10.000000}{12.000000}\selectfont \(\displaystyle 0.05\)}%
\end{pgfscope}%
\begin{pgfscope}%
\pgfsetbuttcap%
\pgfsetroundjoin%
\definecolor{currentfill}{rgb}{0.000000,0.000000,0.000000}%
\pgfsetfillcolor{currentfill}%
\pgfsetlinewidth{0.803000pt}%
\definecolor{currentstroke}{rgb}{0.000000,0.000000,0.000000}%
\pgfsetstrokecolor{currentstroke}%
\pgfsetdash{}{0pt}%
\pgfsys@defobject{currentmarker}{\pgfqpoint{-0.048611in}{0.000000in}}{\pgfqpoint{0.000000in}{0.000000in}}{%
\pgfpathmoveto{\pgfqpoint{0.000000in}{0.000000in}}%
\pgfpathlineto{\pgfqpoint{-0.048611in}{0.000000in}}%
\pgfusepath{stroke,fill}%
}%
\begin{pgfscope}%
\pgfsys@transformshift{0.374692in}{0.622908in}%
\pgfsys@useobject{currentmarker}{}%
\end{pgfscope}%
\end{pgfscope}%
\begin{pgfscope}%
\pgftext[x=0.100000in,y=0.575080in,left,base]{\rmfamily\fontsize{10.000000}{12.000000}\selectfont \(\displaystyle 3.6\)}%
\end{pgfscope}%
\begin{pgfscope}%
\pgfsetbuttcap%
\pgfsetroundjoin%
\definecolor{currentfill}{rgb}{0.000000,0.000000,0.000000}%
\pgfsetfillcolor{currentfill}%
\pgfsetlinewidth{0.803000pt}%
\definecolor{currentstroke}{rgb}{0.000000,0.000000,0.000000}%
\pgfsetstrokecolor{currentstroke}%
\pgfsetdash{}{0pt}%
\pgfsys@defobject{currentmarker}{\pgfqpoint{-0.048611in}{0.000000in}}{\pgfqpoint{0.000000in}{0.000000in}}{%
\pgfpathmoveto{\pgfqpoint{0.000000in}{0.000000in}}%
\pgfpathlineto{\pgfqpoint{-0.048611in}{0.000000in}}%
\pgfusepath{stroke,fill}%
}%
\begin{pgfscope}%
\pgfsys@transformshift{0.374692in}{1.067280in}%
\pgfsys@useobject{currentmarker}{}%
\end{pgfscope}%
\end{pgfscope}%
\begin{pgfscope}%
\pgftext[x=0.100000in,y=1.019453in,left,base]{\rmfamily\fontsize{10.000000}{12.000000}\selectfont \(\displaystyle 3.7\)}%
\end{pgfscope}%
\begin{pgfscope}%
\pgfsetbuttcap%
\pgfsetroundjoin%
\definecolor{currentfill}{rgb}{0.000000,0.000000,0.000000}%
\pgfsetfillcolor{currentfill}%
\pgfsetlinewidth{0.803000pt}%
\definecolor{currentstroke}{rgb}{0.000000,0.000000,0.000000}%
\pgfsetstrokecolor{currentstroke}%
\pgfsetdash{}{0pt}%
\pgfsys@defobject{currentmarker}{\pgfqpoint{-0.048611in}{0.000000in}}{\pgfqpoint{0.000000in}{0.000000in}}{%
\pgfpathmoveto{\pgfqpoint{0.000000in}{0.000000in}}%
\pgfpathlineto{\pgfqpoint{-0.048611in}{0.000000in}}%
\pgfusepath{stroke,fill}%
}%
\begin{pgfscope}%
\pgfsys@transformshift{0.374692in}{1.511652in}%
\pgfsys@useobject{currentmarker}{}%
\end{pgfscope}%
\end{pgfscope}%
\begin{pgfscope}%
\pgftext[x=0.100000in,y=1.463825in,left,base]{\rmfamily\fontsize{10.000000}{12.000000}\selectfont \(\displaystyle 3.8\)}%
\end{pgfscope}%
\begin{pgfscope}%
\pgfsetbuttcap%
\pgfsetroundjoin%
\definecolor{currentfill}{rgb}{0.000000,0.000000,0.000000}%
\pgfsetfillcolor{currentfill}%
\pgfsetlinewidth{0.803000pt}%
\definecolor{currentstroke}{rgb}{0.000000,0.000000,0.000000}%
\pgfsetstrokecolor{currentstroke}%
\pgfsetdash{}{0pt}%
\pgfsys@defobject{currentmarker}{\pgfqpoint{-0.048611in}{0.000000in}}{\pgfqpoint{0.000000in}{0.000000in}}{%
\pgfpathmoveto{\pgfqpoint{0.000000in}{0.000000in}}%
\pgfpathlineto{\pgfqpoint{-0.048611in}{0.000000in}}%
\pgfusepath{stroke,fill}%
}%
\begin{pgfscope}%
\pgfsys@transformshift{0.374692in}{1.956024in}%
\pgfsys@useobject{currentmarker}{}%
\end{pgfscope}%
\end{pgfscope}%
\begin{pgfscope}%
\pgftext[x=0.100000in,y=1.908197in,left,base]{\rmfamily\fontsize{10.000000}{12.000000}\selectfont \(\displaystyle 3.9\)}%
\end{pgfscope}%
\begin{pgfscope}%
\pgfsetbuttcap%
\pgfsetroundjoin%
\definecolor{currentfill}{rgb}{0.000000,0.000000,0.000000}%
\pgfsetfillcolor{currentfill}%
\pgfsetlinewidth{0.803000pt}%
\definecolor{currentstroke}{rgb}{0.000000,0.000000,0.000000}%
\pgfsetstrokecolor{currentstroke}%
\pgfsetdash{}{0pt}%
\pgfsys@defobject{currentmarker}{\pgfqpoint{-0.048611in}{0.000000in}}{\pgfqpoint{0.000000in}{0.000000in}}{%
\pgfpathmoveto{\pgfqpoint{0.000000in}{0.000000in}}%
\pgfpathlineto{\pgfqpoint{-0.048611in}{0.000000in}}%
\pgfusepath{stroke,fill}%
}%
\begin{pgfscope}%
\pgfsys@transformshift{0.374692in}{2.400396in}%
\pgfsys@useobject{currentmarker}{}%
\end{pgfscope}%
\end{pgfscope}%
\begin{pgfscope}%
\pgftext[x=0.100000in,y=2.352569in,left,base]{\rmfamily\fontsize{10.000000}{12.000000}\selectfont \(\displaystyle 4.0\)}%
\end{pgfscope}%
\begin{pgfscope}%
\pgfsetbuttcap%
\pgfsetroundjoin%
\definecolor{currentfill}{rgb}{0.000000,0.000000,0.000000}%
\pgfsetfillcolor{currentfill}%
\pgfsetlinewidth{0.803000pt}%
\definecolor{currentstroke}{rgb}{0.000000,0.000000,0.000000}%
\pgfsetstrokecolor{currentstroke}%
\pgfsetdash{}{0pt}%
\pgfsys@defobject{currentmarker}{\pgfqpoint{-0.048611in}{0.000000in}}{\pgfqpoint{0.000000in}{0.000000in}}{%
\pgfpathmoveto{\pgfqpoint{0.000000in}{0.000000in}}%
\pgfpathlineto{\pgfqpoint{-0.048611in}{0.000000in}}%
\pgfusepath{stroke,fill}%
}%
\begin{pgfscope}%
\pgfsys@transformshift{0.374692in}{2.844769in}%
\pgfsys@useobject{currentmarker}{}%
\end{pgfscope}%
\end{pgfscope}%
\begin{pgfscope}%
\pgftext[x=0.100000in,y=2.796941in,left,base]{\rmfamily\fontsize{10.000000}{12.000000}\selectfont \(\displaystyle 4.1\)}%
\end{pgfscope}%
\begin{pgfscope}%
\pgfsetrectcap%
\pgfsetmiterjoin%
\pgfsetlinewidth{0.803000pt}%
\definecolor{currentstroke}{rgb}{0.000000,0.000000,0.000000}%
\pgfsetstrokecolor{currentstroke}%
\pgfsetdash{}{0pt}%
\pgfpathmoveto{\pgfqpoint{0.374692in}{0.319877in}}%
\pgfpathlineto{\pgfqpoint{0.374692in}{2.925408in}}%
\pgfusepath{stroke}%
\end{pgfscope}%
\begin{pgfscope}%
\pgfsetrectcap%
\pgfsetmiterjoin%
\pgfsetlinewidth{0.803000pt}%
\definecolor{currentstroke}{rgb}{0.000000,0.000000,0.000000}%
\pgfsetstrokecolor{currentstroke}%
\pgfsetdash{}{0pt}%
\pgfpathmoveto{\pgfqpoint{1.954366in}{0.319877in}}%
\pgfpathlineto{\pgfqpoint{1.954366in}{2.925408in}}%
\pgfusepath{stroke}%
\end{pgfscope}%
\begin{pgfscope}%
\pgfsetrectcap%
\pgfsetmiterjoin%
\pgfsetlinewidth{0.803000pt}%
\definecolor{currentstroke}{rgb}{0.000000,0.000000,0.000000}%
\pgfsetstrokecolor{currentstroke}%
\pgfsetdash{}{0pt}%
\pgfpathmoveto{\pgfqpoint{0.374692in}{0.319877in}}%
\pgfpathlineto{\pgfqpoint{1.954366in}{0.319877in}}%
\pgfusepath{stroke}%
\end{pgfscope}%
\begin{pgfscope}%
\pgfsetrectcap%
\pgfsetmiterjoin%
\pgfsetlinewidth{0.803000pt}%
\definecolor{currentstroke}{rgb}{0.000000,0.000000,0.000000}%
\pgfsetstrokecolor{currentstroke}%
\pgfsetdash{}{0pt}%
\pgfpathmoveto{\pgfqpoint{0.374692in}{2.925408in}}%
\pgfpathlineto{\pgfqpoint{1.954366in}{2.925408in}}%
\pgfusepath{stroke}%
\end{pgfscope}%
\begin{pgfscope}%
\pgfpathrectangle{\pgfqpoint{2.053095in}{0.319877in}}{\pgfqpoint{0.130277in}{2.605531in}} %
\pgfusepath{clip}%
\pgfsetbuttcap%
\pgfsetmiterjoin%
\definecolor{currentfill}{rgb}{1.000000,1.000000,1.000000}%
\pgfsetfillcolor{currentfill}%
\pgfsetlinewidth{0.010037pt}%
\definecolor{currentstroke}{rgb}{1.000000,1.000000,1.000000}%
\pgfsetstrokecolor{currentstroke}%
\pgfsetdash{}{0pt}%
\pgfpathmoveto{\pgfqpoint{2.053095in}{0.319877in}}%
\pgfpathlineto{\pgfqpoint{2.053095in}{0.330055in}}%
\pgfpathlineto{\pgfqpoint{2.053095in}{2.915230in}}%
\pgfpathlineto{\pgfqpoint{2.053095in}{2.925408in}}%
\pgfpathlineto{\pgfqpoint{2.183372in}{2.925408in}}%
\pgfpathlineto{\pgfqpoint{2.183372in}{2.915230in}}%
\pgfpathlineto{\pgfqpoint{2.183372in}{0.330055in}}%
\pgfpathlineto{\pgfqpoint{2.183372in}{0.319877in}}%
\pgfpathclose%
\pgfusepath{stroke,fill}%
\end{pgfscope}%
\begin{pgfscope}%
\pgfsys@transformshift{2.050000in}{0.320408in}%
\pgftext[left,bottom]{\pgfimage[interpolate=true,width=0.130000in,height=2.610000in]{RnnNorm_vs_dq_Ti_500K-img1.png}}%
\end{pgfscope}%
\begin{pgfscope}%
\pgfsetbuttcap%
\pgfsetroundjoin%
\definecolor{currentfill}{rgb}{0.000000,0.000000,0.000000}%
\pgfsetfillcolor{currentfill}%
\pgfsetlinewidth{0.803000pt}%
\definecolor{currentstroke}{rgb}{0.000000,0.000000,0.000000}%
\pgfsetstrokecolor{currentstroke}%
\pgfsetdash{}{0pt}%
\pgfsys@defobject{currentmarker}{\pgfqpoint{0.000000in}{0.000000in}}{\pgfqpoint{0.048611in}{0.000000in}}{%
\pgfpathmoveto{\pgfqpoint{0.000000in}{0.000000in}}%
\pgfpathlineto{\pgfqpoint{0.048611in}{0.000000in}}%
\pgfusepath{stroke,fill}%
}%
\begin{pgfscope}%
\pgfsys@transformshift{2.183372in}{0.319877in}%
\pgfsys@useobject{currentmarker}{}%
\end{pgfscope}%
\end{pgfscope}%
\begin{pgfscope}%
\pgftext[x=2.280594in,y=0.272050in,left,base]{\rmfamily\fontsize{10.000000}{12.000000}\selectfont \(\displaystyle 0\)}%
\end{pgfscope}%
\begin{pgfscope}%
\pgfsetbuttcap%
\pgfsetroundjoin%
\definecolor{currentfill}{rgb}{0.000000,0.000000,0.000000}%
\pgfsetfillcolor{currentfill}%
\pgfsetlinewidth{0.803000pt}%
\definecolor{currentstroke}{rgb}{0.000000,0.000000,0.000000}%
\pgfsetstrokecolor{currentstroke}%
\pgfsetdash{}{0pt}%
\pgfsys@defobject{currentmarker}{\pgfqpoint{0.000000in}{0.000000in}}{\pgfqpoint{0.048611in}{0.000000in}}{%
\pgfpathmoveto{\pgfqpoint{0.000000in}{0.000000in}}%
\pgfpathlineto{\pgfqpoint{0.048611in}{0.000000in}}%
\pgfusepath{stroke,fill}%
}%
\begin{pgfscope}%
\pgfsys@transformshift{2.183372in}{0.793610in}%
\pgfsys@useobject{currentmarker}{}%
\end{pgfscope}%
\end{pgfscope}%
\begin{pgfscope}%
\pgftext[x=2.280594in,y=0.745782in,left,base]{\rmfamily\fontsize{10.000000}{12.000000}\selectfont \(\displaystyle 10\)}%
\end{pgfscope}%
\begin{pgfscope}%
\pgfsetbuttcap%
\pgfsetroundjoin%
\definecolor{currentfill}{rgb}{0.000000,0.000000,0.000000}%
\pgfsetfillcolor{currentfill}%
\pgfsetlinewidth{0.803000pt}%
\definecolor{currentstroke}{rgb}{0.000000,0.000000,0.000000}%
\pgfsetstrokecolor{currentstroke}%
\pgfsetdash{}{0pt}%
\pgfsys@defobject{currentmarker}{\pgfqpoint{0.000000in}{0.000000in}}{\pgfqpoint{0.048611in}{0.000000in}}{%
\pgfpathmoveto{\pgfqpoint{0.000000in}{0.000000in}}%
\pgfpathlineto{\pgfqpoint{0.048611in}{0.000000in}}%
\pgfusepath{stroke,fill}%
}%
\begin{pgfscope}%
\pgfsys@transformshift{2.183372in}{1.267343in}%
\pgfsys@useobject{currentmarker}{}%
\end{pgfscope}%
\end{pgfscope}%
\begin{pgfscope}%
\pgftext[x=2.280594in,y=1.219515in,left,base]{\rmfamily\fontsize{10.000000}{12.000000}\selectfont \(\displaystyle 20\)}%
\end{pgfscope}%
\begin{pgfscope}%
\pgfsetbuttcap%
\pgfsetroundjoin%
\definecolor{currentfill}{rgb}{0.000000,0.000000,0.000000}%
\pgfsetfillcolor{currentfill}%
\pgfsetlinewidth{0.803000pt}%
\definecolor{currentstroke}{rgb}{0.000000,0.000000,0.000000}%
\pgfsetstrokecolor{currentstroke}%
\pgfsetdash{}{0pt}%
\pgfsys@defobject{currentmarker}{\pgfqpoint{0.000000in}{0.000000in}}{\pgfqpoint{0.048611in}{0.000000in}}{%
\pgfpathmoveto{\pgfqpoint{0.000000in}{0.000000in}}%
\pgfpathlineto{\pgfqpoint{0.048611in}{0.000000in}}%
\pgfusepath{stroke,fill}%
}%
\begin{pgfscope}%
\pgfsys@transformshift{2.183372in}{1.741076in}%
\pgfsys@useobject{currentmarker}{}%
\end{pgfscope}%
\end{pgfscope}%
\begin{pgfscope}%
\pgftext[x=2.280594in,y=1.693248in,left,base]{\rmfamily\fontsize{10.000000}{12.000000}\selectfont \(\displaystyle 30\)}%
\end{pgfscope}%
\begin{pgfscope}%
\pgfsetbuttcap%
\pgfsetroundjoin%
\definecolor{currentfill}{rgb}{0.000000,0.000000,0.000000}%
\pgfsetfillcolor{currentfill}%
\pgfsetlinewidth{0.803000pt}%
\definecolor{currentstroke}{rgb}{0.000000,0.000000,0.000000}%
\pgfsetstrokecolor{currentstroke}%
\pgfsetdash{}{0pt}%
\pgfsys@defobject{currentmarker}{\pgfqpoint{0.000000in}{0.000000in}}{\pgfqpoint{0.048611in}{0.000000in}}{%
\pgfpathmoveto{\pgfqpoint{0.000000in}{0.000000in}}%
\pgfpathlineto{\pgfqpoint{0.048611in}{0.000000in}}%
\pgfusepath{stroke,fill}%
}%
\begin{pgfscope}%
\pgfsys@transformshift{2.183372in}{2.214809in}%
\pgfsys@useobject{currentmarker}{}%
\end{pgfscope}%
\end{pgfscope}%
\begin{pgfscope}%
\pgftext[x=2.280594in,y=2.166981in,left,base]{\rmfamily\fontsize{10.000000}{12.000000}\selectfont \(\displaystyle 40\)}%
\end{pgfscope}%
\begin{pgfscope}%
\pgfsetbuttcap%
\pgfsetroundjoin%
\definecolor{currentfill}{rgb}{0.000000,0.000000,0.000000}%
\pgfsetfillcolor{currentfill}%
\pgfsetlinewidth{0.803000pt}%
\definecolor{currentstroke}{rgb}{0.000000,0.000000,0.000000}%
\pgfsetstrokecolor{currentstroke}%
\pgfsetdash{}{0pt}%
\pgfsys@defobject{currentmarker}{\pgfqpoint{0.000000in}{0.000000in}}{\pgfqpoint{0.048611in}{0.000000in}}{%
\pgfpathmoveto{\pgfqpoint{0.000000in}{0.000000in}}%
\pgfpathlineto{\pgfqpoint{0.048611in}{0.000000in}}%
\pgfusepath{stroke,fill}%
}%
\begin{pgfscope}%
\pgfsys@transformshift{2.183372in}{2.688541in}%
\pgfsys@useobject{currentmarker}{}%
\end{pgfscope}%
\end{pgfscope}%
\begin{pgfscope}%
\pgftext[x=2.280594in,y=2.640714in,left,base]{\rmfamily\fontsize{10.000000}{12.000000}\selectfont \(\displaystyle 50\)}%
\end{pgfscope}%
\begin{pgfscope}%
\pgfsetbuttcap%
\pgfsetmiterjoin%
\pgfsetlinewidth{0.803000pt}%
\definecolor{currentstroke}{rgb}{0.000000,0.000000,0.000000}%
\pgfsetstrokecolor{currentstroke}%
\pgfsetdash{}{0pt}%
\pgfpathmoveto{\pgfqpoint{2.053095in}{0.319877in}}%
\pgfpathlineto{\pgfqpoint{2.053095in}{0.330055in}}%
\pgfpathlineto{\pgfqpoint{2.053095in}{2.915230in}}%
\pgfpathlineto{\pgfqpoint{2.053095in}{2.925408in}}%
\pgfpathlineto{\pgfqpoint{2.183372in}{2.925408in}}%
\pgfpathlineto{\pgfqpoint{2.183372in}{2.915230in}}%
\pgfpathlineto{\pgfqpoint{2.183372in}{0.330055in}}%
\pgfpathlineto{\pgfqpoint{2.183372in}{0.319877in}}%
\pgfpathclose%
\pgfusepath{stroke}%
\end{pgfscope}%
\end{pgfpicture}%
\makeatother%
\endgroup%

    \vspace*{-0.4cm}
	\caption{500 K. Bin size $0.018e$}
	\end{subfigure}
	\quad
	\begin{subfigure}[b]{0.45\textwidth}
	\hspace*{-0.4cm}
	%% Creator: Matplotlib, PGF backend
%%
%% To include the figure in your LaTeX document, write
%%   \input{<filename>.pgf}
%%
%% Make sure the required packages are loaded in your preamble
%%   \usepackage{pgf}
%%
%% Figures using additional raster images can only be included by \input if
%% they are in the same directory as the main LaTeX file. For loading figures
%% from other directories you can use the `import` package
%%   \usepackage{import}
%% and then include the figures with
%%   \import{<path to file>}{<filename>.pgf}
%%
%% Matplotlib used the following preamble
%%   \usepackage[utf8x]{inputenc}
%%   \usepackage[T1]{fontenc}
%%
\begingroup%
\makeatletter%
\begin{pgfpicture}%
\pgfpathrectangle{\pgfpointorigin}{\pgfqpoint{2.519483in}{3.060408in}}%
\pgfusepath{use as bounding box, clip}%
\begin{pgfscope}%
\pgfsetbuttcap%
\pgfsetmiterjoin%
\definecolor{currentfill}{rgb}{1.000000,1.000000,1.000000}%
\pgfsetfillcolor{currentfill}%
\pgfsetlinewidth{0.000000pt}%
\definecolor{currentstroke}{rgb}{1.000000,1.000000,1.000000}%
\pgfsetstrokecolor{currentstroke}%
\pgfsetdash{}{0pt}%
\pgfpathmoveto{\pgfqpoint{0.000000in}{0.000000in}}%
\pgfpathlineto{\pgfqpoint{2.519483in}{0.000000in}}%
\pgfpathlineto{\pgfqpoint{2.519483in}{3.060408in}}%
\pgfpathlineto{\pgfqpoint{0.000000in}{3.060408in}}%
\pgfpathclose%
\pgfusepath{fill}%
\end{pgfscope}%
\begin{pgfscope}%
\pgfsetbuttcap%
\pgfsetmiterjoin%
\definecolor{currentfill}{rgb}{1.000000,1.000000,1.000000}%
\pgfsetfillcolor{currentfill}%
\pgfsetlinewidth{0.000000pt}%
\definecolor{currentstroke}{rgb}{0.000000,0.000000,0.000000}%
\pgfsetstrokecolor{currentstroke}%
\pgfsetstrokeopacity{0.000000}%
\pgfsetdash{}{0pt}%
\pgfpathmoveto{\pgfqpoint{0.374692in}{0.319877in}}%
\pgfpathlineto{\pgfqpoint{1.954366in}{0.319877in}}%
\pgfpathlineto{\pgfqpoint{1.954366in}{2.925408in}}%
\pgfpathlineto{\pgfqpoint{0.374692in}{2.925408in}}%
\pgfpathclose%
\pgfusepath{fill}%
\end{pgfscope}%
\begin{pgfscope}%
\pgfpathrectangle{\pgfqpoint{0.374692in}{0.319877in}}{\pgfqpoint{1.579674in}{2.605531in}} %
\pgfusepath{clip}%
\pgfsys@transformshift{0.374692in}{0.319877in}%
\pgftext[left,bottom]{\pgfimage[interpolate=true,width=1.580000in,height=2.610000in]{RnnNorm_vs_dq_Ti_1000K-img0.png}}%
\end{pgfscope}%
\begin{pgfscope}%
\pgfpathrectangle{\pgfqpoint{0.374692in}{0.319877in}}{\pgfqpoint{1.579674in}{2.605531in}} %
\pgfusepath{clip}%
\pgfsetbuttcap%
\pgfsetroundjoin%
\definecolor{currentfill}{rgb}{1.000000,0.752941,0.796078}%
\pgfsetfillcolor{currentfill}%
\pgfsetlinewidth{1.003750pt}%
\definecolor{currentstroke}{rgb}{1.000000,0.752941,0.796078}%
\pgfsetstrokecolor{currentstroke}%
\pgfsetdash{}{0pt}%
\pgfpathmoveto{\pgfqpoint{0.473422in}{2.108763in}}%
\pgfpathcurveto{\pgfqpoint{0.484472in}{2.108763in}}{\pgfqpoint{0.495071in}{2.113153in}}{\pgfqpoint{0.502884in}{2.120967in}}%
\pgfpathcurveto{\pgfqpoint{0.510698in}{2.128780in}}{\pgfqpoint{0.515088in}{2.139379in}}{\pgfqpoint{0.515088in}{2.150430in}}%
\pgfpathcurveto{\pgfqpoint{0.515088in}{2.161480in}}{\pgfqpoint{0.510698in}{2.172079in}}{\pgfqpoint{0.502884in}{2.179892in}}%
\pgfpathcurveto{\pgfqpoint{0.495071in}{2.187706in}}{\pgfqpoint{0.484472in}{2.192096in}}{\pgfqpoint{0.473422in}{2.192096in}}%
\pgfpathcurveto{\pgfqpoint{0.462371in}{2.192096in}}{\pgfqpoint{0.451772in}{2.187706in}}{\pgfqpoint{0.443959in}{2.179892in}}%
\pgfpathcurveto{\pgfqpoint{0.436145in}{2.172079in}}{\pgfqpoint{0.431755in}{2.161480in}}{\pgfqpoint{0.431755in}{2.150430in}}%
\pgfpathcurveto{\pgfqpoint{0.431755in}{2.139379in}}{\pgfqpoint{0.436145in}{2.128780in}}{\pgfqpoint{0.443959in}{2.120967in}}%
\pgfpathcurveto{\pgfqpoint{0.451772in}{2.113153in}}{\pgfqpoint{0.462371in}{2.108763in}}{\pgfqpoint{0.473422in}{2.108763in}}%
\pgfpathclose%
\pgfusepath{stroke,fill}%
\end{pgfscope}%
\begin{pgfscope}%
\pgfpathrectangle{\pgfqpoint{0.374692in}{0.319877in}}{\pgfqpoint{1.579674in}{2.605531in}} %
\pgfusepath{clip}%
\pgfsetbuttcap%
\pgfsetroundjoin%
\definecolor{currentfill}{rgb}{1.000000,0.752941,0.796078}%
\pgfsetfillcolor{currentfill}%
\pgfsetlinewidth{1.003750pt}%
\definecolor{currentstroke}{rgb}{1.000000,0.752941,0.796078}%
\pgfsetstrokecolor{currentstroke}%
\pgfsetdash{}{0pt}%
\pgfpathmoveto{\pgfqpoint{0.670881in}{1.986599in}}%
\pgfpathcurveto{\pgfqpoint{0.681931in}{1.986599in}}{\pgfqpoint{0.692530in}{1.990989in}}{\pgfqpoint{0.700344in}{1.998803in}}%
\pgfpathcurveto{\pgfqpoint{0.708157in}{2.006616in}}{\pgfqpoint{0.712547in}{2.017215in}}{\pgfqpoint{0.712547in}{2.028265in}}%
\pgfpathcurveto{\pgfqpoint{0.712547in}{2.039316in}}{\pgfqpoint{0.708157in}{2.049915in}}{\pgfqpoint{0.700344in}{2.057728in}}%
\pgfpathcurveto{\pgfqpoint{0.692530in}{2.065542in}}{\pgfqpoint{0.681931in}{2.069932in}}{\pgfqpoint{0.670881in}{2.069932in}}%
\pgfpathcurveto{\pgfqpoint{0.659831in}{2.069932in}}{\pgfqpoint{0.649232in}{2.065542in}}{\pgfqpoint{0.641418in}{2.057728in}}%
\pgfpathcurveto{\pgfqpoint{0.633604in}{2.049915in}}{\pgfqpoint{0.629214in}{2.039316in}}{\pgfqpoint{0.629214in}{2.028265in}}%
\pgfpathcurveto{\pgfqpoint{0.629214in}{2.017215in}}{\pgfqpoint{0.633604in}{2.006616in}}{\pgfqpoint{0.641418in}{1.998803in}}%
\pgfpathcurveto{\pgfqpoint{0.649232in}{1.990989in}}{\pgfqpoint{0.659831in}{1.986599in}}{\pgfqpoint{0.670881in}{1.986599in}}%
\pgfpathclose%
\pgfusepath{stroke,fill}%
\end{pgfscope}%
\begin{pgfscope}%
\pgfpathrectangle{\pgfqpoint{0.374692in}{0.319877in}}{\pgfqpoint{1.579674in}{2.605531in}} %
\pgfusepath{clip}%
\pgfsetbuttcap%
\pgfsetroundjoin%
\definecolor{currentfill}{rgb}{1.000000,0.752941,0.796078}%
\pgfsetfillcolor{currentfill}%
\pgfsetlinewidth{1.003750pt}%
\definecolor{currentstroke}{rgb}{1.000000,0.752941,0.796078}%
\pgfsetstrokecolor{currentstroke}%
\pgfsetdash{}{0pt}%
\pgfpathmoveto{\pgfqpoint{0.868340in}{1.848892in}}%
\pgfpathcurveto{\pgfqpoint{0.879390in}{1.848892in}}{\pgfqpoint{0.889989in}{1.853282in}}{\pgfqpoint{0.897803in}{1.861095in}}%
\pgfpathcurveto{\pgfqpoint{0.905616in}{1.868909in}}{\pgfqpoint{0.910007in}{1.879508in}}{\pgfqpoint{0.910007in}{1.890558in}}%
\pgfpathcurveto{\pgfqpoint{0.910007in}{1.901608in}}{\pgfqpoint{0.905616in}{1.912207in}}{\pgfqpoint{0.897803in}{1.920021in}}%
\pgfpathcurveto{\pgfqpoint{0.889989in}{1.927835in}}{\pgfqpoint{0.879390in}{1.932225in}}{\pgfqpoint{0.868340in}{1.932225in}}%
\pgfpathcurveto{\pgfqpoint{0.857290in}{1.932225in}}{\pgfqpoint{0.846691in}{1.927835in}}{\pgfqpoint{0.838877in}{1.920021in}}%
\pgfpathcurveto{\pgfqpoint{0.831064in}{1.912207in}}{\pgfqpoint{0.826673in}{1.901608in}}{\pgfqpoint{0.826673in}{1.890558in}}%
\pgfpathcurveto{\pgfqpoint{0.826673in}{1.879508in}}{\pgfqpoint{0.831064in}{1.868909in}}{\pgfqpoint{0.838877in}{1.861095in}}%
\pgfpathcurveto{\pgfqpoint{0.846691in}{1.853282in}}{\pgfqpoint{0.857290in}{1.848892in}}{\pgfqpoint{0.868340in}{1.848892in}}%
\pgfpathclose%
\pgfusepath{stroke,fill}%
\end{pgfscope}%
\begin{pgfscope}%
\pgfpathrectangle{\pgfqpoint{0.374692in}{0.319877in}}{\pgfqpoint{1.579674in}{2.605531in}} %
\pgfusepath{clip}%
\pgfsetbuttcap%
\pgfsetroundjoin%
\definecolor{currentfill}{rgb}{1.000000,0.752941,0.796078}%
\pgfsetfillcolor{currentfill}%
\pgfsetlinewidth{1.003750pt}%
\definecolor{currentstroke}{rgb}{1.000000,0.752941,0.796078}%
\pgfsetstrokecolor{currentstroke}%
\pgfsetdash{}{0pt}%
\pgfpathmoveto{\pgfqpoint{1.065799in}{1.687654in}}%
\pgfpathcurveto{\pgfqpoint{1.076849in}{1.687654in}}{\pgfqpoint{1.087448in}{1.692044in}}{\pgfqpoint{1.095262in}{1.699858in}}%
\pgfpathcurveto{\pgfqpoint{1.103076in}{1.707671in}}{\pgfqpoint{1.107466in}{1.718271in}}{\pgfqpoint{1.107466in}{1.729321in}}%
\pgfpathcurveto{\pgfqpoint{1.107466in}{1.740371in}}{\pgfqpoint{1.103076in}{1.750970in}}{\pgfqpoint{1.095262in}{1.758783in}}%
\pgfpathcurveto{\pgfqpoint{1.087448in}{1.766597in}}{\pgfqpoint{1.076849in}{1.770987in}}{\pgfqpoint{1.065799in}{1.770987in}}%
\pgfpathcurveto{\pgfqpoint{1.054749in}{1.770987in}}{\pgfqpoint{1.044150in}{1.766597in}}{\pgfqpoint{1.036336in}{1.758783in}}%
\pgfpathcurveto{\pgfqpoint{1.028523in}{1.750970in}}{\pgfqpoint{1.024133in}{1.740371in}}{\pgfqpoint{1.024133in}{1.729321in}}%
\pgfpathcurveto{\pgfqpoint{1.024133in}{1.718271in}}{\pgfqpoint{1.028523in}{1.707671in}}{\pgfqpoint{1.036336in}{1.699858in}}%
\pgfpathcurveto{\pgfqpoint{1.044150in}{1.692044in}}{\pgfqpoint{1.054749in}{1.687654in}}{\pgfqpoint{1.065799in}{1.687654in}}%
\pgfpathclose%
\pgfusepath{stroke,fill}%
\end{pgfscope}%
\begin{pgfscope}%
\pgfpathrectangle{\pgfqpoint{0.374692in}{0.319877in}}{\pgfqpoint{1.579674in}{2.605531in}} %
\pgfusepath{clip}%
\pgfsetbuttcap%
\pgfsetroundjoin%
\definecolor{currentfill}{rgb}{1.000000,0.752941,0.796078}%
\pgfsetfillcolor{currentfill}%
\pgfsetlinewidth{1.003750pt}%
\definecolor{currentstroke}{rgb}{1.000000,0.752941,0.796078}%
\pgfsetstrokecolor{currentstroke}%
\pgfsetdash{}{0pt}%
\pgfpathmoveto{\pgfqpoint{1.263258in}{1.558999in}}%
\pgfpathcurveto{\pgfqpoint{1.274309in}{1.558999in}}{\pgfqpoint{1.284908in}{1.563390in}}{\pgfqpoint{1.292721in}{1.571203in}}%
\pgfpathcurveto{\pgfqpoint{1.300535in}{1.579017in}}{\pgfqpoint{1.304925in}{1.589616in}}{\pgfqpoint{1.304925in}{1.600666in}}%
\pgfpathcurveto{\pgfqpoint{1.304925in}{1.611716in}}{\pgfqpoint{1.300535in}{1.622315in}}{\pgfqpoint{1.292721in}{1.630129in}}%
\pgfpathcurveto{\pgfqpoint{1.284908in}{1.637943in}}{\pgfqpoint{1.274309in}{1.642333in}}{\pgfqpoint{1.263258in}{1.642333in}}%
\pgfpathcurveto{\pgfqpoint{1.252208in}{1.642333in}}{\pgfqpoint{1.241609in}{1.637943in}}{\pgfqpoint{1.233796in}{1.630129in}}%
\pgfpathcurveto{\pgfqpoint{1.225982in}{1.622315in}}{\pgfqpoint{1.221592in}{1.611716in}}{\pgfqpoint{1.221592in}{1.600666in}}%
\pgfpathcurveto{\pgfqpoint{1.221592in}{1.589616in}}{\pgfqpoint{1.225982in}{1.579017in}}{\pgfqpoint{1.233796in}{1.571203in}}%
\pgfpathcurveto{\pgfqpoint{1.241609in}{1.563390in}}{\pgfqpoint{1.252208in}{1.558999in}}{\pgfqpoint{1.263258in}{1.558999in}}%
\pgfpathclose%
\pgfusepath{stroke,fill}%
\end{pgfscope}%
\begin{pgfscope}%
\pgfpathrectangle{\pgfqpoint{0.374692in}{0.319877in}}{\pgfqpoint{1.579674in}{2.605531in}} %
\pgfusepath{clip}%
\pgfsetbuttcap%
\pgfsetroundjoin%
\definecolor{currentfill}{rgb}{1.000000,0.752941,0.796078}%
\pgfsetfillcolor{currentfill}%
\pgfsetlinewidth{1.003750pt}%
\definecolor{currentstroke}{rgb}{1.000000,0.752941,0.796078}%
\pgfsetstrokecolor{currentstroke}%
\pgfsetdash{}{0pt}%
\pgfpathmoveto{\pgfqpoint{1.460718in}{1.523278in}}%
\pgfpathcurveto{\pgfqpoint{1.471768in}{1.523278in}}{\pgfqpoint{1.482367in}{1.527668in}}{\pgfqpoint{1.490180in}{1.535482in}}%
\pgfpathcurveto{\pgfqpoint{1.497994in}{1.543295in}}{\pgfqpoint{1.502384in}{1.553894in}}{\pgfqpoint{1.502384in}{1.564944in}}%
\pgfpathcurveto{\pgfqpoint{1.502384in}{1.575995in}}{\pgfqpoint{1.497994in}{1.586594in}}{\pgfqpoint{1.490180in}{1.594407in}}%
\pgfpathcurveto{\pgfqpoint{1.482367in}{1.602221in}}{\pgfqpoint{1.471768in}{1.606611in}}{\pgfqpoint{1.460718in}{1.606611in}}%
\pgfpathcurveto{\pgfqpoint{1.449668in}{1.606611in}}{\pgfqpoint{1.439069in}{1.602221in}}{\pgfqpoint{1.431255in}{1.594407in}}%
\pgfpathcurveto{\pgfqpoint{1.423441in}{1.586594in}}{\pgfqpoint{1.419051in}{1.575995in}}{\pgfqpoint{1.419051in}{1.564944in}}%
\pgfpathcurveto{\pgfqpoint{1.419051in}{1.553894in}}{\pgfqpoint{1.423441in}{1.543295in}}{\pgfqpoint{1.431255in}{1.535482in}}%
\pgfpathcurveto{\pgfqpoint{1.439069in}{1.527668in}}{\pgfqpoint{1.449668in}{1.523278in}}{\pgfqpoint{1.460718in}{1.523278in}}%
\pgfpathclose%
\pgfusepath{stroke,fill}%
\end{pgfscope}%
\begin{pgfscope}%
\pgfpathrectangle{\pgfqpoint{0.374692in}{0.319877in}}{\pgfqpoint{1.579674in}{2.605531in}} %
\pgfusepath{clip}%
\pgfsetbuttcap%
\pgfsetroundjoin%
\definecolor{currentfill}{rgb}{1.000000,0.752941,0.796078}%
\pgfsetfillcolor{currentfill}%
\pgfsetlinewidth{1.003750pt}%
\definecolor{currentstroke}{rgb}{1.000000,0.752941,0.796078}%
\pgfsetstrokecolor{currentstroke}%
\pgfsetdash{}{0pt}%
\pgfpathmoveto{\pgfqpoint{1.658177in}{1.334620in}}%
\pgfpathcurveto{\pgfqpoint{1.669227in}{1.334620in}}{\pgfqpoint{1.679826in}{1.339010in}}{\pgfqpoint{1.687640in}{1.346824in}}%
\pgfpathcurveto{\pgfqpoint{1.695453in}{1.354637in}}{\pgfqpoint{1.699844in}{1.365236in}}{\pgfqpoint{1.699844in}{1.376286in}}%
\pgfpathcurveto{\pgfqpoint{1.699844in}{1.387336in}}{\pgfqpoint{1.695453in}{1.397936in}}{\pgfqpoint{1.687640in}{1.405749in}}%
\pgfpathcurveto{\pgfqpoint{1.679826in}{1.413563in}}{\pgfqpoint{1.669227in}{1.417953in}}{\pgfqpoint{1.658177in}{1.417953in}}%
\pgfpathcurveto{\pgfqpoint{1.647127in}{1.417953in}}{\pgfqpoint{1.636528in}{1.413563in}}{\pgfqpoint{1.628714in}{1.405749in}}%
\pgfpathcurveto{\pgfqpoint{1.620901in}{1.397936in}}{\pgfqpoint{1.616510in}{1.387336in}}{\pgfqpoint{1.616510in}{1.376286in}}%
\pgfpathcurveto{\pgfqpoint{1.616510in}{1.365236in}}{\pgfqpoint{1.620901in}{1.354637in}}{\pgfqpoint{1.628714in}{1.346824in}}%
\pgfpathcurveto{\pgfqpoint{1.636528in}{1.339010in}}{\pgfqpoint{1.647127in}{1.334620in}}{\pgfqpoint{1.658177in}{1.334620in}}%
\pgfpathclose%
\pgfusepath{stroke,fill}%
\end{pgfscope}%
\begin{pgfscope}%
\pgfpathrectangle{\pgfqpoint{0.374692in}{0.319877in}}{\pgfqpoint{1.579674in}{2.605531in}} %
\pgfusepath{clip}%
\pgfsetbuttcap%
\pgfsetroundjoin%
\definecolor{currentfill}{rgb}{1.000000,0.752941,0.796078}%
\pgfsetfillcolor{currentfill}%
\pgfsetlinewidth{1.003750pt}%
\definecolor{currentstroke}{rgb}{1.000000,0.752941,0.796078}%
\pgfsetstrokecolor{currentstroke}%
\pgfsetdash{}{0pt}%
\pgfpathmoveto{\pgfqpoint{1.855636in}{1.380551in}}%
\pgfpathcurveto{\pgfqpoint{1.866686in}{1.380551in}}{\pgfqpoint{1.877285in}{1.384941in}}{\pgfqpoint{1.885099in}{1.392754in}}%
\pgfpathcurveto{\pgfqpoint{1.892913in}{1.400568in}}{\pgfqpoint{1.897303in}{1.411167in}}{\pgfqpoint{1.897303in}{1.422217in}}%
\pgfpathcurveto{\pgfqpoint{1.897303in}{1.433267in}}{\pgfqpoint{1.892913in}{1.443866in}}{\pgfqpoint{1.885099in}{1.451680in}}%
\pgfpathcurveto{\pgfqpoint{1.877285in}{1.459494in}}{\pgfqpoint{1.866686in}{1.463884in}}{\pgfqpoint{1.855636in}{1.463884in}}%
\pgfpathcurveto{\pgfqpoint{1.844586in}{1.463884in}}{\pgfqpoint{1.833987in}{1.459494in}}{\pgfqpoint{1.826173in}{1.451680in}}%
\pgfpathcurveto{\pgfqpoint{1.818360in}{1.443866in}}{\pgfqpoint{1.813969in}{1.433267in}}{\pgfqpoint{1.813969in}{1.422217in}}%
\pgfpathcurveto{\pgfqpoint{1.813969in}{1.411167in}}{\pgfqpoint{1.818360in}{1.400568in}}{\pgfqpoint{1.826173in}{1.392754in}}%
\pgfpathcurveto{\pgfqpoint{1.833987in}{1.384941in}}{\pgfqpoint{1.844586in}{1.380551in}}{\pgfqpoint{1.855636in}{1.380551in}}%
\pgfpathclose%
\pgfusepath{stroke,fill}%
\end{pgfscope}%
\begin{pgfscope}%
\pgfsetbuttcap%
\pgfsetroundjoin%
\definecolor{currentfill}{rgb}{0.000000,0.000000,0.000000}%
\pgfsetfillcolor{currentfill}%
\pgfsetlinewidth{0.803000pt}%
\definecolor{currentstroke}{rgb}{0.000000,0.000000,0.000000}%
\pgfsetstrokecolor{currentstroke}%
\pgfsetdash{}{0pt}%
\pgfsys@defobject{currentmarker}{\pgfqpoint{0.000000in}{-0.048611in}}{\pgfqpoint{0.000000in}{0.000000in}}{%
\pgfpathmoveto{\pgfqpoint{0.000000in}{0.000000in}}%
\pgfpathlineto{\pgfqpoint{0.000000in}{-0.048611in}}%
\pgfusepath{stroke,fill}%
}%
\begin{pgfscope}%
\pgfsys@transformshift{0.670881in}{0.319877in}%
\pgfsys@useobject{currentmarker}{}%
\end{pgfscope}%
\end{pgfscope}%
\begin{pgfscope}%
\pgftext[x=0.670881in,y=0.222655in,,top]{\rmfamily\fontsize{10.000000}{12.000000}\selectfont \(\displaystyle -0.05\)}%
\end{pgfscope}%
\begin{pgfscope}%
\pgfsetbuttcap%
\pgfsetroundjoin%
\definecolor{currentfill}{rgb}{0.000000,0.000000,0.000000}%
\pgfsetfillcolor{currentfill}%
\pgfsetlinewidth{0.803000pt}%
\definecolor{currentstroke}{rgb}{0.000000,0.000000,0.000000}%
\pgfsetstrokecolor{currentstroke}%
\pgfsetdash{}{0pt}%
\pgfsys@defobject{currentmarker}{\pgfqpoint{0.000000in}{-0.048611in}}{\pgfqpoint{0.000000in}{0.000000in}}{%
\pgfpathmoveto{\pgfqpoint{0.000000in}{0.000000in}}%
\pgfpathlineto{\pgfqpoint{0.000000in}{-0.048611in}}%
\pgfusepath{stroke,fill}%
}%
\begin{pgfscope}%
\pgfsys@transformshift{1.164529in}{0.319877in}%
\pgfsys@useobject{currentmarker}{}%
\end{pgfscope}%
\end{pgfscope}%
\begin{pgfscope}%
\pgftext[x=1.164529in,y=0.222655in,,top]{\rmfamily\fontsize{10.000000}{12.000000}\selectfont \(\displaystyle 0.00\)}%
\end{pgfscope}%
\begin{pgfscope}%
\pgfsetbuttcap%
\pgfsetroundjoin%
\definecolor{currentfill}{rgb}{0.000000,0.000000,0.000000}%
\pgfsetfillcolor{currentfill}%
\pgfsetlinewidth{0.803000pt}%
\definecolor{currentstroke}{rgb}{0.000000,0.000000,0.000000}%
\pgfsetstrokecolor{currentstroke}%
\pgfsetdash{}{0pt}%
\pgfsys@defobject{currentmarker}{\pgfqpoint{0.000000in}{-0.048611in}}{\pgfqpoint{0.000000in}{0.000000in}}{%
\pgfpathmoveto{\pgfqpoint{0.000000in}{0.000000in}}%
\pgfpathlineto{\pgfqpoint{0.000000in}{-0.048611in}}%
\pgfusepath{stroke,fill}%
}%
\begin{pgfscope}%
\pgfsys@transformshift{1.658177in}{0.319877in}%
\pgfsys@useobject{currentmarker}{}%
\end{pgfscope}%
\end{pgfscope}%
\begin{pgfscope}%
\pgftext[x=1.658177in,y=0.222655in,,top]{\rmfamily\fontsize{10.000000}{12.000000}\selectfont \(\displaystyle 0.05\)}%
\end{pgfscope}%
\begin{pgfscope}%
\pgfsetbuttcap%
\pgfsetroundjoin%
\definecolor{currentfill}{rgb}{0.000000,0.000000,0.000000}%
\pgfsetfillcolor{currentfill}%
\pgfsetlinewidth{0.803000pt}%
\definecolor{currentstroke}{rgb}{0.000000,0.000000,0.000000}%
\pgfsetstrokecolor{currentstroke}%
\pgfsetdash{}{0pt}%
\pgfsys@defobject{currentmarker}{\pgfqpoint{-0.048611in}{0.000000in}}{\pgfqpoint{0.000000in}{0.000000in}}{%
\pgfpathmoveto{\pgfqpoint{0.000000in}{0.000000in}}%
\pgfpathlineto{\pgfqpoint{-0.048611in}{0.000000in}}%
\pgfusepath{stroke,fill}%
}%
\begin{pgfscope}%
\pgfsys@transformshift{0.374692in}{0.622908in}%
\pgfsys@useobject{currentmarker}{}%
\end{pgfscope}%
\end{pgfscope}%
\begin{pgfscope}%
\pgftext[x=0.100000in,y=0.575080in,left,base]{\rmfamily\fontsize{10.000000}{12.000000}\selectfont \(\displaystyle 3.6\)}%
\end{pgfscope}%
\begin{pgfscope}%
\pgfsetbuttcap%
\pgfsetroundjoin%
\definecolor{currentfill}{rgb}{0.000000,0.000000,0.000000}%
\pgfsetfillcolor{currentfill}%
\pgfsetlinewidth{0.803000pt}%
\definecolor{currentstroke}{rgb}{0.000000,0.000000,0.000000}%
\pgfsetstrokecolor{currentstroke}%
\pgfsetdash{}{0pt}%
\pgfsys@defobject{currentmarker}{\pgfqpoint{-0.048611in}{0.000000in}}{\pgfqpoint{0.000000in}{0.000000in}}{%
\pgfpathmoveto{\pgfqpoint{0.000000in}{0.000000in}}%
\pgfpathlineto{\pgfqpoint{-0.048611in}{0.000000in}}%
\pgfusepath{stroke,fill}%
}%
\begin{pgfscope}%
\pgfsys@transformshift{0.374692in}{1.067280in}%
\pgfsys@useobject{currentmarker}{}%
\end{pgfscope}%
\end{pgfscope}%
\begin{pgfscope}%
\pgftext[x=0.100000in,y=1.019453in,left,base]{\rmfamily\fontsize{10.000000}{12.000000}\selectfont \(\displaystyle 3.7\)}%
\end{pgfscope}%
\begin{pgfscope}%
\pgfsetbuttcap%
\pgfsetroundjoin%
\definecolor{currentfill}{rgb}{0.000000,0.000000,0.000000}%
\pgfsetfillcolor{currentfill}%
\pgfsetlinewidth{0.803000pt}%
\definecolor{currentstroke}{rgb}{0.000000,0.000000,0.000000}%
\pgfsetstrokecolor{currentstroke}%
\pgfsetdash{}{0pt}%
\pgfsys@defobject{currentmarker}{\pgfqpoint{-0.048611in}{0.000000in}}{\pgfqpoint{0.000000in}{0.000000in}}{%
\pgfpathmoveto{\pgfqpoint{0.000000in}{0.000000in}}%
\pgfpathlineto{\pgfqpoint{-0.048611in}{0.000000in}}%
\pgfusepath{stroke,fill}%
}%
\begin{pgfscope}%
\pgfsys@transformshift{0.374692in}{1.511652in}%
\pgfsys@useobject{currentmarker}{}%
\end{pgfscope}%
\end{pgfscope}%
\begin{pgfscope}%
\pgftext[x=0.100000in,y=1.463825in,left,base]{\rmfamily\fontsize{10.000000}{12.000000}\selectfont \(\displaystyle 3.8\)}%
\end{pgfscope}%
\begin{pgfscope}%
\pgfsetbuttcap%
\pgfsetroundjoin%
\definecolor{currentfill}{rgb}{0.000000,0.000000,0.000000}%
\pgfsetfillcolor{currentfill}%
\pgfsetlinewidth{0.803000pt}%
\definecolor{currentstroke}{rgb}{0.000000,0.000000,0.000000}%
\pgfsetstrokecolor{currentstroke}%
\pgfsetdash{}{0pt}%
\pgfsys@defobject{currentmarker}{\pgfqpoint{-0.048611in}{0.000000in}}{\pgfqpoint{0.000000in}{0.000000in}}{%
\pgfpathmoveto{\pgfqpoint{0.000000in}{0.000000in}}%
\pgfpathlineto{\pgfqpoint{-0.048611in}{0.000000in}}%
\pgfusepath{stroke,fill}%
}%
\begin{pgfscope}%
\pgfsys@transformshift{0.374692in}{1.956024in}%
\pgfsys@useobject{currentmarker}{}%
\end{pgfscope}%
\end{pgfscope}%
\begin{pgfscope}%
\pgftext[x=0.100000in,y=1.908197in,left,base]{\rmfamily\fontsize{10.000000}{12.000000}\selectfont \(\displaystyle 3.9\)}%
\end{pgfscope}%
\begin{pgfscope}%
\pgfsetbuttcap%
\pgfsetroundjoin%
\definecolor{currentfill}{rgb}{0.000000,0.000000,0.000000}%
\pgfsetfillcolor{currentfill}%
\pgfsetlinewidth{0.803000pt}%
\definecolor{currentstroke}{rgb}{0.000000,0.000000,0.000000}%
\pgfsetstrokecolor{currentstroke}%
\pgfsetdash{}{0pt}%
\pgfsys@defobject{currentmarker}{\pgfqpoint{-0.048611in}{0.000000in}}{\pgfqpoint{0.000000in}{0.000000in}}{%
\pgfpathmoveto{\pgfqpoint{0.000000in}{0.000000in}}%
\pgfpathlineto{\pgfqpoint{-0.048611in}{0.000000in}}%
\pgfusepath{stroke,fill}%
}%
\begin{pgfscope}%
\pgfsys@transformshift{0.374692in}{2.400396in}%
\pgfsys@useobject{currentmarker}{}%
\end{pgfscope}%
\end{pgfscope}%
\begin{pgfscope}%
\pgftext[x=0.100000in,y=2.352569in,left,base]{\rmfamily\fontsize{10.000000}{12.000000}\selectfont \(\displaystyle 4.0\)}%
\end{pgfscope}%
\begin{pgfscope}%
\pgfsetbuttcap%
\pgfsetroundjoin%
\definecolor{currentfill}{rgb}{0.000000,0.000000,0.000000}%
\pgfsetfillcolor{currentfill}%
\pgfsetlinewidth{0.803000pt}%
\definecolor{currentstroke}{rgb}{0.000000,0.000000,0.000000}%
\pgfsetstrokecolor{currentstroke}%
\pgfsetdash{}{0pt}%
\pgfsys@defobject{currentmarker}{\pgfqpoint{-0.048611in}{0.000000in}}{\pgfqpoint{0.000000in}{0.000000in}}{%
\pgfpathmoveto{\pgfqpoint{0.000000in}{0.000000in}}%
\pgfpathlineto{\pgfqpoint{-0.048611in}{0.000000in}}%
\pgfusepath{stroke,fill}%
}%
\begin{pgfscope}%
\pgfsys@transformshift{0.374692in}{2.844769in}%
\pgfsys@useobject{currentmarker}{}%
\end{pgfscope}%
\end{pgfscope}%
\begin{pgfscope}%
\pgftext[x=0.100000in,y=2.796941in,left,base]{\rmfamily\fontsize{10.000000}{12.000000}\selectfont \(\displaystyle 4.1\)}%
\end{pgfscope}%
\begin{pgfscope}%
\pgfsetrectcap%
\pgfsetmiterjoin%
\pgfsetlinewidth{0.803000pt}%
\definecolor{currentstroke}{rgb}{0.000000,0.000000,0.000000}%
\pgfsetstrokecolor{currentstroke}%
\pgfsetdash{}{0pt}%
\pgfpathmoveto{\pgfqpoint{0.374692in}{0.319877in}}%
\pgfpathlineto{\pgfqpoint{0.374692in}{2.925408in}}%
\pgfusepath{stroke}%
\end{pgfscope}%
\begin{pgfscope}%
\pgfsetrectcap%
\pgfsetmiterjoin%
\pgfsetlinewidth{0.803000pt}%
\definecolor{currentstroke}{rgb}{0.000000,0.000000,0.000000}%
\pgfsetstrokecolor{currentstroke}%
\pgfsetdash{}{0pt}%
\pgfpathmoveto{\pgfqpoint{1.954366in}{0.319877in}}%
\pgfpathlineto{\pgfqpoint{1.954366in}{2.925408in}}%
\pgfusepath{stroke}%
\end{pgfscope}%
\begin{pgfscope}%
\pgfsetrectcap%
\pgfsetmiterjoin%
\pgfsetlinewidth{0.803000pt}%
\definecolor{currentstroke}{rgb}{0.000000,0.000000,0.000000}%
\pgfsetstrokecolor{currentstroke}%
\pgfsetdash{}{0pt}%
\pgfpathmoveto{\pgfqpoint{0.374692in}{0.319877in}}%
\pgfpathlineto{\pgfqpoint{1.954366in}{0.319877in}}%
\pgfusepath{stroke}%
\end{pgfscope}%
\begin{pgfscope}%
\pgfsetrectcap%
\pgfsetmiterjoin%
\pgfsetlinewidth{0.803000pt}%
\definecolor{currentstroke}{rgb}{0.000000,0.000000,0.000000}%
\pgfsetstrokecolor{currentstroke}%
\pgfsetdash{}{0pt}%
\pgfpathmoveto{\pgfqpoint{0.374692in}{2.925408in}}%
\pgfpathlineto{\pgfqpoint{1.954366in}{2.925408in}}%
\pgfusepath{stroke}%
\end{pgfscope}%
\begin{pgfscope}%
\pgfpathrectangle{\pgfqpoint{2.053095in}{0.319877in}}{\pgfqpoint{0.130277in}{2.605531in}} %
\pgfusepath{clip}%
\pgfsetbuttcap%
\pgfsetmiterjoin%
\definecolor{currentfill}{rgb}{1.000000,1.000000,1.000000}%
\pgfsetfillcolor{currentfill}%
\pgfsetlinewidth{0.010037pt}%
\definecolor{currentstroke}{rgb}{1.000000,1.000000,1.000000}%
\pgfsetstrokecolor{currentstroke}%
\pgfsetdash{}{0pt}%
\pgfpathmoveto{\pgfqpoint{2.053095in}{0.319877in}}%
\pgfpathlineto{\pgfqpoint{2.053095in}{0.330055in}}%
\pgfpathlineto{\pgfqpoint{2.053095in}{2.915230in}}%
\pgfpathlineto{\pgfqpoint{2.053095in}{2.925408in}}%
\pgfpathlineto{\pgfqpoint{2.183372in}{2.925408in}}%
\pgfpathlineto{\pgfqpoint{2.183372in}{2.915230in}}%
\pgfpathlineto{\pgfqpoint{2.183372in}{0.330055in}}%
\pgfpathlineto{\pgfqpoint{2.183372in}{0.319877in}}%
\pgfpathclose%
\pgfusepath{stroke,fill}%
\end{pgfscope}%
\begin{pgfscope}%
\pgfsys@transformshift{2.050000in}{0.320408in}%
\pgftext[left,bottom]{\pgfimage[interpolate=true,width=0.130000in,height=2.610000in]{RnnNorm_vs_dq_Ti_1000K-img1.png}}%
\end{pgfscope}%
\begin{pgfscope}%
\pgfsetbuttcap%
\pgfsetroundjoin%
\definecolor{currentfill}{rgb}{0.000000,0.000000,0.000000}%
\pgfsetfillcolor{currentfill}%
\pgfsetlinewidth{0.803000pt}%
\definecolor{currentstroke}{rgb}{0.000000,0.000000,0.000000}%
\pgfsetstrokecolor{currentstroke}%
\pgfsetdash{}{0pt}%
\pgfsys@defobject{currentmarker}{\pgfqpoint{0.000000in}{0.000000in}}{\pgfqpoint{0.048611in}{0.000000in}}{%
\pgfpathmoveto{\pgfqpoint{0.000000in}{0.000000in}}%
\pgfpathlineto{\pgfqpoint{0.048611in}{0.000000in}}%
\pgfusepath{stroke,fill}%
}%
\begin{pgfscope}%
\pgfsys@transformshift{2.183372in}{0.319877in}%
\pgfsys@useobject{currentmarker}{}%
\end{pgfscope}%
\end{pgfscope}%
\begin{pgfscope}%
\pgftext[x=2.280594in,y=0.272050in,left,base]{\rmfamily\fontsize{10.000000}{12.000000}\selectfont \(\displaystyle 0\)}%
\end{pgfscope}%
\begin{pgfscope}%
\pgfsetbuttcap%
\pgfsetroundjoin%
\definecolor{currentfill}{rgb}{0.000000,0.000000,0.000000}%
\pgfsetfillcolor{currentfill}%
\pgfsetlinewidth{0.803000pt}%
\definecolor{currentstroke}{rgb}{0.000000,0.000000,0.000000}%
\pgfsetstrokecolor{currentstroke}%
\pgfsetdash{}{0pt}%
\pgfsys@defobject{currentmarker}{\pgfqpoint{0.000000in}{0.000000in}}{\pgfqpoint{0.048611in}{0.000000in}}{%
\pgfpathmoveto{\pgfqpoint{0.000000in}{0.000000in}}%
\pgfpathlineto{\pgfqpoint{0.048611in}{0.000000in}}%
\pgfusepath{stroke,fill}%
}%
\begin{pgfscope}%
\pgfsys@transformshift{2.183372in}{0.793610in}%
\pgfsys@useobject{currentmarker}{}%
\end{pgfscope}%
\end{pgfscope}%
\begin{pgfscope}%
\pgftext[x=2.280594in,y=0.745782in,left,base]{\rmfamily\fontsize{10.000000}{12.000000}\selectfont \(\displaystyle 10\)}%
\end{pgfscope}%
\begin{pgfscope}%
\pgfsetbuttcap%
\pgfsetroundjoin%
\definecolor{currentfill}{rgb}{0.000000,0.000000,0.000000}%
\pgfsetfillcolor{currentfill}%
\pgfsetlinewidth{0.803000pt}%
\definecolor{currentstroke}{rgb}{0.000000,0.000000,0.000000}%
\pgfsetstrokecolor{currentstroke}%
\pgfsetdash{}{0pt}%
\pgfsys@defobject{currentmarker}{\pgfqpoint{0.000000in}{0.000000in}}{\pgfqpoint{0.048611in}{0.000000in}}{%
\pgfpathmoveto{\pgfqpoint{0.000000in}{0.000000in}}%
\pgfpathlineto{\pgfqpoint{0.048611in}{0.000000in}}%
\pgfusepath{stroke,fill}%
}%
\begin{pgfscope}%
\pgfsys@transformshift{2.183372in}{1.267343in}%
\pgfsys@useobject{currentmarker}{}%
\end{pgfscope}%
\end{pgfscope}%
\begin{pgfscope}%
\pgftext[x=2.280594in,y=1.219515in,left,base]{\rmfamily\fontsize{10.000000}{12.000000}\selectfont \(\displaystyle 20\)}%
\end{pgfscope}%
\begin{pgfscope}%
\pgfsetbuttcap%
\pgfsetroundjoin%
\definecolor{currentfill}{rgb}{0.000000,0.000000,0.000000}%
\pgfsetfillcolor{currentfill}%
\pgfsetlinewidth{0.803000pt}%
\definecolor{currentstroke}{rgb}{0.000000,0.000000,0.000000}%
\pgfsetstrokecolor{currentstroke}%
\pgfsetdash{}{0pt}%
\pgfsys@defobject{currentmarker}{\pgfqpoint{0.000000in}{0.000000in}}{\pgfqpoint{0.048611in}{0.000000in}}{%
\pgfpathmoveto{\pgfqpoint{0.000000in}{0.000000in}}%
\pgfpathlineto{\pgfqpoint{0.048611in}{0.000000in}}%
\pgfusepath{stroke,fill}%
}%
\begin{pgfscope}%
\pgfsys@transformshift{2.183372in}{1.741076in}%
\pgfsys@useobject{currentmarker}{}%
\end{pgfscope}%
\end{pgfscope}%
\begin{pgfscope}%
\pgftext[x=2.280594in,y=1.693248in,left,base]{\rmfamily\fontsize{10.000000}{12.000000}\selectfont \(\displaystyle 30\)}%
\end{pgfscope}%
\begin{pgfscope}%
\pgfsetbuttcap%
\pgfsetroundjoin%
\definecolor{currentfill}{rgb}{0.000000,0.000000,0.000000}%
\pgfsetfillcolor{currentfill}%
\pgfsetlinewidth{0.803000pt}%
\definecolor{currentstroke}{rgb}{0.000000,0.000000,0.000000}%
\pgfsetstrokecolor{currentstroke}%
\pgfsetdash{}{0pt}%
\pgfsys@defobject{currentmarker}{\pgfqpoint{0.000000in}{0.000000in}}{\pgfqpoint{0.048611in}{0.000000in}}{%
\pgfpathmoveto{\pgfqpoint{0.000000in}{0.000000in}}%
\pgfpathlineto{\pgfqpoint{0.048611in}{0.000000in}}%
\pgfusepath{stroke,fill}%
}%
\begin{pgfscope}%
\pgfsys@transformshift{2.183372in}{2.214809in}%
\pgfsys@useobject{currentmarker}{}%
\end{pgfscope}%
\end{pgfscope}%
\begin{pgfscope}%
\pgftext[x=2.280594in,y=2.166981in,left,base]{\rmfamily\fontsize{10.000000}{12.000000}\selectfont \(\displaystyle 40\)}%
\end{pgfscope}%
\begin{pgfscope}%
\pgfsetbuttcap%
\pgfsetroundjoin%
\definecolor{currentfill}{rgb}{0.000000,0.000000,0.000000}%
\pgfsetfillcolor{currentfill}%
\pgfsetlinewidth{0.803000pt}%
\definecolor{currentstroke}{rgb}{0.000000,0.000000,0.000000}%
\pgfsetstrokecolor{currentstroke}%
\pgfsetdash{}{0pt}%
\pgfsys@defobject{currentmarker}{\pgfqpoint{0.000000in}{0.000000in}}{\pgfqpoint{0.048611in}{0.000000in}}{%
\pgfpathmoveto{\pgfqpoint{0.000000in}{0.000000in}}%
\pgfpathlineto{\pgfqpoint{0.048611in}{0.000000in}}%
\pgfusepath{stroke,fill}%
}%
\begin{pgfscope}%
\pgfsys@transformshift{2.183372in}{2.688541in}%
\pgfsys@useobject{currentmarker}{}%
\end{pgfscope}%
\end{pgfscope}%
\begin{pgfscope}%
\pgftext[x=2.280594in,y=2.640714in,left,base]{\rmfamily\fontsize{10.000000}{12.000000}\selectfont \(\displaystyle 50\)}%
\end{pgfscope}%
\begin{pgfscope}%
\pgfsetbuttcap%
\pgfsetmiterjoin%
\pgfsetlinewidth{0.803000pt}%
\definecolor{currentstroke}{rgb}{0.000000,0.000000,0.000000}%
\pgfsetstrokecolor{currentstroke}%
\pgfsetdash{}{0pt}%
\pgfpathmoveto{\pgfqpoint{2.053095in}{0.319877in}}%
\pgfpathlineto{\pgfqpoint{2.053095in}{0.330055in}}%
\pgfpathlineto{\pgfqpoint{2.053095in}{2.915230in}}%
\pgfpathlineto{\pgfqpoint{2.053095in}{2.925408in}}%
\pgfpathlineto{\pgfqpoint{2.183372in}{2.925408in}}%
\pgfpathlineto{\pgfqpoint{2.183372in}{2.915230in}}%
\pgfpathlineto{\pgfqpoint{2.183372in}{0.330055in}}%
\pgfpathlineto{\pgfqpoint{2.183372in}{0.319877in}}%
\pgfpathclose%
\pgfusepath{stroke}%
\end{pgfscope}%
\end{pgfpicture}%
\makeatother%
\endgroup%

    \vspace*{-0.4cm}
	\caption{1000 K. Bin size $0.020e$}
	\end{subfigure}
\caption{Change in distance to nearest neighbours of Ti vs change in Ti charge}
\label{on_site_RnnNorm_vs_dq}
\end{figure}

4) Figure \ref{on_site_RnnStDev_vs_dq} compares the change in charge to the standard deviation of the distance to nearest neighbour Oxygen ions as compared to the average distance to the shell for a given Ti ion. Defining the expression in equation \ref{RnnAve} as $\bar{\text{R}}^{\text{NN}}_{i,I}$, the exact expression represented by the y-axis is:
\begin{align*}
R^{\text{NN}}_{s,i;I} &\equiv \sqrt{\sum_{\alpha = x,y,z}\left(R_{s,I}^{\alpha}-R_{i,I}^{\alpha}\right)^2} \\
y_{i,I} &\equiv \sqrt{\frac{\sum_{s\in \text{NN}_i}\left(R^{\text{NN}}_{s,i;I} - \bar{\text{R}}^{\text{NN}}_{i,I} \right)^2}{6-1}}
\end{align*}
which stands for the standard deviation of the nearest neighbour bond length from the average. 

\begin{figure}[h!]
\centering
	\begin{subfigure}[b]{0.45\textwidth}
	\hspace*{-0.4cm}
	%% Creator: Matplotlib, PGF backend
%%
%% To include the figure in your LaTeX document, write
%%   \input{<filename>.pgf}
%%
%% Make sure the required packages are loaded in your preamble
%%   \usepackage{pgf}
%%
%% Figures using additional raster images can only be included by \input if
%% they are in the same directory as the main LaTeX file. For loading figures
%% from other directories you can use the `import` package
%%   \usepackage{import}
%% and then include the figures with
%%   \import{<path to file>}{<filename>.pgf}
%%
%% Matplotlib used the following preamble
%%   \usepackage[utf8x]{inputenc}
%%   \usepackage[T1]{fontenc}
%%
\begingroup%
\makeatletter%
\begin{pgfpicture}%
\pgfpathrectangle{\pgfpointorigin}{\pgfqpoint{2.535687in}{3.060408in}}%
\pgfusepath{use as bounding box, clip}%
\begin{pgfscope}%
\pgfsetbuttcap%
\pgfsetmiterjoin%
\definecolor{currentfill}{rgb}{1.000000,1.000000,1.000000}%
\pgfsetfillcolor{currentfill}%
\pgfsetlinewidth{0.000000pt}%
\definecolor{currentstroke}{rgb}{1.000000,1.000000,1.000000}%
\pgfsetstrokecolor{currentstroke}%
\pgfsetdash{}{0pt}%
\pgfpathmoveto{\pgfqpoint{0.000000in}{0.000000in}}%
\pgfpathlineto{\pgfqpoint{2.535687in}{0.000000in}}%
\pgfpathlineto{\pgfqpoint{2.535687in}{3.060408in}}%
\pgfpathlineto{\pgfqpoint{0.000000in}{3.060408in}}%
\pgfpathclose%
\pgfusepath{fill}%
\end{pgfscope}%
\begin{pgfscope}%
\pgfsetbuttcap%
\pgfsetmiterjoin%
\definecolor{currentfill}{rgb}{1.000000,1.000000,1.000000}%
\pgfsetfillcolor{currentfill}%
\pgfsetlinewidth{0.000000pt}%
\definecolor{currentstroke}{rgb}{0.000000,0.000000,0.000000}%
\pgfsetstrokecolor{currentstroke}%
\pgfsetstrokeopacity{0.000000}%
\pgfsetdash{}{0pt}%
\pgfpathmoveto{\pgfqpoint{0.374692in}{0.319877in}}%
\pgfpathlineto{\pgfqpoint{1.867946in}{0.319877in}}%
\pgfpathlineto{\pgfqpoint{1.867946in}{2.925408in}}%
\pgfpathlineto{\pgfqpoint{0.374692in}{2.925408in}}%
\pgfpathclose%
\pgfusepath{fill}%
\end{pgfscope}%
\begin{pgfscope}%
\pgfpathrectangle{\pgfqpoint{0.374692in}{0.319877in}}{\pgfqpoint{1.493254in}{2.605531in}} %
\pgfusepath{clip}%
\pgfsys@transformshift{0.374692in}{0.319877in}%
\pgftext[left,bottom]{\pgfimage[interpolate=true,width=1.500000in,height=2.610000in]{RnnStDev_vs_dq_Ti_100K-img0.png}}%
\end{pgfscope}%
\begin{pgfscope}%
\pgfpathrectangle{\pgfqpoint{0.374692in}{0.319877in}}{\pgfqpoint{1.493254in}{2.605531in}} %
\pgfusepath{clip}%
\pgfsetbuttcap%
\pgfsetroundjoin%
\definecolor{currentfill}{rgb}{1.000000,0.752941,0.796078}%
\pgfsetfillcolor{currentfill}%
\pgfsetlinewidth{1.003750pt}%
\definecolor{currentstroke}{rgb}{1.000000,0.752941,0.796078}%
\pgfsetstrokecolor{currentstroke}%
\pgfsetdash{}{0pt}%
\pgfpathmoveto{\pgfqpoint{0.922218in}{0.950816in}}%
\pgfpathcurveto{\pgfqpoint{0.933268in}{0.950816in}}{\pgfqpoint{0.943867in}{0.955207in}}{\pgfqpoint{0.951681in}{0.963020in}}%
\pgfpathcurveto{\pgfqpoint{0.959495in}{0.970834in}}{\pgfqpoint{0.963885in}{0.981433in}}{\pgfqpoint{0.963885in}{0.992483in}}%
\pgfpathcurveto{\pgfqpoint{0.963885in}{1.003533in}}{\pgfqpoint{0.959495in}{1.014132in}}{\pgfqpoint{0.951681in}{1.021946in}}%
\pgfpathcurveto{\pgfqpoint{0.943867in}{1.029759in}}{\pgfqpoint{0.933268in}{1.034150in}}{\pgfqpoint{0.922218in}{1.034150in}}%
\pgfpathcurveto{\pgfqpoint{0.911168in}{1.034150in}}{\pgfqpoint{0.900569in}{1.029759in}}{\pgfqpoint{0.892756in}{1.021946in}}%
\pgfpathcurveto{\pgfqpoint{0.884942in}{1.014132in}}{\pgfqpoint{0.880552in}{1.003533in}}{\pgfqpoint{0.880552in}{0.992483in}}%
\pgfpathcurveto{\pgfqpoint{0.880552in}{0.981433in}}{\pgfqpoint{0.884942in}{0.970834in}}{\pgfqpoint{0.892756in}{0.963020in}}%
\pgfpathcurveto{\pgfqpoint{0.900569in}{0.955207in}}{\pgfqpoint{0.911168in}{0.950816in}}{\pgfqpoint{0.922218in}{0.950816in}}%
\pgfpathclose%
\pgfusepath{stroke,fill}%
\end{pgfscope}%
\begin{pgfscope}%
\pgfpathrectangle{\pgfqpoint{0.374692in}{0.319877in}}{\pgfqpoint{1.493254in}{2.605531in}} %
\pgfusepath{clip}%
\pgfsetbuttcap%
\pgfsetroundjoin%
\definecolor{currentfill}{rgb}{1.000000,0.752941,0.796078}%
\pgfsetfillcolor{currentfill}%
\pgfsetlinewidth{1.003750pt}%
\definecolor{currentstroke}{rgb}{1.000000,0.752941,0.796078}%
\pgfsetstrokecolor{currentstroke}%
\pgfsetdash{}{0pt}%
\pgfpathmoveto{\pgfqpoint{1.021769in}{1.051358in}}%
\pgfpathcurveto{\pgfqpoint{1.032819in}{1.051358in}}{\pgfqpoint{1.043418in}{1.055748in}}{\pgfqpoint{1.051231in}{1.063562in}}%
\pgfpathcurveto{\pgfqpoint{1.059045in}{1.071376in}}{\pgfqpoint{1.063435in}{1.081975in}}{\pgfqpoint{1.063435in}{1.093025in}}%
\pgfpathcurveto{\pgfqpoint{1.063435in}{1.104075in}}{\pgfqpoint{1.059045in}{1.114674in}}{\pgfqpoint{1.051231in}{1.122488in}}%
\pgfpathcurveto{\pgfqpoint{1.043418in}{1.130301in}}{\pgfqpoint{1.032819in}{1.134691in}}{\pgfqpoint{1.021769in}{1.134691in}}%
\pgfpathcurveto{\pgfqpoint{1.010718in}{1.134691in}}{\pgfqpoint{1.000119in}{1.130301in}}{\pgfqpoint{0.992306in}{1.122488in}}%
\pgfpathcurveto{\pgfqpoint{0.984492in}{1.114674in}}{\pgfqpoint{0.980102in}{1.104075in}}{\pgfqpoint{0.980102in}{1.093025in}}%
\pgfpathcurveto{\pgfqpoint{0.980102in}{1.081975in}}{\pgfqpoint{0.984492in}{1.071376in}}{\pgfqpoint{0.992306in}{1.063562in}}%
\pgfpathcurveto{\pgfqpoint{1.000119in}{1.055748in}}{\pgfqpoint{1.010718in}{1.051358in}}{\pgfqpoint{1.021769in}{1.051358in}}%
\pgfpathclose%
\pgfusepath{stroke,fill}%
\end{pgfscope}%
\begin{pgfscope}%
\pgfpathrectangle{\pgfqpoint{0.374692in}{0.319877in}}{\pgfqpoint{1.493254in}{2.605531in}} %
\pgfusepath{clip}%
\pgfsetbuttcap%
\pgfsetroundjoin%
\definecolor{currentfill}{rgb}{1.000000,0.752941,0.796078}%
\pgfsetfillcolor{currentfill}%
\pgfsetlinewidth{1.003750pt}%
\definecolor{currentstroke}{rgb}{1.000000,0.752941,0.796078}%
\pgfsetstrokecolor{currentstroke}%
\pgfsetdash{}{0pt}%
\pgfpathmoveto{\pgfqpoint{1.121319in}{1.090079in}}%
\pgfpathcurveto{\pgfqpoint{1.132369in}{1.090079in}}{\pgfqpoint{1.142968in}{1.094469in}}{\pgfqpoint{1.150782in}{1.102283in}}%
\pgfpathcurveto{\pgfqpoint{1.158595in}{1.110096in}}{\pgfqpoint{1.162986in}{1.120695in}}{\pgfqpoint{1.162986in}{1.131746in}}%
\pgfpathcurveto{\pgfqpoint{1.162986in}{1.142796in}}{\pgfqpoint{1.158595in}{1.153395in}}{\pgfqpoint{1.150782in}{1.161208in}}%
\pgfpathcurveto{\pgfqpoint{1.142968in}{1.169022in}}{\pgfqpoint{1.132369in}{1.173412in}}{\pgfqpoint{1.121319in}{1.173412in}}%
\pgfpathcurveto{\pgfqpoint{1.110269in}{1.173412in}}{\pgfqpoint{1.099670in}{1.169022in}}{\pgfqpoint{1.091856in}{1.161208in}}%
\pgfpathcurveto{\pgfqpoint{1.084042in}{1.153395in}}{\pgfqpoint{1.079652in}{1.142796in}}{\pgfqpoint{1.079652in}{1.131746in}}%
\pgfpathcurveto{\pgfqpoint{1.079652in}{1.120695in}}{\pgfqpoint{1.084042in}{1.110096in}}{\pgfqpoint{1.091856in}{1.102283in}}%
\pgfpathcurveto{\pgfqpoint{1.099670in}{1.094469in}}{\pgfqpoint{1.110269in}{1.090079in}}{\pgfqpoint{1.121319in}{1.090079in}}%
\pgfpathclose%
\pgfusepath{stroke,fill}%
\end{pgfscope}%
\begin{pgfscope}%
\pgfpathrectangle{\pgfqpoint{0.374692in}{0.319877in}}{\pgfqpoint{1.493254in}{2.605531in}} %
\pgfusepath{clip}%
\pgfsetbuttcap%
\pgfsetroundjoin%
\definecolor{currentfill}{rgb}{1.000000,0.752941,0.796078}%
\pgfsetfillcolor{currentfill}%
\pgfsetlinewidth{1.003750pt}%
\definecolor{currentstroke}{rgb}{1.000000,0.752941,0.796078}%
\pgfsetstrokecolor{currentstroke}%
\pgfsetdash{}{0pt}%
\pgfpathmoveto{\pgfqpoint{1.220869in}{1.115248in}}%
\pgfpathcurveto{\pgfqpoint{1.231919in}{1.115248in}}{\pgfqpoint{1.242518in}{1.119638in}}{\pgfqpoint{1.250332in}{1.127452in}}%
\pgfpathcurveto{\pgfqpoint{1.258146in}{1.135265in}}{\pgfqpoint{1.262536in}{1.145864in}}{\pgfqpoint{1.262536in}{1.156914in}}%
\pgfpathcurveto{\pgfqpoint{1.262536in}{1.167964in}}{\pgfqpoint{1.258146in}{1.178563in}}{\pgfqpoint{1.250332in}{1.186377in}}%
\pgfpathcurveto{\pgfqpoint{1.242518in}{1.194191in}}{\pgfqpoint{1.231919in}{1.198581in}}{\pgfqpoint{1.220869in}{1.198581in}}%
\pgfpathcurveto{\pgfqpoint{1.209819in}{1.198581in}}{\pgfqpoint{1.199220in}{1.194191in}}{\pgfqpoint{1.191406in}{1.186377in}}%
\pgfpathcurveto{\pgfqpoint{1.183593in}{1.178563in}}{\pgfqpoint{1.179202in}{1.167964in}}{\pgfqpoint{1.179202in}{1.156914in}}%
\pgfpathcurveto{\pgfqpoint{1.179202in}{1.145864in}}{\pgfqpoint{1.183593in}{1.135265in}}{\pgfqpoint{1.191406in}{1.127452in}}%
\pgfpathcurveto{\pgfqpoint{1.199220in}{1.119638in}}{\pgfqpoint{1.209819in}{1.115248in}}{\pgfqpoint{1.220869in}{1.115248in}}%
\pgfpathclose%
\pgfusepath{stroke,fill}%
\end{pgfscope}%
\begin{pgfscope}%
\pgfpathrectangle{\pgfqpoint{0.374692in}{0.319877in}}{\pgfqpoint{1.493254in}{2.605531in}} %
\pgfusepath{clip}%
\pgfsetbuttcap%
\pgfsetroundjoin%
\definecolor{currentfill}{rgb}{1.000000,0.752941,0.796078}%
\pgfsetfillcolor{currentfill}%
\pgfsetlinewidth{1.003750pt}%
\definecolor{currentstroke}{rgb}{1.000000,0.752941,0.796078}%
\pgfsetstrokecolor{currentstroke}%
\pgfsetdash{}{0pt}%
\pgfpathmoveto{\pgfqpoint{1.320419in}{1.215287in}}%
\pgfpathcurveto{\pgfqpoint{1.331469in}{1.215287in}}{\pgfqpoint{1.342069in}{1.219678in}}{\pgfqpoint{1.349882in}{1.227491in}}%
\pgfpathcurveto{\pgfqpoint{1.357696in}{1.235305in}}{\pgfqpoint{1.362086in}{1.245904in}}{\pgfqpoint{1.362086in}{1.256954in}}%
\pgfpathcurveto{\pgfqpoint{1.362086in}{1.268004in}}{\pgfqpoint{1.357696in}{1.278603in}}{\pgfqpoint{1.349882in}{1.286417in}}%
\pgfpathcurveto{\pgfqpoint{1.342069in}{1.294231in}}{\pgfqpoint{1.331469in}{1.298621in}}{\pgfqpoint{1.320419in}{1.298621in}}%
\pgfpathcurveto{\pgfqpoint{1.309369in}{1.298621in}}{\pgfqpoint{1.298770in}{1.294231in}}{\pgfqpoint{1.290957in}{1.286417in}}%
\pgfpathcurveto{\pgfqpoint{1.283143in}{1.278603in}}{\pgfqpoint{1.278753in}{1.268004in}}{\pgfqpoint{1.278753in}{1.256954in}}%
\pgfpathcurveto{\pgfqpoint{1.278753in}{1.245904in}}{\pgfqpoint{1.283143in}{1.235305in}}{\pgfqpoint{1.290957in}{1.227491in}}%
\pgfpathcurveto{\pgfqpoint{1.298770in}{1.219678in}}{\pgfqpoint{1.309369in}{1.215287in}}{\pgfqpoint{1.320419in}{1.215287in}}%
\pgfpathclose%
\pgfusepath{stroke,fill}%
\end{pgfscope}%
\begin{pgfscope}%
\pgfsetbuttcap%
\pgfsetroundjoin%
\definecolor{currentfill}{rgb}{0.000000,0.000000,0.000000}%
\pgfsetfillcolor{currentfill}%
\pgfsetlinewidth{0.803000pt}%
\definecolor{currentstroke}{rgb}{0.000000,0.000000,0.000000}%
\pgfsetstrokecolor{currentstroke}%
\pgfsetdash{}{0pt}%
\pgfsys@defobject{currentmarker}{\pgfqpoint{0.000000in}{-0.048611in}}{\pgfqpoint{0.000000in}{0.000000in}}{%
\pgfpathmoveto{\pgfqpoint{0.000000in}{0.000000in}}%
\pgfpathlineto{\pgfqpoint{0.000000in}{-0.048611in}}%
\pgfusepath{stroke,fill}%
}%
\begin{pgfscope}%
\pgfsys@transformshift{0.654677in}{0.319877in}%
\pgfsys@useobject{currentmarker}{}%
\end{pgfscope}%
\end{pgfscope}%
\begin{pgfscope}%
\pgftext[x=0.654677in,y=0.222655in,,top]{\rmfamily\fontsize{10.000000}{12.000000}\selectfont \(\displaystyle -0.05\)}%
\end{pgfscope}%
\begin{pgfscope}%
\pgfsetbuttcap%
\pgfsetroundjoin%
\definecolor{currentfill}{rgb}{0.000000,0.000000,0.000000}%
\pgfsetfillcolor{currentfill}%
\pgfsetlinewidth{0.803000pt}%
\definecolor{currentstroke}{rgb}{0.000000,0.000000,0.000000}%
\pgfsetstrokecolor{currentstroke}%
\pgfsetdash{}{0pt}%
\pgfsys@defobject{currentmarker}{\pgfqpoint{0.000000in}{-0.048611in}}{\pgfqpoint{0.000000in}{0.000000in}}{%
\pgfpathmoveto{\pgfqpoint{0.000000in}{0.000000in}}%
\pgfpathlineto{\pgfqpoint{0.000000in}{-0.048611in}}%
\pgfusepath{stroke,fill}%
}%
\begin{pgfscope}%
\pgfsys@transformshift{1.121319in}{0.319877in}%
\pgfsys@useobject{currentmarker}{}%
\end{pgfscope}%
\end{pgfscope}%
\begin{pgfscope}%
\pgftext[x=1.121319in,y=0.222655in,,top]{\rmfamily\fontsize{10.000000}{12.000000}\selectfont \(\displaystyle 0.00\)}%
\end{pgfscope}%
\begin{pgfscope}%
\pgfsetbuttcap%
\pgfsetroundjoin%
\definecolor{currentfill}{rgb}{0.000000,0.000000,0.000000}%
\pgfsetfillcolor{currentfill}%
\pgfsetlinewidth{0.803000pt}%
\definecolor{currentstroke}{rgb}{0.000000,0.000000,0.000000}%
\pgfsetstrokecolor{currentstroke}%
\pgfsetdash{}{0pt}%
\pgfsys@defobject{currentmarker}{\pgfqpoint{0.000000in}{-0.048611in}}{\pgfqpoint{0.000000in}{0.000000in}}{%
\pgfpathmoveto{\pgfqpoint{0.000000in}{0.000000in}}%
\pgfpathlineto{\pgfqpoint{0.000000in}{-0.048611in}}%
\pgfusepath{stroke,fill}%
}%
\begin{pgfscope}%
\pgfsys@transformshift{1.587961in}{0.319877in}%
\pgfsys@useobject{currentmarker}{}%
\end{pgfscope}%
\end{pgfscope}%
\begin{pgfscope}%
\pgftext[x=1.587961in,y=0.222655in,,top]{\rmfamily\fontsize{10.000000}{12.000000}\selectfont \(\displaystyle 0.05\)}%
\end{pgfscope}%
\begin{pgfscope}%
\pgfsetbuttcap%
\pgfsetroundjoin%
\definecolor{currentfill}{rgb}{0.000000,0.000000,0.000000}%
\pgfsetfillcolor{currentfill}%
\pgfsetlinewidth{0.803000pt}%
\definecolor{currentstroke}{rgb}{0.000000,0.000000,0.000000}%
\pgfsetstrokecolor{currentstroke}%
\pgfsetdash{}{0pt}%
\pgfsys@defobject{currentmarker}{\pgfqpoint{-0.048611in}{0.000000in}}{\pgfqpoint{0.000000in}{0.000000in}}{%
\pgfpathmoveto{\pgfqpoint{0.000000in}{0.000000in}}%
\pgfpathlineto{\pgfqpoint{-0.048611in}{0.000000in}}%
\pgfusepath{stroke,fill}%
}%
\begin{pgfscope}%
\pgfsys@transformshift{0.374692in}{0.436809in}%
\pgfsys@useobject{currentmarker}{}%
\end{pgfscope}%
\end{pgfscope}%
\begin{pgfscope}%
\pgftext[x=0.100000in,y=0.388981in,left,base]{\rmfamily\fontsize{10.000000}{12.000000}\selectfont \(\displaystyle 0.0\)}%
\end{pgfscope}%
\begin{pgfscope}%
\pgfsetbuttcap%
\pgfsetroundjoin%
\definecolor{currentfill}{rgb}{0.000000,0.000000,0.000000}%
\pgfsetfillcolor{currentfill}%
\pgfsetlinewidth{0.803000pt}%
\definecolor{currentstroke}{rgb}{0.000000,0.000000,0.000000}%
\pgfsetstrokecolor{currentstroke}%
\pgfsetdash{}{0pt}%
\pgfsys@defobject{currentmarker}{\pgfqpoint{-0.048611in}{0.000000in}}{\pgfqpoint{0.000000in}{0.000000in}}{%
\pgfpathmoveto{\pgfqpoint{0.000000in}{0.000000in}}%
\pgfpathlineto{\pgfqpoint{-0.048611in}{0.000000in}}%
\pgfusepath{stroke,fill}%
}%
\begin{pgfscope}%
\pgfsys@transformshift{0.374692in}{0.740854in}%
\pgfsys@useobject{currentmarker}{}%
\end{pgfscope}%
\end{pgfscope}%
\begin{pgfscope}%
\pgftext[x=0.100000in,y=0.693026in,left,base]{\rmfamily\fontsize{10.000000}{12.000000}\selectfont \(\displaystyle 0.1\)}%
\end{pgfscope}%
\begin{pgfscope}%
\pgfsetbuttcap%
\pgfsetroundjoin%
\definecolor{currentfill}{rgb}{0.000000,0.000000,0.000000}%
\pgfsetfillcolor{currentfill}%
\pgfsetlinewidth{0.803000pt}%
\definecolor{currentstroke}{rgb}{0.000000,0.000000,0.000000}%
\pgfsetstrokecolor{currentstroke}%
\pgfsetdash{}{0pt}%
\pgfsys@defobject{currentmarker}{\pgfqpoint{-0.048611in}{0.000000in}}{\pgfqpoint{0.000000in}{0.000000in}}{%
\pgfpathmoveto{\pgfqpoint{0.000000in}{0.000000in}}%
\pgfpathlineto{\pgfqpoint{-0.048611in}{0.000000in}}%
\pgfusepath{stroke,fill}%
}%
\begin{pgfscope}%
\pgfsys@transformshift{0.374692in}{1.044899in}%
\pgfsys@useobject{currentmarker}{}%
\end{pgfscope}%
\end{pgfscope}%
\begin{pgfscope}%
\pgftext[x=0.100000in,y=0.997071in,left,base]{\rmfamily\fontsize{10.000000}{12.000000}\selectfont \(\displaystyle 0.2\)}%
\end{pgfscope}%
\begin{pgfscope}%
\pgfsetbuttcap%
\pgfsetroundjoin%
\definecolor{currentfill}{rgb}{0.000000,0.000000,0.000000}%
\pgfsetfillcolor{currentfill}%
\pgfsetlinewidth{0.803000pt}%
\definecolor{currentstroke}{rgb}{0.000000,0.000000,0.000000}%
\pgfsetstrokecolor{currentstroke}%
\pgfsetdash{}{0pt}%
\pgfsys@defobject{currentmarker}{\pgfqpoint{-0.048611in}{0.000000in}}{\pgfqpoint{0.000000in}{0.000000in}}{%
\pgfpathmoveto{\pgfqpoint{0.000000in}{0.000000in}}%
\pgfpathlineto{\pgfqpoint{-0.048611in}{0.000000in}}%
\pgfusepath{stroke,fill}%
}%
\begin{pgfscope}%
\pgfsys@transformshift{0.374692in}{1.348944in}%
\pgfsys@useobject{currentmarker}{}%
\end{pgfscope}%
\end{pgfscope}%
\begin{pgfscope}%
\pgftext[x=0.100000in,y=1.301116in,left,base]{\rmfamily\fontsize{10.000000}{12.000000}\selectfont \(\displaystyle 0.3\)}%
\end{pgfscope}%
\begin{pgfscope}%
\pgfsetbuttcap%
\pgfsetroundjoin%
\definecolor{currentfill}{rgb}{0.000000,0.000000,0.000000}%
\pgfsetfillcolor{currentfill}%
\pgfsetlinewidth{0.803000pt}%
\definecolor{currentstroke}{rgb}{0.000000,0.000000,0.000000}%
\pgfsetstrokecolor{currentstroke}%
\pgfsetdash{}{0pt}%
\pgfsys@defobject{currentmarker}{\pgfqpoint{-0.048611in}{0.000000in}}{\pgfqpoint{0.000000in}{0.000000in}}{%
\pgfpathmoveto{\pgfqpoint{0.000000in}{0.000000in}}%
\pgfpathlineto{\pgfqpoint{-0.048611in}{0.000000in}}%
\pgfusepath{stroke,fill}%
}%
\begin{pgfscope}%
\pgfsys@transformshift{0.374692in}{1.652989in}%
\pgfsys@useobject{currentmarker}{}%
\end{pgfscope}%
\end{pgfscope}%
\begin{pgfscope}%
\pgftext[x=0.100000in,y=1.605161in,left,base]{\rmfamily\fontsize{10.000000}{12.000000}\selectfont \(\displaystyle 0.4\)}%
\end{pgfscope}%
\begin{pgfscope}%
\pgfsetbuttcap%
\pgfsetroundjoin%
\definecolor{currentfill}{rgb}{0.000000,0.000000,0.000000}%
\pgfsetfillcolor{currentfill}%
\pgfsetlinewidth{0.803000pt}%
\definecolor{currentstroke}{rgb}{0.000000,0.000000,0.000000}%
\pgfsetstrokecolor{currentstroke}%
\pgfsetdash{}{0pt}%
\pgfsys@defobject{currentmarker}{\pgfqpoint{-0.048611in}{0.000000in}}{\pgfqpoint{0.000000in}{0.000000in}}{%
\pgfpathmoveto{\pgfqpoint{0.000000in}{0.000000in}}%
\pgfpathlineto{\pgfqpoint{-0.048611in}{0.000000in}}%
\pgfusepath{stroke,fill}%
}%
\begin{pgfscope}%
\pgfsys@transformshift{0.374692in}{1.957034in}%
\pgfsys@useobject{currentmarker}{}%
\end{pgfscope}%
\end{pgfscope}%
\begin{pgfscope}%
\pgftext[x=0.100000in,y=1.909207in,left,base]{\rmfamily\fontsize{10.000000}{12.000000}\selectfont \(\displaystyle 0.5\)}%
\end{pgfscope}%
\begin{pgfscope}%
\pgfsetbuttcap%
\pgfsetroundjoin%
\definecolor{currentfill}{rgb}{0.000000,0.000000,0.000000}%
\pgfsetfillcolor{currentfill}%
\pgfsetlinewidth{0.803000pt}%
\definecolor{currentstroke}{rgb}{0.000000,0.000000,0.000000}%
\pgfsetstrokecolor{currentstroke}%
\pgfsetdash{}{0pt}%
\pgfsys@defobject{currentmarker}{\pgfqpoint{-0.048611in}{0.000000in}}{\pgfqpoint{0.000000in}{0.000000in}}{%
\pgfpathmoveto{\pgfqpoint{0.000000in}{0.000000in}}%
\pgfpathlineto{\pgfqpoint{-0.048611in}{0.000000in}}%
\pgfusepath{stroke,fill}%
}%
\begin{pgfscope}%
\pgfsys@transformshift{0.374692in}{2.261079in}%
\pgfsys@useobject{currentmarker}{}%
\end{pgfscope}%
\end{pgfscope}%
\begin{pgfscope}%
\pgftext[x=0.100000in,y=2.213252in,left,base]{\rmfamily\fontsize{10.000000}{12.000000}\selectfont \(\displaystyle 0.6\)}%
\end{pgfscope}%
\begin{pgfscope}%
\pgfsetbuttcap%
\pgfsetroundjoin%
\definecolor{currentfill}{rgb}{0.000000,0.000000,0.000000}%
\pgfsetfillcolor{currentfill}%
\pgfsetlinewidth{0.803000pt}%
\definecolor{currentstroke}{rgb}{0.000000,0.000000,0.000000}%
\pgfsetstrokecolor{currentstroke}%
\pgfsetdash{}{0pt}%
\pgfsys@defobject{currentmarker}{\pgfqpoint{-0.048611in}{0.000000in}}{\pgfqpoint{0.000000in}{0.000000in}}{%
\pgfpathmoveto{\pgfqpoint{0.000000in}{0.000000in}}%
\pgfpathlineto{\pgfqpoint{-0.048611in}{0.000000in}}%
\pgfusepath{stroke,fill}%
}%
\begin{pgfscope}%
\pgfsys@transformshift{0.374692in}{2.565124in}%
\pgfsys@useobject{currentmarker}{}%
\end{pgfscope}%
\end{pgfscope}%
\begin{pgfscope}%
\pgftext[x=0.100000in,y=2.517297in,left,base]{\rmfamily\fontsize{10.000000}{12.000000}\selectfont \(\displaystyle 0.7\)}%
\end{pgfscope}%
\begin{pgfscope}%
\pgfsetbuttcap%
\pgfsetroundjoin%
\definecolor{currentfill}{rgb}{0.000000,0.000000,0.000000}%
\pgfsetfillcolor{currentfill}%
\pgfsetlinewidth{0.803000pt}%
\definecolor{currentstroke}{rgb}{0.000000,0.000000,0.000000}%
\pgfsetstrokecolor{currentstroke}%
\pgfsetdash{}{0pt}%
\pgfsys@defobject{currentmarker}{\pgfqpoint{-0.048611in}{0.000000in}}{\pgfqpoint{0.000000in}{0.000000in}}{%
\pgfpathmoveto{\pgfqpoint{0.000000in}{0.000000in}}%
\pgfpathlineto{\pgfqpoint{-0.048611in}{0.000000in}}%
\pgfusepath{stroke,fill}%
}%
\begin{pgfscope}%
\pgfsys@transformshift{0.374692in}{2.869169in}%
\pgfsys@useobject{currentmarker}{}%
\end{pgfscope}%
\end{pgfscope}%
\begin{pgfscope}%
\pgftext[x=0.100000in,y=2.821342in,left,base]{\rmfamily\fontsize{10.000000}{12.000000}\selectfont \(\displaystyle 0.8\)}%
\end{pgfscope}%
\begin{pgfscope}%
\pgfsetrectcap%
\pgfsetmiterjoin%
\pgfsetlinewidth{0.803000pt}%
\definecolor{currentstroke}{rgb}{0.000000,0.000000,0.000000}%
\pgfsetstrokecolor{currentstroke}%
\pgfsetdash{}{0pt}%
\pgfpathmoveto{\pgfqpoint{0.374692in}{0.319877in}}%
\pgfpathlineto{\pgfqpoint{0.374692in}{2.925408in}}%
\pgfusepath{stroke}%
\end{pgfscope}%
\begin{pgfscope}%
\pgfsetrectcap%
\pgfsetmiterjoin%
\pgfsetlinewidth{0.803000pt}%
\definecolor{currentstroke}{rgb}{0.000000,0.000000,0.000000}%
\pgfsetstrokecolor{currentstroke}%
\pgfsetdash{}{0pt}%
\pgfpathmoveto{\pgfqpoint{1.867946in}{0.319877in}}%
\pgfpathlineto{\pgfqpoint{1.867946in}{2.925408in}}%
\pgfusepath{stroke}%
\end{pgfscope}%
\begin{pgfscope}%
\pgfsetrectcap%
\pgfsetmiterjoin%
\pgfsetlinewidth{0.803000pt}%
\definecolor{currentstroke}{rgb}{0.000000,0.000000,0.000000}%
\pgfsetstrokecolor{currentstroke}%
\pgfsetdash{}{0pt}%
\pgfpathmoveto{\pgfqpoint{0.374692in}{0.319877in}}%
\pgfpathlineto{\pgfqpoint{1.867946in}{0.319877in}}%
\pgfusepath{stroke}%
\end{pgfscope}%
\begin{pgfscope}%
\pgfsetrectcap%
\pgfsetmiterjoin%
\pgfsetlinewidth{0.803000pt}%
\definecolor{currentstroke}{rgb}{0.000000,0.000000,0.000000}%
\pgfsetstrokecolor{currentstroke}%
\pgfsetdash{}{0pt}%
\pgfpathmoveto{\pgfqpoint{0.374692in}{2.925408in}}%
\pgfpathlineto{\pgfqpoint{1.867946in}{2.925408in}}%
\pgfusepath{stroke}%
\end{pgfscope}%
\begin{pgfscope}%
\pgfpathrectangle{\pgfqpoint{1.961274in}{0.319877in}}{\pgfqpoint{0.130277in}{2.605531in}} %
\pgfusepath{clip}%
\pgfsetbuttcap%
\pgfsetmiterjoin%
\definecolor{currentfill}{rgb}{1.000000,1.000000,1.000000}%
\pgfsetfillcolor{currentfill}%
\pgfsetlinewidth{0.010037pt}%
\definecolor{currentstroke}{rgb}{1.000000,1.000000,1.000000}%
\pgfsetstrokecolor{currentstroke}%
\pgfsetdash{}{0pt}%
\pgfpathmoveto{\pgfqpoint{1.961274in}{0.319877in}}%
\pgfpathlineto{\pgfqpoint{1.961274in}{0.330055in}}%
\pgfpathlineto{\pgfqpoint{1.961274in}{2.915230in}}%
\pgfpathlineto{\pgfqpoint{1.961274in}{2.925408in}}%
\pgfpathlineto{\pgfqpoint{2.091551in}{2.925408in}}%
\pgfpathlineto{\pgfqpoint{2.091551in}{2.915230in}}%
\pgfpathlineto{\pgfqpoint{2.091551in}{0.330055in}}%
\pgfpathlineto{\pgfqpoint{2.091551in}{0.319877in}}%
\pgfpathclose%
\pgfusepath{stroke,fill}%
\end{pgfscope}%
\begin{pgfscope}%
\pgfsys@transformshift{1.960000in}{0.320408in}%
\pgftext[left,bottom]{\pgfimage[interpolate=true,width=0.130000in,height=2.610000in]{RnnStDev_vs_dq_Ti_100K-img1.png}}%
\end{pgfscope}%
\begin{pgfscope}%
\pgfsetbuttcap%
\pgfsetroundjoin%
\definecolor{currentfill}{rgb}{0.000000,0.000000,0.000000}%
\pgfsetfillcolor{currentfill}%
\pgfsetlinewidth{0.803000pt}%
\definecolor{currentstroke}{rgb}{0.000000,0.000000,0.000000}%
\pgfsetstrokecolor{currentstroke}%
\pgfsetdash{}{0pt}%
\pgfsys@defobject{currentmarker}{\pgfqpoint{0.000000in}{0.000000in}}{\pgfqpoint{0.048611in}{0.000000in}}{%
\pgfpathmoveto{\pgfqpoint{0.000000in}{0.000000in}}%
\pgfpathlineto{\pgfqpoint{0.048611in}{0.000000in}}%
\pgfusepath{stroke,fill}%
}%
\begin{pgfscope}%
\pgfsys@transformshift{2.091551in}{0.319877in}%
\pgfsys@useobject{currentmarker}{}%
\end{pgfscope}%
\end{pgfscope}%
\begin{pgfscope}%
\pgftext[x=2.188773in,y=0.272050in,left,base]{\rmfamily\fontsize{10.000000}{12.000000}\selectfont \(\displaystyle 0.0\)}%
\end{pgfscope}%
\begin{pgfscope}%
\pgfsetbuttcap%
\pgfsetroundjoin%
\definecolor{currentfill}{rgb}{0.000000,0.000000,0.000000}%
\pgfsetfillcolor{currentfill}%
\pgfsetlinewidth{0.803000pt}%
\definecolor{currentstroke}{rgb}{0.000000,0.000000,0.000000}%
\pgfsetstrokecolor{currentstroke}%
\pgfsetdash{}{0pt}%
\pgfsys@defobject{currentmarker}{\pgfqpoint{0.000000in}{0.000000in}}{\pgfqpoint{0.048611in}{0.000000in}}{%
\pgfpathmoveto{\pgfqpoint{0.000000in}{0.000000in}}%
\pgfpathlineto{\pgfqpoint{0.048611in}{0.000000in}}%
\pgfusepath{stroke,fill}%
}%
\begin{pgfscope}%
\pgfsys@transformshift{2.091551in}{0.662710in}%
\pgfsys@useobject{currentmarker}{}%
\end{pgfscope}%
\end{pgfscope}%
\begin{pgfscope}%
\pgftext[x=2.188773in,y=0.614883in,left,base]{\rmfamily\fontsize{10.000000}{12.000000}\selectfont \(\displaystyle 2.5\)}%
\end{pgfscope}%
\begin{pgfscope}%
\pgfsetbuttcap%
\pgfsetroundjoin%
\definecolor{currentfill}{rgb}{0.000000,0.000000,0.000000}%
\pgfsetfillcolor{currentfill}%
\pgfsetlinewidth{0.803000pt}%
\definecolor{currentstroke}{rgb}{0.000000,0.000000,0.000000}%
\pgfsetstrokecolor{currentstroke}%
\pgfsetdash{}{0pt}%
\pgfsys@defobject{currentmarker}{\pgfqpoint{0.000000in}{0.000000in}}{\pgfqpoint{0.048611in}{0.000000in}}{%
\pgfpathmoveto{\pgfqpoint{0.000000in}{0.000000in}}%
\pgfpathlineto{\pgfqpoint{0.048611in}{0.000000in}}%
\pgfusepath{stroke,fill}%
}%
\begin{pgfscope}%
\pgfsys@transformshift{2.091551in}{1.005543in}%
\pgfsys@useobject{currentmarker}{}%
\end{pgfscope}%
\end{pgfscope}%
\begin{pgfscope}%
\pgftext[x=2.188773in,y=0.957716in,left,base]{\rmfamily\fontsize{10.000000}{12.000000}\selectfont \(\displaystyle 5.0\)}%
\end{pgfscope}%
\begin{pgfscope}%
\pgfsetbuttcap%
\pgfsetroundjoin%
\definecolor{currentfill}{rgb}{0.000000,0.000000,0.000000}%
\pgfsetfillcolor{currentfill}%
\pgfsetlinewidth{0.803000pt}%
\definecolor{currentstroke}{rgb}{0.000000,0.000000,0.000000}%
\pgfsetstrokecolor{currentstroke}%
\pgfsetdash{}{0pt}%
\pgfsys@defobject{currentmarker}{\pgfqpoint{0.000000in}{0.000000in}}{\pgfqpoint{0.048611in}{0.000000in}}{%
\pgfpathmoveto{\pgfqpoint{0.000000in}{0.000000in}}%
\pgfpathlineto{\pgfqpoint{0.048611in}{0.000000in}}%
\pgfusepath{stroke,fill}%
}%
\begin{pgfscope}%
\pgfsys@transformshift{2.091551in}{1.348376in}%
\pgfsys@useobject{currentmarker}{}%
\end{pgfscope}%
\end{pgfscope}%
\begin{pgfscope}%
\pgftext[x=2.188773in,y=1.300548in,left,base]{\rmfamily\fontsize{10.000000}{12.000000}\selectfont \(\displaystyle 7.5\)}%
\end{pgfscope}%
\begin{pgfscope}%
\pgfsetbuttcap%
\pgfsetroundjoin%
\definecolor{currentfill}{rgb}{0.000000,0.000000,0.000000}%
\pgfsetfillcolor{currentfill}%
\pgfsetlinewidth{0.803000pt}%
\definecolor{currentstroke}{rgb}{0.000000,0.000000,0.000000}%
\pgfsetstrokecolor{currentstroke}%
\pgfsetdash{}{0pt}%
\pgfsys@defobject{currentmarker}{\pgfqpoint{0.000000in}{0.000000in}}{\pgfqpoint{0.048611in}{0.000000in}}{%
\pgfpathmoveto{\pgfqpoint{0.000000in}{0.000000in}}%
\pgfpathlineto{\pgfqpoint{0.048611in}{0.000000in}}%
\pgfusepath{stroke,fill}%
}%
\begin{pgfscope}%
\pgfsys@transformshift{2.091551in}{1.691209in}%
\pgfsys@useobject{currentmarker}{}%
\end{pgfscope}%
\end{pgfscope}%
\begin{pgfscope}%
\pgftext[x=2.188773in,y=1.643381in,left,base]{\rmfamily\fontsize{10.000000}{12.000000}\selectfont \(\displaystyle 10.0\)}%
\end{pgfscope}%
\begin{pgfscope}%
\pgfsetbuttcap%
\pgfsetroundjoin%
\definecolor{currentfill}{rgb}{0.000000,0.000000,0.000000}%
\pgfsetfillcolor{currentfill}%
\pgfsetlinewidth{0.803000pt}%
\definecolor{currentstroke}{rgb}{0.000000,0.000000,0.000000}%
\pgfsetstrokecolor{currentstroke}%
\pgfsetdash{}{0pt}%
\pgfsys@defobject{currentmarker}{\pgfqpoint{0.000000in}{0.000000in}}{\pgfqpoint{0.048611in}{0.000000in}}{%
\pgfpathmoveto{\pgfqpoint{0.000000in}{0.000000in}}%
\pgfpathlineto{\pgfqpoint{0.048611in}{0.000000in}}%
\pgfusepath{stroke,fill}%
}%
\begin{pgfscope}%
\pgfsys@transformshift{2.091551in}{2.034042in}%
\pgfsys@useobject{currentmarker}{}%
\end{pgfscope}%
\end{pgfscope}%
\begin{pgfscope}%
\pgftext[x=2.188773in,y=1.986214in,left,base]{\rmfamily\fontsize{10.000000}{12.000000}\selectfont \(\displaystyle 12.5\)}%
\end{pgfscope}%
\begin{pgfscope}%
\pgfsetbuttcap%
\pgfsetroundjoin%
\definecolor{currentfill}{rgb}{0.000000,0.000000,0.000000}%
\pgfsetfillcolor{currentfill}%
\pgfsetlinewidth{0.803000pt}%
\definecolor{currentstroke}{rgb}{0.000000,0.000000,0.000000}%
\pgfsetstrokecolor{currentstroke}%
\pgfsetdash{}{0pt}%
\pgfsys@defobject{currentmarker}{\pgfqpoint{0.000000in}{0.000000in}}{\pgfqpoint{0.048611in}{0.000000in}}{%
\pgfpathmoveto{\pgfqpoint{0.000000in}{0.000000in}}%
\pgfpathlineto{\pgfqpoint{0.048611in}{0.000000in}}%
\pgfusepath{stroke,fill}%
}%
\begin{pgfscope}%
\pgfsys@transformshift{2.091551in}{2.376875in}%
\pgfsys@useobject{currentmarker}{}%
\end{pgfscope}%
\end{pgfscope}%
\begin{pgfscope}%
\pgftext[x=2.188773in,y=2.329047in,left,base]{\rmfamily\fontsize{10.000000}{12.000000}\selectfont \(\displaystyle 15.0\)}%
\end{pgfscope}%
\begin{pgfscope}%
\pgfsetbuttcap%
\pgfsetroundjoin%
\definecolor{currentfill}{rgb}{0.000000,0.000000,0.000000}%
\pgfsetfillcolor{currentfill}%
\pgfsetlinewidth{0.803000pt}%
\definecolor{currentstroke}{rgb}{0.000000,0.000000,0.000000}%
\pgfsetstrokecolor{currentstroke}%
\pgfsetdash{}{0pt}%
\pgfsys@defobject{currentmarker}{\pgfqpoint{0.000000in}{0.000000in}}{\pgfqpoint{0.048611in}{0.000000in}}{%
\pgfpathmoveto{\pgfqpoint{0.000000in}{0.000000in}}%
\pgfpathlineto{\pgfqpoint{0.048611in}{0.000000in}}%
\pgfusepath{stroke,fill}%
}%
\begin{pgfscope}%
\pgfsys@transformshift{2.091551in}{2.719708in}%
\pgfsys@useobject{currentmarker}{}%
\end{pgfscope}%
\end{pgfscope}%
\begin{pgfscope}%
\pgftext[x=2.188773in,y=2.671880in,left,base]{\rmfamily\fontsize{10.000000}{12.000000}\selectfont \(\displaystyle 17.5\)}%
\end{pgfscope}%
\begin{pgfscope}%
\pgfsetbuttcap%
\pgfsetmiterjoin%
\pgfsetlinewidth{0.803000pt}%
\definecolor{currentstroke}{rgb}{0.000000,0.000000,0.000000}%
\pgfsetstrokecolor{currentstroke}%
\pgfsetdash{}{0pt}%
\pgfpathmoveto{\pgfqpoint{1.961274in}{0.319877in}}%
\pgfpathlineto{\pgfqpoint{1.961274in}{0.330055in}}%
\pgfpathlineto{\pgfqpoint{1.961274in}{2.915230in}}%
\pgfpathlineto{\pgfqpoint{1.961274in}{2.925408in}}%
\pgfpathlineto{\pgfqpoint{2.091551in}{2.925408in}}%
\pgfpathlineto{\pgfqpoint{2.091551in}{2.915230in}}%
\pgfpathlineto{\pgfqpoint{2.091551in}{0.330055in}}%
\pgfpathlineto{\pgfqpoint{2.091551in}{0.319877in}}%
\pgfpathclose%
\pgfusepath{stroke}%
\end{pgfscope}%
\end{pgfpicture}%
\makeatother%
\endgroup%

	\vspace*{-0.4cm}
	\caption{100 K. Bin size $0.0105e$}
	\end{subfigure}
	\hspace{0.6cm}
	\begin{subfigure}[b]{0.45\textwidth}
	\hspace*{-0.4cm}
	%% Creator: Matplotlib, PGF backend
%%
%% To include the figure in your LaTeX document, write
%%   \input{<filename>.pgf}
%%
%% Make sure the required packages are loaded in your preamble
%%   \usepackage{pgf}
%%
%% Figures using additional raster images can only be included by \input if
%% they are in the same directory as the main LaTeX file. For loading figures
%% from other directories you can use the `import` package
%%   \usepackage{import}
%% and then include the figures with
%%   \import{<path to file>}{<filename>.pgf}
%%
%% Matplotlib used the following preamble
%%   \usepackage[utf8x]{inputenc}
%%   \usepackage[T1]{fontenc}
%%
\begingroup%
\makeatletter%
\begin{pgfpicture}%
\pgfpathrectangle{\pgfpointorigin}{\pgfqpoint{2.535687in}{3.060408in}}%
\pgfusepath{use as bounding box, clip}%
\begin{pgfscope}%
\pgfsetbuttcap%
\pgfsetmiterjoin%
\definecolor{currentfill}{rgb}{1.000000,1.000000,1.000000}%
\pgfsetfillcolor{currentfill}%
\pgfsetlinewidth{0.000000pt}%
\definecolor{currentstroke}{rgb}{1.000000,1.000000,1.000000}%
\pgfsetstrokecolor{currentstroke}%
\pgfsetdash{}{0pt}%
\pgfpathmoveto{\pgfqpoint{0.000000in}{0.000000in}}%
\pgfpathlineto{\pgfqpoint{2.535687in}{0.000000in}}%
\pgfpathlineto{\pgfqpoint{2.535687in}{3.060408in}}%
\pgfpathlineto{\pgfqpoint{0.000000in}{3.060408in}}%
\pgfpathclose%
\pgfusepath{fill}%
\end{pgfscope}%
\begin{pgfscope}%
\pgfsetbuttcap%
\pgfsetmiterjoin%
\definecolor{currentfill}{rgb}{1.000000,1.000000,1.000000}%
\pgfsetfillcolor{currentfill}%
\pgfsetlinewidth{0.000000pt}%
\definecolor{currentstroke}{rgb}{0.000000,0.000000,0.000000}%
\pgfsetstrokecolor{currentstroke}%
\pgfsetstrokeopacity{0.000000}%
\pgfsetdash{}{0pt}%
\pgfpathmoveto{\pgfqpoint{0.374692in}{0.319877in}}%
\pgfpathlineto{\pgfqpoint{1.867946in}{0.319877in}}%
\pgfpathlineto{\pgfqpoint{1.867946in}{2.925408in}}%
\pgfpathlineto{\pgfqpoint{0.374692in}{2.925408in}}%
\pgfpathclose%
\pgfusepath{fill}%
\end{pgfscope}%
\begin{pgfscope}%
\pgfpathrectangle{\pgfqpoint{0.374692in}{0.319877in}}{\pgfqpoint{1.493254in}{2.605531in}} %
\pgfusepath{clip}%
\pgfsys@transformshift{0.374692in}{0.319877in}%
\pgftext[left,bottom]{\pgfimage[interpolate=true,width=1.500000in,height=2.610000in]{RnnStDev_vs_dq_Ti_200K-img0.png}}%
\end{pgfscope}%
\begin{pgfscope}%
\pgfpathrectangle{\pgfqpoint{0.374692in}{0.319877in}}{\pgfqpoint{1.493254in}{2.605531in}} %
\pgfusepath{clip}%
\pgfsetbuttcap%
\pgfsetroundjoin%
\definecolor{currentfill}{rgb}{1.000000,0.752941,0.796078}%
\pgfsetfillcolor{currentfill}%
\pgfsetlinewidth{1.003750pt}%
\definecolor{currentstroke}{rgb}{1.000000,0.752941,0.796078}%
\pgfsetstrokecolor{currentstroke}%
\pgfsetdash{}{0pt}%
\pgfpathmoveto{\pgfqpoint{0.854666in}{1.131484in}}%
\pgfpathcurveto{\pgfqpoint{0.865717in}{1.131484in}}{\pgfqpoint{0.876316in}{1.135874in}}{\pgfqpoint{0.884129in}{1.143688in}}%
\pgfpathcurveto{\pgfqpoint{0.891943in}{1.151501in}}{\pgfqpoint{0.896333in}{1.162100in}}{\pgfqpoint{0.896333in}{1.173151in}}%
\pgfpathcurveto{\pgfqpoint{0.896333in}{1.184201in}}{\pgfqpoint{0.891943in}{1.194800in}}{\pgfqpoint{0.884129in}{1.202613in}}%
\pgfpathcurveto{\pgfqpoint{0.876316in}{1.210427in}}{\pgfqpoint{0.865717in}{1.214817in}}{\pgfqpoint{0.854666in}{1.214817in}}%
\pgfpathcurveto{\pgfqpoint{0.843616in}{1.214817in}}{\pgfqpoint{0.833017in}{1.210427in}}{\pgfqpoint{0.825204in}{1.202613in}}%
\pgfpathcurveto{\pgfqpoint{0.817390in}{1.194800in}}{\pgfqpoint{0.813000in}{1.184201in}}{\pgfqpoint{0.813000in}{1.173151in}}%
\pgfpathcurveto{\pgfqpoint{0.813000in}{1.162100in}}{\pgfqpoint{0.817390in}{1.151501in}}{\pgfqpoint{0.825204in}{1.143688in}}%
\pgfpathcurveto{\pgfqpoint{0.833017in}{1.135874in}}{\pgfqpoint{0.843616in}{1.131484in}}{\pgfqpoint{0.854666in}{1.131484in}}%
\pgfpathclose%
\pgfusepath{stroke,fill}%
\end{pgfscope}%
\begin{pgfscope}%
\pgfpathrectangle{\pgfqpoint{0.374692in}{0.319877in}}{\pgfqpoint{1.493254in}{2.605531in}} %
\pgfusepath{clip}%
\pgfsetbuttcap%
\pgfsetroundjoin%
\definecolor{currentfill}{rgb}{1.000000,0.752941,0.796078}%
\pgfsetfillcolor{currentfill}%
\pgfsetlinewidth{1.003750pt}%
\definecolor{currentstroke}{rgb}{1.000000,0.752941,0.796078}%
\pgfsetstrokecolor{currentstroke}%
\pgfsetdash{}{0pt}%
\pgfpathmoveto{\pgfqpoint{0.961327in}{1.041306in}}%
\pgfpathcurveto{\pgfqpoint{0.972377in}{1.041306in}}{\pgfqpoint{0.982977in}{1.045696in}}{\pgfqpoint{0.990790in}{1.053509in}}%
\pgfpathcurveto{\pgfqpoint{0.998604in}{1.061323in}}{\pgfqpoint{1.002994in}{1.071922in}}{\pgfqpoint{1.002994in}{1.082972in}}%
\pgfpathcurveto{\pgfqpoint{1.002994in}{1.094022in}}{\pgfqpoint{0.998604in}{1.104621in}}{\pgfqpoint{0.990790in}{1.112435in}}%
\pgfpathcurveto{\pgfqpoint{0.982977in}{1.120249in}}{\pgfqpoint{0.972377in}{1.124639in}}{\pgfqpoint{0.961327in}{1.124639in}}%
\pgfpathcurveto{\pgfqpoint{0.950277in}{1.124639in}}{\pgfqpoint{0.939678in}{1.120249in}}{\pgfqpoint{0.931865in}{1.112435in}}%
\pgfpathcurveto{\pgfqpoint{0.924051in}{1.104621in}}{\pgfqpoint{0.919661in}{1.094022in}}{\pgfqpoint{0.919661in}{1.082972in}}%
\pgfpathcurveto{\pgfqpoint{0.919661in}{1.071922in}}{\pgfqpoint{0.924051in}{1.061323in}}{\pgfqpoint{0.931865in}{1.053509in}}%
\pgfpathcurveto{\pgfqpoint{0.939678in}{1.045696in}}{\pgfqpoint{0.950277in}{1.041306in}}{\pgfqpoint{0.961327in}{1.041306in}}%
\pgfpathclose%
\pgfusepath{stroke,fill}%
\end{pgfscope}%
\begin{pgfscope}%
\pgfpathrectangle{\pgfqpoint{0.374692in}{0.319877in}}{\pgfqpoint{1.493254in}{2.605531in}} %
\pgfusepath{clip}%
\pgfsetbuttcap%
\pgfsetroundjoin%
\definecolor{currentfill}{rgb}{1.000000,0.752941,0.796078}%
\pgfsetfillcolor{currentfill}%
\pgfsetlinewidth{1.003750pt}%
\definecolor{currentstroke}{rgb}{1.000000,0.752941,0.796078}%
\pgfsetstrokecolor{currentstroke}%
\pgfsetdash{}{0pt}%
\pgfpathmoveto{\pgfqpoint{1.067988in}{1.056875in}}%
\pgfpathcurveto{\pgfqpoint{1.079038in}{1.056875in}}{\pgfqpoint{1.089638in}{1.061265in}}{\pgfqpoint{1.097451in}{1.069079in}}%
\pgfpathcurveto{\pgfqpoint{1.105265in}{1.076892in}}{\pgfqpoint{1.109655in}{1.087492in}}{\pgfqpoint{1.109655in}{1.098542in}}%
\pgfpathcurveto{\pgfqpoint{1.109655in}{1.109592in}}{\pgfqpoint{1.105265in}{1.120191in}}{\pgfqpoint{1.097451in}{1.128004in}}%
\pgfpathcurveto{\pgfqpoint{1.089638in}{1.135818in}}{\pgfqpoint{1.079038in}{1.140208in}}{\pgfqpoint{1.067988in}{1.140208in}}%
\pgfpathcurveto{\pgfqpoint{1.056938in}{1.140208in}}{\pgfqpoint{1.046339in}{1.135818in}}{\pgfqpoint{1.038526in}{1.128004in}}%
\pgfpathcurveto{\pgfqpoint{1.030712in}{1.120191in}}{\pgfqpoint{1.026322in}{1.109592in}}{\pgfqpoint{1.026322in}{1.098542in}}%
\pgfpathcurveto{\pgfqpoint{1.026322in}{1.087492in}}{\pgfqpoint{1.030712in}{1.076892in}}{\pgfqpoint{1.038526in}{1.069079in}}%
\pgfpathcurveto{\pgfqpoint{1.046339in}{1.061265in}}{\pgfqpoint{1.056938in}{1.056875in}}{\pgfqpoint{1.067988in}{1.056875in}}%
\pgfpathclose%
\pgfusepath{stroke,fill}%
\end{pgfscope}%
\begin{pgfscope}%
\pgfpathrectangle{\pgfqpoint{0.374692in}{0.319877in}}{\pgfqpoint{1.493254in}{2.605531in}} %
\pgfusepath{clip}%
\pgfsetbuttcap%
\pgfsetroundjoin%
\definecolor{currentfill}{rgb}{1.000000,0.752941,0.796078}%
\pgfsetfillcolor{currentfill}%
\pgfsetlinewidth{1.003750pt}%
\definecolor{currentstroke}{rgb}{1.000000,0.752941,0.796078}%
\pgfsetstrokecolor{currentstroke}%
\pgfsetdash{}{0pt}%
\pgfpathmoveto{\pgfqpoint{1.174649in}{1.122061in}}%
\pgfpathcurveto{\pgfqpoint{1.185699in}{1.122061in}}{\pgfqpoint{1.196298in}{1.126451in}}{\pgfqpoint{1.204112in}{1.134265in}}%
\pgfpathcurveto{\pgfqpoint{1.211926in}{1.142078in}}{\pgfqpoint{1.216316in}{1.152677in}}{\pgfqpoint{1.216316in}{1.163728in}}%
\pgfpathcurveto{\pgfqpoint{1.216316in}{1.174778in}}{\pgfqpoint{1.211926in}{1.185377in}}{\pgfqpoint{1.204112in}{1.193190in}}%
\pgfpathcurveto{\pgfqpoint{1.196298in}{1.201004in}}{\pgfqpoint{1.185699in}{1.205394in}}{\pgfqpoint{1.174649in}{1.205394in}}%
\pgfpathcurveto{\pgfqpoint{1.163599in}{1.205394in}}{\pgfqpoint{1.153000in}{1.201004in}}{\pgfqpoint{1.145187in}{1.193190in}}%
\pgfpathcurveto{\pgfqpoint{1.137373in}{1.185377in}}{\pgfqpoint{1.132983in}{1.174778in}}{\pgfqpoint{1.132983in}{1.163728in}}%
\pgfpathcurveto{\pgfqpoint{1.132983in}{1.152677in}}{\pgfqpoint{1.137373in}{1.142078in}}{\pgfqpoint{1.145187in}{1.134265in}}%
\pgfpathcurveto{\pgfqpoint{1.153000in}{1.126451in}}{\pgfqpoint{1.163599in}{1.122061in}}{\pgfqpoint{1.174649in}{1.122061in}}%
\pgfpathclose%
\pgfusepath{stroke,fill}%
\end{pgfscope}%
\begin{pgfscope}%
\pgfpathrectangle{\pgfqpoint{0.374692in}{0.319877in}}{\pgfqpoint{1.493254in}{2.605531in}} %
\pgfusepath{clip}%
\pgfsetbuttcap%
\pgfsetroundjoin%
\definecolor{currentfill}{rgb}{1.000000,0.752941,0.796078}%
\pgfsetfillcolor{currentfill}%
\pgfsetlinewidth{1.003750pt}%
\definecolor{currentstroke}{rgb}{1.000000,0.752941,0.796078}%
\pgfsetstrokecolor{currentstroke}%
\pgfsetdash{}{0pt}%
\pgfpathmoveto{\pgfqpoint{1.281310in}{1.149838in}}%
\pgfpathcurveto{\pgfqpoint{1.292360in}{1.149838in}}{\pgfqpoint{1.302959in}{1.154228in}}{\pgfqpoint{1.310773in}{1.162041in}}%
\pgfpathcurveto{\pgfqpoint{1.318587in}{1.169855in}}{\pgfqpoint{1.322977in}{1.180454in}}{\pgfqpoint{1.322977in}{1.191504in}}%
\pgfpathcurveto{\pgfqpoint{1.322977in}{1.202554in}}{\pgfqpoint{1.318587in}{1.213153in}}{\pgfqpoint{1.310773in}{1.220967in}}%
\pgfpathcurveto{\pgfqpoint{1.302959in}{1.228781in}}{\pgfqpoint{1.292360in}{1.233171in}}{\pgfqpoint{1.281310in}{1.233171in}}%
\pgfpathcurveto{\pgfqpoint{1.270260in}{1.233171in}}{\pgfqpoint{1.259661in}{1.228781in}}{\pgfqpoint{1.251848in}{1.220967in}}%
\pgfpathcurveto{\pgfqpoint{1.244034in}{1.213153in}}{\pgfqpoint{1.239644in}{1.202554in}}{\pgfqpoint{1.239644in}{1.191504in}}%
\pgfpathcurveto{\pgfqpoint{1.239644in}{1.180454in}}{\pgfqpoint{1.244034in}{1.169855in}}{\pgfqpoint{1.251848in}{1.162041in}}%
\pgfpathcurveto{\pgfqpoint{1.259661in}{1.154228in}}{\pgfqpoint{1.270260in}{1.149838in}}{\pgfqpoint{1.281310in}{1.149838in}}%
\pgfpathclose%
\pgfusepath{stroke,fill}%
\end{pgfscope}%
\begin{pgfscope}%
\pgfpathrectangle{\pgfqpoint{0.374692in}{0.319877in}}{\pgfqpoint{1.493254in}{2.605531in}} %
\pgfusepath{clip}%
\pgfsetbuttcap%
\pgfsetroundjoin%
\definecolor{currentfill}{rgb}{1.000000,0.752941,0.796078}%
\pgfsetfillcolor{currentfill}%
\pgfsetlinewidth{1.003750pt}%
\definecolor{currentstroke}{rgb}{1.000000,0.752941,0.796078}%
\pgfsetstrokecolor{currentstroke}%
\pgfsetdash{}{0pt}%
\pgfpathmoveto{\pgfqpoint{1.387971in}{1.238143in}}%
\pgfpathcurveto{\pgfqpoint{1.399021in}{1.238143in}}{\pgfqpoint{1.409620in}{1.242533in}}{\pgfqpoint{1.417434in}{1.250347in}}%
\pgfpathcurveto{\pgfqpoint{1.425248in}{1.258160in}}{\pgfqpoint{1.429638in}{1.268760in}}{\pgfqpoint{1.429638in}{1.279810in}}%
\pgfpathcurveto{\pgfqpoint{1.429638in}{1.290860in}}{\pgfqpoint{1.425248in}{1.301459in}}{\pgfqpoint{1.417434in}{1.309272in}}%
\pgfpathcurveto{\pgfqpoint{1.409620in}{1.317086in}}{\pgfqpoint{1.399021in}{1.321476in}}{\pgfqpoint{1.387971in}{1.321476in}}%
\pgfpathcurveto{\pgfqpoint{1.376921in}{1.321476in}}{\pgfqpoint{1.366322in}{1.317086in}}{\pgfqpoint{1.358509in}{1.309272in}}%
\pgfpathcurveto{\pgfqpoint{1.350695in}{1.301459in}}{\pgfqpoint{1.346305in}{1.290860in}}{\pgfqpoint{1.346305in}{1.279810in}}%
\pgfpathcurveto{\pgfqpoint{1.346305in}{1.268760in}}{\pgfqpoint{1.350695in}{1.258160in}}{\pgfqpoint{1.358509in}{1.250347in}}%
\pgfpathcurveto{\pgfqpoint{1.366322in}{1.242533in}}{\pgfqpoint{1.376921in}{1.238143in}}{\pgfqpoint{1.387971in}{1.238143in}}%
\pgfpathclose%
\pgfusepath{stroke,fill}%
\end{pgfscope}%
\begin{pgfscope}%
\pgfpathrectangle{\pgfqpoint{0.374692in}{0.319877in}}{\pgfqpoint{1.493254in}{2.605531in}} %
\pgfusepath{clip}%
\pgfsetbuttcap%
\pgfsetroundjoin%
\definecolor{currentfill}{rgb}{1.000000,0.752941,0.796078}%
\pgfsetfillcolor{currentfill}%
\pgfsetlinewidth{1.003750pt}%
\definecolor{currentstroke}{rgb}{1.000000,0.752941,0.796078}%
\pgfsetstrokecolor{currentstroke}%
\pgfsetdash{}{0pt}%
\pgfpathmoveto{\pgfqpoint{1.494632in}{1.238143in}}%
\pgfpathcurveto{\pgfqpoint{1.505682in}{1.238143in}}{\pgfqpoint{1.516281in}{1.242533in}}{\pgfqpoint{1.524095in}{1.250347in}}%
\pgfpathcurveto{\pgfqpoint{1.531909in}{1.258160in}}{\pgfqpoint{1.536299in}{1.268760in}}{\pgfqpoint{1.536299in}{1.279810in}}%
\pgfpathcurveto{\pgfqpoint{1.536299in}{1.290860in}}{\pgfqpoint{1.531909in}{1.301459in}}{\pgfqpoint{1.524095in}{1.309272in}}%
\pgfpathcurveto{\pgfqpoint{1.516281in}{1.317086in}}{\pgfqpoint{1.505682in}{1.321476in}}{\pgfqpoint{1.494632in}{1.321476in}}%
\pgfpathcurveto{\pgfqpoint{1.483582in}{1.321476in}}{\pgfqpoint{1.472983in}{1.317086in}}{\pgfqpoint{1.465170in}{1.309272in}}%
\pgfpathcurveto{\pgfqpoint{1.457356in}{1.301459in}}{\pgfqpoint{1.452966in}{1.290860in}}{\pgfqpoint{1.452966in}{1.279810in}}%
\pgfpathcurveto{\pgfqpoint{1.452966in}{1.268760in}}{\pgfqpoint{1.457356in}{1.258160in}}{\pgfqpoint{1.465170in}{1.250347in}}%
\pgfpathcurveto{\pgfqpoint{1.472983in}{1.242533in}}{\pgfqpoint{1.483582in}{1.238143in}}{\pgfqpoint{1.494632in}{1.238143in}}%
\pgfpathclose%
\pgfusepath{stroke,fill}%
\end{pgfscope}%
\begin{pgfscope}%
\pgfsetbuttcap%
\pgfsetroundjoin%
\definecolor{currentfill}{rgb}{0.000000,0.000000,0.000000}%
\pgfsetfillcolor{currentfill}%
\pgfsetlinewidth{0.803000pt}%
\definecolor{currentstroke}{rgb}{0.000000,0.000000,0.000000}%
\pgfsetstrokecolor{currentstroke}%
\pgfsetdash{}{0pt}%
\pgfsys@defobject{currentmarker}{\pgfqpoint{0.000000in}{-0.048611in}}{\pgfqpoint{0.000000in}{0.000000in}}{%
\pgfpathmoveto{\pgfqpoint{0.000000in}{0.000000in}}%
\pgfpathlineto{\pgfqpoint{0.000000in}{-0.048611in}}%
\pgfusepath{stroke,fill}%
}%
\begin{pgfscope}%
\pgfsys@transformshift{0.654677in}{0.319877in}%
\pgfsys@useobject{currentmarker}{}%
\end{pgfscope}%
\end{pgfscope}%
\begin{pgfscope}%
\pgftext[x=0.654677in,y=0.222655in,,top]{\rmfamily\fontsize{10.000000}{12.000000}\selectfont \(\displaystyle -0.05\)}%
\end{pgfscope}%
\begin{pgfscope}%
\pgfsetbuttcap%
\pgfsetroundjoin%
\definecolor{currentfill}{rgb}{0.000000,0.000000,0.000000}%
\pgfsetfillcolor{currentfill}%
\pgfsetlinewidth{0.803000pt}%
\definecolor{currentstroke}{rgb}{0.000000,0.000000,0.000000}%
\pgfsetstrokecolor{currentstroke}%
\pgfsetdash{}{0pt}%
\pgfsys@defobject{currentmarker}{\pgfqpoint{0.000000in}{-0.048611in}}{\pgfqpoint{0.000000in}{0.000000in}}{%
\pgfpathmoveto{\pgfqpoint{0.000000in}{0.000000in}}%
\pgfpathlineto{\pgfqpoint{0.000000in}{-0.048611in}}%
\pgfusepath{stroke,fill}%
}%
\begin{pgfscope}%
\pgfsys@transformshift{1.121319in}{0.319877in}%
\pgfsys@useobject{currentmarker}{}%
\end{pgfscope}%
\end{pgfscope}%
\begin{pgfscope}%
\pgftext[x=1.121319in,y=0.222655in,,top]{\rmfamily\fontsize{10.000000}{12.000000}\selectfont \(\displaystyle 0.00\)}%
\end{pgfscope}%
\begin{pgfscope}%
\pgfsetbuttcap%
\pgfsetroundjoin%
\definecolor{currentfill}{rgb}{0.000000,0.000000,0.000000}%
\pgfsetfillcolor{currentfill}%
\pgfsetlinewidth{0.803000pt}%
\definecolor{currentstroke}{rgb}{0.000000,0.000000,0.000000}%
\pgfsetstrokecolor{currentstroke}%
\pgfsetdash{}{0pt}%
\pgfsys@defobject{currentmarker}{\pgfqpoint{0.000000in}{-0.048611in}}{\pgfqpoint{0.000000in}{0.000000in}}{%
\pgfpathmoveto{\pgfqpoint{0.000000in}{0.000000in}}%
\pgfpathlineto{\pgfqpoint{0.000000in}{-0.048611in}}%
\pgfusepath{stroke,fill}%
}%
\begin{pgfscope}%
\pgfsys@transformshift{1.587961in}{0.319877in}%
\pgfsys@useobject{currentmarker}{}%
\end{pgfscope}%
\end{pgfscope}%
\begin{pgfscope}%
\pgftext[x=1.587961in,y=0.222655in,,top]{\rmfamily\fontsize{10.000000}{12.000000}\selectfont \(\displaystyle 0.05\)}%
\end{pgfscope}%
\begin{pgfscope}%
\pgfsetbuttcap%
\pgfsetroundjoin%
\definecolor{currentfill}{rgb}{0.000000,0.000000,0.000000}%
\pgfsetfillcolor{currentfill}%
\pgfsetlinewidth{0.803000pt}%
\definecolor{currentstroke}{rgb}{0.000000,0.000000,0.000000}%
\pgfsetstrokecolor{currentstroke}%
\pgfsetdash{}{0pt}%
\pgfsys@defobject{currentmarker}{\pgfqpoint{-0.048611in}{0.000000in}}{\pgfqpoint{0.000000in}{0.000000in}}{%
\pgfpathmoveto{\pgfqpoint{0.000000in}{0.000000in}}%
\pgfpathlineto{\pgfqpoint{-0.048611in}{0.000000in}}%
\pgfusepath{stroke,fill}%
}%
\begin{pgfscope}%
\pgfsys@transformshift{0.374692in}{0.436809in}%
\pgfsys@useobject{currentmarker}{}%
\end{pgfscope}%
\end{pgfscope}%
\begin{pgfscope}%
\pgftext[x=0.100000in,y=0.388981in,left,base]{\rmfamily\fontsize{10.000000}{12.000000}\selectfont \(\displaystyle 0.0\)}%
\end{pgfscope}%
\begin{pgfscope}%
\pgfsetbuttcap%
\pgfsetroundjoin%
\definecolor{currentfill}{rgb}{0.000000,0.000000,0.000000}%
\pgfsetfillcolor{currentfill}%
\pgfsetlinewidth{0.803000pt}%
\definecolor{currentstroke}{rgb}{0.000000,0.000000,0.000000}%
\pgfsetstrokecolor{currentstroke}%
\pgfsetdash{}{0pt}%
\pgfsys@defobject{currentmarker}{\pgfqpoint{-0.048611in}{0.000000in}}{\pgfqpoint{0.000000in}{0.000000in}}{%
\pgfpathmoveto{\pgfqpoint{0.000000in}{0.000000in}}%
\pgfpathlineto{\pgfqpoint{-0.048611in}{0.000000in}}%
\pgfusepath{stroke,fill}%
}%
\begin{pgfscope}%
\pgfsys@transformshift{0.374692in}{0.740854in}%
\pgfsys@useobject{currentmarker}{}%
\end{pgfscope}%
\end{pgfscope}%
\begin{pgfscope}%
\pgftext[x=0.100000in,y=0.693026in,left,base]{\rmfamily\fontsize{10.000000}{12.000000}\selectfont \(\displaystyle 0.1\)}%
\end{pgfscope}%
\begin{pgfscope}%
\pgfsetbuttcap%
\pgfsetroundjoin%
\definecolor{currentfill}{rgb}{0.000000,0.000000,0.000000}%
\pgfsetfillcolor{currentfill}%
\pgfsetlinewidth{0.803000pt}%
\definecolor{currentstroke}{rgb}{0.000000,0.000000,0.000000}%
\pgfsetstrokecolor{currentstroke}%
\pgfsetdash{}{0pt}%
\pgfsys@defobject{currentmarker}{\pgfqpoint{-0.048611in}{0.000000in}}{\pgfqpoint{0.000000in}{0.000000in}}{%
\pgfpathmoveto{\pgfqpoint{0.000000in}{0.000000in}}%
\pgfpathlineto{\pgfqpoint{-0.048611in}{0.000000in}}%
\pgfusepath{stroke,fill}%
}%
\begin{pgfscope}%
\pgfsys@transformshift{0.374692in}{1.044899in}%
\pgfsys@useobject{currentmarker}{}%
\end{pgfscope}%
\end{pgfscope}%
\begin{pgfscope}%
\pgftext[x=0.100000in,y=0.997071in,left,base]{\rmfamily\fontsize{10.000000}{12.000000}\selectfont \(\displaystyle 0.2\)}%
\end{pgfscope}%
\begin{pgfscope}%
\pgfsetbuttcap%
\pgfsetroundjoin%
\definecolor{currentfill}{rgb}{0.000000,0.000000,0.000000}%
\pgfsetfillcolor{currentfill}%
\pgfsetlinewidth{0.803000pt}%
\definecolor{currentstroke}{rgb}{0.000000,0.000000,0.000000}%
\pgfsetstrokecolor{currentstroke}%
\pgfsetdash{}{0pt}%
\pgfsys@defobject{currentmarker}{\pgfqpoint{-0.048611in}{0.000000in}}{\pgfqpoint{0.000000in}{0.000000in}}{%
\pgfpathmoveto{\pgfqpoint{0.000000in}{0.000000in}}%
\pgfpathlineto{\pgfqpoint{-0.048611in}{0.000000in}}%
\pgfusepath{stroke,fill}%
}%
\begin{pgfscope}%
\pgfsys@transformshift{0.374692in}{1.348944in}%
\pgfsys@useobject{currentmarker}{}%
\end{pgfscope}%
\end{pgfscope}%
\begin{pgfscope}%
\pgftext[x=0.100000in,y=1.301116in,left,base]{\rmfamily\fontsize{10.000000}{12.000000}\selectfont \(\displaystyle 0.3\)}%
\end{pgfscope}%
\begin{pgfscope}%
\pgfsetbuttcap%
\pgfsetroundjoin%
\definecolor{currentfill}{rgb}{0.000000,0.000000,0.000000}%
\pgfsetfillcolor{currentfill}%
\pgfsetlinewidth{0.803000pt}%
\definecolor{currentstroke}{rgb}{0.000000,0.000000,0.000000}%
\pgfsetstrokecolor{currentstroke}%
\pgfsetdash{}{0pt}%
\pgfsys@defobject{currentmarker}{\pgfqpoint{-0.048611in}{0.000000in}}{\pgfqpoint{0.000000in}{0.000000in}}{%
\pgfpathmoveto{\pgfqpoint{0.000000in}{0.000000in}}%
\pgfpathlineto{\pgfqpoint{-0.048611in}{0.000000in}}%
\pgfusepath{stroke,fill}%
}%
\begin{pgfscope}%
\pgfsys@transformshift{0.374692in}{1.652989in}%
\pgfsys@useobject{currentmarker}{}%
\end{pgfscope}%
\end{pgfscope}%
\begin{pgfscope}%
\pgftext[x=0.100000in,y=1.605161in,left,base]{\rmfamily\fontsize{10.000000}{12.000000}\selectfont \(\displaystyle 0.4\)}%
\end{pgfscope}%
\begin{pgfscope}%
\pgfsetbuttcap%
\pgfsetroundjoin%
\definecolor{currentfill}{rgb}{0.000000,0.000000,0.000000}%
\pgfsetfillcolor{currentfill}%
\pgfsetlinewidth{0.803000pt}%
\definecolor{currentstroke}{rgb}{0.000000,0.000000,0.000000}%
\pgfsetstrokecolor{currentstroke}%
\pgfsetdash{}{0pt}%
\pgfsys@defobject{currentmarker}{\pgfqpoint{-0.048611in}{0.000000in}}{\pgfqpoint{0.000000in}{0.000000in}}{%
\pgfpathmoveto{\pgfqpoint{0.000000in}{0.000000in}}%
\pgfpathlineto{\pgfqpoint{-0.048611in}{0.000000in}}%
\pgfusepath{stroke,fill}%
}%
\begin{pgfscope}%
\pgfsys@transformshift{0.374692in}{1.957034in}%
\pgfsys@useobject{currentmarker}{}%
\end{pgfscope}%
\end{pgfscope}%
\begin{pgfscope}%
\pgftext[x=0.100000in,y=1.909207in,left,base]{\rmfamily\fontsize{10.000000}{12.000000}\selectfont \(\displaystyle 0.5\)}%
\end{pgfscope}%
\begin{pgfscope}%
\pgfsetbuttcap%
\pgfsetroundjoin%
\definecolor{currentfill}{rgb}{0.000000,0.000000,0.000000}%
\pgfsetfillcolor{currentfill}%
\pgfsetlinewidth{0.803000pt}%
\definecolor{currentstroke}{rgb}{0.000000,0.000000,0.000000}%
\pgfsetstrokecolor{currentstroke}%
\pgfsetdash{}{0pt}%
\pgfsys@defobject{currentmarker}{\pgfqpoint{-0.048611in}{0.000000in}}{\pgfqpoint{0.000000in}{0.000000in}}{%
\pgfpathmoveto{\pgfqpoint{0.000000in}{0.000000in}}%
\pgfpathlineto{\pgfqpoint{-0.048611in}{0.000000in}}%
\pgfusepath{stroke,fill}%
}%
\begin{pgfscope}%
\pgfsys@transformshift{0.374692in}{2.261079in}%
\pgfsys@useobject{currentmarker}{}%
\end{pgfscope}%
\end{pgfscope}%
\begin{pgfscope}%
\pgftext[x=0.100000in,y=2.213252in,left,base]{\rmfamily\fontsize{10.000000}{12.000000}\selectfont \(\displaystyle 0.6\)}%
\end{pgfscope}%
\begin{pgfscope}%
\pgfsetbuttcap%
\pgfsetroundjoin%
\definecolor{currentfill}{rgb}{0.000000,0.000000,0.000000}%
\pgfsetfillcolor{currentfill}%
\pgfsetlinewidth{0.803000pt}%
\definecolor{currentstroke}{rgb}{0.000000,0.000000,0.000000}%
\pgfsetstrokecolor{currentstroke}%
\pgfsetdash{}{0pt}%
\pgfsys@defobject{currentmarker}{\pgfqpoint{-0.048611in}{0.000000in}}{\pgfqpoint{0.000000in}{0.000000in}}{%
\pgfpathmoveto{\pgfqpoint{0.000000in}{0.000000in}}%
\pgfpathlineto{\pgfqpoint{-0.048611in}{0.000000in}}%
\pgfusepath{stroke,fill}%
}%
\begin{pgfscope}%
\pgfsys@transformshift{0.374692in}{2.565124in}%
\pgfsys@useobject{currentmarker}{}%
\end{pgfscope}%
\end{pgfscope}%
\begin{pgfscope}%
\pgftext[x=0.100000in,y=2.517297in,left,base]{\rmfamily\fontsize{10.000000}{12.000000}\selectfont \(\displaystyle 0.7\)}%
\end{pgfscope}%
\begin{pgfscope}%
\pgfsetbuttcap%
\pgfsetroundjoin%
\definecolor{currentfill}{rgb}{0.000000,0.000000,0.000000}%
\pgfsetfillcolor{currentfill}%
\pgfsetlinewidth{0.803000pt}%
\definecolor{currentstroke}{rgb}{0.000000,0.000000,0.000000}%
\pgfsetstrokecolor{currentstroke}%
\pgfsetdash{}{0pt}%
\pgfsys@defobject{currentmarker}{\pgfqpoint{-0.048611in}{0.000000in}}{\pgfqpoint{0.000000in}{0.000000in}}{%
\pgfpathmoveto{\pgfqpoint{0.000000in}{0.000000in}}%
\pgfpathlineto{\pgfqpoint{-0.048611in}{0.000000in}}%
\pgfusepath{stroke,fill}%
}%
\begin{pgfscope}%
\pgfsys@transformshift{0.374692in}{2.869169in}%
\pgfsys@useobject{currentmarker}{}%
\end{pgfscope}%
\end{pgfscope}%
\begin{pgfscope}%
\pgftext[x=0.100000in,y=2.821342in,left,base]{\rmfamily\fontsize{10.000000}{12.000000}\selectfont \(\displaystyle 0.8\)}%
\end{pgfscope}%
\begin{pgfscope}%
\pgfsetrectcap%
\pgfsetmiterjoin%
\pgfsetlinewidth{0.803000pt}%
\definecolor{currentstroke}{rgb}{0.000000,0.000000,0.000000}%
\pgfsetstrokecolor{currentstroke}%
\pgfsetdash{}{0pt}%
\pgfpathmoveto{\pgfqpoint{0.374692in}{0.319877in}}%
\pgfpathlineto{\pgfqpoint{0.374692in}{2.925408in}}%
\pgfusepath{stroke}%
\end{pgfscope}%
\begin{pgfscope}%
\pgfsetrectcap%
\pgfsetmiterjoin%
\pgfsetlinewidth{0.803000pt}%
\definecolor{currentstroke}{rgb}{0.000000,0.000000,0.000000}%
\pgfsetstrokecolor{currentstroke}%
\pgfsetdash{}{0pt}%
\pgfpathmoveto{\pgfqpoint{1.867946in}{0.319877in}}%
\pgfpathlineto{\pgfqpoint{1.867946in}{2.925408in}}%
\pgfusepath{stroke}%
\end{pgfscope}%
\begin{pgfscope}%
\pgfsetrectcap%
\pgfsetmiterjoin%
\pgfsetlinewidth{0.803000pt}%
\definecolor{currentstroke}{rgb}{0.000000,0.000000,0.000000}%
\pgfsetstrokecolor{currentstroke}%
\pgfsetdash{}{0pt}%
\pgfpathmoveto{\pgfqpoint{0.374692in}{0.319877in}}%
\pgfpathlineto{\pgfqpoint{1.867946in}{0.319877in}}%
\pgfusepath{stroke}%
\end{pgfscope}%
\begin{pgfscope}%
\pgfsetrectcap%
\pgfsetmiterjoin%
\pgfsetlinewidth{0.803000pt}%
\definecolor{currentstroke}{rgb}{0.000000,0.000000,0.000000}%
\pgfsetstrokecolor{currentstroke}%
\pgfsetdash{}{0pt}%
\pgfpathmoveto{\pgfqpoint{0.374692in}{2.925408in}}%
\pgfpathlineto{\pgfqpoint{1.867946in}{2.925408in}}%
\pgfusepath{stroke}%
\end{pgfscope}%
\begin{pgfscope}%
\pgfpathrectangle{\pgfqpoint{1.961274in}{0.319877in}}{\pgfqpoint{0.130277in}{2.605531in}} %
\pgfusepath{clip}%
\pgfsetbuttcap%
\pgfsetmiterjoin%
\definecolor{currentfill}{rgb}{1.000000,1.000000,1.000000}%
\pgfsetfillcolor{currentfill}%
\pgfsetlinewidth{0.010037pt}%
\definecolor{currentstroke}{rgb}{1.000000,1.000000,1.000000}%
\pgfsetstrokecolor{currentstroke}%
\pgfsetdash{}{0pt}%
\pgfpathmoveto{\pgfqpoint{1.961274in}{0.319877in}}%
\pgfpathlineto{\pgfqpoint{1.961274in}{0.330055in}}%
\pgfpathlineto{\pgfqpoint{1.961274in}{2.915230in}}%
\pgfpathlineto{\pgfqpoint{1.961274in}{2.925408in}}%
\pgfpathlineto{\pgfqpoint{2.091551in}{2.925408in}}%
\pgfpathlineto{\pgfqpoint{2.091551in}{2.915230in}}%
\pgfpathlineto{\pgfqpoint{2.091551in}{0.330055in}}%
\pgfpathlineto{\pgfqpoint{2.091551in}{0.319877in}}%
\pgfpathclose%
\pgfusepath{stroke,fill}%
\end{pgfscope}%
\begin{pgfscope}%
\pgfsys@transformshift{1.960000in}{0.320408in}%
\pgftext[left,bottom]{\pgfimage[interpolate=true,width=0.130000in,height=2.610000in]{RnnStDev_vs_dq_Ti_200K-img1.png}}%
\end{pgfscope}%
\begin{pgfscope}%
\pgfsetbuttcap%
\pgfsetroundjoin%
\definecolor{currentfill}{rgb}{0.000000,0.000000,0.000000}%
\pgfsetfillcolor{currentfill}%
\pgfsetlinewidth{0.803000pt}%
\definecolor{currentstroke}{rgb}{0.000000,0.000000,0.000000}%
\pgfsetstrokecolor{currentstroke}%
\pgfsetdash{}{0pt}%
\pgfsys@defobject{currentmarker}{\pgfqpoint{0.000000in}{0.000000in}}{\pgfqpoint{0.048611in}{0.000000in}}{%
\pgfpathmoveto{\pgfqpoint{0.000000in}{0.000000in}}%
\pgfpathlineto{\pgfqpoint{0.048611in}{0.000000in}}%
\pgfusepath{stroke,fill}%
}%
\begin{pgfscope}%
\pgfsys@transformshift{2.091551in}{0.319877in}%
\pgfsys@useobject{currentmarker}{}%
\end{pgfscope}%
\end{pgfscope}%
\begin{pgfscope}%
\pgftext[x=2.188773in,y=0.272050in,left,base]{\rmfamily\fontsize{10.000000}{12.000000}\selectfont \(\displaystyle 0.0\)}%
\end{pgfscope}%
\begin{pgfscope}%
\pgfsetbuttcap%
\pgfsetroundjoin%
\definecolor{currentfill}{rgb}{0.000000,0.000000,0.000000}%
\pgfsetfillcolor{currentfill}%
\pgfsetlinewidth{0.803000pt}%
\definecolor{currentstroke}{rgb}{0.000000,0.000000,0.000000}%
\pgfsetstrokecolor{currentstroke}%
\pgfsetdash{}{0pt}%
\pgfsys@defobject{currentmarker}{\pgfqpoint{0.000000in}{0.000000in}}{\pgfqpoint{0.048611in}{0.000000in}}{%
\pgfpathmoveto{\pgfqpoint{0.000000in}{0.000000in}}%
\pgfpathlineto{\pgfqpoint{0.048611in}{0.000000in}}%
\pgfusepath{stroke,fill}%
}%
\begin{pgfscope}%
\pgfsys@transformshift{2.091551in}{0.662710in}%
\pgfsys@useobject{currentmarker}{}%
\end{pgfscope}%
\end{pgfscope}%
\begin{pgfscope}%
\pgftext[x=2.188773in,y=0.614883in,left,base]{\rmfamily\fontsize{10.000000}{12.000000}\selectfont \(\displaystyle 2.5\)}%
\end{pgfscope}%
\begin{pgfscope}%
\pgfsetbuttcap%
\pgfsetroundjoin%
\definecolor{currentfill}{rgb}{0.000000,0.000000,0.000000}%
\pgfsetfillcolor{currentfill}%
\pgfsetlinewidth{0.803000pt}%
\definecolor{currentstroke}{rgb}{0.000000,0.000000,0.000000}%
\pgfsetstrokecolor{currentstroke}%
\pgfsetdash{}{0pt}%
\pgfsys@defobject{currentmarker}{\pgfqpoint{0.000000in}{0.000000in}}{\pgfqpoint{0.048611in}{0.000000in}}{%
\pgfpathmoveto{\pgfqpoint{0.000000in}{0.000000in}}%
\pgfpathlineto{\pgfqpoint{0.048611in}{0.000000in}}%
\pgfusepath{stroke,fill}%
}%
\begin{pgfscope}%
\pgfsys@transformshift{2.091551in}{1.005543in}%
\pgfsys@useobject{currentmarker}{}%
\end{pgfscope}%
\end{pgfscope}%
\begin{pgfscope}%
\pgftext[x=2.188773in,y=0.957716in,left,base]{\rmfamily\fontsize{10.000000}{12.000000}\selectfont \(\displaystyle 5.0\)}%
\end{pgfscope}%
\begin{pgfscope}%
\pgfsetbuttcap%
\pgfsetroundjoin%
\definecolor{currentfill}{rgb}{0.000000,0.000000,0.000000}%
\pgfsetfillcolor{currentfill}%
\pgfsetlinewidth{0.803000pt}%
\definecolor{currentstroke}{rgb}{0.000000,0.000000,0.000000}%
\pgfsetstrokecolor{currentstroke}%
\pgfsetdash{}{0pt}%
\pgfsys@defobject{currentmarker}{\pgfqpoint{0.000000in}{0.000000in}}{\pgfqpoint{0.048611in}{0.000000in}}{%
\pgfpathmoveto{\pgfqpoint{0.000000in}{0.000000in}}%
\pgfpathlineto{\pgfqpoint{0.048611in}{0.000000in}}%
\pgfusepath{stroke,fill}%
}%
\begin{pgfscope}%
\pgfsys@transformshift{2.091551in}{1.348376in}%
\pgfsys@useobject{currentmarker}{}%
\end{pgfscope}%
\end{pgfscope}%
\begin{pgfscope}%
\pgftext[x=2.188773in,y=1.300548in,left,base]{\rmfamily\fontsize{10.000000}{12.000000}\selectfont \(\displaystyle 7.5\)}%
\end{pgfscope}%
\begin{pgfscope}%
\pgfsetbuttcap%
\pgfsetroundjoin%
\definecolor{currentfill}{rgb}{0.000000,0.000000,0.000000}%
\pgfsetfillcolor{currentfill}%
\pgfsetlinewidth{0.803000pt}%
\definecolor{currentstroke}{rgb}{0.000000,0.000000,0.000000}%
\pgfsetstrokecolor{currentstroke}%
\pgfsetdash{}{0pt}%
\pgfsys@defobject{currentmarker}{\pgfqpoint{0.000000in}{0.000000in}}{\pgfqpoint{0.048611in}{0.000000in}}{%
\pgfpathmoveto{\pgfqpoint{0.000000in}{0.000000in}}%
\pgfpathlineto{\pgfqpoint{0.048611in}{0.000000in}}%
\pgfusepath{stroke,fill}%
}%
\begin{pgfscope}%
\pgfsys@transformshift{2.091551in}{1.691209in}%
\pgfsys@useobject{currentmarker}{}%
\end{pgfscope}%
\end{pgfscope}%
\begin{pgfscope}%
\pgftext[x=2.188773in,y=1.643381in,left,base]{\rmfamily\fontsize{10.000000}{12.000000}\selectfont \(\displaystyle 10.0\)}%
\end{pgfscope}%
\begin{pgfscope}%
\pgfsetbuttcap%
\pgfsetroundjoin%
\definecolor{currentfill}{rgb}{0.000000,0.000000,0.000000}%
\pgfsetfillcolor{currentfill}%
\pgfsetlinewidth{0.803000pt}%
\definecolor{currentstroke}{rgb}{0.000000,0.000000,0.000000}%
\pgfsetstrokecolor{currentstroke}%
\pgfsetdash{}{0pt}%
\pgfsys@defobject{currentmarker}{\pgfqpoint{0.000000in}{0.000000in}}{\pgfqpoint{0.048611in}{0.000000in}}{%
\pgfpathmoveto{\pgfqpoint{0.000000in}{0.000000in}}%
\pgfpathlineto{\pgfqpoint{0.048611in}{0.000000in}}%
\pgfusepath{stroke,fill}%
}%
\begin{pgfscope}%
\pgfsys@transformshift{2.091551in}{2.034042in}%
\pgfsys@useobject{currentmarker}{}%
\end{pgfscope}%
\end{pgfscope}%
\begin{pgfscope}%
\pgftext[x=2.188773in,y=1.986214in,left,base]{\rmfamily\fontsize{10.000000}{12.000000}\selectfont \(\displaystyle 12.5\)}%
\end{pgfscope}%
\begin{pgfscope}%
\pgfsetbuttcap%
\pgfsetroundjoin%
\definecolor{currentfill}{rgb}{0.000000,0.000000,0.000000}%
\pgfsetfillcolor{currentfill}%
\pgfsetlinewidth{0.803000pt}%
\definecolor{currentstroke}{rgb}{0.000000,0.000000,0.000000}%
\pgfsetstrokecolor{currentstroke}%
\pgfsetdash{}{0pt}%
\pgfsys@defobject{currentmarker}{\pgfqpoint{0.000000in}{0.000000in}}{\pgfqpoint{0.048611in}{0.000000in}}{%
\pgfpathmoveto{\pgfqpoint{0.000000in}{0.000000in}}%
\pgfpathlineto{\pgfqpoint{0.048611in}{0.000000in}}%
\pgfusepath{stroke,fill}%
}%
\begin{pgfscope}%
\pgfsys@transformshift{2.091551in}{2.376875in}%
\pgfsys@useobject{currentmarker}{}%
\end{pgfscope}%
\end{pgfscope}%
\begin{pgfscope}%
\pgftext[x=2.188773in,y=2.329047in,left,base]{\rmfamily\fontsize{10.000000}{12.000000}\selectfont \(\displaystyle 15.0\)}%
\end{pgfscope}%
\begin{pgfscope}%
\pgfsetbuttcap%
\pgfsetroundjoin%
\definecolor{currentfill}{rgb}{0.000000,0.000000,0.000000}%
\pgfsetfillcolor{currentfill}%
\pgfsetlinewidth{0.803000pt}%
\definecolor{currentstroke}{rgb}{0.000000,0.000000,0.000000}%
\pgfsetstrokecolor{currentstroke}%
\pgfsetdash{}{0pt}%
\pgfsys@defobject{currentmarker}{\pgfqpoint{0.000000in}{0.000000in}}{\pgfqpoint{0.048611in}{0.000000in}}{%
\pgfpathmoveto{\pgfqpoint{0.000000in}{0.000000in}}%
\pgfpathlineto{\pgfqpoint{0.048611in}{0.000000in}}%
\pgfusepath{stroke,fill}%
}%
\begin{pgfscope}%
\pgfsys@transformshift{2.091551in}{2.719708in}%
\pgfsys@useobject{currentmarker}{}%
\end{pgfscope}%
\end{pgfscope}%
\begin{pgfscope}%
\pgftext[x=2.188773in,y=2.671880in,left,base]{\rmfamily\fontsize{10.000000}{12.000000}\selectfont \(\displaystyle 17.5\)}%
\end{pgfscope}%
\begin{pgfscope}%
\pgfsetbuttcap%
\pgfsetmiterjoin%
\pgfsetlinewidth{0.803000pt}%
\definecolor{currentstroke}{rgb}{0.000000,0.000000,0.000000}%
\pgfsetstrokecolor{currentstroke}%
\pgfsetdash{}{0pt}%
\pgfpathmoveto{\pgfqpoint{1.961274in}{0.319877in}}%
\pgfpathlineto{\pgfqpoint{1.961274in}{0.330055in}}%
\pgfpathlineto{\pgfqpoint{1.961274in}{2.915230in}}%
\pgfpathlineto{\pgfqpoint{1.961274in}{2.925408in}}%
\pgfpathlineto{\pgfqpoint{2.091551in}{2.925408in}}%
\pgfpathlineto{\pgfqpoint{2.091551in}{2.915230in}}%
\pgfpathlineto{\pgfqpoint{2.091551in}{0.330055in}}%
\pgfpathlineto{\pgfqpoint{2.091551in}{0.319877in}}%
\pgfpathclose%
\pgfusepath{stroke}%
\end{pgfscope}%
\end{pgfpicture}%
\makeatother%
\endgroup%

	\vspace*{-0.4cm}
	\caption{200 K. Bin size $0.011e$}
	\end{subfigure}
	\quad
	\begin{subfigure}[b]{0.45\textwidth}
	\hspace*{-0.4cm}
	%% Creator: Matplotlib, PGF backend
%%
%% To include the figure in your LaTeX document, write
%%   \input{<filename>.pgf}
%%
%% Make sure the required packages are loaded in your preamble
%%   \usepackage{pgf}
%%
%% Figures using additional raster images can only be included by \input if
%% they are in the same directory as the main LaTeX file. For loading figures
%% from other directories you can use the `import` package
%%   \usepackage{import}
%% and then include the figures with
%%   \import{<path to file>}{<filename>.pgf}
%%
%% Matplotlib used the following preamble
%%   \usepackage[utf8x]{inputenc}
%%   \usepackage[T1]{fontenc}
%%
\begingroup%
\makeatletter%
\begin{pgfpicture}%
\pgfpathrectangle{\pgfpointorigin}{\pgfqpoint{2.535687in}{3.060408in}}%
\pgfusepath{use as bounding box, clip}%
\begin{pgfscope}%
\pgfsetbuttcap%
\pgfsetmiterjoin%
\definecolor{currentfill}{rgb}{1.000000,1.000000,1.000000}%
\pgfsetfillcolor{currentfill}%
\pgfsetlinewidth{0.000000pt}%
\definecolor{currentstroke}{rgb}{1.000000,1.000000,1.000000}%
\pgfsetstrokecolor{currentstroke}%
\pgfsetdash{}{0pt}%
\pgfpathmoveto{\pgfqpoint{0.000000in}{0.000000in}}%
\pgfpathlineto{\pgfqpoint{2.535687in}{0.000000in}}%
\pgfpathlineto{\pgfqpoint{2.535687in}{3.060408in}}%
\pgfpathlineto{\pgfqpoint{0.000000in}{3.060408in}}%
\pgfpathclose%
\pgfusepath{fill}%
\end{pgfscope}%
\begin{pgfscope}%
\pgfsetbuttcap%
\pgfsetmiterjoin%
\definecolor{currentfill}{rgb}{1.000000,1.000000,1.000000}%
\pgfsetfillcolor{currentfill}%
\pgfsetlinewidth{0.000000pt}%
\definecolor{currentstroke}{rgb}{0.000000,0.000000,0.000000}%
\pgfsetstrokecolor{currentstroke}%
\pgfsetstrokeopacity{0.000000}%
\pgfsetdash{}{0pt}%
\pgfpathmoveto{\pgfqpoint{0.374692in}{0.319877in}}%
\pgfpathlineto{\pgfqpoint{1.867946in}{0.319877in}}%
\pgfpathlineto{\pgfqpoint{1.867946in}{2.925408in}}%
\pgfpathlineto{\pgfqpoint{0.374692in}{2.925408in}}%
\pgfpathclose%
\pgfusepath{fill}%
\end{pgfscope}%
\begin{pgfscope}%
\pgfpathrectangle{\pgfqpoint{0.374692in}{0.319877in}}{\pgfqpoint{1.493254in}{2.605531in}} %
\pgfusepath{clip}%
\pgfsys@transformshift{0.374692in}{0.319877in}%
\pgftext[left,bottom]{\pgfimage[interpolate=true,width=1.500000in,height=2.610000in]{RnnStDev_vs_dq_Ti_300K-img0.png}}%
\end{pgfscope}%
\begin{pgfscope}%
\pgfpathrectangle{\pgfqpoint{0.374692in}{0.319877in}}{\pgfqpoint{1.493254in}{2.605531in}} %
\pgfusepath{clip}%
\pgfsetbuttcap%
\pgfsetroundjoin%
\definecolor{currentfill}{rgb}{1.000000,0.752941,0.796078}%
\pgfsetfillcolor{currentfill}%
\pgfsetlinewidth{1.003750pt}%
\definecolor{currentstroke}{rgb}{1.000000,0.752941,0.796078}%
\pgfsetstrokecolor{currentstroke}%
\pgfsetdash{}{0pt}%
\pgfpathmoveto{\pgfqpoint{0.714068in}{1.283854in}}%
\pgfpathcurveto{\pgfqpoint{0.725118in}{1.283854in}}{\pgfqpoint{0.735717in}{1.288244in}}{\pgfqpoint{0.743531in}{1.296058in}}%
\pgfpathcurveto{\pgfqpoint{0.751344in}{1.303872in}}{\pgfqpoint{0.755734in}{1.314471in}}{\pgfqpoint{0.755734in}{1.325521in}}%
\pgfpathcurveto{\pgfqpoint{0.755734in}{1.336571in}}{\pgfqpoint{0.751344in}{1.347170in}}{\pgfqpoint{0.743531in}{1.354983in}}%
\pgfpathcurveto{\pgfqpoint{0.735717in}{1.362797in}}{\pgfqpoint{0.725118in}{1.367187in}}{\pgfqpoint{0.714068in}{1.367187in}}%
\pgfpathcurveto{\pgfqpoint{0.703018in}{1.367187in}}{\pgfqpoint{0.692419in}{1.362797in}}{\pgfqpoint{0.684605in}{1.354983in}}%
\pgfpathcurveto{\pgfqpoint{0.676791in}{1.347170in}}{\pgfqpoint{0.672401in}{1.336571in}}{\pgfqpoint{0.672401in}{1.325521in}}%
\pgfpathcurveto{\pgfqpoint{0.672401in}{1.314471in}}{\pgfqpoint{0.676791in}{1.303872in}}{\pgfqpoint{0.684605in}{1.296058in}}%
\pgfpathcurveto{\pgfqpoint{0.692419in}{1.288244in}}{\pgfqpoint{0.703018in}{1.283854in}}{\pgfqpoint{0.714068in}{1.283854in}}%
\pgfpathclose%
\pgfusepath{stroke,fill}%
\end{pgfscope}%
\begin{pgfscope}%
\pgfpathrectangle{\pgfqpoint{0.374692in}{0.319877in}}{\pgfqpoint{1.493254in}{2.605531in}} %
\pgfusepath{clip}%
\pgfsetbuttcap%
\pgfsetroundjoin%
\definecolor{currentfill}{rgb}{1.000000,0.752941,0.796078}%
\pgfsetfillcolor{currentfill}%
\pgfsetlinewidth{1.003750pt}%
\definecolor{currentstroke}{rgb}{1.000000,0.752941,0.796078}%
\pgfsetstrokecolor{currentstroke}%
\pgfsetdash{}{0pt}%
\pgfpathmoveto{\pgfqpoint{0.849818in}{0.941021in}}%
\pgfpathcurveto{\pgfqpoint{0.860868in}{0.941021in}}{\pgfqpoint{0.871467in}{0.945411in}}{\pgfqpoint{0.879281in}{0.953225in}}%
\pgfpathcurveto{\pgfqpoint{0.887095in}{0.961039in}}{\pgfqpoint{0.891485in}{0.971638in}}{\pgfqpoint{0.891485in}{0.982688in}}%
\pgfpathcurveto{\pgfqpoint{0.891485in}{0.993738in}}{\pgfqpoint{0.887095in}{1.004337in}}{\pgfqpoint{0.879281in}{1.012151in}}%
\pgfpathcurveto{\pgfqpoint{0.871467in}{1.019964in}}{\pgfqpoint{0.860868in}{1.024354in}}{\pgfqpoint{0.849818in}{1.024354in}}%
\pgfpathcurveto{\pgfqpoint{0.838768in}{1.024354in}}{\pgfqpoint{0.828169in}{1.019964in}}{\pgfqpoint{0.820355in}{1.012151in}}%
\pgfpathcurveto{\pgfqpoint{0.812542in}{1.004337in}}{\pgfqpoint{0.808151in}{0.993738in}}{\pgfqpoint{0.808151in}{0.982688in}}%
\pgfpathcurveto{\pgfqpoint{0.808151in}{0.971638in}}{\pgfqpoint{0.812542in}{0.961039in}}{\pgfqpoint{0.820355in}{0.953225in}}%
\pgfpathcurveto{\pgfqpoint{0.828169in}{0.945411in}}{\pgfqpoint{0.838768in}{0.941021in}}{\pgfqpoint{0.849818in}{0.941021in}}%
\pgfpathclose%
\pgfusepath{stroke,fill}%
\end{pgfscope}%
\begin{pgfscope}%
\pgfpathrectangle{\pgfqpoint{0.374692in}{0.319877in}}{\pgfqpoint{1.493254in}{2.605531in}} %
\pgfusepath{clip}%
\pgfsetbuttcap%
\pgfsetroundjoin%
\definecolor{currentfill}{rgb}{1.000000,0.752941,0.796078}%
\pgfsetfillcolor{currentfill}%
\pgfsetlinewidth{1.003750pt}%
\definecolor{currentstroke}{rgb}{1.000000,0.752941,0.796078}%
\pgfsetstrokecolor{currentstroke}%
\pgfsetdash{}{0pt}%
\pgfpathmoveto{\pgfqpoint{0.985568in}{1.109120in}}%
\pgfpathcurveto{\pgfqpoint{0.996619in}{1.109120in}}{\pgfqpoint{1.007218in}{1.113510in}}{\pgfqpoint{1.015031in}{1.121324in}}%
\pgfpathcurveto{\pgfqpoint{1.022845in}{1.129137in}}{\pgfqpoint{1.027235in}{1.139736in}}{\pgfqpoint{1.027235in}{1.150786in}}%
\pgfpathcurveto{\pgfqpoint{1.027235in}{1.161837in}}{\pgfqpoint{1.022845in}{1.172436in}}{\pgfqpoint{1.015031in}{1.180249in}}%
\pgfpathcurveto{\pgfqpoint{1.007218in}{1.188063in}}{\pgfqpoint{0.996619in}{1.192453in}}{\pgfqpoint{0.985568in}{1.192453in}}%
\pgfpathcurveto{\pgfqpoint{0.974518in}{1.192453in}}{\pgfqpoint{0.963919in}{1.188063in}}{\pgfqpoint{0.956106in}{1.180249in}}%
\pgfpathcurveto{\pgfqpoint{0.948292in}{1.172436in}}{\pgfqpoint{0.943902in}{1.161837in}}{\pgfqpoint{0.943902in}{1.150786in}}%
\pgfpathcurveto{\pgfqpoint{0.943902in}{1.139736in}}{\pgfqpoint{0.948292in}{1.129137in}}{\pgfqpoint{0.956106in}{1.121324in}}%
\pgfpathcurveto{\pgfqpoint{0.963919in}{1.113510in}}{\pgfqpoint{0.974518in}{1.109120in}}{\pgfqpoint{0.985568in}{1.109120in}}%
\pgfpathclose%
\pgfusepath{stroke,fill}%
\end{pgfscope}%
\begin{pgfscope}%
\pgfpathrectangle{\pgfqpoint{0.374692in}{0.319877in}}{\pgfqpoint{1.493254in}{2.605531in}} %
\pgfusepath{clip}%
\pgfsetbuttcap%
\pgfsetroundjoin%
\definecolor{currentfill}{rgb}{1.000000,0.752941,0.796078}%
\pgfsetfillcolor{currentfill}%
\pgfsetlinewidth{1.003750pt}%
\definecolor{currentstroke}{rgb}{1.000000,0.752941,0.796078}%
\pgfsetstrokecolor{currentstroke}%
\pgfsetdash{}{0pt}%
\pgfpathmoveto{\pgfqpoint{1.121319in}{1.130476in}}%
\pgfpathcurveto{\pgfqpoint{1.132369in}{1.130476in}}{\pgfqpoint{1.142968in}{1.134867in}}{\pgfqpoint{1.150782in}{1.142680in}}%
\pgfpathcurveto{\pgfqpoint{1.158595in}{1.150494in}}{\pgfqpoint{1.162986in}{1.161093in}}{\pgfqpoint{1.162986in}{1.172143in}}%
\pgfpathcurveto{\pgfqpoint{1.162986in}{1.183193in}}{\pgfqpoint{1.158595in}{1.193792in}}{\pgfqpoint{1.150782in}{1.201606in}}%
\pgfpathcurveto{\pgfqpoint{1.142968in}{1.209420in}}{\pgfqpoint{1.132369in}{1.213810in}}{\pgfqpoint{1.121319in}{1.213810in}}%
\pgfpathcurveto{\pgfqpoint{1.110269in}{1.213810in}}{\pgfqpoint{1.099670in}{1.209420in}}{\pgfqpoint{1.091856in}{1.201606in}}%
\pgfpathcurveto{\pgfqpoint{1.084042in}{1.193792in}}{\pgfqpoint{1.079652in}{1.183193in}}{\pgfqpoint{1.079652in}{1.172143in}}%
\pgfpathcurveto{\pgfqpoint{1.079652in}{1.161093in}}{\pgfqpoint{1.084042in}{1.150494in}}{\pgfqpoint{1.091856in}{1.142680in}}%
\pgfpathcurveto{\pgfqpoint{1.099670in}{1.134867in}}{\pgfqpoint{1.110269in}{1.130476in}}{\pgfqpoint{1.121319in}{1.130476in}}%
\pgfpathclose%
\pgfusepath{stroke,fill}%
\end{pgfscope}%
\begin{pgfscope}%
\pgfpathrectangle{\pgfqpoint{0.374692in}{0.319877in}}{\pgfqpoint{1.493254in}{2.605531in}} %
\pgfusepath{clip}%
\pgfsetbuttcap%
\pgfsetroundjoin%
\definecolor{currentfill}{rgb}{1.000000,0.752941,0.796078}%
\pgfsetfillcolor{currentfill}%
\pgfsetlinewidth{1.003750pt}%
\definecolor{currentstroke}{rgb}{1.000000,0.752941,0.796078}%
\pgfsetstrokecolor{currentstroke}%
\pgfsetdash{}{0pt}%
\pgfpathmoveto{\pgfqpoint{1.257069in}{1.168053in}}%
\pgfpathcurveto{\pgfqpoint{1.268119in}{1.168053in}}{\pgfqpoint{1.278718in}{1.172443in}}{\pgfqpoint{1.286532in}{1.180257in}}%
\pgfpathcurveto{\pgfqpoint{1.294346in}{1.188070in}}{\pgfqpoint{1.298736in}{1.198669in}}{\pgfqpoint{1.298736in}{1.209719in}}%
\pgfpathcurveto{\pgfqpoint{1.298736in}{1.220769in}}{\pgfqpoint{1.294346in}{1.231369in}}{\pgfqpoint{1.286532in}{1.239182in}}%
\pgfpathcurveto{\pgfqpoint{1.278718in}{1.246996in}}{\pgfqpoint{1.268119in}{1.251386in}}{\pgfqpoint{1.257069in}{1.251386in}}%
\pgfpathcurveto{\pgfqpoint{1.246019in}{1.251386in}}{\pgfqpoint{1.235420in}{1.246996in}}{\pgfqpoint{1.227606in}{1.239182in}}%
\pgfpathcurveto{\pgfqpoint{1.219793in}{1.231369in}}{\pgfqpoint{1.215403in}{1.220769in}}{\pgfqpoint{1.215403in}{1.209719in}}%
\pgfpathcurveto{\pgfqpoint{1.215403in}{1.198669in}}{\pgfqpoint{1.219793in}{1.188070in}}{\pgfqpoint{1.227606in}{1.180257in}}%
\pgfpathcurveto{\pgfqpoint{1.235420in}{1.172443in}}{\pgfqpoint{1.246019in}{1.168053in}}{\pgfqpoint{1.257069in}{1.168053in}}%
\pgfpathclose%
\pgfusepath{stroke,fill}%
\end{pgfscope}%
\begin{pgfscope}%
\pgfpathrectangle{\pgfqpoint{0.374692in}{0.319877in}}{\pgfqpoint{1.493254in}{2.605531in}} %
\pgfusepath{clip}%
\pgfsetbuttcap%
\pgfsetroundjoin%
\definecolor{currentfill}{rgb}{1.000000,0.752941,0.796078}%
\pgfsetfillcolor{currentfill}%
\pgfsetlinewidth{1.003750pt}%
\definecolor{currentstroke}{rgb}{1.000000,0.752941,0.796078}%
\pgfsetstrokecolor{currentstroke}%
\pgfsetdash{}{0pt}%
\pgfpathmoveto{\pgfqpoint{1.392820in}{1.153251in}}%
\pgfpathcurveto{\pgfqpoint{1.403870in}{1.153251in}}{\pgfqpoint{1.414469in}{1.157641in}}{\pgfqpoint{1.422282in}{1.165455in}}%
\pgfpathcurveto{\pgfqpoint{1.430096in}{1.173269in}}{\pgfqpoint{1.434486in}{1.183868in}}{\pgfqpoint{1.434486in}{1.194918in}}%
\pgfpathcurveto{\pgfqpoint{1.434486in}{1.205968in}}{\pgfqpoint{1.430096in}{1.216567in}}{\pgfqpoint{1.422282in}{1.224380in}}%
\pgfpathcurveto{\pgfqpoint{1.414469in}{1.232194in}}{\pgfqpoint{1.403870in}{1.236584in}}{\pgfqpoint{1.392820in}{1.236584in}}%
\pgfpathcurveto{\pgfqpoint{1.381769in}{1.236584in}}{\pgfqpoint{1.371170in}{1.232194in}}{\pgfqpoint{1.363357in}{1.224380in}}%
\pgfpathcurveto{\pgfqpoint{1.355543in}{1.216567in}}{\pgfqpoint{1.351153in}{1.205968in}}{\pgfqpoint{1.351153in}{1.194918in}}%
\pgfpathcurveto{\pgfqpoint{1.351153in}{1.183868in}}{\pgfqpoint{1.355543in}{1.173269in}}{\pgfqpoint{1.363357in}{1.165455in}}%
\pgfpathcurveto{\pgfqpoint{1.371170in}{1.157641in}}{\pgfqpoint{1.381769in}{1.153251in}}{\pgfqpoint{1.392820in}{1.153251in}}%
\pgfpathclose%
\pgfusepath{stroke,fill}%
\end{pgfscope}%
\begin{pgfscope}%
\pgfsetbuttcap%
\pgfsetroundjoin%
\definecolor{currentfill}{rgb}{0.000000,0.000000,0.000000}%
\pgfsetfillcolor{currentfill}%
\pgfsetlinewidth{0.803000pt}%
\definecolor{currentstroke}{rgb}{0.000000,0.000000,0.000000}%
\pgfsetstrokecolor{currentstroke}%
\pgfsetdash{}{0pt}%
\pgfsys@defobject{currentmarker}{\pgfqpoint{0.000000in}{-0.048611in}}{\pgfqpoint{0.000000in}{0.000000in}}{%
\pgfpathmoveto{\pgfqpoint{0.000000in}{0.000000in}}%
\pgfpathlineto{\pgfqpoint{0.000000in}{-0.048611in}}%
\pgfusepath{stroke,fill}%
}%
\begin{pgfscope}%
\pgfsys@transformshift{0.654677in}{0.319877in}%
\pgfsys@useobject{currentmarker}{}%
\end{pgfscope}%
\end{pgfscope}%
\begin{pgfscope}%
\pgftext[x=0.654677in,y=0.222655in,,top]{\rmfamily\fontsize{10.000000}{12.000000}\selectfont \(\displaystyle -0.05\)}%
\end{pgfscope}%
\begin{pgfscope}%
\pgfsetbuttcap%
\pgfsetroundjoin%
\definecolor{currentfill}{rgb}{0.000000,0.000000,0.000000}%
\pgfsetfillcolor{currentfill}%
\pgfsetlinewidth{0.803000pt}%
\definecolor{currentstroke}{rgb}{0.000000,0.000000,0.000000}%
\pgfsetstrokecolor{currentstroke}%
\pgfsetdash{}{0pt}%
\pgfsys@defobject{currentmarker}{\pgfqpoint{0.000000in}{-0.048611in}}{\pgfqpoint{0.000000in}{0.000000in}}{%
\pgfpathmoveto{\pgfqpoint{0.000000in}{0.000000in}}%
\pgfpathlineto{\pgfqpoint{0.000000in}{-0.048611in}}%
\pgfusepath{stroke,fill}%
}%
\begin{pgfscope}%
\pgfsys@transformshift{1.121319in}{0.319877in}%
\pgfsys@useobject{currentmarker}{}%
\end{pgfscope}%
\end{pgfscope}%
\begin{pgfscope}%
\pgftext[x=1.121319in,y=0.222655in,,top]{\rmfamily\fontsize{10.000000}{12.000000}\selectfont \(\displaystyle 0.00\)}%
\end{pgfscope}%
\begin{pgfscope}%
\pgfsetbuttcap%
\pgfsetroundjoin%
\definecolor{currentfill}{rgb}{0.000000,0.000000,0.000000}%
\pgfsetfillcolor{currentfill}%
\pgfsetlinewidth{0.803000pt}%
\definecolor{currentstroke}{rgb}{0.000000,0.000000,0.000000}%
\pgfsetstrokecolor{currentstroke}%
\pgfsetdash{}{0pt}%
\pgfsys@defobject{currentmarker}{\pgfqpoint{0.000000in}{-0.048611in}}{\pgfqpoint{0.000000in}{0.000000in}}{%
\pgfpathmoveto{\pgfqpoint{0.000000in}{0.000000in}}%
\pgfpathlineto{\pgfqpoint{0.000000in}{-0.048611in}}%
\pgfusepath{stroke,fill}%
}%
\begin{pgfscope}%
\pgfsys@transformshift{1.587961in}{0.319877in}%
\pgfsys@useobject{currentmarker}{}%
\end{pgfscope}%
\end{pgfscope}%
\begin{pgfscope}%
\pgftext[x=1.587961in,y=0.222655in,,top]{\rmfamily\fontsize{10.000000}{12.000000}\selectfont \(\displaystyle 0.05\)}%
\end{pgfscope}%
\begin{pgfscope}%
\pgfsetbuttcap%
\pgfsetroundjoin%
\definecolor{currentfill}{rgb}{0.000000,0.000000,0.000000}%
\pgfsetfillcolor{currentfill}%
\pgfsetlinewidth{0.803000pt}%
\definecolor{currentstroke}{rgb}{0.000000,0.000000,0.000000}%
\pgfsetstrokecolor{currentstroke}%
\pgfsetdash{}{0pt}%
\pgfsys@defobject{currentmarker}{\pgfqpoint{-0.048611in}{0.000000in}}{\pgfqpoint{0.000000in}{0.000000in}}{%
\pgfpathmoveto{\pgfqpoint{0.000000in}{0.000000in}}%
\pgfpathlineto{\pgfqpoint{-0.048611in}{0.000000in}}%
\pgfusepath{stroke,fill}%
}%
\begin{pgfscope}%
\pgfsys@transformshift{0.374692in}{0.436809in}%
\pgfsys@useobject{currentmarker}{}%
\end{pgfscope}%
\end{pgfscope}%
\begin{pgfscope}%
\pgftext[x=0.100000in,y=0.388981in,left,base]{\rmfamily\fontsize{10.000000}{12.000000}\selectfont \(\displaystyle 0.0\)}%
\end{pgfscope}%
\begin{pgfscope}%
\pgfsetbuttcap%
\pgfsetroundjoin%
\definecolor{currentfill}{rgb}{0.000000,0.000000,0.000000}%
\pgfsetfillcolor{currentfill}%
\pgfsetlinewidth{0.803000pt}%
\definecolor{currentstroke}{rgb}{0.000000,0.000000,0.000000}%
\pgfsetstrokecolor{currentstroke}%
\pgfsetdash{}{0pt}%
\pgfsys@defobject{currentmarker}{\pgfqpoint{-0.048611in}{0.000000in}}{\pgfqpoint{0.000000in}{0.000000in}}{%
\pgfpathmoveto{\pgfqpoint{0.000000in}{0.000000in}}%
\pgfpathlineto{\pgfqpoint{-0.048611in}{0.000000in}}%
\pgfusepath{stroke,fill}%
}%
\begin{pgfscope}%
\pgfsys@transformshift{0.374692in}{0.740854in}%
\pgfsys@useobject{currentmarker}{}%
\end{pgfscope}%
\end{pgfscope}%
\begin{pgfscope}%
\pgftext[x=0.100000in,y=0.693026in,left,base]{\rmfamily\fontsize{10.000000}{12.000000}\selectfont \(\displaystyle 0.1\)}%
\end{pgfscope}%
\begin{pgfscope}%
\pgfsetbuttcap%
\pgfsetroundjoin%
\definecolor{currentfill}{rgb}{0.000000,0.000000,0.000000}%
\pgfsetfillcolor{currentfill}%
\pgfsetlinewidth{0.803000pt}%
\definecolor{currentstroke}{rgb}{0.000000,0.000000,0.000000}%
\pgfsetstrokecolor{currentstroke}%
\pgfsetdash{}{0pt}%
\pgfsys@defobject{currentmarker}{\pgfqpoint{-0.048611in}{0.000000in}}{\pgfqpoint{0.000000in}{0.000000in}}{%
\pgfpathmoveto{\pgfqpoint{0.000000in}{0.000000in}}%
\pgfpathlineto{\pgfqpoint{-0.048611in}{0.000000in}}%
\pgfusepath{stroke,fill}%
}%
\begin{pgfscope}%
\pgfsys@transformshift{0.374692in}{1.044899in}%
\pgfsys@useobject{currentmarker}{}%
\end{pgfscope}%
\end{pgfscope}%
\begin{pgfscope}%
\pgftext[x=0.100000in,y=0.997071in,left,base]{\rmfamily\fontsize{10.000000}{12.000000}\selectfont \(\displaystyle 0.2\)}%
\end{pgfscope}%
\begin{pgfscope}%
\pgfsetbuttcap%
\pgfsetroundjoin%
\definecolor{currentfill}{rgb}{0.000000,0.000000,0.000000}%
\pgfsetfillcolor{currentfill}%
\pgfsetlinewidth{0.803000pt}%
\definecolor{currentstroke}{rgb}{0.000000,0.000000,0.000000}%
\pgfsetstrokecolor{currentstroke}%
\pgfsetdash{}{0pt}%
\pgfsys@defobject{currentmarker}{\pgfqpoint{-0.048611in}{0.000000in}}{\pgfqpoint{0.000000in}{0.000000in}}{%
\pgfpathmoveto{\pgfqpoint{0.000000in}{0.000000in}}%
\pgfpathlineto{\pgfqpoint{-0.048611in}{0.000000in}}%
\pgfusepath{stroke,fill}%
}%
\begin{pgfscope}%
\pgfsys@transformshift{0.374692in}{1.348944in}%
\pgfsys@useobject{currentmarker}{}%
\end{pgfscope}%
\end{pgfscope}%
\begin{pgfscope}%
\pgftext[x=0.100000in,y=1.301116in,left,base]{\rmfamily\fontsize{10.000000}{12.000000}\selectfont \(\displaystyle 0.3\)}%
\end{pgfscope}%
\begin{pgfscope}%
\pgfsetbuttcap%
\pgfsetroundjoin%
\definecolor{currentfill}{rgb}{0.000000,0.000000,0.000000}%
\pgfsetfillcolor{currentfill}%
\pgfsetlinewidth{0.803000pt}%
\definecolor{currentstroke}{rgb}{0.000000,0.000000,0.000000}%
\pgfsetstrokecolor{currentstroke}%
\pgfsetdash{}{0pt}%
\pgfsys@defobject{currentmarker}{\pgfqpoint{-0.048611in}{0.000000in}}{\pgfqpoint{0.000000in}{0.000000in}}{%
\pgfpathmoveto{\pgfqpoint{0.000000in}{0.000000in}}%
\pgfpathlineto{\pgfqpoint{-0.048611in}{0.000000in}}%
\pgfusepath{stroke,fill}%
}%
\begin{pgfscope}%
\pgfsys@transformshift{0.374692in}{1.652989in}%
\pgfsys@useobject{currentmarker}{}%
\end{pgfscope}%
\end{pgfscope}%
\begin{pgfscope}%
\pgftext[x=0.100000in,y=1.605161in,left,base]{\rmfamily\fontsize{10.000000}{12.000000}\selectfont \(\displaystyle 0.4\)}%
\end{pgfscope}%
\begin{pgfscope}%
\pgfsetbuttcap%
\pgfsetroundjoin%
\definecolor{currentfill}{rgb}{0.000000,0.000000,0.000000}%
\pgfsetfillcolor{currentfill}%
\pgfsetlinewidth{0.803000pt}%
\definecolor{currentstroke}{rgb}{0.000000,0.000000,0.000000}%
\pgfsetstrokecolor{currentstroke}%
\pgfsetdash{}{0pt}%
\pgfsys@defobject{currentmarker}{\pgfqpoint{-0.048611in}{0.000000in}}{\pgfqpoint{0.000000in}{0.000000in}}{%
\pgfpathmoveto{\pgfqpoint{0.000000in}{0.000000in}}%
\pgfpathlineto{\pgfqpoint{-0.048611in}{0.000000in}}%
\pgfusepath{stroke,fill}%
}%
\begin{pgfscope}%
\pgfsys@transformshift{0.374692in}{1.957034in}%
\pgfsys@useobject{currentmarker}{}%
\end{pgfscope}%
\end{pgfscope}%
\begin{pgfscope}%
\pgftext[x=0.100000in,y=1.909207in,left,base]{\rmfamily\fontsize{10.000000}{12.000000}\selectfont \(\displaystyle 0.5\)}%
\end{pgfscope}%
\begin{pgfscope}%
\pgfsetbuttcap%
\pgfsetroundjoin%
\definecolor{currentfill}{rgb}{0.000000,0.000000,0.000000}%
\pgfsetfillcolor{currentfill}%
\pgfsetlinewidth{0.803000pt}%
\definecolor{currentstroke}{rgb}{0.000000,0.000000,0.000000}%
\pgfsetstrokecolor{currentstroke}%
\pgfsetdash{}{0pt}%
\pgfsys@defobject{currentmarker}{\pgfqpoint{-0.048611in}{0.000000in}}{\pgfqpoint{0.000000in}{0.000000in}}{%
\pgfpathmoveto{\pgfqpoint{0.000000in}{0.000000in}}%
\pgfpathlineto{\pgfqpoint{-0.048611in}{0.000000in}}%
\pgfusepath{stroke,fill}%
}%
\begin{pgfscope}%
\pgfsys@transformshift{0.374692in}{2.261079in}%
\pgfsys@useobject{currentmarker}{}%
\end{pgfscope}%
\end{pgfscope}%
\begin{pgfscope}%
\pgftext[x=0.100000in,y=2.213252in,left,base]{\rmfamily\fontsize{10.000000}{12.000000}\selectfont \(\displaystyle 0.6\)}%
\end{pgfscope}%
\begin{pgfscope}%
\pgfsetbuttcap%
\pgfsetroundjoin%
\definecolor{currentfill}{rgb}{0.000000,0.000000,0.000000}%
\pgfsetfillcolor{currentfill}%
\pgfsetlinewidth{0.803000pt}%
\definecolor{currentstroke}{rgb}{0.000000,0.000000,0.000000}%
\pgfsetstrokecolor{currentstroke}%
\pgfsetdash{}{0pt}%
\pgfsys@defobject{currentmarker}{\pgfqpoint{-0.048611in}{0.000000in}}{\pgfqpoint{0.000000in}{0.000000in}}{%
\pgfpathmoveto{\pgfqpoint{0.000000in}{0.000000in}}%
\pgfpathlineto{\pgfqpoint{-0.048611in}{0.000000in}}%
\pgfusepath{stroke,fill}%
}%
\begin{pgfscope}%
\pgfsys@transformshift{0.374692in}{2.565124in}%
\pgfsys@useobject{currentmarker}{}%
\end{pgfscope}%
\end{pgfscope}%
\begin{pgfscope}%
\pgftext[x=0.100000in,y=2.517297in,left,base]{\rmfamily\fontsize{10.000000}{12.000000}\selectfont \(\displaystyle 0.7\)}%
\end{pgfscope}%
\begin{pgfscope}%
\pgfsetbuttcap%
\pgfsetroundjoin%
\definecolor{currentfill}{rgb}{0.000000,0.000000,0.000000}%
\pgfsetfillcolor{currentfill}%
\pgfsetlinewidth{0.803000pt}%
\definecolor{currentstroke}{rgb}{0.000000,0.000000,0.000000}%
\pgfsetstrokecolor{currentstroke}%
\pgfsetdash{}{0pt}%
\pgfsys@defobject{currentmarker}{\pgfqpoint{-0.048611in}{0.000000in}}{\pgfqpoint{0.000000in}{0.000000in}}{%
\pgfpathmoveto{\pgfqpoint{0.000000in}{0.000000in}}%
\pgfpathlineto{\pgfqpoint{-0.048611in}{0.000000in}}%
\pgfusepath{stroke,fill}%
}%
\begin{pgfscope}%
\pgfsys@transformshift{0.374692in}{2.869169in}%
\pgfsys@useobject{currentmarker}{}%
\end{pgfscope}%
\end{pgfscope}%
\begin{pgfscope}%
\pgftext[x=0.100000in,y=2.821342in,left,base]{\rmfamily\fontsize{10.000000}{12.000000}\selectfont \(\displaystyle 0.8\)}%
\end{pgfscope}%
\begin{pgfscope}%
\pgfsetrectcap%
\pgfsetmiterjoin%
\pgfsetlinewidth{0.803000pt}%
\definecolor{currentstroke}{rgb}{0.000000,0.000000,0.000000}%
\pgfsetstrokecolor{currentstroke}%
\pgfsetdash{}{0pt}%
\pgfpathmoveto{\pgfqpoint{0.374692in}{0.319877in}}%
\pgfpathlineto{\pgfqpoint{0.374692in}{2.925408in}}%
\pgfusepath{stroke}%
\end{pgfscope}%
\begin{pgfscope}%
\pgfsetrectcap%
\pgfsetmiterjoin%
\pgfsetlinewidth{0.803000pt}%
\definecolor{currentstroke}{rgb}{0.000000,0.000000,0.000000}%
\pgfsetstrokecolor{currentstroke}%
\pgfsetdash{}{0pt}%
\pgfpathmoveto{\pgfqpoint{1.867946in}{0.319877in}}%
\pgfpathlineto{\pgfqpoint{1.867946in}{2.925408in}}%
\pgfusepath{stroke}%
\end{pgfscope}%
\begin{pgfscope}%
\pgfsetrectcap%
\pgfsetmiterjoin%
\pgfsetlinewidth{0.803000pt}%
\definecolor{currentstroke}{rgb}{0.000000,0.000000,0.000000}%
\pgfsetstrokecolor{currentstroke}%
\pgfsetdash{}{0pt}%
\pgfpathmoveto{\pgfqpoint{0.374692in}{0.319877in}}%
\pgfpathlineto{\pgfqpoint{1.867946in}{0.319877in}}%
\pgfusepath{stroke}%
\end{pgfscope}%
\begin{pgfscope}%
\pgfsetrectcap%
\pgfsetmiterjoin%
\pgfsetlinewidth{0.803000pt}%
\definecolor{currentstroke}{rgb}{0.000000,0.000000,0.000000}%
\pgfsetstrokecolor{currentstroke}%
\pgfsetdash{}{0pt}%
\pgfpathmoveto{\pgfqpoint{0.374692in}{2.925408in}}%
\pgfpathlineto{\pgfqpoint{1.867946in}{2.925408in}}%
\pgfusepath{stroke}%
\end{pgfscope}%
\begin{pgfscope}%
\pgfpathrectangle{\pgfqpoint{1.961274in}{0.319877in}}{\pgfqpoint{0.130277in}{2.605531in}} %
\pgfusepath{clip}%
\pgfsetbuttcap%
\pgfsetmiterjoin%
\definecolor{currentfill}{rgb}{1.000000,1.000000,1.000000}%
\pgfsetfillcolor{currentfill}%
\pgfsetlinewidth{0.010037pt}%
\definecolor{currentstroke}{rgb}{1.000000,1.000000,1.000000}%
\pgfsetstrokecolor{currentstroke}%
\pgfsetdash{}{0pt}%
\pgfpathmoveto{\pgfqpoint{1.961274in}{0.319877in}}%
\pgfpathlineto{\pgfqpoint{1.961274in}{0.330055in}}%
\pgfpathlineto{\pgfqpoint{1.961274in}{2.915230in}}%
\pgfpathlineto{\pgfqpoint{1.961274in}{2.925408in}}%
\pgfpathlineto{\pgfqpoint{2.091551in}{2.925408in}}%
\pgfpathlineto{\pgfqpoint{2.091551in}{2.915230in}}%
\pgfpathlineto{\pgfqpoint{2.091551in}{0.330055in}}%
\pgfpathlineto{\pgfqpoint{2.091551in}{0.319877in}}%
\pgfpathclose%
\pgfusepath{stroke,fill}%
\end{pgfscope}%
\begin{pgfscope}%
\pgfsys@transformshift{1.960000in}{0.320408in}%
\pgftext[left,bottom]{\pgfimage[interpolate=true,width=0.130000in,height=2.610000in]{RnnStDev_vs_dq_Ti_300K-img1.png}}%
\end{pgfscope}%
\begin{pgfscope}%
\pgfsetbuttcap%
\pgfsetroundjoin%
\definecolor{currentfill}{rgb}{0.000000,0.000000,0.000000}%
\pgfsetfillcolor{currentfill}%
\pgfsetlinewidth{0.803000pt}%
\definecolor{currentstroke}{rgb}{0.000000,0.000000,0.000000}%
\pgfsetstrokecolor{currentstroke}%
\pgfsetdash{}{0pt}%
\pgfsys@defobject{currentmarker}{\pgfqpoint{0.000000in}{0.000000in}}{\pgfqpoint{0.048611in}{0.000000in}}{%
\pgfpathmoveto{\pgfqpoint{0.000000in}{0.000000in}}%
\pgfpathlineto{\pgfqpoint{0.048611in}{0.000000in}}%
\pgfusepath{stroke,fill}%
}%
\begin{pgfscope}%
\pgfsys@transformshift{2.091551in}{0.319877in}%
\pgfsys@useobject{currentmarker}{}%
\end{pgfscope}%
\end{pgfscope}%
\begin{pgfscope}%
\pgftext[x=2.188773in,y=0.272050in,left,base]{\rmfamily\fontsize{10.000000}{12.000000}\selectfont \(\displaystyle 0.0\)}%
\end{pgfscope}%
\begin{pgfscope}%
\pgfsetbuttcap%
\pgfsetroundjoin%
\definecolor{currentfill}{rgb}{0.000000,0.000000,0.000000}%
\pgfsetfillcolor{currentfill}%
\pgfsetlinewidth{0.803000pt}%
\definecolor{currentstroke}{rgb}{0.000000,0.000000,0.000000}%
\pgfsetstrokecolor{currentstroke}%
\pgfsetdash{}{0pt}%
\pgfsys@defobject{currentmarker}{\pgfqpoint{0.000000in}{0.000000in}}{\pgfqpoint{0.048611in}{0.000000in}}{%
\pgfpathmoveto{\pgfqpoint{0.000000in}{0.000000in}}%
\pgfpathlineto{\pgfqpoint{0.048611in}{0.000000in}}%
\pgfusepath{stroke,fill}%
}%
\begin{pgfscope}%
\pgfsys@transformshift{2.091551in}{0.662710in}%
\pgfsys@useobject{currentmarker}{}%
\end{pgfscope}%
\end{pgfscope}%
\begin{pgfscope}%
\pgftext[x=2.188773in,y=0.614883in,left,base]{\rmfamily\fontsize{10.000000}{12.000000}\selectfont \(\displaystyle 2.5\)}%
\end{pgfscope}%
\begin{pgfscope}%
\pgfsetbuttcap%
\pgfsetroundjoin%
\definecolor{currentfill}{rgb}{0.000000,0.000000,0.000000}%
\pgfsetfillcolor{currentfill}%
\pgfsetlinewidth{0.803000pt}%
\definecolor{currentstroke}{rgb}{0.000000,0.000000,0.000000}%
\pgfsetstrokecolor{currentstroke}%
\pgfsetdash{}{0pt}%
\pgfsys@defobject{currentmarker}{\pgfqpoint{0.000000in}{0.000000in}}{\pgfqpoint{0.048611in}{0.000000in}}{%
\pgfpathmoveto{\pgfqpoint{0.000000in}{0.000000in}}%
\pgfpathlineto{\pgfqpoint{0.048611in}{0.000000in}}%
\pgfusepath{stroke,fill}%
}%
\begin{pgfscope}%
\pgfsys@transformshift{2.091551in}{1.005543in}%
\pgfsys@useobject{currentmarker}{}%
\end{pgfscope}%
\end{pgfscope}%
\begin{pgfscope}%
\pgftext[x=2.188773in,y=0.957716in,left,base]{\rmfamily\fontsize{10.000000}{12.000000}\selectfont \(\displaystyle 5.0\)}%
\end{pgfscope}%
\begin{pgfscope}%
\pgfsetbuttcap%
\pgfsetroundjoin%
\definecolor{currentfill}{rgb}{0.000000,0.000000,0.000000}%
\pgfsetfillcolor{currentfill}%
\pgfsetlinewidth{0.803000pt}%
\definecolor{currentstroke}{rgb}{0.000000,0.000000,0.000000}%
\pgfsetstrokecolor{currentstroke}%
\pgfsetdash{}{0pt}%
\pgfsys@defobject{currentmarker}{\pgfqpoint{0.000000in}{0.000000in}}{\pgfqpoint{0.048611in}{0.000000in}}{%
\pgfpathmoveto{\pgfqpoint{0.000000in}{0.000000in}}%
\pgfpathlineto{\pgfqpoint{0.048611in}{0.000000in}}%
\pgfusepath{stroke,fill}%
}%
\begin{pgfscope}%
\pgfsys@transformshift{2.091551in}{1.348376in}%
\pgfsys@useobject{currentmarker}{}%
\end{pgfscope}%
\end{pgfscope}%
\begin{pgfscope}%
\pgftext[x=2.188773in,y=1.300548in,left,base]{\rmfamily\fontsize{10.000000}{12.000000}\selectfont \(\displaystyle 7.5\)}%
\end{pgfscope}%
\begin{pgfscope}%
\pgfsetbuttcap%
\pgfsetroundjoin%
\definecolor{currentfill}{rgb}{0.000000,0.000000,0.000000}%
\pgfsetfillcolor{currentfill}%
\pgfsetlinewidth{0.803000pt}%
\definecolor{currentstroke}{rgb}{0.000000,0.000000,0.000000}%
\pgfsetstrokecolor{currentstroke}%
\pgfsetdash{}{0pt}%
\pgfsys@defobject{currentmarker}{\pgfqpoint{0.000000in}{0.000000in}}{\pgfqpoint{0.048611in}{0.000000in}}{%
\pgfpathmoveto{\pgfqpoint{0.000000in}{0.000000in}}%
\pgfpathlineto{\pgfqpoint{0.048611in}{0.000000in}}%
\pgfusepath{stroke,fill}%
}%
\begin{pgfscope}%
\pgfsys@transformshift{2.091551in}{1.691209in}%
\pgfsys@useobject{currentmarker}{}%
\end{pgfscope}%
\end{pgfscope}%
\begin{pgfscope}%
\pgftext[x=2.188773in,y=1.643381in,left,base]{\rmfamily\fontsize{10.000000}{12.000000}\selectfont \(\displaystyle 10.0\)}%
\end{pgfscope}%
\begin{pgfscope}%
\pgfsetbuttcap%
\pgfsetroundjoin%
\definecolor{currentfill}{rgb}{0.000000,0.000000,0.000000}%
\pgfsetfillcolor{currentfill}%
\pgfsetlinewidth{0.803000pt}%
\definecolor{currentstroke}{rgb}{0.000000,0.000000,0.000000}%
\pgfsetstrokecolor{currentstroke}%
\pgfsetdash{}{0pt}%
\pgfsys@defobject{currentmarker}{\pgfqpoint{0.000000in}{0.000000in}}{\pgfqpoint{0.048611in}{0.000000in}}{%
\pgfpathmoveto{\pgfqpoint{0.000000in}{0.000000in}}%
\pgfpathlineto{\pgfqpoint{0.048611in}{0.000000in}}%
\pgfusepath{stroke,fill}%
}%
\begin{pgfscope}%
\pgfsys@transformshift{2.091551in}{2.034042in}%
\pgfsys@useobject{currentmarker}{}%
\end{pgfscope}%
\end{pgfscope}%
\begin{pgfscope}%
\pgftext[x=2.188773in,y=1.986214in,left,base]{\rmfamily\fontsize{10.000000}{12.000000}\selectfont \(\displaystyle 12.5\)}%
\end{pgfscope}%
\begin{pgfscope}%
\pgfsetbuttcap%
\pgfsetroundjoin%
\definecolor{currentfill}{rgb}{0.000000,0.000000,0.000000}%
\pgfsetfillcolor{currentfill}%
\pgfsetlinewidth{0.803000pt}%
\definecolor{currentstroke}{rgb}{0.000000,0.000000,0.000000}%
\pgfsetstrokecolor{currentstroke}%
\pgfsetdash{}{0pt}%
\pgfsys@defobject{currentmarker}{\pgfqpoint{0.000000in}{0.000000in}}{\pgfqpoint{0.048611in}{0.000000in}}{%
\pgfpathmoveto{\pgfqpoint{0.000000in}{0.000000in}}%
\pgfpathlineto{\pgfqpoint{0.048611in}{0.000000in}}%
\pgfusepath{stroke,fill}%
}%
\begin{pgfscope}%
\pgfsys@transformshift{2.091551in}{2.376875in}%
\pgfsys@useobject{currentmarker}{}%
\end{pgfscope}%
\end{pgfscope}%
\begin{pgfscope}%
\pgftext[x=2.188773in,y=2.329047in,left,base]{\rmfamily\fontsize{10.000000}{12.000000}\selectfont \(\displaystyle 15.0\)}%
\end{pgfscope}%
\begin{pgfscope}%
\pgfsetbuttcap%
\pgfsetroundjoin%
\definecolor{currentfill}{rgb}{0.000000,0.000000,0.000000}%
\pgfsetfillcolor{currentfill}%
\pgfsetlinewidth{0.803000pt}%
\definecolor{currentstroke}{rgb}{0.000000,0.000000,0.000000}%
\pgfsetstrokecolor{currentstroke}%
\pgfsetdash{}{0pt}%
\pgfsys@defobject{currentmarker}{\pgfqpoint{0.000000in}{0.000000in}}{\pgfqpoint{0.048611in}{0.000000in}}{%
\pgfpathmoveto{\pgfqpoint{0.000000in}{0.000000in}}%
\pgfpathlineto{\pgfqpoint{0.048611in}{0.000000in}}%
\pgfusepath{stroke,fill}%
}%
\begin{pgfscope}%
\pgfsys@transformshift{2.091551in}{2.719708in}%
\pgfsys@useobject{currentmarker}{}%
\end{pgfscope}%
\end{pgfscope}%
\begin{pgfscope}%
\pgftext[x=2.188773in,y=2.671880in,left,base]{\rmfamily\fontsize{10.000000}{12.000000}\selectfont \(\displaystyle 17.5\)}%
\end{pgfscope}%
\begin{pgfscope}%
\pgfsetbuttcap%
\pgfsetmiterjoin%
\pgfsetlinewidth{0.803000pt}%
\definecolor{currentstroke}{rgb}{0.000000,0.000000,0.000000}%
\pgfsetstrokecolor{currentstroke}%
\pgfsetdash{}{0pt}%
\pgfpathmoveto{\pgfqpoint{1.961274in}{0.319877in}}%
\pgfpathlineto{\pgfqpoint{1.961274in}{0.330055in}}%
\pgfpathlineto{\pgfqpoint{1.961274in}{2.915230in}}%
\pgfpathlineto{\pgfqpoint{1.961274in}{2.925408in}}%
\pgfpathlineto{\pgfqpoint{2.091551in}{2.925408in}}%
\pgfpathlineto{\pgfqpoint{2.091551in}{2.915230in}}%
\pgfpathlineto{\pgfqpoint{2.091551in}{0.330055in}}%
\pgfpathlineto{\pgfqpoint{2.091551in}{0.319877in}}%
\pgfpathclose%
\pgfusepath{stroke}%
\end{pgfscope}%
\end{pgfpicture}%
\makeatother%
\endgroup%

    \vspace*{-0.4cm}
	\caption{300 K. Bin size $0.014e$}
	\end{subfigure}
	\hspace{0.6cm}
	\begin{subfigure}[b]{0.45\textwidth}
	\hspace*{-0.4cm}
	%% Creator: Matplotlib, PGF backend
%%
%% To include the figure in your LaTeX document, write
%%   \input{<filename>.pgf}
%%
%% Make sure the required packages are loaded in your preamble
%%   \usepackage{pgf}
%%
%% Figures using additional raster images can only be included by \input if
%% they are in the same directory as the main LaTeX file. For loading figures
%% from other directories you can use the `import` package
%%   \usepackage{import}
%% and then include the figures with
%%   \import{<path to file>}{<filename>.pgf}
%%
%% Matplotlib used the following preamble
%%   \usepackage[utf8x]{inputenc}
%%   \usepackage[T1]{fontenc}
%%
\begingroup%
\makeatletter%
\begin{pgfpicture}%
\pgfpathrectangle{\pgfpointorigin}{\pgfqpoint{2.535687in}{3.060408in}}%
\pgfusepath{use as bounding box, clip}%
\begin{pgfscope}%
\pgfsetbuttcap%
\pgfsetmiterjoin%
\definecolor{currentfill}{rgb}{1.000000,1.000000,1.000000}%
\pgfsetfillcolor{currentfill}%
\pgfsetlinewidth{0.000000pt}%
\definecolor{currentstroke}{rgb}{1.000000,1.000000,1.000000}%
\pgfsetstrokecolor{currentstroke}%
\pgfsetdash{}{0pt}%
\pgfpathmoveto{\pgfqpoint{0.000000in}{0.000000in}}%
\pgfpathlineto{\pgfqpoint{2.535687in}{0.000000in}}%
\pgfpathlineto{\pgfqpoint{2.535687in}{3.060408in}}%
\pgfpathlineto{\pgfqpoint{0.000000in}{3.060408in}}%
\pgfpathclose%
\pgfusepath{fill}%
\end{pgfscope}%
\begin{pgfscope}%
\pgfsetbuttcap%
\pgfsetmiterjoin%
\definecolor{currentfill}{rgb}{1.000000,1.000000,1.000000}%
\pgfsetfillcolor{currentfill}%
\pgfsetlinewidth{0.000000pt}%
\definecolor{currentstroke}{rgb}{0.000000,0.000000,0.000000}%
\pgfsetstrokecolor{currentstroke}%
\pgfsetstrokeopacity{0.000000}%
\pgfsetdash{}{0pt}%
\pgfpathmoveto{\pgfqpoint{0.374692in}{0.319877in}}%
\pgfpathlineto{\pgfqpoint{1.867946in}{0.319877in}}%
\pgfpathlineto{\pgfqpoint{1.867946in}{2.925408in}}%
\pgfpathlineto{\pgfqpoint{0.374692in}{2.925408in}}%
\pgfpathclose%
\pgfusepath{fill}%
\end{pgfscope}%
\begin{pgfscope}%
\pgfpathrectangle{\pgfqpoint{0.374692in}{0.319877in}}{\pgfqpoint{1.493254in}{2.605531in}} %
\pgfusepath{clip}%
\pgfsys@transformshift{0.374692in}{0.319877in}%
\pgftext[left,bottom]{\pgfimage[interpolate=true,width=1.500000in,height=2.610000in]{RnnStDev_vs_dq_Ti_500K-img0.png}}%
\end{pgfscope}%
\begin{pgfscope}%
\pgfpathrectangle{\pgfqpoint{0.374692in}{0.319877in}}{\pgfqpoint{1.493254in}{2.605531in}} %
\pgfusepath{clip}%
\pgfsetbuttcap%
\pgfsetroundjoin%
\definecolor{currentfill}{rgb}{1.000000,0.752941,0.796078}%
\pgfsetfillcolor{currentfill}%
\pgfsetlinewidth{1.003750pt}%
\definecolor{currentstroke}{rgb}{1.000000,0.752941,0.796078}%
\pgfsetstrokecolor{currentstroke}%
\pgfsetdash{}{0pt}%
\pgfpathmoveto{\pgfqpoint{0.654677in}{1.055299in}}%
\pgfpathcurveto{\pgfqpoint{0.665727in}{1.055299in}}{\pgfqpoint{0.676326in}{1.059689in}}{\pgfqpoint{0.684140in}{1.067503in}}%
\pgfpathcurveto{\pgfqpoint{0.691953in}{1.075316in}}{\pgfqpoint{0.696344in}{1.085915in}}{\pgfqpoint{0.696344in}{1.096965in}}%
\pgfpathcurveto{\pgfqpoint{0.696344in}{1.108016in}}{\pgfqpoint{0.691953in}{1.118615in}}{\pgfqpoint{0.684140in}{1.126428in}}%
\pgfpathcurveto{\pgfqpoint{0.676326in}{1.134242in}}{\pgfqpoint{0.665727in}{1.138632in}}{\pgfqpoint{0.654677in}{1.138632in}}%
\pgfpathcurveto{\pgfqpoint{0.643627in}{1.138632in}}{\pgfqpoint{0.633028in}{1.134242in}}{\pgfqpoint{0.625214in}{1.126428in}}%
\pgfpathcurveto{\pgfqpoint{0.617401in}{1.118615in}}{\pgfqpoint{0.613010in}{1.108016in}}{\pgfqpoint{0.613010in}{1.096965in}}%
\pgfpathcurveto{\pgfqpoint{0.613010in}{1.085915in}}{\pgfqpoint{0.617401in}{1.075316in}}{\pgfqpoint{0.625214in}{1.067503in}}%
\pgfpathcurveto{\pgfqpoint{0.633028in}{1.059689in}}{\pgfqpoint{0.643627in}{1.055299in}}{\pgfqpoint{0.654677in}{1.055299in}}%
\pgfpathclose%
\pgfusepath{stroke,fill}%
\end{pgfscope}%
\begin{pgfscope}%
\pgfpathrectangle{\pgfqpoint{0.374692in}{0.319877in}}{\pgfqpoint{1.493254in}{2.605531in}} %
\pgfusepath{clip}%
\pgfsetbuttcap%
\pgfsetroundjoin%
\definecolor{currentfill}{rgb}{1.000000,0.752941,0.796078}%
\pgfsetfillcolor{currentfill}%
\pgfsetlinewidth{1.003750pt}%
\definecolor{currentstroke}{rgb}{1.000000,0.752941,0.796078}%
\pgfsetstrokecolor{currentstroke}%
\pgfsetdash{}{0pt}%
\pgfpathmoveto{\pgfqpoint{0.841334in}{1.209910in}}%
\pgfpathcurveto{\pgfqpoint{0.852384in}{1.209910in}}{\pgfqpoint{0.862983in}{1.214300in}}{\pgfqpoint{0.870797in}{1.222114in}}%
\pgfpathcurveto{\pgfqpoint{0.878610in}{1.229927in}}{\pgfqpoint{0.883000in}{1.240526in}}{\pgfqpoint{0.883000in}{1.251576in}}%
\pgfpathcurveto{\pgfqpoint{0.883000in}{1.262626in}}{\pgfqpoint{0.878610in}{1.273226in}}{\pgfqpoint{0.870797in}{1.281039in}}%
\pgfpathcurveto{\pgfqpoint{0.862983in}{1.288853in}}{\pgfqpoint{0.852384in}{1.293243in}}{\pgfqpoint{0.841334in}{1.293243in}}%
\pgfpathcurveto{\pgfqpoint{0.830284in}{1.293243in}}{\pgfqpoint{0.819685in}{1.288853in}}{\pgfqpoint{0.811871in}{1.281039in}}%
\pgfpathcurveto{\pgfqpoint{0.804057in}{1.273226in}}{\pgfqpoint{0.799667in}{1.262626in}}{\pgfqpoint{0.799667in}{1.251576in}}%
\pgfpathcurveto{\pgfqpoint{0.799667in}{1.240526in}}{\pgfqpoint{0.804057in}{1.229927in}}{\pgfqpoint{0.811871in}{1.222114in}}%
\pgfpathcurveto{\pgfqpoint{0.819685in}{1.214300in}}{\pgfqpoint{0.830284in}{1.209910in}}{\pgfqpoint{0.841334in}{1.209910in}}%
\pgfpathclose%
\pgfusepath{stroke,fill}%
\end{pgfscope}%
\begin{pgfscope}%
\pgfpathrectangle{\pgfqpoint{0.374692in}{0.319877in}}{\pgfqpoint{1.493254in}{2.605531in}} %
\pgfusepath{clip}%
\pgfsetbuttcap%
\pgfsetroundjoin%
\definecolor{currentfill}{rgb}{1.000000,0.752941,0.796078}%
\pgfsetfillcolor{currentfill}%
\pgfsetlinewidth{1.003750pt}%
\definecolor{currentstroke}{rgb}{1.000000,0.752941,0.796078}%
\pgfsetstrokecolor{currentstroke}%
\pgfsetdash{}{0pt}%
\pgfpathmoveto{\pgfqpoint{1.027990in}{1.236684in}}%
\pgfpathcurveto{\pgfqpoint{1.039041in}{1.236684in}}{\pgfqpoint{1.049640in}{1.241074in}}{\pgfqpoint{1.057453in}{1.248888in}}%
\pgfpathcurveto{\pgfqpoint{1.065267in}{1.256702in}}{\pgfqpoint{1.069657in}{1.267301in}}{\pgfqpoint{1.069657in}{1.278351in}}%
\pgfpathcurveto{\pgfqpoint{1.069657in}{1.289401in}}{\pgfqpoint{1.065267in}{1.300000in}}{\pgfqpoint{1.057453in}{1.307814in}}%
\pgfpathcurveto{\pgfqpoint{1.049640in}{1.315627in}}{\pgfqpoint{1.039041in}{1.320017in}}{\pgfqpoint{1.027990in}{1.320017in}}%
\pgfpathcurveto{\pgfqpoint{1.016940in}{1.320017in}}{\pgfqpoint{1.006341in}{1.315627in}}{\pgfqpoint{0.998528in}{1.307814in}}%
\pgfpathcurveto{\pgfqpoint{0.990714in}{1.300000in}}{\pgfqpoint{0.986324in}{1.289401in}}{\pgfqpoint{0.986324in}{1.278351in}}%
\pgfpathcurveto{\pgfqpoint{0.986324in}{1.267301in}}{\pgfqpoint{0.990714in}{1.256702in}}{\pgfqpoint{0.998528in}{1.248888in}}%
\pgfpathcurveto{\pgfqpoint{1.006341in}{1.241074in}}{\pgfqpoint{1.016940in}{1.236684in}}{\pgfqpoint{1.027990in}{1.236684in}}%
\pgfpathclose%
\pgfusepath{stroke,fill}%
\end{pgfscope}%
\begin{pgfscope}%
\pgfpathrectangle{\pgfqpoint{0.374692in}{0.319877in}}{\pgfqpoint{1.493254in}{2.605531in}} %
\pgfusepath{clip}%
\pgfsetbuttcap%
\pgfsetroundjoin%
\definecolor{currentfill}{rgb}{1.000000,0.752941,0.796078}%
\pgfsetfillcolor{currentfill}%
\pgfsetlinewidth{1.003750pt}%
\definecolor{currentstroke}{rgb}{1.000000,0.752941,0.796078}%
\pgfsetstrokecolor{currentstroke}%
\pgfsetdash{}{0pt}%
\pgfpathmoveto{\pgfqpoint{1.214647in}{1.249924in}}%
\pgfpathcurveto{\pgfqpoint{1.225697in}{1.249924in}}{\pgfqpoint{1.236296in}{1.254314in}}{\pgfqpoint{1.244110in}{1.262128in}}%
\pgfpathcurveto{\pgfqpoint{1.251924in}{1.269942in}}{\pgfqpoint{1.256314in}{1.280541in}}{\pgfqpoint{1.256314in}{1.291591in}}%
\pgfpathcurveto{\pgfqpoint{1.256314in}{1.302641in}}{\pgfqpoint{1.251924in}{1.313240in}}{\pgfqpoint{1.244110in}{1.321054in}}%
\pgfpathcurveto{\pgfqpoint{1.236296in}{1.328867in}}{\pgfqpoint{1.225697in}{1.333258in}}{\pgfqpoint{1.214647in}{1.333258in}}%
\pgfpathcurveto{\pgfqpoint{1.203597in}{1.333258in}}{\pgfqpoint{1.192998in}{1.328867in}}{\pgfqpoint{1.185184in}{1.321054in}}%
\pgfpathcurveto{\pgfqpoint{1.177371in}{1.313240in}}{\pgfqpoint{1.172981in}{1.302641in}}{\pgfqpoint{1.172981in}{1.291591in}}%
\pgfpathcurveto{\pgfqpoint{1.172981in}{1.280541in}}{\pgfqpoint{1.177371in}{1.269942in}}{\pgfqpoint{1.185184in}{1.262128in}}%
\pgfpathcurveto{\pgfqpoint{1.192998in}{1.254314in}}{\pgfqpoint{1.203597in}{1.249924in}}{\pgfqpoint{1.214647in}{1.249924in}}%
\pgfpathclose%
\pgfusepath{stroke,fill}%
\end{pgfscope}%
\begin{pgfscope}%
\pgfpathrectangle{\pgfqpoint{0.374692in}{0.319877in}}{\pgfqpoint{1.493254in}{2.605531in}} %
\pgfusepath{clip}%
\pgfsetbuttcap%
\pgfsetroundjoin%
\definecolor{currentfill}{rgb}{1.000000,0.752941,0.796078}%
\pgfsetfillcolor{currentfill}%
\pgfsetlinewidth{1.003750pt}%
\definecolor{currentstroke}{rgb}{1.000000,0.752941,0.796078}%
\pgfsetstrokecolor{currentstroke}%
\pgfsetdash{}{0pt}%
\pgfpathmoveto{\pgfqpoint{1.401304in}{1.251203in}}%
\pgfpathcurveto{\pgfqpoint{1.412354in}{1.251203in}}{\pgfqpoint{1.422953in}{1.255594in}}{\pgfqpoint{1.430767in}{1.263407in}}%
\pgfpathcurveto{\pgfqpoint{1.438580in}{1.271221in}}{\pgfqpoint{1.442971in}{1.281820in}}{\pgfqpoint{1.442971in}{1.292870in}}%
\pgfpathcurveto{\pgfqpoint{1.442971in}{1.303920in}}{\pgfqpoint{1.438580in}{1.314519in}}{\pgfqpoint{1.430767in}{1.322333in}}%
\pgfpathcurveto{\pgfqpoint{1.422953in}{1.330146in}}{\pgfqpoint{1.412354in}{1.334537in}}{\pgfqpoint{1.401304in}{1.334537in}}%
\pgfpathcurveto{\pgfqpoint{1.390254in}{1.334537in}}{\pgfqpoint{1.379655in}{1.330146in}}{\pgfqpoint{1.371841in}{1.322333in}}%
\pgfpathcurveto{\pgfqpoint{1.364028in}{1.314519in}}{\pgfqpoint{1.359637in}{1.303920in}}{\pgfqpoint{1.359637in}{1.292870in}}%
\pgfpathcurveto{\pgfqpoint{1.359637in}{1.281820in}}{\pgfqpoint{1.364028in}{1.271221in}}{\pgfqpoint{1.371841in}{1.263407in}}%
\pgfpathcurveto{\pgfqpoint{1.379655in}{1.255594in}}{\pgfqpoint{1.390254in}{1.251203in}}{\pgfqpoint{1.401304in}{1.251203in}}%
\pgfpathclose%
\pgfusepath{stroke,fill}%
\end{pgfscope}%
\begin{pgfscope}%
\pgfpathrectangle{\pgfqpoint{0.374692in}{0.319877in}}{\pgfqpoint{1.493254in}{2.605531in}} %
\pgfusepath{clip}%
\pgfsetbuttcap%
\pgfsetroundjoin%
\definecolor{currentfill}{rgb}{1.000000,0.752941,0.796078}%
\pgfsetfillcolor{currentfill}%
\pgfsetlinewidth{1.003750pt}%
\definecolor{currentstroke}{rgb}{1.000000,0.752941,0.796078}%
\pgfsetstrokecolor{currentstroke}%
\pgfsetdash{}{0pt}%
\pgfpathmoveto{\pgfqpoint{1.587961in}{1.512409in}}%
\pgfpathcurveto{\pgfqpoint{1.599011in}{1.512409in}}{\pgfqpoint{1.609610in}{1.516800in}}{\pgfqpoint{1.617423in}{1.524613in}}%
\pgfpathcurveto{\pgfqpoint{1.625237in}{1.532427in}}{\pgfqpoint{1.629627in}{1.543026in}}{\pgfqpoint{1.629627in}{1.554076in}}%
\pgfpathcurveto{\pgfqpoint{1.629627in}{1.565126in}}{\pgfqpoint{1.625237in}{1.575725in}}{\pgfqpoint{1.617423in}{1.583539in}}%
\pgfpathcurveto{\pgfqpoint{1.609610in}{1.591352in}}{\pgfqpoint{1.599011in}{1.595743in}}{\pgfqpoint{1.587961in}{1.595743in}}%
\pgfpathcurveto{\pgfqpoint{1.576911in}{1.595743in}}{\pgfqpoint{1.566311in}{1.591352in}}{\pgfqpoint{1.558498in}{1.583539in}}%
\pgfpathcurveto{\pgfqpoint{1.550684in}{1.575725in}}{\pgfqpoint{1.546294in}{1.565126in}}{\pgfqpoint{1.546294in}{1.554076in}}%
\pgfpathcurveto{\pgfqpoint{1.546294in}{1.543026in}}{\pgfqpoint{1.550684in}{1.532427in}}{\pgfqpoint{1.558498in}{1.524613in}}%
\pgfpathcurveto{\pgfqpoint{1.566311in}{1.516800in}}{\pgfqpoint{1.576911in}{1.512409in}}{\pgfqpoint{1.587961in}{1.512409in}}%
\pgfpathclose%
\pgfusepath{stroke,fill}%
\end{pgfscope}%
\begin{pgfscope}%
\pgfpathrectangle{\pgfqpoint{0.374692in}{0.319877in}}{\pgfqpoint{1.493254in}{2.605531in}} %
\pgfusepath{clip}%
\pgfsetbuttcap%
\pgfsetroundjoin%
\definecolor{currentfill}{rgb}{1.000000,0.752941,0.796078}%
\pgfsetfillcolor{currentfill}%
\pgfsetlinewidth{1.003750pt}%
\definecolor{currentstroke}{rgb}{1.000000,0.752941,0.796078}%
\pgfsetstrokecolor{currentstroke}%
\pgfsetdash{}{0pt}%
\pgfpathmoveto{\pgfqpoint{1.774617in}{1.283854in}}%
\pgfpathcurveto{\pgfqpoint{1.785668in}{1.283854in}}{\pgfqpoint{1.796267in}{1.288244in}}{\pgfqpoint{1.804080in}{1.296058in}}%
\pgfpathcurveto{\pgfqpoint{1.811894in}{1.303872in}}{\pgfqpoint{1.816284in}{1.314471in}}{\pgfqpoint{1.816284in}{1.325521in}}%
\pgfpathcurveto{\pgfqpoint{1.816284in}{1.336571in}}{\pgfqpoint{1.811894in}{1.347170in}}{\pgfqpoint{1.804080in}{1.354983in}}%
\pgfpathcurveto{\pgfqpoint{1.796267in}{1.362797in}}{\pgfqpoint{1.785668in}{1.367187in}}{\pgfqpoint{1.774617in}{1.367187in}}%
\pgfpathcurveto{\pgfqpoint{1.763567in}{1.367187in}}{\pgfqpoint{1.752968in}{1.362797in}}{\pgfqpoint{1.745155in}{1.354983in}}%
\pgfpathcurveto{\pgfqpoint{1.737341in}{1.347170in}}{\pgfqpoint{1.732951in}{1.336571in}}{\pgfqpoint{1.732951in}{1.325521in}}%
\pgfpathcurveto{\pgfqpoint{1.732951in}{1.314471in}}{\pgfqpoint{1.737341in}{1.303872in}}{\pgfqpoint{1.745155in}{1.296058in}}%
\pgfpathcurveto{\pgfqpoint{1.752968in}{1.288244in}}{\pgfqpoint{1.763567in}{1.283854in}}{\pgfqpoint{1.774617in}{1.283854in}}%
\pgfpathclose%
\pgfusepath{stroke,fill}%
\end{pgfscope}%
\begin{pgfscope}%
\pgfsetbuttcap%
\pgfsetroundjoin%
\definecolor{currentfill}{rgb}{0.000000,0.000000,0.000000}%
\pgfsetfillcolor{currentfill}%
\pgfsetlinewidth{0.803000pt}%
\definecolor{currentstroke}{rgb}{0.000000,0.000000,0.000000}%
\pgfsetstrokecolor{currentstroke}%
\pgfsetdash{}{0pt}%
\pgfsys@defobject{currentmarker}{\pgfqpoint{0.000000in}{-0.048611in}}{\pgfqpoint{0.000000in}{0.000000in}}{%
\pgfpathmoveto{\pgfqpoint{0.000000in}{0.000000in}}%
\pgfpathlineto{\pgfqpoint{0.000000in}{-0.048611in}}%
\pgfusepath{stroke,fill}%
}%
\begin{pgfscope}%
\pgfsys@transformshift{0.654677in}{0.319877in}%
\pgfsys@useobject{currentmarker}{}%
\end{pgfscope}%
\end{pgfscope}%
\begin{pgfscope}%
\pgftext[x=0.654677in,y=0.222655in,,top]{\rmfamily\fontsize{10.000000}{12.000000}\selectfont \(\displaystyle -0.05\)}%
\end{pgfscope}%
\begin{pgfscope}%
\pgfsetbuttcap%
\pgfsetroundjoin%
\definecolor{currentfill}{rgb}{0.000000,0.000000,0.000000}%
\pgfsetfillcolor{currentfill}%
\pgfsetlinewidth{0.803000pt}%
\definecolor{currentstroke}{rgb}{0.000000,0.000000,0.000000}%
\pgfsetstrokecolor{currentstroke}%
\pgfsetdash{}{0pt}%
\pgfsys@defobject{currentmarker}{\pgfqpoint{0.000000in}{-0.048611in}}{\pgfqpoint{0.000000in}{0.000000in}}{%
\pgfpathmoveto{\pgfqpoint{0.000000in}{0.000000in}}%
\pgfpathlineto{\pgfqpoint{0.000000in}{-0.048611in}}%
\pgfusepath{stroke,fill}%
}%
\begin{pgfscope}%
\pgfsys@transformshift{1.121319in}{0.319877in}%
\pgfsys@useobject{currentmarker}{}%
\end{pgfscope}%
\end{pgfscope}%
\begin{pgfscope}%
\pgftext[x=1.121319in,y=0.222655in,,top]{\rmfamily\fontsize{10.000000}{12.000000}\selectfont \(\displaystyle 0.00\)}%
\end{pgfscope}%
\begin{pgfscope}%
\pgfsetbuttcap%
\pgfsetroundjoin%
\definecolor{currentfill}{rgb}{0.000000,0.000000,0.000000}%
\pgfsetfillcolor{currentfill}%
\pgfsetlinewidth{0.803000pt}%
\definecolor{currentstroke}{rgb}{0.000000,0.000000,0.000000}%
\pgfsetstrokecolor{currentstroke}%
\pgfsetdash{}{0pt}%
\pgfsys@defobject{currentmarker}{\pgfqpoint{0.000000in}{-0.048611in}}{\pgfqpoint{0.000000in}{0.000000in}}{%
\pgfpathmoveto{\pgfqpoint{0.000000in}{0.000000in}}%
\pgfpathlineto{\pgfqpoint{0.000000in}{-0.048611in}}%
\pgfusepath{stroke,fill}%
}%
\begin{pgfscope}%
\pgfsys@transformshift{1.587961in}{0.319877in}%
\pgfsys@useobject{currentmarker}{}%
\end{pgfscope}%
\end{pgfscope}%
\begin{pgfscope}%
\pgftext[x=1.587961in,y=0.222655in,,top]{\rmfamily\fontsize{10.000000}{12.000000}\selectfont \(\displaystyle 0.05\)}%
\end{pgfscope}%
\begin{pgfscope}%
\pgfsetbuttcap%
\pgfsetroundjoin%
\definecolor{currentfill}{rgb}{0.000000,0.000000,0.000000}%
\pgfsetfillcolor{currentfill}%
\pgfsetlinewidth{0.803000pt}%
\definecolor{currentstroke}{rgb}{0.000000,0.000000,0.000000}%
\pgfsetstrokecolor{currentstroke}%
\pgfsetdash{}{0pt}%
\pgfsys@defobject{currentmarker}{\pgfqpoint{-0.048611in}{0.000000in}}{\pgfqpoint{0.000000in}{0.000000in}}{%
\pgfpathmoveto{\pgfqpoint{0.000000in}{0.000000in}}%
\pgfpathlineto{\pgfqpoint{-0.048611in}{0.000000in}}%
\pgfusepath{stroke,fill}%
}%
\begin{pgfscope}%
\pgfsys@transformshift{0.374692in}{0.436809in}%
\pgfsys@useobject{currentmarker}{}%
\end{pgfscope}%
\end{pgfscope}%
\begin{pgfscope}%
\pgftext[x=0.100000in,y=0.388981in,left,base]{\rmfamily\fontsize{10.000000}{12.000000}\selectfont \(\displaystyle 0.0\)}%
\end{pgfscope}%
\begin{pgfscope}%
\pgfsetbuttcap%
\pgfsetroundjoin%
\definecolor{currentfill}{rgb}{0.000000,0.000000,0.000000}%
\pgfsetfillcolor{currentfill}%
\pgfsetlinewidth{0.803000pt}%
\definecolor{currentstroke}{rgb}{0.000000,0.000000,0.000000}%
\pgfsetstrokecolor{currentstroke}%
\pgfsetdash{}{0pt}%
\pgfsys@defobject{currentmarker}{\pgfqpoint{-0.048611in}{0.000000in}}{\pgfqpoint{0.000000in}{0.000000in}}{%
\pgfpathmoveto{\pgfqpoint{0.000000in}{0.000000in}}%
\pgfpathlineto{\pgfqpoint{-0.048611in}{0.000000in}}%
\pgfusepath{stroke,fill}%
}%
\begin{pgfscope}%
\pgfsys@transformshift{0.374692in}{0.740854in}%
\pgfsys@useobject{currentmarker}{}%
\end{pgfscope}%
\end{pgfscope}%
\begin{pgfscope}%
\pgftext[x=0.100000in,y=0.693026in,left,base]{\rmfamily\fontsize{10.000000}{12.000000}\selectfont \(\displaystyle 0.1\)}%
\end{pgfscope}%
\begin{pgfscope}%
\pgfsetbuttcap%
\pgfsetroundjoin%
\definecolor{currentfill}{rgb}{0.000000,0.000000,0.000000}%
\pgfsetfillcolor{currentfill}%
\pgfsetlinewidth{0.803000pt}%
\definecolor{currentstroke}{rgb}{0.000000,0.000000,0.000000}%
\pgfsetstrokecolor{currentstroke}%
\pgfsetdash{}{0pt}%
\pgfsys@defobject{currentmarker}{\pgfqpoint{-0.048611in}{0.000000in}}{\pgfqpoint{0.000000in}{0.000000in}}{%
\pgfpathmoveto{\pgfqpoint{0.000000in}{0.000000in}}%
\pgfpathlineto{\pgfqpoint{-0.048611in}{0.000000in}}%
\pgfusepath{stroke,fill}%
}%
\begin{pgfscope}%
\pgfsys@transformshift{0.374692in}{1.044899in}%
\pgfsys@useobject{currentmarker}{}%
\end{pgfscope}%
\end{pgfscope}%
\begin{pgfscope}%
\pgftext[x=0.100000in,y=0.997071in,left,base]{\rmfamily\fontsize{10.000000}{12.000000}\selectfont \(\displaystyle 0.2\)}%
\end{pgfscope}%
\begin{pgfscope}%
\pgfsetbuttcap%
\pgfsetroundjoin%
\definecolor{currentfill}{rgb}{0.000000,0.000000,0.000000}%
\pgfsetfillcolor{currentfill}%
\pgfsetlinewidth{0.803000pt}%
\definecolor{currentstroke}{rgb}{0.000000,0.000000,0.000000}%
\pgfsetstrokecolor{currentstroke}%
\pgfsetdash{}{0pt}%
\pgfsys@defobject{currentmarker}{\pgfqpoint{-0.048611in}{0.000000in}}{\pgfqpoint{0.000000in}{0.000000in}}{%
\pgfpathmoveto{\pgfqpoint{0.000000in}{0.000000in}}%
\pgfpathlineto{\pgfqpoint{-0.048611in}{0.000000in}}%
\pgfusepath{stroke,fill}%
}%
\begin{pgfscope}%
\pgfsys@transformshift{0.374692in}{1.348944in}%
\pgfsys@useobject{currentmarker}{}%
\end{pgfscope}%
\end{pgfscope}%
\begin{pgfscope}%
\pgftext[x=0.100000in,y=1.301116in,left,base]{\rmfamily\fontsize{10.000000}{12.000000}\selectfont \(\displaystyle 0.3\)}%
\end{pgfscope}%
\begin{pgfscope}%
\pgfsetbuttcap%
\pgfsetroundjoin%
\definecolor{currentfill}{rgb}{0.000000,0.000000,0.000000}%
\pgfsetfillcolor{currentfill}%
\pgfsetlinewidth{0.803000pt}%
\definecolor{currentstroke}{rgb}{0.000000,0.000000,0.000000}%
\pgfsetstrokecolor{currentstroke}%
\pgfsetdash{}{0pt}%
\pgfsys@defobject{currentmarker}{\pgfqpoint{-0.048611in}{0.000000in}}{\pgfqpoint{0.000000in}{0.000000in}}{%
\pgfpathmoveto{\pgfqpoint{0.000000in}{0.000000in}}%
\pgfpathlineto{\pgfqpoint{-0.048611in}{0.000000in}}%
\pgfusepath{stroke,fill}%
}%
\begin{pgfscope}%
\pgfsys@transformshift{0.374692in}{1.652989in}%
\pgfsys@useobject{currentmarker}{}%
\end{pgfscope}%
\end{pgfscope}%
\begin{pgfscope}%
\pgftext[x=0.100000in,y=1.605161in,left,base]{\rmfamily\fontsize{10.000000}{12.000000}\selectfont \(\displaystyle 0.4\)}%
\end{pgfscope}%
\begin{pgfscope}%
\pgfsetbuttcap%
\pgfsetroundjoin%
\definecolor{currentfill}{rgb}{0.000000,0.000000,0.000000}%
\pgfsetfillcolor{currentfill}%
\pgfsetlinewidth{0.803000pt}%
\definecolor{currentstroke}{rgb}{0.000000,0.000000,0.000000}%
\pgfsetstrokecolor{currentstroke}%
\pgfsetdash{}{0pt}%
\pgfsys@defobject{currentmarker}{\pgfqpoint{-0.048611in}{0.000000in}}{\pgfqpoint{0.000000in}{0.000000in}}{%
\pgfpathmoveto{\pgfqpoint{0.000000in}{0.000000in}}%
\pgfpathlineto{\pgfqpoint{-0.048611in}{0.000000in}}%
\pgfusepath{stroke,fill}%
}%
\begin{pgfscope}%
\pgfsys@transformshift{0.374692in}{1.957034in}%
\pgfsys@useobject{currentmarker}{}%
\end{pgfscope}%
\end{pgfscope}%
\begin{pgfscope}%
\pgftext[x=0.100000in,y=1.909207in,left,base]{\rmfamily\fontsize{10.000000}{12.000000}\selectfont \(\displaystyle 0.5\)}%
\end{pgfscope}%
\begin{pgfscope}%
\pgfsetbuttcap%
\pgfsetroundjoin%
\definecolor{currentfill}{rgb}{0.000000,0.000000,0.000000}%
\pgfsetfillcolor{currentfill}%
\pgfsetlinewidth{0.803000pt}%
\definecolor{currentstroke}{rgb}{0.000000,0.000000,0.000000}%
\pgfsetstrokecolor{currentstroke}%
\pgfsetdash{}{0pt}%
\pgfsys@defobject{currentmarker}{\pgfqpoint{-0.048611in}{0.000000in}}{\pgfqpoint{0.000000in}{0.000000in}}{%
\pgfpathmoveto{\pgfqpoint{0.000000in}{0.000000in}}%
\pgfpathlineto{\pgfqpoint{-0.048611in}{0.000000in}}%
\pgfusepath{stroke,fill}%
}%
\begin{pgfscope}%
\pgfsys@transformshift{0.374692in}{2.261079in}%
\pgfsys@useobject{currentmarker}{}%
\end{pgfscope}%
\end{pgfscope}%
\begin{pgfscope}%
\pgftext[x=0.100000in,y=2.213252in,left,base]{\rmfamily\fontsize{10.000000}{12.000000}\selectfont \(\displaystyle 0.6\)}%
\end{pgfscope}%
\begin{pgfscope}%
\pgfsetbuttcap%
\pgfsetroundjoin%
\definecolor{currentfill}{rgb}{0.000000,0.000000,0.000000}%
\pgfsetfillcolor{currentfill}%
\pgfsetlinewidth{0.803000pt}%
\definecolor{currentstroke}{rgb}{0.000000,0.000000,0.000000}%
\pgfsetstrokecolor{currentstroke}%
\pgfsetdash{}{0pt}%
\pgfsys@defobject{currentmarker}{\pgfqpoint{-0.048611in}{0.000000in}}{\pgfqpoint{0.000000in}{0.000000in}}{%
\pgfpathmoveto{\pgfqpoint{0.000000in}{0.000000in}}%
\pgfpathlineto{\pgfqpoint{-0.048611in}{0.000000in}}%
\pgfusepath{stroke,fill}%
}%
\begin{pgfscope}%
\pgfsys@transformshift{0.374692in}{2.565124in}%
\pgfsys@useobject{currentmarker}{}%
\end{pgfscope}%
\end{pgfscope}%
\begin{pgfscope}%
\pgftext[x=0.100000in,y=2.517297in,left,base]{\rmfamily\fontsize{10.000000}{12.000000}\selectfont \(\displaystyle 0.7\)}%
\end{pgfscope}%
\begin{pgfscope}%
\pgfsetbuttcap%
\pgfsetroundjoin%
\definecolor{currentfill}{rgb}{0.000000,0.000000,0.000000}%
\pgfsetfillcolor{currentfill}%
\pgfsetlinewidth{0.803000pt}%
\definecolor{currentstroke}{rgb}{0.000000,0.000000,0.000000}%
\pgfsetstrokecolor{currentstroke}%
\pgfsetdash{}{0pt}%
\pgfsys@defobject{currentmarker}{\pgfqpoint{-0.048611in}{0.000000in}}{\pgfqpoint{0.000000in}{0.000000in}}{%
\pgfpathmoveto{\pgfqpoint{0.000000in}{0.000000in}}%
\pgfpathlineto{\pgfqpoint{-0.048611in}{0.000000in}}%
\pgfusepath{stroke,fill}%
}%
\begin{pgfscope}%
\pgfsys@transformshift{0.374692in}{2.869169in}%
\pgfsys@useobject{currentmarker}{}%
\end{pgfscope}%
\end{pgfscope}%
\begin{pgfscope}%
\pgftext[x=0.100000in,y=2.821342in,left,base]{\rmfamily\fontsize{10.000000}{12.000000}\selectfont \(\displaystyle 0.8\)}%
\end{pgfscope}%
\begin{pgfscope}%
\pgfsetrectcap%
\pgfsetmiterjoin%
\pgfsetlinewidth{0.803000pt}%
\definecolor{currentstroke}{rgb}{0.000000,0.000000,0.000000}%
\pgfsetstrokecolor{currentstroke}%
\pgfsetdash{}{0pt}%
\pgfpathmoveto{\pgfqpoint{0.374692in}{0.319877in}}%
\pgfpathlineto{\pgfqpoint{0.374692in}{2.925408in}}%
\pgfusepath{stroke}%
\end{pgfscope}%
\begin{pgfscope}%
\pgfsetrectcap%
\pgfsetmiterjoin%
\pgfsetlinewidth{0.803000pt}%
\definecolor{currentstroke}{rgb}{0.000000,0.000000,0.000000}%
\pgfsetstrokecolor{currentstroke}%
\pgfsetdash{}{0pt}%
\pgfpathmoveto{\pgfqpoint{1.867946in}{0.319877in}}%
\pgfpathlineto{\pgfqpoint{1.867946in}{2.925408in}}%
\pgfusepath{stroke}%
\end{pgfscope}%
\begin{pgfscope}%
\pgfsetrectcap%
\pgfsetmiterjoin%
\pgfsetlinewidth{0.803000pt}%
\definecolor{currentstroke}{rgb}{0.000000,0.000000,0.000000}%
\pgfsetstrokecolor{currentstroke}%
\pgfsetdash{}{0pt}%
\pgfpathmoveto{\pgfqpoint{0.374692in}{0.319877in}}%
\pgfpathlineto{\pgfqpoint{1.867946in}{0.319877in}}%
\pgfusepath{stroke}%
\end{pgfscope}%
\begin{pgfscope}%
\pgfsetrectcap%
\pgfsetmiterjoin%
\pgfsetlinewidth{0.803000pt}%
\definecolor{currentstroke}{rgb}{0.000000,0.000000,0.000000}%
\pgfsetstrokecolor{currentstroke}%
\pgfsetdash{}{0pt}%
\pgfpathmoveto{\pgfqpoint{0.374692in}{2.925408in}}%
\pgfpathlineto{\pgfqpoint{1.867946in}{2.925408in}}%
\pgfusepath{stroke}%
\end{pgfscope}%
\begin{pgfscope}%
\pgfpathrectangle{\pgfqpoint{1.961274in}{0.319877in}}{\pgfqpoint{0.130277in}{2.605531in}} %
\pgfusepath{clip}%
\pgfsetbuttcap%
\pgfsetmiterjoin%
\definecolor{currentfill}{rgb}{1.000000,1.000000,1.000000}%
\pgfsetfillcolor{currentfill}%
\pgfsetlinewidth{0.010037pt}%
\definecolor{currentstroke}{rgb}{1.000000,1.000000,1.000000}%
\pgfsetstrokecolor{currentstroke}%
\pgfsetdash{}{0pt}%
\pgfpathmoveto{\pgfqpoint{1.961274in}{0.319877in}}%
\pgfpathlineto{\pgfqpoint{1.961274in}{0.330055in}}%
\pgfpathlineto{\pgfqpoint{1.961274in}{2.915230in}}%
\pgfpathlineto{\pgfqpoint{1.961274in}{2.925408in}}%
\pgfpathlineto{\pgfqpoint{2.091551in}{2.925408in}}%
\pgfpathlineto{\pgfqpoint{2.091551in}{2.915230in}}%
\pgfpathlineto{\pgfqpoint{2.091551in}{0.330055in}}%
\pgfpathlineto{\pgfqpoint{2.091551in}{0.319877in}}%
\pgfpathclose%
\pgfusepath{stroke,fill}%
\end{pgfscope}%
\begin{pgfscope}%
\pgfsys@transformshift{1.960000in}{0.320408in}%
\pgftext[left,bottom]{\pgfimage[interpolate=true,width=0.130000in,height=2.610000in]{RnnStDev_vs_dq_Ti_500K-img1.png}}%
\end{pgfscope}%
\begin{pgfscope}%
\pgfsetbuttcap%
\pgfsetroundjoin%
\definecolor{currentfill}{rgb}{0.000000,0.000000,0.000000}%
\pgfsetfillcolor{currentfill}%
\pgfsetlinewidth{0.803000pt}%
\definecolor{currentstroke}{rgb}{0.000000,0.000000,0.000000}%
\pgfsetstrokecolor{currentstroke}%
\pgfsetdash{}{0pt}%
\pgfsys@defobject{currentmarker}{\pgfqpoint{0.000000in}{0.000000in}}{\pgfqpoint{0.048611in}{0.000000in}}{%
\pgfpathmoveto{\pgfqpoint{0.000000in}{0.000000in}}%
\pgfpathlineto{\pgfqpoint{0.048611in}{0.000000in}}%
\pgfusepath{stroke,fill}%
}%
\begin{pgfscope}%
\pgfsys@transformshift{2.091551in}{0.319877in}%
\pgfsys@useobject{currentmarker}{}%
\end{pgfscope}%
\end{pgfscope}%
\begin{pgfscope}%
\pgftext[x=2.188773in,y=0.272050in,left,base]{\rmfamily\fontsize{10.000000}{12.000000}\selectfont \(\displaystyle 0.0\)}%
\end{pgfscope}%
\begin{pgfscope}%
\pgfsetbuttcap%
\pgfsetroundjoin%
\definecolor{currentfill}{rgb}{0.000000,0.000000,0.000000}%
\pgfsetfillcolor{currentfill}%
\pgfsetlinewidth{0.803000pt}%
\definecolor{currentstroke}{rgb}{0.000000,0.000000,0.000000}%
\pgfsetstrokecolor{currentstroke}%
\pgfsetdash{}{0pt}%
\pgfsys@defobject{currentmarker}{\pgfqpoint{0.000000in}{0.000000in}}{\pgfqpoint{0.048611in}{0.000000in}}{%
\pgfpathmoveto{\pgfqpoint{0.000000in}{0.000000in}}%
\pgfpathlineto{\pgfqpoint{0.048611in}{0.000000in}}%
\pgfusepath{stroke,fill}%
}%
\begin{pgfscope}%
\pgfsys@transformshift{2.091551in}{0.662710in}%
\pgfsys@useobject{currentmarker}{}%
\end{pgfscope}%
\end{pgfscope}%
\begin{pgfscope}%
\pgftext[x=2.188773in,y=0.614883in,left,base]{\rmfamily\fontsize{10.000000}{12.000000}\selectfont \(\displaystyle 2.5\)}%
\end{pgfscope}%
\begin{pgfscope}%
\pgfsetbuttcap%
\pgfsetroundjoin%
\definecolor{currentfill}{rgb}{0.000000,0.000000,0.000000}%
\pgfsetfillcolor{currentfill}%
\pgfsetlinewidth{0.803000pt}%
\definecolor{currentstroke}{rgb}{0.000000,0.000000,0.000000}%
\pgfsetstrokecolor{currentstroke}%
\pgfsetdash{}{0pt}%
\pgfsys@defobject{currentmarker}{\pgfqpoint{0.000000in}{0.000000in}}{\pgfqpoint{0.048611in}{0.000000in}}{%
\pgfpathmoveto{\pgfqpoint{0.000000in}{0.000000in}}%
\pgfpathlineto{\pgfqpoint{0.048611in}{0.000000in}}%
\pgfusepath{stroke,fill}%
}%
\begin{pgfscope}%
\pgfsys@transformshift{2.091551in}{1.005543in}%
\pgfsys@useobject{currentmarker}{}%
\end{pgfscope}%
\end{pgfscope}%
\begin{pgfscope}%
\pgftext[x=2.188773in,y=0.957716in,left,base]{\rmfamily\fontsize{10.000000}{12.000000}\selectfont \(\displaystyle 5.0\)}%
\end{pgfscope}%
\begin{pgfscope}%
\pgfsetbuttcap%
\pgfsetroundjoin%
\definecolor{currentfill}{rgb}{0.000000,0.000000,0.000000}%
\pgfsetfillcolor{currentfill}%
\pgfsetlinewidth{0.803000pt}%
\definecolor{currentstroke}{rgb}{0.000000,0.000000,0.000000}%
\pgfsetstrokecolor{currentstroke}%
\pgfsetdash{}{0pt}%
\pgfsys@defobject{currentmarker}{\pgfqpoint{0.000000in}{0.000000in}}{\pgfqpoint{0.048611in}{0.000000in}}{%
\pgfpathmoveto{\pgfqpoint{0.000000in}{0.000000in}}%
\pgfpathlineto{\pgfqpoint{0.048611in}{0.000000in}}%
\pgfusepath{stroke,fill}%
}%
\begin{pgfscope}%
\pgfsys@transformshift{2.091551in}{1.348376in}%
\pgfsys@useobject{currentmarker}{}%
\end{pgfscope}%
\end{pgfscope}%
\begin{pgfscope}%
\pgftext[x=2.188773in,y=1.300548in,left,base]{\rmfamily\fontsize{10.000000}{12.000000}\selectfont \(\displaystyle 7.5\)}%
\end{pgfscope}%
\begin{pgfscope}%
\pgfsetbuttcap%
\pgfsetroundjoin%
\definecolor{currentfill}{rgb}{0.000000,0.000000,0.000000}%
\pgfsetfillcolor{currentfill}%
\pgfsetlinewidth{0.803000pt}%
\definecolor{currentstroke}{rgb}{0.000000,0.000000,0.000000}%
\pgfsetstrokecolor{currentstroke}%
\pgfsetdash{}{0pt}%
\pgfsys@defobject{currentmarker}{\pgfqpoint{0.000000in}{0.000000in}}{\pgfqpoint{0.048611in}{0.000000in}}{%
\pgfpathmoveto{\pgfqpoint{0.000000in}{0.000000in}}%
\pgfpathlineto{\pgfqpoint{0.048611in}{0.000000in}}%
\pgfusepath{stroke,fill}%
}%
\begin{pgfscope}%
\pgfsys@transformshift{2.091551in}{1.691209in}%
\pgfsys@useobject{currentmarker}{}%
\end{pgfscope}%
\end{pgfscope}%
\begin{pgfscope}%
\pgftext[x=2.188773in,y=1.643381in,left,base]{\rmfamily\fontsize{10.000000}{12.000000}\selectfont \(\displaystyle 10.0\)}%
\end{pgfscope}%
\begin{pgfscope}%
\pgfsetbuttcap%
\pgfsetroundjoin%
\definecolor{currentfill}{rgb}{0.000000,0.000000,0.000000}%
\pgfsetfillcolor{currentfill}%
\pgfsetlinewidth{0.803000pt}%
\definecolor{currentstroke}{rgb}{0.000000,0.000000,0.000000}%
\pgfsetstrokecolor{currentstroke}%
\pgfsetdash{}{0pt}%
\pgfsys@defobject{currentmarker}{\pgfqpoint{0.000000in}{0.000000in}}{\pgfqpoint{0.048611in}{0.000000in}}{%
\pgfpathmoveto{\pgfqpoint{0.000000in}{0.000000in}}%
\pgfpathlineto{\pgfqpoint{0.048611in}{0.000000in}}%
\pgfusepath{stroke,fill}%
}%
\begin{pgfscope}%
\pgfsys@transformshift{2.091551in}{2.034042in}%
\pgfsys@useobject{currentmarker}{}%
\end{pgfscope}%
\end{pgfscope}%
\begin{pgfscope}%
\pgftext[x=2.188773in,y=1.986214in,left,base]{\rmfamily\fontsize{10.000000}{12.000000}\selectfont \(\displaystyle 12.5\)}%
\end{pgfscope}%
\begin{pgfscope}%
\pgfsetbuttcap%
\pgfsetroundjoin%
\definecolor{currentfill}{rgb}{0.000000,0.000000,0.000000}%
\pgfsetfillcolor{currentfill}%
\pgfsetlinewidth{0.803000pt}%
\definecolor{currentstroke}{rgb}{0.000000,0.000000,0.000000}%
\pgfsetstrokecolor{currentstroke}%
\pgfsetdash{}{0pt}%
\pgfsys@defobject{currentmarker}{\pgfqpoint{0.000000in}{0.000000in}}{\pgfqpoint{0.048611in}{0.000000in}}{%
\pgfpathmoveto{\pgfqpoint{0.000000in}{0.000000in}}%
\pgfpathlineto{\pgfqpoint{0.048611in}{0.000000in}}%
\pgfusepath{stroke,fill}%
}%
\begin{pgfscope}%
\pgfsys@transformshift{2.091551in}{2.376875in}%
\pgfsys@useobject{currentmarker}{}%
\end{pgfscope}%
\end{pgfscope}%
\begin{pgfscope}%
\pgftext[x=2.188773in,y=2.329047in,left,base]{\rmfamily\fontsize{10.000000}{12.000000}\selectfont \(\displaystyle 15.0\)}%
\end{pgfscope}%
\begin{pgfscope}%
\pgfsetbuttcap%
\pgfsetroundjoin%
\definecolor{currentfill}{rgb}{0.000000,0.000000,0.000000}%
\pgfsetfillcolor{currentfill}%
\pgfsetlinewidth{0.803000pt}%
\definecolor{currentstroke}{rgb}{0.000000,0.000000,0.000000}%
\pgfsetstrokecolor{currentstroke}%
\pgfsetdash{}{0pt}%
\pgfsys@defobject{currentmarker}{\pgfqpoint{0.000000in}{0.000000in}}{\pgfqpoint{0.048611in}{0.000000in}}{%
\pgfpathmoveto{\pgfqpoint{0.000000in}{0.000000in}}%
\pgfpathlineto{\pgfqpoint{0.048611in}{0.000000in}}%
\pgfusepath{stroke,fill}%
}%
\begin{pgfscope}%
\pgfsys@transformshift{2.091551in}{2.719708in}%
\pgfsys@useobject{currentmarker}{}%
\end{pgfscope}%
\end{pgfscope}%
\begin{pgfscope}%
\pgftext[x=2.188773in,y=2.671880in,left,base]{\rmfamily\fontsize{10.000000}{12.000000}\selectfont \(\displaystyle 17.5\)}%
\end{pgfscope}%
\begin{pgfscope}%
\pgfsetbuttcap%
\pgfsetmiterjoin%
\pgfsetlinewidth{0.803000pt}%
\definecolor{currentstroke}{rgb}{0.000000,0.000000,0.000000}%
\pgfsetstrokecolor{currentstroke}%
\pgfsetdash{}{0pt}%
\pgfpathmoveto{\pgfqpoint{1.961274in}{0.319877in}}%
\pgfpathlineto{\pgfqpoint{1.961274in}{0.330055in}}%
\pgfpathlineto{\pgfqpoint{1.961274in}{2.915230in}}%
\pgfpathlineto{\pgfqpoint{1.961274in}{2.925408in}}%
\pgfpathlineto{\pgfqpoint{2.091551in}{2.925408in}}%
\pgfpathlineto{\pgfqpoint{2.091551in}{2.915230in}}%
\pgfpathlineto{\pgfqpoint{2.091551in}{0.330055in}}%
\pgfpathlineto{\pgfqpoint{2.091551in}{0.319877in}}%
\pgfpathclose%
\pgfusepath{stroke}%
\end{pgfscope}%
\end{pgfpicture}%
\makeatother%
\endgroup%

    \vspace*{-0.4cm}
	\caption{500 K. Bin size $0.018e$}
	\end{subfigure}
	\quad
	\begin{subfigure}[b]{0.45\textwidth}
	\hspace*{-0.4cm}
	%% Creator: Matplotlib, PGF backend
%%
%% To include the figure in your LaTeX document, write
%%   \input{<filename>.pgf}
%%
%% Make sure the required packages are loaded in your preamble
%%   \usepackage{pgf}
%%
%% Figures using additional raster images can only be included by \input if
%% they are in the same directory as the main LaTeX file. For loading figures
%% from other directories you can use the `import` package
%%   \usepackage{import}
%% and then include the figures with
%%   \import{<path to file>}{<filename>.pgf}
%%
%% Matplotlib used the following preamble
%%   \usepackage[utf8x]{inputenc}
%%   \usepackage[T1]{fontenc}
%%
\begingroup%
\makeatletter%
\begin{pgfpicture}%
\pgfpathrectangle{\pgfpointorigin}{\pgfqpoint{2.535687in}{3.060408in}}%
\pgfusepath{use as bounding box, clip}%
\begin{pgfscope}%
\pgfsetbuttcap%
\pgfsetmiterjoin%
\definecolor{currentfill}{rgb}{1.000000,1.000000,1.000000}%
\pgfsetfillcolor{currentfill}%
\pgfsetlinewidth{0.000000pt}%
\definecolor{currentstroke}{rgb}{1.000000,1.000000,1.000000}%
\pgfsetstrokecolor{currentstroke}%
\pgfsetdash{}{0pt}%
\pgfpathmoveto{\pgfqpoint{0.000000in}{0.000000in}}%
\pgfpathlineto{\pgfqpoint{2.535687in}{0.000000in}}%
\pgfpathlineto{\pgfqpoint{2.535687in}{3.060408in}}%
\pgfpathlineto{\pgfqpoint{0.000000in}{3.060408in}}%
\pgfpathclose%
\pgfusepath{fill}%
\end{pgfscope}%
\begin{pgfscope}%
\pgfsetbuttcap%
\pgfsetmiterjoin%
\definecolor{currentfill}{rgb}{1.000000,1.000000,1.000000}%
\pgfsetfillcolor{currentfill}%
\pgfsetlinewidth{0.000000pt}%
\definecolor{currentstroke}{rgb}{0.000000,0.000000,0.000000}%
\pgfsetstrokecolor{currentstroke}%
\pgfsetstrokeopacity{0.000000}%
\pgfsetdash{}{0pt}%
\pgfpathmoveto{\pgfqpoint{0.374692in}{0.319877in}}%
\pgfpathlineto{\pgfqpoint{1.867946in}{0.319877in}}%
\pgfpathlineto{\pgfqpoint{1.867946in}{2.925408in}}%
\pgfpathlineto{\pgfqpoint{0.374692in}{2.925408in}}%
\pgfpathclose%
\pgfusepath{fill}%
\end{pgfscope}%
\begin{pgfscope}%
\pgfpathrectangle{\pgfqpoint{0.374692in}{0.319877in}}{\pgfqpoint{1.493254in}{2.605531in}} %
\pgfusepath{clip}%
\pgfsys@transformshift{0.374692in}{0.319877in}%
\pgftext[left,bottom]{\pgfimage[interpolate=true,width=1.500000in,height=2.610000in]{RnnStDev_vs_dq_Ti_1000K-img0.png}}%
\end{pgfscope}%
\begin{pgfscope}%
\pgfpathrectangle{\pgfqpoint{0.374692in}{0.319877in}}{\pgfqpoint{1.493254in}{2.605531in}} %
\pgfusepath{clip}%
\pgfsetbuttcap%
\pgfsetroundjoin%
\definecolor{currentfill}{rgb}{1.000000,0.752941,0.796078}%
\pgfsetfillcolor{currentfill}%
\pgfsetlinewidth{1.003750pt}%
\definecolor{currentstroke}{rgb}{1.000000,0.752941,0.796078}%
\pgfsetstrokecolor{currentstroke}%
\pgfsetdash{}{0pt}%
\pgfpathmoveto{\pgfqpoint{0.468020in}{1.283854in}}%
\pgfpathcurveto{\pgfqpoint{0.479070in}{1.283854in}}{\pgfqpoint{0.489669in}{1.288244in}}{\pgfqpoint{0.497483in}{1.296058in}}%
\pgfpathcurveto{\pgfqpoint{0.505297in}{1.303872in}}{\pgfqpoint{0.509687in}{1.314471in}}{\pgfqpoint{0.509687in}{1.325521in}}%
\pgfpathcurveto{\pgfqpoint{0.509687in}{1.336571in}}{\pgfqpoint{0.505297in}{1.347170in}}{\pgfqpoint{0.497483in}{1.354983in}}%
\pgfpathcurveto{\pgfqpoint{0.489669in}{1.362797in}}{\pgfqpoint{0.479070in}{1.367187in}}{\pgfqpoint{0.468020in}{1.367187in}}%
\pgfpathcurveto{\pgfqpoint{0.456970in}{1.367187in}}{\pgfqpoint{0.446371in}{1.362797in}}{\pgfqpoint{0.438558in}{1.354983in}}%
\pgfpathcurveto{\pgfqpoint{0.430744in}{1.347170in}}{\pgfqpoint{0.426354in}{1.336571in}}{\pgfqpoint{0.426354in}{1.325521in}}%
\pgfpathcurveto{\pgfqpoint{0.426354in}{1.314471in}}{\pgfqpoint{0.430744in}{1.303872in}}{\pgfqpoint{0.438558in}{1.296058in}}%
\pgfpathcurveto{\pgfqpoint{0.446371in}{1.288244in}}{\pgfqpoint{0.456970in}{1.283854in}}{\pgfqpoint{0.468020in}{1.283854in}}%
\pgfpathclose%
\pgfusepath{stroke,fill}%
\end{pgfscope}%
\begin{pgfscope}%
\pgfpathrectangle{\pgfqpoint{0.374692in}{0.319877in}}{\pgfqpoint{1.493254in}{2.605531in}} %
\pgfusepath{clip}%
\pgfsetbuttcap%
\pgfsetroundjoin%
\definecolor{currentfill}{rgb}{1.000000,0.752941,0.796078}%
\pgfsetfillcolor{currentfill}%
\pgfsetlinewidth{1.003750pt}%
\definecolor{currentstroke}{rgb}{1.000000,0.752941,0.796078}%
\pgfsetstrokecolor{currentstroke}%
\pgfsetdash{}{0pt}%
\pgfpathmoveto{\pgfqpoint{0.654677in}{1.460168in}}%
\pgfpathcurveto{\pgfqpoint{0.665727in}{1.460168in}}{\pgfqpoint{0.676326in}{1.464558in}}{\pgfqpoint{0.684140in}{1.472372in}}%
\pgfpathcurveto{\pgfqpoint{0.691953in}{1.480186in}}{\pgfqpoint{0.696344in}{1.490785in}}{\pgfqpoint{0.696344in}{1.501835in}}%
\pgfpathcurveto{\pgfqpoint{0.696344in}{1.512885in}}{\pgfqpoint{0.691953in}{1.523484in}}{\pgfqpoint{0.684140in}{1.531298in}}%
\pgfpathcurveto{\pgfqpoint{0.676326in}{1.539111in}}{\pgfqpoint{0.665727in}{1.543501in}}{\pgfqpoint{0.654677in}{1.543501in}}%
\pgfpathcurveto{\pgfqpoint{0.643627in}{1.543501in}}{\pgfqpoint{0.633028in}{1.539111in}}{\pgfqpoint{0.625214in}{1.531298in}}%
\pgfpathcurveto{\pgfqpoint{0.617401in}{1.523484in}}{\pgfqpoint{0.613010in}{1.512885in}}{\pgfqpoint{0.613010in}{1.501835in}}%
\pgfpathcurveto{\pgfqpoint{0.613010in}{1.490785in}}{\pgfqpoint{0.617401in}{1.480186in}}{\pgfqpoint{0.625214in}{1.472372in}}%
\pgfpathcurveto{\pgfqpoint{0.633028in}{1.464558in}}{\pgfqpoint{0.643627in}{1.460168in}}{\pgfqpoint{0.654677in}{1.460168in}}%
\pgfpathclose%
\pgfusepath{stroke,fill}%
\end{pgfscope}%
\begin{pgfscope}%
\pgfpathrectangle{\pgfqpoint{0.374692in}{0.319877in}}{\pgfqpoint{1.493254in}{2.605531in}} %
\pgfusepath{clip}%
\pgfsetbuttcap%
\pgfsetroundjoin%
\definecolor{currentfill}{rgb}{1.000000,0.752941,0.796078}%
\pgfsetfillcolor{currentfill}%
\pgfsetlinewidth{1.003750pt}%
\definecolor{currentstroke}{rgb}{1.000000,0.752941,0.796078}%
\pgfsetstrokecolor{currentstroke}%
\pgfsetdash{}{0pt}%
\pgfpathmoveto{\pgfqpoint{0.841334in}{1.493752in}}%
\pgfpathcurveto{\pgfqpoint{0.852384in}{1.493752in}}{\pgfqpoint{0.862983in}{1.498142in}}{\pgfqpoint{0.870797in}{1.505956in}}%
\pgfpathcurveto{\pgfqpoint{0.878610in}{1.513769in}}{\pgfqpoint{0.883000in}{1.524368in}}{\pgfqpoint{0.883000in}{1.535418in}}%
\pgfpathcurveto{\pgfqpoint{0.883000in}{1.546469in}}{\pgfqpoint{0.878610in}{1.557068in}}{\pgfqpoint{0.870797in}{1.564881in}}%
\pgfpathcurveto{\pgfqpoint{0.862983in}{1.572695in}}{\pgfqpoint{0.852384in}{1.577085in}}{\pgfqpoint{0.841334in}{1.577085in}}%
\pgfpathcurveto{\pgfqpoint{0.830284in}{1.577085in}}{\pgfqpoint{0.819685in}{1.572695in}}{\pgfqpoint{0.811871in}{1.564881in}}%
\pgfpathcurveto{\pgfqpoint{0.804057in}{1.557068in}}{\pgfqpoint{0.799667in}{1.546469in}}{\pgfqpoint{0.799667in}{1.535418in}}%
\pgfpathcurveto{\pgfqpoint{0.799667in}{1.524368in}}{\pgfqpoint{0.804057in}{1.513769in}}{\pgfqpoint{0.811871in}{1.505956in}}%
\pgfpathcurveto{\pgfqpoint{0.819685in}{1.498142in}}{\pgfqpoint{0.830284in}{1.493752in}}{\pgfqpoint{0.841334in}{1.493752in}}%
\pgfpathclose%
\pgfusepath{stroke,fill}%
\end{pgfscope}%
\begin{pgfscope}%
\pgfpathrectangle{\pgfqpoint{0.374692in}{0.319877in}}{\pgfqpoint{1.493254in}{2.605531in}} %
\pgfusepath{clip}%
\pgfsetbuttcap%
\pgfsetroundjoin%
\definecolor{currentfill}{rgb}{1.000000,0.752941,0.796078}%
\pgfsetfillcolor{currentfill}%
\pgfsetlinewidth{1.003750pt}%
\definecolor{currentstroke}{rgb}{1.000000,0.752941,0.796078}%
\pgfsetstrokecolor{currentstroke}%
\pgfsetdash{}{0pt}%
\pgfpathmoveto{\pgfqpoint{1.027990in}{1.445317in}}%
\pgfpathcurveto{\pgfqpoint{1.039041in}{1.445317in}}{\pgfqpoint{1.049640in}{1.449708in}}{\pgfqpoint{1.057453in}{1.457521in}}%
\pgfpathcurveto{\pgfqpoint{1.065267in}{1.465335in}}{\pgfqpoint{1.069657in}{1.475934in}}{\pgfqpoint{1.069657in}{1.486984in}}%
\pgfpathcurveto{\pgfqpoint{1.069657in}{1.498034in}}{\pgfqpoint{1.065267in}{1.508633in}}{\pgfqpoint{1.057453in}{1.516447in}}%
\pgfpathcurveto{\pgfqpoint{1.049640in}{1.524260in}}{\pgfqpoint{1.039041in}{1.528651in}}{\pgfqpoint{1.027990in}{1.528651in}}%
\pgfpathcurveto{\pgfqpoint{1.016940in}{1.528651in}}{\pgfqpoint{1.006341in}{1.524260in}}{\pgfqpoint{0.998528in}{1.516447in}}%
\pgfpathcurveto{\pgfqpoint{0.990714in}{1.508633in}}{\pgfqpoint{0.986324in}{1.498034in}}{\pgfqpoint{0.986324in}{1.486984in}}%
\pgfpathcurveto{\pgfqpoint{0.986324in}{1.475934in}}{\pgfqpoint{0.990714in}{1.465335in}}{\pgfqpoint{0.998528in}{1.457521in}}%
\pgfpathcurveto{\pgfqpoint{1.006341in}{1.449708in}}{\pgfqpoint{1.016940in}{1.445317in}}{\pgfqpoint{1.027990in}{1.445317in}}%
\pgfpathclose%
\pgfusepath{stroke,fill}%
\end{pgfscope}%
\begin{pgfscope}%
\pgfpathrectangle{\pgfqpoint{0.374692in}{0.319877in}}{\pgfqpoint{1.493254in}{2.605531in}} %
\pgfusepath{clip}%
\pgfsetbuttcap%
\pgfsetroundjoin%
\definecolor{currentfill}{rgb}{1.000000,0.752941,0.796078}%
\pgfsetfillcolor{currentfill}%
\pgfsetlinewidth{1.003750pt}%
\definecolor{currentstroke}{rgb}{1.000000,0.752941,0.796078}%
\pgfsetstrokecolor{currentstroke}%
\pgfsetdash{}{0pt}%
\pgfpathmoveto{\pgfqpoint{1.214647in}{1.446249in}}%
\pgfpathcurveto{\pgfqpoint{1.225697in}{1.446249in}}{\pgfqpoint{1.236296in}{1.450639in}}{\pgfqpoint{1.244110in}{1.458452in}}%
\pgfpathcurveto{\pgfqpoint{1.251924in}{1.466266in}}{\pgfqpoint{1.256314in}{1.476865in}}{\pgfqpoint{1.256314in}{1.487915in}}%
\pgfpathcurveto{\pgfqpoint{1.256314in}{1.498965in}}{\pgfqpoint{1.251924in}{1.509564in}}{\pgfqpoint{1.244110in}{1.517378in}}%
\pgfpathcurveto{\pgfqpoint{1.236296in}{1.525192in}}{\pgfqpoint{1.225697in}{1.529582in}}{\pgfqpoint{1.214647in}{1.529582in}}%
\pgfpathcurveto{\pgfqpoint{1.203597in}{1.529582in}}{\pgfqpoint{1.192998in}{1.525192in}}{\pgfqpoint{1.185184in}{1.517378in}}%
\pgfpathcurveto{\pgfqpoint{1.177371in}{1.509564in}}{\pgfqpoint{1.172981in}{1.498965in}}{\pgfqpoint{1.172981in}{1.487915in}}%
\pgfpathcurveto{\pgfqpoint{1.172981in}{1.476865in}}{\pgfqpoint{1.177371in}{1.466266in}}{\pgfqpoint{1.185184in}{1.458452in}}%
\pgfpathcurveto{\pgfqpoint{1.192998in}{1.450639in}}{\pgfqpoint{1.203597in}{1.446249in}}{\pgfqpoint{1.214647in}{1.446249in}}%
\pgfpathclose%
\pgfusepath{stroke,fill}%
\end{pgfscope}%
\begin{pgfscope}%
\pgfpathrectangle{\pgfqpoint{0.374692in}{0.319877in}}{\pgfqpoint{1.493254in}{2.605531in}} %
\pgfusepath{clip}%
\pgfsetbuttcap%
\pgfsetroundjoin%
\definecolor{currentfill}{rgb}{1.000000,0.752941,0.796078}%
\pgfsetfillcolor{currentfill}%
\pgfsetlinewidth{1.003750pt}%
\definecolor{currentstroke}{rgb}{1.000000,0.752941,0.796078}%
\pgfsetstrokecolor{currentstroke}%
\pgfsetdash{}{0pt}%
\pgfpathmoveto{\pgfqpoint{1.401304in}{1.519682in}}%
\pgfpathcurveto{\pgfqpoint{1.412354in}{1.519682in}}{\pgfqpoint{1.422953in}{1.524072in}}{\pgfqpoint{1.430767in}{1.531885in}}%
\pgfpathcurveto{\pgfqpoint{1.438580in}{1.539699in}}{\pgfqpoint{1.442971in}{1.550298in}}{\pgfqpoint{1.442971in}{1.561348in}}%
\pgfpathcurveto{\pgfqpoint{1.442971in}{1.572398in}}{\pgfqpoint{1.438580in}{1.582997in}}{\pgfqpoint{1.430767in}{1.590811in}}%
\pgfpathcurveto{\pgfqpoint{1.422953in}{1.598625in}}{\pgfqpoint{1.412354in}{1.603015in}}{\pgfqpoint{1.401304in}{1.603015in}}%
\pgfpathcurveto{\pgfqpoint{1.390254in}{1.603015in}}{\pgfqpoint{1.379655in}{1.598625in}}{\pgfqpoint{1.371841in}{1.590811in}}%
\pgfpathcurveto{\pgfqpoint{1.364028in}{1.582997in}}{\pgfqpoint{1.359637in}{1.572398in}}{\pgfqpoint{1.359637in}{1.561348in}}%
\pgfpathcurveto{\pgfqpoint{1.359637in}{1.550298in}}{\pgfqpoint{1.364028in}{1.539699in}}{\pgfqpoint{1.371841in}{1.531885in}}%
\pgfpathcurveto{\pgfqpoint{1.379655in}{1.524072in}}{\pgfqpoint{1.390254in}{1.519682in}}{\pgfqpoint{1.401304in}{1.519682in}}%
\pgfpathclose%
\pgfusepath{stroke,fill}%
\end{pgfscope}%
\begin{pgfscope}%
\pgfpathrectangle{\pgfqpoint{0.374692in}{0.319877in}}{\pgfqpoint{1.493254in}{2.605531in}} %
\pgfusepath{clip}%
\pgfsetbuttcap%
\pgfsetroundjoin%
\definecolor{currentfill}{rgb}{1.000000,0.752941,0.796078}%
\pgfsetfillcolor{currentfill}%
\pgfsetlinewidth{1.003750pt}%
\definecolor{currentstroke}{rgb}{1.000000,0.752941,0.796078}%
\pgfsetstrokecolor{currentstroke}%
\pgfsetdash{}{0pt}%
\pgfpathmoveto{\pgfqpoint{1.587961in}{1.592404in}}%
\pgfpathcurveto{\pgfqpoint{1.599011in}{1.592404in}}{\pgfqpoint{1.609610in}{1.596794in}}{\pgfqpoint{1.617423in}{1.604608in}}%
\pgfpathcurveto{\pgfqpoint{1.625237in}{1.612421in}}{\pgfqpoint{1.629627in}{1.623020in}}{\pgfqpoint{1.629627in}{1.634070in}}%
\pgfpathcurveto{\pgfqpoint{1.629627in}{1.645121in}}{\pgfqpoint{1.625237in}{1.655720in}}{\pgfqpoint{1.617423in}{1.663533in}}%
\pgfpathcurveto{\pgfqpoint{1.609610in}{1.671347in}}{\pgfqpoint{1.599011in}{1.675737in}}{\pgfqpoint{1.587961in}{1.675737in}}%
\pgfpathcurveto{\pgfqpoint{1.576911in}{1.675737in}}{\pgfqpoint{1.566311in}{1.671347in}}{\pgfqpoint{1.558498in}{1.663533in}}%
\pgfpathcurveto{\pgfqpoint{1.550684in}{1.655720in}}{\pgfqpoint{1.546294in}{1.645121in}}{\pgfqpoint{1.546294in}{1.634070in}}%
\pgfpathcurveto{\pgfqpoint{1.546294in}{1.623020in}}{\pgfqpoint{1.550684in}{1.612421in}}{\pgfqpoint{1.558498in}{1.604608in}}%
\pgfpathcurveto{\pgfqpoint{1.566311in}{1.596794in}}{\pgfqpoint{1.576911in}{1.592404in}}{\pgfqpoint{1.587961in}{1.592404in}}%
\pgfpathclose%
\pgfusepath{stroke,fill}%
\end{pgfscope}%
\begin{pgfscope}%
\pgfpathrectangle{\pgfqpoint{0.374692in}{0.319877in}}{\pgfqpoint{1.493254in}{2.605531in}} %
\pgfusepath{clip}%
\pgfsetbuttcap%
\pgfsetroundjoin%
\definecolor{currentfill}{rgb}{1.000000,0.752941,0.796078}%
\pgfsetfillcolor{currentfill}%
\pgfsetlinewidth{1.003750pt}%
\definecolor{currentstroke}{rgb}{1.000000,0.752941,0.796078}%
\pgfsetstrokecolor{currentstroke}%
\pgfsetdash{}{0pt}%
\pgfpathmoveto{\pgfqpoint{1.774617in}{1.695254in}}%
\pgfpathcurveto{\pgfqpoint{1.785668in}{1.695254in}}{\pgfqpoint{1.796267in}{1.699644in}}{\pgfqpoint{1.804080in}{1.707457in}}%
\pgfpathcurveto{\pgfqpoint{1.811894in}{1.715271in}}{\pgfqpoint{1.816284in}{1.725870in}}{\pgfqpoint{1.816284in}{1.736920in}}%
\pgfpathcurveto{\pgfqpoint{1.816284in}{1.747970in}}{\pgfqpoint{1.811894in}{1.758569in}}{\pgfqpoint{1.804080in}{1.766383in}}%
\pgfpathcurveto{\pgfqpoint{1.796267in}{1.774197in}}{\pgfqpoint{1.785668in}{1.778587in}}{\pgfqpoint{1.774617in}{1.778587in}}%
\pgfpathcurveto{\pgfqpoint{1.763567in}{1.778587in}}{\pgfqpoint{1.752968in}{1.774197in}}{\pgfqpoint{1.745155in}{1.766383in}}%
\pgfpathcurveto{\pgfqpoint{1.737341in}{1.758569in}}{\pgfqpoint{1.732951in}{1.747970in}}{\pgfqpoint{1.732951in}{1.736920in}}%
\pgfpathcurveto{\pgfqpoint{1.732951in}{1.725870in}}{\pgfqpoint{1.737341in}{1.715271in}}{\pgfqpoint{1.745155in}{1.707457in}}%
\pgfpathcurveto{\pgfqpoint{1.752968in}{1.699644in}}{\pgfqpoint{1.763567in}{1.695254in}}{\pgfqpoint{1.774617in}{1.695254in}}%
\pgfpathclose%
\pgfusepath{stroke,fill}%
\end{pgfscope}%
\begin{pgfscope}%
\pgfsetbuttcap%
\pgfsetroundjoin%
\definecolor{currentfill}{rgb}{0.000000,0.000000,0.000000}%
\pgfsetfillcolor{currentfill}%
\pgfsetlinewidth{0.803000pt}%
\definecolor{currentstroke}{rgb}{0.000000,0.000000,0.000000}%
\pgfsetstrokecolor{currentstroke}%
\pgfsetdash{}{0pt}%
\pgfsys@defobject{currentmarker}{\pgfqpoint{0.000000in}{-0.048611in}}{\pgfqpoint{0.000000in}{0.000000in}}{%
\pgfpathmoveto{\pgfqpoint{0.000000in}{0.000000in}}%
\pgfpathlineto{\pgfqpoint{0.000000in}{-0.048611in}}%
\pgfusepath{stroke,fill}%
}%
\begin{pgfscope}%
\pgfsys@transformshift{0.654677in}{0.319877in}%
\pgfsys@useobject{currentmarker}{}%
\end{pgfscope}%
\end{pgfscope}%
\begin{pgfscope}%
\pgftext[x=0.654677in,y=0.222655in,,top]{\rmfamily\fontsize{10.000000}{12.000000}\selectfont \(\displaystyle -0.05\)}%
\end{pgfscope}%
\begin{pgfscope}%
\pgfsetbuttcap%
\pgfsetroundjoin%
\definecolor{currentfill}{rgb}{0.000000,0.000000,0.000000}%
\pgfsetfillcolor{currentfill}%
\pgfsetlinewidth{0.803000pt}%
\definecolor{currentstroke}{rgb}{0.000000,0.000000,0.000000}%
\pgfsetstrokecolor{currentstroke}%
\pgfsetdash{}{0pt}%
\pgfsys@defobject{currentmarker}{\pgfqpoint{0.000000in}{-0.048611in}}{\pgfqpoint{0.000000in}{0.000000in}}{%
\pgfpathmoveto{\pgfqpoint{0.000000in}{0.000000in}}%
\pgfpathlineto{\pgfqpoint{0.000000in}{-0.048611in}}%
\pgfusepath{stroke,fill}%
}%
\begin{pgfscope}%
\pgfsys@transformshift{1.121319in}{0.319877in}%
\pgfsys@useobject{currentmarker}{}%
\end{pgfscope}%
\end{pgfscope}%
\begin{pgfscope}%
\pgftext[x=1.121319in,y=0.222655in,,top]{\rmfamily\fontsize{10.000000}{12.000000}\selectfont \(\displaystyle 0.00\)}%
\end{pgfscope}%
\begin{pgfscope}%
\pgfsetbuttcap%
\pgfsetroundjoin%
\definecolor{currentfill}{rgb}{0.000000,0.000000,0.000000}%
\pgfsetfillcolor{currentfill}%
\pgfsetlinewidth{0.803000pt}%
\definecolor{currentstroke}{rgb}{0.000000,0.000000,0.000000}%
\pgfsetstrokecolor{currentstroke}%
\pgfsetdash{}{0pt}%
\pgfsys@defobject{currentmarker}{\pgfqpoint{0.000000in}{-0.048611in}}{\pgfqpoint{0.000000in}{0.000000in}}{%
\pgfpathmoveto{\pgfqpoint{0.000000in}{0.000000in}}%
\pgfpathlineto{\pgfqpoint{0.000000in}{-0.048611in}}%
\pgfusepath{stroke,fill}%
}%
\begin{pgfscope}%
\pgfsys@transformshift{1.587961in}{0.319877in}%
\pgfsys@useobject{currentmarker}{}%
\end{pgfscope}%
\end{pgfscope}%
\begin{pgfscope}%
\pgftext[x=1.587961in,y=0.222655in,,top]{\rmfamily\fontsize{10.000000}{12.000000}\selectfont \(\displaystyle 0.05\)}%
\end{pgfscope}%
\begin{pgfscope}%
\pgfsetbuttcap%
\pgfsetroundjoin%
\definecolor{currentfill}{rgb}{0.000000,0.000000,0.000000}%
\pgfsetfillcolor{currentfill}%
\pgfsetlinewidth{0.803000pt}%
\definecolor{currentstroke}{rgb}{0.000000,0.000000,0.000000}%
\pgfsetstrokecolor{currentstroke}%
\pgfsetdash{}{0pt}%
\pgfsys@defobject{currentmarker}{\pgfqpoint{-0.048611in}{0.000000in}}{\pgfqpoint{0.000000in}{0.000000in}}{%
\pgfpathmoveto{\pgfqpoint{0.000000in}{0.000000in}}%
\pgfpathlineto{\pgfqpoint{-0.048611in}{0.000000in}}%
\pgfusepath{stroke,fill}%
}%
\begin{pgfscope}%
\pgfsys@transformshift{0.374692in}{0.436809in}%
\pgfsys@useobject{currentmarker}{}%
\end{pgfscope}%
\end{pgfscope}%
\begin{pgfscope}%
\pgftext[x=0.100000in,y=0.388981in,left,base]{\rmfamily\fontsize{10.000000}{12.000000}\selectfont \(\displaystyle 0.0\)}%
\end{pgfscope}%
\begin{pgfscope}%
\pgfsetbuttcap%
\pgfsetroundjoin%
\definecolor{currentfill}{rgb}{0.000000,0.000000,0.000000}%
\pgfsetfillcolor{currentfill}%
\pgfsetlinewidth{0.803000pt}%
\definecolor{currentstroke}{rgb}{0.000000,0.000000,0.000000}%
\pgfsetstrokecolor{currentstroke}%
\pgfsetdash{}{0pt}%
\pgfsys@defobject{currentmarker}{\pgfqpoint{-0.048611in}{0.000000in}}{\pgfqpoint{0.000000in}{0.000000in}}{%
\pgfpathmoveto{\pgfqpoint{0.000000in}{0.000000in}}%
\pgfpathlineto{\pgfqpoint{-0.048611in}{0.000000in}}%
\pgfusepath{stroke,fill}%
}%
\begin{pgfscope}%
\pgfsys@transformshift{0.374692in}{0.740854in}%
\pgfsys@useobject{currentmarker}{}%
\end{pgfscope}%
\end{pgfscope}%
\begin{pgfscope}%
\pgftext[x=0.100000in,y=0.693026in,left,base]{\rmfamily\fontsize{10.000000}{12.000000}\selectfont \(\displaystyle 0.1\)}%
\end{pgfscope}%
\begin{pgfscope}%
\pgfsetbuttcap%
\pgfsetroundjoin%
\definecolor{currentfill}{rgb}{0.000000,0.000000,0.000000}%
\pgfsetfillcolor{currentfill}%
\pgfsetlinewidth{0.803000pt}%
\definecolor{currentstroke}{rgb}{0.000000,0.000000,0.000000}%
\pgfsetstrokecolor{currentstroke}%
\pgfsetdash{}{0pt}%
\pgfsys@defobject{currentmarker}{\pgfqpoint{-0.048611in}{0.000000in}}{\pgfqpoint{0.000000in}{0.000000in}}{%
\pgfpathmoveto{\pgfqpoint{0.000000in}{0.000000in}}%
\pgfpathlineto{\pgfqpoint{-0.048611in}{0.000000in}}%
\pgfusepath{stroke,fill}%
}%
\begin{pgfscope}%
\pgfsys@transformshift{0.374692in}{1.044899in}%
\pgfsys@useobject{currentmarker}{}%
\end{pgfscope}%
\end{pgfscope}%
\begin{pgfscope}%
\pgftext[x=0.100000in,y=0.997071in,left,base]{\rmfamily\fontsize{10.000000}{12.000000}\selectfont \(\displaystyle 0.2\)}%
\end{pgfscope}%
\begin{pgfscope}%
\pgfsetbuttcap%
\pgfsetroundjoin%
\definecolor{currentfill}{rgb}{0.000000,0.000000,0.000000}%
\pgfsetfillcolor{currentfill}%
\pgfsetlinewidth{0.803000pt}%
\definecolor{currentstroke}{rgb}{0.000000,0.000000,0.000000}%
\pgfsetstrokecolor{currentstroke}%
\pgfsetdash{}{0pt}%
\pgfsys@defobject{currentmarker}{\pgfqpoint{-0.048611in}{0.000000in}}{\pgfqpoint{0.000000in}{0.000000in}}{%
\pgfpathmoveto{\pgfqpoint{0.000000in}{0.000000in}}%
\pgfpathlineto{\pgfqpoint{-0.048611in}{0.000000in}}%
\pgfusepath{stroke,fill}%
}%
\begin{pgfscope}%
\pgfsys@transformshift{0.374692in}{1.348944in}%
\pgfsys@useobject{currentmarker}{}%
\end{pgfscope}%
\end{pgfscope}%
\begin{pgfscope}%
\pgftext[x=0.100000in,y=1.301116in,left,base]{\rmfamily\fontsize{10.000000}{12.000000}\selectfont \(\displaystyle 0.3\)}%
\end{pgfscope}%
\begin{pgfscope}%
\pgfsetbuttcap%
\pgfsetroundjoin%
\definecolor{currentfill}{rgb}{0.000000,0.000000,0.000000}%
\pgfsetfillcolor{currentfill}%
\pgfsetlinewidth{0.803000pt}%
\definecolor{currentstroke}{rgb}{0.000000,0.000000,0.000000}%
\pgfsetstrokecolor{currentstroke}%
\pgfsetdash{}{0pt}%
\pgfsys@defobject{currentmarker}{\pgfqpoint{-0.048611in}{0.000000in}}{\pgfqpoint{0.000000in}{0.000000in}}{%
\pgfpathmoveto{\pgfqpoint{0.000000in}{0.000000in}}%
\pgfpathlineto{\pgfqpoint{-0.048611in}{0.000000in}}%
\pgfusepath{stroke,fill}%
}%
\begin{pgfscope}%
\pgfsys@transformshift{0.374692in}{1.652989in}%
\pgfsys@useobject{currentmarker}{}%
\end{pgfscope}%
\end{pgfscope}%
\begin{pgfscope}%
\pgftext[x=0.100000in,y=1.605161in,left,base]{\rmfamily\fontsize{10.000000}{12.000000}\selectfont \(\displaystyle 0.4\)}%
\end{pgfscope}%
\begin{pgfscope}%
\pgfsetbuttcap%
\pgfsetroundjoin%
\definecolor{currentfill}{rgb}{0.000000,0.000000,0.000000}%
\pgfsetfillcolor{currentfill}%
\pgfsetlinewidth{0.803000pt}%
\definecolor{currentstroke}{rgb}{0.000000,0.000000,0.000000}%
\pgfsetstrokecolor{currentstroke}%
\pgfsetdash{}{0pt}%
\pgfsys@defobject{currentmarker}{\pgfqpoint{-0.048611in}{0.000000in}}{\pgfqpoint{0.000000in}{0.000000in}}{%
\pgfpathmoveto{\pgfqpoint{0.000000in}{0.000000in}}%
\pgfpathlineto{\pgfqpoint{-0.048611in}{0.000000in}}%
\pgfusepath{stroke,fill}%
}%
\begin{pgfscope}%
\pgfsys@transformshift{0.374692in}{1.957034in}%
\pgfsys@useobject{currentmarker}{}%
\end{pgfscope}%
\end{pgfscope}%
\begin{pgfscope}%
\pgftext[x=0.100000in,y=1.909207in,left,base]{\rmfamily\fontsize{10.000000}{12.000000}\selectfont \(\displaystyle 0.5\)}%
\end{pgfscope}%
\begin{pgfscope}%
\pgfsetbuttcap%
\pgfsetroundjoin%
\definecolor{currentfill}{rgb}{0.000000,0.000000,0.000000}%
\pgfsetfillcolor{currentfill}%
\pgfsetlinewidth{0.803000pt}%
\definecolor{currentstroke}{rgb}{0.000000,0.000000,0.000000}%
\pgfsetstrokecolor{currentstroke}%
\pgfsetdash{}{0pt}%
\pgfsys@defobject{currentmarker}{\pgfqpoint{-0.048611in}{0.000000in}}{\pgfqpoint{0.000000in}{0.000000in}}{%
\pgfpathmoveto{\pgfqpoint{0.000000in}{0.000000in}}%
\pgfpathlineto{\pgfqpoint{-0.048611in}{0.000000in}}%
\pgfusepath{stroke,fill}%
}%
\begin{pgfscope}%
\pgfsys@transformshift{0.374692in}{2.261079in}%
\pgfsys@useobject{currentmarker}{}%
\end{pgfscope}%
\end{pgfscope}%
\begin{pgfscope}%
\pgftext[x=0.100000in,y=2.213252in,left,base]{\rmfamily\fontsize{10.000000}{12.000000}\selectfont \(\displaystyle 0.6\)}%
\end{pgfscope}%
\begin{pgfscope}%
\pgfsetbuttcap%
\pgfsetroundjoin%
\definecolor{currentfill}{rgb}{0.000000,0.000000,0.000000}%
\pgfsetfillcolor{currentfill}%
\pgfsetlinewidth{0.803000pt}%
\definecolor{currentstroke}{rgb}{0.000000,0.000000,0.000000}%
\pgfsetstrokecolor{currentstroke}%
\pgfsetdash{}{0pt}%
\pgfsys@defobject{currentmarker}{\pgfqpoint{-0.048611in}{0.000000in}}{\pgfqpoint{0.000000in}{0.000000in}}{%
\pgfpathmoveto{\pgfqpoint{0.000000in}{0.000000in}}%
\pgfpathlineto{\pgfqpoint{-0.048611in}{0.000000in}}%
\pgfusepath{stroke,fill}%
}%
\begin{pgfscope}%
\pgfsys@transformshift{0.374692in}{2.565124in}%
\pgfsys@useobject{currentmarker}{}%
\end{pgfscope}%
\end{pgfscope}%
\begin{pgfscope}%
\pgftext[x=0.100000in,y=2.517297in,left,base]{\rmfamily\fontsize{10.000000}{12.000000}\selectfont \(\displaystyle 0.7\)}%
\end{pgfscope}%
\begin{pgfscope}%
\pgfsetbuttcap%
\pgfsetroundjoin%
\definecolor{currentfill}{rgb}{0.000000,0.000000,0.000000}%
\pgfsetfillcolor{currentfill}%
\pgfsetlinewidth{0.803000pt}%
\definecolor{currentstroke}{rgb}{0.000000,0.000000,0.000000}%
\pgfsetstrokecolor{currentstroke}%
\pgfsetdash{}{0pt}%
\pgfsys@defobject{currentmarker}{\pgfqpoint{-0.048611in}{0.000000in}}{\pgfqpoint{0.000000in}{0.000000in}}{%
\pgfpathmoveto{\pgfqpoint{0.000000in}{0.000000in}}%
\pgfpathlineto{\pgfqpoint{-0.048611in}{0.000000in}}%
\pgfusepath{stroke,fill}%
}%
\begin{pgfscope}%
\pgfsys@transformshift{0.374692in}{2.869169in}%
\pgfsys@useobject{currentmarker}{}%
\end{pgfscope}%
\end{pgfscope}%
\begin{pgfscope}%
\pgftext[x=0.100000in,y=2.821342in,left,base]{\rmfamily\fontsize{10.000000}{12.000000}\selectfont \(\displaystyle 0.8\)}%
\end{pgfscope}%
\begin{pgfscope}%
\pgfsetrectcap%
\pgfsetmiterjoin%
\pgfsetlinewidth{0.803000pt}%
\definecolor{currentstroke}{rgb}{0.000000,0.000000,0.000000}%
\pgfsetstrokecolor{currentstroke}%
\pgfsetdash{}{0pt}%
\pgfpathmoveto{\pgfqpoint{0.374692in}{0.319877in}}%
\pgfpathlineto{\pgfqpoint{0.374692in}{2.925408in}}%
\pgfusepath{stroke}%
\end{pgfscope}%
\begin{pgfscope}%
\pgfsetrectcap%
\pgfsetmiterjoin%
\pgfsetlinewidth{0.803000pt}%
\definecolor{currentstroke}{rgb}{0.000000,0.000000,0.000000}%
\pgfsetstrokecolor{currentstroke}%
\pgfsetdash{}{0pt}%
\pgfpathmoveto{\pgfqpoint{1.867946in}{0.319877in}}%
\pgfpathlineto{\pgfqpoint{1.867946in}{2.925408in}}%
\pgfusepath{stroke}%
\end{pgfscope}%
\begin{pgfscope}%
\pgfsetrectcap%
\pgfsetmiterjoin%
\pgfsetlinewidth{0.803000pt}%
\definecolor{currentstroke}{rgb}{0.000000,0.000000,0.000000}%
\pgfsetstrokecolor{currentstroke}%
\pgfsetdash{}{0pt}%
\pgfpathmoveto{\pgfqpoint{0.374692in}{0.319877in}}%
\pgfpathlineto{\pgfqpoint{1.867946in}{0.319877in}}%
\pgfusepath{stroke}%
\end{pgfscope}%
\begin{pgfscope}%
\pgfsetrectcap%
\pgfsetmiterjoin%
\pgfsetlinewidth{0.803000pt}%
\definecolor{currentstroke}{rgb}{0.000000,0.000000,0.000000}%
\pgfsetstrokecolor{currentstroke}%
\pgfsetdash{}{0pt}%
\pgfpathmoveto{\pgfqpoint{0.374692in}{2.925408in}}%
\pgfpathlineto{\pgfqpoint{1.867946in}{2.925408in}}%
\pgfusepath{stroke}%
\end{pgfscope}%
\begin{pgfscope}%
\pgfpathrectangle{\pgfqpoint{1.961274in}{0.319877in}}{\pgfqpoint{0.130277in}{2.605531in}} %
\pgfusepath{clip}%
\pgfsetbuttcap%
\pgfsetmiterjoin%
\definecolor{currentfill}{rgb}{1.000000,1.000000,1.000000}%
\pgfsetfillcolor{currentfill}%
\pgfsetlinewidth{0.010037pt}%
\definecolor{currentstroke}{rgb}{1.000000,1.000000,1.000000}%
\pgfsetstrokecolor{currentstroke}%
\pgfsetdash{}{0pt}%
\pgfpathmoveto{\pgfqpoint{1.961274in}{0.319877in}}%
\pgfpathlineto{\pgfqpoint{1.961274in}{0.330055in}}%
\pgfpathlineto{\pgfqpoint{1.961274in}{2.915230in}}%
\pgfpathlineto{\pgfqpoint{1.961274in}{2.925408in}}%
\pgfpathlineto{\pgfqpoint{2.091551in}{2.925408in}}%
\pgfpathlineto{\pgfqpoint{2.091551in}{2.915230in}}%
\pgfpathlineto{\pgfqpoint{2.091551in}{0.330055in}}%
\pgfpathlineto{\pgfqpoint{2.091551in}{0.319877in}}%
\pgfpathclose%
\pgfusepath{stroke,fill}%
\end{pgfscope}%
\begin{pgfscope}%
\pgfsys@transformshift{1.960000in}{0.320408in}%
\pgftext[left,bottom]{\pgfimage[interpolate=true,width=0.130000in,height=2.610000in]{RnnStDev_vs_dq_Ti_1000K-img1.png}}%
\end{pgfscope}%
\begin{pgfscope}%
\pgfsetbuttcap%
\pgfsetroundjoin%
\definecolor{currentfill}{rgb}{0.000000,0.000000,0.000000}%
\pgfsetfillcolor{currentfill}%
\pgfsetlinewidth{0.803000pt}%
\definecolor{currentstroke}{rgb}{0.000000,0.000000,0.000000}%
\pgfsetstrokecolor{currentstroke}%
\pgfsetdash{}{0pt}%
\pgfsys@defobject{currentmarker}{\pgfqpoint{0.000000in}{0.000000in}}{\pgfqpoint{0.048611in}{0.000000in}}{%
\pgfpathmoveto{\pgfqpoint{0.000000in}{0.000000in}}%
\pgfpathlineto{\pgfqpoint{0.048611in}{0.000000in}}%
\pgfusepath{stroke,fill}%
}%
\begin{pgfscope}%
\pgfsys@transformshift{2.091551in}{0.319877in}%
\pgfsys@useobject{currentmarker}{}%
\end{pgfscope}%
\end{pgfscope}%
\begin{pgfscope}%
\pgftext[x=2.188773in,y=0.272050in,left,base]{\rmfamily\fontsize{10.000000}{12.000000}\selectfont \(\displaystyle 0.0\)}%
\end{pgfscope}%
\begin{pgfscope}%
\pgfsetbuttcap%
\pgfsetroundjoin%
\definecolor{currentfill}{rgb}{0.000000,0.000000,0.000000}%
\pgfsetfillcolor{currentfill}%
\pgfsetlinewidth{0.803000pt}%
\definecolor{currentstroke}{rgb}{0.000000,0.000000,0.000000}%
\pgfsetstrokecolor{currentstroke}%
\pgfsetdash{}{0pt}%
\pgfsys@defobject{currentmarker}{\pgfqpoint{0.000000in}{0.000000in}}{\pgfqpoint{0.048611in}{0.000000in}}{%
\pgfpathmoveto{\pgfqpoint{0.000000in}{0.000000in}}%
\pgfpathlineto{\pgfqpoint{0.048611in}{0.000000in}}%
\pgfusepath{stroke,fill}%
}%
\begin{pgfscope}%
\pgfsys@transformshift{2.091551in}{0.662710in}%
\pgfsys@useobject{currentmarker}{}%
\end{pgfscope}%
\end{pgfscope}%
\begin{pgfscope}%
\pgftext[x=2.188773in,y=0.614883in,left,base]{\rmfamily\fontsize{10.000000}{12.000000}\selectfont \(\displaystyle 2.5\)}%
\end{pgfscope}%
\begin{pgfscope}%
\pgfsetbuttcap%
\pgfsetroundjoin%
\definecolor{currentfill}{rgb}{0.000000,0.000000,0.000000}%
\pgfsetfillcolor{currentfill}%
\pgfsetlinewidth{0.803000pt}%
\definecolor{currentstroke}{rgb}{0.000000,0.000000,0.000000}%
\pgfsetstrokecolor{currentstroke}%
\pgfsetdash{}{0pt}%
\pgfsys@defobject{currentmarker}{\pgfqpoint{0.000000in}{0.000000in}}{\pgfqpoint{0.048611in}{0.000000in}}{%
\pgfpathmoveto{\pgfqpoint{0.000000in}{0.000000in}}%
\pgfpathlineto{\pgfqpoint{0.048611in}{0.000000in}}%
\pgfusepath{stroke,fill}%
}%
\begin{pgfscope}%
\pgfsys@transformshift{2.091551in}{1.005543in}%
\pgfsys@useobject{currentmarker}{}%
\end{pgfscope}%
\end{pgfscope}%
\begin{pgfscope}%
\pgftext[x=2.188773in,y=0.957716in,left,base]{\rmfamily\fontsize{10.000000}{12.000000}\selectfont \(\displaystyle 5.0\)}%
\end{pgfscope}%
\begin{pgfscope}%
\pgfsetbuttcap%
\pgfsetroundjoin%
\definecolor{currentfill}{rgb}{0.000000,0.000000,0.000000}%
\pgfsetfillcolor{currentfill}%
\pgfsetlinewidth{0.803000pt}%
\definecolor{currentstroke}{rgb}{0.000000,0.000000,0.000000}%
\pgfsetstrokecolor{currentstroke}%
\pgfsetdash{}{0pt}%
\pgfsys@defobject{currentmarker}{\pgfqpoint{0.000000in}{0.000000in}}{\pgfqpoint{0.048611in}{0.000000in}}{%
\pgfpathmoveto{\pgfqpoint{0.000000in}{0.000000in}}%
\pgfpathlineto{\pgfqpoint{0.048611in}{0.000000in}}%
\pgfusepath{stroke,fill}%
}%
\begin{pgfscope}%
\pgfsys@transformshift{2.091551in}{1.348376in}%
\pgfsys@useobject{currentmarker}{}%
\end{pgfscope}%
\end{pgfscope}%
\begin{pgfscope}%
\pgftext[x=2.188773in,y=1.300548in,left,base]{\rmfamily\fontsize{10.000000}{12.000000}\selectfont \(\displaystyle 7.5\)}%
\end{pgfscope}%
\begin{pgfscope}%
\pgfsetbuttcap%
\pgfsetroundjoin%
\definecolor{currentfill}{rgb}{0.000000,0.000000,0.000000}%
\pgfsetfillcolor{currentfill}%
\pgfsetlinewidth{0.803000pt}%
\definecolor{currentstroke}{rgb}{0.000000,0.000000,0.000000}%
\pgfsetstrokecolor{currentstroke}%
\pgfsetdash{}{0pt}%
\pgfsys@defobject{currentmarker}{\pgfqpoint{0.000000in}{0.000000in}}{\pgfqpoint{0.048611in}{0.000000in}}{%
\pgfpathmoveto{\pgfqpoint{0.000000in}{0.000000in}}%
\pgfpathlineto{\pgfqpoint{0.048611in}{0.000000in}}%
\pgfusepath{stroke,fill}%
}%
\begin{pgfscope}%
\pgfsys@transformshift{2.091551in}{1.691209in}%
\pgfsys@useobject{currentmarker}{}%
\end{pgfscope}%
\end{pgfscope}%
\begin{pgfscope}%
\pgftext[x=2.188773in,y=1.643381in,left,base]{\rmfamily\fontsize{10.000000}{12.000000}\selectfont \(\displaystyle 10.0\)}%
\end{pgfscope}%
\begin{pgfscope}%
\pgfsetbuttcap%
\pgfsetroundjoin%
\definecolor{currentfill}{rgb}{0.000000,0.000000,0.000000}%
\pgfsetfillcolor{currentfill}%
\pgfsetlinewidth{0.803000pt}%
\definecolor{currentstroke}{rgb}{0.000000,0.000000,0.000000}%
\pgfsetstrokecolor{currentstroke}%
\pgfsetdash{}{0pt}%
\pgfsys@defobject{currentmarker}{\pgfqpoint{0.000000in}{0.000000in}}{\pgfqpoint{0.048611in}{0.000000in}}{%
\pgfpathmoveto{\pgfqpoint{0.000000in}{0.000000in}}%
\pgfpathlineto{\pgfqpoint{0.048611in}{0.000000in}}%
\pgfusepath{stroke,fill}%
}%
\begin{pgfscope}%
\pgfsys@transformshift{2.091551in}{2.034042in}%
\pgfsys@useobject{currentmarker}{}%
\end{pgfscope}%
\end{pgfscope}%
\begin{pgfscope}%
\pgftext[x=2.188773in,y=1.986214in,left,base]{\rmfamily\fontsize{10.000000}{12.000000}\selectfont \(\displaystyle 12.5\)}%
\end{pgfscope}%
\begin{pgfscope}%
\pgfsetbuttcap%
\pgfsetroundjoin%
\definecolor{currentfill}{rgb}{0.000000,0.000000,0.000000}%
\pgfsetfillcolor{currentfill}%
\pgfsetlinewidth{0.803000pt}%
\definecolor{currentstroke}{rgb}{0.000000,0.000000,0.000000}%
\pgfsetstrokecolor{currentstroke}%
\pgfsetdash{}{0pt}%
\pgfsys@defobject{currentmarker}{\pgfqpoint{0.000000in}{0.000000in}}{\pgfqpoint{0.048611in}{0.000000in}}{%
\pgfpathmoveto{\pgfqpoint{0.000000in}{0.000000in}}%
\pgfpathlineto{\pgfqpoint{0.048611in}{0.000000in}}%
\pgfusepath{stroke,fill}%
}%
\begin{pgfscope}%
\pgfsys@transformshift{2.091551in}{2.376875in}%
\pgfsys@useobject{currentmarker}{}%
\end{pgfscope}%
\end{pgfscope}%
\begin{pgfscope}%
\pgftext[x=2.188773in,y=2.329047in,left,base]{\rmfamily\fontsize{10.000000}{12.000000}\selectfont \(\displaystyle 15.0\)}%
\end{pgfscope}%
\begin{pgfscope}%
\pgfsetbuttcap%
\pgfsetroundjoin%
\definecolor{currentfill}{rgb}{0.000000,0.000000,0.000000}%
\pgfsetfillcolor{currentfill}%
\pgfsetlinewidth{0.803000pt}%
\definecolor{currentstroke}{rgb}{0.000000,0.000000,0.000000}%
\pgfsetstrokecolor{currentstroke}%
\pgfsetdash{}{0pt}%
\pgfsys@defobject{currentmarker}{\pgfqpoint{0.000000in}{0.000000in}}{\pgfqpoint{0.048611in}{0.000000in}}{%
\pgfpathmoveto{\pgfqpoint{0.000000in}{0.000000in}}%
\pgfpathlineto{\pgfqpoint{0.048611in}{0.000000in}}%
\pgfusepath{stroke,fill}%
}%
\begin{pgfscope}%
\pgfsys@transformshift{2.091551in}{2.719708in}%
\pgfsys@useobject{currentmarker}{}%
\end{pgfscope}%
\end{pgfscope}%
\begin{pgfscope}%
\pgftext[x=2.188773in,y=2.671880in,left,base]{\rmfamily\fontsize{10.000000}{12.000000}\selectfont \(\displaystyle 17.5\)}%
\end{pgfscope}%
\begin{pgfscope}%
\pgfsetbuttcap%
\pgfsetmiterjoin%
\pgfsetlinewidth{0.803000pt}%
\definecolor{currentstroke}{rgb}{0.000000,0.000000,0.000000}%
\pgfsetstrokecolor{currentstroke}%
\pgfsetdash{}{0pt}%
\pgfpathmoveto{\pgfqpoint{1.961274in}{0.319877in}}%
\pgfpathlineto{\pgfqpoint{1.961274in}{0.330055in}}%
\pgfpathlineto{\pgfqpoint{1.961274in}{2.915230in}}%
\pgfpathlineto{\pgfqpoint{1.961274in}{2.925408in}}%
\pgfpathlineto{\pgfqpoint{2.091551in}{2.925408in}}%
\pgfpathlineto{\pgfqpoint{2.091551in}{2.915230in}}%
\pgfpathlineto{\pgfqpoint{2.091551in}{0.330055in}}%
\pgfpathlineto{\pgfqpoint{2.091551in}{0.319877in}}%
\pgfpathclose%
\pgfusepath{stroke}%
\end{pgfscope}%
\end{pgfpicture}%
\makeatother%
\endgroup%

    \vspace*{-0.4cm}
	\caption{1000 K. Bin size $0.020e$}
	\end{subfigure}
\caption{Change in St.Dev of distance to nearest neighbours of Ti vs change in Ti charge}
\label{on_site_RnnStDev_vs_dq}
\end{figure}

5) Figure \ref{on_site_Perr_vs_dq} now examines teh relation between the change in charge on a Ti ion and its own dipole moment. The exact quantity represented by the y-axis is defined as:
\begin{align*}
y_{i,I} \equiv \sqrt{\sum_{\alpha = x,y,z}\left(p_{i,I}^{\alpha}(\{q_l\})-p_{i,I}^{\alpha}(\bar{q}_{\text{Ba}},\bar{q}_{\text{Ti}},\bar{q}_{\text{O}})\right)^2}
\end{align*}
where $i\in \{\text{Ti}_1,...,\text{Ti}_{27}\}$ and $I\in\{\text{MD}_1,...,\text{MD}_{10}\}$; the normalization was not performed because dipoles vary by at least two orders of magnitude on different ions\footnote{Normalizing, as a consequence, leads to a wide amplification of the y-axis.}. 

\begin{figure}[h!]
\centering
	\begin{subfigure}[b]{0.45\textwidth}
	\hspace*{-0.4cm}
	%% Creator: Matplotlib, PGF backend
%%
%% To include the figure in your LaTeX document, write
%%   \input{<filename>.pgf}
%%
%% Make sure the required packages are loaded in your preamble
%%   \usepackage{pgf}
%%
%% Figures using additional raster images can only be included by \input if
%% they are in the same directory as the main LaTeX file. For loading figures
%% from other directories you can use the `import` package
%%   \usepackage{import}
%% and then include the figures with
%%   \import{<path to file>}{<filename>.pgf}
%%
%% Matplotlib used the following preamble
%%   \usepackage[utf8x]{inputenc}
%%   \usepackage[T1]{fontenc}
%%
\begingroup%
\makeatletter%
\begin{pgfpicture}%
\pgfpathrectangle{\pgfpointorigin}{\pgfqpoint{2.518842in}{3.060408in}}%
\pgfusepath{use as bounding box, clip}%
\begin{pgfscope}%
\pgfsetbuttcap%
\pgfsetmiterjoin%
\definecolor{currentfill}{rgb}{1.000000,1.000000,1.000000}%
\pgfsetfillcolor{currentfill}%
\pgfsetlinewidth{0.000000pt}%
\definecolor{currentstroke}{rgb}{1.000000,1.000000,1.000000}%
\pgfsetstrokecolor{currentstroke}%
\pgfsetdash{}{0pt}%
\pgfpathmoveto{\pgfqpoint{0.000000in}{0.000000in}}%
\pgfpathlineto{\pgfqpoint{2.518842in}{0.000000in}}%
\pgfpathlineto{\pgfqpoint{2.518842in}{3.060408in}}%
\pgfpathlineto{\pgfqpoint{0.000000in}{3.060408in}}%
\pgfpathclose%
\pgfusepath{fill}%
\end{pgfscope}%
\begin{pgfscope}%
\pgfsetbuttcap%
\pgfsetmiterjoin%
\definecolor{currentfill}{rgb}{1.000000,1.000000,1.000000}%
\pgfsetfillcolor{currentfill}%
\pgfsetlinewidth{0.000000pt}%
\definecolor{currentstroke}{rgb}{0.000000,0.000000,0.000000}%
\pgfsetstrokecolor{currentstroke}%
\pgfsetstrokeopacity{0.000000}%
\pgfsetdash{}{0pt}%
\pgfpathmoveto{\pgfqpoint{0.374692in}{0.319877in}}%
\pgfpathlineto{\pgfqpoint{1.954366in}{0.319877in}}%
\pgfpathlineto{\pgfqpoint{1.954366in}{2.912580in}}%
\pgfpathlineto{\pgfqpoint{0.374692in}{2.912580in}}%
\pgfpathclose%
\pgfusepath{fill}%
\end{pgfscope}%
\begin{pgfscope}%
\pgfpathrectangle{\pgfqpoint{0.374692in}{0.319877in}}{\pgfqpoint{1.579674in}{2.592703in}} %
\pgfusepath{clip}%
\pgfsys@transformshift{0.374692in}{0.319877in}%
\pgftext[left,bottom]{\pgfimage[interpolate=true,width=1.580000in,height=2.600000in]{Perr_vs_dq_Ti_100K-img0.png}}%
\end{pgfscope}%
\begin{pgfscope}%
\pgfpathrectangle{\pgfqpoint{0.374692in}{0.319877in}}{\pgfqpoint{1.579674in}{2.592703in}} %
\pgfusepath{clip}%
\pgfsetbuttcap%
\pgfsetroundjoin%
\definecolor{currentfill}{rgb}{1.000000,0.752941,0.796078}%
\pgfsetfillcolor{currentfill}%
\pgfsetlinewidth{1.003750pt}%
\definecolor{currentstroke}{rgb}{1.000000,0.752941,0.796078}%
\pgfsetstrokecolor{currentstroke}%
\pgfsetdash{}{0pt}%
\pgfpathmoveto{\pgfqpoint{0.953906in}{0.816954in}}%
\pgfpathcurveto{\pgfqpoint{0.964956in}{0.816954in}}{\pgfqpoint{0.975555in}{0.821344in}}{\pgfqpoint{0.983368in}{0.829158in}}%
\pgfpathcurveto{\pgfqpoint{0.991182in}{0.836972in}}{\pgfqpoint{0.995572in}{0.847571in}}{\pgfqpoint{0.995572in}{0.858621in}}%
\pgfpathcurveto{\pgfqpoint{0.995572in}{0.869671in}}{\pgfqpoint{0.991182in}{0.880270in}}{\pgfqpoint{0.983368in}{0.888084in}}%
\pgfpathcurveto{\pgfqpoint{0.975555in}{0.895897in}}{\pgfqpoint{0.964956in}{0.900287in}}{\pgfqpoint{0.953906in}{0.900287in}}%
\pgfpathcurveto{\pgfqpoint{0.942856in}{0.900287in}}{\pgfqpoint{0.932257in}{0.895897in}}{\pgfqpoint{0.924443in}{0.888084in}}%
\pgfpathcurveto{\pgfqpoint{0.916629in}{0.880270in}}{\pgfqpoint{0.912239in}{0.869671in}}{\pgfqpoint{0.912239in}{0.858621in}}%
\pgfpathcurveto{\pgfqpoint{0.912239in}{0.847571in}}{\pgfqpoint{0.916629in}{0.836972in}}{\pgfqpoint{0.924443in}{0.829158in}}%
\pgfpathcurveto{\pgfqpoint{0.932257in}{0.821344in}}{\pgfqpoint{0.942856in}{0.816954in}}{\pgfqpoint{0.953906in}{0.816954in}}%
\pgfpathclose%
\pgfusepath{stroke,fill}%
\end{pgfscope}%
\begin{pgfscope}%
\pgfpathrectangle{\pgfqpoint{0.374692in}{0.319877in}}{\pgfqpoint{1.579674in}{2.592703in}} %
\pgfusepath{clip}%
\pgfsetbuttcap%
\pgfsetroundjoin%
\definecolor{currentfill}{rgb}{1.000000,0.752941,0.796078}%
\pgfsetfillcolor{currentfill}%
\pgfsetlinewidth{1.003750pt}%
\definecolor{currentstroke}{rgb}{1.000000,0.752941,0.796078}%
\pgfsetstrokecolor{currentstroke}%
\pgfsetdash{}{0pt}%
\pgfpathmoveto{\pgfqpoint{1.059217in}{0.876944in}}%
\pgfpathcurveto{\pgfqpoint{1.070267in}{0.876944in}}{\pgfqpoint{1.080866in}{0.881334in}}{\pgfqpoint{1.088680in}{0.889147in}}%
\pgfpathcurveto{\pgfqpoint{1.096494in}{0.896961in}}{\pgfqpoint{1.100884in}{0.907560in}}{\pgfqpoint{1.100884in}{0.918610in}}%
\pgfpathcurveto{\pgfqpoint{1.100884in}{0.929660in}}{\pgfqpoint{1.096494in}{0.940259in}}{\pgfqpoint{1.088680in}{0.948073in}}%
\pgfpathcurveto{\pgfqpoint{1.080866in}{0.955887in}}{\pgfqpoint{1.070267in}{0.960277in}}{\pgfqpoint{1.059217in}{0.960277in}}%
\pgfpathcurveto{\pgfqpoint{1.048167in}{0.960277in}}{\pgfqpoint{1.037568in}{0.955887in}}{\pgfqpoint{1.029754in}{0.948073in}}%
\pgfpathcurveto{\pgfqpoint{1.021941in}{0.940259in}}{\pgfqpoint{1.017551in}{0.929660in}}{\pgfqpoint{1.017551in}{0.918610in}}%
\pgfpathcurveto{\pgfqpoint{1.017551in}{0.907560in}}{\pgfqpoint{1.021941in}{0.896961in}}{\pgfqpoint{1.029754in}{0.889147in}}%
\pgfpathcurveto{\pgfqpoint{1.037568in}{0.881334in}}{\pgfqpoint{1.048167in}{0.876944in}}{\pgfqpoint{1.059217in}{0.876944in}}%
\pgfpathclose%
\pgfusepath{stroke,fill}%
\end{pgfscope}%
\begin{pgfscope}%
\pgfpathrectangle{\pgfqpoint{0.374692in}{0.319877in}}{\pgfqpoint{1.579674in}{2.592703in}} %
\pgfusepath{clip}%
\pgfsetbuttcap%
\pgfsetroundjoin%
\definecolor{currentfill}{rgb}{1.000000,0.752941,0.796078}%
\pgfsetfillcolor{currentfill}%
\pgfsetlinewidth{1.003750pt}%
\definecolor{currentstroke}{rgb}{1.000000,0.752941,0.796078}%
\pgfsetstrokecolor{currentstroke}%
\pgfsetdash{}{0pt}%
\pgfpathmoveto{\pgfqpoint{1.164529in}{0.993282in}}%
\pgfpathcurveto{\pgfqpoint{1.175579in}{0.993282in}}{\pgfqpoint{1.186178in}{0.997672in}}{\pgfqpoint{1.193992in}{1.005486in}}%
\pgfpathcurveto{\pgfqpoint{1.201805in}{1.013299in}}{\pgfqpoint{1.206196in}{1.023899in}}{\pgfqpoint{1.206196in}{1.034949in}}%
\pgfpathcurveto{\pgfqpoint{1.206196in}{1.045999in}}{\pgfqpoint{1.201805in}{1.056598in}}{\pgfqpoint{1.193992in}{1.064411in}}%
\pgfpathcurveto{\pgfqpoint{1.186178in}{1.072225in}}{\pgfqpoint{1.175579in}{1.076615in}}{\pgfqpoint{1.164529in}{1.076615in}}%
\pgfpathcurveto{\pgfqpoint{1.153479in}{1.076615in}}{\pgfqpoint{1.142880in}{1.072225in}}{\pgfqpoint{1.135066in}{1.064411in}}%
\pgfpathcurveto{\pgfqpoint{1.127252in}{1.056598in}}{\pgfqpoint{1.122862in}{1.045999in}}{\pgfqpoint{1.122862in}{1.034949in}}%
\pgfpathcurveto{\pgfqpoint{1.122862in}{1.023899in}}{\pgfqpoint{1.127252in}{1.013299in}}{\pgfqpoint{1.135066in}{1.005486in}}%
\pgfpathcurveto{\pgfqpoint{1.142880in}{0.997672in}}{\pgfqpoint{1.153479in}{0.993282in}}{\pgfqpoint{1.164529in}{0.993282in}}%
\pgfpathclose%
\pgfusepath{stroke,fill}%
\end{pgfscope}%
\begin{pgfscope}%
\pgfpathrectangle{\pgfqpoint{0.374692in}{0.319877in}}{\pgfqpoint{1.579674in}{2.592703in}} %
\pgfusepath{clip}%
\pgfsetbuttcap%
\pgfsetroundjoin%
\definecolor{currentfill}{rgb}{1.000000,0.752941,0.796078}%
\pgfsetfillcolor{currentfill}%
\pgfsetlinewidth{1.003750pt}%
\definecolor{currentstroke}{rgb}{1.000000,0.752941,0.796078}%
\pgfsetstrokecolor{currentstroke}%
\pgfsetdash{}{0pt}%
\pgfpathmoveto{\pgfqpoint{1.269840in}{0.926064in}}%
\pgfpathcurveto{\pgfqpoint{1.280891in}{0.926064in}}{\pgfqpoint{1.291490in}{0.930455in}}{\pgfqpoint{1.299303in}{0.938268in}}%
\pgfpathcurveto{\pgfqpoint{1.307117in}{0.946082in}}{\pgfqpoint{1.311507in}{0.956681in}}{\pgfqpoint{1.311507in}{0.967731in}}%
\pgfpathcurveto{\pgfqpoint{1.311507in}{0.978781in}}{\pgfqpoint{1.307117in}{0.989380in}}{\pgfqpoint{1.299303in}{0.997194in}}%
\pgfpathcurveto{\pgfqpoint{1.291490in}{1.005007in}}{\pgfqpoint{1.280891in}{1.009398in}}{\pgfqpoint{1.269840in}{1.009398in}}%
\pgfpathcurveto{\pgfqpoint{1.258790in}{1.009398in}}{\pgfqpoint{1.248191in}{1.005007in}}{\pgfqpoint{1.240378in}{0.997194in}}%
\pgfpathcurveto{\pgfqpoint{1.232564in}{0.989380in}}{\pgfqpoint{1.228174in}{0.978781in}}{\pgfqpoint{1.228174in}{0.967731in}}%
\pgfpathcurveto{\pgfqpoint{1.228174in}{0.956681in}}{\pgfqpoint{1.232564in}{0.946082in}}{\pgfqpoint{1.240378in}{0.938268in}}%
\pgfpathcurveto{\pgfqpoint{1.248191in}{0.930455in}}{\pgfqpoint{1.258790in}{0.926064in}}{\pgfqpoint{1.269840in}{0.926064in}}%
\pgfpathclose%
\pgfusepath{stroke,fill}%
\end{pgfscope}%
\begin{pgfscope}%
\pgfpathrectangle{\pgfqpoint{0.374692in}{0.319877in}}{\pgfqpoint{1.579674in}{2.592703in}} %
\pgfusepath{clip}%
\pgfsetbuttcap%
\pgfsetroundjoin%
\definecolor{currentfill}{rgb}{1.000000,0.752941,0.796078}%
\pgfsetfillcolor{currentfill}%
\pgfsetlinewidth{1.003750pt}%
\definecolor{currentstroke}{rgb}{1.000000,0.752941,0.796078}%
\pgfsetstrokecolor{currentstroke}%
\pgfsetdash{}{0pt}%
\pgfpathmoveto{\pgfqpoint{1.375152in}{0.500816in}}%
\pgfpathcurveto{\pgfqpoint{1.386202in}{0.500816in}}{\pgfqpoint{1.396801in}{0.505207in}}{\pgfqpoint{1.404615in}{0.513020in}}%
\pgfpathcurveto{\pgfqpoint{1.412428in}{0.520834in}}{\pgfqpoint{1.416819in}{0.531433in}}{\pgfqpoint{1.416819in}{0.542483in}}%
\pgfpathcurveto{\pgfqpoint{1.416819in}{0.553533in}}{\pgfqpoint{1.412428in}{0.564132in}}{\pgfqpoint{1.404615in}{0.571946in}}%
\pgfpathcurveto{\pgfqpoint{1.396801in}{0.579760in}}{\pgfqpoint{1.386202in}{0.584150in}}{\pgfqpoint{1.375152in}{0.584150in}}%
\pgfpathcurveto{\pgfqpoint{1.364102in}{0.584150in}}{\pgfqpoint{1.353503in}{0.579760in}}{\pgfqpoint{1.345689in}{0.571946in}}%
\pgfpathcurveto{\pgfqpoint{1.337876in}{0.564132in}}{\pgfqpoint{1.333485in}{0.553533in}}{\pgfqpoint{1.333485in}{0.542483in}}%
\pgfpathcurveto{\pgfqpoint{1.333485in}{0.531433in}}{\pgfqpoint{1.337876in}{0.520834in}}{\pgfqpoint{1.345689in}{0.513020in}}%
\pgfpathcurveto{\pgfqpoint{1.353503in}{0.505207in}}{\pgfqpoint{1.364102in}{0.500816in}}{\pgfqpoint{1.375152in}{0.500816in}}%
\pgfpathclose%
\pgfusepath{stroke,fill}%
\end{pgfscope}%
\begin{pgfscope}%
\pgfsetbuttcap%
\pgfsetroundjoin%
\definecolor{currentfill}{rgb}{0.000000,0.000000,0.000000}%
\pgfsetfillcolor{currentfill}%
\pgfsetlinewidth{0.803000pt}%
\definecolor{currentstroke}{rgb}{0.000000,0.000000,0.000000}%
\pgfsetstrokecolor{currentstroke}%
\pgfsetdash{}{0pt}%
\pgfsys@defobject{currentmarker}{\pgfqpoint{0.000000in}{-0.048611in}}{\pgfqpoint{0.000000in}{0.000000in}}{%
\pgfpathmoveto{\pgfqpoint{0.000000in}{0.000000in}}%
\pgfpathlineto{\pgfqpoint{0.000000in}{-0.048611in}}%
\pgfusepath{stroke,fill}%
}%
\begin{pgfscope}%
\pgfsys@transformshift{0.670881in}{0.319877in}%
\pgfsys@useobject{currentmarker}{}%
\end{pgfscope}%
\end{pgfscope}%
\begin{pgfscope}%
\pgftext[x=0.670881in,y=0.222655in,,top]{\rmfamily\fontsize{10.000000}{12.000000}\selectfont \(\displaystyle -0.05\)}%
\end{pgfscope}%
\begin{pgfscope}%
\pgfsetbuttcap%
\pgfsetroundjoin%
\definecolor{currentfill}{rgb}{0.000000,0.000000,0.000000}%
\pgfsetfillcolor{currentfill}%
\pgfsetlinewidth{0.803000pt}%
\definecolor{currentstroke}{rgb}{0.000000,0.000000,0.000000}%
\pgfsetstrokecolor{currentstroke}%
\pgfsetdash{}{0pt}%
\pgfsys@defobject{currentmarker}{\pgfqpoint{0.000000in}{-0.048611in}}{\pgfqpoint{0.000000in}{0.000000in}}{%
\pgfpathmoveto{\pgfqpoint{0.000000in}{0.000000in}}%
\pgfpathlineto{\pgfqpoint{0.000000in}{-0.048611in}}%
\pgfusepath{stroke,fill}%
}%
\begin{pgfscope}%
\pgfsys@transformshift{1.164529in}{0.319877in}%
\pgfsys@useobject{currentmarker}{}%
\end{pgfscope}%
\end{pgfscope}%
\begin{pgfscope}%
\pgftext[x=1.164529in,y=0.222655in,,top]{\rmfamily\fontsize{10.000000}{12.000000}\selectfont \(\displaystyle 0.00\)}%
\end{pgfscope}%
\begin{pgfscope}%
\pgfsetbuttcap%
\pgfsetroundjoin%
\definecolor{currentfill}{rgb}{0.000000,0.000000,0.000000}%
\pgfsetfillcolor{currentfill}%
\pgfsetlinewidth{0.803000pt}%
\definecolor{currentstroke}{rgb}{0.000000,0.000000,0.000000}%
\pgfsetstrokecolor{currentstroke}%
\pgfsetdash{}{0pt}%
\pgfsys@defobject{currentmarker}{\pgfqpoint{0.000000in}{-0.048611in}}{\pgfqpoint{0.000000in}{0.000000in}}{%
\pgfpathmoveto{\pgfqpoint{0.000000in}{0.000000in}}%
\pgfpathlineto{\pgfqpoint{0.000000in}{-0.048611in}}%
\pgfusepath{stroke,fill}%
}%
\begin{pgfscope}%
\pgfsys@transformshift{1.658177in}{0.319877in}%
\pgfsys@useobject{currentmarker}{}%
\end{pgfscope}%
\end{pgfscope}%
\begin{pgfscope}%
\pgftext[x=1.658177in,y=0.222655in,,top]{\rmfamily\fontsize{10.000000}{12.000000}\selectfont \(\displaystyle 0.05\)}%
\end{pgfscope}%
\begin{pgfscope}%
\pgfsetbuttcap%
\pgfsetroundjoin%
\definecolor{currentfill}{rgb}{0.000000,0.000000,0.000000}%
\pgfsetfillcolor{currentfill}%
\pgfsetlinewidth{0.803000pt}%
\definecolor{currentstroke}{rgb}{0.000000,0.000000,0.000000}%
\pgfsetstrokecolor{currentstroke}%
\pgfsetdash{}{0pt}%
\pgfsys@defobject{currentmarker}{\pgfqpoint{-0.048611in}{0.000000in}}{\pgfqpoint{0.000000in}{0.000000in}}{%
\pgfpathmoveto{\pgfqpoint{0.000000in}{0.000000in}}%
\pgfpathlineto{\pgfqpoint{-0.048611in}{0.000000in}}%
\pgfusepath{stroke,fill}%
}%
\begin{pgfscope}%
\pgfsys@transformshift{0.374692in}{0.319877in}%
\pgfsys@useobject{currentmarker}{}%
\end{pgfscope}%
\end{pgfscope}%
\begin{pgfscope}%
\pgftext[x=0.100000in,y=0.272050in,left,base]{\rmfamily\fontsize{10.000000}{12.000000}\selectfont \(\displaystyle 0.0\)}%
\end{pgfscope}%
\begin{pgfscope}%
\pgfsetbuttcap%
\pgfsetroundjoin%
\definecolor{currentfill}{rgb}{0.000000,0.000000,0.000000}%
\pgfsetfillcolor{currentfill}%
\pgfsetlinewidth{0.803000pt}%
\definecolor{currentstroke}{rgb}{0.000000,0.000000,0.000000}%
\pgfsetstrokecolor{currentstroke}%
\pgfsetdash{}{0pt}%
\pgfsys@defobject{currentmarker}{\pgfqpoint{-0.048611in}{0.000000in}}{\pgfqpoint{0.000000in}{0.000000in}}{%
\pgfpathmoveto{\pgfqpoint{0.000000in}{0.000000in}}%
\pgfpathlineto{\pgfqpoint{-0.048611in}{0.000000in}}%
\pgfusepath{stroke,fill}%
}%
\begin{pgfscope}%
\pgfsys@transformshift{0.374692in}{0.838418in}%
\pgfsys@useobject{currentmarker}{}%
\end{pgfscope}%
\end{pgfscope}%
\begin{pgfscope}%
\pgftext[x=0.100000in,y=0.790590in,left,base]{\rmfamily\fontsize{10.000000}{12.000000}\selectfont \(\displaystyle 0.2\)}%
\end{pgfscope}%
\begin{pgfscope}%
\pgfsetbuttcap%
\pgfsetroundjoin%
\definecolor{currentfill}{rgb}{0.000000,0.000000,0.000000}%
\pgfsetfillcolor{currentfill}%
\pgfsetlinewidth{0.803000pt}%
\definecolor{currentstroke}{rgb}{0.000000,0.000000,0.000000}%
\pgfsetstrokecolor{currentstroke}%
\pgfsetdash{}{0pt}%
\pgfsys@defobject{currentmarker}{\pgfqpoint{-0.048611in}{0.000000in}}{\pgfqpoint{0.000000in}{0.000000in}}{%
\pgfpathmoveto{\pgfqpoint{0.000000in}{0.000000in}}%
\pgfpathlineto{\pgfqpoint{-0.048611in}{0.000000in}}%
\pgfusepath{stroke,fill}%
}%
\begin{pgfscope}%
\pgfsys@transformshift{0.374692in}{1.356958in}%
\pgfsys@useobject{currentmarker}{}%
\end{pgfscope}%
\end{pgfscope}%
\begin{pgfscope}%
\pgftext[x=0.100000in,y=1.309131in,left,base]{\rmfamily\fontsize{10.000000}{12.000000}\selectfont \(\displaystyle 0.4\)}%
\end{pgfscope}%
\begin{pgfscope}%
\pgfsetbuttcap%
\pgfsetroundjoin%
\definecolor{currentfill}{rgb}{0.000000,0.000000,0.000000}%
\pgfsetfillcolor{currentfill}%
\pgfsetlinewidth{0.803000pt}%
\definecolor{currentstroke}{rgb}{0.000000,0.000000,0.000000}%
\pgfsetstrokecolor{currentstroke}%
\pgfsetdash{}{0pt}%
\pgfsys@defobject{currentmarker}{\pgfqpoint{-0.048611in}{0.000000in}}{\pgfqpoint{0.000000in}{0.000000in}}{%
\pgfpathmoveto{\pgfqpoint{0.000000in}{0.000000in}}%
\pgfpathlineto{\pgfqpoint{-0.048611in}{0.000000in}}%
\pgfusepath{stroke,fill}%
}%
\begin{pgfscope}%
\pgfsys@transformshift{0.374692in}{1.875499in}%
\pgfsys@useobject{currentmarker}{}%
\end{pgfscope}%
\end{pgfscope}%
\begin{pgfscope}%
\pgftext[x=0.100000in,y=1.827671in,left,base]{\rmfamily\fontsize{10.000000}{12.000000}\selectfont \(\displaystyle 0.6\)}%
\end{pgfscope}%
\begin{pgfscope}%
\pgfsetbuttcap%
\pgfsetroundjoin%
\definecolor{currentfill}{rgb}{0.000000,0.000000,0.000000}%
\pgfsetfillcolor{currentfill}%
\pgfsetlinewidth{0.803000pt}%
\definecolor{currentstroke}{rgb}{0.000000,0.000000,0.000000}%
\pgfsetstrokecolor{currentstroke}%
\pgfsetdash{}{0pt}%
\pgfsys@defobject{currentmarker}{\pgfqpoint{-0.048611in}{0.000000in}}{\pgfqpoint{0.000000in}{0.000000in}}{%
\pgfpathmoveto{\pgfqpoint{0.000000in}{0.000000in}}%
\pgfpathlineto{\pgfqpoint{-0.048611in}{0.000000in}}%
\pgfusepath{stroke,fill}%
}%
\begin{pgfscope}%
\pgfsys@transformshift{0.374692in}{2.394040in}%
\pgfsys@useobject{currentmarker}{}%
\end{pgfscope}%
\end{pgfscope}%
\begin{pgfscope}%
\pgftext[x=0.100000in,y=2.346212in,left,base]{\rmfamily\fontsize{10.000000}{12.000000}\selectfont \(\displaystyle 0.8\)}%
\end{pgfscope}%
\begin{pgfscope}%
\pgfsetbuttcap%
\pgfsetroundjoin%
\definecolor{currentfill}{rgb}{0.000000,0.000000,0.000000}%
\pgfsetfillcolor{currentfill}%
\pgfsetlinewidth{0.803000pt}%
\definecolor{currentstroke}{rgb}{0.000000,0.000000,0.000000}%
\pgfsetstrokecolor{currentstroke}%
\pgfsetdash{}{0pt}%
\pgfsys@defobject{currentmarker}{\pgfqpoint{-0.048611in}{0.000000in}}{\pgfqpoint{0.000000in}{0.000000in}}{%
\pgfpathmoveto{\pgfqpoint{0.000000in}{0.000000in}}%
\pgfpathlineto{\pgfqpoint{-0.048611in}{0.000000in}}%
\pgfusepath{stroke,fill}%
}%
\begin{pgfscope}%
\pgfsys@transformshift{0.374692in}{2.912580in}%
\pgfsys@useobject{currentmarker}{}%
\end{pgfscope}%
\end{pgfscope}%
\begin{pgfscope}%
\pgftext[x=0.100000in,y=2.864752in,left,base]{\rmfamily\fontsize{10.000000}{12.000000}\selectfont \(\displaystyle 1.0\)}%
\end{pgfscope}%
\begin{pgfscope}%
\pgfsetrectcap%
\pgfsetmiterjoin%
\pgfsetlinewidth{0.803000pt}%
\definecolor{currentstroke}{rgb}{0.000000,0.000000,0.000000}%
\pgfsetstrokecolor{currentstroke}%
\pgfsetdash{}{0pt}%
\pgfpathmoveto{\pgfqpoint{0.374692in}{0.319877in}}%
\pgfpathlineto{\pgfqpoint{0.374692in}{2.912580in}}%
\pgfusepath{stroke}%
\end{pgfscope}%
\begin{pgfscope}%
\pgfsetrectcap%
\pgfsetmiterjoin%
\pgfsetlinewidth{0.803000pt}%
\definecolor{currentstroke}{rgb}{0.000000,0.000000,0.000000}%
\pgfsetstrokecolor{currentstroke}%
\pgfsetdash{}{0pt}%
\pgfpathmoveto{\pgfqpoint{1.954366in}{0.319877in}}%
\pgfpathlineto{\pgfqpoint{1.954366in}{2.912580in}}%
\pgfusepath{stroke}%
\end{pgfscope}%
\begin{pgfscope}%
\pgfsetrectcap%
\pgfsetmiterjoin%
\pgfsetlinewidth{0.803000pt}%
\definecolor{currentstroke}{rgb}{0.000000,0.000000,0.000000}%
\pgfsetstrokecolor{currentstroke}%
\pgfsetdash{}{0pt}%
\pgfpathmoveto{\pgfqpoint{0.374692in}{0.319877in}}%
\pgfpathlineto{\pgfqpoint{1.954366in}{0.319877in}}%
\pgfusepath{stroke}%
\end{pgfscope}%
\begin{pgfscope}%
\pgfsetrectcap%
\pgfsetmiterjoin%
\pgfsetlinewidth{0.803000pt}%
\definecolor{currentstroke}{rgb}{0.000000,0.000000,0.000000}%
\pgfsetstrokecolor{currentstroke}%
\pgfsetdash{}{0pt}%
\pgfpathmoveto{\pgfqpoint{0.374692in}{2.912580in}}%
\pgfpathlineto{\pgfqpoint{1.954366in}{2.912580in}}%
\pgfusepath{stroke}%
\end{pgfscope}%
\begin{pgfscope}%
\pgfpathrectangle{\pgfqpoint{2.053095in}{0.319877in}}{\pgfqpoint{0.129635in}{2.592703in}} %
\pgfusepath{clip}%
\pgfsetbuttcap%
\pgfsetmiterjoin%
\definecolor{currentfill}{rgb}{1.000000,1.000000,1.000000}%
\pgfsetfillcolor{currentfill}%
\pgfsetlinewidth{0.010037pt}%
\definecolor{currentstroke}{rgb}{1.000000,1.000000,1.000000}%
\pgfsetstrokecolor{currentstroke}%
\pgfsetdash{}{0pt}%
\pgfpathmoveto{\pgfqpoint{2.053095in}{0.319877in}}%
\pgfpathlineto{\pgfqpoint{2.053095in}{0.330005in}}%
\pgfpathlineto{\pgfqpoint{2.053095in}{2.902452in}}%
\pgfpathlineto{\pgfqpoint{2.053095in}{2.912580in}}%
\pgfpathlineto{\pgfqpoint{2.182730in}{2.912580in}}%
\pgfpathlineto{\pgfqpoint{2.182730in}{2.902452in}}%
\pgfpathlineto{\pgfqpoint{2.182730in}{0.330005in}}%
\pgfpathlineto{\pgfqpoint{2.182730in}{0.319877in}}%
\pgfpathclose%
\pgfusepath{stroke,fill}%
\end{pgfscope}%
\begin{pgfscope}%
\pgfsys@transformshift{2.050000in}{0.320408in}%
\pgftext[left,bottom]{\pgfimage[interpolate=true,width=0.130000in,height=2.590000in]{Perr_vs_dq_Ti_100K-img1.png}}%
\end{pgfscope}%
\begin{pgfscope}%
\pgfsetbuttcap%
\pgfsetroundjoin%
\definecolor{currentfill}{rgb}{0.000000,0.000000,0.000000}%
\pgfsetfillcolor{currentfill}%
\pgfsetlinewidth{0.803000pt}%
\definecolor{currentstroke}{rgb}{0.000000,0.000000,0.000000}%
\pgfsetstrokecolor{currentstroke}%
\pgfsetdash{}{0pt}%
\pgfsys@defobject{currentmarker}{\pgfqpoint{0.000000in}{0.000000in}}{\pgfqpoint{0.048611in}{0.000000in}}{%
\pgfpathmoveto{\pgfqpoint{0.000000in}{0.000000in}}%
\pgfpathlineto{\pgfqpoint{0.048611in}{0.000000in}}%
\pgfusepath{stroke,fill}%
}%
\begin{pgfscope}%
\pgfsys@transformshift{2.182730in}{0.319877in}%
\pgfsys@useobject{currentmarker}{}%
\end{pgfscope}%
\end{pgfscope}%
\begin{pgfscope}%
\pgftext[x=2.279953in,y=0.272050in,left,base]{\rmfamily\fontsize{10.000000}{12.000000}\selectfont \(\displaystyle 0\)}%
\end{pgfscope}%
\begin{pgfscope}%
\pgfsetbuttcap%
\pgfsetroundjoin%
\definecolor{currentfill}{rgb}{0.000000,0.000000,0.000000}%
\pgfsetfillcolor{currentfill}%
\pgfsetlinewidth{0.803000pt}%
\definecolor{currentstroke}{rgb}{0.000000,0.000000,0.000000}%
\pgfsetstrokecolor{currentstroke}%
\pgfsetdash{}{0pt}%
\pgfsys@defobject{currentmarker}{\pgfqpoint{0.000000in}{0.000000in}}{\pgfqpoint{0.048611in}{0.000000in}}{%
\pgfpathmoveto{\pgfqpoint{0.000000in}{0.000000in}}%
\pgfpathlineto{\pgfqpoint{0.048611in}{0.000000in}}%
\pgfusepath{stroke,fill}%
}%
\begin{pgfscope}%
\pgfsys@transformshift{2.182730in}{0.718755in}%
\pgfsys@useobject{currentmarker}{}%
\end{pgfscope}%
\end{pgfscope}%
\begin{pgfscope}%
\pgftext[x=2.279953in,y=0.670927in,left,base]{\rmfamily\fontsize{10.000000}{12.000000}\selectfont \(\displaystyle 2\)}%
\end{pgfscope}%
\begin{pgfscope}%
\pgfsetbuttcap%
\pgfsetroundjoin%
\definecolor{currentfill}{rgb}{0.000000,0.000000,0.000000}%
\pgfsetfillcolor{currentfill}%
\pgfsetlinewidth{0.803000pt}%
\definecolor{currentstroke}{rgb}{0.000000,0.000000,0.000000}%
\pgfsetstrokecolor{currentstroke}%
\pgfsetdash{}{0pt}%
\pgfsys@defobject{currentmarker}{\pgfqpoint{0.000000in}{0.000000in}}{\pgfqpoint{0.048611in}{0.000000in}}{%
\pgfpathmoveto{\pgfqpoint{0.000000in}{0.000000in}}%
\pgfpathlineto{\pgfqpoint{0.048611in}{0.000000in}}%
\pgfusepath{stroke,fill}%
}%
\begin{pgfscope}%
\pgfsys@transformshift{2.182730in}{1.117632in}%
\pgfsys@useobject{currentmarker}{}%
\end{pgfscope}%
\end{pgfscope}%
\begin{pgfscope}%
\pgftext[x=2.279953in,y=1.069804in,left,base]{\rmfamily\fontsize{10.000000}{12.000000}\selectfont \(\displaystyle 4\)}%
\end{pgfscope}%
\begin{pgfscope}%
\pgfsetbuttcap%
\pgfsetroundjoin%
\definecolor{currentfill}{rgb}{0.000000,0.000000,0.000000}%
\pgfsetfillcolor{currentfill}%
\pgfsetlinewidth{0.803000pt}%
\definecolor{currentstroke}{rgb}{0.000000,0.000000,0.000000}%
\pgfsetstrokecolor{currentstroke}%
\pgfsetdash{}{0pt}%
\pgfsys@defobject{currentmarker}{\pgfqpoint{0.000000in}{0.000000in}}{\pgfqpoint{0.048611in}{0.000000in}}{%
\pgfpathmoveto{\pgfqpoint{0.000000in}{0.000000in}}%
\pgfpathlineto{\pgfqpoint{0.048611in}{0.000000in}}%
\pgfusepath{stroke,fill}%
}%
\begin{pgfscope}%
\pgfsys@transformshift{2.182730in}{1.516509in}%
\pgfsys@useobject{currentmarker}{}%
\end{pgfscope}%
\end{pgfscope}%
\begin{pgfscope}%
\pgftext[x=2.279953in,y=1.468682in,left,base]{\rmfamily\fontsize{10.000000}{12.000000}\selectfont \(\displaystyle 6\)}%
\end{pgfscope}%
\begin{pgfscope}%
\pgfsetbuttcap%
\pgfsetroundjoin%
\definecolor{currentfill}{rgb}{0.000000,0.000000,0.000000}%
\pgfsetfillcolor{currentfill}%
\pgfsetlinewidth{0.803000pt}%
\definecolor{currentstroke}{rgb}{0.000000,0.000000,0.000000}%
\pgfsetstrokecolor{currentstroke}%
\pgfsetdash{}{0pt}%
\pgfsys@defobject{currentmarker}{\pgfqpoint{0.000000in}{0.000000in}}{\pgfqpoint{0.048611in}{0.000000in}}{%
\pgfpathmoveto{\pgfqpoint{0.000000in}{0.000000in}}%
\pgfpathlineto{\pgfqpoint{0.048611in}{0.000000in}}%
\pgfusepath{stroke,fill}%
}%
\begin{pgfscope}%
\pgfsys@transformshift{2.182730in}{1.915387in}%
\pgfsys@useobject{currentmarker}{}%
\end{pgfscope}%
\end{pgfscope}%
\begin{pgfscope}%
\pgftext[x=2.279953in,y=1.867559in,left,base]{\rmfamily\fontsize{10.000000}{12.000000}\selectfont \(\displaystyle 8\)}%
\end{pgfscope}%
\begin{pgfscope}%
\pgfsetbuttcap%
\pgfsetroundjoin%
\definecolor{currentfill}{rgb}{0.000000,0.000000,0.000000}%
\pgfsetfillcolor{currentfill}%
\pgfsetlinewidth{0.803000pt}%
\definecolor{currentstroke}{rgb}{0.000000,0.000000,0.000000}%
\pgfsetstrokecolor{currentstroke}%
\pgfsetdash{}{0pt}%
\pgfsys@defobject{currentmarker}{\pgfqpoint{0.000000in}{0.000000in}}{\pgfqpoint{0.048611in}{0.000000in}}{%
\pgfpathmoveto{\pgfqpoint{0.000000in}{0.000000in}}%
\pgfpathlineto{\pgfqpoint{0.048611in}{0.000000in}}%
\pgfusepath{stroke,fill}%
}%
\begin{pgfscope}%
\pgfsys@transformshift{2.182730in}{2.314264in}%
\pgfsys@useobject{currentmarker}{}%
\end{pgfscope}%
\end{pgfscope}%
\begin{pgfscope}%
\pgftext[x=2.279953in,y=2.266436in,left,base]{\rmfamily\fontsize{10.000000}{12.000000}\selectfont \(\displaystyle 10\)}%
\end{pgfscope}%
\begin{pgfscope}%
\pgfsetbuttcap%
\pgfsetroundjoin%
\definecolor{currentfill}{rgb}{0.000000,0.000000,0.000000}%
\pgfsetfillcolor{currentfill}%
\pgfsetlinewidth{0.803000pt}%
\definecolor{currentstroke}{rgb}{0.000000,0.000000,0.000000}%
\pgfsetstrokecolor{currentstroke}%
\pgfsetdash{}{0pt}%
\pgfsys@defobject{currentmarker}{\pgfqpoint{0.000000in}{0.000000in}}{\pgfqpoint{0.048611in}{0.000000in}}{%
\pgfpathmoveto{\pgfqpoint{0.000000in}{0.000000in}}%
\pgfpathlineto{\pgfqpoint{0.048611in}{0.000000in}}%
\pgfusepath{stroke,fill}%
}%
\begin{pgfscope}%
\pgfsys@transformshift{2.182730in}{2.713141in}%
\pgfsys@useobject{currentmarker}{}%
\end{pgfscope}%
\end{pgfscope}%
\begin{pgfscope}%
\pgftext[x=2.279953in,y=2.665314in,left,base]{\rmfamily\fontsize{10.000000}{12.000000}\selectfont \(\displaystyle 12\)}%
\end{pgfscope}%
\begin{pgfscope}%
\pgfsetbuttcap%
\pgfsetmiterjoin%
\pgfsetlinewidth{0.803000pt}%
\definecolor{currentstroke}{rgb}{0.000000,0.000000,0.000000}%
\pgfsetstrokecolor{currentstroke}%
\pgfsetdash{}{0pt}%
\pgfpathmoveto{\pgfqpoint{2.053095in}{0.319877in}}%
\pgfpathlineto{\pgfqpoint{2.053095in}{0.330005in}}%
\pgfpathlineto{\pgfqpoint{2.053095in}{2.902452in}}%
\pgfpathlineto{\pgfqpoint{2.053095in}{2.912580in}}%
\pgfpathlineto{\pgfqpoint{2.182730in}{2.912580in}}%
\pgfpathlineto{\pgfqpoint{2.182730in}{2.902452in}}%
\pgfpathlineto{\pgfqpoint{2.182730in}{0.330005in}}%
\pgfpathlineto{\pgfqpoint{2.182730in}{0.319877in}}%
\pgfpathclose%
\pgfusepath{stroke}%
\end{pgfscope}%
\end{pgfpicture}%
\makeatother%
\endgroup%

	\vspace*{-0.4cm}
	\caption{100 K. Bin size $0.0105e$}
	\end{subfigure}
	\hspace{0.6cm}
	\begin{subfigure}[b]{0.45\textwidth}
	\hspace*{-0.4cm}
	%% Creator: Matplotlib, PGF backend
%%
%% To include the figure in your LaTeX document, write
%%   \input{<filename>.pgf}
%%
%% Make sure the required packages are loaded in your preamble
%%   \usepackage{pgf}
%%
%% Figures using additional raster images can only be included by \input if
%% they are in the same directory as the main LaTeX file. For loading figures
%% from other directories you can use the `import` package
%%   \usepackage{import}
%% and then include the figures with
%%   \import{<path to file>}{<filename>.pgf}
%%
%% Matplotlib used the following preamble
%%   \usepackage[utf8x]{inputenc}
%%   \usepackage[T1]{fontenc}
%%
\begingroup%
\makeatletter%
\begin{pgfpicture}%
\pgfpathrectangle{\pgfpointorigin}{\pgfqpoint{2.518842in}{3.060408in}}%
\pgfusepath{use as bounding box, clip}%
\begin{pgfscope}%
\pgfsetbuttcap%
\pgfsetmiterjoin%
\definecolor{currentfill}{rgb}{1.000000,1.000000,1.000000}%
\pgfsetfillcolor{currentfill}%
\pgfsetlinewidth{0.000000pt}%
\definecolor{currentstroke}{rgb}{1.000000,1.000000,1.000000}%
\pgfsetstrokecolor{currentstroke}%
\pgfsetdash{}{0pt}%
\pgfpathmoveto{\pgfqpoint{0.000000in}{0.000000in}}%
\pgfpathlineto{\pgfqpoint{2.518842in}{0.000000in}}%
\pgfpathlineto{\pgfqpoint{2.518842in}{3.060408in}}%
\pgfpathlineto{\pgfqpoint{0.000000in}{3.060408in}}%
\pgfpathclose%
\pgfusepath{fill}%
\end{pgfscope}%
\begin{pgfscope}%
\pgfsetbuttcap%
\pgfsetmiterjoin%
\definecolor{currentfill}{rgb}{1.000000,1.000000,1.000000}%
\pgfsetfillcolor{currentfill}%
\pgfsetlinewidth{0.000000pt}%
\definecolor{currentstroke}{rgb}{0.000000,0.000000,0.000000}%
\pgfsetstrokecolor{currentstroke}%
\pgfsetstrokeopacity{0.000000}%
\pgfsetdash{}{0pt}%
\pgfpathmoveto{\pgfqpoint{0.374692in}{0.319877in}}%
\pgfpathlineto{\pgfqpoint{1.954366in}{0.319877in}}%
\pgfpathlineto{\pgfqpoint{1.954366in}{2.912580in}}%
\pgfpathlineto{\pgfqpoint{0.374692in}{2.912580in}}%
\pgfpathclose%
\pgfusepath{fill}%
\end{pgfscope}%
\begin{pgfscope}%
\pgfpathrectangle{\pgfqpoint{0.374692in}{0.319877in}}{\pgfqpoint{1.579674in}{2.592703in}} %
\pgfusepath{clip}%
\pgfsys@transformshift{0.374692in}{0.319877in}%
\pgftext[left,bottom]{\pgfimage[interpolate=true,width=1.580000in,height=2.600000in]{Perr_vs_dq_Ti_200K-img0.png}}%
\end{pgfscope}%
\begin{pgfscope}%
\pgfpathrectangle{\pgfqpoint{0.374692in}{0.319877in}}{\pgfqpoint{1.579674in}{2.592703in}} %
\pgfusepath{clip}%
\pgfsetbuttcap%
\pgfsetroundjoin%
\definecolor{currentfill}{rgb}{1.000000,0.752941,0.796078}%
\pgfsetfillcolor{currentfill}%
\pgfsetlinewidth{1.003750pt}%
\definecolor{currentstroke}{rgb}{1.000000,0.752941,0.796078}%
\pgfsetstrokecolor{currentstroke}%
\pgfsetdash{}{0pt}%
\pgfpathmoveto{\pgfqpoint{0.882444in}{1.509090in}}%
\pgfpathcurveto{\pgfqpoint{0.893494in}{1.509090in}}{\pgfqpoint{0.904093in}{1.513480in}}{\pgfqpoint{0.911907in}{1.521294in}}%
\pgfpathcurveto{\pgfqpoint{0.919721in}{1.529107in}}{\pgfqpoint{0.924111in}{1.539706in}}{\pgfqpoint{0.924111in}{1.550756in}}%
\pgfpathcurveto{\pgfqpoint{0.924111in}{1.561807in}}{\pgfqpoint{0.919721in}{1.572406in}}{\pgfqpoint{0.911907in}{1.580219in}}%
\pgfpathcurveto{\pgfqpoint{0.904093in}{1.588033in}}{\pgfqpoint{0.893494in}{1.592423in}}{\pgfqpoint{0.882444in}{1.592423in}}%
\pgfpathcurveto{\pgfqpoint{0.871394in}{1.592423in}}{\pgfqpoint{0.860795in}{1.588033in}}{\pgfqpoint{0.852981in}{1.580219in}}%
\pgfpathcurveto{\pgfqpoint{0.845168in}{1.572406in}}{\pgfqpoint{0.840778in}{1.561807in}}{\pgfqpoint{0.840778in}{1.550756in}}%
\pgfpathcurveto{\pgfqpoint{0.840778in}{1.539706in}}{\pgfqpoint{0.845168in}{1.529107in}}{\pgfqpoint{0.852981in}{1.521294in}}%
\pgfpathcurveto{\pgfqpoint{0.860795in}{1.513480in}}{\pgfqpoint{0.871394in}{1.509090in}}{\pgfqpoint{0.882444in}{1.509090in}}%
\pgfpathclose%
\pgfusepath{stroke,fill}%
\end{pgfscope}%
\begin{pgfscope}%
\pgfpathrectangle{\pgfqpoint{0.374692in}{0.319877in}}{\pgfqpoint{1.579674in}{2.592703in}} %
\pgfusepath{clip}%
\pgfsetbuttcap%
\pgfsetroundjoin%
\definecolor{currentfill}{rgb}{1.000000,0.752941,0.796078}%
\pgfsetfillcolor{currentfill}%
\pgfsetlinewidth{1.003750pt}%
\definecolor{currentstroke}{rgb}{1.000000,0.752941,0.796078}%
\pgfsetstrokecolor{currentstroke}%
\pgfsetdash{}{0pt}%
\pgfpathmoveto{\pgfqpoint{0.995278in}{1.410213in}}%
\pgfpathcurveto{\pgfqpoint{1.006328in}{1.410213in}}{\pgfqpoint{1.016927in}{1.414604in}}{\pgfqpoint{1.024741in}{1.422417in}}%
\pgfpathcurveto{\pgfqpoint{1.032554in}{1.430231in}}{\pgfqpoint{1.036945in}{1.440830in}}{\pgfqpoint{1.036945in}{1.451880in}}%
\pgfpathcurveto{\pgfqpoint{1.036945in}{1.462930in}}{\pgfqpoint{1.032554in}{1.473529in}}{\pgfqpoint{1.024741in}{1.481343in}}%
\pgfpathcurveto{\pgfqpoint{1.016927in}{1.489156in}}{\pgfqpoint{1.006328in}{1.493547in}}{\pgfqpoint{0.995278in}{1.493547in}}%
\pgfpathcurveto{\pgfqpoint{0.984228in}{1.493547in}}{\pgfqpoint{0.973629in}{1.489156in}}{\pgfqpoint{0.965815in}{1.481343in}}%
\pgfpathcurveto{\pgfqpoint{0.958002in}{1.473529in}}{\pgfqpoint{0.953611in}{1.462930in}}{\pgfqpoint{0.953611in}{1.451880in}}%
\pgfpathcurveto{\pgfqpoint{0.953611in}{1.440830in}}{\pgfqpoint{0.958002in}{1.430231in}}{\pgfqpoint{0.965815in}{1.422417in}}%
\pgfpathcurveto{\pgfqpoint{0.973629in}{1.414604in}}{\pgfqpoint{0.984228in}{1.410213in}}{\pgfqpoint{0.995278in}{1.410213in}}%
\pgfpathclose%
\pgfusepath{stroke,fill}%
\end{pgfscope}%
\begin{pgfscope}%
\pgfpathrectangle{\pgfqpoint{0.374692in}{0.319877in}}{\pgfqpoint{1.579674in}{2.592703in}} %
\pgfusepath{clip}%
\pgfsetbuttcap%
\pgfsetroundjoin%
\definecolor{currentfill}{rgb}{1.000000,0.752941,0.796078}%
\pgfsetfillcolor{currentfill}%
\pgfsetlinewidth{1.003750pt}%
\definecolor{currentstroke}{rgb}{1.000000,0.752941,0.796078}%
\pgfsetstrokecolor{currentstroke}%
\pgfsetdash{}{0pt}%
\pgfpathmoveto{\pgfqpoint{1.108112in}{1.468452in}}%
\pgfpathcurveto{\pgfqpoint{1.119162in}{1.468452in}}{\pgfqpoint{1.129761in}{1.472842in}}{\pgfqpoint{1.137575in}{1.480656in}}%
\pgfpathcurveto{\pgfqpoint{1.145388in}{1.488469in}}{\pgfqpoint{1.149779in}{1.499068in}}{\pgfqpoint{1.149779in}{1.510118in}}%
\pgfpathcurveto{\pgfqpoint{1.149779in}{1.521169in}}{\pgfqpoint{1.145388in}{1.531768in}}{\pgfqpoint{1.137575in}{1.539581in}}%
\pgfpathcurveto{\pgfqpoint{1.129761in}{1.547395in}}{\pgfqpoint{1.119162in}{1.551785in}}{\pgfqpoint{1.108112in}{1.551785in}}%
\pgfpathcurveto{\pgfqpoint{1.097062in}{1.551785in}}{\pgfqpoint{1.086463in}{1.547395in}}{\pgfqpoint{1.078649in}{1.539581in}}%
\pgfpathcurveto{\pgfqpoint{1.070836in}{1.531768in}}{\pgfqpoint{1.066445in}{1.521169in}}{\pgfqpoint{1.066445in}{1.510118in}}%
\pgfpathcurveto{\pgfqpoint{1.066445in}{1.499068in}}{\pgfqpoint{1.070836in}{1.488469in}}{\pgfqpoint{1.078649in}{1.480656in}}%
\pgfpathcurveto{\pgfqpoint{1.086463in}{1.472842in}}{\pgfqpoint{1.097062in}{1.468452in}}{\pgfqpoint{1.108112in}{1.468452in}}%
\pgfpathclose%
\pgfusepath{stroke,fill}%
\end{pgfscope}%
\begin{pgfscope}%
\pgfpathrectangle{\pgfqpoint{0.374692in}{0.319877in}}{\pgfqpoint{1.579674in}{2.592703in}} %
\pgfusepath{clip}%
\pgfsetbuttcap%
\pgfsetroundjoin%
\definecolor{currentfill}{rgb}{1.000000,0.752941,0.796078}%
\pgfsetfillcolor{currentfill}%
\pgfsetlinewidth{1.003750pt}%
\definecolor{currentstroke}{rgb}{1.000000,0.752941,0.796078}%
\pgfsetstrokecolor{currentstroke}%
\pgfsetdash{}{0pt}%
\pgfpathmoveto{\pgfqpoint{1.220946in}{1.489793in}}%
\pgfpathcurveto{\pgfqpoint{1.231996in}{1.489793in}}{\pgfqpoint{1.242595in}{1.494183in}}{\pgfqpoint{1.250409in}{1.501997in}}%
\pgfpathcurveto{\pgfqpoint{1.258222in}{1.509810in}}{\pgfqpoint{1.262612in}{1.520409in}}{\pgfqpoint{1.262612in}{1.531459in}}%
\pgfpathcurveto{\pgfqpoint{1.262612in}{1.542509in}}{\pgfqpoint{1.258222in}{1.553108in}}{\pgfqpoint{1.250409in}{1.560922in}}%
\pgfpathcurveto{\pgfqpoint{1.242595in}{1.568736in}}{\pgfqpoint{1.231996in}{1.573126in}}{\pgfqpoint{1.220946in}{1.573126in}}%
\pgfpathcurveto{\pgfqpoint{1.209896in}{1.573126in}}{\pgfqpoint{1.199297in}{1.568736in}}{\pgfqpoint{1.191483in}{1.560922in}}%
\pgfpathcurveto{\pgfqpoint{1.183669in}{1.553108in}}{\pgfqpoint{1.179279in}{1.542509in}}{\pgfqpoint{1.179279in}{1.531459in}}%
\pgfpathcurveto{\pgfqpoint{1.179279in}{1.520409in}}{\pgfqpoint{1.183669in}{1.509810in}}{\pgfqpoint{1.191483in}{1.501997in}}%
\pgfpathcurveto{\pgfqpoint{1.199297in}{1.494183in}}{\pgfqpoint{1.209896in}{1.489793in}}{\pgfqpoint{1.220946in}{1.489793in}}%
\pgfpathclose%
\pgfusepath{stroke,fill}%
\end{pgfscope}%
\begin{pgfscope}%
\pgfpathrectangle{\pgfqpoint{0.374692in}{0.319877in}}{\pgfqpoint{1.579674in}{2.592703in}} %
\pgfusepath{clip}%
\pgfsetbuttcap%
\pgfsetroundjoin%
\definecolor{currentfill}{rgb}{1.000000,0.752941,0.796078}%
\pgfsetfillcolor{currentfill}%
\pgfsetlinewidth{1.003750pt}%
\definecolor{currentstroke}{rgb}{1.000000,0.752941,0.796078}%
\pgfsetstrokecolor{currentstroke}%
\pgfsetdash{}{0pt}%
\pgfpathmoveto{\pgfqpoint{1.333780in}{1.563848in}}%
\pgfpathcurveto{\pgfqpoint{1.344830in}{1.563848in}}{\pgfqpoint{1.355429in}{1.568239in}}{\pgfqpoint{1.363242in}{1.576052in}}%
\pgfpathcurveto{\pgfqpoint{1.371056in}{1.583866in}}{\pgfqpoint{1.375446in}{1.594465in}}{\pgfqpoint{1.375446in}{1.605515in}}%
\pgfpathcurveto{\pgfqpoint{1.375446in}{1.616565in}}{\pgfqpoint{1.371056in}{1.627164in}}{\pgfqpoint{1.363242in}{1.634978in}}%
\pgfpathcurveto{\pgfqpoint{1.355429in}{1.642791in}}{\pgfqpoint{1.344830in}{1.647182in}}{\pgfqpoint{1.333780in}{1.647182in}}%
\pgfpathcurveto{\pgfqpoint{1.322729in}{1.647182in}}{\pgfqpoint{1.312130in}{1.642791in}}{\pgfqpoint{1.304317in}{1.634978in}}%
\pgfpathcurveto{\pgfqpoint{1.296503in}{1.627164in}}{\pgfqpoint{1.292113in}{1.616565in}}{\pgfqpoint{1.292113in}{1.605515in}}%
\pgfpathcurveto{\pgfqpoint{1.292113in}{1.594465in}}{\pgfqpoint{1.296503in}{1.583866in}}{\pgfqpoint{1.304317in}{1.576052in}}%
\pgfpathcurveto{\pgfqpoint{1.312130in}{1.568239in}}{\pgfqpoint{1.322729in}{1.563848in}}{\pgfqpoint{1.333780in}{1.563848in}}%
\pgfpathclose%
\pgfusepath{stroke,fill}%
\end{pgfscope}%
\begin{pgfscope}%
\pgfpathrectangle{\pgfqpoint{0.374692in}{0.319877in}}{\pgfqpoint{1.579674in}{2.592703in}} %
\pgfusepath{clip}%
\pgfsetbuttcap%
\pgfsetroundjoin%
\definecolor{currentfill}{rgb}{1.000000,0.752941,0.796078}%
\pgfsetfillcolor{currentfill}%
\pgfsetlinewidth{1.003750pt}%
\definecolor{currentstroke}{rgb}{1.000000,0.752941,0.796078}%
\pgfsetstrokecolor{currentstroke}%
\pgfsetdash{}{0pt}%
\pgfpathmoveto{\pgfqpoint{1.446613in}{1.679318in}}%
\pgfpathcurveto{\pgfqpoint{1.457664in}{1.679318in}}{\pgfqpoint{1.468263in}{1.683708in}}{\pgfqpoint{1.476076in}{1.691522in}}%
\pgfpathcurveto{\pgfqpoint{1.483890in}{1.699335in}}{\pgfqpoint{1.488280in}{1.709934in}}{\pgfqpoint{1.488280in}{1.720984in}}%
\pgfpathcurveto{\pgfqpoint{1.488280in}{1.732035in}}{\pgfqpoint{1.483890in}{1.742634in}}{\pgfqpoint{1.476076in}{1.750447in}}%
\pgfpathcurveto{\pgfqpoint{1.468263in}{1.758261in}}{\pgfqpoint{1.457664in}{1.762651in}}{\pgfqpoint{1.446613in}{1.762651in}}%
\pgfpathcurveto{\pgfqpoint{1.435563in}{1.762651in}}{\pgfqpoint{1.424964in}{1.758261in}}{\pgfqpoint{1.417151in}{1.750447in}}%
\pgfpathcurveto{\pgfqpoint{1.409337in}{1.742634in}}{\pgfqpoint{1.404947in}{1.732035in}}{\pgfqpoint{1.404947in}{1.720984in}}%
\pgfpathcurveto{\pgfqpoint{1.404947in}{1.709934in}}{\pgfqpoint{1.409337in}{1.699335in}}{\pgfqpoint{1.417151in}{1.691522in}}%
\pgfpathcurveto{\pgfqpoint{1.424964in}{1.683708in}}{\pgfqpoint{1.435563in}{1.679318in}}{\pgfqpoint{1.446613in}{1.679318in}}%
\pgfpathclose%
\pgfusepath{stroke,fill}%
\end{pgfscope}%
\begin{pgfscope}%
\pgfpathrectangle{\pgfqpoint{0.374692in}{0.319877in}}{\pgfqpoint{1.579674in}{2.592703in}} %
\pgfusepath{clip}%
\pgfsetbuttcap%
\pgfsetroundjoin%
\definecolor{currentfill}{rgb}{1.000000,0.752941,0.796078}%
\pgfsetfillcolor{currentfill}%
\pgfsetlinewidth{1.003750pt}%
\definecolor{currentstroke}{rgb}{1.000000,0.752941,0.796078}%
\pgfsetstrokecolor{currentstroke}%
\pgfsetdash{}{0pt}%
\pgfpathmoveto{\pgfqpoint{1.559447in}{2.045963in}}%
\pgfpathcurveto{\pgfqpoint{1.570497in}{2.045963in}}{\pgfqpoint{1.581096in}{2.050353in}}{\pgfqpoint{1.588910in}{2.058166in}}%
\pgfpathcurveto{\pgfqpoint{1.596724in}{2.065980in}}{\pgfqpoint{1.601114in}{2.076579in}}{\pgfqpoint{1.601114in}{2.087629in}}%
\pgfpathcurveto{\pgfqpoint{1.601114in}{2.098679in}}{\pgfqpoint{1.596724in}{2.109278in}}{\pgfqpoint{1.588910in}{2.117092in}}%
\pgfpathcurveto{\pgfqpoint{1.581096in}{2.124906in}}{\pgfqpoint{1.570497in}{2.129296in}}{\pgfqpoint{1.559447in}{2.129296in}}%
\pgfpathcurveto{\pgfqpoint{1.548397in}{2.129296in}}{\pgfqpoint{1.537798in}{2.124906in}}{\pgfqpoint{1.529985in}{2.117092in}}%
\pgfpathcurveto{\pgfqpoint{1.522171in}{2.109278in}}{\pgfqpoint{1.517781in}{2.098679in}}{\pgfqpoint{1.517781in}{2.087629in}}%
\pgfpathcurveto{\pgfqpoint{1.517781in}{2.076579in}}{\pgfqpoint{1.522171in}{2.065980in}}{\pgfqpoint{1.529985in}{2.058166in}}%
\pgfpathcurveto{\pgfqpoint{1.537798in}{2.050353in}}{\pgfqpoint{1.548397in}{2.045963in}}{\pgfqpoint{1.559447in}{2.045963in}}%
\pgfpathclose%
\pgfusepath{stroke,fill}%
\end{pgfscope}%
\begin{pgfscope}%
\pgfsetbuttcap%
\pgfsetroundjoin%
\definecolor{currentfill}{rgb}{0.000000,0.000000,0.000000}%
\pgfsetfillcolor{currentfill}%
\pgfsetlinewidth{0.803000pt}%
\definecolor{currentstroke}{rgb}{0.000000,0.000000,0.000000}%
\pgfsetstrokecolor{currentstroke}%
\pgfsetdash{}{0pt}%
\pgfsys@defobject{currentmarker}{\pgfqpoint{0.000000in}{-0.048611in}}{\pgfqpoint{0.000000in}{0.000000in}}{%
\pgfpathmoveto{\pgfqpoint{0.000000in}{0.000000in}}%
\pgfpathlineto{\pgfqpoint{0.000000in}{-0.048611in}}%
\pgfusepath{stroke,fill}%
}%
\begin{pgfscope}%
\pgfsys@transformshift{0.670881in}{0.319877in}%
\pgfsys@useobject{currentmarker}{}%
\end{pgfscope}%
\end{pgfscope}%
\begin{pgfscope}%
\pgftext[x=0.670881in,y=0.222655in,,top]{\rmfamily\fontsize{10.000000}{12.000000}\selectfont \(\displaystyle -0.05\)}%
\end{pgfscope}%
\begin{pgfscope}%
\pgfsetbuttcap%
\pgfsetroundjoin%
\definecolor{currentfill}{rgb}{0.000000,0.000000,0.000000}%
\pgfsetfillcolor{currentfill}%
\pgfsetlinewidth{0.803000pt}%
\definecolor{currentstroke}{rgb}{0.000000,0.000000,0.000000}%
\pgfsetstrokecolor{currentstroke}%
\pgfsetdash{}{0pt}%
\pgfsys@defobject{currentmarker}{\pgfqpoint{0.000000in}{-0.048611in}}{\pgfqpoint{0.000000in}{0.000000in}}{%
\pgfpathmoveto{\pgfqpoint{0.000000in}{0.000000in}}%
\pgfpathlineto{\pgfqpoint{0.000000in}{-0.048611in}}%
\pgfusepath{stroke,fill}%
}%
\begin{pgfscope}%
\pgfsys@transformshift{1.164529in}{0.319877in}%
\pgfsys@useobject{currentmarker}{}%
\end{pgfscope}%
\end{pgfscope}%
\begin{pgfscope}%
\pgftext[x=1.164529in,y=0.222655in,,top]{\rmfamily\fontsize{10.000000}{12.000000}\selectfont \(\displaystyle 0.00\)}%
\end{pgfscope}%
\begin{pgfscope}%
\pgfsetbuttcap%
\pgfsetroundjoin%
\definecolor{currentfill}{rgb}{0.000000,0.000000,0.000000}%
\pgfsetfillcolor{currentfill}%
\pgfsetlinewidth{0.803000pt}%
\definecolor{currentstroke}{rgb}{0.000000,0.000000,0.000000}%
\pgfsetstrokecolor{currentstroke}%
\pgfsetdash{}{0pt}%
\pgfsys@defobject{currentmarker}{\pgfqpoint{0.000000in}{-0.048611in}}{\pgfqpoint{0.000000in}{0.000000in}}{%
\pgfpathmoveto{\pgfqpoint{0.000000in}{0.000000in}}%
\pgfpathlineto{\pgfqpoint{0.000000in}{-0.048611in}}%
\pgfusepath{stroke,fill}%
}%
\begin{pgfscope}%
\pgfsys@transformshift{1.658177in}{0.319877in}%
\pgfsys@useobject{currentmarker}{}%
\end{pgfscope}%
\end{pgfscope}%
\begin{pgfscope}%
\pgftext[x=1.658177in,y=0.222655in,,top]{\rmfamily\fontsize{10.000000}{12.000000}\selectfont \(\displaystyle 0.05\)}%
\end{pgfscope}%
\begin{pgfscope}%
\pgfsetbuttcap%
\pgfsetroundjoin%
\definecolor{currentfill}{rgb}{0.000000,0.000000,0.000000}%
\pgfsetfillcolor{currentfill}%
\pgfsetlinewidth{0.803000pt}%
\definecolor{currentstroke}{rgb}{0.000000,0.000000,0.000000}%
\pgfsetstrokecolor{currentstroke}%
\pgfsetdash{}{0pt}%
\pgfsys@defobject{currentmarker}{\pgfqpoint{-0.048611in}{0.000000in}}{\pgfqpoint{0.000000in}{0.000000in}}{%
\pgfpathmoveto{\pgfqpoint{0.000000in}{0.000000in}}%
\pgfpathlineto{\pgfqpoint{-0.048611in}{0.000000in}}%
\pgfusepath{stroke,fill}%
}%
\begin{pgfscope}%
\pgfsys@transformshift{0.374692in}{0.319877in}%
\pgfsys@useobject{currentmarker}{}%
\end{pgfscope}%
\end{pgfscope}%
\begin{pgfscope}%
\pgftext[x=0.100000in,y=0.272050in,left,base]{\rmfamily\fontsize{10.000000}{12.000000}\selectfont \(\displaystyle 0.0\)}%
\end{pgfscope}%
\begin{pgfscope}%
\pgfsetbuttcap%
\pgfsetroundjoin%
\definecolor{currentfill}{rgb}{0.000000,0.000000,0.000000}%
\pgfsetfillcolor{currentfill}%
\pgfsetlinewidth{0.803000pt}%
\definecolor{currentstroke}{rgb}{0.000000,0.000000,0.000000}%
\pgfsetstrokecolor{currentstroke}%
\pgfsetdash{}{0pt}%
\pgfsys@defobject{currentmarker}{\pgfqpoint{-0.048611in}{0.000000in}}{\pgfqpoint{0.000000in}{0.000000in}}{%
\pgfpathmoveto{\pgfqpoint{0.000000in}{0.000000in}}%
\pgfpathlineto{\pgfqpoint{-0.048611in}{0.000000in}}%
\pgfusepath{stroke,fill}%
}%
\begin{pgfscope}%
\pgfsys@transformshift{0.374692in}{0.838418in}%
\pgfsys@useobject{currentmarker}{}%
\end{pgfscope}%
\end{pgfscope}%
\begin{pgfscope}%
\pgftext[x=0.100000in,y=0.790590in,left,base]{\rmfamily\fontsize{10.000000}{12.000000}\selectfont \(\displaystyle 0.2\)}%
\end{pgfscope}%
\begin{pgfscope}%
\pgfsetbuttcap%
\pgfsetroundjoin%
\definecolor{currentfill}{rgb}{0.000000,0.000000,0.000000}%
\pgfsetfillcolor{currentfill}%
\pgfsetlinewidth{0.803000pt}%
\definecolor{currentstroke}{rgb}{0.000000,0.000000,0.000000}%
\pgfsetstrokecolor{currentstroke}%
\pgfsetdash{}{0pt}%
\pgfsys@defobject{currentmarker}{\pgfqpoint{-0.048611in}{0.000000in}}{\pgfqpoint{0.000000in}{0.000000in}}{%
\pgfpathmoveto{\pgfqpoint{0.000000in}{0.000000in}}%
\pgfpathlineto{\pgfqpoint{-0.048611in}{0.000000in}}%
\pgfusepath{stroke,fill}%
}%
\begin{pgfscope}%
\pgfsys@transformshift{0.374692in}{1.356958in}%
\pgfsys@useobject{currentmarker}{}%
\end{pgfscope}%
\end{pgfscope}%
\begin{pgfscope}%
\pgftext[x=0.100000in,y=1.309131in,left,base]{\rmfamily\fontsize{10.000000}{12.000000}\selectfont \(\displaystyle 0.4\)}%
\end{pgfscope}%
\begin{pgfscope}%
\pgfsetbuttcap%
\pgfsetroundjoin%
\definecolor{currentfill}{rgb}{0.000000,0.000000,0.000000}%
\pgfsetfillcolor{currentfill}%
\pgfsetlinewidth{0.803000pt}%
\definecolor{currentstroke}{rgb}{0.000000,0.000000,0.000000}%
\pgfsetstrokecolor{currentstroke}%
\pgfsetdash{}{0pt}%
\pgfsys@defobject{currentmarker}{\pgfqpoint{-0.048611in}{0.000000in}}{\pgfqpoint{0.000000in}{0.000000in}}{%
\pgfpathmoveto{\pgfqpoint{0.000000in}{0.000000in}}%
\pgfpathlineto{\pgfqpoint{-0.048611in}{0.000000in}}%
\pgfusepath{stroke,fill}%
}%
\begin{pgfscope}%
\pgfsys@transformshift{0.374692in}{1.875499in}%
\pgfsys@useobject{currentmarker}{}%
\end{pgfscope}%
\end{pgfscope}%
\begin{pgfscope}%
\pgftext[x=0.100000in,y=1.827671in,left,base]{\rmfamily\fontsize{10.000000}{12.000000}\selectfont \(\displaystyle 0.6\)}%
\end{pgfscope}%
\begin{pgfscope}%
\pgfsetbuttcap%
\pgfsetroundjoin%
\definecolor{currentfill}{rgb}{0.000000,0.000000,0.000000}%
\pgfsetfillcolor{currentfill}%
\pgfsetlinewidth{0.803000pt}%
\definecolor{currentstroke}{rgb}{0.000000,0.000000,0.000000}%
\pgfsetstrokecolor{currentstroke}%
\pgfsetdash{}{0pt}%
\pgfsys@defobject{currentmarker}{\pgfqpoint{-0.048611in}{0.000000in}}{\pgfqpoint{0.000000in}{0.000000in}}{%
\pgfpathmoveto{\pgfqpoint{0.000000in}{0.000000in}}%
\pgfpathlineto{\pgfqpoint{-0.048611in}{0.000000in}}%
\pgfusepath{stroke,fill}%
}%
\begin{pgfscope}%
\pgfsys@transformshift{0.374692in}{2.394040in}%
\pgfsys@useobject{currentmarker}{}%
\end{pgfscope}%
\end{pgfscope}%
\begin{pgfscope}%
\pgftext[x=0.100000in,y=2.346212in,left,base]{\rmfamily\fontsize{10.000000}{12.000000}\selectfont \(\displaystyle 0.8\)}%
\end{pgfscope}%
\begin{pgfscope}%
\pgfsetbuttcap%
\pgfsetroundjoin%
\definecolor{currentfill}{rgb}{0.000000,0.000000,0.000000}%
\pgfsetfillcolor{currentfill}%
\pgfsetlinewidth{0.803000pt}%
\definecolor{currentstroke}{rgb}{0.000000,0.000000,0.000000}%
\pgfsetstrokecolor{currentstroke}%
\pgfsetdash{}{0pt}%
\pgfsys@defobject{currentmarker}{\pgfqpoint{-0.048611in}{0.000000in}}{\pgfqpoint{0.000000in}{0.000000in}}{%
\pgfpathmoveto{\pgfqpoint{0.000000in}{0.000000in}}%
\pgfpathlineto{\pgfqpoint{-0.048611in}{0.000000in}}%
\pgfusepath{stroke,fill}%
}%
\begin{pgfscope}%
\pgfsys@transformshift{0.374692in}{2.912580in}%
\pgfsys@useobject{currentmarker}{}%
\end{pgfscope}%
\end{pgfscope}%
\begin{pgfscope}%
\pgftext[x=0.100000in,y=2.864752in,left,base]{\rmfamily\fontsize{10.000000}{12.000000}\selectfont \(\displaystyle 1.0\)}%
\end{pgfscope}%
\begin{pgfscope}%
\pgfsetrectcap%
\pgfsetmiterjoin%
\pgfsetlinewidth{0.803000pt}%
\definecolor{currentstroke}{rgb}{0.000000,0.000000,0.000000}%
\pgfsetstrokecolor{currentstroke}%
\pgfsetdash{}{0pt}%
\pgfpathmoveto{\pgfqpoint{0.374692in}{0.319877in}}%
\pgfpathlineto{\pgfqpoint{0.374692in}{2.912580in}}%
\pgfusepath{stroke}%
\end{pgfscope}%
\begin{pgfscope}%
\pgfsetrectcap%
\pgfsetmiterjoin%
\pgfsetlinewidth{0.803000pt}%
\definecolor{currentstroke}{rgb}{0.000000,0.000000,0.000000}%
\pgfsetstrokecolor{currentstroke}%
\pgfsetdash{}{0pt}%
\pgfpathmoveto{\pgfqpoint{1.954366in}{0.319877in}}%
\pgfpathlineto{\pgfqpoint{1.954366in}{2.912580in}}%
\pgfusepath{stroke}%
\end{pgfscope}%
\begin{pgfscope}%
\pgfsetrectcap%
\pgfsetmiterjoin%
\pgfsetlinewidth{0.803000pt}%
\definecolor{currentstroke}{rgb}{0.000000,0.000000,0.000000}%
\pgfsetstrokecolor{currentstroke}%
\pgfsetdash{}{0pt}%
\pgfpathmoveto{\pgfqpoint{0.374692in}{0.319877in}}%
\pgfpathlineto{\pgfqpoint{1.954366in}{0.319877in}}%
\pgfusepath{stroke}%
\end{pgfscope}%
\begin{pgfscope}%
\pgfsetrectcap%
\pgfsetmiterjoin%
\pgfsetlinewidth{0.803000pt}%
\definecolor{currentstroke}{rgb}{0.000000,0.000000,0.000000}%
\pgfsetstrokecolor{currentstroke}%
\pgfsetdash{}{0pt}%
\pgfpathmoveto{\pgfqpoint{0.374692in}{2.912580in}}%
\pgfpathlineto{\pgfqpoint{1.954366in}{2.912580in}}%
\pgfusepath{stroke}%
\end{pgfscope}%
\begin{pgfscope}%
\pgfpathrectangle{\pgfqpoint{2.053095in}{0.319877in}}{\pgfqpoint{0.129635in}{2.592703in}} %
\pgfusepath{clip}%
\pgfsetbuttcap%
\pgfsetmiterjoin%
\definecolor{currentfill}{rgb}{1.000000,1.000000,1.000000}%
\pgfsetfillcolor{currentfill}%
\pgfsetlinewidth{0.010037pt}%
\definecolor{currentstroke}{rgb}{1.000000,1.000000,1.000000}%
\pgfsetstrokecolor{currentstroke}%
\pgfsetdash{}{0pt}%
\pgfpathmoveto{\pgfqpoint{2.053095in}{0.319877in}}%
\pgfpathlineto{\pgfqpoint{2.053095in}{0.330005in}}%
\pgfpathlineto{\pgfqpoint{2.053095in}{2.902452in}}%
\pgfpathlineto{\pgfqpoint{2.053095in}{2.912580in}}%
\pgfpathlineto{\pgfqpoint{2.182730in}{2.912580in}}%
\pgfpathlineto{\pgfqpoint{2.182730in}{2.902452in}}%
\pgfpathlineto{\pgfqpoint{2.182730in}{0.330005in}}%
\pgfpathlineto{\pgfqpoint{2.182730in}{0.319877in}}%
\pgfpathclose%
\pgfusepath{stroke,fill}%
\end{pgfscope}%
\begin{pgfscope}%
\pgfsys@transformshift{2.050000in}{0.320408in}%
\pgftext[left,bottom]{\pgfimage[interpolate=true,width=0.130000in,height=2.590000in]{Perr_vs_dq_Ti_200K-img1.png}}%
\end{pgfscope}%
\begin{pgfscope}%
\pgfsetbuttcap%
\pgfsetroundjoin%
\definecolor{currentfill}{rgb}{0.000000,0.000000,0.000000}%
\pgfsetfillcolor{currentfill}%
\pgfsetlinewidth{0.803000pt}%
\definecolor{currentstroke}{rgb}{0.000000,0.000000,0.000000}%
\pgfsetstrokecolor{currentstroke}%
\pgfsetdash{}{0pt}%
\pgfsys@defobject{currentmarker}{\pgfqpoint{0.000000in}{0.000000in}}{\pgfqpoint{0.048611in}{0.000000in}}{%
\pgfpathmoveto{\pgfqpoint{0.000000in}{0.000000in}}%
\pgfpathlineto{\pgfqpoint{0.048611in}{0.000000in}}%
\pgfusepath{stroke,fill}%
}%
\begin{pgfscope}%
\pgfsys@transformshift{2.182730in}{0.319877in}%
\pgfsys@useobject{currentmarker}{}%
\end{pgfscope}%
\end{pgfscope}%
\begin{pgfscope}%
\pgftext[x=2.279953in,y=0.272050in,left,base]{\rmfamily\fontsize{10.000000}{12.000000}\selectfont \(\displaystyle 0\)}%
\end{pgfscope}%
\begin{pgfscope}%
\pgfsetbuttcap%
\pgfsetroundjoin%
\definecolor{currentfill}{rgb}{0.000000,0.000000,0.000000}%
\pgfsetfillcolor{currentfill}%
\pgfsetlinewidth{0.803000pt}%
\definecolor{currentstroke}{rgb}{0.000000,0.000000,0.000000}%
\pgfsetstrokecolor{currentstroke}%
\pgfsetdash{}{0pt}%
\pgfsys@defobject{currentmarker}{\pgfqpoint{0.000000in}{0.000000in}}{\pgfqpoint{0.048611in}{0.000000in}}{%
\pgfpathmoveto{\pgfqpoint{0.000000in}{0.000000in}}%
\pgfpathlineto{\pgfqpoint{0.048611in}{0.000000in}}%
\pgfusepath{stroke,fill}%
}%
\begin{pgfscope}%
\pgfsys@transformshift{2.182730in}{0.718755in}%
\pgfsys@useobject{currentmarker}{}%
\end{pgfscope}%
\end{pgfscope}%
\begin{pgfscope}%
\pgftext[x=2.279953in,y=0.670927in,left,base]{\rmfamily\fontsize{10.000000}{12.000000}\selectfont \(\displaystyle 2\)}%
\end{pgfscope}%
\begin{pgfscope}%
\pgfsetbuttcap%
\pgfsetroundjoin%
\definecolor{currentfill}{rgb}{0.000000,0.000000,0.000000}%
\pgfsetfillcolor{currentfill}%
\pgfsetlinewidth{0.803000pt}%
\definecolor{currentstroke}{rgb}{0.000000,0.000000,0.000000}%
\pgfsetstrokecolor{currentstroke}%
\pgfsetdash{}{0pt}%
\pgfsys@defobject{currentmarker}{\pgfqpoint{0.000000in}{0.000000in}}{\pgfqpoint{0.048611in}{0.000000in}}{%
\pgfpathmoveto{\pgfqpoint{0.000000in}{0.000000in}}%
\pgfpathlineto{\pgfqpoint{0.048611in}{0.000000in}}%
\pgfusepath{stroke,fill}%
}%
\begin{pgfscope}%
\pgfsys@transformshift{2.182730in}{1.117632in}%
\pgfsys@useobject{currentmarker}{}%
\end{pgfscope}%
\end{pgfscope}%
\begin{pgfscope}%
\pgftext[x=2.279953in,y=1.069804in,left,base]{\rmfamily\fontsize{10.000000}{12.000000}\selectfont \(\displaystyle 4\)}%
\end{pgfscope}%
\begin{pgfscope}%
\pgfsetbuttcap%
\pgfsetroundjoin%
\definecolor{currentfill}{rgb}{0.000000,0.000000,0.000000}%
\pgfsetfillcolor{currentfill}%
\pgfsetlinewidth{0.803000pt}%
\definecolor{currentstroke}{rgb}{0.000000,0.000000,0.000000}%
\pgfsetstrokecolor{currentstroke}%
\pgfsetdash{}{0pt}%
\pgfsys@defobject{currentmarker}{\pgfqpoint{0.000000in}{0.000000in}}{\pgfqpoint{0.048611in}{0.000000in}}{%
\pgfpathmoveto{\pgfqpoint{0.000000in}{0.000000in}}%
\pgfpathlineto{\pgfqpoint{0.048611in}{0.000000in}}%
\pgfusepath{stroke,fill}%
}%
\begin{pgfscope}%
\pgfsys@transformshift{2.182730in}{1.516509in}%
\pgfsys@useobject{currentmarker}{}%
\end{pgfscope}%
\end{pgfscope}%
\begin{pgfscope}%
\pgftext[x=2.279953in,y=1.468682in,left,base]{\rmfamily\fontsize{10.000000}{12.000000}\selectfont \(\displaystyle 6\)}%
\end{pgfscope}%
\begin{pgfscope}%
\pgfsetbuttcap%
\pgfsetroundjoin%
\definecolor{currentfill}{rgb}{0.000000,0.000000,0.000000}%
\pgfsetfillcolor{currentfill}%
\pgfsetlinewidth{0.803000pt}%
\definecolor{currentstroke}{rgb}{0.000000,0.000000,0.000000}%
\pgfsetstrokecolor{currentstroke}%
\pgfsetdash{}{0pt}%
\pgfsys@defobject{currentmarker}{\pgfqpoint{0.000000in}{0.000000in}}{\pgfqpoint{0.048611in}{0.000000in}}{%
\pgfpathmoveto{\pgfqpoint{0.000000in}{0.000000in}}%
\pgfpathlineto{\pgfqpoint{0.048611in}{0.000000in}}%
\pgfusepath{stroke,fill}%
}%
\begin{pgfscope}%
\pgfsys@transformshift{2.182730in}{1.915387in}%
\pgfsys@useobject{currentmarker}{}%
\end{pgfscope}%
\end{pgfscope}%
\begin{pgfscope}%
\pgftext[x=2.279953in,y=1.867559in,left,base]{\rmfamily\fontsize{10.000000}{12.000000}\selectfont \(\displaystyle 8\)}%
\end{pgfscope}%
\begin{pgfscope}%
\pgfsetbuttcap%
\pgfsetroundjoin%
\definecolor{currentfill}{rgb}{0.000000,0.000000,0.000000}%
\pgfsetfillcolor{currentfill}%
\pgfsetlinewidth{0.803000pt}%
\definecolor{currentstroke}{rgb}{0.000000,0.000000,0.000000}%
\pgfsetstrokecolor{currentstroke}%
\pgfsetdash{}{0pt}%
\pgfsys@defobject{currentmarker}{\pgfqpoint{0.000000in}{0.000000in}}{\pgfqpoint{0.048611in}{0.000000in}}{%
\pgfpathmoveto{\pgfqpoint{0.000000in}{0.000000in}}%
\pgfpathlineto{\pgfqpoint{0.048611in}{0.000000in}}%
\pgfusepath{stroke,fill}%
}%
\begin{pgfscope}%
\pgfsys@transformshift{2.182730in}{2.314264in}%
\pgfsys@useobject{currentmarker}{}%
\end{pgfscope}%
\end{pgfscope}%
\begin{pgfscope}%
\pgftext[x=2.279953in,y=2.266436in,left,base]{\rmfamily\fontsize{10.000000}{12.000000}\selectfont \(\displaystyle 10\)}%
\end{pgfscope}%
\begin{pgfscope}%
\pgfsetbuttcap%
\pgfsetroundjoin%
\definecolor{currentfill}{rgb}{0.000000,0.000000,0.000000}%
\pgfsetfillcolor{currentfill}%
\pgfsetlinewidth{0.803000pt}%
\definecolor{currentstroke}{rgb}{0.000000,0.000000,0.000000}%
\pgfsetstrokecolor{currentstroke}%
\pgfsetdash{}{0pt}%
\pgfsys@defobject{currentmarker}{\pgfqpoint{0.000000in}{0.000000in}}{\pgfqpoint{0.048611in}{0.000000in}}{%
\pgfpathmoveto{\pgfqpoint{0.000000in}{0.000000in}}%
\pgfpathlineto{\pgfqpoint{0.048611in}{0.000000in}}%
\pgfusepath{stroke,fill}%
}%
\begin{pgfscope}%
\pgfsys@transformshift{2.182730in}{2.713141in}%
\pgfsys@useobject{currentmarker}{}%
\end{pgfscope}%
\end{pgfscope}%
\begin{pgfscope}%
\pgftext[x=2.279953in,y=2.665314in,left,base]{\rmfamily\fontsize{10.000000}{12.000000}\selectfont \(\displaystyle 12\)}%
\end{pgfscope}%
\begin{pgfscope}%
\pgfsetbuttcap%
\pgfsetmiterjoin%
\pgfsetlinewidth{0.803000pt}%
\definecolor{currentstroke}{rgb}{0.000000,0.000000,0.000000}%
\pgfsetstrokecolor{currentstroke}%
\pgfsetdash{}{0pt}%
\pgfpathmoveto{\pgfqpoint{2.053095in}{0.319877in}}%
\pgfpathlineto{\pgfqpoint{2.053095in}{0.330005in}}%
\pgfpathlineto{\pgfqpoint{2.053095in}{2.902452in}}%
\pgfpathlineto{\pgfqpoint{2.053095in}{2.912580in}}%
\pgfpathlineto{\pgfqpoint{2.182730in}{2.912580in}}%
\pgfpathlineto{\pgfqpoint{2.182730in}{2.902452in}}%
\pgfpathlineto{\pgfqpoint{2.182730in}{0.330005in}}%
\pgfpathlineto{\pgfqpoint{2.182730in}{0.319877in}}%
\pgfpathclose%
\pgfusepath{stroke}%
\end{pgfscope}%
\end{pgfpicture}%
\makeatother%
\endgroup%

	\vspace*{-0.4cm}
	\caption{200 K. Bin size $0.011e$}
	\end{subfigure}
	\quad
	\begin{subfigure}[b]{0.45\textwidth}
	\hspace*{-0.4cm}
	%% Creator: Matplotlib, PGF backend
%%
%% To include the figure in your LaTeX document, write
%%   \input{<filename>.pgf}
%%
%% Make sure the required packages are loaded in your preamble
%%   \usepackage{pgf}
%%
%% Figures using additional raster images can only be included by \input if
%% they are in the same directory as the main LaTeX file. For loading figures
%% from other directories you can use the `import` package
%%   \usepackage{import}
%% and then include the figures with
%%   \import{<path to file>}{<filename>.pgf}
%%
%% Matplotlib used the following preamble
%%   \usepackage[utf8x]{inputenc}
%%   \usepackage[T1]{fontenc}
%%
\begingroup%
\makeatletter%
\begin{pgfpicture}%
\pgfpathrectangle{\pgfpointorigin}{\pgfqpoint{2.518842in}{3.060408in}}%
\pgfusepath{use as bounding box, clip}%
\begin{pgfscope}%
\pgfsetbuttcap%
\pgfsetmiterjoin%
\definecolor{currentfill}{rgb}{1.000000,1.000000,1.000000}%
\pgfsetfillcolor{currentfill}%
\pgfsetlinewidth{0.000000pt}%
\definecolor{currentstroke}{rgb}{1.000000,1.000000,1.000000}%
\pgfsetstrokecolor{currentstroke}%
\pgfsetdash{}{0pt}%
\pgfpathmoveto{\pgfqpoint{0.000000in}{0.000000in}}%
\pgfpathlineto{\pgfqpoint{2.518842in}{0.000000in}}%
\pgfpathlineto{\pgfqpoint{2.518842in}{3.060408in}}%
\pgfpathlineto{\pgfqpoint{0.000000in}{3.060408in}}%
\pgfpathclose%
\pgfusepath{fill}%
\end{pgfscope}%
\begin{pgfscope}%
\pgfsetbuttcap%
\pgfsetmiterjoin%
\definecolor{currentfill}{rgb}{1.000000,1.000000,1.000000}%
\pgfsetfillcolor{currentfill}%
\pgfsetlinewidth{0.000000pt}%
\definecolor{currentstroke}{rgb}{0.000000,0.000000,0.000000}%
\pgfsetstrokecolor{currentstroke}%
\pgfsetstrokeopacity{0.000000}%
\pgfsetdash{}{0pt}%
\pgfpathmoveto{\pgfqpoint{0.374692in}{0.319877in}}%
\pgfpathlineto{\pgfqpoint{1.954366in}{0.319877in}}%
\pgfpathlineto{\pgfqpoint{1.954366in}{2.912580in}}%
\pgfpathlineto{\pgfqpoint{0.374692in}{2.912580in}}%
\pgfpathclose%
\pgfusepath{fill}%
\end{pgfscope}%
\begin{pgfscope}%
\pgfpathrectangle{\pgfqpoint{0.374692in}{0.319877in}}{\pgfqpoint{1.579674in}{2.592703in}} %
\pgfusepath{clip}%
\pgfsys@transformshift{0.374692in}{0.319877in}%
\pgftext[left,bottom]{\pgfimage[interpolate=true,width=1.580000in,height=2.600000in]{Perr_vs_dq_Ti_300K-img0.png}}%
\end{pgfscope}%
\begin{pgfscope}%
\pgfpathrectangle{\pgfqpoint{0.374692in}{0.319877in}}{\pgfqpoint{1.579674in}{2.592703in}} %
\pgfusepath{clip}%
\pgfsetbuttcap%
\pgfsetroundjoin%
\definecolor{currentfill}{rgb}{1.000000,0.752941,0.796078}%
\pgfsetfillcolor{currentfill}%
\pgfsetlinewidth{1.003750pt}%
\definecolor{currentstroke}{rgb}{1.000000,0.752941,0.796078}%
\pgfsetstrokecolor{currentstroke}%
\pgfsetdash{}{0pt}%
\pgfpathmoveto{\pgfqpoint{0.733709in}{1.142445in}}%
\pgfpathcurveto{\pgfqpoint{0.744759in}{1.142445in}}{\pgfqpoint{0.755358in}{1.146835in}}{\pgfqpoint{0.763171in}{1.154649in}}%
\pgfpathcurveto{\pgfqpoint{0.770985in}{1.162462in}}{\pgfqpoint{0.775375in}{1.173061in}}{\pgfqpoint{0.775375in}{1.184112in}}%
\pgfpathcurveto{\pgfqpoint{0.775375in}{1.195162in}}{\pgfqpoint{0.770985in}{1.205761in}}{\pgfqpoint{0.763171in}{1.213574in}}%
\pgfpathcurveto{\pgfqpoint{0.755358in}{1.221388in}}{\pgfqpoint{0.744759in}{1.225778in}}{\pgfqpoint{0.733709in}{1.225778in}}%
\pgfpathcurveto{\pgfqpoint{0.722659in}{1.225778in}}{\pgfqpoint{0.712060in}{1.221388in}}{\pgfqpoint{0.704246in}{1.213574in}}%
\pgfpathcurveto{\pgfqpoint{0.696432in}{1.205761in}}{\pgfqpoint{0.692042in}{1.195162in}}{\pgfqpoint{0.692042in}{1.184112in}}%
\pgfpathcurveto{\pgfqpoint{0.692042in}{1.173061in}}{\pgfqpoint{0.696432in}{1.162462in}}{\pgfqpoint{0.704246in}{1.154649in}}%
\pgfpathcurveto{\pgfqpoint{0.712060in}{1.146835in}}{\pgfqpoint{0.722659in}{1.142445in}}{\pgfqpoint{0.733709in}{1.142445in}}%
\pgfpathclose%
\pgfusepath{stroke,fill}%
\end{pgfscope}%
\begin{pgfscope}%
\pgfpathrectangle{\pgfqpoint{0.374692in}{0.319877in}}{\pgfqpoint{1.579674in}{2.592703in}} %
\pgfusepath{clip}%
\pgfsetbuttcap%
\pgfsetroundjoin%
\definecolor{currentfill}{rgb}{1.000000,0.752941,0.796078}%
\pgfsetfillcolor{currentfill}%
\pgfsetlinewidth{1.003750pt}%
\definecolor{currentstroke}{rgb}{1.000000,0.752941,0.796078}%
\pgfsetstrokecolor{currentstroke}%
\pgfsetdash{}{0pt}%
\pgfpathmoveto{\pgfqpoint{0.877315in}{1.266842in}}%
\pgfpathcurveto{\pgfqpoint{0.888366in}{1.266842in}}{\pgfqpoint{0.898965in}{1.271233in}}{\pgfqpoint{0.906778in}{1.279046in}}%
\pgfpathcurveto{\pgfqpoint{0.914592in}{1.286860in}}{\pgfqpoint{0.918982in}{1.297459in}}{\pgfqpoint{0.918982in}{1.308509in}}%
\pgfpathcurveto{\pgfqpoint{0.918982in}{1.319559in}}{\pgfqpoint{0.914592in}{1.330158in}}{\pgfqpoint{0.906778in}{1.337972in}}%
\pgfpathcurveto{\pgfqpoint{0.898965in}{1.345785in}}{\pgfqpoint{0.888366in}{1.350176in}}{\pgfqpoint{0.877315in}{1.350176in}}%
\pgfpathcurveto{\pgfqpoint{0.866265in}{1.350176in}}{\pgfqpoint{0.855666in}{1.345785in}}{\pgfqpoint{0.847853in}{1.337972in}}%
\pgfpathcurveto{\pgfqpoint{0.840039in}{1.330158in}}{\pgfqpoint{0.835649in}{1.319559in}}{\pgfqpoint{0.835649in}{1.308509in}}%
\pgfpathcurveto{\pgfqpoint{0.835649in}{1.297459in}}{\pgfqpoint{0.840039in}{1.286860in}}{\pgfqpoint{0.847853in}{1.279046in}}%
\pgfpathcurveto{\pgfqpoint{0.855666in}{1.271233in}}{\pgfqpoint{0.866265in}{1.266842in}}{\pgfqpoint{0.877315in}{1.266842in}}%
\pgfpathclose%
\pgfusepath{stroke,fill}%
\end{pgfscope}%
\begin{pgfscope}%
\pgfpathrectangle{\pgfqpoint{0.374692in}{0.319877in}}{\pgfqpoint{1.579674in}{2.592703in}} %
\pgfusepath{clip}%
\pgfsetbuttcap%
\pgfsetroundjoin%
\definecolor{currentfill}{rgb}{1.000000,0.752941,0.796078}%
\pgfsetfillcolor{currentfill}%
\pgfsetlinewidth{1.003750pt}%
\definecolor{currentstroke}{rgb}{1.000000,0.752941,0.796078}%
\pgfsetstrokecolor{currentstroke}%
\pgfsetdash{}{0pt}%
\pgfpathmoveto{\pgfqpoint{1.020922in}{1.286906in}}%
\pgfpathcurveto{\pgfqpoint{1.031972in}{1.286906in}}{\pgfqpoint{1.042571in}{1.291297in}}{\pgfqpoint{1.050385in}{1.299110in}}%
\pgfpathcurveto{\pgfqpoint{1.058199in}{1.306924in}}{\pgfqpoint{1.062589in}{1.317523in}}{\pgfqpoint{1.062589in}{1.328573in}}%
\pgfpathcurveto{\pgfqpoint{1.062589in}{1.339623in}}{\pgfqpoint{1.058199in}{1.350222in}}{\pgfqpoint{1.050385in}{1.358036in}}%
\pgfpathcurveto{\pgfqpoint{1.042571in}{1.365849in}}{\pgfqpoint{1.031972in}{1.370240in}}{\pgfqpoint{1.020922in}{1.370240in}}%
\pgfpathcurveto{\pgfqpoint{1.009872in}{1.370240in}}{\pgfqpoint{0.999273in}{1.365849in}}{\pgfqpoint{0.991459in}{1.358036in}}%
\pgfpathcurveto{\pgfqpoint{0.983646in}{1.350222in}}{\pgfqpoint{0.979255in}{1.339623in}}{\pgfqpoint{0.979255in}{1.328573in}}%
\pgfpathcurveto{\pgfqpoint{0.979255in}{1.317523in}}{\pgfqpoint{0.983646in}{1.306924in}}{\pgfqpoint{0.991459in}{1.299110in}}%
\pgfpathcurveto{\pgfqpoint{0.999273in}{1.291297in}}{\pgfqpoint{1.009872in}{1.286906in}}{\pgfqpoint{1.020922in}{1.286906in}}%
\pgfpathclose%
\pgfusepath{stroke,fill}%
\end{pgfscope}%
\begin{pgfscope}%
\pgfpathrectangle{\pgfqpoint{0.374692in}{0.319877in}}{\pgfqpoint{1.579674in}{2.592703in}} %
\pgfusepath{clip}%
\pgfsetbuttcap%
\pgfsetroundjoin%
\definecolor{currentfill}{rgb}{1.000000,0.752941,0.796078}%
\pgfsetfillcolor{currentfill}%
\pgfsetlinewidth{1.003750pt}%
\definecolor{currentstroke}{rgb}{1.000000,0.752941,0.796078}%
\pgfsetstrokecolor{currentstroke}%
\pgfsetdash{}{0pt}%
\pgfpathmoveto{\pgfqpoint{1.164529in}{1.351523in}}%
\pgfpathcurveto{\pgfqpoint{1.175579in}{1.351523in}}{\pgfqpoint{1.186178in}{1.355914in}}{\pgfqpoint{1.193992in}{1.363727in}}%
\pgfpathcurveto{\pgfqpoint{1.201805in}{1.371541in}}{\pgfqpoint{1.206196in}{1.382140in}}{\pgfqpoint{1.206196in}{1.393190in}}%
\pgfpathcurveto{\pgfqpoint{1.206196in}{1.404240in}}{\pgfqpoint{1.201805in}{1.414839in}}{\pgfqpoint{1.193992in}{1.422653in}}%
\pgfpathcurveto{\pgfqpoint{1.186178in}{1.430466in}}{\pgfqpoint{1.175579in}{1.434857in}}{\pgfqpoint{1.164529in}{1.434857in}}%
\pgfpathcurveto{\pgfqpoint{1.153479in}{1.434857in}}{\pgfqpoint{1.142880in}{1.430466in}}{\pgfqpoint{1.135066in}{1.422653in}}%
\pgfpathcurveto{\pgfqpoint{1.127252in}{1.414839in}}{\pgfqpoint{1.122862in}{1.404240in}}{\pgfqpoint{1.122862in}{1.393190in}}%
\pgfpathcurveto{\pgfqpoint{1.122862in}{1.382140in}}{\pgfqpoint{1.127252in}{1.371541in}}{\pgfqpoint{1.135066in}{1.363727in}}%
\pgfpathcurveto{\pgfqpoint{1.142880in}{1.355914in}}{\pgfqpoint{1.153479in}{1.351523in}}{\pgfqpoint{1.164529in}{1.351523in}}%
\pgfpathclose%
\pgfusepath{stroke,fill}%
\end{pgfscope}%
\begin{pgfscope}%
\pgfpathrectangle{\pgfqpoint{0.374692in}{0.319877in}}{\pgfqpoint{1.579674in}{2.592703in}} %
\pgfusepath{clip}%
\pgfsetbuttcap%
\pgfsetroundjoin%
\definecolor{currentfill}{rgb}{1.000000,0.752941,0.796078}%
\pgfsetfillcolor{currentfill}%
\pgfsetlinewidth{1.003750pt}%
\definecolor{currentstroke}{rgb}{1.000000,0.752941,0.796078}%
\pgfsetstrokecolor{currentstroke}%
\pgfsetdash{}{0pt}%
\pgfpathmoveto{\pgfqpoint{1.308136in}{1.501233in}}%
\pgfpathcurveto{\pgfqpoint{1.319186in}{1.501233in}}{\pgfqpoint{1.329785in}{1.505623in}}{\pgfqpoint{1.337598in}{1.513437in}}%
\pgfpathcurveto{\pgfqpoint{1.345412in}{1.521251in}}{\pgfqpoint{1.349802in}{1.531850in}}{\pgfqpoint{1.349802in}{1.542900in}}%
\pgfpathcurveto{\pgfqpoint{1.349802in}{1.553950in}}{\pgfqpoint{1.345412in}{1.564549in}}{\pgfqpoint{1.337598in}{1.572363in}}%
\pgfpathcurveto{\pgfqpoint{1.329785in}{1.580176in}}{\pgfqpoint{1.319186in}{1.584566in}}{\pgfqpoint{1.308136in}{1.584566in}}%
\pgfpathcurveto{\pgfqpoint{1.297085in}{1.584566in}}{\pgfqpoint{1.286486in}{1.580176in}}{\pgfqpoint{1.278673in}{1.572363in}}%
\pgfpathcurveto{\pgfqpoint{1.270859in}{1.564549in}}{\pgfqpoint{1.266469in}{1.553950in}}{\pgfqpoint{1.266469in}{1.542900in}}%
\pgfpathcurveto{\pgfqpoint{1.266469in}{1.531850in}}{\pgfqpoint{1.270859in}{1.521251in}}{\pgfqpoint{1.278673in}{1.513437in}}%
\pgfpathcurveto{\pgfqpoint{1.286486in}{1.505623in}}{\pgfqpoint{1.297085in}{1.501233in}}{\pgfqpoint{1.308136in}{1.501233in}}%
\pgfpathclose%
\pgfusepath{stroke,fill}%
\end{pgfscope}%
\begin{pgfscope}%
\pgfpathrectangle{\pgfqpoint{0.374692in}{0.319877in}}{\pgfqpoint{1.579674in}{2.592703in}} %
\pgfusepath{clip}%
\pgfsetbuttcap%
\pgfsetroundjoin%
\definecolor{currentfill}{rgb}{1.000000,0.752941,0.796078}%
\pgfsetfillcolor{currentfill}%
\pgfsetlinewidth{1.003750pt}%
\definecolor{currentstroke}{rgb}{1.000000,0.752941,0.796078}%
\pgfsetstrokecolor{currentstroke}%
\pgfsetdash{}{0pt}%
\pgfpathmoveto{\pgfqpoint{1.451742in}{1.479160in}}%
\pgfpathcurveto{\pgfqpoint{1.462792in}{1.479160in}}{\pgfqpoint{1.473391in}{1.483550in}}{\pgfqpoint{1.481205in}{1.491363in}}%
\pgfpathcurveto{\pgfqpoint{1.489019in}{1.499177in}}{\pgfqpoint{1.493409in}{1.509776in}}{\pgfqpoint{1.493409in}{1.520826in}}%
\pgfpathcurveto{\pgfqpoint{1.493409in}{1.531876in}}{\pgfqpoint{1.489019in}{1.542475in}}{\pgfqpoint{1.481205in}{1.550289in}}%
\pgfpathcurveto{\pgfqpoint{1.473391in}{1.558103in}}{\pgfqpoint{1.462792in}{1.562493in}}{\pgfqpoint{1.451742in}{1.562493in}}%
\pgfpathcurveto{\pgfqpoint{1.440692in}{1.562493in}}{\pgfqpoint{1.430093in}{1.558103in}}{\pgfqpoint{1.422279in}{1.550289in}}%
\pgfpathcurveto{\pgfqpoint{1.414466in}{1.542475in}}{\pgfqpoint{1.410076in}{1.531876in}}{\pgfqpoint{1.410076in}{1.520826in}}%
\pgfpathcurveto{\pgfqpoint{1.410076in}{1.509776in}}{\pgfqpoint{1.414466in}{1.499177in}}{\pgfqpoint{1.422279in}{1.491363in}}%
\pgfpathcurveto{\pgfqpoint{1.430093in}{1.483550in}}{\pgfqpoint{1.440692in}{1.479160in}}{\pgfqpoint{1.451742in}{1.479160in}}%
\pgfpathclose%
\pgfusepath{stroke,fill}%
\end{pgfscope}%
\begin{pgfscope}%
\pgfsetbuttcap%
\pgfsetroundjoin%
\definecolor{currentfill}{rgb}{0.000000,0.000000,0.000000}%
\pgfsetfillcolor{currentfill}%
\pgfsetlinewidth{0.803000pt}%
\definecolor{currentstroke}{rgb}{0.000000,0.000000,0.000000}%
\pgfsetstrokecolor{currentstroke}%
\pgfsetdash{}{0pt}%
\pgfsys@defobject{currentmarker}{\pgfqpoint{0.000000in}{-0.048611in}}{\pgfqpoint{0.000000in}{0.000000in}}{%
\pgfpathmoveto{\pgfqpoint{0.000000in}{0.000000in}}%
\pgfpathlineto{\pgfqpoint{0.000000in}{-0.048611in}}%
\pgfusepath{stroke,fill}%
}%
\begin{pgfscope}%
\pgfsys@transformshift{0.670881in}{0.319877in}%
\pgfsys@useobject{currentmarker}{}%
\end{pgfscope}%
\end{pgfscope}%
\begin{pgfscope}%
\pgftext[x=0.670881in,y=0.222655in,,top]{\rmfamily\fontsize{10.000000}{12.000000}\selectfont \(\displaystyle -0.05\)}%
\end{pgfscope}%
\begin{pgfscope}%
\pgfsetbuttcap%
\pgfsetroundjoin%
\definecolor{currentfill}{rgb}{0.000000,0.000000,0.000000}%
\pgfsetfillcolor{currentfill}%
\pgfsetlinewidth{0.803000pt}%
\definecolor{currentstroke}{rgb}{0.000000,0.000000,0.000000}%
\pgfsetstrokecolor{currentstroke}%
\pgfsetdash{}{0pt}%
\pgfsys@defobject{currentmarker}{\pgfqpoint{0.000000in}{-0.048611in}}{\pgfqpoint{0.000000in}{0.000000in}}{%
\pgfpathmoveto{\pgfqpoint{0.000000in}{0.000000in}}%
\pgfpathlineto{\pgfqpoint{0.000000in}{-0.048611in}}%
\pgfusepath{stroke,fill}%
}%
\begin{pgfscope}%
\pgfsys@transformshift{1.164529in}{0.319877in}%
\pgfsys@useobject{currentmarker}{}%
\end{pgfscope}%
\end{pgfscope}%
\begin{pgfscope}%
\pgftext[x=1.164529in,y=0.222655in,,top]{\rmfamily\fontsize{10.000000}{12.000000}\selectfont \(\displaystyle 0.00\)}%
\end{pgfscope}%
\begin{pgfscope}%
\pgfsetbuttcap%
\pgfsetroundjoin%
\definecolor{currentfill}{rgb}{0.000000,0.000000,0.000000}%
\pgfsetfillcolor{currentfill}%
\pgfsetlinewidth{0.803000pt}%
\definecolor{currentstroke}{rgb}{0.000000,0.000000,0.000000}%
\pgfsetstrokecolor{currentstroke}%
\pgfsetdash{}{0pt}%
\pgfsys@defobject{currentmarker}{\pgfqpoint{0.000000in}{-0.048611in}}{\pgfqpoint{0.000000in}{0.000000in}}{%
\pgfpathmoveto{\pgfqpoint{0.000000in}{0.000000in}}%
\pgfpathlineto{\pgfqpoint{0.000000in}{-0.048611in}}%
\pgfusepath{stroke,fill}%
}%
\begin{pgfscope}%
\pgfsys@transformshift{1.658177in}{0.319877in}%
\pgfsys@useobject{currentmarker}{}%
\end{pgfscope}%
\end{pgfscope}%
\begin{pgfscope}%
\pgftext[x=1.658177in,y=0.222655in,,top]{\rmfamily\fontsize{10.000000}{12.000000}\selectfont \(\displaystyle 0.05\)}%
\end{pgfscope}%
\begin{pgfscope}%
\pgfsetbuttcap%
\pgfsetroundjoin%
\definecolor{currentfill}{rgb}{0.000000,0.000000,0.000000}%
\pgfsetfillcolor{currentfill}%
\pgfsetlinewidth{0.803000pt}%
\definecolor{currentstroke}{rgb}{0.000000,0.000000,0.000000}%
\pgfsetstrokecolor{currentstroke}%
\pgfsetdash{}{0pt}%
\pgfsys@defobject{currentmarker}{\pgfqpoint{-0.048611in}{0.000000in}}{\pgfqpoint{0.000000in}{0.000000in}}{%
\pgfpathmoveto{\pgfqpoint{0.000000in}{0.000000in}}%
\pgfpathlineto{\pgfqpoint{-0.048611in}{0.000000in}}%
\pgfusepath{stroke,fill}%
}%
\begin{pgfscope}%
\pgfsys@transformshift{0.374692in}{0.319877in}%
\pgfsys@useobject{currentmarker}{}%
\end{pgfscope}%
\end{pgfscope}%
\begin{pgfscope}%
\pgftext[x=0.100000in,y=0.272050in,left,base]{\rmfamily\fontsize{10.000000}{12.000000}\selectfont \(\displaystyle 0.0\)}%
\end{pgfscope}%
\begin{pgfscope}%
\pgfsetbuttcap%
\pgfsetroundjoin%
\definecolor{currentfill}{rgb}{0.000000,0.000000,0.000000}%
\pgfsetfillcolor{currentfill}%
\pgfsetlinewidth{0.803000pt}%
\definecolor{currentstroke}{rgb}{0.000000,0.000000,0.000000}%
\pgfsetstrokecolor{currentstroke}%
\pgfsetdash{}{0pt}%
\pgfsys@defobject{currentmarker}{\pgfqpoint{-0.048611in}{0.000000in}}{\pgfqpoint{0.000000in}{0.000000in}}{%
\pgfpathmoveto{\pgfqpoint{0.000000in}{0.000000in}}%
\pgfpathlineto{\pgfqpoint{-0.048611in}{0.000000in}}%
\pgfusepath{stroke,fill}%
}%
\begin{pgfscope}%
\pgfsys@transformshift{0.374692in}{0.838418in}%
\pgfsys@useobject{currentmarker}{}%
\end{pgfscope}%
\end{pgfscope}%
\begin{pgfscope}%
\pgftext[x=0.100000in,y=0.790590in,left,base]{\rmfamily\fontsize{10.000000}{12.000000}\selectfont \(\displaystyle 0.2\)}%
\end{pgfscope}%
\begin{pgfscope}%
\pgfsetbuttcap%
\pgfsetroundjoin%
\definecolor{currentfill}{rgb}{0.000000,0.000000,0.000000}%
\pgfsetfillcolor{currentfill}%
\pgfsetlinewidth{0.803000pt}%
\definecolor{currentstroke}{rgb}{0.000000,0.000000,0.000000}%
\pgfsetstrokecolor{currentstroke}%
\pgfsetdash{}{0pt}%
\pgfsys@defobject{currentmarker}{\pgfqpoint{-0.048611in}{0.000000in}}{\pgfqpoint{0.000000in}{0.000000in}}{%
\pgfpathmoveto{\pgfqpoint{0.000000in}{0.000000in}}%
\pgfpathlineto{\pgfqpoint{-0.048611in}{0.000000in}}%
\pgfusepath{stroke,fill}%
}%
\begin{pgfscope}%
\pgfsys@transformshift{0.374692in}{1.356958in}%
\pgfsys@useobject{currentmarker}{}%
\end{pgfscope}%
\end{pgfscope}%
\begin{pgfscope}%
\pgftext[x=0.100000in,y=1.309131in,left,base]{\rmfamily\fontsize{10.000000}{12.000000}\selectfont \(\displaystyle 0.4\)}%
\end{pgfscope}%
\begin{pgfscope}%
\pgfsetbuttcap%
\pgfsetroundjoin%
\definecolor{currentfill}{rgb}{0.000000,0.000000,0.000000}%
\pgfsetfillcolor{currentfill}%
\pgfsetlinewidth{0.803000pt}%
\definecolor{currentstroke}{rgb}{0.000000,0.000000,0.000000}%
\pgfsetstrokecolor{currentstroke}%
\pgfsetdash{}{0pt}%
\pgfsys@defobject{currentmarker}{\pgfqpoint{-0.048611in}{0.000000in}}{\pgfqpoint{0.000000in}{0.000000in}}{%
\pgfpathmoveto{\pgfqpoint{0.000000in}{0.000000in}}%
\pgfpathlineto{\pgfqpoint{-0.048611in}{0.000000in}}%
\pgfusepath{stroke,fill}%
}%
\begin{pgfscope}%
\pgfsys@transformshift{0.374692in}{1.875499in}%
\pgfsys@useobject{currentmarker}{}%
\end{pgfscope}%
\end{pgfscope}%
\begin{pgfscope}%
\pgftext[x=0.100000in,y=1.827671in,left,base]{\rmfamily\fontsize{10.000000}{12.000000}\selectfont \(\displaystyle 0.6\)}%
\end{pgfscope}%
\begin{pgfscope}%
\pgfsetbuttcap%
\pgfsetroundjoin%
\definecolor{currentfill}{rgb}{0.000000,0.000000,0.000000}%
\pgfsetfillcolor{currentfill}%
\pgfsetlinewidth{0.803000pt}%
\definecolor{currentstroke}{rgb}{0.000000,0.000000,0.000000}%
\pgfsetstrokecolor{currentstroke}%
\pgfsetdash{}{0pt}%
\pgfsys@defobject{currentmarker}{\pgfqpoint{-0.048611in}{0.000000in}}{\pgfqpoint{0.000000in}{0.000000in}}{%
\pgfpathmoveto{\pgfqpoint{0.000000in}{0.000000in}}%
\pgfpathlineto{\pgfqpoint{-0.048611in}{0.000000in}}%
\pgfusepath{stroke,fill}%
}%
\begin{pgfscope}%
\pgfsys@transformshift{0.374692in}{2.394040in}%
\pgfsys@useobject{currentmarker}{}%
\end{pgfscope}%
\end{pgfscope}%
\begin{pgfscope}%
\pgftext[x=0.100000in,y=2.346212in,left,base]{\rmfamily\fontsize{10.000000}{12.000000}\selectfont \(\displaystyle 0.8\)}%
\end{pgfscope}%
\begin{pgfscope}%
\pgfsetbuttcap%
\pgfsetroundjoin%
\definecolor{currentfill}{rgb}{0.000000,0.000000,0.000000}%
\pgfsetfillcolor{currentfill}%
\pgfsetlinewidth{0.803000pt}%
\definecolor{currentstroke}{rgb}{0.000000,0.000000,0.000000}%
\pgfsetstrokecolor{currentstroke}%
\pgfsetdash{}{0pt}%
\pgfsys@defobject{currentmarker}{\pgfqpoint{-0.048611in}{0.000000in}}{\pgfqpoint{0.000000in}{0.000000in}}{%
\pgfpathmoveto{\pgfqpoint{0.000000in}{0.000000in}}%
\pgfpathlineto{\pgfqpoint{-0.048611in}{0.000000in}}%
\pgfusepath{stroke,fill}%
}%
\begin{pgfscope}%
\pgfsys@transformshift{0.374692in}{2.912580in}%
\pgfsys@useobject{currentmarker}{}%
\end{pgfscope}%
\end{pgfscope}%
\begin{pgfscope}%
\pgftext[x=0.100000in,y=2.864752in,left,base]{\rmfamily\fontsize{10.000000}{12.000000}\selectfont \(\displaystyle 1.0\)}%
\end{pgfscope}%
\begin{pgfscope}%
\pgfsetrectcap%
\pgfsetmiterjoin%
\pgfsetlinewidth{0.803000pt}%
\definecolor{currentstroke}{rgb}{0.000000,0.000000,0.000000}%
\pgfsetstrokecolor{currentstroke}%
\pgfsetdash{}{0pt}%
\pgfpathmoveto{\pgfqpoint{0.374692in}{0.319877in}}%
\pgfpathlineto{\pgfqpoint{0.374692in}{2.912580in}}%
\pgfusepath{stroke}%
\end{pgfscope}%
\begin{pgfscope}%
\pgfsetrectcap%
\pgfsetmiterjoin%
\pgfsetlinewidth{0.803000pt}%
\definecolor{currentstroke}{rgb}{0.000000,0.000000,0.000000}%
\pgfsetstrokecolor{currentstroke}%
\pgfsetdash{}{0pt}%
\pgfpathmoveto{\pgfqpoint{1.954366in}{0.319877in}}%
\pgfpathlineto{\pgfqpoint{1.954366in}{2.912580in}}%
\pgfusepath{stroke}%
\end{pgfscope}%
\begin{pgfscope}%
\pgfsetrectcap%
\pgfsetmiterjoin%
\pgfsetlinewidth{0.803000pt}%
\definecolor{currentstroke}{rgb}{0.000000,0.000000,0.000000}%
\pgfsetstrokecolor{currentstroke}%
\pgfsetdash{}{0pt}%
\pgfpathmoveto{\pgfqpoint{0.374692in}{0.319877in}}%
\pgfpathlineto{\pgfqpoint{1.954366in}{0.319877in}}%
\pgfusepath{stroke}%
\end{pgfscope}%
\begin{pgfscope}%
\pgfsetrectcap%
\pgfsetmiterjoin%
\pgfsetlinewidth{0.803000pt}%
\definecolor{currentstroke}{rgb}{0.000000,0.000000,0.000000}%
\pgfsetstrokecolor{currentstroke}%
\pgfsetdash{}{0pt}%
\pgfpathmoveto{\pgfqpoint{0.374692in}{2.912580in}}%
\pgfpathlineto{\pgfqpoint{1.954366in}{2.912580in}}%
\pgfusepath{stroke}%
\end{pgfscope}%
\begin{pgfscope}%
\pgfpathrectangle{\pgfqpoint{2.053095in}{0.319877in}}{\pgfqpoint{0.129635in}{2.592703in}} %
\pgfusepath{clip}%
\pgfsetbuttcap%
\pgfsetmiterjoin%
\definecolor{currentfill}{rgb}{1.000000,1.000000,1.000000}%
\pgfsetfillcolor{currentfill}%
\pgfsetlinewidth{0.010037pt}%
\definecolor{currentstroke}{rgb}{1.000000,1.000000,1.000000}%
\pgfsetstrokecolor{currentstroke}%
\pgfsetdash{}{0pt}%
\pgfpathmoveto{\pgfqpoint{2.053095in}{0.319877in}}%
\pgfpathlineto{\pgfqpoint{2.053095in}{0.330005in}}%
\pgfpathlineto{\pgfqpoint{2.053095in}{2.902452in}}%
\pgfpathlineto{\pgfqpoint{2.053095in}{2.912580in}}%
\pgfpathlineto{\pgfqpoint{2.182730in}{2.912580in}}%
\pgfpathlineto{\pgfqpoint{2.182730in}{2.902452in}}%
\pgfpathlineto{\pgfqpoint{2.182730in}{0.330005in}}%
\pgfpathlineto{\pgfqpoint{2.182730in}{0.319877in}}%
\pgfpathclose%
\pgfusepath{stroke,fill}%
\end{pgfscope}%
\begin{pgfscope}%
\pgfsys@transformshift{2.050000in}{0.320408in}%
\pgftext[left,bottom]{\pgfimage[interpolate=true,width=0.130000in,height=2.590000in]{Perr_vs_dq_Ti_300K-img1.png}}%
\end{pgfscope}%
\begin{pgfscope}%
\pgfsetbuttcap%
\pgfsetroundjoin%
\definecolor{currentfill}{rgb}{0.000000,0.000000,0.000000}%
\pgfsetfillcolor{currentfill}%
\pgfsetlinewidth{0.803000pt}%
\definecolor{currentstroke}{rgb}{0.000000,0.000000,0.000000}%
\pgfsetstrokecolor{currentstroke}%
\pgfsetdash{}{0pt}%
\pgfsys@defobject{currentmarker}{\pgfqpoint{0.000000in}{0.000000in}}{\pgfqpoint{0.048611in}{0.000000in}}{%
\pgfpathmoveto{\pgfqpoint{0.000000in}{0.000000in}}%
\pgfpathlineto{\pgfqpoint{0.048611in}{0.000000in}}%
\pgfusepath{stroke,fill}%
}%
\begin{pgfscope}%
\pgfsys@transformshift{2.182730in}{0.319877in}%
\pgfsys@useobject{currentmarker}{}%
\end{pgfscope}%
\end{pgfscope}%
\begin{pgfscope}%
\pgftext[x=2.279953in,y=0.272050in,left,base]{\rmfamily\fontsize{10.000000}{12.000000}\selectfont \(\displaystyle 0\)}%
\end{pgfscope}%
\begin{pgfscope}%
\pgfsetbuttcap%
\pgfsetroundjoin%
\definecolor{currentfill}{rgb}{0.000000,0.000000,0.000000}%
\pgfsetfillcolor{currentfill}%
\pgfsetlinewidth{0.803000pt}%
\definecolor{currentstroke}{rgb}{0.000000,0.000000,0.000000}%
\pgfsetstrokecolor{currentstroke}%
\pgfsetdash{}{0pt}%
\pgfsys@defobject{currentmarker}{\pgfqpoint{0.000000in}{0.000000in}}{\pgfqpoint{0.048611in}{0.000000in}}{%
\pgfpathmoveto{\pgfqpoint{0.000000in}{0.000000in}}%
\pgfpathlineto{\pgfqpoint{0.048611in}{0.000000in}}%
\pgfusepath{stroke,fill}%
}%
\begin{pgfscope}%
\pgfsys@transformshift{2.182730in}{0.718755in}%
\pgfsys@useobject{currentmarker}{}%
\end{pgfscope}%
\end{pgfscope}%
\begin{pgfscope}%
\pgftext[x=2.279953in,y=0.670927in,left,base]{\rmfamily\fontsize{10.000000}{12.000000}\selectfont \(\displaystyle 2\)}%
\end{pgfscope}%
\begin{pgfscope}%
\pgfsetbuttcap%
\pgfsetroundjoin%
\definecolor{currentfill}{rgb}{0.000000,0.000000,0.000000}%
\pgfsetfillcolor{currentfill}%
\pgfsetlinewidth{0.803000pt}%
\definecolor{currentstroke}{rgb}{0.000000,0.000000,0.000000}%
\pgfsetstrokecolor{currentstroke}%
\pgfsetdash{}{0pt}%
\pgfsys@defobject{currentmarker}{\pgfqpoint{0.000000in}{0.000000in}}{\pgfqpoint{0.048611in}{0.000000in}}{%
\pgfpathmoveto{\pgfqpoint{0.000000in}{0.000000in}}%
\pgfpathlineto{\pgfqpoint{0.048611in}{0.000000in}}%
\pgfusepath{stroke,fill}%
}%
\begin{pgfscope}%
\pgfsys@transformshift{2.182730in}{1.117632in}%
\pgfsys@useobject{currentmarker}{}%
\end{pgfscope}%
\end{pgfscope}%
\begin{pgfscope}%
\pgftext[x=2.279953in,y=1.069804in,left,base]{\rmfamily\fontsize{10.000000}{12.000000}\selectfont \(\displaystyle 4\)}%
\end{pgfscope}%
\begin{pgfscope}%
\pgfsetbuttcap%
\pgfsetroundjoin%
\definecolor{currentfill}{rgb}{0.000000,0.000000,0.000000}%
\pgfsetfillcolor{currentfill}%
\pgfsetlinewidth{0.803000pt}%
\definecolor{currentstroke}{rgb}{0.000000,0.000000,0.000000}%
\pgfsetstrokecolor{currentstroke}%
\pgfsetdash{}{0pt}%
\pgfsys@defobject{currentmarker}{\pgfqpoint{0.000000in}{0.000000in}}{\pgfqpoint{0.048611in}{0.000000in}}{%
\pgfpathmoveto{\pgfqpoint{0.000000in}{0.000000in}}%
\pgfpathlineto{\pgfqpoint{0.048611in}{0.000000in}}%
\pgfusepath{stroke,fill}%
}%
\begin{pgfscope}%
\pgfsys@transformshift{2.182730in}{1.516509in}%
\pgfsys@useobject{currentmarker}{}%
\end{pgfscope}%
\end{pgfscope}%
\begin{pgfscope}%
\pgftext[x=2.279953in,y=1.468682in,left,base]{\rmfamily\fontsize{10.000000}{12.000000}\selectfont \(\displaystyle 6\)}%
\end{pgfscope}%
\begin{pgfscope}%
\pgfsetbuttcap%
\pgfsetroundjoin%
\definecolor{currentfill}{rgb}{0.000000,0.000000,0.000000}%
\pgfsetfillcolor{currentfill}%
\pgfsetlinewidth{0.803000pt}%
\definecolor{currentstroke}{rgb}{0.000000,0.000000,0.000000}%
\pgfsetstrokecolor{currentstroke}%
\pgfsetdash{}{0pt}%
\pgfsys@defobject{currentmarker}{\pgfqpoint{0.000000in}{0.000000in}}{\pgfqpoint{0.048611in}{0.000000in}}{%
\pgfpathmoveto{\pgfqpoint{0.000000in}{0.000000in}}%
\pgfpathlineto{\pgfqpoint{0.048611in}{0.000000in}}%
\pgfusepath{stroke,fill}%
}%
\begin{pgfscope}%
\pgfsys@transformshift{2.182730in}{1.915387in}%
\pgfsys@useobject{currentmarker}{}%
\end{pgfscope}%
\end{pgfscope}%
\begin{pgfscope}%
\pgftext[x=2.279953in,y=1.867559in,left,base]{\rmfamily\fontsize{10.000000}{12.000000}\selectfont \(\displaystyle 8\)}%
\end{pgfscope}%
\begin{pgfscope}%
\pgfsetbuttcap%
\pgfsetroundjoin%
\definecolor{currentfill}{rgb}{0.000000,0.000000,0.000000}%
\pgfsetfillcolor{currentfill}%
\pgfsetlinewidth{0.803000pt}%
\definecolor{currentstroke}{rgb}{0.000000,0.000000,0.000000}%
\pgfsetstrokecolor{currentstroke}%
\pgfsetdash{}{0pt}%
\pgfsys@defobject{currentmarker}{\pgfqpoint{0.000000in}{0.000000in}}{\pgfqpoint{0.048611in}{0.000000in}}{%
\pgfpathmoveto{\pgfqpoint{0.000000in}{0.000000in}}%
\pgfpathlineto{\pgfqpoint{0.048611in}{0.000000in}}%
\pgfusepath{stroke,fill}%
}%
\begin{pgfscope}%
\pgfsys@transformshift{2.182730in}{2.314264in}%
\pgfsys@useobject{currentmarker}{}%
\end{pgfscope}%
\end{pgfscope}%
\begin{pgfscope}%
\pgftext[x=2.279953in,y=2.266436in,left,base]{\rmfamily\fontsize{10.000000}{12.000000}\selectfont \(\displaystyle 10\)}%
\end{pgfscope}%
\begin{pgfscope}%
\pgfsetbuttcap%
\pgfsetroundjoin%
\definecolor{currentfill}{rgb}{0.000000,0.000000,0.000000}%
\pgfsetfillcolor{currentfill}%
\pgfsetlinewidth{0.803000pt}%
\definecolor{currentstroke}{rgb}{0.000000,0.000000,0.000000}%
\pgfsetstrokecolor{currentstroke}%
\pgfsetdash{}{0pt}%
\pgfsys@defobject{currentmarker}{\pgfqpoint{0.000000in}{0.000000in}}{\pgfqpoint{0.048611in}{0.000000in}}{%
\pgfpathmoveto{\pgfqpoint{0.000000in}{0.000000in}}%
\pgfpathlineto{\pgfqpoint{0.048611in}{0.000000in}}%
\pgfusepath{stroke,fill}%
}%
\begin{pgfscope}%
\pgfsys@transformshift{2.182730in}{2.713141in}%
\pgfsys@useobject{currentmarker}{}%
\end{pgfscope}%
\end{pgfscope}%
\begin{pgfscope}%
\pgftext[x=2.279953in,y=2.665314in,left,base]{\rmfamily\fontsize{10.000000}{12.000000}\selectfont \(\displaystyle 12\)}%
\end{pgfscope}%
\begin{pgfscope}%
\pgfsetbuttcap%
\pgfsetmiterjoin%
\pgfsetlinewidth{0.803000pt}%
\definecolor{currentstroke}{rgb}{0.000000,0.000000,0.000000}%
\pgfsetstrokecolor{currentstroke}%
\pgfsetdash{}{0pt}%
\pgfpathmoveto{\pgfqpoint{2.053095in}{0.319877in}}%
\pgfpathlineto{\pgfqpoint{2.053095in}{0.330005in}}%
\pgfpathlineto{\pgfqpoint{2.053095in}{2.902452in}}%
\pgfpathlineto{\pgfqpoint{2.053095in}{2.912580in}}%
\pgfpathlineto{\pgfqpoint{2.182730in}{2.912580in}}%
\pgfpathlineto{\pgfqpoint{2.182730in}{2.902452in}}%
\pgfpathlineto{\pgfqpoint{2.182730in}{0.330005in}}%
\pgfpathlineto{\pgfqpoint{2.182730in}{0.319877in}}%
\pgfpathclose%
\pgfusepath{stroke}%
\end{pgfscope}%
\end{pgfpicture}%
\makeatother%
\endgroup%

    \vspace*{-0.4cm}
	\caption{300 K. Bin size $0.014e$}
	\end{subfigure}
	\hspace{0.6cm}
	\begin{subfigure}[b]{0.45\textwidth}
	\hspace*{-0.4cm}
	%% Creator: Matplotlib, PGF backend
%%
%% To include the figure in your LaTeX document, write
%%   \input{<filename>.pgf}
%%
%% Make sure the required packages are loaded in your preamble
%%   \usepackage{pgf}
%%
%% Figures using additional raster images can only be included by \input if
%% they are in the same directory as the main LaTeX file. For loading figures
%% from other directories you can use the `import` package
%%   \usepackage{import}
%% and then include the figures with
%%   \import{<path to file>}{<filename>.pgf}
%%
%% Matplotlib used the following preamble
%%   \usepackage[utf8x]{inputenc}
%%   \usepackage[T1]{fontenc}
%%
\begingroup%
\makeatletter%
\begin{pgfpicture}%
\pgfpathrectangle{\pgfpointorigin}{\pgfqpoint{2.518842in}{3.060408in}}%
\pgfusepath{use as bounding box, clip}%
\begin{pgfscope}%
\pgfsetbuttcap%
\pgfsetmiterjoin%
\definecolor{currentfill}{rgb}{1.000000,1.000000,1.000000}%
\pgfsetfillcolor{currentfill}%
\pgfsetlinewidth{0.000000pt}%
\definecolor{currentstroke}{rgb}{1.000000,1.000000,1.000000}%
\pgfsetstrokecolor{currentstroke}%
\pgfsetdash{}{0pt}%
\pgfpathmoveto{\pgfqpoint{0.000000in}{0.000000in}}%
\pgfpathlineto{\pgfqpoint{2.518842in}{0.000000in}}%
\pgfpathlineto{\pgfqpoint{2.518842in}{3.060408in}}%
\pgfpathlineto{\pgfqpoint{0.000000in}{3.060408in}}%
\pgfpathclose%
\pgfusepath{fill}%
\end{pgfscope}%
\begin{pgfscope}%
\pgfsetbuttcap%
\pgfsetmiterjoin%
\definecolor{currentfill}{rgb}{1.000000,1.000000,1.000000}%
\pgfsetfillcolor{currentfill}%
\pgfsetlinewidth{0.000000pt}%
\definecolor{currentstroke}{rgb}{0.000000,0.000000,0.000000}%
\pgfsetstrokecolor{currentstroke}%
\pgfsetstrokeopacity{0.000000}%
\pgfsetdash{}{0pt}%
\pgfpathmoveto{\pgfqpoint{0.374692in}{0.319877in}}%
\pgfpathlineto{\pgfqpoint{1.954366in}{0.319877in}}%
\pgfpathlineto{\pgfqpoint{1.954366in}{2.912580in}}%
\pgfpathlineto{\pgfqpoint{0.374692in}{2.912580in}}%
\pgfpathclose%
\pgfusepath{fill}%
\end{pgfscope}%
\begin{pgfscope}%
\pgfpathrectangle{\pgfqpoint{0.374692in}{0.319877in}}{\pgfqpoint{1.579674in}{2.592703in}} %
\pgfusepath{clip}%
\pgfsys@transformshift{0.374692in}{0.319877in}%
\pgftext[left,bottom]{\pgfimage[interpolate=true,width=1.580000in,height=2.600000in]{Perr_vs_dq_Ti_500K-img0.png}}%
\end{pgfscope}%
\begin{pgfscope}%
\pgfpathrectangle{\pgfqpoint{0.374692in}{0.319877in}}{\pgfqpoint{1.579674in}{2.592703in}} %
\pgfusepath{clip}%
\pgfsetbuttcap%
\pgfsetroundjoin%
\definecolor{currentfill}{rgb}{1.000000,0.752941,0.796078}%
\pgfsetfillcolor{currentfill}%
\pgfsetlinewidth{1.003750pt}%
\definecolor{currentstroke}{rgb}{1.000000,0.752941,0.796078}%
\pgfsetstrokecolor{currentstroke}%
\pgfsetdash{}{0pt}%
\pgfpathmoveto{\pgfqpoint{0.670881in}{1.378145in}}%
\pgfpathcurveto{\pgfqpoint{0.681931in}{1.378145in}}{\pgfqpoint{0.692530in}{1.382535in}}{\pgfqpoint{0.700344in}{1.390349in}}%
\pgfpathcurveto{\pgfqpoint{0.708157in}{1.398163in}}{\pgfqpoint{0.712547in}{1.408762in}}{\pgfqpoint{0.712547in}{1.419812in}}%
\pgfpathcurveto{\pgfqpoint{0.712547in}{1.430862in}}{\pgfqpoint{0.708157in}{1.441461in}}{\pgfqpoint{0.700344in}{1.449275in}}%
\pgfpathcurveto{\pgfqpoint{0.692530in}{1.457088in}}{\pgfqpoint{0.681931in}{1.461479in}}{\pgfqpoint{0.670881in}{1.461479in}}%
\pgfpathcurveto{\pgfqpoint{0.659831in}{1.461479in}}{\pgfqpoint{0.649232in}{1.457088in}}{\pgfqpoint{0.641418in}{1.449275in}}%
\pgfpathcurveto{\pgfqpoint{0.633604in}{1.441461in}}{\pgfqpoint{0.629214in}{1.430862in}}{\pgfqpoint{0.629214in}{1.419812in}}%
\pgfpathcurveto{\pgfqpoint{0.629214in}{1.408762in}}{\pgfqpoint{0.633604in}{1.398163in}}{\pgfqpoint{0.641418in}{1.390349in}}%
\pgfpathcurveto{\pgfqpoint{0.649232in}{1.382535in}}{\pgfqpoint{0.659831in}{1.378145in}}{\pgfqpoint{0.670881in}{1.378145in}}%
\pgfpathclose%
\pgfusepath{stroke,fill}%
\end{pgfscope}%
\begin{pgfscope}%
\pgfpathrectangle{\pgfqpoint{0.374692in}{0.319877in}}{\pgfqpoint{1.579674in}{2.592703in}} %
\pgfusepath{clip}%
\pgfsetbuttcap%
\pgfsetroundjoin%
\definecolor{currentfill}{rgb}{1.000000,0.752941,0.796078}%
\pgfsetfillcolor{currentfill}%
\pgfsetlinewidth{1.003750pt}%
\definecolor{currentstroke}{rgb}{1.000000,0.752941,0.796078}%
\pgfsetstrokecolor{currentstroke}%
\pgfsetdash{}{0pt}%
\pgfpathmoveto{\pgfqpoint{0.868340in}{1.417429in}}%
\pgfpathcurveto{\pgfqpoint{0.879390in}{1.417429in}}{\pgfqpoint{0.889989in}{1.421819in}}{\pgfqpoint{0.897803in}{1.429632in}}%
\pgfpathcurveto{\pgfqpoint{0.905616in}{1.437446in}}{\pgfqpoint{0.910007in}{1.448045in}}{\pgfqpoint{0.910007in}{1.459095in}}%
\pgfpathcurveto{\pgfqpoint{0.910007in}{1.470145in}}{\pgfqpoint{0.905616in}{1.480744in}}{\pgfqpoint{0.897803in}{1.488558in}}%
\pgfpathcurveto{\pgfqpoint{0.889989in}{1.496372in}}{\pgfqpoint{0.879390in}{1.500762in}}{\pgfqpoint{0.868340in}{1.500762in}}%
\pgfpathcurveto{\pgfqpoint{0.857290in}{1.500762in}}{\pgfqpoint{0.846691in}{1.496372in}}{\pgfqpoint{0.838877in}{1.488558in}}%
\pgfpathcurveto{\pgfqpoint{0.831064in}{1.480744in}}{\pgfqpoint{0.826673in}{1.470145in}}{\pgfqpoint{0.826673in}{1.459095in}}%
\pgfpathcurveto{\pgfqpoint{0.826673in}{1.448045in}}{\pgfqpoint{0.831064in}{1.437446in}}{\pgfqpoint{0.838877in}{1.429632in}}%
\pgfpathcurveto{\pgfqpoint{0.846691in}{1.421819in}}{\pgfqpoint{0.857290in}{1.417429in}}{\pgfqpoint{0.868340in}{1.417429in}}%
\pgfpathclose%
\pgfusepath{stroke,fill}%
\end{pgfscope}%
\begin{pgfscope}%
\pgfpathrectangle{\pgfqpoint{0.374692in}{0.319877in}}{\pgfqpoint{1.579674in}{2.592703in}} %
\pgfusepath{clip}%
\pgfsetbuttcap%
\pgfsetroundjoin%
\definecolor{currentfill}{rgb}{1.000000,0.752941,0.796078}%
\pgfsetfillcolor{currentfill}%
\pgfsetlinewidth{1.003750pt}%
\definecolor{currentstroke}{rgb}{1.000000,0.752941,0.796078}%
\pgfsetstrokecolor{currentstroke}%
\pgfsetdash{}{0pt}%
\pgfpathmoveto{\pgfqpoint{1.065799in}{1.470085in}}%
\pgfpathcurveto{\pgfqpoint{1.076849in}{1.470085in}}{\pgfqpoint{1.087448in}{1.474475in}}{\pgfqpoint{1.095262in}{1.482289in}}%
\pgfpathcurveto{\pgfqpoint{1.103076in}{1.490103in}}{\pgfqpoint{1.107466in}{1.500702in}}{\pgfqpoint{1.107466in}{1.511752in}}%
\pgfpathcurveto{\pgfqpoint{1.107466in}{1.522802in}}{\pgfqpoint{1.103076in}{1.533401in}}{\pgfqpoint{1.095262in}{1.541214in}}%
\pgfpathcurveto{\pgfqpoint{1.087448in}{1.549028in}}{\pgfqpoint{1.076849in}{1.553418in}}{\pgfqpoint{1.065799in}{1.553418in}}%
\pgfpathcurveto{\pgfqpoint{1.054749in}{1.553418in}}{\pgfqpoint{1.044150in}{1.549028in}}{\pgfqpoint{1.036336in}{1.541214in}}%
\pgfpathcurveto{\pgfqpoint{1.028523in}{1.533401in}}{\pgfqpoint{1.024133in}{1.522802in}}{\pgfqpoint{1.024133in}{1.511752in}}%
\pgfpathcurveto{\pgfqpoint{1.024133in}{1.500702in}}{\pgfqpoint{1.028523in}{1.490103in}}{\pgfqpoint{1.036336in}{1.482289in}}%
\pgfpathcurveto{\pgfqpoint{1.044150in}{1.474475in}}{\pgfqpoint{1.054749in}{1.470085in}}{\pgfqpoint{1.065799in}{1.470085in}}%
\pgfpathclose%
\pgfusepath{stroke,fill}%
\end{pgfscope}%
\begin{pgfscope}%
\pgfpathrectangle{\pgfqpoint{0.374692in}{0.319877in}}{\pgfqpoint{1.579674in}{2.592703in}} %
\pgfusepath{clip}%
\pgfsetbuttcap%
\pgfsetroundjoin%
\definecolor{currentfill}{rgb}{1.000000,0.752941,0.796078}%
\pgfsetfillcolor{currentfill}%
\pgfsetlinewidth{1.003750pt}%
\definecolor{currentstroke}{rgb}{1.000000,0.752941,0.796078}%
\pgfsetstrokecolor{currentstroke}%
\pgfsetdash{}{0pt}%
\pgfpathmoveto{\pgfqpoint{1.263258in}{1.451852in}}%
\pgfpathcurveto{\pgfqpoint{1.274309in}{1.451852in}}{\pgfqpoint{1.284908in}{1.456242in}}{\pgfqpoint{1.292721in}{1.464056in}}%
\pgfpathcurveto{\pgfqpoint{1.300535in}{1.471870in}}{\pgfqpoint{1.304925in}{1.482469in}}{\pgfqpoint{1.304925in}{1.493519in}}%
\pgfpathcurveto{\pgfqpoint{1.304925in}{1.504569in}}{\pgfqpoint{1.300535in}{1.515168in}}{\pgfqpoint{1.292721in}{1.522982in}}%
\pgfpathcurveto{\pgfqpoint{1.284908in}{1.530795in}}{\pgfqpoint{1.274309in}{1.535185in}}{\pgfqpoint{1.263258in}{1.535185in}}%
\pgfpathcurveto{\pgfqpoint{1.252208in}{1.535185in}}{\pgfqpoint{1.241609in}{1.530795in}}{\pgfqpoint{1.233796in}{1.522982in}}%
\pgfpathcurveto{\pgfqpoint{1.225982in}{1.515168in}}{\pgfqpoint{1.221592in}{1.504569in}}{\pgfqpoint{1.221592in}{1.493519in}}%
\pgfpathcurveto{\pgfqpoint{1.221592in}{1.482469in}}{\pgfqpoint{1.225982in}{1.471870in}}{\pgfqpoint{1.233796in}{1.464056in}}%
\pgfpathcurveto{\pgfqpoint{1.241609in}{1.456242in}}{\pgfqpoint{1.252208in}{1.451852in}}{\pgfqpoint{1.263258in}{1.451852in}}%
\pgfpathclose%
\pgfusepath{stroke,fill}%
\end{pgfscope}%
\begin{pgfscope}%
\pgfpathrectangle{\pgfqpoint{0.374692in}{0.319877in}}{\pgfqpoint{1.579674in}{2.592703in}} %
\pgfusepath{clip}%
\pgfsetbuttcap%
\pgfsetroundjoin%
\definecolor{currentfill}{rgb}{1.000000,0.752941,0.796078}%
\pgfsetfillcolor{currentfill}%
\pgfsetlinewidth{1.003750pt}%
\definecolor{currentstroke}{rgb}{1.000000,0.752941,0.796078}%
\pgfsetstrokecolor{currentstroke}%
\pgfsetdash{}{0pt}%
\pgfpathmoveto{\pgfqpoint{1.460718in}{1.443243in}}%
\pgfpathcurveto{\pgfqpoint{1.471768in}{1.443243in}}{\pgfqpoint{1.482367in}{1.447634in}}{\pgfqpoint{1.490180in}{1.455447in}}%
\pgfpathcurveto{\pgfqpoint{1.497994in}{1.463261in}}{\pgfqpoint{1.502384in}{1.473860in}}{\pgfqpoint{1.502384in}{1.484910in}}%
\pgfpathcurveto{\pgfqpoint{1.502384in}{1.495960in}}{\pgfqpoint{1.497994in}{1.506559in}}{\pgfqpoint{1.490180in}{1.514373in}}%
\pgfpathcurveto{\pgfqpoint{1.482367in}{1.522186in}}{\pgfqpoint{1.471768in}{1.526577in}}{\pgfqpoint{1.460718in}{1.526577in}}%
\pgfpathcurveto{\pgfqpoint{1.449668in}{1.526577in}}{\pgfqpoint{1.439069in}{1.522186in}}{\pgfqpoint{1.431255in}{1.514373in}}%
\pgfpathcurveto{\pgfqpoint{1.423441in}{1.506559in}}{\pgfqpoint{1.419051in}{1.495960in}}{\pgfqpoint{1.419051in}{1.484910in}}%
\pgfpathcurveto{\pgfqpoint{1.419051in}{1.473860in}}{\pgfqpoint{1.423441in}{1.463261in}}{\pgfqpoint{1.431255in}{1.455447in}}%
\pgfpathcurveto{\pgfqpoint{1.439069in}{1.447634in}}{\pgfqpoint{1.449668in}{1.443243in}}{\pgfqpoint{1.460718in}{1.443243in}}%
\pgfpathclose%
\pgfusepath{stroke,fill}%
\end{pgfscope}%
\begin{pgfscope}%
\pgfpathrectangle{\pgfqpoint{0.374692in}{0.319877in}}{\pgfqpoint{1.579674in}{2.592703in}} %
\pgfusepath{clip}%
\pgfsetbuttcap%
\pgfsetroundjoin%
\definecolor{currentfill}{rgb}{1.000000,0.752941,0.796078}%
\pgfsetfillcolor{currentfill}%
\pgfsetlinewidth{1.003750pt}%
\definecolor{currentstroke}{rgb}{1.000000,0.752941,0.796078}%
\pgfsetstrokecolor{currentstroke}%
\pgfsetdash{}{0pt}%
\pgfpathmoveto{\pgfqpoint{1.658177in}{1.947754in}}%
\pgfpathcurveto{\pgfqpoint{1.669227in}{1.947754in}}{\pgfqpoint{1.679826in}{1.952144in}}{\pgfqpoint{1.687640in}{1.959958in}}%
\pgfpathcurveto{\pgfqpoint{1.695453in}{1.967772in}}{\pgfqpoint{1.699844in}{1.978371in}}{\pgfqpoint{1.699844in}{1.989421in}}%
\pgfpathcurveto{\pgfqpoint{1.699844in}{2.000471in}}{\pgfqpoint{1.695453in}{2.011070in}}{\pgfqpoint{1.687640in}{2.018884in}}%
\pgfpathcurveto{\pgfqpoint{1.679826in}{2.026697in}}{\pgfqpoint{1.669227in}{2.031087in}}{\pgfqpoint{1.658177in}{2.031087in}}%
\pgfpathcurveto{\pgfqpoint{1.647127in}{2.031087in}}{\pgfqpoint{1.636528in}{2.026697in}}{\pgfqpoint{1.628714in}{2.018884in}}%
\pgfpathcurveto{\pgfqpoint{1.620901in}{2.011070in}}{\pgfqpoint{1.616510in}{2.000471in}}{\pgfqpoint{1.616510in}{1.989421in}}%
\pgfpathcurveto{\pgfqpoint{1.616510in}{1.978371in}}{\pgfqpoint{1.620901in}{1.967772in}}{\pgfqpoint{1.628714in}{1.959958in}}%
\pgfpathcurveto{\pgfqpoint{1.636528in}{1.952144in}}{\pgfqpoint{1.647127in}{1.947754in}}{\pgfqpoint{1.658177in}{1.947754in}}%
\pgfpathclose%
\pgfusepath{stroke,fill}%
\end{pgfscope}%
\begin{pgfscope}%
\pgfpathrectangle{\pgfqpoint{0.374692in}{0.319877in}}{\pgfqpoint{1.579674in}{2.592703in}} %
\pgfusepath{clip}%
\pgfsetbuttcap%
\pgfsetroundjoin%
\definecolor{currentfill}{rgb}{1.000000,0.752941,0.796078}%
\pgfsetfillcolor{currentfill}%
\pgfsetlinewidth{1.003750pt}%
\definecolor{currentstroke}{rgb}{1.000000,0.752941,0.796078}%
\pgfsetstrokecolor{currentstroke}%
\pgfsetdash{}{0pt}%
\pgfpathmoveto{\pgfqpoint{1.855636in}{1.299578in}}%
\pgfpathcurveto{\pgfqpoint{1.866686in}{1.299578in}}{\pgfqpoint{1.877285in}{1.303969in}}{\pgfqpoint{1.885099in}{1.311782in}}%
\pgfpathcurveto{\pgfqpoint{1.892913in}{1.319596in}}{\pgfqpoint{1.897303in}{1.330195in}}{\pgfqpoint{1.897303in}{1.341245in}}%
\pgfpathcurveto{\pgfqpoint{1.897303in}{1.352295in}}{\pgfqpoint{1.892913in}{1.362894in}}{\pgfqpoint{1.885099in}{1.370708in}}%
\pgfpathcurveto{\pgfqpoint{1.877285in}{1.378522in}}{\pgfqpoint{1.866686in}{1.382912in}}{\pgfqpoint{1.855636in}{1.382912in}}%
\pgfpathcurveto{\pgfqpoint{1.844586in}{1.382912in}}{\pgfqpoint{1.833987in}{1.378522in}}{\pgfqpoint{1.826173in}{1.370708in}}%
\pgfpathcurveto{\pgfqpoint{1.818360in}{1.362894in}}{\pgfqpoint{1.813969in}{1.352295in}}{\pgfqpoint{1.813969in}{1.341245in}}%
\pgfpathcurveto{\pgfqpoint{1.813969in}{1.330195in}}{\pgfqpoint{1.818360in}{1.319596in}}{\pgfqpoint{1.826173in}{1.311782in}}%
\pgfpathcurveto{\pgfqpoint{1.833987in}{1.303969in}}{\pgfqpoint{1.844586in}{1.299578in}}{\pgfqpoint{1.855636in}{1.299578in}}%
\pgfpathclose%
\pgfusepath{stroke,fill}%
\end{pgfscope}%
\begin{pgfscope}%
\pgfsetbuttcap%
\pgfsetroundjoin%
\definecolor{currentfill}{rgb}{0.000000,0.000000,0.000000}%
\pgfsetfillcolor{currentfill}%
\pgfsetlinewidth{0.803000pt}%
\definecolor{currentstroke}{rgb}{0.000000,0.000000,0.000000}%
\pgfsetstrokecolor{currentstroke}%
\pgfsetdash{}{0pt}%
\pgfsys@defobject{currentmarker}{\pgfqpoint{0.000000in}{-0.048611in}}{\pgfqpoint{0.000000in}{0.000000in}}{%
\pgfpathmoveto{\pgfqpoint{0.000000in}{0.000000in}}%
\pgfpathlineto{\pgfqpoint{0.000000in}{-0.048611in}}%
\pgfusepath{stroke,fill}%
}%
\begin{pgfscope}%
\pgfsys@transformshift{0.670881in}{0.319877in}%
\pgfsys@useobject{currentmarker}{}%
\end{pgfscope}%
\end{pgfscope}%
\begin{pgfscope}%
\pgftext[x=0.670881in,y=0.222655in,,top]{\rmfamily\fontsize{10.000000}{12.000000}\selectfont \(\displaystyle -0.05\)}%
\end{pgfscope}%
\begin{pgfscope}%
\pgfsetbuttcap%
\pgfsetroundjoin%
\definecolor{currentfill}{rgb}{0.000000,0.000000,0.000000}%
\pgfsetfillcolor{currentfill}%
\pgfsetlinewidth{0.803000pt}%
\definecolor{currentstroke}{rgb}{0.000000,0.000000,0.000000}%
\pgfsetstrokecolor{currentstroke}%
\pgfsetdash{}{0pt}%
\pgfsys@defobject{currentmarker}{\pgfqpoint{0.000000in}{-0.048611in}}{\pgfqpoint{0.000000in}{0.000000in}}{%
\pgfpathmoveto{\pgfqpoint{0.000000in}{0.000000in}}%
\pgfpathlineto{\pgfqpoint{0.000000in}{-0.048611in}}%
\pgfusepath{stroke,fill}%
}%
\begin{pgfscope}%
\pgfsys@transformshift{1.164529in}{0.319877in}%
\pgfsys@useobject{currentmarker}{}%
\end{pgfscope}%
\end{pgfscope}%
\begin{pgfscope}%
\pgftext[x=1.164529in,y=0.222655in,,top]{\rmfamily\fontsize{10.000000}{12.000000}\selectfont \(\displaystyle 0.00\)}%
\end{pgfscope}%
\begin{pgfscope}%
\pgfsetbuttcap%
\pgfsetroundjoin%
\definecolor{currentfill}{rgb}{0.000000,0.000000,0.000000}%
\pgfsetfillcolor{currentfill}%
\pgfsetlinewidth{0.803000pt}%
\definecolor{currentstroke}{rgb}{0.000000,0.000000,0.000000}%
\pgfsetstrokecolor{currentstroke}%
\pgfsetdash{}{0pt}%
\pgfsys@defobject{currentmarker}{\pgfqpoint{0.000000in}{-0.048611in}}{\pgfqpoint{0.000000in}{0.000000in}}{%
\pgfpathmoveto{\pgfqpoint{0.000000in}{0.000000in}}%
\pgfpathlineto{\pgfqpoint{0.000000in}{-0.048611in}}%
\pgfusepath{stroke,fill}%
}%
\begin{pgfscope}%
\pgfsys@transformshift{1.658177in}{0.319877in}%
\pgfsys@useobject{currentmarker}{}%
\end{pgfscope}%
\end{pgfscope}%
\begin{pgfscope}%
\pgftext[x=1.658177in,y=0.222655in,,top]{\rmfamily\fontsize{10.000000}{12.000000}\selectfont \(\displaystyle 0.05\)}%
\end{pgfscope}%
\begin{pgfscope}%
\pgfsetbuttcap%
\pgfsetroundjoin%
\definecolor{currentfill}{rgb}{0.000000,0.000000,0.000000}%
\pgfsetfillcolor{currentfill}%
\pgfsetlinewidth{0.803000pt}%
\definecolor{currentstroke}{rgb}{0.000000,0.000000,0.000000}%
\pgfsetstrokecolor{currentstroke}%
\pgfsetdash{}{0pt}%
\pgfsys@defobject{currentmarker}{\pgfqpoint{-0.048611in}{0.000000in}}{\pgfqpoint{0.000000in}{0.000000in}}{%
\pgfpathmoveto{\pgfqpoint{0.000000in}{0.000000in}}%
\pgfpathlineto{\pgfqpoint{-0.048611in}{0.000000in}}%
\pgfusepath{stroke,fill}%
}%
\begin{pgfscope}%
\pgfsys@transformshift{0.374692in}{0.319877in}%
\pgfsys@useobject{currentmarker}{}%
\end{pgfscope}%
\end{pgfscope}%
\begin{pgfscope}%
\pgftext[x=0.100000in,y=0.272050in,left,base]{\rmfamily\fontsize{10.000000}{12.000000}\selectfont \(\displaystyle 0.0\)}%
\end{pgfscope}%
\begin{pgfscope}%
\pgfsetbuttcap%
\pgfsetroundjoin%
\definecolor{currentfill}{rgb}{0.000000,0.000000,0.000000}%
\pgfsetfillcolor{currentfill}%
\pgfsetlinewidth{0.803000pt}%
\definecolor{currentstroke}{rgb}{0.000000,0.000000,0.000000}%
\pgfsetstrokecolor{currentstroke}%
\pgfsetdash{}{0pt}%
\pgfsys@defobject{currentmarker}{\pgfqpoint{-0.048611in}{0.000000in}}{\pgfqpoint{0.000000in}{0.000000in}}{%
\pgfpathmoveto{\pgfqpoint{0.000000in}{0.000000in}}%
\pgfpathlineto{\pgfqpoint{-0.048611in}{0.000000in}}%
\pgfusepath{stroke,fill}%
}%
\begin{pgfscope}%
\pgfsys@transformshift{0.374692in}{0.838418in}%
\pgfsys@useobject{currentmarker}{}%
\end{pgfscope}%
\end{pgfscope}%
\begin{pgfscope}%
\pgftext[x=0.100000in,y=0.790590in,left,base]{\rmfamily\fontsize{10.000000}{12.000000}\selectfont \(\displaystyle 0.2\)}%
\end{pgfscope}%
\begin{pgfscope}%
\pgfsetbuttcap%
\pgfsetroundjoin%
\definecolor{currentfill}{rgb}{0.000000,0.000000,0.000000}%
\pgfsetfillcolor{currentfill}%
\pgfsetlinewidth{0.803000pt}%
\definecolor{currentstroke}{rgb}{0.000000,0.000000,0.000000}%
\pgfsetstrokecolor{currentstroke}%
\pgfsetdash{}{0pt}%
\pgfsys@defobject{currentmarker}{\pgfqpoint{-0.048611in}{0.000000in}}{\pgfqpoint{0.000000in}{0.000000in}}{%
\pgfpathmoveto{\pgfqpoint{0.000000in}{0.000000in}}%
\pgfpathlineto{\pgfqpoint{-0.048611in}{0.000000in}}%
\pgfusepath{stroke,fill}%
}%
\begin{pgfscope}%
\pgfsys@transformshift{0.374692in}{1.356958in}%
\pgfsys@useobject{currentmarker}{}%
\end{pgfscope}%
\end{pgfscope}%
\begin{pgfscope}%
\pgftext[x=0.100000in,y=1.309131in,left,base]{\rmfamily\fontsize{10.000000}{12.000000}\selectfont \(\displaystyle 0.4\)}%
\end{pgfscope}%
\begin{pgfscope}%
\pgfsetbuttcap%
\pgfsetroundjoin%
\definecolor{currentfill}{rgb}{0.000000,0.000000,0.000000}%
\pgfsetfillcolor{currentfill}%
\pgfsetlinewidth{0.803000pt}%
\definecolor{currentstroke}{rgb}{0.000000,0.000000,0.000000}%
\pgfsetstrokecolor{currentstroke}%
\pgfsetdash{}{0pt}%
\pgfsys@defobject{currentmarker}{\pgfqpoint{-0.048611in}{0.000000in}}{\pgfqpoint{0.000000in}{0.000000in}}{%
\pgfpathmoveto{\pgfqpoint{0.000000in}{0.000000in}}%
\pgfpathlineto{\pgfqpoint{-0.048611in}{0.000000in}}%
\pgfusepath{stroke,fill}%
}%
\begin{pgfscope}%
\pgfsys@transformshift{0.374692in}{1.875499in}%
\pgfsys@useobject{currentmarker}{}%
\end{pgfscope}%
\end{pgfscope}%
\begin{pgfscope}%
\pgftext[x=0.100000in,y=1.827671in,left,base]{\rmfamily\fontsize{10.000000}{12.000000}\selectfont \(\displaystyle 0.6\)}%
\end{pgfscope}%
\begin{pgfscope}%
\pgfsetbuttcap%
\pgfsetroundjoin%
\definecolor{currentfill}{rgb}{0.000000,0.000000,0.000000}%
\pgfsetfillcolor{currentfill}%
\pgfsetlinewidth{0.803000pt}%
\definecolor{currentstroke}{rgb}{0.000000,0.000000,0.000000}%
\pgfsetstrokecolor{currentstroke}%
\pgfsetdash{}{0pt}%
\pgfsys@defobject{currentmarker}{\pgfqpoint{-0.048611in}{0.000000in}}{\pgfqpoint{0.000000in}{0.000000in}}{%
\pgfpathmoveto{\pgfqpoint{0.000000in}{0.000000in}}%
\pgfpathlineto{\pgfqpoint{-0.048611in}{0.000000in}}%
\pgfusepath{stroke,fill}%
}%
\begin{pgfscope}%
\pgfsys@transformshift{0.374692in}{2.394040in}%
\pgfsys@useobject{currentmarker}{}%
\end{pgfscope}%
\end{pgfscope}%
\begin{pgfscope}%
\pgftext[x=0.100000in,y=2.346212in,left,base]{\rmfamily\fontsize{10.000000}{12.000000}\selectfont \(\displaystyle 0.8\)}%
\end{pgfscope}%
\begin{pgfscope}%
\pgfsetbuttcap%
\pgfsetroundjoin%
\definecolor{currentfill}{rgb}{0.000000,0.000000,0.000000}%
\pgfsetfillcolor{currentfill}%
\pgfsetlinewidth{0.803000pt}%
\definecolor{currentstroke}{rgb}{0.000000,0.000000,0.000000}%
\pgfsetstrokecolor{currentstroke}%
\pgfsetdash{}{0pt}%
\pgfsys@defobject{currentmarker}{\pgfqpoint{-0.048611in}{0.000000in}}{\pgfqpoint{0.000000in}{0.000000in}}{%
\pgfpathmoveto{\pgfqpoint{0.000000in}{0.000000in}}%
\pgfpathlineto{\pgfqpoint{-0.048611in}{0.000000in}}%
\pgfusepath{stroke,fill}%
}%
\begin{pgfscope}%
\pgfsys@transformshift{0.374692in}{2.912580in}%
\pgfsys@useobject{currentmarker}{}%
\end{pgfscope}%
\end{pgfscope}%
\begin{pgfscope}%
\pgftext[x=0.100000in,y=2.864752in,left,base]{\rmfamily\fontsize{10.000000}{12.000000}\selectfont \(\displaystyle 1.0\)}%
\end{pgfscope}%
\begin{pgfscope}%
\pgfsetrectcap%
\pgfsetmiterjoin%
\pgfsetlinewidth{0.803000pt}%
\definecolor{currentstroke}{rgb}{0.000000,0.000000,0.000000}%
\pgfsetstrokecolor{currentstroke}%
\pgfsetdash{}{0pt}%
\pgfpathmoveto{\pgfqpoint{0.374692in}{0.319877in}}%
\pgfpathlineto{\pgfqpoint{0.374692in}{2.912580in}}%
\pgfusepath{stroke}%
\end{pgfscope}%
\begin{pgfscope}%
\pgfsetrectcap%
\pgfsetmiterjoin%
\pgfsetlinewidth{0.803000pt}%
\definecolor{currentstroke}{rgb}{0.000000,0.000000,0.000000}%
\pgfsetstrokecolor{currentstroke}%
\pgfsetdash{}{0pt}%
\pgfpathmoveto{\pgfqpoint{1.954366in}{0.319877in}}%
\pgfpathlineto{\pgfqpoint{1.954366in}{2.912580in}}%
\pgfusepath{stroke}%
\end{pgfscope}%
\begin{pgfscope}%
\pgfsetrectcap%
\pgfsetmiterjoin%
\pgfsetlinewidth{0.803000pt}%
\definecolor{currentstroke}{rgb}{0.000000,0.000000,0.000000}%
\pgfsetstrokecolor{currentstroke}%
\pgfsetdash{}{0pt}%
\pgfpathmoveto{\pgfqpoint{0.374692in}{0.319877in}}%
\pgfpathlineto{\pgfqpoint{1.954366in}{0.319877in}}%
\pgfusepath{stroke}%
\end{pgfscope}%
\begin{pgfscope}%
\pgfsetrectcap%
\pgfsetmiterjoin%
\pgfsetlinewidth{0.803000pt}%
\definecolor{currentstroke}{rgb}{0.000000,0.000000,0.000000}%
\pgfsetstrokecolor{currentstroke}%
\pgfsetdash{}{0pt}%
\pgfpathmoveto{\pgfqpoint{0.374692in}{2.912580in}}%
\pgfpathlineto{\pgfqpoint{1.954366in}{2.912580in}}%
\pgfusepath{stroke}%
\end{pgfscope}%
\begin{pgfscope}%
\pgfpathrectangle{\pgfqpoint{2.053095in}{0.319877in}}{\pgfqpoint{0.129635in}{2.592703in}} %
\pgfusepath{clip}%
\pgfsetbuttcap%
\pgfsetmiterjoin%
\definecolor{currentfill}{rgb}{1.000000,1.000000,1.000000}%
\pgfsetfillcolor{currentfill}%
\pgfsetlinewidth{0.010037pt}%
\definecolor{currentstroke}{rgb}{1.000000,1.000000,1.000000}%
\pgfsetstrokecolor{currentstroke}%
\pgfsetdash{}{0pt}%
\pgfpathmoveto{\pgfqpoint{2.053095in}{0.319877in}}%
\pgfpathlineto{\pgfqpoint{2.053095in}{0.330005in}}%
\pgfpathlineto{\pgfqpoint{2.053095in}{2.902452in}}%
\pgfpathlineto{\pgfqpoint{2.053095in}{2.912580in}}%
\pgfpathlineto{\pgfqpoint{2.182730in}{2.912580in}}%
\pgfpathlineto{\pgfqpoint{2.182730in}{2.902452in}}%
\pgfpathlineto{\pgfqpoint{2.182730in}{0.330005in}}%
\pgfpathlineto{\pgfqpoint{2.182730in}{0.319877in}}%
\pgfpathclose%
\pgfusepath{stroke,fill}%
\end{pgfscope}%
\begin{pgfscope}%
\pgfsys@transformshift{2.050000in}{0.320408in}%
\pgftext[left,bottom]{\pgfimage[interpolate=true,width=0.130000in,height=2.590000in]{Perr_vs_dq_Ti_500K-img1.png}}%
\end{pgfscope}%
\begin{pgfscope}%
\pgfsetbuttcap%
\pgfsetroundjoin%
\definecolor{currentfill}{rgb}{0.000000,0.000000,0.000000}%
\pgfsetfillcolor{currentfill}%
\pgfsetlinewidth{0.803000pt}%
\definecolor{currentstroke}{rgb}{0.000000,0.000000,0.000000}%
\pgfsetstrokecolor{currentstroke}%
\pgfsetdash{}{0pt}%
\pgfsys@defobject{currentmarker}{\pgfqpoint{0.000000in}{0.000000in}}{\pgfqpoint{0.048611in}{0.000000in}}{%
\pgfpathmoveto{\pgfqpoint{0.000000in}{0.000000in}}%
\pgfpathlineto{\pgfqpoint{0.048611in}{0.000000in}}%
\pgfusepath{stroke,fill}%
}%
\begin{pgfscope}%
\pgfsys@transformshift{2.182730in}{0.319877in}%
\pgfsys@useobject{currentmarker}{}%
\end{pgfscope}%
\end{pgfscope}%
\begin{pgfscope}%
\pgftext[x=2.279953in,y=0.272050in,left,base]{\rmfamily\fontsize{10.000000}{12.000000}\selectfont \(\displaystyle 0\)}%
\end{pgfscope}%
\begin{pgfscope}%
\pgfsetbuttcap%
\pgfsetroundjoin%
\definecolor{currentfill}{rgb}{0.000000,0.000000,0.000000}%
\pgfsetfillcolor{currentfill}%
\pgfsetlinewidth{0.803000pt}%
\definecolor{currentstroke}{rgb}{0.000000,0.000000,0.000000}%
\pgfsetstrokecolor{currentstroke}%
\pgfsetdash{}{0pt}%
\pgfsys@defobject{currentmarker}{\pgfqpoint{0.000000in}{0.000000in}}{\pgfqpoint{0.048611in}{0.000000in}}{%
\pgfpathmoveto{\pgfqpoint{0.000000in}{0.000000in}}%
\pgfpathlineto{\pgfqpoint{0.048611in}{0.000000in}}%
\pgfusepath{stroke,fill}%
}%
\begin{pgfscope}%
\pgfsys@transformshift{2.182730in}{0.718755in}%
\pgfsys@useobject{currentmarker}{}%
\end{pgfscope}%
\end{pgfscope}%
\begin{pgfscope}%
\pgftext[x=2.279953in,y=0.670927in,left,base]{\rmfamily\fontsize{10.000000}{12.000000}\selectfont \(\displaystyle 2\)}%
\end{pgfscope}%
\begin{pgfscope}%
\pgfsetbuttcap%
\pgfsetroundjoin%
\definecolor{currentfill}{rgb}{0.000000,0.000000,0.000000}%
\pgfsetfillcolor{currentfill}%
\pgfsetlinewidth{0.803000pt}%
\definecolor{currentstroke}{rgb}{0.000000,0.000000,0.000000}%
\pgfsetstrokecolor{currentstroke}%
\pgfsetdash{}{0pt}%
\pgfsys@defobject{currentmarker}{\pgfqpoint{0.000000in}{0.000000in}}{\pgfqpoint{0.048611in}{0.000000in}}{%
\pgfpathmoveto{\pgfqpoint{0.000000in}{0.000000in}}%
\pgfpathlineto{\pgfqpoint{0.048611in}{0.000000in}}%
\pgfusepath{stroke,fill}%
}%
\begin{pgfscope}%
\pgfsys@transformshift{2.182730in}{1.117632in}%
\pgfsys@useobject{currentmarker}{}%
\end{pgfscope}%
\end{pgfscope}%
\begin{pgfscope}%
\pgftext[x=2.279953in,y=1.069804in,left,base]{\rmfamily\fontsize{10.000000}{12.000000}\selectfont \(\displaystyle 4\)}%
\end{pgfscope}%
\begin{pgfscope}%
\pgfsetbuttcap%
\pgfsetroundjoin%
\definecolor{currentfill}{rgb}{0.000000,0.000000,0.000000}%
\pgfsetfillcolor{currentfill}%
\pgfsetlinewidth{0.803000pt}%
\definecolor{currentstroke}{rgb}{0.000000,0.000000,0.000000}%
\pgfsetstrokecolor{currentstroke}%
\pgfsetdash{}{0pt}%
\pgfsys@defobject{currentmarker}{\pgfqpoint{0.000000in}{0.000000in}}{\pgfqpoint{0.048611in}{0.000000in}}{%
\pgfpathmoveto{\pgfqpoint{0.000000in}{0.000000in}}%
\pgfpathlineto{\pgfqpoint{0.048611in}{0.000000in}}%
\pgfusepath{stroke,fill}%
}%
\begin{pgfscope}%
\pgfsys@transformshift{2.182730in}{1.516509in}%
\pgfsys@useobject{currentmarker}{}%
\end{pgfscope}%
\end{pgfscope}%
\begin{pgfscope}%
\pgftext[x=2.279953in,y=1.468682in,left,base]{\rmfamily\fontsize{10.000000}{12.000000}\selectfont \(\displaystyle 6\)}%
\end{pgfscope}%
\begin{pgfscope}%
\pgfsetbuttcap%
\pgfsetroundjoin%
\definecolor{currentfill}{rgb}{0.000000,0.000000,0.000000}%
\pgfsetfillcolor{currentfill}%
\pgfsetlinewidth{0.803000pt}%
\definecolor{currentstroke}{rgb}{0.000000,0.000000,0.000000}%
\pgfsetstrokecolor{currentstroke}%
\pgfsetdash{}{0pt}%
\pgfsys@defobject{currentmarker}{\pgfqpoint{0.000000in}{0.000000in}}{\pgfqpoint{0.048611in}{0.000000in}}{%
\pgfpathmoveto{\pgfqpoint{0.000000in}{0.000000in}}%
\pgfpathlineto{\pgfqpoint{0.048611in}{0.000000in}}%
\pgfusepath{stroke,fill}%
}%
\begin{pgfscope}%
\pgfsys@transformshift{2.182730in}{1.915387in}%
\pgfsys@useobject{currentmarker}{}%
\end{pgfscope}%
\end{pgfscope}%
\begin{pgfscope}%
\pgftext[x=2.279953in,y=1.867559in,left,base]{\rmfamily\fontsize{10.000000}{12.000000}\selectfont \(\displaystyle 8\)}%
\end{pgfscope}%
\begin{pgfscope}%
\pgfsetbuttcap%
\pgfsetroundjoin%
\definecolor{currentfill}{rgb}{0.000000,0.000000,0.000000}%
\pgfsetfillcolor{currentfill}%
\pgfsetlinewidth{0.803000pt}%
\definecolor{currentstroke}{rgb}{0.000000,0.000000,0.000000}%
\pgfsetstrokecolor{currentstroke}%
\pgfsetdash{}{0pt}%
\pgfsys@defobject{currentmarker}{\pgfqpoint{0.000000in}{0.000000in}}{\pgfqpoint{0.048611in}{0.000000in}}{%
\pgfpathmoveto{\pgfqpoint{0.000000in}{0.000000in}}%
\pgfpathlineto{\pgfqpoint{0.048611in}{0.000000in}}%
\pgfusepath{stroke,fill}%
}%
\begin{pgfscope}%
\pgfsys@transformshift{2.182730in}{2.314264in}%
\pgfsys@useobject{currentmarker}{}%
\end{pgfscope}%
\end{pgfscope}%
\begin{pgfscope}%
\pgftext[x=2.279953in,y=2.266436in,left,base]{\rmfamily\fontsize{10.000000}{12.000000}\selectfont \(\displaystyle 10\)}%
\end{pgfscope}%
\begin{pgfscope}%
\pgfsetbuttcap%
\pgfsetroundjoin%
\definecolor{currentfill}{rgb}{0.000000,0.000000,0.000000}%
\pgfsetfillcolor{currentfill}%
\pgfsetlinewidth{0.803000pt}%
\definecolor{currentstroke}{rgb}{0.000000,0.000000,0.000000}%
\pgfsetstrokecolor{currentstroke}%
\pgfsetdash{}{0pt}%
\pgfsys@defobject{currentmarker}{\pgfqpoint{0.000000in}{0.000000in}}{\pgfqpoint{0.048611in}{0.000000in}}{%
\pgfpathmoveto{\pgfqpoint{0.000000in}{0.000000in}}%
\pgfpathlineto{\pgfqpoint{0.048611in}{0.000000in}}%
\pgfusepath{stroke,fill}%
}%
\begin{pgfscope}%
\pgfsys@transformshift{2.182730in}{2.713141in}%
\pgfsys@useobject{currentmarker}{}%
\end{pgfscope}%
\end{pgfscope}%
\begin{pgfscope}%
\pgftext[x=2.279953in,y=2.665314in,left,base]{\rmfamily\fontsize{10.000000}{12.000000}\selectfont \(\displaystyle 12\)}%
\end{pgfscope}%
\begin{pgfscope}%
\pgfsetbuttcap%
\pgfsetmiterjoin%
\pgfsetlinewidth{0.803000pt}%
\definecolor{currentstroke}{rgb}{0.000000,0.000000,0.000000}%
\pgfsetstrokecolor{currentstroke}%
\pgfsetdash{}{0pt}%
\pgfpathmoveto{\pgfqpoint{2.053095in}{0.319877in}}%
\pgfpathlineto{\pgfqpoint{2.053095in}{0.330005in}}%
\pgfpathlineto{\pgfqpoint{2.053095in}{2.902452in}}%
\pgfpathlineto{\pgfqpoint{2.053095in}{2.912580in}}%
\pgfpathlineto{\pgfqpoint{2.182730in}{2.912580in}}%
\pgfpathlineto{\pgfqpoint{2.182730in}{2.902452in}}%
\pgfpathlineto{\pgfqpoint{2.182730in}{0.330005in}}%
\pgfpathlineto{\pgfqpoint{2.182730in}{0.319877in}}%
\pgfpathclose%
\pgfusepath{stroke}%
\end{pgfscope}%
\end{pgfpicture}%
\makeatother%
\endgroup%

    \vspace*{-0.4cm}
	\caption{500 K. Bin size $0.018e$}
	\end{subfigure}
	\quad
	\begin{subfigure}[b]{0.45\textwidth}
	\hspace*{-0.4cm}
	%% Creator: Matplotlib, PGF backend
%%
%% To include the figure in your LaTeX document, write
%%   \input{<filename>.pgf}
%%
%% Make sure the required packages are loaded in your preamble
%%   \usepackage{pgf}
%%
%% Figures using additional raster images can only be included by \input if
%% they are in the same directory as the main LaTeX file. For loading figures
%% from other directories you can use the `import` package
%%   \usepackage{import}
%% and then include the figures with
%%   \import{<path to file>}{<filename>.pgf}
%%
%% Matplotlib used the following preamble
%%   \usepackage[utf8x]{inputenc}
%%   \usepackage[T1]{fontenc}
%%
\begingroup%
\makeatletter%
\begin{pgfpicture}%
\pgfpathrectangle{\pgfpointorigin}{\pgfqpoint{2.518842in}{3.060408in}}%
\pgfusepath{use as bounding box, clip}%
\begin{pgfscope}%
\pgfsetbuttcap%
\pgfsetmiterjoin%
\definecolor{currentfill}{rgb}{1.000000,1.000000,1.000000}%
\pgfsetfillcolor{currentfill}%
\pgfsetlinewidth{0.000000pt}%
\definecolor{currentstroke}{rgb}{1.000000,1.000000,1.000000}%
\pgfsetstrokecolor{currentstroke}%
\pgfsetdash{}{0pt}%
\pgfpathmoveto{\pgfqpoint{0.000000in}{0.000000in}}%
\pgfpathlineto{\pgfqpoint{2.518842in}{0.000000in}}%
\pgfpathlineto{\pgfqpoint{2.518842in}{3.060408in}}%
\pgfpathlineto{\pgfqpoint{0.000000in}{3.060408in}}%
\pgfpathclose%
\pgfusepath{fill}%
\end{pgfscope}%
\begin{pgfscope}%
\pgfsetbuttcap%
\pgfsetmiterjoin%
\definecolor{currentfill}{rgb}{1.000000,1.000000,1.000000}%
\pgfsetfillcolor{currentfill}%
\pgfsetlinewidth{0.000000pt}%
\definecolor{currentstroke}{rgb}{0.000000,0.000000,0.000000}%
\pgfsetstrokecolor{currentstroke}%
\pgfsetstrokeopacity{0.000000}%
\pgfsetdash{}{0pt}%
\pgfpathmoveto{\pgfqpoint{0.374692in}{0.319877in}}%
\pgfpathlineto{\pgfqpoint{1.954366in}{0.319877in}}%
\pgfpathlineto{\pgfqpoint{1.954366in}{2.912580in}}%
\pgfpathlineto{\pgfqpoint{0.374692in}{2.912580in}}%
\pgfpathclose%
\pgfusepath{fill}%
\end{pgfscope}%
\begin{pgfscope}%
\pgfpathrectangle{\pgfqpoint{0.374692in}{0.319877in}}{\pgfqpoint{1.579674in}{2.592703in}} %
\pgfusepath{clip}%
\pgfsys@transformshift{0.374692in}{0.319877in}%
\pgftext[left,bottom]{\pgfimage[interpolate=true,width=1.580000in,height=2.600000in]{Perr_vs_dq_Ti_1000K-img0.png}}%
\end{pgfscope}%
\begin{pgfscope}%
\pgfpathrectangle{\pgfqpoint{0.374692in}{0.319877in}}{\pgfqpoint{1.579674in}{2.592703in}} %
\pgfusepath{clip}%
\pgfsetbuttcap%
\pgfsetroundjoin%
\definecolor{currentfill}{rgb}{1.000000,0.752941,0.796078}%
\pgfsetfillcolor{currentfill}%
\pgfsetlinewidth{1.003750pt}%
\definecolor{currentstroke}{rgb}{1.000000,0.752941,0.796078}%
\pgfsetstrokecolor{currentstroke}%
\pgfsetdash{}{0pt}%
\pgfpathmoveto{\pgfqpoint{0.473422in}{1.613845in}}%
\pgfpathcurveto{\pgfqpoint{0.484472in}{1.613845in}}{\pgfqpoint{0.495071in}{1.618236in}}{\pgfqpoint{0.502884in}{1.626049in}}%
\pgfpathcurveto{\pgfqpoint{0.510698in}{1.633863in}}{\pgfqpoint{0.515088in}{1.644462in}}{\pgfqpoint{0.515088in}{1.655512in}}%
\pgfpathcurveto{\pgfqpoint{0.515088in}{1.666562in}}{\pgfqpoint{0.510698in}{1.677161in}}{\pgfqpoint{0.502884in}{1.684975in}}%
\pgfpathcurveto{\pgfqpoint{0.495071in}{1.692789in}}{\pgfqpoint{0.484472in}{1.697179in}}{\pgfqpoint{0.473422in}{1.697179in}}%
\pgfpathcurveto{\pgfqpoint{0.462371in}{1.697179in}}{\pgfqpoint{0.451772in}{1.692789in}}{\pgfqpoint{0.443959in}{1.684975in}}%
\pgfpathcurveto{\pgfqpoint{0.436145in}{1.677161in}}{\pgfqpoint{0.431755in}{1.666562in}}{\pgfqpoint{0.431755in}{1.655512in}}%
\pgfpathcurveto{\pgfqpoint{0.431755in}{1.644462in}}{\pgfqpoint{0.436145in}{1.633863in}}{\pgfqpoint{0.443959in}{1.626049in}}%
\pgfpathcurveto{\pgfqpoint{0.451772in}{1.618236in}}{\pgfqpoint{0.462371in}{1.613845in}}{\pgfqpoint{0.473422in}{1.613845in}}%
\pgfpathclose%
\pgfusepath{stroke,fill}%
\end{pgfscope}%
\begin{pgfscope}%
\pgfpathrectangle{\pgfqpoint{0.374692in}{0.319877in}}{\pgfqpoint{1.579674in}{2.592703in}} %
\pgfusepath{clip}%
\pgfsetbuttcap%
\pgfsetroundjoin%
\definecolor{currentfill}{rgb}{1.000000,0.752941,0.796078}%
\pgfsetfillcolor{currentfill}%
\pgfsetlinewidth{1.003750pt}%
\definecolor{currentstroke}{rgb}{1.000000,0.752941,0.796078}%
\pgfsetstrokecolor{currentstroke}%
\pgfsetdash{}{0pt}%
\pgfpathmoveto{\pgfqpoint{0.670881in}{1.507219in}}%
\pgfpathcurveto{\pgfqpoint{0.681931in}{1.507219in}}{\pgfqpoint{0.692530in}{1.511609in}}{\pgfqpoint{0.700344in}{1.519423in}}%
\pgfpathcurveto{\pgfqpoint{0.708157in}{1.527237in}}{\pgfqpoint{0.712547in}{1.537836in}}{\pgfqpoint{0.712547in}{1.548886in}}%
\pgfpathcurveto{\pgfqpoint{0.712547in}{1.559936in}}{\pgfqpoint{0.708157in}{1.570535in}}{\pgfqpoint{0.700344in}{1.578349in}}%
\pgfpathcurveto{\pgfqpoint{0.692530in}{1.586162in}}{\pgfqpoint{0.681931in}{1.590552in}}{\pgfqpoint{0.670881in}{1.590552in}}%
\pgfpathcurveto{\pgfqpoint{0.659831in}{1.590552in}}{\pgfqpoint{0.649232in}{1.586162in}}{\pgfqpoint{0.641418in}{1.578349in}}%
\pgfpathcurveto{\pgfqpoint{0.633604in}{1.570535in}}{\pgfqpoint{0.629214in}{1.559936in}}{\pgfqpoint{0.629214in}{1.548886in}}%
\pgfpathcurveto{\pgfqpoint{0.629214in}{1.537836in}}{\pgfqpoint{0.633604in}{1.527237in}}{\pgfqpoint{0.641418in}{1.519423in}}%
\pgfpathcurveto{\pgfqpoint{0.649232in}{1.511609in}}{\pgfqpoint{0.659831in}{1.507219in}}{\pgfqpoint{0.670881in}{1.507219in}}%
\pgfpathclose%
\pgfusepath{stroke,fill}%
\end{pgfscope}%
\begin{pgfscope}%
\pgfpathrectangle{\pgfqpoint{0.374692in}{0.319877in}}{\pgfqpoint{1.579674in}{2.592703in}} %
\pgfusepath{clip}%
\pgfsetbuttcap%
\pgfsetroundjoin%
\definecolor{currentfill}{rgb}{1.000000,0.752941,0.796078}%
\pgfsetfillcolor{currentfill}%
\pgfsetlinewidth{1.003750pt}%
\definecolor{currentstroke}{rgb}{1.000000,0.752941,0.796078}%
\pgfsetstrokecolor{currentstroke}%
\pgfsetdash{}{0pt}%
\pgfpathmoveto{\pgfqpoint{0.868340in}{1.532072in}}%
\pgfpathcurveto{\pgfqpoint{0.879390in}{1.532072in}}{\pgfqpoint{0.889989in}{1.536462in}}{\pgfqpoint{0.897803in}{1.544276in}}%
\pgfpathcurveto{\pgfqpoint{0.905616in}{1.552089in}}{\pgfqpoint{0.910007in}{1.562688in}}{\pgfqpoint{0.910007in}{1.573739in}}%
\pgfpathcurveto{\pgfqpoint{0.910007in}{1.584789in}}{\pgfqpoint{0.905616in}{1.595388in}}{\pgfqpoint{0.897803in}{1.603201in}}%
\pgfpathcurveto{\pgfqpoint{0.889989in}{1.611015in}}{\pgfqpoint{0.879390in}{1.615405in}}{\pgfqpoint{0.868340in}{1.615405in}}%
\pgfpathcurveto{\pgfqpoint{0.857290in}{1.615405in}}{\pgfqpoint{0.846691in}{1.611015in}}{\pgfqpoint{0.838877in}{1.603201in}}%
\pgfpathcurveto{\pgfqpoint{0.831064in}{1.595388in}}{\pgfqpoint{0.826673in}{1.584789in}}{\pgfqpoint{0.826673in}{1.573739in}}%
\pgfpathcurveto{\pgfqpoint{0.826673in}{1.562688in}}{\pgfqpoint{0.831064in}{1.552089in}}{\pgfqpoint{0.838877in}{1.544276in}}%
\pgfpathcurveto{\pgfqpoint{0.846691in}{1.536462in}}{\pgfqpoint{0.857290in}{1.532072in}}{\pgfqpoint{0.868340in}{1.532072in}}%
\pgfpathclose%
\pgfusepath{stroke,fill}%
\end{pgfscope}%
\begin{pgfscope}%
\pgfpathrectangle{\pgfqpoint{0.374692in}{0.319877in}}{\pgfqpoint{1.579674in}{2.592703in}} %
\pgfusepath{clip}%
\pgfsetbuttcap%
\pgfsetroundjoin%
\definecolor{currentfill}{rgb}{1.000000,0.752941,0.796078}%
\pgfsetfillcolor{currentfill}%
\pgfsetlinewidth{1.003750pt}%
\definecolor{currentstroke}{rgb}{1.000000,0.752941,0.796078}%
\pgfsetstrokecolor{currentstroke}%
\pgfsetdash{}{0pt}%
\pgfpathmoveto{\pgfqpoint{1.065799in}{1.608777in}}%
\pgfpathcurveto{\pgfqpoint{1.076849in}{1.608777in}}{\pgfqpoint{1.087448in}{1.613167in}}{\pgfqpoint{1.095262in}{1.620980in}}%
\pgfpathcurveto{\pgfqpoint{1.103076in}{1.628794in}}{\pgfqpoint{1.107466in}{1.639393in}}{\pgfqpoint{1.107466in}{1.650443in}}%
\pgfpathcurveto{\pgfqpoint{1.107466in}{1.661493in}}{\pgfqpoint{1.103076in}{1.672092in}}{\pgfqpoint{1.095262in}{1.679906in}}%
\pgfpathcurveto{\pgfqpoint{1.087448in}{1.687720in}}{\pgfqpoint{1.076849in}{1.692110in}}{\pgfqpoint{1.065799in}{1.692110in}}%
\pgfpathcurveto{\pgfqpoint{1.054749in}{1.692110in}}{\pgfqpoint{1.044150in}{1.687720in}}{\pgfqpoint{1.036336in}{1.679906in}}%
\pgfpathcurveto{\pgfqpoint{1.028523in}{1.672092in}}{\pgfqpoint{1.024133in}{1.661493in}}{\pgfqpoint{1.024133in}{1.650443in}}%
\pgfpathcurveto{\pgfqpoint{1.024133in}{1.639393in}}{\pgfqpoint{1.028523in}{1.628794in}}{\pgfqpoint{1.036336in}{1.620980in}}%
\pgfpathcurveto{\pgfqpoint{1.044150in}{1.613167in}}{\pgfqpoint{1.054749in}{1.608777in}}{\pgfqpoint{1.065799in}{1.608777in}}%
\pgfpathclose%
\pgfusepath{stroke,fill}%
\end{pgfscope}%
\begin{pgfscope}%
\pgfpathrectangle{\pgfqpoint{0.374692in}{0.319877in}}{\pgfqpoint{1.579674in}{2.592703in}} %
\pgfusepath{clip}%
\pgfsetbuttcap%
\pgfsetroundjoin%
\definecolor{currentfill}{rgb}{1.000000,0.752941,0.796078}%
\pgfsetfillcolor{currentfill}%
\pgfsetlinewidth{1.003750pt}%
\definecolor{currentstroke}{rgb}{1.000000,0.752941,0.796078}%
\pgfsetstrokecolor{currentstroke}%
\pgfsetdash{}{0pt}%
\pgfpathmoveto{\pgfqpoint{1.263258in}{1.584900in}}%
\pgfpathcurveto{\pgfqpoint{1.274309in}{1.584900in}}{\pgfqpoint{1.284908in}{1.589290in}}{\pgfqpoint{1.292721in}{1.597104in}}%
\pgfpathcurveto{\pgfqpoint{1.300535in}{1.604917in}}{\pgfqpoint{1.304925in}{1.615516in}}{\pgfqpoint{1.304925in}{1.626566in}}%
\pgfpathcurveto{\pgfqpoint{1.304925in}{1.637617in}}{\pgfqpoint{1.300535in}{1.648216in}}{\pgfqpoint{1.292721in}{1.656029in}}%
\pgfpathcurveto{\pgfqpoint{1.284908in}{1.663843in}}{\pgfqpoint{1.274309in}{1.668233in}}{\pgfqpoint{1.263258in}{1.668233in}}%
\pgfpathcurveto{\pgfqpoint{1.252208in}{1.668233in}}{\pgfqpoint{1.241609in}{1.663843in}}{\pgfqpoint{1.233796in}{1.656029in}}%
\pgfpathcurveto{\pgfqpoint{1.225982in}{1.648216in}}{\pgfqpoint{1.221592in}{1.637617in}}{\pgfqpoint{1.221592in}{1.626566in}}%
\pgfpathcurveto{\pgfqpoint{1.221592in}{1.615516in}}{\pgfqpoint{1.225982in}{1.604917in}}{\pgfqpoint{1.233796in}{1.597104in}}%
\pgfpathcurveto{\pgfqpoint{1.241609in}{1.589290in}}{\pgfqpoint{1.252208in}{1.584900in}}{\pgfqpoint{1.263258in}{1.584900in}}%
\pgfpathclose%
\pgfusepath{stroke,fill}%
\end{pgfscope}%
\begin{pgfscope}%
\pgfpathrectangle{\pgfqpoint{0.374692in}{0.319877in}}{\pgfqpoint{1.579674in}{2.592703in}} %
\pgfusepath{clip}%
\pgfsetbuttcap%
\pgfsetroundjoin%
\definecolor{currentfill}{rgb}{1.000000,0.752941,0.796078}%
\pgfsetfillcolor{currentfill}%
\pgfsetlinewidth{1.003750pt}%
\definecolor{currentstroke}{rgb}{1.000000,0.752941,0.796078}%
\pgfsetstrokecolor{currentstroke}%
\pgfsetdash{}{0pt}%
\pgfpathmoveto{\pgfqpoint{1.460718in}{1.613845in}}%
\pgfpathcurveto{\pgfqpoint{1.471768in}{1.613845in}}{\pgfqpoint{1.482367in}{1.618236in}}{\pgfqpoint{1.490180in}{1.626049in}}%
\pgfpathcurveto{\pgfqpoint{1.497994in}{1.633863in}}{\pgfqpoint{1.502384in}{1.644462in}}{\pgfqpoint{1.502384in}{1.655512in}}%
\pgfpathcurveto{\pgfqpoint{1.502384in}{1.666562in}}{\pgfqpoint{1.497994in}{1.677161in}}{\pgfqpoint{1.490180in}{1.684975in}}%
\pgfpathcurveto{\pgfqpoint{1.482367in}{1.692789in}}{\pgfqpoint{1.471768in}{1.697179in}}{\pgfqpoint{1.460718in}{1.697179in}}%
\pgfpathcurveto{\pgfqpoint{1.449668in}{1.697179in}}{\pgfqpoint{1.439069in}{1.692789in}}{\pgfqpoint{1.431255in}{1.684975in}}%
\pgfpathcurveto{\pgfqpoint{1.423441in}{1.677161in}}{\pgfqpoint{1.419051in}{1.666562in}}{\pgfqpoint{1.419051in}{1.655512in}}%
\pgfpathcurveto{\pgfqpoint{1.419051in}{1.644462in}}{\pgfqpoint{1.423441in}{1.633863in}}{\pgfqpoint{1.431255in}{1.626049in}}%
\pgfpathcurveto{\pgfqpoint{1.439069in}{1.618236in}}{\pgfqpoint{1.449668in}{1.613845in}}{\pgfqpoint{1.460718in}{1.613845in}}%
\pgfpathclose%
\pgfusepath{stroke,fill}%
\end{pgfscope}%
\begin{pgfscope}%
\pgfpathrectangle{\pgfqpoint{0.374692in}{0.319877in}}{\pgfqpoint{1.579674in}{2.592703in}} %
\pgfusepath{clip}%
\pgfsetbuttcap%
\pgfsetroundjoin%
\definecolor{currentfill}{rgb}{1.000000,0.752941,0.796078}%
\pgfsetfillcolor{currentfill}%
\pgfsetlinewidth{1.003750pt}%
\definecolor{currentstroke}{rgb}{1.000000,0.752941,0.796078}%
\pgfsetstrokecolor{currentstroke}%
\pgfsetdash{}{0pt}%
\pgfpathmoveto{\pgfqpoint{1.658177in}{1.751337in}}%
\pgfpathcurveto{\pgfqpoint{1.669227in}{1.751337in}}{\pgfqpoint{1.679826in}{1.755728in}}{\pgfqpoint{1.687640in}{1.763541in}}%
\pgfpathcurveto{\pgfqpoint{1.695453in}{1.771355in}}{\pgfqpoint{1.699844in}{1.781954in}}{\pgfqpoint{1.699844in}{1.793004in}}%
\pgfpathcurveto{\pgfqpoint{1.699844in}{1.804054in}}{\pgfqpoint{1.695453in}{1.814653in}}{\pgfqpoint{1.687640in}{1.822467in}}%
\pgfpathcurveto{\pgfqpoint{1.679826in}{1.830280in}}{\pgfqpoint{1.669227in}{1.834671in}}{\pgfqpoint{1.658177in}{1.834671in}}%
\pgfpathcurveto{\pgfqpoint{1.647127in}{1.834671in}}{\pgfqpoint{1.636528in}{1.830280in}}{\pgfqpoint{1.628714in}{1.822467in}}%
\pgfpathcurveto{\pgfqpoint{1.620901in}{1.814653in}}{\pgfqpoint{1.616510in}{1.804054in}}{\pgfqpoint{1.616510in}{1.793004in}}%
\pgfpathcurveto{\pgfqpoint{1.616510in}{1.781954in}}{\pgfqpoint{1.620901in}{1.771355in}}{\pgfqpoint{1.628714in}{1.763541in}}%
\pgfpathcurveto{\pgfqpoint{1.636528in}{1.755728in}}{\pgfqpoint{1.647127in}{1.751337in}}{\pgfqpoint{1.658177in}{1.751337in}}%
\pgfpathclose%
\pgfusepath{stroke,fill}%
\end{pgfscope}%
\begin{pgfscope}%
\pgfpathrectangle{\pgfqpoint{0.374692in}{0.319877in}}{\pgfqpoint{1.579674in}{2.592703in}} %
\pgfusepath{clip}%
\pgfsetbuttcap%
\pgfsetroundjoin%
\definecolor{currentfill}{rgb}{1.000000,0.752941,0.796078}%
\pgfsetfillcolor{currentfill}%
\pgfsetlinewidth{1.003750pt}%
\definecolor{currentstroke}{rgb}{1.000000,0.752941,0.796078}%
\pgfsetstrokecolor{currentstroke}%
\pgfsetdash{}{0pt}%
\pgfpathmoveto{\pgfqpoint{1.855636in}{1.692412in}}%
\pgfpathcurveto{\pgfqpoint{1.866686in}{1.692412in}}{\pgfqpoint{1.877285in}{1.696802in}}{\pgfqpoint{1.885099in}{1.704616in}}%
\pgfpathcurveto{\pgfqpoint{1.892913in}{1.712430in}}{\pgfqpoint{1.897303in}{1.723029in}}{\pgfqpoint{1.897303in}{1.734079in}}%
\pgfpathcurveto{\pgfqpoint{1.897303in}{1.745129in}}{\pgfqpoint{1.892913in}{1.755728in}}{\pgfqpoint{1.885099in}{1.763542in}}%
\pgfpathcurveto{\pgfqpoint{1.877285in}{1.771355in}}{\pgfqpoint{1.866686in}{1.775746in}}{\pgfqpoint{1.855636in}{1.775746in}}%
\pgfpathcurveto{\pgfqpoint{1.844586in}{1.775746in}}{\pgfqpoint{1.833987in}{1.771355in}}{\pgfqpoint{1.826173in}{1.763542in}}%
\pgfpathcurveto{\pgfqpoint{1.818360in}{1.755728in}}{\pgfqpoint{1.813969in}{1.745129in}}{\pgfqpoint{1.813969in}{1.734079in}}%
\pgfpathcurveto{\pgfqpoint{1.813969in}{1.723029in}}{\pgfqpoint{1.818360in}{1.712430in}}{\pgfqpoint{1.826173in}{1.704616in}}%
\pgfpathcurveto{\pgfqpoint{1.833987in}{1.696802in}}{\pgfqpoint{1.844586in}{1.692412in}}{\pgfqpoint{1.855636in}{1.692412in}}%
\pgfpathclose%
\pgfusepath{stroke,fill}%
\end{pgfscope}%
\begin{pgfscope}%
\pgfsetbuttcap%
\pgfsetroundjoin%
\definecolor{currentfill}{rgb}{0.000000,0.000000,0.000000}%
\pgfsetfillcolor{currentfill}%
\pgfsetlinewidth{0.803000pt}%
\definecolor{currentstroke}{rgb}{0.000000,0.000000,0.000000}%
\pgfsetstrokecolor{currentstroke}%
\pgfsetdash{}{0pt}%
\pgfsys@defobject{currentmarker}{\pgfqpoint{0.000000in}{-0.048611in}}{\pgfqpoint{0.000000in}{0.000000in}}{%
\pgfpathmoveto{\pgfqpoint{0.000000in}{0.000000in}}%
\pgfpathlineto{\pgfqpoint{0.000000in}{-0.048611in}}%
\pgfusepath{stroke,fill}%
}%
\begin{pgfscope}%
\pgfsys@transformshift{0.670881in}{0.319877in}%
\pgfsys@useobject{currentmarker}{}%
\end{pgfscope}%
\end{pgfscope}%
\begin{pgfscope}%
\pgftext[x=0.670881in,y=0.222655in,,top]{\rmfamily\fontsize{10.000000}{12.000000}\selectfont \(\displaystyle -0.05\)}%
\end{pgfscope}%
\begin{pgfscope}%
\pgfsetbuttcap%
\pgfsetroundjoin%
\definecolor{currentfill}{rgb}{0.000000,0.000000,0.000000}%
\pgfsetfillcolor{currentfill}%
\pgfsetlinewidth{0.803000pt}%
\definecolor{currentstroke}{rgb}{0.000000,0.000000,0.000000}%
\pgfsetstrokecolor{currentstroke}%
\pgfsetdash{}{0pt}%
\pgfsys@defobject{currentmarker}{\pgfqpoint{0.000000in}{-0.048611in}}{\pgfqpoint{0.000000in}{0.000000in}}{%
\pgfpathmoveto{\pgfqpoint{0.000000in}{0.000000in}}%
\pgfpathlineto{\pgfqpoint{0.000000in}{-0.048611in}}%
\pgfusepath{stroke,fill}%
}%
\begin{pgfscope}%
\pgfsys@transformshift{1.164529in}{0.319877in}%
\pgfsys@useobject{currentmarker}{}%
\end{pgfscope}%
\end{pgfscope}%
\begin{pgfscope}%
\pgftext[x=1.164529in,y=0.222655in,,top]{\rmfamily\fontsize{10.000000}{12.000000}\selectfont \(\displaystyle 0.00\)}%
\end{pgfscope}%
\begin{pgfscope}%
\pgfsetbuttcap%
\pgfsetroundjoin%
\definecolor{currentfill}{rgb}{0.000000,0.000000,0.000000}%
\pgfsetfillcolor{currentfill}%
\pgfsetlinewidth{0.803000pt}%
\definecolor{currentstroke}{rgb}{0.000000,0.000000,0.000000}%
\pgfsetstrokecolor{currentstroke}%
\pgfsetdash{}{0pt}%
\pgfsys@defobject{currentmarker}{\pgfqpoint{0.000000in}{-0.048611in}}{\pgfqpoint{0.000000in}{0.000000in}}{%
\pgfpathmoveto{\pgfqpoint{0.000000in}{0.000000in}}%
\pgfpathlineto{\pgfqpoint{0.000000in}{-0.048611in}}%
\pgfusepath{stroke,fill}%
}%
\begin{pgfscope}%
\pgfsys@transformshift{1.658177in}{0.319877in}%
\pgfsys@useobject{currentmarker}{}%
\end{pgfscope}%
\end{pgfscope}%
\begin{pgfscope}%
\pgftext[x=1.658177in,y=0.222655in,,top]{\rmfamily\fontsize{10.000000}{12.000000}\selectfont \(\displaystyle 0.05\)}%
\end{pgfscope}%
\begin{pgfscope}%
\pgfsetbuttcap%
\pgfsetroundjoin%
\definecolor{currentfill}{rgb}{0.000000,0.000000,0.000000}%
\pgfsetfillcolor{currentfill}%
\pgfsetlinewidth{0.803000pt}%
\definecolor{currentstroke}{rgb}{0.000000,0.000000,0.000000}%
\pgfsetstrokecolor{currentstroke}%
\pgfsetdash{}{0pt}%
\pgfsys@defobject{currentmarker}{\pgfqpoint{-0.048611in}{0.000000in}}{\pgfqpoint{0.000000in}{0.000000in}}{%
\pgfpathmoveto{\pgfqpoint{0.000000in}{0.000000in}}%
\pgfpathlineto{\pgfqpoint{-0.048611in}{0.000000in}}%
\pgfusepath{stroke,fill}%
}%
\begin{pgfscope}%
\pgfsys@transformshift{0.374692in}{0.319877in}%
\pgfsys@useobject{currentmarker}{}%
\end{pgfscope}%
\end{pgfscope}%
\begin{pgfscope}%
\pgftext[x=0.100000in,y=0.272050in,left,base]{\rmfamily\fontsize{10.000000}{12.000000}\selectfont \(\displaystyle 0.0\)}%
\end{pgfscope}%
\begin{pgfscope}%
\pgfsetbuttcap%
\pgfsetroundjoin%
\definecolor{currentfill}{rgb}{0.000000,0.000000,0.000000}%
\pgfsetfillcolor{currentfill}%
\pgfsetlinewidth{0.803000pt}%
\definecolor{currentstroke}{rgb}{0.000000,0.000000,0.000000}%
\pgfsetstrokecolor{currentstroke}%
\pgfsetdash{}{0pt}%
\pgfsys@defobject{currentmarker}{\pgfqpoint{-0.048611in}{0.000000in}}{\pgfqpoint{0.000000in}{0.000000in}}{%
\pgfpathmoveto{\pgfqpoint{0.000000in}{0.000000in}}%
\pgfpathlineto{\pgfqpoint{-0.048611in}{0.000000in}}%
\pgfusepath{stroke,fill}%
}%
\begin{pgfscope}%
\pgfsys@transformshift{0.374692in}{0.838418in}%
\pgfsys@useobject{currentmarker}{}%
\end{pgfscope}%
\end{pgfscope}%
\begin{pgfscope}%
\pgftext[x=0.100000in,y=0.790590in,left,base]{\rmfamily\fontsize{10.000000}{12.000000}\selectfont \(\displaystyle 0.2\)}%
\end{pgfscope}%
\begin{pgfscope}%
\pgfsetbuttcap%
\pgfsetroundjoin%
\definecolor{currentfill}{rgb}{0.000000,0.000000,0.000000}%
\pgfsetfillcolor{currentfill}%
\pgfsetlinewidth{0.803000pt}%
\definecolor{currentstroke}{rgb}{0.000000,0.000000,0.000000}%
\pgfsetstrokecolor{currentstroke}%
\pgfsetdash{}{0pt}%
\pgfsys@defobject{currentmarker}{\pgfqpoint{-0.048611in}{0.000000in}}{\pgfqpoint{0.000000in}{0.000000in}}{%
\pgfpathmoveto{\pgfqpoint{0.000000in}{0.000000in}}%
\pgfpathlineto{\pgfqpoint{-0.048611in}{0.000000in}}%
\pgfusepath{stroke,fill}%
}%
\begin{pgfscope}%
\pgfsys@transformshift{0.374692in}{1.356958in}%
\pgfsys@useobject{currentmarker}{}%
\end{pgfscope}%
\end{pgfscope}%
\begin{pgfscope}%
\pgftext[x=0.100000in,y=1.309131in,left,base]{\rmfamily\fontsize{10.000000}{12.000000}\selectfont \(\displaystyle 0.4\)}%
\end{pgfscope}%
\begin{pgfscope}%
\pgfsetbuttcap%
\pgfsetroundjoin%
\definecolor{currentfill}{rgb}{0.000000,0.000000,0.000000}%
\pgfsetfillcolor{currentfill}%
\pgfsetlinewidth{0.803000pt}%
\definecolor{currentstroke}{rgb}{0.000000,0.000000,0.000000}%
\pgfsetstrokecolor{currentstroke}%
\pgfsetdash{}{0pt}%
\pgfsys@defobject{currentmarker}{\pgfqpoint{-0.048611in}{0.000000in}}{\pgfqpoint{0.000000in}{0.000000in}}{%
\pgfpathmoveto{\pgfqpoint{0.000000in}{0.000000in}}%
\pgfpathlineto{\pgfqpoint{-0.048611in}{0.000000in}}%
\pgfusepath{stroke,fill}%
}%
\begin{pgfscope}%
\pgfsys@transformshift{0.374692in}{1.875499in}%
\pgfsys@useobject{currentmarker}{}%
\end{pgfscope}%
\end{pgfscope}%
\begin{pgfscope}%
\pgftext[x=0.100000in,y=1.827671in,left,base]{\rmfamily\fontsize{10.000000}{12.000000}\selectfont \(\displaystyle 0.6\)}%
\end{pgfscope}%
\begin{pgfscope}%
\pgfsetbuttcap%
\pgfsetroundjoin%
\definecolor{currentfill}{rgb}{0.000000,0.000000,0.000000}%
\pgfsetfillcolor{currentfill}%
\pgfsetlinewidth{0.803000pt}%
\definecolor{currentstroke}{rgb}{0.000000,0.000000,0.000000}%
\pgfsetstrokecolor{currentstroke}%
\pgfsetdash{}{0pt}%
\pgfsys@defobject{currentmarker}{\pgfqpoint{-0.048611in}{0.000000in}}{\pgfqpoint{0.000000in}{0.000000in}}{%
\pgfpathmoveto{\pgfqpoint{0.000000in}{0.000000in}}%
\pgfpathlineto{\pgfqpoint{-0.048611in}{0.000000in}}%
\pgfusepath{stroke,fill}%
}%
\begin{pgfscope}%
\pgfsys@transformshift{0.374692in}{2.394040in}%
\pgfsys@useobject{currentmarker}{}%
\end{pgfscope}%
\end{pgfscope}%
\begin{pgfscope}%
\pgftext[x=0.100000in,y=2.346212in,left,base]{\rmfamily\fontsize{10.000000}{12.000000}\selectfont \(\displaystyle 0.8\)}%
\end{pgfscope}%
\begin{pgfscope}%
\pgfsetbuttcap%
\pgfsetroundjoin%
\definecolor{currentfill}{rgb}{0.000000,0.000000,0.000000}%
\pgfsetfillcolor{currentfill}%
\pgfsetlinewidth{0.803000pt}%
\definecolor{currentstroke}{rgb}{0.000000,0.000000,0.000000}%
\pgfsetstrokecolor{currentstroke}%
\pgfsetdash{}{0pt}%
\pgfsys@defobject{currentmarker}{\pgfqpoint{-0.048611in}{0.000000in}}{\pgfqpoint{0.000000in}{0.000000in}}{%
\pgfpathmoveto{\pgfqpoint{0.000000in}{0.000000in}}%
\pgfpathlineto{\pgfqpoint{-0.048611in}{0.000000in}}%
\pgfusepath{stroke,fill}%
}%
\begin{pgfscope}%
\pgfsys@transformshift{0.374692in}{2.912580in}%
\pgfsys@useobject{currentmarker}{}%
\end{pgfscope}%
\end{pgfscope}%
\begin{pgfscope}%
\pgftext[x=0.100000in,y=2.864752in,left,base]{\rmfamily\fontsize{10.000000}{12.000000}\selectfont \(\displaystyle 1.0\)}%
\end{pgfscope}%
\begin{pgfscope}%
\pgfsetrectcap%
\pgfsetmiterjoin%
\pgfsetlinewidth{0.803000pt}%
\definecolor{currentstroke}{rgb}{0.000000,0.000000,0.000000}%
\pgfsetstrokecolor{currentstroke}%
\pgfsetdash{}{0pt}%
\pgfpathmoveto{\pgfqpoint{0.374692in}{0.319877in}}%
\pgfpathlineto{\pgfqpoint{0.374692in}{2.912580in}}%
\pgfusepath{stroke}%
\end{pgfscope}%
\begin{pgfscope}%
\pgfsetrectcap%
\pgfsetmiterjoin%
\pgfsetlinewidth{0.803000pt}%
\definecolor{currentstroke}{rgb}{0.000000,0.000000,0.000000}%
\pgfsetstrokecolor{currentstroke}%
\pgfsetdash{}{0pt}%
\pgfpathmoveto{\pgfqpoint{1.954366in}{0.319877in}}%
\pgfpathlineto{\pgfqpoint{1.954366in}{2.912580in}}%
\pgfusepath{stroke}%
\end{pgfscope}%
\begin{pgfscope}%
\pgfsetrectcap%
\pgfsetmiterjoin%
\pgfsetlinewidth{0.803000pt}%
\definecolor{currentstroke}{rgb}{0.000000,0.000000,0.000000}%
\pgfsetstrokecolor{currentstroke}%
\pgfsetdash{}{0pt}%
\pgfpathmoveto{\pgfqpoint{0.374692in}{0.319877in}}%
\pgfpathlineto{\pgfqpoint{1.954366in}{0.319877in}}%
\pgfusepath{stroke}%
\end{pgfscope}%
\begin{pgfscope}%
\pgfsetrectcap%
\pgfsetmiterjoin%
\pgfsetlinewidth{0.803000pt}%
\definecolor{currentstroke}{rgb}{0.000000,0.000000,0.000000}%
\pgfsetstrokecolor{currentstroke}%
\pgfsetdash{}{0pt}%
\pgfpathmoveto{\pgfqpoint{0.374692in}{2.912580in}}%
\pgfpathlineto{\pgfqpoint{1.954366in}{2.912580in}}%
\pgfusepath{stroke}%
\end{pgfscope}%
\begin{pgfscope}%
\pgfpathrectangle{\pgfqpoint{2.053095in}{0.319877in}}{\pgfqpoint{0.129635in}{2.592703in}} %
\pgfusepath{clip}%
\pgfsetbuttcap%
\pgfsetmiterjoin%
\definecolor{currentfill}{rgb}{1.000000,1.000000,1.000000}%
\pgfsetfillcolor{currentfill}%
\pgfsetlinewidth{0.010037pt}%
\definecolor{currentstroke}{rgb}{1.000000,1.000000,1.000000}%
\pgfsetstrokecolor{currentstroke}%
\pgfsetdash{}{0pt}%
\pgfpathmoveto{\pgfqpoint{2.053095in}{0.319877in}}%
\pgfpathlineto{\pgfqpoint{2.053095in}{0.330005in}}%
\pgfpathlineto{\pgfqpoint{2.053095in}{2.902452in}}%
\pgfpathlineto{\pgfqpoint{2.053095in}{2.912580in}}%
\pgfpathlineto{\pgfqpoint{2.182730in}{2.912580in}}%
\pgfpathlineto{\pgfqpoint{2.182730in}{2.902452in}}%
\pgfpathlineto{\pgfqpoint{2.182730in}{0.330005in}}%
\pgfpathlineto{\pgfqpoint{2.182730in}{0.319877in}}%
\pgfpathclose%
\pgfusepath{stroke,fill}%
\end{pgfscope}%
\begin{pgfscope}%
\pgfsys@transformshift{2.050000in}{0.320408in}%
\pgftext[left,bottom]{\pgfimage[interpolate=true,width=0.130000in,height=2.590000in]{Perr_vs_dq_Ti_1000K-img1.png}}%
\end{pgfscope}%
\begin{pgfscope}%
\pgfsetbuttcap%
\pgfsetroundjoin%
\definecolor{currentfill}{rgb}{0.000000,0.000000,0.000000}%
\pgfsetfillcolor{currentfill}%
\pgfsetlinewidth{0.803000pt}%
\definecolor{currentstroke}{rgb}{0.000000,0.000000,0.000000}%
\pgfsetstrokecolor{currentstroke}%
\pgfsetdash{}{0pt}%
\pgfsys@defobject{currentmarker}{\pgfqpoint{0.000000in}{0.000000in}}{\pgfqpoint{0.048611in}{0.000000in}}{%
\pgfpathmoveto{\pgfqpoint{0.000000in}{0.000000in}}%
\pgfpathlineto{\pgfqpoint{0.048611in}{0.000000in}}%
\pgfusepath{stroke,fill}%
}%
\begin{pgfscope}%
\pgfsys@transformshift{2.182730in}{0.319877in}%
\pgfsys@useobject{currentmarker}{}%
\end{pgfscope}%
\end{pgfscope}%
\begin{pgfscope}%
\pgftext[x=2.279953in,y=0.272050in,left,base]{\rmfamily\fontsize{10.000000}{12.000000}\selectfont \(\displaystyle 0\)}%
\end{pgfscope}%
\begin{pgfscope}%
\pgfsetbuttcap%
\pgfsetroundjoin%
\definecolor{currentfill}{rgb}{0.000000,0.000000,0.000000}%
\pgfsetfillcolor{currentfill}%
\pgfsetlinewidth{0.803000pt}%
\definecolor{currentstroke}{rgb}{0.000000,0.000000,0.000000}%
\pgfsetstrokecolor{currentstroke}%
\pgfsetdash{}{0pt}%
\pgfsys@defobject{currentmarker}{\pgfqpoint{0.000000in}{0.000000in}}{\pgfqpoint{0.048611in}{0.000000in}}{%
\pgfpathmoveto{\pgfqpoint{0.000000in}{0.000000in}}%
\pgfpathlineto{\pgfqpoint{0.048611in}{0.000000in}}%
\pgfusepath{stroke,fill}%
}%
\begin{pgfscope}%
\pgfsys@transformshift{2.182730in}{0.718755in}%
\pgfsys@useobject{currentmarker}{}%
\end{pgfscope}%
\end{pgfscope}%
\begin{pgfscope}%
\pgftext[x=2.279953in,y=0.670927in,left,base]{\rmfamily\fontsize{10.000000}{12.000000}\selectfont \(\displaystyle 2\)}%
\end{pgfscope}%
\begin{pgfscope}%
\pgfsetbuttcap%
\pgfsetroundjoin%
\definecolor{currentfill}{rgb}{0.000000,0.000000,0.000000}%
\pgfsetfillcolor{currentfill}%
\pgfsetlinewidth{0.803000pt}%
\definecolor{currentstroke}{rgb}{0.000000,0.000000,0.000000}%
\pgfsetstrokecolor{currentstroke}%
\pgfsetdash{}{0pt}%
\pgfsys@defobject{currentmarker}{\pgfqpoint{0.000000in}{0.000000in}}{\pgfqpoint{0.048611in}{0.000000in}}{%
\pgfpathmoveto{\pgfqpoint{0.000000in}{0.000000in}}%
\pgfpathlineto{\pgfqpoint{0.048611in}{0.000000in}}%
\pgfusepath{stroke,fill}%
}%
\begin{pgfscope}%
\pgfsys@transformshift{2.182730in}{1.117632in}%
\pgfsys@useobject{currentmarker}{}%
\end{pgfscope}%
\end{pgfscope}%
\begin{pgfscope}%
\pgftext[x=2.279953in,y=1.069804in,left,base]{\rmfamily\fontsize{10.000000}{12.000000}\selectfont \(\displaystyle 4\)}%
\end{pgfscope}%
\begin{pgfscope}%
\pgfsetbuttcap%
\pgfsetroundjoin%
\definecolor{currentfill}{rgb}{0.000000,0.000000,0.000000}%
\pgfsetfillcolor{currentfill}%
\pgfsetlinewidth{0.803000pt}%
\definecolor{currentstroke}{rgb}{0.000000,0.000000,0.000000}%
\pgfsetstrokecolor{currentstroke}%
\pgfsetdash{}{0pt}%
\pgfsys@defobject{currentmarker}{\pgfqpoint{0.000000in}{0.000000in}}{\pgfqpoint{0.048611in}{0.000000in}}{%
\pgfpathmoveto{\pgfqpoint{0.000000in}{0.000000in}}%
\pgfpathlineto{\pgfqpoint{0.048611in}{0.000000in}}%
\pgfusepath{stroke,fill}%
}%
\begin{pgfscope}%
\pgfsys@transformshift{2.182730in}{1.516509in}%
\pgfsys@useobject{currentmarker}{}%
\end{pgfscope}%
\end{pgfscope}%
\begin{pgfscope}%
\pgftext[x=2.279953in,y=1.468682in,left,base]{\rmfamily\fontsize{10.000000}{12.000000}\selectfont \(\displaystyle 6\)}%
\end{pgfscope}%
\begin{pgfscope}%
\pgfsetbuttcap%
\pgfsetroundjoin%
\definecolor{currentfill}{rgb}{0.000000,0.000000,0.000000}%
\pgfsetfillcolor{currentfill}%
\pgfsetlinewidth{0.803000pt}%
\definecolor{currentstroke}{rgb}{0.000000,0.000000,0.000000}%
\pgfsetstrokecolor{currentstroke}%
\pgfsetdash{}{0pt}%
\pgfsys@defobject{currentmarker}{\pgfqpoint{0.000000in}{0.000000in}}{\pgfqpoint{0.048611in}{0.000000in}}{%
\pgfpathmoveto{\pgfqpoint{0.000000in}{0.000000in}}%
\pgfpathlineto{\pgfqpoint{0.048611in}{0.000000in}}%
\pgfusepath{stroke,fill}%
}%
\begin{pgfscope}%
\pgfsys@transformshift{2.182730in}{1.915387in}%
\pgfsys@useobject{currentmarker}{}%
\end{pgfscope}%
\end{pgfscope}%
\begin{pgfscope}%
\pgftext[x=2.279953in,y=1.867559in,left,base]{\rmfamily\fontsize{10.000000}{12.000000}\selectfont \(\displaystyle 8\)}%
\end{pgfscope}%
\begin{pgfscope}%
\pgfsetbuttcap%
\pgfsetroundjoin%
\definecolor{currentfill}{rgb}{0.000000,0.000000,0.000000}%
\pgfsetfillcolor{currentfill}%
\pgfsetlinewidth{0.803000pt}%
\definecolor{currentstroke}{rgb}{0.000000,0.000000,0.000000}%
\pgfsetstrokecolor{currentstroke}%
\pgfsetdash{}{0pt}%
\pgfsys@defobject{currentmarker}{\pgfqpoint{0.000000in}{0.000000in}}{\pgfqpoint{0.048611in}{0.000000in}}{%
\pgfpathmoveto{\pgfqpoint{0.000000in}{0.000000in}}%
\pgfpathlineto{\pgfqpoint{0.048611in}{0.000000in}}%
\pgfusepath{stroke,fill}%
}%
\begin{pgfscope}%
\pgfsys@transformshift{2.182730in}{2.314264in}%
\pgfsys@useobject{currentmarker}{}%
\end{pgfscope}%
\end{pgfscope}%
\begin{pgfscope}%
\pgftext[x=2.279953in,y=2.266436in,left,base]{\rmfamily\fontsize{10.000000}{12.000000}\selectfont \(\displaystyle 10\)}%
\end{pgfscope}%
\begin{pgfscope}%
\pgfsetbuttcap%
\pgfsetroundjoin%
\definecolor{currentfill}{rgb}{0.000000,0.000000,0.000000}%
\pgfsetfillcolor{currentfill}%
\pgfsetlinewidth{0.803000pt}%
\definecolor{currentstroke}{rgb}{0.000000,0.000000,0.000000}%
\pgfsetstrokecolor{currentstroke}%
\pgfsetdash{}{0pt}%
\pgfsys@defobject{currentmarker}{\pgfqpoint{0.000000in}{0.000000in}}{\pgfqpoint{0.048611in}{0.000000in}}{%
\pgfpathmoveto{\pgfqpoint{0.000000in}{0.000000in}}%
\pgfpathlineto{\pgfqpoint{0.048611in}{0.000000in}}%
\pgfusepath{stroke,fill}%
}%
\begin{pgfscope}%
\pgfsys@transformshift{2.182730in}{2.713141in}%
\pgfsys@useobject{currentmarker}{}%
\end{pgfscope}%
\end{pgfscope}%
\begin{pgfscope}%
\pgftext[x=2.279953in,y=2.665314in,left,base]{\rmfamily\fontsize{10.000000}{12.000000}\selectfont \(\displaystyle 12\)}%
\end{pgfscope}%
\begin{pgfscope}%
\pgfsetbuttcap%
\pgfsetmiterjoin%
\pgfsetlinewidth{0.803000pt}%
\definecolor{currentstroke}{rgb}{0.000000,0.000000,0.000000}%
\pgfsetstrokecolor{currentstroke}%
\pgfsetdash{}{0pt}%
\pgfpathmoveto{\pgfqpoint{2.053095in}{0.319877in}}%
\pgfpathlineto{\pgfqpoint{2.053095in}{0.330005in}}%
\pgfpathlineto{\pgfqpoint{2.053095in}{2.902452in}}%
\pgfpathlineto{\pgfqpoint{2.053095in}{2.912580in}}%
\pgfpathlineto{\pgfqpoint{2.182730in}{2.912580in}}%
\pgfpathlineto{\pgfqpoint{2.182730in}{2.902452in}}%
\pgfpathlineto{\pgfqpoint{2.182730in}{0.330005in}}%
\pgfpathlineto{\pgfqpoint{2.182730in}{0.319877in}}%
\pgfpathclose%
\pgfusepath{stroke}%
\end{pgfscope}%
\end{pgfpicture}%
\makeatother%
\endgroup%

    \vspace*{-0.4cm}
	\caption{1000 K. Bin size $0.020e$}
	\end{subfigure}
\caption{Change in dipole on an ion vs its change in charge}
\label{on_site_Perr_vs_dq}
\end{figure}

6) Figure \ref{on_site_PerrNN_vs_dq} looks at the relation between the change in charge on a Ti ion and the dipole moments of its nearest neighbour Oxygen shell. The exact quantity represented by the y-axis is defined as:
\begin{align*}
y_{i,I} \equiv (1/6)\sum_{s\in \text{NN}_i}\sqrt{\sum_{\alpha = x,y,z}\left(p_{s,I}^{\alpha}(\{q_l\})-p_{s,I}^{\alpha}(\bar{q}_{\text{Ba}},\bar{q}_{\text{Ti}},\bar{q}_{\text{O}})\right)^2}
\end{align*}
where $i\in \{\text{Ti}_1,...,\text{Ti}_{27}\}$, $I\in\{\text{MD}_1,...,\text{MD}_{10}\}$, $s\in\text{NN}_i=\{s=1,...,6 : \text{R}_{is} \text{ is n. n.}\}$ and the normalization was not performed for the reasons mentioned earlier. 

\begin{figure}[h!]
\centering
	\begin{subfigure}[b]{0.45\textwidth}
	\hspace*{-0.4cm}
	%% Creator: Matplotlib, PGF backend
%%
%% To include the figure in your LaTeX document, write
%%   \input{<filename>.pgf}
%%
%% Make sure the required packages are loaded in your preamble
%%   \usepackage{pgf}
%%
%% Figures using additional raster images can only be included by \input if
%% they are in the same directory as the main LaTeX file. For loading figures
%% from other directories you can use the `import` package
%%   \usepackage{import}
%% and then include the figures with
%%   \import{<path to file>}{<filename>.pgf}
%%
%% Matplotlib used the following preamble
%%   \usepackage[utf8x]{inputenc}
%%   \usepackage[T1]{fontenc}
%%
\begingroup%
\makeatletter%
\begin{pgfpicture}%
\pgfpathrectangle{\pgfpointorigin}{\pgfqpoint{2.518842in}{3.060408in}}%
\pgfusepath{use as bounding box, clip}%
\begin{pgfscope}%
\pgfsetbuttcap%
\pgfsetmiterjoin%
\definecolor{currentfill}{rgb}{1.000000,1.000000,1.000000}%
\pgfsetfillcolor{currentfill}%
\pgfsetlinewidth{0.000000pt}%
\definecolor{currentstroke}{rgb}{1.000000,1.000000,1.000000}%
\pgfsetstrokecolor{currentstroke}%
\pgfsetdash{}{0pt}%
\pgfpathmoveto{\pgfqpoint{0.000000in}{0.000000in}}%
\pgfpathlineto{\pgfqpoint{2.518842in}{0.000000in}}%
\pgfpathlineto{\pgfqpoint{2.518842in}{3.060408in}}%
\pgfpathlineto{\pgfqpoint{0.000000in}{3.060408in}}%
\pgfpathclose%
\pgfusepath{fill}%
\end{pgfscope}%
\begin{pgfscope}%
\pgfsetbuttcap%
\pgfsetmiterjoin%
\definecolor{currentfill}{rgb}{1.000000,1.000000,1.000000}%
\pgfsetfillcolor{currentfill}%
\pgfsetlinewidth{0.000000pt}%
\definecolor{currentstroke}{rgb}{0.000000,0.000000,0.000000}%
\pgfsetstrokecolor{currentstroke}%
\pgfsetstrokeopacity{0.000000}%
\pgfsetdash{}{0pt}%
\pgfpathmoveto{\pgfqpoint{0.374692in}{0.319877in}}%
\pgfpathlineto{\pgfqpoint{1.954366in}{0.319877in}}%
\pgfpathlineto{\pgfqpoint{1.954366in}{2.912580in}}%
\pgfpathlineto{\pgfqpoint{0.374692in}{2.912580in}}%
\pgfpathclose%
\pgfusepath{fill}%
\end{pgfscope}%
\begin{pgfscope}%
\pgfpathrectangle{\pgfqpoint{0.374692in}{0.319877in}}{\pgfqpoint{1.579674in}{2.592703in}} %
\pgfusepath{clip}%
\pgfsys@transformshift{0.374692in}{0.319877in}%
\pgftext[left,bottom]{\pgfimage[interpolate=true,width=1.580000in,height=2.600000in]{PerrNN_vs_dq_Ti_100K-img0.png}}%
\end{pgfscope}%
\begin{pgfscope}%
\pgfpathrectangle{\pgfqpoint{0.374692in}{0.319877in}}{\pgfqpoint{1.579674in}{2.592703in}} %
\pgfusepath{clip}%
\pgfsetbuttcap%
\pgfsetroundjoin%
\definecolor{currentfill}{rgb}{1.000000,0.752941,0.796078}%
\pgfsetfillcolor{currentfill}%
\pgfsetlinewidth{1.003750pt}%
\definecolor{currentstroke}{rgb}{1.000000,0.752941,0.796078}%
\pgfsetstrokecolor{currentstroke}%
\pgfsetdash{}{0pt}%
\pgfpathmoveto{\pgfqpoint{0.953906in}{0.789088in}}%
\pgfpathcurveto{\pgfqpoint{0.964956in}{0.789088in}}{\pgfqpoint{0.975555in}{0.793478in}}{\pgfqpoint{0.983368in}{0.801292in}}%
\pgfpathcurveto{\pgfqpoint{0.991182in}{0.809106in}}{\pgfqpoint{0.995572in}{0.819705in}}{\pgfqpoint{0.995572in}{0.830755in}}%
\pgfpathcurveto{\pgfqpoint{0.995572in}{0.841805in}}{\pgfqpoint{0.991182in}{0.852404in}}{\pgfqpoint{0.983368in}{0.860218in}}%
\pgfpathcurveto{\pgfqpoint{0.975555in}{0.868031in}}{\pgfqpoint{0.964956in}{0.872421in}}{\pgfqpoint{0.953906in}{0.872421in}}%
\pgfpathcurveto{\pgfqpoint{0.942856in}{0.872421in}}{\pgfqpoint{0.932257in}{0.868031in}}{\pgfqpoint{0.924443in}{0.860218in}}%
\pgfpathcurveto{\pgfqpoint{0.916629in}{0.852404in}}{\pgfqpoint{0.912239in}{0.841805in}}{\pgfqpoint{0.912239in}{0.830755in}}%
\pgfpathcurveto{\pgfqpoint{0.912239in}{0.819705in}}{\pgfqpoint{0.916629in}{0.809106in}}{\pgfqpoint{0.924443in}{0.801292in}}%
\pgfpathcurveto{\pgfqpoint{0.932257in}{0.793478in}}{\pgfqpoint{0.942856in}{0.789088in}}{\pgfqpoint{0.953906in}{0.789088in}}%
\pgfpathclose%
\pgfusepath{stroke,fill}%
\end{pgfscope}%
\begin{pgfscope}%
\pgfpathrectangle{\pgfqpoint{0.374692in}{0.319877in}}{\pgfqpoint{1.579674in}{2.592703in}} %
\pgfusepath{clip}%
\pgfsetbuttcap%
\pgfsetroundjoin%
\definecolor{currentfill}{rgb}{1.000000,0.752941,0.796078}%
\pgfsetfillcolor{currentfill}%
\pgfsetlinewidth{1.003750pt}%
\definecolor{currentstroke}{rgb}{1.000000,0.752941,0.796078}%
\pgfsetstrokecolor{currentstroke}%
\pgfsetdash{}{0pt}%
\pgfpathmoveto{\pgfqpoint{1.059217in}{0.847002in}}%
\pgfpathcurveto{\pgfqpoint{1.070267in}{0.847002in}}{\pgfqpoint{1.080866in}{0.851392in}}{\pgfqpoint{1.088680in}{0.859206in}}%
\pgfpathcurveto{\pgfqpoint{1.096494in}{0.867020in}}{\pgfqpoint{1.100884in}{0.877619in}}{\pgfqpoint{1.100884in}{0.888669in}}%
\pgfpathcurveto{\pgfqpoint{1.100884in}{0.899719in}}{\pgfqpoint{1.096494in}{0.910318in}}{\pgfqpoint{1.088680in}{0.918132in}}%
\pgfpathcurveto{\pgfqpoint{1.080866in}{0.925945in}}{\pgfqpoint{1.070267in}{0.930336in}}{\pgfqpoint{1.059217in}{0.930336in}}%
\pgfpathcurveto{\pgfqpoint{1.048167in}{0.930336in}}{\pgfqpoint{1.037568in}{0.925945in}}{\pgfqpoint{1.029754in}{0.918132in}}%
\pgfpathcurveto{\pgfqpoint{1.021941in}{0.910318in}}{\pgfqpoint{1.017551in}{0.899719in}}{\pgfqpoint{1.017551in}{0.888669in}}%
\pgfpathcurveto{\pgfqpoint{1.017551in}{0.877619in}}{\pgfqpoint{1.021941in}{0.867020in}}{\pgfqpoint{1.029754in}{0.859206in}}%
\pgfpathcurveto{\pgfqpoint{1.037568in}{0.851392in}}{\pgfqpoint{1.048167in}{0.847002in}}{\pgfqpoint{1.059217in}{0.847002in}}%
\pgfpathclose%
\pgfusepath{stroke,fill}%
\end{pgfscope}%
\begin{pgfscope}%
\pgfpathrectangle{\pgfqpoint{0.374692in}{0.319877in}}{\pgfqpoint{1.579674in}{2.592703in}} %
\pgfusepath{clip}%
\pgfsetbuttcap%
\pgfsetroundjoin%
\definecolor{currentfill}{rgb}{1.000000,0.752941,0.796078}%
\pgfsetfillcolor{currentfill}%
\pgfsetlinewidth{1.003750pt}%
\definecolor{currentstroke}{rgb}{1.000000,0.752941,0.796078}%
\pgfsetstrokecolor{currentstroke}%
\pgfsetdash{}{0pt}%
\pgfpathmoveto{\pgfqpoint{1.164529in}{0.942907in}}%
\pgfpathcurveto{\pgfqpoint{1.175579in}{0.942907in}}{\pgfqpoint{1.186178in}{0.947297in}}{\pgfqpoint{1.193992in}{0.955110in}}%
\pgfpathcurveto{\pgfqpoint{1.201805in}{0.962924in}}{\pgfqpoint{1.206196in}{0.973523in}}{\pgfqpoint{1.206196in}{0.984573in}}%
\pgfpathcurveto{\pgfqpoint{1.206196in}{0.995623in}}{\pgfqpoint{1.201805in}{1.006222in}}{\pgfqpoint{1.193992in}{1.014036in}}%
\pgfpathcurveto{\pgfqpoint{1.186178in}{1.021850in}}{\pgfqpoint{1.175579in}{1.026240in}}{\pgfqpoint{1.164529in}{1.026240in}}%
\pgfpathcurveto{\pgfqpoint{1.153479in}{1.026240in}}{\pgfqpoint{1.142880in}{1.021850in}}{\pgfqpoint{1.135066in}{1.014036in}}%
\pgfpathcurveto{\pgfqpoint{1.127252in}{1.006222in}}{\pgfqpoint{1.122862in}{0.995623in}}{\pgfqpoint{1.122862in}{0.984573in}}%
\pgfpathcurveto{\pgfqpoint{1.122862in}{0.973523in}}{\pgfqpoint{1.127252in}{0.962924in}}{\pgfqpoint{1.135066in}{0.955110in}}%
\pgfpathcurveto{\pgfqpoint{1.142880in}{0.947297in}}{\pgfqpoint{1.153479in}{0.942907in}}{\pgfqpoint{1.164529in}{0.942907in}}%
\pgfpathclose%
\pgfusepath{stroke,fill}%
\end{pgfscope}%
\begin{pgfscope}%
\pgfpathrectangle{\pgfqpoint{0.374692in}{0.319877in}}{\pgfqpoint{1.579674in}{2.592703in}} %
\pgfusepath{clip}%
\pgfsetbuttcap%
\pgfsetroundjoin%
\definecolor{currentfill}{rgb}{1.000000,0.752941,0.796078}%
\pgfsetfillcolor{currentfill}%
\pgfsetlinewidth{1.003750pt}%
\definecolor{currentstroke}{rgb}{1.000000,0.752941,0.796078}%
\pgfsetstrokecolor{currentstroke}%
\pgfsetdash{}{0pt}%
\pgfpathmoveto{\pgfqpoint{1.269840in}{0.883517in}}%
\pgfpathcurveto{\pgfqpoint{1.280891in}{0.883517in}}{\pgfqpoint{1.291490in}{0.887907in}}{\pgfqpoint{1.299303in}{0.895721in}}%
\pgfpathcurveto{\pgfqpoint{1.307117in}{0.903534in}}{\pgfqpoint{1.311507in}{0.914133in}}{\pgfqpoint{1.311507in}{0.925183in}}%
\pgfpathcurveto{\pgfqpoint{1.311507in}{0.936233in}}{\pgfqpoint{1.307117in}{0.946832in}}{\pgfqpoint{1.299303in}{0.954646in}}%
\pgfpathcurveto{\pgfqpoint{1.291490in}{0.962460in}}{\pgfqpoint{1.280891in}{0.966850in}}{\pgfqpoint{1.269840in}{0.966850in}}%
\pgfpathcurveto{\pgfqpoint{1.258790in}{0.966850in}}{\pgfqpoint{1.248191in}{0.962460in}}{\pgfqpoint{1.240378in}{0.954646in}}%
\pgfpathcurveto{\pgfqpoint{1.232564in}{0.946832in}}{\pgfqpoint{1.228174in}{0.936233in}}{\pgfqpoint{1.228174in}{0.925183in}}%
\pgfpathcurveto{\pgfqpoint{1.228174in}{0.914133in}}{\pgfqpoint{1.232564in}{0.903534in}}{\pgfqpoint{1.240378in}{0.895721in}}%
\pgfpathcurveto{\pgfqpoint{1.248191in}{0.887907in}}{\pgfqpoint{1.258790in}{0.883517in}}{\pgfqpoint{1.269840in}{0.883517in}}%
\pgfpathclose%
\pgfusepath{stroke,fill}%
\end{pgfscope}%
\begin{pgfscope}%
\pgfpathrectangle{\pgfqpoint{0.374692in}{0.319877in}}{\pgfqpoint{1.579674in}{2.592703in}} %
\pgfusepath{clip}%
\pgfsetbuttcap%
\pgfsetroundjoin%
\definecolor{currentfill}{rgb}{1.000000,0.752941,0.796078}%
\pgfsetfillcolor{currentfill}%
\pgfsetlinewidth{1.003750pt}%
\definecolor{currentstroke}{rgb}{1.000000,0.752941,0.796078}%
\pgfsetstrokecolor{currentstroke}%
\pgfsetdash{}{0pt}%
\pgfpathmoveto{\pgfqpoint{1.375152in}{0.471918in}}%
\pgfpathcurveto{\pgfqpoint{1.386202in}{0.471918in}}{\pgfqpoint{1.396801in}{0.476309in}}{\pgfqpoint{1.404615in}{0.484122in}}%
\pgfpathcurveto{\pgfqpoint{1.412428in}{0.491936in}}{\pgfqpoint{1.416819in}{0.502535in}}{\pgfqpoint{1.416819in}{0.513585in}}%
\pgfpathcurveto{\pgfqpoint{1.416819in}{0.524635in}}{\pgfqpoint{1.412428in}{0.535234in}}{\pgfqpoint{1.404615in}{0.543048in}}%
\pgfpathcurveto{\pgfqpoint{1.396801in}{0.550861in}}{\pgfqpoint{1.386202in}{0.555252in}}{\pgfqpoint{1.375152in}{0.555252in}}%
\pgfpathcurveto{\pgfqpoint{1.364102in}{0.555252in}}{\pgfqpoint{1.353503in}{0.550861in}}{\pgfqpoint{1.345689in}{0.543048in}}%
\pgfpathcurveto{\pgfqpoint{1.337876in}{0.535234in}}{\pgfqpoint{1.333485in}{0.524635in}}{\pgfqpoint{1.333485in}{0.513585in}}%
\pgfpathcurveto{\pgfqpoint{1.333485in}{0.502535in}}{\pgfqpoint{1.337876in}{0.491936in}}{\pgfqpoint{1.345689in}{0.484122in}}%
\pgfpathcurveto{\pgfqpoint{1.353503in}{0.476309in}}{\pgfqpoint{1.364102in}{0.471918in}}{\pgfqpoint{1.375152in}{0.471918in}}%
\pgfpathclose%
\pgfusepath{stroke,fill}%
\end{pgfscope}%
\begin{pgfscope}%
\pgfsetbuttcap%
\pgfsetroundjoin%
\definecolor{currentfill}{rgb}{0.000000,0.000000,0.000000}%
\pgfsetfillcolor{currentfill}%
\pgfsetlinewidth{0.803000pt}%
\definecolor{currentstroke}{rgb}{0.000000,0.000000,0.000000}%
\pgfsetstrokecolor{currentstroke}%
\pgfsetdash{}{0pt}%
\pgfsys@defobject{currentmarker}{\pgfqpoint{0.000000in}{-0.048611in}}{\pgfqpoint{0.000000in}{0.000000in}}{%
\pgfpathmoveto{\pgfqpoint{0.000000in}{0.000000in}}%
\pgfpathlineto{\pgfqpoint{0.000000in}{-0.048611in}}%
\pgfusepath{stroke,fill}%
}%
\begin{pgfscope}%
\pgfsys@transformshift{0.670881in}{0.319877in}%
\pgfsys@useobject{currentmarker}{}%
\end{pgfscope}%
\end{pgfscope}%
\begin{pgfscope}%
\pgftext[x=0.670881in,y=0.222655in,,top]{\rmfamily\fontsize{10.000000}{12.000000}\selectfont \(\displaystyle -0.05\)}%
\end{pgfscope}%
\begin{pgfscope}%
\pgfsetbuttcap%
\pgfsetroundjoin%
\definecolor{currentfill}{rgb}{0.000000,0.000000,0.000000}%
\pgfsetfillcolor{currentfill}%
\pgfsetlinewidth{0.803000pt}%
\definecolor{currentstroke}{rgb}{0.000000,0.000000,0.000000}%
\pgfsetstrokecolor{currentstroke}%
\pgfsetdash{}{0pt}%
\pgfsys@defobject{currentmarker}{\pgfqpoint{0.000000in}{-0.048611in}}{\pgfqpoint{0.000000in}{0.000000in}}{%
\pgfpathmoveto{\pgfqpoint{0.000000in}{0.000000in}}%
\pgfpathlineto{\pgfqpoint{0.000000in}{-0.048611in}}%
\pgfusepath{stroke,fill}%
}%
\begin{pgfscope}%
\pgfsys@transformshift{1.164529in}{0.319877in}%
\pgfsys@useobject{currentmarker}{}%
\end{pgfscope}%
\end{pgfscope}%
\begin{pgfscope}%
\pgftext[x=1.164529in,y=0.222655in,,top]{\rmfamily\fontsize{10.000000}{12.000000}\selectfont \(\displaystyle 0.00\)}%
\end{pgfscope}%
\begin{pgfscope}%
\pgfsetbuttcap%
\pgfsetroundjoin%
\definecolor{currentfill}{rgb}{0.000000,0.000000,0.000000}%
\pgfsetfillcolor{currentfill}%
\pgfsetlinewidth{0.803000pt}%
\definecolor{currentstroke}{rgb}{0.000000,0.000000,0.000000}%
\pgfsetstrokecolor{currentstroke}%
\pgfsetdash{}{0pt}%
\pgfsys@defobject{currentmarker}{\pgfqpoint{0.000000in}{-0.048611in}}{\pgfqpoint{0.000000in}{0.000000in}}{%
\pgfpathmoveto{\pgfqpoint{0.000000in}{0.000000in}}%
\pgfpathlineto{\pgfqpoint{0.000000in}{-0.048611in}}%
\pgfusepath{stroke,fill}%
}%
\begin{pgfscope}%
\pgfsys@transformshift{1.658177in}{0.319877in}%
\pgfsys@useobject{currentmarker}{}%
\end{pgfscope}%
\end{pgfscope}%
\begin{pgfscope}%
\pgftext[x=1.658177in,y=0.222655in,,top]{\rmfamily\fontsize{10.000000}{12.000000}\selectfont \(\displaystyle 0.05\)}%
\end{pgfscope}%
\begin{pgfscope}%
\pgfsetbuttcap%
\pgfsetroundjoin%
\definecolor{currentfill}{rgb}{0.000000,0.000000,0.000000}%
\pgfsetfillcolor{currentfill}%
\pgfsetlinewidth{0.803000pt}%
\definecolor{currentstroke}{rgb}{0.000000,0.000000,0.000000}%
\pgfsetstrokecolor{currentstroke}%
\pgfsetdash{}{0pt}%
\pgfsys@defobject{currentmarker}{\pgfqpoint{-0.048611in}{0.000000in}}{\pgfqpoint{0.000000in}{0.000000in}}{%
\pgfpathmoveto{\pgfqpoint{0.000000in}{0.000000in}}%
\pgfpathlineto{\pgfqpoint{-0.048611in}{0.000000in}}%
\pgfusepath{stroke,fill}%
}%
\begin{pgfscope}%
\pgfsys@transformshift{0.374692in}{0.319877in}%
\pgfsys@useobject{currentmarker}{}%
\end{pgfscope}%
\end{pgfscope}%
\begin{pgfscope}%
\pgftext[x=0.100000in,y=0.272050in,left,base]{\rmfamily\fontsize{10.000000}{12.000000}\selectfont \(\displaystyle 0.0\)}%
\end{pgfscope}%
\begin{pgfscope}%
\pgfsetbuttcap%
\pgfsetroundjoin%
\definecolor{currentfill}{rgb}{0.000000,0.000000,0.000000}%
\pgfsetfillcolor{currentfill}%
\pgfsetlinewidth{0.803000pt}%
\definecolor{currentstroke}{rgb}{0.000000,0.000000,0.000000}%
\pgfsetstrokecolor{currentstroke}%
\pgfsetdash{}{0pt}%
\pgfsys@defobject{currentmarker}{\pgfqpoint{-0.048611in}{0.000000in}}{\pgfqpoint{0.000000in}{0.000000in}}{%
\pgfpathmoveto{\pgfqpoint{0.000000in}{0.000000in}}%
\pgfpathlineto{\pgfqpoint{-0.048611in}{0.000000in}}%
\pgfusepath{stroke,fill}%
}%
\begin{pgfscope}%
\pgfsys@transformshift{0.374692in}{0.838418in}%
\pgfsys@useobject{currentmarker}{}%
\end{pgfscope}%
\end{pgfscope}%
\begin{pgfscope}%
\pgftext[x=0.100000in,y=0.790590in,left,base]{\rmfamily\fontsize{10.000000}{12.000000}\selectfont \(\displaystyle 0.1\)}%
\end{pgfscope}%
\begin{pgfscope}%
\pgfsetbuttcap%
\pgfsetroundjoin%
\definecolor{currentfill}{rgb}{0.000000,0.000000,0.000000}%
\pgfsetfillcolor{currentfill}%
\pgfsetlinewidth{0.803000pt}%
\definecolor{currentstroke}{rgb}{0.000000,0.000000,0.000000}%
\pgfsetstrokecolor{currentstroke}%
\pgfsetdash{}{0pt}%
\pgfsys@defobject{currentmarker}{\pgfqpoint{-0.048611in}{0.000000in}}{\pgfqpoint{0.000000in}{0.000000in}}{%
\pgfpathmoveto{\pgfqpoint{0.000000in}{0.000000in}}%
\pgfpathlineto{\pgfqpoint{-0.048611in}{0.000000in}}%
\pgfusepath{stroke,fill}%
}%
\begin{pgfscope}%
\pgfsys@transformshift{0.374692in}{1.356958in}%
\pgfsys@useobject{currentmarker}{}%
\end{pgfscope}%
\end{pgfscope}%
\begin{pgfscope}%
\pgftext[x=0.100000in,y=1.309131in,left,base]{\rmfamily\fontsize{10.000000}{12.000000}\selectfont \(\displaystyle 0.2\)}%
\end{pgfscope}%
\begin{pgfscope}%
\pgfsetbuttcap%
\pgfsetroundjoin%
\definecolor{currentfill}{rgb}{0.000000,0.000000,0.000000}%
\pgfsetfillcolor{currentfill}%
\pgfsetlinewidth{0.803000pt}%
\definecolor{currentstroke}{rgb}{0.000000,0.000000,0.000000}%
\pgfsetstrokecolor{currentstroke}%
\pgfsetdash{}{0pt}%
\pgfsys@defobject{currentmarker}{\pgfqpoint{-0.048611in}{0.000000in}}{\pgfqpoint{0.000000in}{0.000000in}}{%
\pgfpathmoveto{\pgfqpoint{0.000000in}{0.000000in}}%
\pgfpathlineto{\pgfqpoint{-0.048611in}{0.000000in}}%
\pgfusepath{stroke,fill}%
}%
\begin{pgfscope}%
\pgfsys@transformshift{0.374692in}{1.875499in}%
\pgfsys@useobject{currentmarker}{}%
\end{pgfscope}%
\end{pgfscope}%
\begin{pgfscope}%
\pgftext[x=0.100000in,y=1.827671in,left,base]{\rmfamily\fontsize{10.000000}{12.000000}\selectfont \(\displaystyle 0.3\)}%
\end{pgfscope}%
\begin{pgfscope}%
\pgfsetbuttcap%
\pgfsetroundjoin%
\definecolor{currentfill}{rgb}{0.000000,0.000000,0.000000}%
\pgfsetfillcolor{currentfill}%
\pgfsetlinewidth{0.803000pt}%
\definecolor{currentstroke}{rgb}{0.000000,0.000000,0.000000}%
\pgfsetstrokecolor{currentstroke}%
\pgfsetdash{}{0pt}%
\pgfsys@defobject{currentmarker}{\pgfqpoint{-0.048611in}{0.000000in}}{\pgfqpoint{0.000000in}{0.000000in}}{%
\pgfpathmoveto{\pgfqpoint{0.000000in}{0.000000in}}%
\pgfpathlineto{\pgfqpoint{-0.048611in}{0.000000in}}%
\pgfusepath{stroke,fill}%
}%
\begin{pgfscope}%
\pgfsys@transformshift{0.374692in}{2.394040in}%
\pgfsys@useobject{currentmarker}{}%
\end{pgfscope}%
\end{pgfscope}%
\begin{pgfscope}%
\pgftext[x=0.100000in,y=2.346212in,left,base]{\rmfamily\fontsize{10.000000}{12.000000}\selectfont \(\displaystyle 0.4\)}%
\end{pgfscope}%
\begin{pgfscope}%
\pgfsetbuttcap%
\pgfsetroundjoin%
\definecolor{currentfill}{rgb}{0.000000,0.000000,0.000000}%
\pgfsetfillcolor{currentfill}%
\pgfsetlinewidth{0.803000pt}%
\definecolor{currentstroke}{rgb}{0.000000,0.000000,0.000000}%
\pgfsetstrokecolor{currentstroke}%
\pgfsetdash{}{0pt}%
\pgfsys@defobject{currentmarker}{\pgfqpoint{-0.048611in}{0.000000in}}{\pgfqpoint{0.000000in}{0.000000in}}{%
\pgfpathmoveto{\pgfqpoint{0.000000in}{0.000000in}}%
\pgfpathlineto{\pgfqpoint{-0.048611in}{0.000000in}}%
\pgfusepath{stroke,fill}%
}%
\begin{pgfscope}%
\pgfsys@transformshift{0.374692in}{2.912580in}%
\pgfsys@useobject{currentmarker}{}%
\end{pgfscope}%
\end{pgfscope}%
\begin{pgfscope}%
\pgftext[x=0.100000in,y=2.864752in,left,base]{\rmfamily\fontsize{10.000000}{12.000000}\selectfont \(\displaystyle 0.5\)}%
\end{pgfscope}%
\begin{pgfscope}%
\pgfsetrectcap%
\pgfsetmiterjoin%
\pgfsetlinewidth{0.803000pt}%
\definecolor{currentstroke}{rgb}{0.000000,0.000000,0.000000}%
\pgfsetstrokecolor{currentstroke}%
\pgfsetdash{}{0pt}%
\pgfpathmoveto{\pgfqpoint{0.374692in}{0.319877in}}%
\pgfpathlineto{\pgfqpoint{0.374692in}{2.912580in}}%
\pgfusepath{stroke}%
\end{pgfscope}%
\begin{pgfscope}%
\pgfsetrectcap%
\pgfsetmiterjoin%
\pgfsetlinewidth{0.803000pt}%
\definecolor{currentstroke}{rgb}{0.000000,0.000000,0.000000}%
\pgfsetstrokecolor{currentstroke}%
\pgfsetdash{}{0pt}%
\pgfpathmoveto{\pgfqpoint{1.954366in}{0.319877in}}%
\pgfpathlineto{\pgfqpoint{1.954366in}{2.912580in}}%
\pgfusepath{stroke}%
\end{pgfscope}%
\begin{pgfscope}%
\pgfsetrectcap%
\pgfsetmiterjoin%
\pgfsetlinewidth{0.803000pt}%
\definecolor{currentstroke}{rgb}{0.000000,0.000000,0.000000}%
\pgfsetstrokecolor{currentstroke}%
\pgfsetdash{}{0pt}%
\pgfpathmoveto{\pgfqpoint{0.374692in}{0.319877in}}%
\pgfpathlineto{\pgfqpoint{1.954366in}{0.319877in}}%
\pgfusepath{stroke}%
\end{pgfscope}%
\begin{pgfscope}%
\pgfsetrectcap%
\pgfsetmiterjoin%
\pgfsetlinewidth{0.803000pt}%
\definecolor{currentstroke}{rgb}{0.000000,0.000000,0.000000}%
\pgfsetstrokecolor{currentstroke}%
\pgfsetdash{}{0pt}%
\pgfpathmoveto{\pgfqpoint{0.374692in}{2.912580in}}%
\pgfpathlineto{\pgfqpoint{1.954366in}{2.912580in}}%
\pgfusepath{stroke}%
\end{pgfscope}%
\begin{pgfscope}%
\pgfpathrectangle{\pgfqpoint{2.053095in}{0.319877in}}{\pgfqpoint{0.129635in}{2.592703in}} %
\pgfusepath{clip}%
\pgfsetbuttcap%
\pgfsetmiterjoin%
\definecolor{currentfill}{rgb}{1.000000,1.000000,1.000000}%
\pgfsetfillcolor{currentfill}%
\pgfsetlinewidth{0.010037pt}%
\definecolor{currentstroke}{rgb}{1.000000,1.000000,1.000000}%
\pgfsetstrokecolor{currentstroke}%
\pgfsetdash{}{0pt}%
\pgfpathmoveto{\pgfqpoint{2.053095in}{0.319877in}}%
\pgfpathlineto{\pgfqpoint{2.053095in}{0.330005in}}%
\pgfpathlineto{\pgfqpoint{2.053095in}{2.902452in}}%
\pgfpathlineto{\pgfqpoint{2.053095in}{2.912580in}}%
\pgfpathlineto{\pgfqpoint{2.182730in}{2.912580in}}%
\pgfpathlineto{\pgfqpoint{2.182730in}{2.902452in}}%
\pgfpathlineto{\pgfqpoint{2.182730in}{0.330005in}}%
\pgfpathlineto{\pgfqpoint{2.182730in}{0.319877in}}%
\pgfpathclose%
\pgfusepath{stroke,fill}%
\end{pgfscope}%
\begin{pgfscope}%
\pgfsys@transformshift{2.050000in}{0.320408in}%
\pgftext[left,bottom]{\pgfimage[interpolate=true,width=0.130000in,height=2.590000in]{PerrNN_vs_dq_Ti_100K-img1.png}}%
\end{pgfscope}%
\begin{pgfscope}%
\pgfsetbuttcap%
\pgfsetroundjoin%
\definecolor{currentfill}{rgb}{0.000000,0.000000,0.000000}%
\pgfsetfillcolor{currentfill}%
\pgfsetlinewidth{0.803000pt}%
\definecolor{currentstroke}{rgb}{0.000000,0.000000,0.000000}%
\pgfsetstrokecolor{currentstroke}%
\pgfsetdash{}{0pt}%
\pgfsys@defobject{currentmarker}{\pgfqpoint{0.000000in}{0.000000in}}{\pgfqpoint{0.048611in}{0.000000in}}{%
\pgfpathmoveto{\pgfqpoint{0.000000in}{0.000000in}}%
\pgfpathlineto{\pgfqpoint{0.048611in}{0.000000in}}%
\pgfusepath{stroke,fill}%
}%
\begin{pgfscope}%
\pgfsys@transformshift{2.182730in}{0.319877in}%
\pgfsys@useobject{currentmarker}{}%
\end{pgfscope}%
\end{pgfscope}%
\begin{pgfscope}%
\pgftext[x=2.279953in,y=0.272050in,left,base]{\rmfamily\fontsize{10.000000}{12.000000}\selectfont \(\displaystyle 0\)}%
\end{pgfscope}%
\begin{pgfscope}%
\pgfsetbuttcap%
\pgfsetroundjoin%
\definecolor{currentfill}{rgb}{0.000000,0.000000,0.000000}%
\pgfsetfillcolor{currentfill}%
\pgfsetlinewidth{0.803000pt}%
\definecolor{currentstroke}{rgb}{0.000000,0.000000,0.000000}%
\pgfsetstrokecolor{currentstroke}%
\pgfsetdash{}{0pt}%
\pgfsys@defobject{currentmarker}{\pgfqpoint{0.000000in}{0.000000in}}{\pgfqpoint{0.048611in}{0.000000in}}{%
\pgfpathmoveto{\pgfqpoint{0.000000in}{0.000000in}}%
\pgfpathlineto{\pgfqpoint{0.048611in}{0.000000in}}%
\pgfusepath{stroke,fill}%
}%
\begin{pgfscope}%
\pgfsys@transformshift{2.182730in}{0.607955in}%
\pgfsys@useobject{currentmarker}{}%
\end{pgfscope}%
\end{pgfscope}%
\begin{pgfscope}%
\pgftext[x=2.279953in,y=0.560128in,left,base]{\rmfamily\fontsize{10.000000}{12.000000}\selectfont \(\displaystyle 2\)}%
\end{pgfscope}%
\begin{pgfscope}%
\pgfsetbuttcap%
\pgfsetroundjoin%
\definecolor{currentfill}{rgb}{0.000000,0.000000,0.000000}%
\pgfsetfillcolor{currentfill}%
\pgfsetlinewidth{0.803000pt}%
\definecolor{currentstroke}{rgb}{0.000000,0.000000,0.000000}%
\pgfsetstrokecolor{currentstroke}%
\pgfsetdash{}{0pt}%
\pgfsys@defobject{currentmarker}{\pgfqpoint{0.000000in}{0.000000in}}{\pgfqpoint{0.048611in}{0.000000in}}{%
\pgfpathmoveto{\pgfqpoint{0.000000in}{0.000000in}}%
\pgfpathlineto{\pgfqpoint{0.048611in}{0.000000in}}%
\pgfusepath{stroke,fill}%
}%
\begin{pgfscope}%
\pgfsys@transformshift{2.182730in}{0.896034in}%
\pgfsys@useobject{currentmarker}{}%
\end{pgfscope}%
\end{pgfscope}%
\begin{pgfscope}%
\pgftext[x=2.279953in,y=0.848206in,left,base]{\rmfamily\fontsize{10.000000}{12.000000}\selectfont \(\displaystyle 4\)}%
\end{pgfscope}%
\begin{pgfscope}%
\pgfsetbuttcap%
\pgfsetroundjoin%
\definecolor{currentfill}{rgb}{0.000000,0.000000,0.000000}%
\pgfsetfillcolor{currentfill}%
\pgfsetlinewidth{0.803000pt}%
\definecolor{currentstroke}{rgb}{0.000000,0.000000,0.000000}%
\pgfsetstrokecolor{currentstroke}%
\pgfsetdash{}{0pt}%
\pgfsys@defobject{currentmarker}{\pgfqpoint{0.000000in}{0.000000in}}{\pgfqpoint{0.048611in}{0.000000in}}{%
\pgfpathmoveto{\pgfqpoint{0.000000in}{0.000000in}}%
\pgfpathlineto{\pgfqpoint{0.048611in}{0.000000in}}%
\pgfusepath{stroke,fill}%
}%
\begin{pgfscope}%
\pgfsys@transformshift{2.182730in}{1.184112in}%
\pgfsys@useobject{currentmarker}{}%
\end{pgfscope}%
\end{pgfscope}%
\begin{pgfscope}%
\pgftext[x=2.279953in,y=1.136284in,left,base]{\rmfamily\fontsize{10.000000}{12.000000}\selectfont \(\displaystyle 6\)}%
\end{pgfscope}%
\begin{pgfscope}%
\pgfsetbuttcap%
\pgfsetroundjoin%
\definecolor{currentfill}{rgb}{0.000000,0.000000,0.000000}%
\pgfsetfillcolor{currentfill}%
\pgfsetlinewidth{0.803000pt}%
\definecolor{currentstroke}{rgb}{0.000000,0.000000,0.000000}%
\pgfsetstrokecolor{currentstroke}%
\pgfsetdash{}{0pt}%
\pgfsys@defobject{currentmarker}{\pgfqpoint{0.000000in}{0.000000in}}{\pgfqpoint{0.048611in}{0.000000in}}{%
\pgfpathmoveto{\pgfqpoint{0.000000in}{0.000000in}}%
\pgfpathlineto{\pgfqpoint{0.048611in}{0.000000in}}%
\pgfusepath{stroke,fill}%
}%
\begin{pgfscope}%
\pgfsys@transformshift{2.182730in}{1.472190in}%
\pgfsys@useobject{currentmarker}{}%
\end{pgfscope}%
\end{pgfscope}%
\begin{pgfscope}%
\pgftext[x=2.279953in,y=1.424362in,left,base]{\rmfamily\fontsize{10.000000}{12.000000}\selectfont \(\displaystyle 8\)}%
\end{pgfscope}%
\begin{pgfscope}%
\pgfsetbuttcap%
\pgfsetroundjoin%
\definecolor{currentfill}{rgb}{0.000000,0.000000,0.000000}%
\pgfsetfillcolor{currentfill}%
\pgfsetlinewidth{0.803000pt}%
\definecolor{currentstroke}{rgb}{0.000000,0.000000,0.000000}%
\pgfsetstrokecolor{currentstroke}%
\pgfsetdash{}{0pt}%
\pgfsys@defobject{currentmarker}{\pgfqpoint{0.000000in}{0.000000in}}{\pgfqpoint{0.048611in}{0.000000in}}{%
\pgfpathmoveto{\pgfqpoint{0.000000in}{0.000000in}}%
\pgfpathlineto{\pgfqpoint{0.048611in}{0.000000in}}%
\pgfusepath{stroke,fill}%
}%
\begin{pgfscope}%
\pgfsys@transformshift{2.182730in}{1.760268in}%
\pgfsys@useobject{currentmarker}{}%
\end{pgfscope}%
\end{pgfscope}%
\begin{pgfscope}%
\pgftext[x=2.279953in,y=1.712440in,left,base]{\rmfamily\fontsize{10.000000}{12.000000}\selectfont \(\displaystyle 10\)}%
\end{pgfscope}%
\begin{pgfscope}%
\pgfsetbuttcap%
\pgfsetroundjoin%
\definecolor{currentfill}{rgb}{0.000000,0.000000,0.000000}%
\pgfsetfillcolor{currentfill}%
\pgfsetlinewidth{0.803000pt}%
\definecolor{currentstroke}{rgb}{0.000000,0.000000,0.000000}%
\pgfsetstrokecolor{currentstroke}%
\pgfsetdash{}{0pt}%
\pgfsys@defobject{currentmarker}{\pgfqpoint{0.000000in}{0.000000in}}{\pgfqpoint{0.048611in}{0.000000in}}{%
\pgfpathmoveto{\pgfqpoint{0.000000in}{0.000000in}}%
\pgfpathlineto{\pgfqpoint{0.048611in}{0.000000in}}%
\pgfusepath{stroke,fill}%
}%
\begin{pgfscope}%
\pgfsys@transformshift{2.182730in}{2.048346in}%
\pgfsys@useobject{currentmarker}{}%
\end{pgfscope}%
\end{pgfscope}%
\begin{pgfscope}%
\pgftext[x=2.279953in,y=2.000518in,left,base]{\rmfamily\fontsize{10.000000}{12.000000}\selectfont \(\displaystyle 12\)}%
\end{pgfscope}%
\begin{pgfscope}%
\pgfsetbuttcap%
\pgfsetroundjoin%
\definecolor{currentfill}{rgb}{0.000000,0.000000,0.000000}%
\pgfsetfillcolor{currentfill}%
\pgfsetlinewidth{0.803000pt}%
\definecolor{currentstroke}{rgb}{0.000000,0.000000,0.000000}%
\pgfsetstrokecolor{currentstroke}%
\pgfsetdash{}{0pt}%
\pgfsys@defobject{currentmarker}{\pgfqpoint{0.000000in}{0.000000in}}{\pgfqpoint{0.048611in}{0.000000in}}{%
\pgfpathmoveto{\pgfqpoint{0.000000in}{0.000000in}}%
\pgfpathlineto{\pgfqpoint{0.048611in}{0.000000in}}%
\pgfusepath{stroke,fill}%
}%
\begin{pgfscope}%
\pgfsys@transformshift{2.182730in}{2.336424in}%
\pgfsys@useobject{currentmarker}{}%
\end{pgfscope}%
\end{pgfscope}%
\begin{pgfscope}%
\pgftext[x=2.279953in,y=2.288596in,left,base]{\rmfamily\fontsize{10.000000}{12.000000}\selectfont \(\displaystyle 14\)}%
\end{pgfscope}%
\begin{pgfscope}%
\pgfsetbuttcap%
\pgfsetroundjoin%
\definecolor{currentfill}{rgb}{0.000000,0.000000,0.000000}%
\pgfsetfillcolor{currentfill}%
\pgfsetlinewidth{0.803000pt}%
\definecolor{currentstroke}{rgb}{0.000000,0.000000,0.000000}%
\pgfsetstrokecolor{currentstroke}%
\pgfsetdash{}{0pt}%
\pgfsys@defobject{currentmarker}{\pgfqpoint{0.000000in}{0.000000in}}{\pgfqpoint{0.048611in}{0.000000in}}{%
\pgfpathmoveto{\pgfqpoint{0.000000in}{0.000000in}}%
\pgfpathlineto{\pgfqpoint{0.048611in}{0.000000in}}%
\pgfusepath{stroke,fill}%
}%
\begin{pgfscope}%
\pgfsys@transformshift{2.182730in}{2.624502in}%
\pgfsys@useobject{currentmarker}{}%
\end{pgfscope}%
\end{pgfscope}%
\begin{pgfscope}%
\pgftext[x=2.279953in,y=2.576674in,left,base]{\rmfamily\fontsize{10.000000}{12.000000}\selectfont \(\displaystyle 16\)}%
\end{pgfscope}%
\begin{pgfscope}%
\pgfsetbuttcap%
\pgfsetroundjoin%
\definecolor{currentfill}{rgb}{0.000000,0.000000,0.000000}%
\pgfsetfillcolor{currentfill}%
\pgfsetlinewidth{0.803000pt}%
\definecolor{currentstroke}{rgb}{0.000000,0.000000,0.000000}%
\pgfsetstrokecolor{currentstroke}%
\pgfsetdash{}{0pt}%
\pgfsys@defobject{currentmarker}{\pgfqpoint{0.000000in}{0.000000in}}{\pgfqpoint{0.048611in}{0.000000in}}{%
\pgfpathmoveto{\pgfqpoint{0.000000in}{0.000000in}}%
\pgfpathlineto{\pgfqpoint{0.048611in}{0.000000in}}%
\pgfusepath{stroke,fill}%
}%
\begin{pgfscope}%
\pgfsys@transformshift{2.182730in}{2.912580in}%
\pgfsys@useobject{currentmarker}{}%
\end{pgfscope}%
\end{pgfscope}%
\begin{pgfscope}%
\pgftext[x=2.279953in,y=2.864752in,left,base]{\rmfamily\fontsize{10.000000}{12.000000}\selectfont \(\displaystyle 18\)}%
\end{pgfscope}%
\begin{pgfscope}%
\pgfsetbuttcap%
\pgfsetmiterjoin%
\pgfsetlinewidth{0.803000pt}%
\definecolor{currentstroke}{rgb}{0.000000,0.000000,0.000000}%
\pgfsetstrokecolor{currentstroke}%
\pgfsetdash{}{0pt}%
\pgfpathmoveto{\pgfqpoint{2.053095in}{0.319877in}}%
\pgfpathlineto{\pgfqpoint{2.053095in}{0.330005in}}%
\pgfpathlineto{\pgfqpoint{2.053095in}{2.902452in}}%
\pgfpathlineto{\pgfqpoint{2.053095in}{2.912580in}}%
\pgfpathlineto{\pgfqpoint{2.182730in}{2.912580in}}%
\pgfpathlineto{\pgfqpoint{2.182730in}{2.902452in}}%
\pgfpathlineto{\pgfqpoint{2.182730in}{0.330005in}}%
\pgfpathlineto{\pgfqpoint{2.182730in}{0.319877in}}%
\pgfpathclose%
\pgfusepath{stroke}%
\end{pgfscope}%
\end{pgfpicture}%
\makeatother%
\endgroup%

	\vspace*{-0.4cm}
	\caption{100 K. Bin size $0.0105e$}
	\end{subfigure}
	\hspace{0.6cm}
	\begin{subfigure}[b]{0.45\textwidth}
	\hspace*{-0.4cm}
	%% Creator: Matplotlib, PGF backend
%%
%% To include the figure in your LaTeX document, write
%%   \input{<filename>.pgf}
%%
%% Make sure the required packages are loaded in your preamble
%%   \usepackage{pgf}
%%
%% Figures using additional raster images can only be included by \input if
%% they are in the same directory as the main LaTeX file. For loading figures
%% from other directories you can use the `import` package
%%   \usepackage{import}
%% and then include the figures with
%%   \import{<path to file>}{<filename>.pgf}
%%
%% Matplotlib used the following preamble
%%   \usepackage[utf8x]{inputenc}
%%   \usepackage[T1]{fontenc}
%%
\begingroup%
\makeatletter%
\begin{pgfpicture}%
\pgfpathrectangle{\pgfpointorigin}{\pgfqpoint{2.518842in}{3.060408in}}%
\pgfusepath{use as bounding box, clip}%
\begin{pgfscope}%
\pgfsetbuttcap%
\pgfsetmiterjoin%
\definecolor{currentfill}{rgb}{1.000000,1.000000,1.000000}%
\pgfsetfillcolor{currentfill}%
\pgfsetlinewidth{0.000000pt}%
\definecolor{currentstroke}{rgb}{1.000000,1.000000,1.000000}%
\pgfsetstrokecolor{currentstroke}%
\pgfsetdash{}{0pt}%
\pgfpathmoveto{\pgfqpoint{0.000000in}{0.000000in}}%
\pgfpathlineto{\pgfqpoint{2.518842in}{0.000000in}}%
\pgfpathlineto{\pgfqpoint{2.518842in}{3.060408in}}%
\pgfpathlineto{\pgfqpoint{0.000000in}{3.060408in}}%
\pgfpathclose%
\pgfusepath{fill}%
\end{pgfscope}%
\begin{pgfscope}%
\pgfsetbuttcap%
\pgfsetmiterjoin%
\definecolor{currentfill}{rgb}{1.000000,1.000000,1.000000}%
\pgfsetfillcolor{currentfill}%
\pgfsetlinewidth{0.000000pt}%
\definecolor{currentstroke}{rgb}{0.000000,0.000000,0.000000}%
\pgfsetstrokecolor{currentstroke}%
\pgfsetstrokeopacity{0.000000}%
\pgfsetdash{}{0pt}%
\pgfpathmoveto{\pgfqpoint{0.374692in}{0.319877in}}%
\pgfpathlineto{\pgfqpoint{1.954366in}{0.319877in}}%
\pgfpathlineto{\pgfqpoint{1.954366in}{2.912580in}}%
\pgfpathlineto{\pgfqpoint{0.374692in}{2.912580in}}%
\pgfpathclose%
\pgfusepath{fill}%
\end{pgfscope}%
\begin{pgfscope}%
\pgfpathrectangle{\pgfqpoint{0.374692in}{0.319877in}}{\pgfqpoint{1.579674in}{2.592703in}} %
\pgfusepath{clip}%
\pgfsys@transformshift{0.374692in}{0.319877in}%
\pgftext[left,bottom]{\pgfimage[interpolate=true,width=1.580000in,height=2.600000in]{PerrNN_vs_dq_Ti_200K-img0.png}}%
\end{pgfscope}%
\begin{pgfscope}%
\pgfpathrectangle{\pgfqpoint{0.374692in}{0.319877in}}{\pgfqpoint{1.579674in}{2.592703in}} %
\pgfusepath{clip}%
\pgfsetbuttcap%
\pgfsetroundjoin%
\definecolor{currentfill}{rgb}{1.000000,0.752941,0.796078}%
\pgfsetfillcolor{currentfill}%
\pgfsetlinewidth{1.003750pt}%
\definecolor{currentstroke}{rgb}{1.000000,0.752941,0.796078}%
\pgfsetstrokecolor{currentstroke}%
\pgfsetdash{}{0pt}%
\pgfpathmoveto{\pgfqpoint{0.882444in}{1.440457in}}%
\pgfpathcurveto{\pgfqpoint{0.893494in}{1.440457in}}{\pgfqpoint{0.904093in}{1.444847in}}{\pgfqpoint{0.911907in}{1.452661in}}%
\pgfpathcurveto{\pgfqpoint{0.919721in}{1.460474in}}{\pgfqpoint{0.924111in}{1.471073in}}{\pgfqpoint{0.924111in}{1.482123in}}%
\pgfpathcurveto{\pgfqpoint{0.924111in}{1.493174in}}{\pgfqpoint{0.919721in}{1.503773in}}{\pgfqpoint{0.911907in}{1.511586in}}%
\pgfpathcurveto{\pgfqpoint{0.904093in}{1.519400in}}{\pgfqpoint{0.893494in}{1.523790in}}{\pgfqpoint{0.882444in}{1.523790in}}%
\pgfpathcurveto{\pgfqpoint{0.871394in}{1.523790in}}{\pgfqpoint{0.860795in}{1.519400in}}{\pgfqpoint{0.852981in}{1.511586in}}%
\pgfpathcurveto{\pgfqpoint{0.845168in}{1.503773in}}{\pgfqpoint{0.840778in}{1.493174in}}{\pgfqpoint{0.840778in}{1.482123in}}%
\pgfpathcurveto{\pgfqpoint{0.840778in}{1.471073in}}{\pgfqpoint{0.845168in}{1.460474in}}{\pgfqpoint{0.852981in}{1.452661in}}%
\pgfpathcurveto{\pgfqpoint{0.860795in}{1.444847in}}{\pgfqpoint{0.871394in}{1.440457in}}{\pgfqpoint{0.882444in}{1.440457in}}%
\pgfpathclose%
\pgfusepath{stroke,fill}%
\end{pgfscope}%
\begin{pgfscope}%
\pgfpathrectangle{\pgfqpoint{0.374692in}{0.319877in}}{\pgfqpoint{1.579674in}{2.592703in}} %
\pgfusepath{clip}%
\pgfsetbuttcap%
\pgfsetroundjoin%
\definecolor{currentfill}{rgb}{1.000000,0.752941,0.796078}%
\pgfsetfillcolor{currentfill}%
\pgfsetlinewidth{1.003750pt}%
\definecolor{currentstroke}{rgb}{1.000000,0.752941,0.796078}%
\pgfsetstrokecolor{currentstroke}%
\pgfsetdash{}{0pt}%
\pgfpathmoveto{\pgfqpoint{0.995278in}{1.402141in}}%
\pgfpathcurveto{\pgfqpoint{1.006328in}{1.402141in}}{\pgfqpoint{1.016927in}{1.406531in}}{\pgfqpoint{1.024741in}{1.414345in}}%
\pgfpathcurveto{\pgfqpoint{1.032554in}{1.422158in}}{\pgfqpoint{1.036945in}{1.432757in}}{\pgfqpoint{1.036945in}{1.443808in}}%
\pgfpathcurveto{\pgfqpoint{1.036945in}{1.454858in}}{\pgfqpoint{1.032554in}{1.465457in}}{\pgfqpoint{1.024741in}{1.473270in}}%
\pgfpathcurveto{\pgfqpoint{1.016927in}{1.481084in}}{\pgfqpoint{1.006328in}{1.485474in}}{\pgfqpoint{0.995278in}{1.485474in}}%
\pgfpathcurveto{\pgfqpoint{0.984228in}{1.485474in}}{\pgfqpoint{0.973629in}{1.481084in}}{\pgfqpoint{0.965815in}{1.473270in}}%
\pgfpathcurveto{\pgfqpoint{0.958002in}{1.465457in}}{\pgfqpoint{0.953611in}{1.454858in}}{\pgfqpoint{0.953611in}{1.443808in}}%
\pgfpathcurveto{\pgfqpoint{0.953611in}{1.432757in}}{\pgfqpoint{0.958002in}{1.422158in}}{\pgfqpoint{0.965815in}{1.414345in}}%
\pgfpathcurveto{\pgfqpoint{0.973629in}{1.406531in}}{\pgfqpoint{0.984228in}{1.402141in}}{\pgfqpoint{0.995278in}{1.402141in}}%
\pgfpathclose%
\pgfusepath{stroke,fill}%
\end{pgfscope}%
\begin{pgfscope}%
\pgfpathrectangle{\pgfqpoint{0.374692in}{0.319877in}}{\pgfqpoint{1.579674in}{2.592703in}} %
\pgfusepath{clip}%
\pgfsetbuttcap%
\pgfsetroundjoin%
\definecolor{currentfill}{rgb}{1.000000,0.752941,0.796078}%
\pgfsetfillcolor{currentfill}%
\pgfsetlinewidth{1.003750pt}%
\definecolor{currentstroke}{rgb}{1.000000,0.752941,0.796078}%
\pgfsetstrokecolor{currentstroke}%
\pgfsetdash{}{0pt}%
\pgfpathmoveto{\pgfqpoint{1.108112in}{1.441484in}}%
\pgfpathcurveto{\pgfqpoint{1.119162in}{1.441484in}}{\pgfqpoint{1.129761in}{1.445875in}}{\pgfqpoint{1.137575in}{1.453688in}}%
\pgfpathcurveto{\pgfqpoint{1.145388in}{1.461502in}}{\pgfqpoint{1.149779in}{1.472101in}}{\pgfqpoint{1.149779in}{1.483151in}}%
\pgfpathcurveto{\pgfqpoint{1.149779in}{1.494201in}}{\pgfqpoint{1.145388in}{1.504800in}}{\pgfqpoint{1.137575in}{1.512614in}}%
\pgfpathcurveto{\pgfqpoint{1.129761in}{1.520427in}}{\pgfqpoint{1.119162in}{1.524818in}}{\pgfqpoint{1.108112in}{1.524818in}}%
\pgfpathcurveto{\pgfqpoint{1.097062in}{1.524818in}}{\pgfqpoint{1.086463in}{1.520427in}}{\pgfqpoint{1.078649in}{1.512614in}}%
\pgfpathcurveto{\pgfqpoint{1.070836in}{1.504800in}}{\pgfqpoint{1.066445in}{1.494201in}}{\pgfqpoint{1.066445in}{1.483151in}}%
\pgfpathcurveto{\pgfqpoint{1.066445in}{1.472101in}}{\pgfqpoint{1.070836in}{1.461502in}}{\pgfqpoint{1.078649in}{1.453688in}}%
\pgfpathcurveto{\pgfqpoint{1.086463in}{1.445875in}}{\pgfqpoint{1.097062in}{1.441484in}}{\pgfqpoint{1.108112in}{1.441484in}}%
\pgfpathclose%
\pgfusepath{stroke,fill}%
\end{pgfscope}%
\begin{pgfscope}%
\pgfpathrectangle{\pgfqpoint{0.374692in}{0.319877in}}{\pgfqpoint{1.579674in}{2.592703in}} %
\pgfusepath{clip}%
\pgfsetbuttcap%
\pgfsetroundjoin%
\definecolor{currentfill}{rgb}{1.000000,0.752941,0.796078}%
\pgfsetfillcolor{currentfill}%
\pgfsetlinewidth{1.003750pt}%
\definecolor{currentstroke}{rgb}{1.000000,0.752941,0.796078}%
\pgfsetstrokecolor{currentstroke}%
\pgfsetdash{}{0pt}%
\pgfpathmoveto{\pgfqpoint{1.220946in}{1.472219in}}%
\pgfpathcurveto{\pgfqpoint{1.231996in}{1.472219in}}{\pgfqpoint{1.242595in}{1.476609in}}{\pgfqpoint{1.250409in}{1.484422in}}%
\pgfpathcurveto{\pgfqpoint{1.258222in}{1.492236in}}{\pgfqpoint{1.262612in}{1.502835in}}{\pgfqpoint{1.262612in}{1.513885in}}%
\pgfpathcurveto{\pgfqpoint{1.262612in}{1.524935in}}{\pgfqpoint{1.258222in}{1.535534in}}{\pgfqpoint{1.250409in}{1.543348in}}%
\pgfpathcurveto{\pgfqpoint{1.242595in}{1.551162in}}{\pgfqpoint{1.231996in}{1.555552in}}{\pgfqpoint{1.220946in}{1.555552in}}%
\pgfpathcurveto{\pgfqpoint{1.209896in}{1.555552in}}{\pgfqpoint{1.199297in}{1.551162in}}{\pgfqpoint{1.191483in}{1.543348in}}%
\pgfpathcurveto{\pgfqpoint{1.183669in}{1.535534in}}{\pgfqpoint{1.179279in}{1.524935in}}{\pgfqpoint{1.179279in}{1.513885in}}%
\pgfpathcurveto{\pgfqpoint{1.179279in}{1.502835in}}{\pgfqpoint{1.183669in}{1.492236in}}{\pgfqpoint{1.191483in}{1.484422in}}%
\pgfpathcurveto{\pgfqpoint{1.199297in}{1.476609in}}{\pgfqpoint{1.209896in}{1.472219in}}{\pgfqpoint{1.220946in}{1.472219in}}%
\pgfpathclose%
\pgfusepath{stroke,fill}%
\end{pgfscope}%
\begin{pgfscope}%
\pgfpathrectangle{\pgfqpoint{0.374692in}{0.319877in}}{\pgfqpoint{1.579674in}{2.592703in}} %
\pgfusepath{clip}%
\pgfsetbuttcap%
\pgfsetroundjoin%
\definecolor{currentfill}{rgb}{1.000000,0.752941,0.796078}%
\pgfsetfillcolor{currentfill}%
\pgfsetlinewidth{1.003750pt}%
\definecolor{currentstroke}{rgb}{1.000000,0.752941,0.796078}%
\pgfsetstrokecolor{currentstroke}%
\pgfsetdash{}{0pt}%
\pgfpathmoveto{\pgfqpoint{1.333780in}{1.499382in}}%
\pgfpathcurveto{\pgfqpoint{1.344830in}{1.499382in}}{\pgfqpoint{1.355429in}{1.503772in}}{\pgfqpoint{1.363242in}{1.511586in}}%
\pgfpathcurveto{\pgfqpoint{1.371056in}{1.519399in}}{\pgfqpoint{1.375446in}{1.529998in}}{\pgfqpoint{1.375446in}{1.541048in}}%
\pgfpathcurveto{\pgfqpoint{1.375446in}{1.552099in}}{\pgfqpoint{1.371056in}{1.562698in}}{\pgfqpoint{1.363242in}{1.570511in}}%
\pgfpathcurveto{\pgfqpoint{1.355429in}{1.578325in}}{\pgfqpoint{1.344830in}{1.582715in}}{\pgfqpoint{1.333780in}{1.582715in}}%
\pgfpathcurveto{\pgfqpoint{1.322729in}{1.582715in}}{\pgfqpoint{1.312130in}{1.578325in}}{\pgfqpoint{1.304317in}{1.570511in}}%
\pgfpathcurveto{\pgfqpoint{1.296503in}{1.562698in}}{\pgfqpoint{1.292113in}{1.552099in}}{\pgfqpoint{1.292113in}{1.541048in}}%
\pgfpathcurveto{\pgfqpoint{1.292113in}{1.529998in}}{\pgfqpoint{1.296503in}{1.519399in}}{\pgfqpoint{1.304317in}{1.511586in}}%
\pgfpathcurveto{\pgfqpoint{1.312130in}{1.503772in}}{\pgfqpoint{1.322729in}{1.499382in}}{\pgfqpoint{1.333780in}{1.499382in}}%
\pgfpathclose%
\pgfusepath{stroke,fill}%
\end{pgfscope}%
\begin{pgfscope}%
\pgfpathrectangle{\pgfqpoint{0.374692in}{0.319877in}}{\pgfqpoint{1.579674in}{2.592703in}} %
\pgfusepath{clip}%
\pgfsetbuttcap%
\pgfsetroundjoin%
\definecolor{currentfill}{rgb}{1.000000,0.752941,0.796078}%
\pgfsetfillcolor{currentfill}%
\pgfsetlinewidth{1.003750pt}%
\definecolor{currentstroke}{rgb}{1.000000,0.752941,0.796078}%
\pgfsetstrokecolor{currentstroke}%
\pgfsetdash{}{0pt}%
\pgfpathmoveto{\pgfqpoint{1.446613in}{1.559661in}}%
\pgfpathcurveto{\pgfqpoint{1.457664in}{1.559661in}}{\pgfqpoint{1.468263in}{1.564052in}}{\pgfqpoint{1.476076in}{1.571865in}}%
\pgfpathcurveto{\pgfqpoint{1.483890in}{1.579679in}}{\pgfqpoint{1.488280in}{1.590278in}}{\pgfqpoint{1.488280in}{1.601328in}}%
\pgfpathcurveto{\pgfqpoint{1.488280in}{1.612378in}}{\pgfqpoint{1.483890in}{1.622977in}}{\pgfqpoint{1.476076in}{1.630791in}}%
\pgfpathcurveto{\pgfqpoint{1.468263in}{1.638605in}}{\pgfqpoint{1.457664in}{1.642995in}}{\pgfqpoint{1.446613in}{1.642995in}}%
\pgfpathcurveto{\pgfqpoint{1.435563in}{1.642995in}}{\pgfqpoint{1.424964in}{1.638605in}}{\pgfqpoint{1.417151in}{1.630791in}}%
\pgfpathcurveto{\pgfqpoint{1.409337in}{1.622977in}}{\pgfqpoint{1.404947in}{1.612378in}}{\pgfqpoint{1.404947in}{1.601328in}}%
\pgfpathcurveto{\pgfqpoint{1.404947in}{1.590278in}}{\pgfqpoint{1.409337in}{1.579679in}}{\pgfqpoint{1.417151in}{1.571865in}}%
\pgfpathcurveto{\pgfqpoint{1.424964in}{1.564052in}}{\pgfqpoint{1.435563in}{1.559661in}}{\pgfqpoint{1.446613in}{1.559661in}}%
\pgfpathclose%
\pgfusepath{stroke,fill}%
\end{pgfscope}%
\begin{pgfscope}%
\pgfpathrectangle{\pgfqpoint{0.374692in}{0.319877in}}{\pgfqpoint{1.579674in}{2.592703in}} %
\pgfusepath{clip}%
\pgfsetbuttcap%
\pgfsetroundjoin%
\definecolor{currentfill}{rgb}{1.000000,0.752941,0.796078}%
\pgfsetfillcolor{currentfill}%
\pgfsetlinewidth{1.003750pt}%
\definecolor{currentstroke}{rgb}{1.000000,0.752941,0.796078}%
\pgfsetstrokecolor{currentstroke}%
\pgfsetdash{}{0pt}%
\pgfpathmoveto{\pgfqpoint{1.559447in}{1.976878in}}%
\pgfpathcurveto{\pgfqpoint{1.570497in}{1.976878in}}{\pgfqpoint{1.581096in}{1.981268in}}{\pgfqpoint{1.588910in}{1.989082in}}%
\pgfpathcurveto{\pgfqpoint{1.596724in}{1.996896in}}{\pgfqpoint{1.601114in}{2.007495in}}{\pgfqpoint{1.601114in}{2.018545in}}%
\pgfpathcurveto{\pgfqpoint{1.601114in}{2.029595in}}{\pgfqpoint{1.596724in}{2.040194in}}{\pgfqpoint{1.588910in}{2.048007in}}%
\pgfpathcurveto{\pgfqpoint{1.581096in}{2.055821in}}{\pgfqpoint{1.570497in}{2.060211in}}{\pgfqpoint{1.559447in}{2.060211in}}%
\pgfpathcurveto{\pgfqpoint{1.548397in}{2.060211in}}{\pgfqpoint{1.537798in}{2.055821in}}{\pgfqpoint{1.529985in}{2.048007in}}%
\pgfpathcurveto{\pgfqpoint{1.522171in}{2.040194in}}{\pgfqpoint{1.517781in}{2.029595in}}{\pgfqpoint{1.517781in}{2.018545in}}%
\pgfpathcurveto{\pgfqpoint{1.517781in}{2.007495in}}{\pgfqpoint{1.522171in}{1.996896in}}{\pgfqpoint{1.529985in}{1.989082in}}%
\pgfpathcurveto{\pgfqpoint{1.537798in}{1.981268in}}{\pgfqpoint{1.548397in}{1.976878in}}{\pgfqpoint{1.559447in}{1.976878in}}%
\pgfpathclose%
\pgfusepath{stroke,fill}%
\end{pgfscope}%
\begin{pgfscope}%
\pgfsetbuttcap%
\pgfsetroundjoin%
\definecolor{currentfill}{rgb}{0.000000,0.000000,0.000000}%
\pgfsetfillcolor{currentfill}%
\pgfsetlinewidth{0.803000pt}%
\definecolor{currentstroke}{rgb}{0.000000,0.000000,0.000000}%
\pgfsetstrokecolor{currentstroke}%
\pgfsetdash{}{0pt}%
\pgfsys@defobject{currentmarker}{\pgfqpoint{0.000000in}{-0.048611in}}{\pgfqpoint{0.000000in}{0.000000in}}{%
\pgfpathmoveto{\pgfqpoint{0.000000in}{0.000000in}}%
\pgfpathlineto{\pgfqpoint{0.000000in}{-0.048611in}}%
\pgfusepath{stroke,fill}%
}%
\begin{pgfscope}%
\pgfsys@transformshift{0.670881in}{0.319877in}%
\pgfsys@useobject{currentmarker}{}%
\end{pgfscope}%
\end{pgfscope}%
\begin{pgfscope}%
\pgftext[x=0.670881in,y=0.222655in,,top]{\rmfamily\fontsize{10.000000}{12.000000}\selectfont \(\displaystyle -0.05\)}%
\end{pgfscope}%
\begin{pgfscope}%
\pgfsetbuttcap%
\pgfsetroundjoin%
\definecolor{currentfill}{rgb}{0.000000,0.000000,0.000000}%
\pgfsetfillcolor{currentfill}%
\pgfsetlinewidth{0.803000pt}%
\definecolor{currentstroke}{rgb}{0.000000,0.000000,0.000000}%
\pgfsetstrokecolor{currentstroke}%
\pgfsetdash{}{0pt}%
\pgfsys@defobject{currentmarker}{\pgfqpoint{0.000000in}{-0.048611in}}{\pgfqpoint{0.000000in}{0.000000in}}{%
\pgfpathmoveto{\pgfqpoint{0.000000in}{0.000000in}}%
\pgfpathlineto{\pgfqpoint{0.000000in}{-0.048611in}}%
\pgfusepath{stroke,fill}%
}%
\begin{pgfscope}%
\pgfsys@transformshift{1.164529in}{0.319877in}%
\pgfsys@useobject{currentmarker}{}%
\end{pgfscope}%
\end{pgfscope}%
\begin{pgfscope}%
\pgftext[x=1.164529in,y=0.222655in,,top]{\rmfamily\fontsize{10.000000}{12.000000}\selectfont \(\displaystyle 0.00\)}%
\end{pgfscope}%
\begin{pgfscope}%
\pgfsetbuttcap%
\pgfsetroundjoin%
\definecolor{currentfill}{rgb}{0.000000,0.000000,0.000000}%
\pgfsetfillcolor{currentfill}%
\pgfsetlinewidth{0.803000pt}%
\definecolor{currentstroke}{rgb}{0.000000,0.000000,0.000000}%
\pgfsetstrokecolor{currentstroke}%
\pgfsetdash{}{0pt}%
\pgfsys@defobject{currentmarker}{\pgfqpoint{0.000000in}{-0.048611in}}{\pgfqpoint{0.000000in}{0.000000in}}{%
\pgfpathmoveto{\pgfqpoint{0.000000in}{0.000000in}}%
\pgfpathlineto{\pgfqpoint{0.000000in}{-0.048611in}}%
\pgfusepath{stroke,fill}%
}%
\begin{pgfscope}%
\pgfsys@transformshift{1.658177in}{0.319877in}%
\pgfsys@useobject{currentmarker}{}%
\end{pgfscope}%
\end{pgfscope}%
\begin{pgfscope}%
\pgftext[x=1.658177in,y=0.222655in,,top]{\rmfamily\fontsize{10.000000}{12.000000}\selectfont \(\displaystyle 0.05\)}%
\end{pgfscope}%
\begin{pgfscope}%
\pgfsetbuttcap%
\pgfsetroundjoin%
\definecolor{currentfill}{rgb}{0.000000,0.000000,0.000000}%
\pgfsetfillcolor{currentfill}%
\pgfsetlinewidth{0.803000pt}%
\definecolor{currentstroke}{rgb}{0.000000,0.000000,0.000000}%
\pgfsetstrokecolor{currentstroke}%
\pgfsetdash{}{0pt}%
\pgfsys@defobject{currentmarker}{\pgfqpoint{-0.048611in}{0.000000in}}{\pgfqpoint{0.000000in}{0.000000in}}{%
\pgfpathmoveto{\pgfqpoint{0.000000in}{0.000000in}}%
\pgfpathlineto{\pgfqpoint{-0.048611in}{0.000000in}}%
\pgfusepath{stroke,fill}%
}%
\begin{pgfscope}%
\pgfsys@transformshift{0.374692in}{0.319877in}%
\pgfsys@useobject{currentmarker}{}%
\end{pgfscope}%
\end{pgfscope}%
\begin{pgfscope}%
\pgftext[x=0.100000in,y=0.272050in,left,base]{\rmfamily\fontsize{10.000000}{12.000000}\selectfont \(\displaystyle 0.0\)}%
\end{pgfscope}%
\begin{pgfscope}%
\pgfsetbuttcap%
\pgfsetroundjoin%
\definecolor{currentfill}{rgb}{0.000000,0.000000,0.000000}%
\pgfsetfillcolor{currentfill}%
\pgfsetlinewidth{0.803000pt}%
\definecolor{currentstroke}{rgb}{0.000000,0.000000,0.000000}%
\pgfsetstrokecolor{currentstroke}%
\pgfsetdash{}{0pt}%
\pgfsys@defobject{currentmarker}{\pgfqpoint{-0.048611in}{0.000000in}}{\pgfqpoint{0.000000in}{0.000000in}}{%
\pgfpathmoveto{\pgfqpoint{0.000000in}{0.000000in}}%
\pgfpathlineto{\pgfqpoint{-0.048611in}{0.000000in}}%
\pgfusepath{stroke,fill}%
}%
\begin{pgfscope}%
\pgfsys@transformshift{0.374692in}{0.838418in}%
\pgfsys@useobject{currentmarker}{}%
\end{pgfscope}%
\end{pgfscope}%
\begin{pgfscope}%
\pgftext[x=0.100000in,y=0.790590in,left,base]{\rmfamily\fontsize{10.000000}{12.000000}\selectfont \(\displaystyle 0.1\)}%
\end{pgfscope}%
\begin{pgfscope}%
\pgfsetbuttcap%
\pgfsetroundjoin%
\definecolor{currentfill}{rgb}{0.000000,0.000000,0.000000}%
\pgfsetfillcolor{currentfill}%
\pgfsetlinewidth{0.803000pt}%
\definecolor{currentstroke}{rgb}{0.000000,0.000000,0.000000}%
\pgfsetstrokecolor{currentstroke}%
\pgfsetdash{}{0pt}%
\pgfsys@defobject{currentmarker}{\pgfqpoint{-0.048611in}{0.000000in}}{\pgfqpoint{0.000000in}{0.000000in}}{%
\pgfpathmoveto{\pgfqpoint{0.000000in}{0.000000in}}%
\pgfpathlineto{\pgfqpoint{-0.048611in}{0.000000in}}%
\pgfusepath{stroke,fill}%
}%
\begin{pgfscope}%
\pgfsys@transformshift{0.374692in}{1.356958in}%
\pgfsys@useobject{currentmarker}{}%
\end{pgfscope}%
\end{pgfscope}%
\begin{pgfscope}%
\pgftext[x=0.100000in,y=1.309131in,left,base]{\rmfamily\fontsize{10.000000}{12.000000}\selectfont \(\displaystyle 0.2\)}%
\end{pgfscope}%
\begin{pgfscope}%
\pgfsetbuttcap%
\pgfsetroundjoin%
\definecolor{currentfill}{rgb}{0.000000,0.000000,0.000000}%
\pgfsetfillcolor{currentfill}%
\pgfsetlinewidth{0.803000pt}%
\definecolor{currentstroke}{rgb}{0.000000,0.000000,0.000000}%
\pgfsetstrokecolor{currentstroke}%
\pgfsetdash{}{0pt}%
\pgfsys@defobject{currentmarker}{\pgfqpoint{-0.048611in}{0.000000in}}{\pgfqpoint{0.000000in}{0.000000in}}{%
\pgfpathmoveto{\pgfqpoint{0.000000in}{0.000000in}}%
\pgfpathlineto{\pgfqpoint{-0.048611in}{0.000000in}}%
\pgfusepath{stroke,fill}%
}%
\begin{pgfscope}%
\pgfsys@transformshift{0.374692in}{1.875499in}%
\pgfsys@useobject{currentmarker}{}%
\end{pgfscope}%
\end{pgfscope}%
\begin{pgfscope}%
\pgftext[x=0.100000in,y=1.827671in,left,base]{\rmfamily\fontsize{10.000000}{12.000000}\selectfont \(\displaystyle 0.3\)}%
\end{pgfscope}%
\begin{pgfscope}%
\pgfsetbuttcap%
\pgfsetroundjoin%
\definecolor{currentfill}{rgb}{0.000000,0.000000,0.000000}%
\pgfsetfillcolor{currentfill}%
\pgfsetlinewidth{0.803000pt}%
\definecolor{currentstroke}{rgb}{0.000000,0.000000,0.000000}%
\pgfsetstrokecolor{currentstroke}%
\pgfsetdash{}{0pt}%
\pgfsys@defobject{currentmarker}{\pgfqpoint{-0.048611in}{0.000000in}}{\pgfqpoint{0.000000in}{0.000000in}}{%
\pgfpathmoveto{\pgfqpoint{0.000000in}{0.000000in}}%
\pgfpathlineto{\pgfqpoint{-0.048611in}{0.000000in}}%
\pgfusepath{stroke,fill}%
}%
\begin{pgfscope}%
\pgfsys@transformshift{0.374692in}{2.394040in}%
\pgfsys@useobject{currentmarker}{}%
\end{pgfscope}%
\end{pgfscope}%
\begin{pgfscope}%
\pgftext[x=0.100000in,y=2.346212in,left,base]{\rmfamily\fontsize{10.000000}{12.000000}\selectfont \(\displaystyle 0.4\)}%
\end{pgfscope}%
\begin{pgfscope}%
\pgfsetbuttcap%
\pgfsetroundjoin%
\definecolor{currentfill}{rgb}{0.000000,0.000000,0.000000}%
\pgfsetfillcolor{currentfill}%
\pgfsetlinewidth{0.803000pt}%
\definecolor{currentstroke}{rgb}{0.000000,0.000000,0.000000}%
\pgfsetstrokecolor{currentstroke}%
\pgfsetdash{}{0pt}%
\pgfsys@defobject{currentmarker}{\pgfqpoint{-0.048611in}{0.000000in}}{\pgfqpoint{0.000000in}{0.000000in}}{%
\pgfpathmoveto{\pgfqpoint{0.000000in}{0.000000in}}%
\pgfpathlineto{\pgfqpoint{-0.048611in}{0.000000in}}%
\pgfusepath{stroke,fill}%
}%
\begin{pgfscope}%
\pgfsys@transformshift{0.374692in}{2.912580in}%
\pgfsys@useobject{currentmarker}{}%
\end{pgfscope}%
\end{pgfscope}%
\begin{pgfscope}%
\pgftext[x=0.100000in,y=2.864752in,left,base]{\rmfamily\fontsize{10.000000}{12.000000}\selectfont \(\displaystyle 0.5\)}%
\end{pgfscope}%
\begin{pgfscope}%
\pgfsetrectcap%
\pgfsetmiterjoin%
\pgfsetlinewidth{0.803000pt}%
\definecolor{currentstroke}{rgb}{0.000000,0.000000,0.000000}%
\pgfsetstrokecolor{currentstroke}%
\pgfsetdash{}{0pt}%
\pgfpathmoveto{\pgfqpoint{0.374692in}{0.319877in}}%
\pgfpathlineto{\pgfqpoint{0.374692in}{2.912580in}}%
\pgfusepath{stroke}%
\end{pgfscope}%
\begin{pgfscope}%
\pgfsetrectcap%
\pgfsetmiterjoin%
\pgfsetlinewidth{0.803000pt}%
\definecolor{currentstroke}{rgb}{0.000000,0.000000,0.000000}%
\pgfsetstrokecolor{currentstroke}%
\pgfsetdash{}{0pt}%
\pgfpathmoveto{\pgfqpoint{1.954366in}{0.319877in}}%
\pgfpathlineto{\pgfqpoint{1.954366in}{2.912580in}}%
\pgfusepath{stroke}%
\end{pgfscope}%
\begin{pgfscope}%
\pgfsetrectcap%
\pgfsetmiterjoin%
\pgfsetlinewidth{0.803000pt}%
\definecolor{currentstroke}{rgb}{0.000000,0.000000,0.000000}%
\pgfsetstrokecolor{currentstroke}%
\pgfsetdash{}{0pt}%
\pgfpathmoveto{\pgfqpoint{0.374692in}{0.319877in}}%
\pgfpathlineto{\pgfqpoint{1.954366in}{0.319877in}}%
\pgfusepath{stroke}%
\end{pgfscope}%
\begin{pgfscope}%
\pgfsetrectcap%
\pgfsetmiterjoin%
\pgfsetlinewidth{0.803000pt}%
\definecolor{currentstroke}{rgb}{0.000000,0.000000,0.000000}%
\pgfsetstrokecolor{currentstroke}%
\pgfsetdash{}{0pt}%
\pgfpathmoveto{\pgfqpoint{0.374692in}{2.912580in}}%
\pgfpathlineto{\pgfqpoint{1.954366in}{2.912580in}}%
\pgfusepath{stroke}%
\end{pgfscope}%
\begin{pgfscope}%
\pgfpathrectangle{\pgfqpoint{2.053095in}{0.319877in}}{\pgfqpoint{0.129635in}{2.592703in}} %
\pgfusepath{clip}%
\pgfsetbuttcap%
\pgfsetmiterjoin%
\definecolor{currentfill}{rgb}{1.000000,1.000000,1.000000}%
\pgfsetfillcolor{currentfill}%
\pgfsetlinewidth{0.010037pt}%
\definecolor{currentstroke}{rgb}{1.000000,1.000000,1.000000}%
\pgfsetstrokecolor{currentstroke}%
\pgfsetdash{}{0pt}%
\pgfpathmoveto{\pgfqpoint{2.053095in}{0.319877in}}%
\pgfpathlineto{\pgfqpoint{2.053095in}{0.330005in}}%
\pgfpathlineto{\pgfqpoint{2.053095in}{2.902452in}}%
\pgfpathlineto{\pgfqpoint{2.053095in}{2.912580in}}%
\pgfpathlineto{\pgfqpoint{2.182730in}{2.912580in}}%
\pgfpathlineto{\pgfqpoint{2.182730in}{2.902452in}}%
\pgfpathlineto{\pgfqpoint{2.182730in}{0.330005in}}%
\pgfpathlineto{\pgfqpoint{2.182730in}{0.319877in}}%
\pgfpathclose%
\pgfusepath{stroke,fill}%
\end{pgfscope}%
\begin{pgfscope}%
\pgfsys@transformshift{2.050000in}{0.320408in}%
\pgftext[left,bottom]{\pgfimage[interpolate=true,width=0.130000in,height=2.590000in]{PerrNN_vs_dq_Ti_200K-img1.png}}%
\end{pgfscope}%
\begin{pgfscope}%
\pgfsetbuttcap%
\pgfsetroundjoin%
\definecolor{currentfill}{rgb}{0.000000,0.000000,0.000000}%
\pgfsetfillcolor{currentfill}%
\pgfsetlinewidth{0.803000pt}%
\definecolor{currentstroke}{rgb}{0.000000,0.000000,0.000000}%
\pgfsetstrokecolor{currentstroke}%
\pgfsetdash{}{0pt}%
\pgfsys@defobject{currentmarker}{\pgfqpoint{0.000000in}{0.000000in}}{\pgfqpoint{0.048611in}{0.000000in}}{%
\pgfpathmoveto{\pgfqpoint{0.000000in}{0.000000in}}%
\pgfpathlineto{\pgfqpoint{0.048611in}{0.000000in}}%
\pgfusepath{stroke,fill}%
}%
\begin{pgfscope}%
\pgfsys@transformshift{2.182730in}{0.319877in}%
\pgfsys@useobject{currentmarker}{}%
\end{pgfscope}%
\end{pgfscope}%
\begin{pgfscope}%
\pgftext[x=2.279953in,y=0.272050in,left,base]{\rmfamily\fontsize{10.000000}{12.000000}\selectfont \(\displaystyle 0\)}%
\end{pgfscope}%
\begin{pgfscope}%
\pgfsetbuttcap%
\pgfsetroundjoin%
\definecolor{currentfill}{rgb}{0.000000,0.000000,0.000000}%
\pgfsetfillcolor{currentfill}%
\pgfsetlinewidth{0.803000pt}%
\definecolor{currentstroke}{rgb}{0.000000,0.000000,0.000000}%
\pgfsetstrokecolor{currentstroke}%
\pgfsetdash{}{0pt}%
\pgfsys@defobject{currentmarker}{\pgfqpoint{0.000000in}{0.000000in}}{\pgfqpoint{0.048611in}{0.000000in}}{%
\pgfpathmoveto{\pgfqpoint{0.000000in}{0.000000in}}%
\pgfpathlineto{\pgfqpoint{0.048611in}{0.000000in}}%
\pgfusepath{stroke,fill}%
}%
\begin{pgfscope}%
\pgfsys@transformshift{2.182730in}{0.607955in}%
\pgfsys@useobject{currentmarker}{}%
\end{pgfscope}%
\end{pgfscope}%
\begin{pgfscope}%
\pgftext[x=2.279953in,y=0.560128in,left,base]{\rmfamily\fontsize{10.000000}{12.000000}\selectfont \(\displaystyle 2\)}%
\end{pgfscope}%
\begin{pgfscope}%
\pgfsetbuttcap%
\pgfsetroundjoin%
\definecolor{currentfill}{rgb}{0.000000,0.000000,0.000000}%
\pgfsetfillcolor{currentfill}%
\pgfsetlinewidth{0.803000pt}%
\definecolor{currentstroke}{rgb}{0.000000,0.000000,0.000000}%
\pgfsetstrokecolor{currentstroke}%
\pgfsetdash{}{0pt}%
\pgfsys@defobject{currentmarker}{\pgfqpoint{0.000000in}{0.000000in}}{\pgfqpoint{0.048611in}{0.000000in}}{%
\pgfpathmoveto{\pgfqpoint{0.000000in}{0.000000in}}%
\pgfpathlineto{\pgfqpoint{0.048611in}{0.000000in}}%
\pgfusepath{stroke,fill}%
}%
\begin{pgfscope}%
\pgfsys@transformshift{2.182730in}{0.896034in}%
\pgfsys@useobject{currentmarker}{}%
\end{pgfscope}%
\end{pgfscope}%
\begin{pgfscope}%
\pgftext[x=2.279953in,y=0.848206in,left,base]{\rmfamily\fontsize{10.000000}{12.000000}\selectfont \(\displaystyle 4\)}%
\end{pgfscope}%
\begin{pgfscope}%
\pgfsetbuttcap%
\pgfsetroundjoin%
\definecolor{currentfill}{rgb}{0.000000,0.000000,0.000000}%
\pgfsetfillcolor{currentfill}%
\pgfsetlinewidth{0.803000pt}%
\definecolor{currentstroke}{rgb}{0.000000,0.000000,0.000000}%
\pgfsetstrokecolor{currentstroke}%
\pgfsetdash{}{0pt}%
\pgfsys@defobject{currentmarker}{\pgfqpoint{0.000000in}{0.000000in}}{\pgfqpoint{0.048611in}{0.000000in}}{%
\pgfpathmoveto{\pgfqpoint{0.000000in}{0.000000in}}%
\pgfpathlineto{\pgfqpoint{0.048611in}{0.000000in}}%
\pgfusepath{stroke,fill}%
}%
\begin{pgfscope}%
\pgfsys@transformshift{2.182730in}{1.184112in}%
\pgfsys@useobject{currentmarker}{}%
\end{pgfscope}%
\end{pgfscope}%
\begin{pgfscope}%
\pgftext[x=2.279953in,y=1.136284in,left,base]{\rmfamily\fontsize{10.000000}{12.000000}\selectfont \(\displaystyle 6\)}%
\end{pgfscope}%
\begin{pgfscope}%
\pgfsetbuttcap%
\pgfsetroundjoin%
\definecolor{currentfill}{rgb}{0.000000,0.000000,0.000000}%
\pgfsetfillcolor{currentfill}%
\pgfsetlinewidth{0.803000pt}%
\definecolor{currentstroke}{rgb}{0.000000,0.000000,0.000000}%
\pgfsetstrokecolor{currentstroke}%
\pgfsetdash{}{0pt}%
\pgfsys@defobject{currentmarker}{\pgfqpoint{0.000000in}{0.000000in}}{\pgfqpoint{0.048611in}{0.000000in}}{%
\pgfpathmoveto{\pgfqpoint{0.000000in}{0.000000in}}%
\pgfpathlineto{\pgfqpoint{0.048611in}{0.000000in}}%
\pgfusepath{stroke,fill}%
}%
\begin{pgfscope}%
\pgfsys@transformshift{2.182730in}{1.472190in}%
\pgfsys@useobject{currentmarker}{}%
\end{pgfscope}%
\end{pgfscope}%
\begin{pgfscope}%
\pgftext[x=2.279953in,y=1.424362in,left,base]{\rmfamily\fontsize{10.000000}{12.000000}\selectfont \(\displaystyle 8\)}%
\end{pgfscope}%
\begin{pgfscope}%
\pgfsetbuttcap%
\pgfsetroundjoin%
\definecolor{currentfill}{rgb}{0.000000,0.000000,0.000000}%
\pgfsetfillcolor{currentfill}%
\pgfsetlinewidth{0.803000pt}%
\definecolor{currentstroke}{rgb}{0.000000,0.000000,0.000000}%
\pgfsetstrokecolor{currentstroke}%
\pgfsetdash{}{0pt}%
\pgfsys@defobject{currentmarker}{\pgfqpoint{0.000000in}{0.000000in}}{\pgfqpoint{0.048611in}{0.000000in}}{%
\pgfpathmoveto{\pgfqpoint{0.000000in}{0.000000in}}%
\pgfpathlineto{\pgfqpoint{0.048611in}{0.000000in}}%
\pgfusepath{stroke,fill}%
}%
\begin{pgfscope}%
\pgfsys@transformshift{2.182730in}{1.760268in}%
\pgfsys@useobject{currentmarker}{}%
\end{pgfscope}%
\end{pgfscope}%
\begin{pgfscope}%
\pgftext[x=2.279953in,y=1.712440in,left,base]{\rmfamily\fontsize{10.000000}{12.000000}\selectfont \(\displaystyle 10\)}%
\end{pgfscope}%
\begin{pgfscope}%
\pgfsetbuttcap%
\pgfsetroundjoin%
\definecolor{currentfill}{rgb}{0.000000,0.000000,0.000000}%
\pgfsetfillcolor{currentfill}%
\pgfsetlinewidth{0.803000pt}%
\definecolor{currentstroke}{rgb}{0.000000,0.000000,0.000000}%
\pgfsetstrokecolor{currentstroke}%
\pgfsetdash{}{0pt}%
\pgfsys@defobject{currentmarker}{\pgfqpoint{0.000000in}{0.000000in}}{\pgfqpoint{0.048611in}{0.000000in}}{%
\pgfpathmoveto{\pgfqpoint{0.000000in}{0.000000in}}%
\pgfpathlineto{\pgfqpoint{0.048611in}{0.000000in}}%
\pgfusepath{stroke,fill}%
}%
\begin{pgfscope}%
\pgfsys@transformshift{2.182730in}{2.048346in}%
\pgfsys@useobject{currentmarker}{}%
\end{pgfscope}%
\end{pgfscope}%
\begin{pgfscope}%
\pgftext[x=2.279953in,y=2.000518in,left,base]{\rmfamily\fontsize{10.000000}{12.000000}\selectfont \(\displaystyle 12\)}%
\end{pgfscope}%
\begin{pgfscope}%
\pgfsetbuttcap%
\pgfsetroundjoin%
\definecolor{currentfill}{rgb}{0.000000,0.000000,0.000000}%
\pgfsetfillcolor{currentfill}%
\pgfsetlinewidth{0.803000pt}%
\definecolor{currentstroke}{rgb}{0.000000,0.000000,0.000000}%
\pgfsetstrokecolor{currentstroke}%
\pgfsetdash{}{0pt}%
\pgfsys@defobject{currentmarker}{\pgfqpoint{0.000000in}{0.000000in}}{\pgfqpoint{0.048611in}{0.000000in}}{%
\pgfpathmoveto{\pgfqpoint{0.000000in}{0.000000in}}%
\pgfpathlineto{\pgfqpoint{0.048611in}{0.000000in}}%
\pgfusepath{stroke,fill}%
}%
\begin{pgfscope}%
\pgfsys@transformshift{2.182730in}{2.336424in}%
\pgfsys@useobject{currentmarker}{}%
\end{pgfscope}%
\end{pgfscope}%
\begin{pgfscope}%
\pgftext[x=2.279953in,y=2.288596in,left,base]{\rmfamily\fontsize{10.000000}{12.000000}\selectfont \(\displaystyle 14\)}%
\end{pgfscope}%
\begin{pgfscope}%
\pgfsetbuttcap%
\pgfsetroundjoin%
\definecolor{currentfill}{rgb}{0.000000,0.000000,0.000000}%
\pgfsetfillcolor{currentfill}%
\pgfsetlinewidth{0.803000pt}%
\definecolor{currentstroke}{rgb}{0.000000,0.000000,0.000000}%
\pgfsetstrokecolor{currentstroke}%
\pgfsetdash{}{0pt}%
\pgfsys@defobject{currentmarker}{\pgfqpoint{0.000000in}{0.000000in}}{\pgfqpoint{0.048611in}{0.000000in}}{%
\pgfpathmoveto{\pgfqpoint{0.000000in}{0.000000in}}%
\pgfpathlineto{\pgfqpoint{0.048611in}{0.000000in}}%
\pgfusepath{stroke,fill}%
}%
\begin{pgfscope}%
\pgfsys@transformshift{2.182730in}{2.624502in}%
\pgfsys@useobject{currentmarker}{}%
\end{pgfscope}%
\end{pgfscope}%
\begin{pgfscope}%
\pgftext[x=2.279953in,y=2.576674in,left,base]{\rmfamily\fontsize{10.000000}{12.000000}\selectfont \(\displaystyle 16\)}%
\end{pgfscope}%
\begin{pgfscope}%
\pgfsetbuttcap%
\pgfsetroundjoin%
\definecolor{currentfill}{rgb}{0.000000,0.000000,0.000000}%
\pgfsetfillcolor{currentfill}%
\pgfsetlinewidth{0.803000pt}%
\definecolor{currentstroke}{rgb}{0.000000,0.000000,0.000000}%
\pgfsetstrokecolor{currentstroke}%
\pgfsetdash{}{0pt}%
\pgfsys@defobject{currentmarker}{\pgfqpoint{0.000000in}{0.000000in}}{\pgfqpoint{0.048611in}{0.000000in}}{%
\pgfpathmoveto{\pgfqpoint{0.000000in}{0.000000in}}%
\pgfpathlineto{\pgfqpoint{0.048611in}{0.000000in}}%
\pgfusepath{stroke,fill}%
}%
\begin{pgfscope}%
\pgfsys@transformshift{2.182730in}{2.912580in}%
\pgfsys@useobject{currentmarker}{}%
\end{pgfscope}%
\end{pgfscope}%
\begin{pgfscope}%
\pgftext[x=2.279953in,y=2.864752in,left,base]{\rmfamily\fontsize{10.000000}{12.000000}\selectfont \(\displaystyle 18\)}%
\end{pgfscope}%
\begin{pgfscope}%
\pgfsetbuttcap%
\pgfsetmiterjoin%
\pgfsetlinewidth{0.803000pt}%
\definecolor{currentstroke}{rgb}{0.000000,0.000000,0.000000}%
\pgfsetstrokecolor{currentstroke}%
\pgfsetdash{}{0pt}%
\pgfpathmoveto{\pgfqpoint{2.053095in}{0.319877in}}%
\pgfpathlineto{\pgfqpoint{2.053095in}{0.330005in}}%
\pgfpathlineto{\pgfqpoint{2.053095in}{2.902452in}}%
\pgfpathlineto{\pgfqpoint{2.053095in}{2.912580in}}%
\pgfpathlineto{\pgfqpoint{2.182730in}{2.912580in}}%
\pgfpathlineto{\pgfqpoint{2.182730in}{2.902452in}}%
\pgfpathlineto{\pgfqpoint{2.182730in}{0.330005in}}%
\pgfpathlineto{\pgfqpoint{2.182730in}{0.319877in}}%
\pgfpathclose%
\pgfusepath{stroke}%
\end{pgfscope}%
\end{pgfpicture}%
\makeatother%
\endgroup%

	\vspace*{-0.4cm}
	\caption{200 K. Bin size $0.011e$}
	\end{subfigure}
	\quad
	\begin{subfigure}[b]{0.45\textwidth}
	\hspace*{-0.4cm}
	%% Creator: Matplotlib, PGF backend
%%
%% To include the figure in your LaTeX document, write
%%   \input{<filename>.pgf}
%%
%% Make sure the required packages are loaded in your preamble
%%   \usepackage{pgf}
%%
%% Figures using additional raster images can only be included by \input if
%% they are in the same directory as the main LaTeX file. For loading figures
%% from other directories you can use the `import` package
%%   \usepackage{import}
%% and then include the figures with
%%   \import{<path to file>}{<filename>.pgf}
%%
%% Matplotlib used the following preamble
%%   \usepackage[utf8x]{inputenc}
%%   \usepackage[T1]{fontenc}
%%
\begingroup%
\makeatletter%
\begin{pgfpicture}%
\pgfpathrectangle{\pgfpointorigin}{\pgfqpoint{2.518842in}{3.060408in}}%
\pgfusepath{use as bounding box, clip}%
\begin{pgfscope}%
\pgfsetbuttcap%
\pgfsetmiterjoin%
\definecolor{currentfill}{rgb}{1.000000,1.000000,1.000000}%
\pgfsetfillcolor{currentfill}%
\pgfsetlinewidth{0.000000pt}%
\definecolor{currentstroke}{rgb}{1.000000,1.000000,1.000000}%
\pgfsetstrokecolor{currentstroke}%
\pgfsetdash{}{0pt}%
\pgfpathmoveto{\pgfqpoint{0.000000in}{0.000000in}}%
\pgfpathlineto{\pgfqpoint{2.518842in}{0.000000in}}%
\pgfpathlineto{\pgfqpoint{2.518842in}{3.060408in}}%
\pgfpathlineto{\pgfqpoint{0.000000in}{3.060408in}}%
\pgfpathclose%
\pgfusepath{fill}%
\end{pgfscope}%
\begin{pgfscope}%
\pgfsetbuttcap%
\pgfsetmiterjoin%
\definecolor{currentfill}{rgb}{1.000000,1.000000,1.000000}%
\pgfsetfillcolor{currentfill}%
\pgfsetlinewidth{0.000000pt}%
\definecolor{currentstroke}{rgb}{0.000000,0.000000,0.000000}%
\pgfsetstrokecolor{currentstroke}%
\pgfsetstrokeopacity{0.000000}%
\pgfsetdash{}{0pt}%
\pgfpathmoveto{\pgfqpoint{0.374692in}{0.319877in}}%
\pgfpathlineto{\pgfqpoint{1.954366in}{0.319877in}}%
\pgfpathlineto{\pgfqpoint{1.954366in}{2.912580in}}%
\pgfpathlineto{\pgfqpoint{0.374692in}{2.912580in}}%
\pgfpathclose%
\pgfusepath{fill}%
\end{pgfscope}%
\begin{pgfscope}%
\pgfpathrectangle{\pgfqpoint{0.374692in}{0.319877in}}{\pgfqpoint{1.579674in}{2.592703in}} %
\pgfusepath{clip}%
\pgfsys@transformshift{0.374692in}{0.319877in}%
\pgftext[left,bottom]{\pgfimage[interpolate=true,width=1.580000in,height=2.600000in]{PerrNN_vs_dq_Ti_300K-img0.png}}%
\end{pgfscope}%
\begin{pgfscope}%
\pgfpathrectangle{\pgfqpoint{0.374692in}{0.319877in}}{\pgfqpoint{1.579674in}{2.592703in}} %
\pgfusepath{clip}%
\pgfsetbuttcap%
\pgfsetroundjoin%
\definecolor{currentfill}{rgb}{1.000000,0.752941,0.796078}%
\pgfsetfillcolor{currentfill}%
\pgfsetlinewidth{1.003750pt}%
\definecolor{currentstroke}{rgb}{1.000000,0.752941,0.796078}%
\pgfsetstrokecolor{currentstroke}%
\pgfsetdash{}{0pt}%
\pgfpathmoveto{\pgfqpoint{0.733709in}{1.172246in}}%
\pgfpathcurveto{\pgfqpoint{0.744759in}{1.172246in}}{\pgfqpoint{0.755358in}{1.176636in}}{\pgfqpoint{0.763171in}{1.184450in}}%
\pgfpathcurveto{\pgfqpoint{0.770985in}{1.192264in}}{\pgfqpoint{0.775375in}{1.202863in}}{\pgfqpoint{0.775375in}{1.213913in}}%
\pgfpathcurveto{\pgfqpoint{0.775375in}{1.224963in}}{\pgfqpoint{0.770985in}{1.235562in}}{\pgfqpoint{0.763171in}{1.243376in}}%
\pgfpathcurveto{\pgfqpoint{0.755358in}{1.251189in}}{\pgfqpoint{0.744759in}{1.255579in}}{\pgfqpoint{0.733709in}{1.255579in}}%
\pgfpathcurveto{\pgfqpoint{0.722659in}{1.255579in}}{\pgfqpoint{0.712060in}{1.251189in}}{\pgfqpoint{0.704246in}{1.243376in}}%
\pgfpathcurveto{\pgfqpoint{0.696432in}{1.235562in}}{\pgfqpoint{0.692042in}{1.224963in}}{\pgfqpoint{0.692042in}{1.213913in}}%
\pgfpathcurveto{\pgfqpoint{0.692042in}{1.202863in}}{\pgfqpoint{0.696432in}{1.192264in}}{\pgfqpoint{0.704246in}{1.184450in}}%
\pgfpathcurveto{\pgfqpoint{0.712060in}{1.176636in}}{\pgfqpoint{0.722659in}{1.172246in}}{\pgfqpoint{0.733709in}{1.172246in}}%
\pgfpathclose%
\pgfusepath{stroke,fill}%
\end{pgfscope}%
\begin{pgfscope}%
\pgfpathrectangle{\pgfqpoint{0.374692in}{0.319877in}}{\pgfqpoint{1.579674in}{2.592703in}} %
\pgfusepath{clip}%
\pgfsetbuttcap%
\pgfsetroundjoin%
\definecolor{currentfill}{rgb}{1.000000,0.752941,0.796078}%
\pgfsetfillcolor{currentfill}%
\pgfsetlinewidth{1.003750pt}%
\definecolor{currentstroke}{rgb}{1.000000,0.752941,0.796078}%
\pgfsetstrokecolor{currentstroke}%
\pgfsetdash{}{0pt}%
\pgfpathmoveto{\pgfqpoint{0.877315in}{1.254199in}}%
\pgfpathcurveto{\pgfqpoint{0.888366in}{1.254199in}}{\pgfqpoint{0.898965in}{1.258590in}}{\pgfqpoint{0.906778in}{1.266403in}}%
\pgfpathcurveto{\pgfqpoint{0.914592in}{1.274217in}}{\pgfqpoint{0.918982in}{1.284816in}}{\pgfqpoint{0.918982in}{1.295866in}}%
\pgfpathcurveto{\pgfqpoint{0.918982in}{1.306916in}}{\pgfqpoint{0.914592in}{1.317515in}}{\pgfqpoint{0.906778in}{1.325329in}}%
\pgfpathcurveto{\pgfqpoint{0.898965in}{1.333142in}}{\pgfqpoint{0.888366in}{1.337533in}}{\pgfqpoint{0.877315in}{1.337533in}}%
\pgfpathcurveto{\pgfqpoint{0.866265in}{1.337533in}}{\pgfqpoint{0.855666in}{1.333142in}}{\pgfqpoint{0.847853in}{1.325329in}}%
\pgfpathcurveto{\pgfqpoint{0.840039in}{1.317515in}}{\pgfqpoint{0.835649in}{1.306916in}}{\pgfqpoint{0.835649in}{1.295866in}}%
\pgfpathcurveto{\pgfqpoint{0.835649in}{1.284816in}}{\pgfqpoint{0.840039in}{1.274217in}}{\pgfqpoint{0.847853in}{1.266403in}}%
\pgfpathcurveto{\pgfqpoint{0.855666in}{1.258590in}}{\pgfqpoint{0.866265in}{1.254199in}}{\pgfqpoint{0.877315in}{1.254199in}}%
\pgfpathclose%
\pgfusepath{stroke,fill}%
\end{pgfscope}%
\begin{pgfscope}%
\pgfpathrectangle{\pgfqpoint{0.374692in}{0.319877in}}{\pgfqpoint{1.579674in}{2.592703in}} %
\pgfusepath{clip}%
\pgfsetbuttcap%
\pgfsetroundjoin%
\definecolor{currentfill}{rgb}{1.000000,0.752941,0.796078}%
\pgfsetfillcolor{currentfill}%
\pgfsetlinewidth{1.003750pt}%
\definecolor{currentstroke}{rgb}{1.000000,0.752941,0.796078}%
\pgfsetstrokecolor{currentstroke}%
\pgfsetdash{}{0pt}%
\pgfpathmoveto{\pgfqpoint{1.020922in}{1.265976in}}%
\pgfpathcurveto{\pgfqpoint{1.031972in}{1.265976in}}{\pgfqpoint{1.042571in}{1.270366in}}{\pgfqpoint{1.050385in}{1.278180in}}%
\pgfpathcurveto{\pgfqpoint{1.058199in}{1.285993in}}{\pgfqpoint{1.062589in}{1.296592in}}{\pgfqpoint{1.062589in}{1.307642in}}%
\pgfpathcurveto{\pgfqpoint{1.062589in}{1.318692in}}{\pgfqpoint{1.058199in}{1.329291in}}{\pgfqpoint{1.050385in}{1.337105in}}%
\pgfpathcurveto{\pgfqpoint{1.042571in}{1.344919in}}{\pgfqpoint{1.031972in}{1.349309in}}{\pgfqpoint{1.020922in}{1.349309in}}%
\pgfpathcurveto{\pgfqpoint{1.009872in}{1.349309in}}{\pgfqpoint{0.999273in}{1.344919in}}{\pgfqpoint{0.991459in}{1.337105in}}%
\pgfpathcurveto{\pgfqpoint{0.983646in}{1.329291in}}{\pgfqpoint{0.979255in}{1.318692in}}{\pgfqpoint{0.979255in}{1.307642in}}%
\pgfpathcurveto{\pgfqpoint{0.979255in}{1.296592in}}{\pgfqpoint{0.983646in}{1.285993in}}{\pgfqpoint{0.991459in}{1.278180in}}%
\pgfpathcurveto{\pgfqpoint{0.999273in}{1.270366in}}{\pgfqpoint{1.009872in}{1.265976in}}{\pgfqpoint{1.020922in}{1.265976in}}%
\pgfpathclose%
\pgfusepath{stroke,fill}%
\end{pgfscope}%
\begin{pgfscope}%
\pgfpathrectangle{\pgfqpoint{0.374692in}{0.319877in}}{\pgfqpoint{1.579674in}{2.592703in}} %
\pgfusepath{clip}%
\pgfsetbuttcap%
\pgfsetroundjoin%
\definecolor{currentfill}{rgb}{1.000000,0.752941,0.796078}%
\pgfsetfillcolor{currentfill}%
\pgfsetlinewidth{1.003750pt}%
\definecolor{currentstroke}{rgb}{1.000000,0.752941,0.796078}%
\pgfsetstrokecolor{currentstroke}%
\pgfsetdash{}{0pt}%
\pgfpathmoveto{\pgfqpoint{1.164529in}{1.337754in}}%
\pgfpathcurveto{\pgfqpoint{1.175579in}{1.337754in}}{\pgfqpoint{1.186178in}{1.342144in}}{\pgfqpoint{1.193992in}{1.349957in}}%
\pgfpathcurveto{\pgfqpoint{1.201805in}{1.357771in}}{\pgfqpoint{1.206196in}{1.368370in}}{\pgfqpoint{1.206196in}{1.379420in}}%
\pgfpathcurveto{\pgfqpoint{1.206196in}{1.390470in}}{\pgfqpoint{1.201805in}{1.401069in}}{\pgfqpoint{1.193992in}{1.408883in}}%
\pgfpathcurveto{\pgfqpoint{1.186178in}{1.416697in}}{\pgfqpoint{1.175579in}{1.421087in}}{\pgfqpoint{1.164529in}{1.421087in}}%
\pgfpathcurveto{\pgfqpoint{1.153479in}{1.421087in}}{\pgfqpoint{1.142880in}{1.416697in}}{\pgfqpoint{1.135066in}{1.408883in}}%
\pgfpathcurveto{\pgfqpoint{1.127252in}{1.401069in}}{\pgfqpoint{1.122862in}{1.390470in}}{\pgfqpoint{1.122862in}{1.379420in}}%
\pgfpathcurveto{\pgfqpoint{1.122862in}{1.368370in}}{\pgfqpoint{1.127252in}{1.357771in}}{\pgfqpoint{1.135066in}{1.349957in}}%
\pgfpathcurveto{\pgfqpoint{1.142880in}{1.342144in}}{\pgfqpoint{1.153479in}{1.337754in}}{\pgfqpoint{1.164529in}{1.337754in}}%
\pgfpathclose%
\pgfusepath{stroke,fill}%
\end{pgfscope}%
\begin{pgfscope}%
\pgfpathrectangle{\pgfqpoint{0.374692in}{0.319877in}}{\pgfqpoint{1.579674in}{2.592703in}} %
\pgfusepath{clip}%
\pgfsetbuttcap%
\pgfsetroundjoin%
\definecolor{currentfill}{rgb}{1.000000,0.752941,0.796078}%
\pgfsetfillcolor{currentfill}%
\pgfsetlinewidth{1.003750pt}%
\definecolor{currentstroke}{rgb}{1.000000,0.752941,0.796078}%
\pgfsetstrokecolor{currentstroke}%
\pgfsetdash{}{0pt}%
\pgfpathmoveto{\pgfqpoint{1.308136in}{1.418106in}}%
\pgfpathcurveto{\pgfqpoint{1.319186in}{1.418106in}}{\pgfqpoint{1.329785in}{1.422496in}}{\pgfqpoint{1.337598in}{1.430310in}}%
\pgfpathcurveto{\pgfqpoint{1.345412in}{1.438123in}}{\pgfqpoint{1.349802in}{1.448722in}}{\pgfqpoint{1.349802in}{1.459773in}}%
\pgfpathcurveto{\pgfqpoint{1.349802in}{1.470823in}}{\pgfqpoint{1.345412in}{1.481422in}}{\pgfqpoint{1.337598in}{1.489235in}}%
\pgfpathcurveto{\pgfqpoint{1.329785in}{1.497049in}}{\pgfqpoint{1.319186in}{1.501439in}}{\pgfqpoint{1.308136in}{1.501439in}}%
\pgfpathcurveto{\pgfqpoint{1.297085in}{1.501439in}}{\pgfqpoint{1.286486in}{1.497049in}}{\pgfqpoint{1.278673in}{1.489235in}}%
\pgfpathcurveto{\pgfqpoint{1.270859in}{1.481422in}}{\pgfqpoint{1.266469in}{1.470823in}}{\pgfqpoint{1.266469in}{1.459773in}}%
\pgfpathcurveto{\pgfqpoint{1.266469in}{1.448722in}}{\pgfqpoint{1.270859in}{1.438123in}}{\pgfqpoint{1.278673in}{1.430310in}}%
\pgfpathcurveto{\pgfqpoint{1.286486in}{1.422496in}}{\pgfqpoint{1.297085in}{1.418106in}}{\pgfqpoint{1.308136in}{1.418106in}}%
\pgfpathclose%
\pgfusepath{stroke,fill}%
\end{pgfscope}%
\begin{pgfscope}%
\pgfpathrectangle{\pgfqpoint{0.374692in}{0.319877in}}{\pgfqpoint{1.579674in}{2.592703in}} %
\pgfusepath{clip}%
\pgfsetbuttcap%
\pgfsetroundjoin%
\definecolor{currentfill}{rgb}{1.000000,0.752941,0.796078}%
\pgfsetfillcolor{currentfill}%
\pgfsetlinewidth{1.003750pt}%
\definecolor{currentstroke}{rgb}{1.000000,0.752941,0.796078}%
\pgfsetstrokecolor{currentstroke}%
\pgfsetdash{}{0pt}%
\pgfpathmoveto{\pgfqpoint{1.451742in}{1.395755in}}%
\pgfpathcurveto{\pgfqpoint{1.462792in}{1.395755in}}{\pgfqpoint{1.473391in}{1.400145in}}{\pgfqpoint{1.481205in}{1.407959in}}%
\pgfpathcurveto{\pgfqpoint{1.489019in}{1.415772in}}{\pgfqpoint{1.493409in}{1.426372in}}{\pgfqpoint{1.493409in}{1.437422in}}%
\pgfpathcurveto{\pgfqpoint{1.493409in}{1.448472in}}{\pgfqpoint{1.489019in}{1.459071in}}{\pgfqpoint{1.481205in}{1.466884in}}%
\pgfpathcurveto{\pgfqpoint{1.473391in}{1.474698in}}{\pgfqpoint{1.462792in}{1.479088in}}{\pgfqpoint{1.451742in}{1.479088in}}%
\pgfpathcurveto{\pgfqpoint{1.440692in}{1.479088in}}{\pgfqpoint{1.430093in}{1.474698in}}{\pgfqpoint{1.422279in}{1.466884in}}%
\pgfpathcurveto{\pgfqpoint{1.414466in}{1.459071in}}{\pgfqpoint{1.410076in}{1.448472in}}{\pgfqpoint{1.410076in}{1.437422in}}%
\pgfpathcurveto{\pgfqpoint{1.410076in}{1.426372in}}{\pgfqpoint{1.414466in}{1.415772in}}{\pgfqpoint{1.422279in}{1.407959in}}%
\pgfpathcurveto{\pgfqpoint{1.430093in}{1.400145in}}{\pgfqpoint{1.440692in}{1.395755in}}{\pgfqpoint{1.451742in}{1.395755in}}%
\pgfpathclose%
\pgfusepath{stroke,fill}%
\end{pgfscope}%
\begin{pgfscope}%
\pgfsetbuttcap%
\pgfsetroundjoin%
\definecolor{currentfill}{rgb}{0.000000,0.000000,0.000000}%
\pgfsetfillcolor{currentfill}%
\pgfsetlinewidth{0.803000pt}%
\definecolor{currentstroke}{rgb}{0.000000,0.000000,0.000000}%
\pgfsetstrokecolor{currentstroke}%
\pgfsetdash{}{0pt}%
\pgfsys@defobject{currentmarker}{\pgfqpoint{0.000000in}{-0.048611in}}{\pgfqpoint{0.000000in}{0.000000in}}{%
\pgfpathmoveto{\pgfqpoint{0.000000in}{0.000000in}}%
\pgfpathlineto{\pgfqpoint{0.000000in}{-0.048611in}}%
\pgfusepath{stroke,fill}%
}%
\begin{pgfscope}%
\pgfsys@transformshift{0.670881in}{0.319877in}%
\pgfsys@useobject{currentmarker}{}%
\end{pgfscope}%
\end{pgfscope}%
\begin{pgfscope}%
\pgftext[x=0.670881in,y=0.222655in,,top]{\rmfamily\fontsize{10.000000}{12.000000}\selectfont \(\displaystyle -0.05\)}%
\end{pgfscope}%
\begin{pgfscope}%
\pgfsetbuttcap%
\pgfsetroundjoin%
\definecolor{currentfill}{rgb}{0.000000,0.000000,0.000000}%
\pgfsetfillcolor{currentfill}%
\pgfsetlinewidth{0.803000pt}%
\definecolor{currentstroke}{rgb}{0.000000,0.000000,0.000000}%
\pgfsetstrokecolor{currentstroke}%
\pgfsetdash{}{0pt}%
\pgfsys@defobject{currentmarker}{\pgfqpoint{0.000000in}{-0.048611in}}{\pgfqpoint{0.000000in}{0.000000in}}{%
\pgfpathmoveto{\pgfqpoint{0.000000in}{0.000000in}}%
\pgfpathlineto{\pgfqpoint{0.000000in}{-0.048611in}}%
\pgfusepath{stroke,fill}%
}%
\begin{pgfscope}%
\pgfsys@transformshift{1.164529in}{0.319877in}%
\pgfsys@useobject{currentmarker}{}%
\end{pgfscope}%
\end{pgfscope}%
\begin{pgfscope}%
\pgftext[x=1.164529in,y=0.222655in,,top]{\rmfamily\fontsize{10.000000}{12.000000}\selectfont \(\displaystyle 0.00\)}%
\end{pgfscope}%
\begin{pgfscope}%
\pgfsetbuttcap%
\pgfsetroundjoin%
\definecolor{currentfill}{rgb}{0.000000,0.000000,0.000000}%
\pgfsetfillcolor{currentfill}%
\pgfsetlinewidth{0.803000pt}%
\definecolor{currentstroke}{rgb}{0.000000,0.000000,0.000000}%
\pgfsetstrokecolor{currentstroke}%
\pgfsetdash{}{0pt}%
\pgfsys@defobject{currentmarker}{\pgfqpoint{0.000000in}{-0.048611in}}{\pgfqpoint{0.000000in}{0.000000in}}{%
\pgfpathmoveto{\pgfqpoint{0.000000in}{0.000000in}}%
\pgfpathlineto{\pgfqpoint{0.000000in}{-0.048611in}}%
\pgfusepath{stroke,fill}%
}%
\begin{pgfscope}%
\pgfsys@transformshift{1.658177in}{0.319877in}%
\pgfsys@useobject{currentmarker}{}%
\end{pgfscope}%
\end{pgfscope}%
\begin{pgfscope}%
\pgftext[x=1.658177in,y=0.222655in,,top]{\rmfamily\fontsize{10.000000}{12.000000}\selectfont \(\displaystyle 0.05\)}%
\end{pgfscope}%
\begin{pgfscope}%
\pgfsetbuttcap%
\pgfsetroundjoin%
\definecolor{currentfill}{rgb}{0.000000,0.000000,0.000000}%
\pgfsetfillcolor{currentfill}%
\pgfsetlinewidth{0.803000pt}%
\definecolor{currentstroke}{rgb}{0.000000,0.000000,0.000000}%
\pgfsetstrokecolor{currentstroke}%
\pgfsetdash{}{0pt}%
\pgfsys@defobject{currentmarker}{\pgfqpoint{-0.048611in}{0.000000in}}{\pgfqpoint{0.000000in}{0.000000in}}{%
\pgfpathmoveto{\pgfqpoint{0.000000in}{0.000000in}}%
\pgfpathlineto{\pgfqpoint{-0.048611in}{0.000000in}}%
\pgfusepath{stroke,fill}%
}%
\begin{pgfscope}%
\pgfsys@transformshift{0.374692in}{0.319877in}%
\pgfsys@useobject{currentmarker}{}%
\end{pgfscope}%
\end{pgfscope}%
\begin{pgfscope}%
\pgftext[x=0.100000in,y=0.272050in,left,base]{\rmfamily\fontsize{10.000000}{12.000000}\selectfont \(\displaystyle 0.0\)}%
\end{pgfscope}%
\begin{pgfscope}%
\pgfsetbuttcap%
\pgfsetroundjoin%
\definecolor{currentfill}{rgb}{0.000000,0.000000,0.000000}%
\pgfsetfillcolor{currentfill}%
\pgfsetlinewidth{0.803000pt}%
\definecolor{currentstroke}{rgb}{0.000000,0.000000,0.000000}%
\pgfsetstrokecolor{currentstroke}%
\pgfsetdash{}{0pt}%
\pgfsys@defobject{currentmarker}{\pgfqpoint{-0.048611in}{0.000000in}}{\pgfqpoint{0.000000in}{0.000000in}}{%
\pgfpathmoveto{\pgfqpoint{0.000000in}{0.000000in}}%
\pgfpathlineto{\pgfqpoint{-0.048611in}{0.000000in}}%
\pgfusepath{stroke,fill}%
}%
\begin{pgfscope}%
\pgfsys@transformshift{0.374692in}{0.838418in}%
\pgfsys@useobject{currentmarker}{}%
\end{pgfscope}%
\end{pgfscope}%
\begin{pgfscope}%
\pgftext[x=0.100000in,y=0.790590in,left,base]{\rmfamily\fontsize{10.000000}{12.000000}\selectfont \(\displaystyle 0.1\)}%
\end{pgfscope}%
\begin{pgfscope}%
\pgfsetbuttcap%
\pgfsetroundjoin%
\definecolor{currentfill}{rgb}{0.000000,0.000000,0.000000}%
\pgfsetfillcolor{currentfill}%
\pgfsetlinewidth{0.803000pt}%
\definecolor{currentstroke}{rgb}{0.000000,0.000000,0.000000}%
\pgfsetstrokecolor{currentstroke}%
\pgfsetdash{}{0pt}%
\pgfsys@defobject{currentmarker}{\pgfqpoint{-0.048611in}{0.000000in}}{\pgfqpoint{0.000000in}{0.000000in}}{%
\pgfpathmoveto{\pgfqpoint{0.000000in}{0.000000in}}%
\pgfpathlineto{\pgfqpoint{-0.048611in}{0.000000in}}%
\pgfusepath{stroke,fill}%
}%
\begin{pgfscope}%
\pgfsys@transformshift{0.374692in}{1.356958in}%
\pgfsys@useobject{currentmarker}{}%
\end{pgfscope}%
\end{pgfscope}%
\begin{pgfscope}%
\pgftext[x=0.100000in,y=1.309131in,left,base]{\rmfamily\fontsize{10.000000}{12.000000}\selectfont \(\displaystyle 0.2\)}%
\end{pgfscope}%
\begin{pgfscope}%
\pgfsetbuttcap%
\pgfsetroundjoin%
\definecolor{currentfill}{rgb}{0.000000,0.000000,0.000000}%
\pgfsetfillcolor{currentfill}%
\pgfsetlinewidth{0.803000pt}%
\definecolor{currentstroke}{rgb}{0.000000,0.000000,0.000000}%
\pgfsetstrokecolor{currentstroke}%
\pgfsetdash{}{0pt}%
\pgfsys@defobject{currentmarker}{\pgfqpoint{-0.048611in}{0.000000in}}{\pgfqpoint{0.000000in}{0.000000in}}{%
\pgfpathmoveto{\pgfqpoint{0.000000in}{0.000000in}}%
\pgfpathlineto{\pgfqpoint{-0.048611in}{0.000000in}}%
\pgfusepath{stroke,fill}%
}%
\begin{pgfscope}%
\pgfsys@transformshift{0.374692in}{1.875499in}%
\pgfsys@useobject{currentmarker}{}%
\end{pgfscope}%
\end{pgfscope}%
\begin{pgfscope}%
\pgftext[x=0.100000in,y=1.827671in,left,base]{\rmfamily\fontsize{10.000000}{12.000000}\selectfont \(\displaystyle 0.3\)}%
\end{pgfscope}%
\begin{pgfscope}%
\pgfsetbuttcap%
\pgfsetroundjoin%
\definecolor{currentfill}{rgb}{0.000000,0.000000,0.000000}%
\pgfsetfillcolor{currentfill}%
\pgfsetlinewidth{0.803000pt}%
\definecolor{currentstroke}{rgb}{0.000000,0.000000,0.000000}%
\pgfsetstrokecolor{currentstroke}%
\pgfsetdash{}{0pt}%
\pgfsys@defobject{currentmarker}{\pgfqpoint{-0.048611in}{0.000000in}}{\pgfqpoint{0.000000in}{0.000000in}}{%
\pgfpathmoveto{\pgfqpoint{0.000000in}{0.000000in}}%
\pgfpathlineto{\pgfqpoint{-0.048611in}{0.000000in}}%
\pgfusepath{stroke,fill}%
}%
\begin{pgfscope}%
\pgfsys@transformshift{0.374692in}{2.394040in}%
\pgfsys@useobject{currentmarker}{}%
\end{pgfscope}%
\end{pgfscope}%
\begin{pgfscope}%
\pgftext[x=0.100000in,y=2.346212in,left,base]{\rmfamily\fontsize{10.000000}{12.000000}\selectfont \(\displaystyle 0.4\)}%
\end{pgfscope}%
\begin{pgfscope}%
\pgfsetbuttcap%
\pgfsetroundjoin%
\definecolor{currentfill}{rgb}{0.000000,0.000000,0.000000}%
\pgfsetfillcolor{currentfill}%
\pgfsetlinewidth{0.803000pt}%
\definecolor{currentstroke}{rgb}{0.000000,0.000000,0.000000}%
\pgfsetstrokecolor{currentstroke}%
\pgfsetdash{}{0pt}%
\pgfsys@defobject{currentmarker}{\pgfqpoint{-0.048611in}{0.000000in}}{\pgfqpoint{0.000000in}{0.000000in}}{%
\pgfpathmoveto{\pgfqpoint{0.000000in}{0.000000in}}%
\pgfpathlineto{\pgfqpoint{-0.048611in}{0.000000in}}%
\pgfusepath{stroke,fill}%
}%
\begin{pgfscope}%
\pgfsys@transformshift{0.374692in}{2.912580in}%
\pgfsys@useobject{currentmarker}{}%
\end{pgfscope}%
\end{pgfscope}%
\begin{pgfscope}%
\pgftext[x=0.100000in,y=2.864752in,left,base]{\rmfamily\fontsize{10.000000}{12.000000}\selectfont \(\displaystyle 0.5\)}%
\end{pgfscope}%
\begin{pgfscope}%
\pgfsetrectcap%
\pgfsetmiterjoin%
\pgfsetlinewidth{0.803000pt}%
\definecolor{currentstroke}{rgb}{0.000000,0.000000,0.000000}%
\pgfsetstrokecolor{currentstroke}%
\pgfsetdash{}{0pt}%
\pgfpathmoveto{\pgfqpoint{0.374692in}{0.319877in}}%
\pgfpathlineto{\pgfqpoint{0.374692in}{2.912580in}}%
\pgfusepath{stroke}%
\end{pgfscope}%
\begin{pgfscope}%
\pgfsetrectcap%
\pgfsetmiterjoin%
\pgfsetlinewidth{0.803000pt}%
\definecolor{currentstroke}{rgb}{0.000000,0.000000,0.000000}%
\pgfsetstrokecolor{currentstroke}%
\pgfsetdash{}{0pt}%
\pgfpathmoveto{\pgfqpoint{1.954366in}{0.319877in}}%
\pgfpathlineto{\pgfqpoint{1.954366in}{2.912580in}}%
\pgfusepath{stroke}%
\end{pgfscope}%
\begin{pgfscope}%
\pgfsetrectcap%
\pgfsetmiterjoin%
\pgfsetlinewidth{0.803000pt}%
\definecolor{currentstroke}{rgb}{0.000000,0.000000,0.000000}%
\pgfsetstrokecolor{currentstroke}%
\pgfsetdash{}{0pt}%
\pgfpathmoveto{\pgfqpoint{0.374692in}{0.319877in}}%
\pgfpathlineto{\pgfqpoint{1.954366in}{0.319877in}}%
\pgfusepath{stroke}%
\end{pgfscope}%
\begin{pgfscope}%
\pgfsetrectcap%
\pgfsetmiterjoin%
\pgfsetlinewidth{0.803000pt}%
\definecolor{currentstroke}{rgb}{0.000000,0.000000,0.000000}%
\pgfsetstrokecolor{currentstroke}%
\pgfsetdash{}{0pt}%
\pgfpathmoveto{\pgfqpoint{0.374692in}{2.912580in}}%
\pgfpathlineto{\pgfqpoint{1.954366in}{2.912580in}}%
\pgfusepath{stroke}%
\end{pgfscope}%
\begin{pgfscope}%
\pgfpathrectangle{\pgfqpoint{2.053095in}{0.319877in}}{\pgfqpoint{0.129635in}{2.592703in}} %
\pgfusepath{clip}%
\pgfsetbuttcap%
\pgfsetmiterjoin%
\definecolor{currentfill}{rgb}{1.000000,1.000000,1.000000}%
\pgfsetfillcolor{currentfill}%
\pgfsetlinewidth{0.010037pt}%
\definecolor{currentstroke}{rgb}{1.000000,1.000000,1.000000}%
\pgfsetstrokecolor{currentstroke}%
\pgfsetdash{}{0pt}%
\pgfpathmoveto{\pgfqpoint{2.053095in}{0.319877in}}%
\pgfpathlineto{\pgfqpoint{2.053095in}{0.330005in}}%
\pgfpathlineto{\pgfqpoint{2.053095in}{2.902452in}}%
\pgfpathlineto{\pgfqpoint{2.053095in}{2.912580in}}%
\pgfpathlineto{\pgfqpoint{2.182730in}{2.912580in}}%
\pgfpathlineto{\pgfqpoint{2.182730in}{2.902452in}}%
\pgfpathlineto{\pgfqpoint{2.182730in}{0.330005in}}%
\pgfpathlineto{\pgfqpoint{2.182730in}{0.319877in}}%
\pgfpathclose%
\pgfusepath{stroke,fill}%
\end{pgfscope}%
\begin{pgfscope}%
\pgfsys@transformshift{2.050000in}{0.320408in}%
\pgftext[left,bottom]{\pgfimage[interpolate=true,width=0.130000in,height=2.590000in]{PerrNN_vs_dq_Ti_300K-img1.png}}%
\end{pgfscope}%
\begin{pgfscope}%
\pgfsetbuttcap%
\pgfsetroundjoin%
\definecolor{currentfill}{rgb}{0.000000,0.000000,0.000000}%
\pgfsetfillcolor{currentfill}%
\pgfsetlinewidth{0.803000pt}%
\definecolor{currentstroke}{rgb}{0.000000,0.000000,0.000000}%
\pgfsetstrokecolor{currentstroke}%
\pgfsetdash{}{0pt}%
\pgfsys@defobject{currentmarker}{\pgfqpoint{0.000000in}{0.000000in}}{\pgfqpoint{0.048611in}{0.000000in}}{%
\pgfpathmoveto{\pgfqpoint{0.000000in}{0.000000in}}%
\pgfpathlineto{\pgfqpoint{0.048611in}{0.000000in}}%
\pgfusepath{stroke,fill}%
}%
\begin{pgfscope}%
\pgfsys@transformshift{2.182730in}{0.319877in}%
\pgfsys@useobject{currentmarker}{}%
\end{pgfscope}%
\end{pgfscope}%
\begin{pgfscope}%
\pgftext[x=2.279953in,y=0.272050in,left,base]{\rmfamily\fontsize{10.000000}{12.000000}\selectfont \(\displaystyle 0\)}%
\end{pgfscope}%
\begin{pgfscope}%
\pgfsetbuttcap%
\pgfsetroundjoin%
\definecolor{currentfill}{rgb}{0.000000,0.000000,0.000000}%
\pgfsetfillcolor{currentfill}%
\pgfsetlinewidth{0.803000pt}%
\definecolor{currentstroke}{rgb}{0.000000,0.000000,0.000000}%
\pgfsetstrokecolor{currentstroke}%
\pgfsetdash{}{0pt}%
\pgfsys@defobject{currentmarker}{\pgfqpoint{0.000000in}{0.000000in}}{\pgfqpoint{0.048611in}{0.000000in}}{%
\pgfpathmoveto{\pgfqpoint{0.000000in}{0.000000in}}%
\pgfpathlineto{\pgfqpoint{0.048611in}{0.000000in}}%
\pgfusepath{stroke,fill}%
}%
\begin{pgfscope}%
\pgfsys@transformshift{2.182730in}{0.607955in}%
\pgfsys@useobject{currentmarker}{}%
\end{pgfscope}%
\end{pgfscope}%
\begin{pgfscope}%
\pgftext[x=2.279953in,y=0.560128in,left,base]{\rmfamily\fontsize{10.000000}{12.000000}\selectfont \(\displaystyle 2\)}%
\end{pgfscope}%
\begin{pgfscope}%
\pgfsetbuttcap%
\pgfsetroundjoin%
\definecolor{currentfill}{rgb}{0.000000,0.000000,0.000000}%
\pgfsetfillcolor{currentfill}%
\pgfsetlinewidth{0.803000pt}%
\definecolor{currentstroke}{rgb}{0.000000,0.000000,0.000000}%
\pgfsetstrokecolor{currentstroke}%
\pgfsetdash{}{0pt}%
\pgfsys@defobject{currentmarker}{\pgfqpoint{0.000000in}{0.000000in}}{\pgfqpoint{0.048611in}{0.000000in}}{%
\pgfpathmoveto{\pgfqpoint{0.000000in}{0.000000in}}%
\pgfpathlineto{\pgfqpoint{0.048611in}{0.000000in}}%
\pgfusepath{stroke,fill}%
}%
\begin{pgfscope}%
\pgfsys@transformshift{2.182730in}{0.896034in}%
\pgfsys@useobject{currentmarker}{}%
\end{pgfscope}%
\end{pgfscope}%
\begin{pgfscope}%
\pgftext[x=2.279953in,y=0.848206in,left,base]{\rmfamily\fontsize{10.000000}{12.000000}\selectfont \(\displaystyle 4\)}%
\end{pgfscope}%
\begin{pgfscope}%
\pgfsetbuttcap%
\pgfsetroundjoin%
\definecolor{currentfill}{rgb}{0.000000,0.000000,0.000000}%
\pgfsetfillcolor{currentfill}%
\pgfsetlinewidth{0.803000pt}%
\definecolor{currentstroke}{rgb}{0.000000,0.000000,0.000000}%
\pgfsetstrokecolor{currentstroke}%
\pgfsetdash{}{0pt}%
\pgfsys@defobject{currentmarker}{\pgfqpoint{0.000000in}{0.000000in}}{\pgfqpoint{0.048611in}{0.000000in}}{%
\pgfpathmoveto{\pgfqpoint{0.000000in}{0.000000in}}%
\pgfpathlineto{\pgfqpoint{0.048611in}{0.000000in}}%
\pgfusepath{stroke,fill}%
}%
\begin{pgfscope}%
\pgfsys@transformshift{2.182730in}{1.184112in}%
\pgfsys@useobject{currentmarker}{}%
\end{pgfscope}%
\end{pgfscope}%
\begin{pgfscope}%
\pgftext[x=2.279953in,y=1.136284in,left,base]{\rmfamily\fontsize{10.000000}{12.000000}\selectfont \(\displaystyle 6\)}%
\end{pgfscope}%
\begin{pgfscope}%
\pgfsetbuttcap%
\pgfsetroundjoin%
\definecolor{currentfill}{rgb}{0.000000,0.000000,0.000000}%
\pgfsetfillcolor{currentfill}%
\pgfsetlinewidth{0.803000pt}%
\definecolor{currentstroke}{rgb}{0.000000,0.000000,0.000000}%
\pgfsetstrokecolor{currentstroke}%
\pgfsetdash{}{0pt}%
\pgfsys@defobject{currentmarker}{\pgfqpoint{0.000000in}{0.000000in}}{\pgfqpoint{0.048611in}{0.000000in}}{%
\pgfpathmoveto{\pgfqpoint{0.000000in}{0.000000in}}%
\pgfpathlineto{\pgfqpoint{0.048611in}{0.000000in}}%
\pgfusepath{stroke,fill}%
}%
\begin{pgfscope}%
\pgfsys@transformshift{2.182730in}{1.472190in}%
\pgfsys@useobject{currentmarker}{}%
\end{pgfscope}%
\end{pgfscope}%
\begin{pgfscope}%
\pgftext[x=2.279953in,y=1.424362in,left,base]{\rmfamily\fontsize{10.000000}{12.000000}\selectfont \(\displaystyle 8\)}%
\end{pgfscope}%
\begin{pgfscope}%
\pgfsetbuttcap%
\pgfsetroundjoin%
\definecolor{currentfill}{rgb}{0.000000,0.000000,0.000000}%
\pgfsetfillcolor{currentfill}%
\pgfsetlinewidth{0.803000pt}%
\definecolor{currentstroke}{rgb}{0.000000,0.000000,0.000000}%
\pgfsetstrokecolor{currentstroke}%
\pgfsetdash{}{0pt}%
\pgfsys@defobject{currentmarker}{\pgfqpoint{0.000000in}{0.000000in}}{\pgfqpoint{0.048611in}{0.000000in}}{%
\pgfpathmoveto{\pgfqpoint{0.000000in}{0.000000in}}%
\pgfpathlineto{\pgfqpoint{0.048611in}{0.000000in}}%
\pgfusepath{stroke,fill}%
}%
\begin{pgfscope}%
\pgfsys@transformshift{2.182730in}{1.760268in}%
\pgfsys@useobject{currentmarker}{}%
\end{pgfscope}%
\end{pgfscope}%
\begin{pgfscope}%
\pgftext[x=2.279953in,y=1.712440in,left,base]{\rmfamily\fontsize{10.000000}{12.000000}\selectfont \(\displaystyle 10\)}%
\end{pgfscope}%
\begin{pgfscope}%
\pgfsetbuttcap%
\pgfsetroundjoin%
\definecolor{currentfill}{rgb}{0.000000,0.000000,0.000000}%
\pgfsetfillcolor{currentfill}%
\pgfsetlinewidth{0.803000pt}%
\definecolor{currentstroke}{rgb}{0.000000,0.000000,0.000000}%
\pgfsetstrokecolor{currentstroke}%
\pgfsetdash{}{0pt}%
\pgfsys@defobject{currentmarker}{\pgfqpoint{0.000000in}{0.000000in}}{\pgfqpoint{0.048611in}{0.000000in}}{%
\pgfpathmoveto{\pgfqpoint{0.000000in}{0.000000in}}%
\pgfpathlineto{\pgfqpoint{0.048611in}{0.000000in}}%
\pgfusepath{stroke,fill}%
}%
\begin{pgfscope}%
\pgfsys@transformshift{2.182730in}{2.048346in}%
\pgfsys@useobject{currentmarker}{}%
\end{pgfscope}%
\end{pgfscope}%
\begin{pgfscope}%
\pgftext[x=2.279953in,y=2.000518in,left,base]{\rmfamily\fontsize{10.000000}{12.000000}\selectfont \(\displaystyle 12\)}%
\end{pgfscope}%
\begin{pgfscope}%
\pgfsetbuttcap%
\pgfsetroundjoin%
\definecolor{currentfill}{rgb}{0.000000,0.000000,0.000000}%
\pgfsetfillcolor{currentfill}%
\pgfsetlinewidth{0.803000pt}%
\definecolor{currentstroke}{rgb}{0.000000,0.000000,0.000000}%
\pgfsetstrokecolor{currentstroke}%
\pgfsetdash{}{0pt}%
\pgfsys@defobject{currentmarker}{\pgfqpoint{0.000000in}{0.000000in}}{\pgfqpoint{0.048611in}{0.000000in}}{%
\pgfpathmoveto{\pgfqpoint{0.000000in}{0.000000in}}%
\pgfpathlineto{\pgfqpoint{0.048611in}{0.000000in}}%
\pgfusepath{stroke,fill}%
}%
\begin{pgfscope}%
\pgfsys@transformshift{2.182730in}{2.336424in}%
\pgfsys@useobject{currentmarker}{}%
\end{pgfscope}%
\end{pgfscope}%
\begin{pgfscope}%
\pgftext[x=2.279953in,y=2.288596in,left,base]{\rmfamily\fontsize{10.000000}{12.000000}\selectfont \(\displaystyle 14\)}%
\end{pgfscope}%
\begin{pgfscope}%
\pgfsetbuttcap%
\pgfsetroundjoin%
\definecolor{currentfill}{rgb}{0.000000,0.000000,0.000000}%
\pgfsetfillcolor{currentfill}%
\pgfsetlinewidth{0.803000pt}%
\definecolor{currentstroke}{rgb}{0.000000,0.000000,0.000000}%
\pgfsetstrokecolor{currentstroke}%
\pgfsetdash{}{0pt}%
\pgfsys@defobject{currentmarker}{\pgfqpoint{0.000000in}{0.000000in}}{\pgfqpoint{0.048611in}{0.000000in}}{%
\pgfpathmoveto{\pgfqpoint{0.000000in}{0.000000in}}%
\pgfpathlineto{\pgfqpoint{0.048611in}{0.000000in}}%
\pgfusepath{stroke,fill}%
}%
\begin{pgfscope}%
\pgfsys@transformshift{2.182730in}{2.624502in}%
\pgfsys@useobject{currentmarker}{}%
\end{pgfscope}%
\end{pgfscope}%
\begin{pgfscope}%
\pgftext[x=2.279953in,y=2.576674in,left,base]{\rmfamily\fontsize{10.000000}{12.000000}\selectfont \(\displaystyle 16\)}%
\end{pgfscope}%
\begin{pgfscope}%
\pgfsetbuttcap%
\pgfsetroundjoin%
\definecolor{currentfill}{rgb}{0.000000,0.000000,0.000000}%
\pgfsetfillcolor{currentfill}%
\pgfsetlinewidth{0.803000pt}%
\definecolor{currentstroke}{rgb}{0.000000,0.000000,0.000000}%
\pgfsetstrokecolor{currentstroke}%
\pgfsetdash{}{0pt}%
\pgfsys@defobject{currentmarker}{\pgfqpoint{0.000000in}{0.000000in}}{\pgfqpoint{0.048611in}{0.000000in}}{%
\pgfpathmoveto{\pgfqpoint{0.000000in}{0.000000in}}%
\pgfpathlineto{\pgfqpoint{0.048611in}{0.000000in}}%
\pgfusepath{stroke,fill}%
}%
\begin{pgfscope}%
\pgfsys@transformshift{2.182730in}{2.912580in}%
\pgfsys@useobject{currentmarker}{}%
\end{pgfscope}%
\end{pgfscope}%
\begin{pgfscope}%
\pgftext[x=2.279953in,y=2.864752in,left,base]{\rmfamily\fontsize{10.000000}{12.000000}\selectfont \(\displaystyle 18\)}%
\end{pgfscope}%
\begin{pgfscope}%
\pgfsetbuttcap%
\pgfsetmiterjoin%
\pgfsetlinewidth{0.803000pt}%
\definecolor{currentstroke}{rgb}{0.000000,0.000000,0.000000}%
\pgfsetstrokecolor{currentstroke}%
\pgfsetdash{}{0pt}%
\pgfpathmoveto{\pgfqpoint{2.053095in}{0.319877in}}%
\pgfpathlineto{\pgfqpoint{2.053095in}{0.330005in}}%
\pgfpathlineto{\pgfqpoint{2.053095in}{2.902452in}}%
\pgfpathlineto{\pgfqpoint{2.053095in}{2.912580in}}%
\pgfpathlineto{\pgfqpoint{2.182730in}{2.912580in}}%
\pgfpathlineto{\pgfqpoint{2.182730in}{2.902452in}}%
\pgfpathlineto{\pgfqpoint{2.182730in}{0.330005in}}%
\pgfpathlineto{\pgfqpoint{2.182730in}{0.319877in}}%
\pgfpathclose%
\pgfusepath{stroke}%
\end{pgfscope}%
\end{pgfpicture}%
\makeatother%
\endgroup%

    \vspace*{-0.4cm}
	\caption{300 K. Bin size $0.014e$}
	\end{subfigure}
	\hspace{0.6cm}
	\begin{subfigure}[b]{0.45\textwidth}
	\hspace*{-0.4cm}
	%% Creator: Matplotlib, PGF backend
%%
%% To include the figure in your LaTeX document, write
%%   \input{<filename>.pgf}
%%
%% Make sure the required packages are loaded in your preamble
%%   \usepackage{pgf}
%%
%% Figures using additional raster images can only be included by \input if
%% they are in the same directory as the main LaTeX file. For loading figures
%% from other directories you can use the `import` package
%%   \usepackage{import}
%% and then include the figures with
%%   \import{<path to file>}{<filename>.pgf}
%%
%% Matplotlib used the following preamble
%%   \usepackage[utf8x]{inputenc}
%%   \usepackage[T1]{fontenc}
%%
\begingroup%
\makeatletter%
\begin{pgfpicture}%
\pgfpathrectangle{\pgfpointorigin}{\pgfqpoint{2.518842in}{3.060408in}}%
\pgfusepath{use as bounding box, clip}%
\begin{pgfscope}%
\pgfsetbuttcap%
\pgfsetmiterjoin%
\definecolor{currentfill}{rgb}{1.000000,1.000000,1.000000}%
\pgfsetfillcolor{currentfill}%
\pgfsetlinewidth{0.000000pt}%
\definecolor{currentstroke}{rgb}{1.000000,1.000000,1.000000}%
\pgfsetstrokecolor{currentstroke}%
\pgfsetdash{}{0pt}%
\pgfpathmoveto{\pgfqpoint{0.000000in}{0.000000in}}%
\pgfpathlineto{\pgfqpoint{2.518842in}{0.000000in}}%
\pgfpathlineto{\pgfqpoint{2.518842in}{3.060408in}}%
\pgfpathlineto{\pgfqpoint{0.000000in}{3.060408in}}%
\pgfpathclose%
\pgfusepath{fill}%
\end{pgfscope}%
\begin{pgfscope}%
\pgfsetbuttcap%
\pgfsetmiterjoin%
\definecolor{currentfill}{rgb}{1.000000,1.000000,1.000000}%
\pgfsetfillcolor{currentfill}%
\pgfsetlinewidth{0.000000pt}%
\definecolor{currentstroke}{rgb}{0.000000,0.000000,0.000000}%
\pgfsetstrokecolor{currentstroke}%
\pgfsetstrokeopacity{0.000000}%
\pgfsetdash{}{0pt}%
\pgfpathmoveto{\pgfqpoint{0.374692in}{0.319877in}}%
\pgfpathlineto{\pgfqpoint{1.954366in}{0.319877in}}%
\pgfpathlineto{\pgfqpoint{1.954366in}{2.912580in}}%
\pgfpathlineto{\pgfqpoint{0.374692in}{2.912580in}}%
\pgfpathclose%
\pgfusepath{fill}%
\end{pgfscope}%
\begin{pgfscope}%
\pgfpathrectangle{\pgfqpoint{0.374692in}{0.319877in}}{\pgfqpoint{1.579674in}{2.592703in}} %
\pgfusepath{clip}%
\pgfsys@transformshift{0.374692in}{0.319877in}%
\pgftext[left,bottom]{\pgfimage[interpolate=true,width=1.580000in,height=2.600000in]{PerrNN_vs_dq_Ti_500K-img0.png}}%
\end{pgfscope}%
\begin{pgfscope}%
\pgfpathrectangle{\pgfqpoint{0.374692in}{0.319877in}}{\pgfqpoint{1.579674in}{2.592703in}} %
\pgfusepath{clip}%
\pgfsetbuttcap%
\pgfsetroundjoin%
\definecolor{currentfill}{rgb}{1.000000,0.752941,0.796078}%
\pgfsetfillcolor{currentfill}%
\pgfsetlinewidth{1.003750pt}%
\definecolor{currentstroke}{rgb}{1.000000,0.752941,0.796078}%
\pgfsetstrokecolor{currentstroke}%
\pgfsetdash{}{0pt}%
\pgfpathmoveto{\pgfqpoint{0.670881in}{1.529860in}}%
\pgfpathcurveto{\pgfqpoint{0.681931in}{1.529860in}}{\pgfqpoint{0.692530in}{1.534251in}}{\pgfqpoint{0.700344in}{1.542064in}}%
\pgfpathcurveto{\pgfqpoint{0.708157in}{1.549878in}}{\pgfqpoint{0.712547in}{1.560477in}}{\pgfqpoint{0.712547in}{1.571527in}}%
\pgfpathcurveto{\pgfqpoint{0.712547in}{1.582577in}}{\pgfqpoint{0.708157in}{1.593176in}}{\pgfqpoint{0.700344in}{1.600990in}}%
\pgfpathcurveto{\pgfqpoint{0.692530in}{1.608803in}}{\pgfqpoint{0.681931in}{1.613194in}}{\pgfqpoint{0.670881in}{1.613194in}}%
\pgfpathcurveto{\pgfqpoint{0.659831in}{1.613194in}}{\pgfqpoint{0.649232in}{1.608803in}}{\pgfqpoint{0.641418in}{1.600990in}}%
\pgfpathcurveto{\pgfqpoint{0.633604in}{1.593176in}}{\pgfqpoint{0.629214in}{1.582577in}}{\pgfqpoint{0.629214in}{1.571527in}}%
\pgfpathcurveto{\pgfqpoint{0.629214in}{1.560477in}}{\pgfqpoint{0.633604in}{1.549878in}}{\pgfqpoint{0.641418in}{1.542064in}}%
\pgfpathcurveto{\pgfqpoint{0.649232in}{1.534251in}}{\pgfqpoint{0.659831in}{1.529860in}}{\pgfqpoint{0.670881in}{1.529860in}}%
\pgfpathclose%
\pgfusepath{stroke,fill}%
\end{pgfscope}%
\begin{pgfscope}%
\pgfpathrectangle{\pgfqpoint{0.374692in}{0.319877in}}{\pgfqpoint{1.579674in}{2.592703in}} %
\pgfusepath{clip}%
\pgfsetbuttcap%
\pgfsetroundjoin%
\definecolor{currentfill}{rgb}{1.000000,0.752941,0.796078}%
\pgfsetfillcolor{currentfill}%
\pgfsetlinewidth{1.003750pt}%
\definecolor{currentstroke}{rgb}{1.000000,0.752941,0.796078}%
\pgfsetstrokecolor{currentstroke}%
\pgfsetdash{}{0pt}%
\pgfpathmoveto{\pgfqpoint{0.868340in}{1.406273in}}%
\pgfpathcurveto{\pgfqpoint{0.879390in}{1.406273in}}{\pgfqpoint{0.889989in}{1.410663in}}{\pgfqpoint{0.897803in}{1.418477in}}%
\pgfpathcurveto{\pgfqpoint{0.905616in}{1.426291in}}{\pgfqpoint{0.910007in}{1.436890in}}{\pgfqpoint{0.910007in}{1.447940in}}%
\pgfpathcurveto{\pgfqpoint{0.910007in}{1.458990in}}{\pgfqpoint{0.905616in}{1.469589in}}{\pgfqpoint{0.897803in}{1.477402in}}%
\pgfpathcurveto{\pgfqpoint{0.889989in}{1.485216in}}{\pgfqpoint{0.879390in}{1.489606in}}{\pgfqpoint{0.868340in}{1.489606in}}%
\pgfpathcurveto{\pgfqpoint{0.857290in}{1.489606in}}{\pgfqpoint{0.846691in}{1.485216in}}{\pgfqpoint{0.838877in}{1.477402in}}%
\pgfpathcurveto{\pgfqpoint{0.831064in}{1.469589in}}{\pgfqpoint{0.826673in}{1.458990in}}{\pgfqpoint{0.826673in}{1.447940in}}%
\pgfpathcurveto{\pgfqpoint{0.826673in}{1.436890in}}{\pgfqpoint{0.831064in}{1.426291in}}{\pgfqpoint{0.838877in}{1.418477in}}%
\pgfpathcurveto{\pgfqpoint{0.846691in}{1.410663in}}{\pgfqpoint{0.857290in}{1.406273in}}{\pgfqpoint{0.868340in}{1.406273in}}%
\pgfpathclose%
\pgfusepath{stroke,fill}%
\end{pgfscope}%
\begin{pgfscope}%
\pgfpathrectangle{\pgfqpoint{0.374692in}{0.319877in}}{\pgfqpoint{1.579674in}{2.592703in}} %
\pgfusepath{clip}%
\pgfsetbuttcap%
\pgfsetroundjoin%
\definecolor{currentfill}{rgb}{1.000000,0.752941,0.796078}%
\pgfsetfillcolor{currentfill}%
\pgfsetlinewidth{1.003750pt}%
\definecolor{currentstroke}{rgb}{1.000000,0.752941,0.796078}%
\pgfsetstrokecolor{currentstroke}%
\pgfsetdash{}{0pt}%
\pgfpathmoveto{\pgfqpoint{1.065799in}{1.456625in}}%
\pgfpathcurveto{\pgfqpoint{1.076849in}{1.456625in}}{\pgfqpoint{1.087448in}{1.461016in}}{\pgfqpoint{1.095262in}{1.468829in}}%
\pgfpathcurveto{\pgfqpoint{1.103076in}{1.476643in}}{\pgfqpoint{1.107466in}{1.487242in}}{\pgfqpoint{1.107466in}{1.498292in}}%
\pgfpathcurveto{\pgfqpoint{1.107466in}{1.509342in}}{\pgfqpoint{1.103076in}{1.519941in}}{\pgfqpoint{1.095262in}{1.527755in}}%
\pgfpathcurveto{\pgfqpoint{1.087448in}{1.535569in}}{\pgfqpoint{1.076849in}{1.539959in}}{\pgfqpoint{1.065799in}{1.539959in}}%
\pgfpathcurveto{\pgfqpoint{1.054749in}{1.539959in}}{\pgfqpoint{1.044150in}{1.535569in}}{\pgfqpoint{1.036336in}{1.527755in}}%
\pgfpathcurveto{\pgfqpoint{1.028523in}{1.519941in}}{\pgfqpoint{1.024133in}{1.509342in}}{\pgfqpoint{1.024133in}{1.498292in}}%
\pgfpathcurveto{\pgfqpoint{1.024133in}{1.487242in}}{\pgfqpoint{1.028523in}{1.476643in}}{\pgfqpoint{1.036336in}{1.468829in}}%
\pgfpathcurveto{\pgfqpoint{1.044150in}{1.461016in}}{\pgfqpoint{1.054749in}{1.456625in}}{\pgfqpoint{1.065799in}{1.456625in}}%
\pgfpathclose%
\pgfusepath{stroke,fill}%
\end{pgfscope}%
\begin{pgfscope}%
\pgfpathrectangle{\pgfqpoint{0.374692in}{0.319877in}}{\pgfqpoint{1.579674in}{2.592703in}} %
\pgfusepath{clip}%
\pgfsetbuttcap%
\pgfsetroundjoin%
\definecolor{currentfill}{rgb}{1.000000,0.752941,0.796078}%
\pgfsetfillcolor{currentfill}%
\pgfsetlinewidth{1.003750pt}%
\definecolor{currentstroke}{rgb}{1.000000,0.752941,0.796078}%
\pgfsetstrokecolor{currentstroke}%
\pgfsetdash{}{0pt}%
\pgfpathmoveto{\pgfqpoint{1.263258in}{1.447830in}}%
\pgfpathcurveto{\pgfqpoint{1.274309in}{1.447830in}}{\pgfqpoint{1.284908in}{1.452220in}}{\pgfqpoint{1.292721in}{1.460034in}}%
\pgfpathcurveto{\pgfqpoint{1.300535in}{1.467848in}}{\pgfqpoint{1.304925in}{1.478447in}}{\pgfqpoint{1.304925in}{1.489497in}}%
\pgfpathcurveto{\pgfqpoint{1.304925in}{1.500547in}}{\pgfqpoint{1.300535in}{1.511146in}}{\pgfqpoint{1.292721in}{1.518960in}}%
\pgfpathcurveto{\pgfqpoint{1.284908in}{1.526773in}}{\pgfqpoint{1.274309in}{1.531164in}}{\pgfqpoint{1.263258in}{1.531164in}}%
\pgfpathcurveto{\pgfqpoint{1.252208in}{1.531164in}}{\pgfqpoint{1.241609in}{1.526773in}}{\pgfqpoint{1.233796in}{1.518960in}}%
\pgfpathcurveto{\pgfqpoint{1.225982in}{1.511146in}}{\pgfqpoint{1.221592in}{1.500547in}}{\pgfqpoint{1.221592in}{1.489497in}}%
\pgfpathcurveto{\pgfqpoint{1.221592in}{1.478447in}}{\pgfqpoint{1.225982in}{1.467848in}}{\pgfqpoint{1.233796in}{1.460034in}}%
\pgfpathcurveto{\pgfqpoint{1.241609in}{1.452220in}}{\pgfqpoint{1.252208in}{1.447830in}}{\pgfqpoint{1.263258in}{1.447830in}}%
\pgfpathclose%
\pgfusepath{stroke,fill}%
\end{pgfscope}%
\begin{pgfscope}%
\pgfpathrectangle{\pgfqpoint{0.374692in}{0.319877in}}{\pgfqpoint{1.579674in}{2.592703in}} %
\pgfusepath{clip}%
\pgfsetbuttcap%
\pgfsetroundjoin%
\definecolor{currentfill}{rgb}{1.000000,0.752941,0.796078}%
\pgfsetfillcolor{currentfill}%
\pgfsetlinewidth{1.003750pt}%
\definecolor{currentstroke}{rgb}{1.000000,0.752941,0.796078}%
\pgfsetstrokecolor{currentstroke}%
\pgfsetdash{}{0pt}%
\pgfpathmoveto{\pgfqpoint{1.460718in}{1.499208in}}%
\pgfpathcurveto{\pgfqpoint{1.471768in}{1.499208in}}{\pgfqpoint{1.482367in}{1.503598in}}{\pgfqpoint{1.490180in}{1.511412in}}%
\pgfpathcurveto{\pgfqpoint{1.497994in}{1.519225in}}{\pgfqpoint{1.502384in}{1.529824in}}{\pgfqpoint{1.502384in}{1.540874in}}%
\pgfpathcurveto{\pgfqpoint{1.502384in}{1.551924in}}{\pgfqpoint{1.497994in}{1.562523in}}{\pgfqpoint{1.490180in}{1.570337in}}%
\pgfpathcurveto{\pgfqpoint{1.482367in}{1.578151in}}{\pgfqpoint{1.471768in}{1.582541in}}{\pgfqpoint{1.460718in}{1.582541in}}%
\pgfpathcurveto{\pgfqpoint{1.449668in}{1.582541in}}{\pgfqpoint{1.439069in}{1.578151in}}{\pgfqpoint{1.431255in}{1.570337in}}%
\pgfpathcurveto{\pgfqpoint{1.423441in}{1.562523in}}{\pgfqpoint{1.419051in}{1.551924in}}{\pgfqpoint{1.419051in}{1.540874in}}%
\pgfpathcurveto{\pgfqpoint{1.419051in}{1.529824in}}{\pgfqpoint{1.423441in}{1.519225in}}{\pgfqpoint{1.431255in}{1.511412in}}%
\pgfpathcurveto{\pgfqpoint{1.439069in}{1.503598in}}{\pgfqpoint{1.449668in}{1.499208in}}{\pgfqpoint{1.460718in}{1.499208in}}%
\pgfpathclose%
\pgfusepath{stroke,fill}%
\end{pgfscope}%
\begin{pgfscope}%
\pgfpathrectangle{\pgfqpoint{0.374692in}{0.319877in}}{\pgfqpoint{1.579674in}{2.592703in}} %
\pgfusepath{clip}%
\pgfsetbuttcap%
\pgfsetroundjoin%
\definecolor{currentfill}{rgb}{1.000000,0.752941,0.796078}%
\pgfsetfillcolor{currentfill}%
\pgfsetlinewidth{1.003750pt}%
\definecolor{currentstroke}{rgb}{1.000000,0.752941,0.796078}%
\pgfsetstrokecolor{currentstroke}%
\pgfsetdash{}{0pt}%
\pgfpathmoveto{\pgfqpoint{1.658177in}{1.820422in}}%
\pgfpathcurveto{\pgfqpoint{1.669227in}{1.820422in}}{\pgfqpoint{1.679826in}{1.824812in}}{\pgfqpoint{1.687640in}{1.832626in}}%
\pgfpathcurveto{\pgfqpoint{1.695453in}{1.840439in}}{\pgfqpoint{1.699844in}{1.851038in}}{\pgfqpoint{1.699844in}{1.862088in}}%
\pgfpathcurveto{\pgfqpoint{1.699844in}{1.873139in}}{\pgfqpoint{1.695453in}{1.883738in}}{\pgfqpoint{1.687640in}{1.891551in}}%
\pgfpathcurveto{\pgfqpoint{1.679826in}{1.899365in}}{\pgfqpoint{1.669227in}{1.903755in}}{\pgfqpoint{1.658177in}{1.903755in}}%
\pgfpathcurveto{\pgfqpoint{1.647127in}{1.903755in}}{\pgfqpoint{1.636528in}{1.899365in}}{\pgfqpoint{1.628714in}{1.891551in}}%
\pgfpathcurveto{\pgfqpoint{1.620901in}{1.883738in}}{\pgfqpoint{1.616510in}{1.873139in}}{\pgfqpoint{1.616510in}{1.862088in}}%
\pgfpathcurveto{\pgfqpoint{1.616510in}{1.851038in}}{\pgfqpoint{1.620901in}{1.840439in}}{\pgfqpoint{1.628714in}{1.832626in}}%
\pgfpathcurveto{\pgfqpoint{1.636528in}{1.824812in}}{\pgfqpoint{1.647127in}{1.820422in}}{\pgfqpoint{1.658177in}{1.820422in}}%
\pgfpathclose%
\pgfusepath{stroke,fill}%
\end{pgfscope}%
\begin{pgfscope}%
\pgfpathrectangle{\pgfqpoint{0.374692in}{0.319877in}}{\pgfqpoint{1.579674in}{2.592703in}} %
\pgfusepath{clip}%
\pgfsetbuttcap%
\pgfsetroundjoin%
\definecolor{currentfill}{rgb}{1.000000,0.752941,0.796078}%
\pgfsetfillcolor{currentfill}%
\pgfsetlinewidth{1.003750pt}%
\definecolor{currentstroke}{rgb}{1.000000,0.752941,0.796078}%
\pgfsetstrokecolor{currentstroke}%
\pgfsetdash{}{0pt}%
\pgfpathmoveto{\pgfqpoint{1.855636in}{1.172246in}}%
\pgfpathcurveto{\pgfqpoint{1.866686in}{1.172246in}}{\pgfqpoint{1.877285in}{1.176636in}}{\pgfqpoint{1.885099in}{1.184450in}}%
\pgfpathcurveto{\pgfqpoint{1.892913in}{1.192264in}}{\pgfqpoint{1.897303in}{1.202863in}}{\pgfqpoint{1.897303in}{1.213913in}}%
\pgfpathcurveto{\pgfqpoint{1.897303in}{1.224963in}}{\pgfqpoint{1.892913in}{1.235562in}}{\pgfqpoint{1.885099in}{1.243376in}}%
\pgfpathcurveto{\pgfqpoint{1.877285in}{1.251189in}}{\pgfqpoint{1.866686in}{1.255579in}}{\pgfqpoint{1.855636in}{1.255579in}}%
\pgfpathcurveto{\pgfqpoint{1.844586in}{1.255579in}}{\pgfqpoint{1.833987in}{1.251189in}}{\pgfqpoint{1.826173in}{1.243376in}}%
\pgfpathcurveto{\pgfqpoint{1.818360in}{1.235562in}}{\pgfqpoint{1.813969in}{1.224963in}}{\pgfqpoint{1.813969in}{1.213913in}}%
\pgfpathcurveto{\pgfqpoint{1.813969in}{1.202863in}}{\pgfqpoint{1.818360in}{1.192264in}}{\pgfqpoint{1.826173in}{1.184450in}}%
\pgfpathcurveto{\pgfqpoint{1.833987in}{1.176636in}}{\pgfqpoint{1.844586in}{1.172246in}}{\pgfqpoint{1.855636in}{1.172246in}}%
\pgfpathclose%
\pgfusepath{stroke,fill}%
\end{pgfscope}%
\begin{pgfscope}%
\pgfsetbuttcap%
\pgfsetroundjoin%
\definecolor{currentfill}{rgb}{0.000000,0.000000,0.000000}%
\pgfsetfillcolor{currentfill}%
\pgfsetlinewidth{0.803000pt}%
\definecolor{currentstroke}{rgb}{0.000000,0.000000,0.000000}%
\pgfsetstrokecolor{currentstroke}%
\pgfsetdash{}{0pt}%
\pgfsys@defobject{currentmarker}{\pgfqpoint{0.000000in}{-0.048611in}}{\pgfqpoint{0.000000in}{0.000000in}}{%
\pgfpathmoveto{\pgfqpoint{0.000000in}{0.000000in}}%
\pgfpathlineto{\pgfqpoint{0.000000in}{-0.048611in}}%
\pgfusepath{stroke,fill}%
}%
\begin{pgfscope}%
\pgfsys@transformshift{0.670881in}{0.319877in}%
\pgfsys@useobject{currentmarker}{}%
\end{pgfscope}%
\end{pgfscope}%
\begin{pgfscope}%
\pgftext[x=0.670881in,y=0.222655in,,top]{\rmfamily\fontsize{10.000000}{12.000000}\selectfont \(\displaystyle -0.05\)}%
\end{pgfscope}%
\begin{pgfscope}%
\pgfsetbuttcap%
\pgfsetroundjoin%
\definecolor{currentfill}{rgb}{0.000000,0.000000,0.000000}%
\pgfsetfillcolor{currentfill}%
\pgfsetlinewidth{0.803000pt}%
\definecolor{currentstroke}{rgb}{0.000000,0.000000,0.000000}%
\pgfsetstrokecolor{currentstroke}%
\pgfsetdash{}{0pt}%
\pgfsys@defobject{currentmarker}{\pgfqpoint{0.000000in}{-0.048611in}}{\pgfqpoint{0.000000in}{0.000000in}}{%
\pgfpathmoveto{\pgfqpoint{0.000000in}{0.000000in}}%
\pgfpathlineto{\pgfqpoint{0.000000in}{-0.048611in}}%
\pgfusepath{stroke,fill}%
}%
\begin{pgfscope}%
\pgfsys@transformshift{1.164529in}{0.319877in}%
\pgfsys@useobject{currentmarker}{}%
\end{pgfscope}%
\end{pgfscope}%
\begin{pgfscope}%
\pgftext[x=1.164529in,y=0.222655in,,top]{\rmfamily\fontsize{10.000000}{12.000000}\selectfont \(\displaystyle 0.00\)}%
\end{pgfscope}%
\begin{pgfscope}%
\pgfsetbuttcap%
\pgfsetroundjoin%
\definecolor{currentfill}{rgb}{0.000000,0.000000,0.000000}%
\pgfsetfillcolor{currentfill}%
\pgfsetlinewidth{0.803000pt}%
\definecolor{currentstroke}{rgb}{0.000000,0.000000,0.000000}%
\pgfsetstrokecolor{currentstroke}%
\pgfsetdash{}{0pt}%
\pgfsys@defobject{currentmarker}{\pgfqpoint{0.000000in}{-0.048611in}}{\pgfqpoint{0.000000in}{0.000000in}}{%
\pgfpathmoveto{\pgfqpoint{0.000000in}{0.000000in}}%
\pgfpathlineto{\pgfqpoint{0.000000in}{-0.048611in}}%
\pgfusepath{stroke,fill}%
}%
\begin{pgfscope}%
\pgfsys@transformshift{1.658177in}{0.319877in}%
\pgfsys@useobject{currentmarker}{}%
\end{pgfscope}%
\end{pgfscope}%
\begin{pgfscope}%
\pgftext[x=1.658177in,y=0.222655in,,top]{\rmfamily\fontsize{10.000000}{12.000000}\selectfont \(\displaystyle 0.05\)}%
\end{pgfscope}%
\begin{pgfscope}%
\pgfsetbuttcap%
\pgfsetroundjoin%
\definecolor{currentfill}{rgb}{0.000000,0.000000,0.000000}%
\pgfsetfillcolor{currentfill}%
\pgfsetlinewidth{0.803000pt}%
\definecolor{currentstroke}{rgb}{0.000000,0.000000,0.000000}%
\pgfsetstrokecolor{currentstroke}%
\pgfsetdash{}{0pt}%
\pgfsys@defobject{currentmarker}{\pgfqpoint{-0.048611in}{0.000000in}}{\pgfqpoint{0.000000in}{0.000000in}}{%
\pgfpathmoveto{\pgfqpoint{0.000000in}{0.000000in}}%
\pgfpathlineto{\pgfqpoint{-0.048611in}{0.000000in}}%
\pgfusepath{stroke,fill}%
}%
\begin{pgfscope}%
\pgfsys@transformshift{0.374692in}{0.319877in}%
\pgfsys@useobject{currentmarker}{}%
\end{pgfscope}%
\end{pgfscope}%
\begin{pgfscope}%
\pgftext[x=0.100000in,y=0.272050in,left,base]{\rmfamily\fontsize{10.000000}{12.000000}\selectfont \(\displaystyle 0.0\)}%
\end{pgfscope}%
\begin{pgfscope}%
\pgfsetbuttcap%
\pgfsetroundjoin%
\definecolor{currentfill}{rgb}{0.000000,0.000000,0.000000}%
\pgfsetfillcolor{currentfill}%
\pgfsetlinewidth{0.803000pt}%
\definecolor{currentstroke}{rgb}{0.000000,0.000000,0.000000}%
\pgfsetstrokecolor{currentstroke}%
\pgfsetdash{}{0pt}%
\pgfsys@defobject{currentmarker}{\pgfqpoint{-0.048611in}{0.000000in}}{\pgfqpoint{0.000000in}{0.000000in}}{%
\pgfpathmoveto{\pgfqpoint{0.000000in}{0.000000in}}%
\pgfpathlineto{\pgfqpoint{-0.048611in}{0.000000in}}%
\pgfusepath{stroke,fill}%
}%
\begin{pgfscope}%
\pgfsys@transformshift{0.374692in}{0.838418in}%
\pgfsys@useobject{currentmarker}{}%
\end{pgfscope}%
\end{pgfscope}%
\begin{pgfscope}%
\pgftext[x=0.100000in,y=0.790590in,left,base]{\rmfamily\fontsize{10.000000}{12.000000}\selectfont \(\displaystyle 0.1\)}%
\end{pgfscope}%
\begin{pgfscope}%
\pgfsetbuttcap%
\pgfsetroundjoin%
\definecolor{currentfill}{rgb}{0.000000,0.000000,0.000000}%
\pgfsetfillcolor{currentfill}%
\pgfsetlinewidth{0.803000pt}%
\definecolor{currentstroke}{rgb}{0.000000,0.000000,0.000000}%
\pgfsetstrokecolor{currentstroke}%
\pgfsetdash{}{0pt}%
\pgfsys@defobject{currentmarker}{\pgfqpoint{-0.048611in}{0.000000in}}{\pgfqpoint{0.000000in}{0.000000in}}{%
\pgfpathmoveto{\pgfqpoint{0.000000in}{0.000000in}}%
\pgfpathlineto{\pgfqpoint{-0.048611in}{0.000000in}}%
\pgfusepath{stroke,fill}%
}%
\begin{pgfscope}%
\pgfsys@transformshift{0.374692in}{1.356958in}%
\pgfsys@useobject{currentmarker}{}%
\end{pgfscope}%
\end{pgfscope}%
\begin{pgfscope}%
\pgftext[x=0.100000in,y=1.309131in,left,base]{\rmfamily\fontsize{10.000000}{12.000000}\selectfont \(\displaystyle 0.2\)}%
\end{pgfscope}%
\begin{pgfscope}%
\pgfsetbuttcap%
\pgfsetroundjoin%
\definecolor{currentfill}{rgb}{0.000000,0.000000,0.000000}%
\pgfsetfillcolor{currentfill}%
\pgfsetlinewidth{0.803000pt}%
\definecolor{currentstroke}{rgb}{0.000000,0.000000,0.000000}%
\pgfsetstrokecolor{currentstroke}%
\pgfsetdash{}{0pt}%
\pgfsys@defobject{currentmarker}{\pgfqpoint{-0.048611in}{0.000000in}}{\pgfqpoint{0.000000in}{0.000000in}}{%
\pgfpathmoveto{\pgfqpoint{0.000000in}{0.000000in}}%
\pgfpathlineto{\pgfqpoint{-0.048611in}{0.000000in}}%
\pgfusepath{stroke,fill}%
}%
\begin{pgfscope}%
\pgfsys@transformshift{0.374692in}{1.875499in}%
\pgfsys@useobject{currentmarker}{}%
\end{pgfscope}%
\end{pgfscope}%
\begin{pgfscope}%
\pgftext[x=0.100000in,y=1.827671in,left,base]{\rmfamily\fontsize{10.000000}{12.000000}\selectfont \(\displaystyle 0.3\)}%
\end{pgfscope}%
\begin{pgfscope}%
\pgfsetbuttcap%
\pgfsetroundjoin%
\definecolor{currentfill}{rgb}{0.000000,0.000000,0.000000}%
\pgfsetfillcolor{currentfill}%
\pgfsetlinewidth{0.803000pt}%
\definecolor{currentstroke}{rgb}{0.000000,0.000000,0.000000}%
\pgfsetstrokecolor{currentstroke}%
\pgfsetdash{}{0pt}%
\pgfsys@defobject{currentmarker}{\pgfqpoint{-0.048611in}{0.000000in}}{\pgfqpoint{0.000000in}{0.000000in}}{%
\pgfpathmoveto{\pgfqpoint{0.000000in}{0.000000in}}%
\pgfpathlineto{\pgfqpoint{-0.048611in}{0.000000in}}%
\pgfusepath{stroke,fill}%
}%
\begin{pgfscope}%
\pgfsys@transformshift{0.374692in}{2.394040in}%
\pgfsys@useobject{currentmarker}{}%
\end{pgfscope}%
\end{pgfscope}%
\begin{pgfscope}%
\pgftext[x=0.100000in,y=2.346212in,left,base]{\rmfamily\fontsize{10.000000}{12.000000}\selectfont \(\displaystyle 0.4\)}%
\end{pgfscope}%
\begin{pgfscope}%
\pgfsetbuttcap%
\pgfsetroundjoin%
\definecolor{currentfill}{rgb}{0.000000,0.000000,0.000000}%
\pgfsetfillcolor{currentfill}%
\pgfsetlinewidth{0.803000pt}%
\definecolor{currentstroke}{rgb}{0.000000,0.000000,0.000000}%
\pgfsetstrokecolor{currentstroke}%
\pgfsetdash{}{0pt}%
\pgfsys@defobject{currentmarker}{\pgfqpoint{-0.048611in}{0.000000in}}{\pgfqpoint{0.000000in}{0.000000in}}{%
\pgfpathmoveto{\pgfqpoint{0.000000in}{0.000000in}}%
\pgfpathlineto{\pgfqpoint{-0.048611in}{0.000000in}}%
\pgfusepath{stroke,fill}%
}%
\begin{pgfscope}%
\pgfsys@transformshift{0.374692in}{2.912580in}%
\pgfsys@useobject{currentmarker}{}%
\end{pgfscope}%
\end{pgfscope}%
\begin{pgfscope}%
\pgftext[x=0.100000in,y=2.864752in,left,base]{\rmfamily\fontsize{10.000000}{12.000000}\selectfont \(\displaystyle 0.5\)}%
\end{pgfscope}%
\begin{pgfscope}%
\pgfsetrectcap%
\pgfsetmiterjoin%
\pgfsetlinewidth{0.803000pt}%
\definecolor{currentstroke}{rgb}{0.000000,0.000000,0.000000}%
\pgfsetstrokecolor{currentstroke}%
\pgfsetdash{}{0pt}%
\pgfpathmoveto{\pgfqpoint{0.374692in}{0.319877in}}%
\pgfpathlineto{\pgfqpoint{0.374692in}{2.912580in}}%
\pgfusepath{stroke}%
\end{pgfscope}%
\begin{pgfscope}%
\pgfsetrectcap%
\pgfsetmiterjoin%
\pgfsetlinewidth{0.803000pt}%
\definecolor{currentstroke}{rgb}{0.000000,0.000000,0.000000}%
\pgfsetstrokecolor{currentstroke}%
\pgfsetdash{}{0pt}%
\pgfpathmoveto{\pgfqpoint{1.954366in}{0.319877in}}%
\pgfpathlineto{\pgfqpoint{1.954366in}{2.912580in}}%
\pgfusepath{stroke}%
\end{pgfscope}%
\begin{pgfscope}%
\pgfsetrectcap%
\pgfsetmiterjoin%
\pgfsetlinewidth{0.803000pt}%
\definecolor{currentstroke}{rgb}{0.000000,0.000000,0.000000}%
\pgfsetstrokecolor{currentstroke}%
\pgfsetdash{}{0pt}%
\pgfpathmoveto{\pgfqpoint{0.374692in}{0.319877in}}%
\pgfpathlineto{\pgfqpoint{1.954366in}{0.319877in}}%
\pgfusepath{stroke}%
\end{pgfscope}%
\begin{pgfscope}%
\pgfsetrectcap%
\pgfsetmiterjoin%
\pgfsetlinewidth{0.803000pt}%
\definecolor{currentstroke}{rgb}{0.000000,0.000000,0.000000}%
\pgfsetstrokecolor{currentstroke}%
\pgfsetdash{}{0pt}%
\pgfpathmoveto{\pgfqpoint{0.374692in}{2.912580in}}%
\pgfpathlineto{\pgfqpoint{1.954366in}{2.912580in}}%
\pgfusepath{stroke}%
\end{pgfscope}%
\begin{pgfscope}%
\pgfpathrectangle{\pgfqpoint{2.053095in}{0.319877in}}{\pgfqpoint{0.129635in}{2.592703in}} %
\pgfusepath{clip}%
\pgfsetbuttcap%
\pgfsetmiterjoin%
\definecolor{currentfill}{rgb}{1.000000,1.000000,1.000000}%
\pgfsetfillcolor{currentfill}%
\pgfsetlinewidth{0.010037pt}%
\definecolor{currentstroke}{rgb}{1.000000,1.000000,1.000000}%
\pgfsetstrokecolor{currentstroke}%
\pgfsetdash{}{0pt}%
\pgfpathmoveto{\pgfqpoint{2.053095in}{0.319877in}}%
\pgfpathlineto{\pgfqpoint{2.053095in}{0.330005in}}%
\pgfpathlineto{\pgfqpoint{2.053095in}{2.902452in}}%
\pgfpathlineto{\pgfqpoint{2.053095in}{2.912580in}}%
\pgfpathlineto{\pgfqpoint{2.182730in}{2.912580in}}%
\pgfpathlineto{\pgfqpoint{2.182730in}{2.902452in}}%
\pgfpathlineto{\pgfqpoint{2.182730in}{0.330005in}}%
\pgfpathlineto{\pgfqpoint{2.182730in}{0.319877in}}%
\pgfpathclose%
\pgfusepath{stroke,fill}%
\end{pgfscope}%
\begin{pgfscope}%
\pgfsys@transformshift{2.050000in}{0.320408in}%
\pgftext[left,bottom]{\pgfimage[interpolate=true,width=0.130000in,height=2.590000in]{PerrNN_vs_dq_Ti_500K-img1.png}}%
\end{pgfscope}%
\begin{pgfscope}%
\pgfsetbuttcap%
\pgfsetroundjoin%
\definecolor{currentfill}{rgb}{0.000000,0.000000,0.000000}%
\pgfsetfillcolor{currentfill}%
\pgfsetlinewidth{0.803000pt}%
\definecolor{currentstroke}{rgb}{0.000000,0.000000,0.000000}%
\pgfsetstrokecolor{currentstroke}%
\pgfsetdash{}{0pt}%
\pgfsys@defobject{currentmarker}{\pgfqpoint{0.000000in}{0.000000in}}{\pgfqpoint{0.048611in}{0.000000in}}{%
\pgfpathmoveto{\pgfqpoint{0.000000in}{0.000000in}}%
\pgfpathlineto{\pgfqpoint{0.048611in}{0.000000in}}%
\pgfusepath{stroke,fill}%
}%
\begin{pgfscope}%
\pgfsys@transformshift{2.182730in}{0.319877in}%
\pgfsys@useobject{currentmarker}{}%
\end{pgfscope}%
\end{pgfscope}%
\begin{pgfscope}%
\pgftext[x=2.279953in,y=0.272050in,left,base]{\rmfamily\fontsize{10.000000}{12.000000}\selectfont \(\displaystyle 0\)}%
\end{pgfscope}%
\begin{pgfscope}%
\pgfsetbuttcap%
\pgfsetroundjoin%
\definecolor{currentfill}{rgb}{0.000000,0.000000,0.000000}%
\pgfsetfillcolor{currentfill}%
\pgfsetlinewidth{0.803000pt}%
\definecolor{currentstroke}{rgb}{0.000000,0.000000,0.000000}%
\pgfsetstrokecolor{currentstroke}%
\pgfsetdash{}{0pt}%
\pgfsys@defobject{currentmarker}{\pgfqpoint{0.000000in}{0.000000in}}{\pgfqpoint{0.048611in}{0.000000in}}{%
\pgfpathmoveto{\pgfqpoint{0.000000in}{0.000000in}}%
\pgfpathlineto{\pgfqpoint{0.048611in}{0.000000in}}%
\pgfusepath{stroke,fill}%
}%
\begin{pgfscope}%
\pgfsys@transformshift{2.182730in}{0.607955in}%
\pgfsys@useobject{currentmarker}{}%
\end{pgfscope}%
\end{pgfscope}%
\begin{pgfscope}%
\pgftext[x=2.279953in,y=0.560128in,left,base]{\rmfamily\fontsize{10.000000}{12.000000}\selectfont \(\displaystyle 2\)}%
\end{pgfscope}%
\begin{pgfscope}%
\pgfsetbuttcap%
\pgfsetroundjoin%
\definecolor{currentfill}{rgb}{0.000000,0.000000,0.000000}%
\pgfsetfillcolor{currentfill}%
\pgfsetlinewidth{0.803000pt}%
\definecolor{currentstroke}{rgb}{0.000000,0.000000,0.000000}%
\pgfsetstrokecolor{currentstroke}%
\pgfsetdash{}{0pt}%
\pgfsys@defobject{currentmarker}{\pgfqpoint{0.000000in}{0.000000in}}{\pgfqpoint{0.048611in}{0.000000in}}{%
\pgfpathmoveto{\pgfqpoint{0.000000in}{0.000000in}}%
\pgfpathlineto{\pgfqpoint{0.048611in}{0.000000in}}%
\pgfusepath{stroke,fill}%
}%
\begin{pgfscope}%
\pgfsys@transformshift{2.182730in}{0.896034in}%
\pgfsys@useobject{currentmarker}{}%
\end{pgfscope}%
\end{pgfscope}%
\begin{pgfscope}%
\pgftext[x=2.279953in,y=0.848206in,left,base]{\rmfamily\fontsize{10.000000}{12.000000}\selectfont \(\displaystyle 4\)}%
\end{pgfscope}%
\begin{pgfscope}%
\pgfsetbuttcap%
\pgfsetroundjoin%
\definecolor{currentfill}{rgb}{0.000000,0.000000,0.000000}%
\pgfsetfillcolor{currentfill}%
\pgfsetlinewidth{0.803000pt}%
\definecolor{currentstroke}{rgb}{0.000000,0.000000,0.000000}%
\pgfsetstrokecolor{currentstroke}%
\pgfsetdash{}{0pt}%
\pgfsys@defobject{currentmarker}{\pgfqpoint{0.000000in}{0.000000in}}{\pgfqpoint{0.048611in}{0.000000in}}{%
\pgfpathmoveto{\pgfqpoint{0.000000in}{0.000000in}}%
\pgfpathlineto{\pgfqpoint{0.048611in}{0.000000in}}%
\pgfusepath{stroke,fill}%
}%
\begin{pgfscope}%
\pgfsys@transformshift{2.182730in}{1.184112in}%
\pgfsys@useobject{currentmarker}{}%
\end{pgfscope}%
\end{pgfscope}%
\begin{pgfscope}%
\pgftext[x=2.279953in,y=1.136284in,left,base]{\rmfamily\fontsize{10.000000}{12.000000}\selectfont \(\displaystyle 6\)}%
\end{pgfscope}%
\begin{pgfscope}%
\pgfsetbuttcap%
\pgfsetroundjoin%
\definecolor{currentfill}{rgb}{0.000000,0.000000,0.000000}%
\pgfsetfillcolor{currentfill}%
\pgfsetlinewidth{0.803000pt}%
\definecolor{currentstroke}{rgb}{0.000000,0.000000,0.000000}%
\pgfsetstrokecolor{currentstroke}%
\pgfsetdash{}{0pt}%
\pgfsys@defobject{currentmarker}{\pgfqpoint{0.000000in}{0.000000in}}{\pgfqpoint{0.048611in}{0.000000in}}{%
\pgfpathmoveto{\pgfqpoint{0.000000in}{0.000000in}}%
\pgfpathlineto{\pgfqpoint{0.048611in}{0.000000in}}%
\pgfusepath{stroke,fill}%
}%
\begin{pgfscope}%
\pgfsys@transformshift{2.182730in}{1.472190in}%
\pgfsys@useobject{currentmarker}{}%
\end{pgfscope}%
\end{pgfscope}%
\begin{pgfscope}%
\pgftext[x=2.279953in,y=1.424362in,left,base]{\rmfamily\fontsize{10.000000}{12.000000}\selectfont \(\displaystyle 8\)}%
\end{pgfscope}%
\begin{pgfscope}%
\pgfsetbuttcap%
\pgfsetroundjoin%
\definecolor{currentfill}{rgb}{0.000000,0.000000,0.000000}%
\pgfsetfillcolor{currentfill}%
\pgfsetlinewidth{0.803000pt}%
\definecolor{currentstroke}{rgb}{0.000000,0.000000,0.000000}%
\pgfsetstrokecolor{currentstroke}%
\pgfsetdash{}{0pt}%
\pgfsys@defobject{currentmarker}{\pgfqpoint{0.000000in}{0.000000in}}{\pgfqpoint{0.048611in}{0.000000in}}{%
\pgfpathmoveto{\pgfqpoint{0.000000in}{0.000000in}}%
\pgfpathlineto{\pgfqpoint{0.048611in}{0.000000in}}%
\pgfusepath{stroke,fill}%
}%
\begin{pgfscope}%
\pgfsys@transformshift{2.182730in}{1.760268in}%
\pgfsys@useobject{currentmarker}{}%
\end{pgfscope}%
\end{pgfscope}%
\begin{pgfscope}%
\pgftext[x=2.279953in,y=1.712440in,left,base]{\rmfamily\fontsize{10.000000}{12.000000}\selectfont \(\displaystyle 10\)}%
\end{pgfscope}%
\begin{pgfscope}%
\pgfsetbuttcap%
\pgfsetroundjoin%
\definecolor{currentfill}{rgb}{0.000000,0.000000,0.000000}%
\pgfsetfillcolor{currentfill}%
\pgfsetlinewidth{0.803000pt}%
\definecolor{currentstroke}{rgb}{0.000000,0.000000,0.000000}%
\pgfsetstrokecolor{currentstroke}%
\pgfsetdash{}{0pt}%
\pgfsys@defobject{currentmarker}{\pgfqpoint{0.000000in}{0.000000in}}{\pgfqpoint{0.048611in}{0.000000in}}{%
\pgfpathmoveto{\pgfqpoint{0.000000in}{0.000000in}}%
\pgfpathlineto{\pgfqpoint{0.048611in}{0.000000in}}%
\pgfusepath{stroke,fill}%
}%
\begin{pgfscope}%
\pgfsys@transformshift{2.182730in}{2.048346in}%
\pgfsys@useobject{currentmarker}{}%
\end{pgfscope}%
\end{pgfscope}%
\begin{pgfscope}%
\pgftext[x=2.279953in,y=2.000518in,left,base]{\rmfamily\fontsize{10.000000}{12.000000}\selectfont \(\displaystyle 12\)}%
\end{pgfscope}%
\begin{pgfscope}%
\pgfsetbuttcap%
\pgfsetroundjoin%
\definecolor{currentfill}{rgb}{0.000000,0.000000,0.000000}%
\pgfsetfillcolor{currentfill}%
\pgfsetlinewidth{0.803000pt}%
\definecolor{currentstroke}{rgb}{0.000000,0.000000,0.000000}%
\pgfsetstrokecolor{currentstroke}%
\pgfsetdash{}{0pt}%
\pgfsys@defobject{currentmarker}{\pgfqpoint{0.000000in}{0.000000in}}{\pgfqpoint{0.048611in}{0.000000in}}{%
\pgfpathmoveto{\pgfqpoint{0.000000in}{0.000000in}}%
\pgfpathlineto{\pgfqpoint{0.048611in}{0.000000in}}%
\pgfusepath{stroke,fill}%
}%
\begin{pgfscope}%
\pgfsys@transformshift{2.182730in}{2.336424in}%
\pgfsys@useobject{currentmarker}{}%
\end{pgfscope}%
\end{pgfscope}%
\begin{pgfscope}%
\pgftext[x=2.279953in,y=2.288596in,left,base]{\rmfamily\fontsize{10.000000}{12.000000}\selectfont \(\displaystyle 14\)}%
\end{pgfscope}%
\begin{pgfscope}%
\pgfsetbuttcap%
\pgfsetroundjoin%
\definecolor{currentfill}{rgb}{0.000000,0.000000,0.000000}%
\pgfsetfillcolor{currentfill}%
\pgfsetlinewidth{0.803000pt}%
\definecolor{currentstroke}{rgb}{0.000000,0.000000,0.000000}%
\pgfsetstrokecolor{currentstroke}%
\pgfsetdash{}{0pt}%
\pgfsys@defobject{currentmarker}{\pgfqpoint{0.000000in}{0.000000in}}{\pgfqpoint{0.048611in}{0.000000in}}{%
\pgfpathmoveto{\pgfqpoint{0.000000in}{0.000000in}}%
\pgfpathlineto{\pgfqpoint{0.048611in}{0.000000in}}%
\pgfusepath{stroke,fill}%
}%
\begin{pgfscope}%
\pgfsys@transformshift{2.182730in}{2.624502in}%
\pgfsys@useobject{currentmarker}{}%
\end{pgfscope}%
\end{pgfscope}%
\begin{pgfscope}%
\pgftext[x=2.279953in,y=2.576674in,left,base]{\rmfamily\fontsize{10.000000}{12.000000}\selectfont \(\displaystyle 16\)}%
\end{pgfscope}%
\begin{pgfscope}%
\pgfsetbuttcap%
\pgfsetroundjoin%
\definecolor{currentfill}{rgb}{0.000000,0.000000,0.000000}%
\pgfsetfillcolor{currentfill}%
\pgfsetlinewidth{0.803000pt}%
\definecolor{currentstroke}{rgb}{0.000000,0.000000,0.000000}%
\pgfsetstrokecolor{currentstroke}%
\pgfsetdash{}{0pt}%
\pgfsys@defobject{currentmarker}{\pgfqpoint{0.000000in}{0.000000in}}{\pgfqpoint{0.048611in}{0.000000in}}{%
\pgfpathmoveto{\pgfqpoint{0.000000in}{0.000000in}}%
\pgfpathlineto{\pgfqpoint{0.048611in}{0.000000in}}%
\pgfusepath{stroke,fill}%
}%
\begin{pgfscope}%
\pgfsys@transformshift{2.182730in}{2.912580in}%
\pgfsys@useobject{currentmarker}{}%
\end{pgfscope}%
\end{pgfscope}%
\begin{pgfscope}%
\pgftext[x=2.279953in,y=2.864752in,left,base]{\rmfamily\fontsize{10.000000}{12.000000}\selectfont \(\displaystyle 18\)}%
\end{pgfscope}%
\begin{pgfscope}%
\pgfsetbuttcap%
\pgfsetmiterjoin%
\pgfsetlinewidth{0.803000pt}%
\definecolor{currentstroke}{rgb}{0.000000,0.000000,0.000000}%
\pgfsetstrokecolor{currentstroke}%
\pgfsetdash{}{0pt}%
\pgfpathmoveto{\pgfqpoint{2.053095in}{0.319877in}}%
\pgfpathlineto{\pgfqpoint{2.053095in}{0.330005in}}%
\pgfpathlineto{\pgfqpoint{2.053095in}{2.902452in}}%
\pgfpathlineto{\pgfqpoint{2.053095in}{2.912580in}}%
\pgfpathlineto{\pgfqpoint{2.182730in}{2.912580in}}%
\pgfpathlineto{\pgfqpoint{2.182730in}{2.902452in}}%
\pgfpathlineto{\pgfqpoint{2.182730in}{0.330005in}}%
\pgfpathlineto{\pgfqpoint{2.182730in}{0.319877in}}%
\pgfpathclose%
\pgfusepath{stroke}%
\end{pgfscope}%
\end{pgfpicture}%
\makeatother%
\endgroup%

    \vspace*{-0.4cm}
	\caption{500 K. Bin size $0.018e$}
	\end{subfigure}
	\quad
	\begin{subfigure}[b]{0.45\textwidth}
	\hspace*{-0.4cm}
	%% Creator: Matplotlib, PGF backend
%%
%% To include the figure in your LaTeX document, write
%%   \input{<filename>.pgf}
%%
%% Make sure the required packages are loaded in your preamble
%%   \usepackage{pgf}
%%
%% Figures using additional raster images can only be included by \input if
%% they are in the same directory as the main LaTeX file. For loading figures
%% from other directories you can use the `import` package
%%   \usepackage{import}
%% and then include the figures with
%%   \import{<path to file>}{<filename>.pgf}
%%
%% Matplotlib used the following preamble
%%   \usepackage[utf8x]{inputenc}
%%   \usepackage[T1]{fontenc}
%%
\begingroup%
\makeatletter%
\begin{pgfpicture}%
\pgfpathrectangle{\pgfpointorigin}{\pgfqpoint{2.518842in}{3.060408in}}%
\pgfusepath{use as bounding box, clip}%
\begin{pgfscope}%
\pgfsetbuttcap%
\pgfsetmiterjoin%
\definecolor{currentfill}{rgb}{1.000000,1.000000,1.000000}%
\pgfsetfillcolor{currentfill}%
\pgfsetlinewidth{0.000000pt}%
\definecolor{currentstroke}{rgb}{1.000000,1.000000,1.000000}%
\pgfsetstrokecolor{currentstroke}%
\pgfsetdash{}{0pt}%
\pgfpathmoveto{\pgfqpoint{0.000000in}{0.000000in}}%
\pgfpathlineto{\pgfqpoint{2.518842in}{0.000000in}}%
\pgfpathlineto{\pgfqpoint{2.518842in}{3.060408in}}%
\pgfpathlineto{\pgfqpoint{0.000000in}{3.060408in}}%
\pgfpathclose%
\pgfusepath{fill}%
\end{pgfscope}%
\begin{pgfscope}%
\pgfsetbuttcap%
\pgfsetmiterjoin%
\definecolor{currentfill}{rgb}{1.000000,1.000000,1.000000}%
\pgfsetfillcolor{currentfill}%
\pgfsetlinewidth{0.000000pt}%
\definecolor{currentstroke}{rgb}{0.000000,0.000000,0.000000}%
\pgfsetstrokecolor{currentstroke}%
\pgfsetstrokeopacity{0.000000}%
\pgfsetdash{}{0pt}%
\pgfpathmoveto{\pgfqpoint{0.374692in}{0.319877in}}%
\pgfpathlineto{\pgfqpoint{1.954366in}{0.319877in}}%
\pgfpathlineto{\pgfqpoint{1.954366in}{2.912580in}}%
\pgfpathlineto{\pgfqpoint{0.374692in}{2.912580in}}%
\pgfpathclose%
\pgfusepath{fill}%
\end{pgfscope}%
\begin{pgfscope}%
\pgfpathrectangle{\pgfqpoint{0.374692in}{0.319877in}}{\pgfqpoint{1.579674in}{2.592703in}} %
\pgfusepath{clip}%
\pgfsys@transformshift{0.374692in}{0.319877in}%
\pgftext[left,bottom]{\pgfimage[interpolate=true,width=1.580000in,height=2.600000in]{PerrNN_vs_dq_Ti_1000K-img0.png}}%
\end{pgfscope}%
\begin{pgfscope}%
\pgfpathrectangle{\pgfqpoint{0.374692in}{0.319877in}}{\pgfqpoint{1.579674in}{2.592703in}} %
\pgfusepath{clip}%
\pgfsetbuttcap%
\pgfsetroundjoin%
\definecolor{currentfill}{rgb}{1.000000,0.752941,0.796078}%
\pgfsetfillcolor{currentfill}%
\pgfsetlinewidth{1.003750pt}%
\definecolor{currentstroke}{rgb}{1.000000,0.752941,0.796078}%
\pgfsetstrokecolor{currentstroke}%
\pgfsetdash{}{0pt}%
\pgfpathmoveto{\pgfqpoint{0.473422in}{1.565622in}}%
\pgfpathcurveto{\pgfqpoint{0.484472in}{1.565622in}}{\pgfqpoint{0.495071in}{1.570012in}}{\pgfqpoint{0.502884in}{1.577826in}}%
\pgfpathcurveto{\pgfqpoint{0.510698in}{1.585639in}}{\pgfqpoint{0.515088in}{1.596238in}}{\pgfqpoint{0.515088in}{1.607288in}}%
\pgfpathcurveto{\pgfqpoint{0.515088in}{1.618338in}}{\pgfqpoint{0.510698in}{1.628938in}}{\pgfqpoint{0.502884in}{1.636751in}}%
\pgfpathcurveto{\pgfqpoint{0.495071in}{1.644565in}}{\pgfqpoint{0.484472in}{1.648955in}}{\pgfqpoint{0.473422in}{1.648955in}}%
\pgfpathcurveto{\pgfqpoint{0.462371in}{1.648955in}}{\pgfqpoint{0.451772in}{1.644565in}}{\pgfqpoint{0.443959in}{1.636751in}}%
\pgfpathcurveto{\pgfqpoint{0.436145in}{1.628938in}}{\pgfqpoint{0.431755in}{1.618338in}}{\pgfqpoint{0.431755in}{1.607288in}}%
\pgfpathcurveto{\pgfqpoint{0.431755in}{1.596238in}}{\pgfqpoint{0.436145in}{1.585639in}}{\pgfqpoint{0.443959in}{1.577826in}}%
\pgfpathcurveto{\pgfqpoint{0.451772in}{1.570012in}}{\pgfqpoint{0.462371in}{1.565622in}}{\pgfqpoint{0.473422in}{1.565622in}}%
\pgfpathclose%
\pgfusepath{stroke,fill}%
\end{pgfscope}%
\begin{pgfscope}%
\pgfpathrectangle{\pgfqpoint{0.374692in}{0.319877in}}{\pgfqpoint{1.579674in}{2.592703in}} %
\pgfusepath{clip}%
\pgfsetbuttcap%
\pgfsetroundjoin%
\definecolor{currentfill}{rgb}{1.000000,0.752941,0.796078}%
\pgfsetfillcolor{currentfill}%
\pgfsetlinewidth{1.003750pt}%
\definecolor{currentstroke}{rgb}{1.000000,0.752941,0.796078}%
\pgfsetstrokecolor{currentstroke}%
\pgfsetdash{}{0pt}%
\pgfpathmoveto{\pgfqpoint{0.670881in}{1.555404in}}%
\pgfpathcurveto{\pgfqpoint{0.681931in}{1.555404in}}{\pgfqpoint{0.692530in}{1.559794in}}{\pgfqpoint{0.700344in}{1.567608in}}%
\pgfpathcurveto{\pgfqpoint{0.708157in}{1.575422in}}{\pgfqpoint{0.712547in}{1.586021in}}{\pgfqpoint{0.712547in}{1.597071in}}%
\pgfpathcurveto{\pgfqpoint{0.712547in}{1.608121in}}{\pgfqpoint{0.708157in}{1.618720in}}{\pgfqpoint{0.700344in}{1.626534in}}%
\pgfpathcurveto{\pgfqpoint{0.692530in}{1.634347in}}{\pgfqpoint{0.681931in}{1.638737in}}{\pgfqpoint{0.670881in}{1.638737in}}%
\pgfpathcurveto{\pgfqpoint{0.659831in}{1.638737in}}{\pgfqpoint{0.649232in}{1.634347in}}{\pgfqpoint{0.641418in}{1.626534in}}%
\pgfpathcurveto{\pgfqpoint{0.633604in}{1.618720in}}{\pgfqpoint{0.629214in}{1.608121in}}{\pgfqpoint{0.629214in}{1.597071in}}%
\pgfpathcurveto{\pgfqpoint{0.629214in}{1.586021in}}{\pgfqpoint{0.633604in}{1.575422in}}{\pgfqpoint{0.641418in}{1.567608in}}%
\pgfpathcurveto{\pgfqpoint{0.649232in}{1.559794in}}{\pgfqpoint{0.659831in}{1.555404in}}{\pgfqpoint{0.670881in}{1.555404in}}%
\pgfpathclose%
\pgfusepath{stroke,fill}%
\end{pgfscope}%
\begin{pgfscope}%
\pgfpathrectangle{\pgfqpoint{0.374692in}{0.319877in}}{\pgfqpoint{1.579674in}{2.592703in}} %
\pgfusepath{clip}%
\pgfsetbuttcap%
\pgfsetroundjoin%
\definecolor{currentfill}{rgb}{1.000000,0.752941,0.796078}%
\pgfsetfillcolor{currentfill}%
\pgfsetlinewidth{1.003750pt}%
\definecolor{currentstroke}{rgb}{1.000000,0.752941,0.796078}%
\pgfsetstrokecolor{currentstroke}%
\pgfsetdash{}{0pt}%
\pgfpathmoveto{\pgfqpoint{0.868340in}{1.586422in}}%
\pgfpathcurveto{\pgfqpoint{0.879390in}{1.586422in}}{\pgfqpoint{0.889989in}{1.590812in}}{\pgfqpoint{0.897803in}{1.598626in}}%
\pgfpathcurveto{\pgfqpoint{0.905616in}{1.606439in}}{\pgfqpoint{0.910007in}{1.617038in}}{\pgfqpoint{0.910007in}{1.628088in}}%
\pgfpathcurveto{\pgfqpoint{0.910007in}{1.639139in}}{\pgfqpoint{0.905616in}{1.649738in}}{\pgfqpoint{0.897803in}{1.657551in}}%
\pgfpathcurveto{\pgfqpoint{0.889989in}{1.665365in}}{\pgfqpoint{0.879390in}{1.669755in}}{\pgfqpoint{0.868340in}{1.669755in}}%
\pgfpathcurveto{\pgfqpoint{0.857290in}{1.669755in}}{\pgfqpoint{0.846691in}{1.665365in}}{\pgfqpoint{0.838877in}{1.657551in}}%
\pgfpathcurveto{\pgfqpoint{0.831064in}{1.649738in}}{\pgfqpoint{0.826673in}{1.639139in}}{\pgfqpoint{0.826673in}{1.628088in}}%
\pgfpathcurveto{\pgfqpoint{0.826673in}{1.617038in}}{\pgfqpoint{0.831064in}{1.606439in}}{\pgfqpoint{0.838877in}{1.598626in}}%
\pgfpathcurveto{\pgfqpoint{0.846691in}{1.590812in}}{\pgfqpoint{0.857290in}{1.586422in}}{\pgfqpoint{0.868340in}{1.586422in}}%
\pgfpathclose%
\pgfusepath{stroke,fill}%
\end{pgfscope}%
\begin{pgfscope}%
\pgfpathrectangle{\pgfqpoint{0.374692in}{0.319877in}}{\pgfqpoint{1.579674in}{2.592703in}} %
\pgfusepath{clip}%
\pgfsetbuttcap%
\pgfsetroundjoin%
\definecolor{currentfill}{rgb}{1.000000,0.752941,0.796078}%
\pgfsetfillcolor{currentfill}%
\pgfsetlinewidth{1.003750pt}%
\definecolor{currentstroke}{rgb}{1.000000,0.752941,0.796078}%
\pgfsetstrokecolor{currentstroke}%
\pgfsetdash{}{0pt}%
\pgfpathmoveto{\pgfqpoint{1.065799in}{1.599076in}}%
\pgfpathcurveto{\pgfqpoint{1.076849in}{1.599076in}}{\pgfqpoint{1.087448in}{1.603466in}}{\pgfqpoint{1.095262in}{1.611280in}}%
\pgfpathcurveto{\pgfqpoint{1.103076in}{1.619093in}}{\pgfqpoint{1.107466in}{1.629692in}}{\pgfqpoint{1.107466in}{1.640743in}}%
\pgfpathcurveto{\pgfqpoint{1.107466in}{1.651793in}}{\pgfqpoint{1.103076in}{1.662392in}}{\pgfqpoint{1.095262in}{1.670205in}}%
\pgfpathcurveto{\pgfqpoint{1.087448in}{1.678019in}}{\pgfqpoint{1.076849in}{1.682409in}}{\pgfqpoint{1.065799in}{1.682409in}}%
\pgfpathcurveto{\pgfqpoint{1.054749in}{1.682409in}}{\pgfqpoint{1.044150in}{1.678019in}}{\pgfqpoint{1.036336in}{1.670205in}}%
\pgfpathcurveto{\pgfqpoint{1.028523in}{1.662392in}}{\pgfqpoint{1.024133in}{1.651793in}}{\pgfqpoint{1.024133in}{1.640743in}}%
\pgfpathcurveto{\pgfqpoint{1.024133in}{1.629692in}}{\pgfqpoint{1.028523in}{1.619093in}}{\pgfqpoint{1.036336in}{1.611280in}}%
\pgfpathcurveto{\pgfqpoint{1.044150in}{1.603466in}}{\pgfqpoint{1.054749in}{1.599076in}}{\pgfqpoint{1.065799in}{1.599076in}}%
\pgfpathclose%
\pgfusepath{stroke,fill}%
\end{pgfscope}%
\begin{pgfscope}%
\pgfpathrectangle{\pgfqpoint{0.374692in}{0.319877in}}{\pgfqpoint{1.579674in}{2.592703in}} %
\pgfusepath{clip}%
\pgfsetbuttcap%
\pgfsetroundjoin%
\definecolor{currentfill}{rgb}{1.000000,0.752941,0.796078}%
\pgfsetfillcolor{currentfill}%
\pgfsetlinewidth{1.003750pt}%
\definecolor{currentstroke}{rgb}{1.000000,0.752941,0.796078}%
\pgfsetstrokecolor{currentstroke}%
\pgfsetdash{}{0pt}%
\pgfpathmoveto{\pgfqpoint{1.263258in}{1.615735in}}%
\pgfpathcurveto{\pgfqpoint{1.274309in}{1.615735in}}{\pgfqpoint{1.284908in}{1.620125in}}{\pgfqpoint{1.292721in}{1.627939in}}%
\pgfpathcurveto{\pgfqpoint{1.300535in}{1.635752in}}{\pgfqpoint{1.304925in}{1.646351in}}{\pgfqpoint{1.304925in}{1.657401in}}%
\pgfpathcurveto{\pgfqpoint{1.304925in}{1.668452in}}{\pgfqpoint{1.300535in}{1.679051in}}{\pgfqpoint{1.292721in}{1.686864in}}%
\pgfpathcurveto{\pgfqpoint{1.284908in}{1.694678in}}{\pgfqpoint{1.274309in}{1.699068in}}{\pgfqpoint{1.263258in}{1.699068in}}%
\pgfpathcurveto{\pgfqpoint{1.252208in}{1.699068in}}{\pgfqpoint{1.241609in}{1.694678in}}{\pgfqpoint{1.233796in}{1.686864in}}%
\pgfpathcurveto{\pgfqpoint{1.225982in}{1.679051in}}{\pgfqpoint{1.221592in}{1.668452in}}{\pgfqpoint{1.221592in}{1.657401in}}%
\pgfpathcurveto{\pgfqpoint{1.221592in}{1.646351in}}{\pgfqpoint{1.225982in}{1.635752in}}{\pgfqpoint{1.233796in}{1.627939in}}%
\pgfpathcurveto{\pgfqpoint{1.241609in}{1.620125in}}{\pgfqpoint{1.252208in}{1.615735in}}{\pgfqpoint{1.263258in}{1.615735in}}%
\pgfpathclose%
\pgfusepath{stroke,fill}%
\end{pgfscope}%
\begin{pgfscope}%
\pgfpathrectangle{\pgfqpoint{0.374692in}{0.319877in}}{\pgfqpoint{1.579674in}{2.592703in}} %
\pgfusepath{clip}%
\pgfsetbuttcap%
\pgfsetroundjoin%
\definecolor{currentfill}{rgb}{1.000000,0.752941,0.796078}%
\pgfsetfillcolor{currentfill}%
\pgfsetlinewidth{1.003750pt}%
\definecolor{currentstroke}{rgb}{1.000000,0.752941,0.796078}%
\pgfsetstrokecolor{currentstroke}%
\pgfsetdash{}{0pt}%
\pgfpathmoveto{\pgfqpoint{1.460718in}{1.613026in}}%
\pgfpathcurveto{\pgfqpoint{1.471768in}{1.613026in}}{\pgfqpoint{1.482367in}{1.617417in}}{\pgfqpoint{1.490180in}{1.625230in}}%
\pgfpathcurveto{\pgfqpoint{1.497994in}{1.633044in}}{\pgfqpoint{1.502384in}{1.643643in}}{\pgfqpoint{1.502384in}{1.654693in}}%
\pgfpathcurveto{\pgfqpoint{1.502384in}{1.665743in}}{\pgfqpoint{1.497994in}{1.676342in}}{\pgfqpoint{1.490180in}{1.684156in}}%
\pgfpathcurveto{\pgfqpoint{1.482367in}{1.691969in}}{\pgfqpoint{1.471768in}{1.696360in}}{\pgfqpoint{1.460718in}{1.696360in}}%
\pgfpathcurveto{\pgfqpoint{1.449668in}{1.696360in}}{\pgfqpoint{1.439069in}{1.691969in}}{\pgfqpoint{1.431255in}{1.684156in}}%
\pgfpathcurveto{\pgfqpoint{1.423441in}{1.676342in}}{\pgfqpoint{1.419051in}{1.665743in}}{\pgfqpoint{1.419051in}{1.654693in}}%
\pgfpathcurveto{\pgfqpoint{1.419051in}{1.643643in}}{\pgfqpoint{1.423441in}{1.633044in}}{\pgfqpoint{1.431255in}{1.625230in}}%
\pgfpathcurveto{\pgfqpoint{1.439069in}{1.617417in}}{\pgfqpoint{1.449668in}{1.613026in}}{\pgfqpoint{1.460718in}{1.613026in}}%
\pgfpathclose%
\pgfusepath{stroke,fill}%
\end{pgfscope}%
\begin{pgfscope}%
\pgfpathrectangle{\pgfqpoint{0.374692in}{0.319877in}}{\pgfqpoint{1.579674in}{2.592703in}} %
\pgfusepath{clip}%
\pgfsetbuttcap%
\pgfsetroundjoin%
\definecolor{currentfill}{rgb}{1.000000,0.752941,0.796078}%
\pgfsetfillcolor{currentfill}%
\pgfsetlinewidth{1.003750pt}%
\definecolor{currentstroke}{rgb}{1.000000,0.752941,0.796078}%
\pgfsetstrokecolor{currentstroke}%
\pgfsetdash{}{0pt}%
\pgfpathmoveto{\pgfqpoint{1.658177in}{1.859536in}}%
\pgfpathcurveto{\pgfqpoint{1.669227in}{1.859536in}}{\pgfqpoint{1.679826in}{1.863926in}}{\pgfqpoint{1.687640in}{1.871740in}}%
\pgfpathcurveto{\pgfqpoint{1.695453in}{1.879553in}}{\pgfqpoint{1.699844in}{1.890152in}}{\pgfqpoint{1.699844in}{1.901203in}}%
\pgfpathcurveto{\pgfqpoint{1.699844in}{1.912253in}}{\pgfqpoint{1.695453in}{1.922852in}}{\pgfqpoint{1.687640in}{1.930665in}}%
\pgfpathcurveto{\pgfqpoint{1.679826in}{1.938479in}}{\pgfqpoint{1.669227in}{1.942869in}}{\pgfqpoint{1.658177in}{1.942869in}}%
\pgfpathcurveto{\pgfqpoint{1.647127in}{1.942869in}}{\pgfqpoint{1.636528in}{1.938479in}}{\pgfqpoint{1.628714in}{1.930665in}}%
\pgfpathcurveto{\pgfqpoint{1.620901in}{1.922852in}}{\pgfqpoint{1.616510in}{1.912253in}}{\pgfqpoint{1.616510in}{1.901203in}}%
\pgfpathcurveto{\pgfqpoint{1.616510in}{1.890152in}}{\pgfqpoint{1.620901in}{1.879553in}}{\pgfqpoint{1.628714in}{1.871740in}}%
\pgfpathcurveto{\pgfqpoint{1.636528in}{1.863926in}}{\pgfqpoint{1.647127in}{1.859536in}}{\pgfqpoint{1.658177in}{1.859536in}}%
\pgfpathclose%
\pgfusepath{stroke,fill}%
\end{pgfscope}%
\begin{pgfscope}%
\pgfpathrectangle{\pgfqpoint{0.374692in}{0.319877in}}{\pgfqpoint{1.579674in}{2.592703in}} %
\pgfusepath{clip}%
\pgfsetbuttcap%
\pgfsetroundjoin%
\definecolor{currentfill}{rgb}{1.000000,0.752941,0.796078}%
\pgfsetfillcolor{currentfill}%
\pgfsetlinewidth{1.003750pt}%
\definecolor{currentstroke}{rgb}{1.000000,0.752941,0.796078}%
\pgfsetstrokecolor{currentstroke}%
\pgfsetdash{}{0pt}%
\pgfpathmoveto{\pgfqpoint{1.855636in}{1.999229in}}%
\pgfpathcurveto{\pgfqpoint{1.866686in}{1.999229in}}{\pgfqpoint{1.877285in}{2.003619in}}{\pgfqpoint{1.885099in}{2.011433in}}%
\pgfpathcurveto{\pgfqpoint{1.892913in}{2.019246in}}{\pgfqpoint{1.897303in}{2.029845in}}{\pgfqpoint{1.897303in}{2.040896in}}%
\pgfpathcurveto{\pgfqpoint{1.897303in}{2.051946in}}{\pgfqpoint{1.892913in}{2.062545in}}{\pgfqpoint{1.885099in}{2.070358in}}%
\pgfpathcurveto{\pgfqpoint{1.877285in}{2.078172in}}{\pgfqpoint{1.866686in}{2.082562in}}{\pgfqpoint{1.855636in}{2.082562in}}%
\pgfpathcurveto{\pgfqpoint{1.844586in}{2.082562in}}{\pgfqpoint{1.833987in}{2.078172in}}{\pgfqpoint{1.826173in}{2.070358in}}%
\pgfpathcurveto{\pgfqpoint{1.818360in}{2.062545in}}{\pgfqpoint{1.813969in}{2.051946in}}{\pgfqpoint{1.813969in}{2.040896in}}%
\pgfpathcurveto{\pgfqpoint{1.813969in}{2.029845in}}{\pgfqpoint{1.818360in}{2.019246in}}{\pgfqpoint{1.826173in}{2.011433in}}%
\pgfpathcurveto{\pgfqpoint{1.833987in}{2.003619in}}{\pgfqpoint{1.844586in}{1.999229in}}{\pgfqpoint{1.855636in}{1.999229in}}%
\pgfpathclose%
\pgfusepath{stroke,fill}%
\end{pgfscope}%
\begin{pgfscope}%
\pgfsetbuttcap%
\pgfsetroundjoin%
\definecolor{currentfill}{rgb}{0.000000,0.000000,0.000000}%
\pgfsetfillcolor{currentfill}%
\pgfsetlinewidth{0.803000pt}%
\definecolor{currentstroke}{rgb}{0.000000,0.000000,0.000000}%
\pgfsetstrokecolor{currentstroke}%
\pgfsetdash{}{0pt}%
\pgfsys@defobject{currentmarker}{\pgfqpoint{0.000000in}{-0.048611in}}{\pgfqpoint{0.000000in}{0.000000in}}{%
\pgfpathmoveto{\pgfqpoint{0.000000in}{0.000000in}}%
\pgfpathlineto{\pgfqpoint{0.000000in}{-0.048611in}}%
\pgfusepath{stroke,fill}%
}%
\begin{pgfscope}%
\pgfsys@transformshift{0.670881in}{0.319877in}%
\pgfsys@useobject{currentmarker}{}%
\end{pgfscope}%
\end{pgfscope}%
\begin{pgfscope}%
\pgftext[x=0.670881in,y=0.222655in,,top]{\rmfamily\fontsize{10.000000}{12.000000}\selectfont \(\displaystyle -0.05\)}%
\end{pgfscope}%
\begin{pgfscope}%
\pgfsetbuttcap%
\pgfsetroundjoin%
\definecolor{currentfill}{rgb}{0.000000,0.000000,0.000000}%
\pgfsetfillcolor{currentfill}%
\pgfsetlinewidth{0.803000pt}%
\definecolor{currentstroke}{rgb}{0.000000,0.000000,0.000000}%
\pgfsetstrokecolor{currentstroke}%
\pgfsetdash{}{0pt}%
\pgfsys@defobject{currentmarker}{\pgfqpoint{0.000000in}{-0.048611in}}{\pgfqpoint{0.000000in}{0.000000in}}{%
\pgfpathmoveto{\pgfqpoint{0.000000in}{0.000000in}}%
\pgfpathlineto{\pgfqpoint{0.000000in}{-0.048611in}}%
\pgfusepath{stroke,fill}%
}%
\begin{pgfscope}%
\pgfsys@transformshift{1.164529in}{0.319877in}%
\pgfsys@useobject{currentmarker}{}%
\end{pgfscope}%
\end{pgfscope}%
\begin{pgfscope}%
\pgftext[x=1.164529in,y=0.222655in,,top]{\rmfamily\fontsize{10.000000}{12.000000}\selectfont \(\displaystyle 0.00\)}%
\end{pgfscope}%
\begin{pgfscope}%
\pgfsetbuttcap%
\pgfsetroundjoin%
\definecolor{currentfill}{rgb}{0.000000,0.000000,0.000000}%
\pgfsetfillcolor{currentfill}%
\pgfsetlinewidth{0.803000pt}%
\definecolor{currentstroke}{rgb}{0.000000,0.000000,0.000000}%
\pgfsetstrokecolor{currentstroke}%
\pgfsetdash{}{0pt}%
\pgfsys@defobject{currentmarker}{\pgfqpoint{0.000000in}{-0.048611in}}{\pgfqpoint{0.000000in}{0.000000in}}{%
\pgfpathmoveto{\pgfqpoint{0.000000in}{0.000000in}}%
\pgfpathlineto{\pgfqpoint{0.000000in}{-0.048611in}}%
\pgfusepath{stroke,fill}%
}%
\begin{pgfscope}%
\pgfsys@transformshift{1.658177in}{0.319877in}%
\pgfsys@useobject{currentmarker}{}%
\end{pgfscope}%
\end{pgfscope}%
\begin{pgfscope}%
\pgftext[x=1.658177in,y=0.222655in,,top]{\rmfamily\fontsize{10.000000}{12.000000}\selectfont \(\displaystyle 0.05\)}%
\end{pgfscope}%
\begin{pgfscope}%
\pgfsetbuttcap%
\pgfsetroundjoin%
\definecolor{currentfill}{rgb}{0.000000,0.000000,0.000000}%
\pgfsetfillcolor{currentfill}%
\pgfsetlinewidth{0.803000pt}%
\definecolor{currentstroke}{rgb}{0.000000,0.000000,0.000000}%
\pgfsetstrokecolor{currentstroke}%
\pgfsetdash{}{0pt}%
\pgfsys@defobject{currentmarker}{\pgfqpoint{-0.048611in}{0.000000in}}{\pgfqpoint{0.000000in}{0.000000in}}{%
\pgfpathmoveto{\pgfqpoint{0.000000in}{0.000000in}}%
\pgfpathlineto{\pgfqpoint{-0.048611in}{0.000000in}}%
\pgfusepath{stroke,fill}%
}%
\begin{pgfscope}%
\pgfsys@transformshift{0.374692in}{0.319877in}%
\pgfsys@useobject{currentmarker}{}%
\end{pgfscope}%
\end{pgfscope}%
\begin{pgfscope}%
\pgftext[x=0.100000in,y=0.272050in,left,base]{\rmfamily\fontsize{10.000000}{12.000000}\selectfont \(\displaystyle 0.0\)}%
\end{pgfscope}%
\begin{pgfscope}%
\pgfsetbuttcap%
\pgfsetroundjoin%
\definecolor{currentfill}{rgb}{0.000000,0.000000,0.000000}%
\pgfsetfillcolor{currentfill}%
\pgfsetlinewidth{0.803000pt}%
\definecolor{currentstroke}{rgb}{0.000000,0.000000,0.000000}%
\pgfsetstrokecolor{currentstroke}%
\pgfsetdash{}{0pt}%
\pgfsys@defobject{currentmarker}{\pgfqpoint{-0.048611in}{0.000000in}}{\pgfqpoint{0.000000in}{0.000000in}}{%
\pgfpathmoveto{\pgfqpoint{0.000000in}{0.000000in}}%
\pgfpathlineto{\pgfqpoint{-0.048611in}{0.000000in}}%
\pgfusepath{stroke,fill}%
}%
\begin{pgfscope}%
\pgfsys@transformshift{0.374692in}{0.838418in}%
\pgfsys@useobject{currentmarker}{}%
\end{pgfscope}%
\end{pgfscope}%
\begin{pgfscope}%
\pgftext[x=0.100000in,y=0.790590in,left,base]{\rmfamily\fontsize{10.000000}{12.000000}\selectfont \(\displaystyle 0.1\)}%
\end{pgfscope}%
\begin{pgfscope}%
\pgfsetbuttcap%
\pgfsetroundjoin%
\definecolor{currentfill}{rgb}{0.000000,0.000000,0.000000}%
\pgfsetfillcolor{currentfill}%
\pgfsetlinewidth{0.803000pt}%
\definecolor{currentstroke}{rgb}{0.000000,0.000000,0.000000}%
\pgfsetstrokecolor{currentstroke}%
\pgfsetdash{}{0pt}%
\pgfsys@defobject{currentmarker}{\pgfqpoint{-0.048611in}{0.000000in}}{\pgfqpoint{0.000000in}{0.000000in}}{%
\pgfpathmoveto{\pgfqpoint{0.000000in}{0.000000in}}%
\pgfpathlineto{\pgfqpoint{-0.048611in}{0.000000in}}%
\pgfusepath{stroke,fill}%
}%
\begin{pgfscope}%
\pgfsys@transformshift{0.374692in}{1.356958in}%
\pgfsys@useobject{currentmarker}{}%
\end{pgfscope}%
\end{pgfscope}%
\begin{pgfscope}%
\pgftext[x=0.100000in,y=1.309131in,left,base]{\rmfamily\fontsize{10.000000}{12.000000}\selectfont \(\displaystyle 0.2\)}%
\end{pgfscope}%
\begin{pgfscope}%
\pgfsetbuttcap%
\pgfsetroundjoin%
\definecolor{currentfill}{rgb}{0.000000,0.000000,0.000000}%
\pgfsetfillcolor{currentfill}%
\pgfsetlinewidth{0.803000pt}%
\definecolor{currentstroke}{rgb}{0.000000,0.000000,0.000000}%
\pgfsetstrokecolor{currentstroke}%
\pgfsetdash{}{0pt}%
\pgfsys@defobject{currentmarker}{\pgfqpoint{-0.048611in}{0.000000in}}{\pgfqpoint{0.000000in}{0.000000in}}{%
\pgfpathmoveto{\pgfqpoint{0.000000in}{0.000000in}}%
\pgfpathlineto{\pgfqpoint{-0.048611in}{0.000000in}}%
\pgfusepath{stroke,fill}%
}%
\begin{pgfscope}%
\pgfsys@transformshift{0.374692in}{1.875499in}%
\pgfsys@useobject{currentmarker}{}%
\end{pgfscope}%
\end{pgfscope}%
\begin{pgfscope}%
\pgftext[x=0.100000in,y=1.827671in,left,base]{\rmfamily\fontsize{10.000000}{12.000000}\selectfont \(\displaystyle 0.3\)}%
\end{pgfscope}%
\begin{pgfscope}%
\pgfsetbuttcap%
\pgfsetroundjoin%
\definecolor{currentfill}{rgb}{0.000000,0.000000,0.000000}%
\pgfsetfillcolor{currentfill}%
\pgfsetlinewidth{0.803000pt}%
\definecolor{currentstroke}{rgb}{0.000000,0.000000,0.000000}%
\pgfsetstrokecolor{currentstroke}%
\pgfsetdash{}{0pt}%
\pgfsys@defobject{currentmarker}{\pgfqpoint{-0.048611in}{0.000000in}}{\pgfqpoint{0.000000in}{0.000000in}}{%
\pgfpathmoveto{\pgfqpoint{0.000000in}{0.000000in}}%
\pgfpathlineto{\pgfqpoint{-0.048611in}{0.000000in}}%
\pgfusepath{stroke,fill}%
}%
\begin{pgfscope}%
\pgfsys@transformshift{0.374692in}{2.394040in}%
\pgfsys@useobject{currentmarker}{}%
\end{pgfscope}%
\end{pgfscope}%
\begin{pgfscope}%
\pgftext[x=0.100000in,y=2.346212in,left,base]{\rmfamily\fontsize{10.000000}{12.000000}\selectfont \(\displaystyle 0.4\)}%
\end{pgfscope}%
\begin{pgfscope}%
\pgfsetbuttcap%
\pgfsetroundjoin%
\definecolor{currentfill}{rgb}{0.000000,0.000000,0.000000}%
\pgfsetfillcolor{currentfill}%
\pgfsetlinewidth{0.803000pt}%
\definecolor{currentstroke}{rgb}{0.000000,0.000000,0.000000}%
\pgfsetstrokecolor{currentstroke}%
\pgfsetdash{}{0pt}%
\pgfsys@defobject{currentmarker}{\pgfqpoint{-0.048611in}{0.000000in}}{\pgfqpoint{0.000000in}{0.000000in}}{%
\pgfpathmoveto{\pgfqpoint{0.000000in}{0.000000in}}%
\pgfpathlineto{\pgfqpoint{-0.048611in}{0.000000in}}%
\pgfusepath{stroke,fill}%
}%
\begin{pgfscope}%
\pgfsys@transformshift{0.374692in}{2.912580in}%
\pgfsys@useobject{currentmarker}{}%
\end{pgfscope}%
\end{pgfscope}%
\begin{pgfscope}%
\pgftext[x=0.100000in,y=2.864752in,left,base]{\rmfamily\fontsize{10.000000}{12.000000}\selectfont \(\displaystyle 0.5\)}%
\end{pgfscope}%
\begin{pgfscope}%
\pgfsetrectcap%
\pgfsetmiterjoin%
\pgfsetlinewidth{0.803000pt}%
\definecolor{currentstroke}{rgb}{0.000000,0.000000,0.000000}%
\pgfsetstrokecolor{currentstroke}%
\pgfsetdash{}{0pt}%
\pgfpathmoveto{\pgfqpoint{0.374692in}{0.319877in}}%
\pgfpathlineto{\pgfqpoint{0.374692in}{2.912580in}}%
\pgfusepath{stroke}%
\end{pgfscope}%
\begin{pgfscope}%
\pgfsetrectcap%
\pgfsetmiterjoin%
\pgfsetlinewidth{0.803000pt}%
\definecolor{currentstroke}{rgb}{0.000000,0.000000,0.000000}%
\pgfsetstrokecolor{currentstroke}%
\pgfsetdash{}{0pt}%
\pgfpathmoveto{\pgfqpoint{1.954366in}{0.319877in}}%
\pgfpathlineto{\pgfqpoint{1.954366in}{2.912580in}}%
\pgfusepath{stroke}%
\end{pgfscope}%
\begin{pgfscope}%
\pgfsetrectcap%
\pgfsetmiterjoin%
\pgfsetlinewidth{0.803000pt}%
\definecolor{currentstroke}{rgb}{0.000000,0.000000,0.000000}%
\pgfsetstrokecolor{currentstroke}%
\pgfsetdash{}{0pt}%
\pgfpathmoveto{\pgfqpoint{0.374692in}{0.319877in}}%
\pgfpathlineto{\pgfqpoint{1.954366in}{0.319877in}}%
\pgfusepath{stroke}%
\end{pgfscope}%
\begin{pgfscope}%
\pgfsetrectcap%
\pgfsetmiterjoin%
\pgfsetlinewidth{0.803000pt}%
\definecolor{currentstroke}{rgb}{0.000000,0.000000,0.000000}%
\pgfsetstrokecolor{currentstroke}%
\pgfsetdash{}{0pt}%
\pgfpathmoveto{\pgfqpoint{0.374692in}{2.912580in}}%
\pgfpathlineto{\pgfqpoint{1.954366in}{2.912580in}}%
\pgfusepath{stroke}%
\end{pgfscope}%
\begin{pgfscope}%
\pgfpathrectangle{\pgfqpoint{2.053095in}{0.319877in}}{\pgfqpoint{0.129635in}{2.592703in}} %
\pgfusepath{clip}%
\pgfsetbuttcap%
\pgfsetmiterjoin%
\definecolor{currentfill}{rgb}{1.000000,1.000000,1.000000}%
\pgfsetfillcolor{currentfill}%
\pgfsetlinewidth{0.010037pt}%
\definecolor{currentstroke}{rgb}{1.000000,1.000000,1.000000}%
\pgfsetstrokecolor{currentstroke}%
\pgfsetdash{}{0pt}%
\pgfpathmoveto{\pgfqpoint{2.053095in}{0.319877in}}%
\pgfpathlineto{\pgfqpoint{2.053095in}{0.330005in}}%
\pgfpathlineto{\pgfqpoint{2.053095in}{2.902452in}}%
\pgfpathlineto{\pgfqpoint{2.053095in}{2.912580in}}%
\pgfpathlineto{\pgfqpoint{2.182730in}{2.912580in}}%
\pgfpathlineto{\pgfqpoint{2.182730in}{2.902452in}}%
\pgfpathlineto{\pgfqpoint{2.182730in}{0.330005in}}%
\pgfpathlineto{\pgfqpoint{2.182730in}{0.319877in}}%
\pgfpathclose%
\pgfusepath{stroke,fill}%
\end{pgfscope}%
\begin{pgfscope}%
\pgfsys@transformshift{2.050000in}{0.320408in}%
\pgftext[left,bottom]{\pgfimage[interpolate=true,width=0.130000in,height=2.590000in]{PerrNN_vs_dq_Ti_1000K-img1.png}}%
\end{pgfscope}%
\begin{pgfscope}%
\pgfsetbuttcap%
\pgfsetroundjoin%
\definecolor{currentfill}{rgb}{0.000000,0.000000,0.000000}%
\pgfsetfillcolor{currentfill}%
\pgfsetlinewidth{0.803000pt}%
\definecolor{currentstroke}{rgb}{0.000000,0.000000,0.000000}%
\pgfsetstrokecolor{currentstroke}%
\pgfsetdash{}{0pt}%
\pgfsys@defobject{currentmarker}{\pgfqpoint{0.000000in}{0.000000in}}{\pgfqpoint{0.048611in}{0.000000in}}{%
\pgfpathmoveto{\pgfqpoint{0.000000in}{0.000000in}}%
\pgfpathlineto{\pgfqpoint{0.048611in}{0.000000in}}%
\pgfusepath{stroke,fill}%
}%
\begin{pgfscope}%
\pgfsys@transformshift{2.182730in}{0.319877in}%
\pgfsys@useobject{currentmarker}{}%
\end{pgfscope}%
\end{pgfscope}%
\begin{pgfscope}%
\pgftext[x=2.279953in,y=0.272050in,left,base]{\rmfamily\fontsize{10.000000}{12.000000}\selectfont \(\displaystyle 0\)}%
\end{pgfscope}%
\begin{pgfscope}%
\pgfsetbuttcap%
\pgfsetroundjoin%
\definecolor{currentfill}{rgb}{0.000000,0.000000,0.000000}%
\pgfsetfillcolor{currentfill}%
\pgfsetlinewidth{0.803000pt}%
\definecolor{currentstroke}{rgb}{0.000000,0.000000,0.000000}%
\pgfsetstrokecolor{currentstroke}%
\pgfsetdash{}{0pt}%
\pgfsys@defobject{currentmarker}{\pgfqpoint{0.000000in}{0.000000in}}{\pgfqpoint{0.048611in}{0.000000in}}{%
\pgfpathmoveto{\pgfqpoint{0.000000in}{0.000000in}}%
\pgfpathlineto{\pgfqpoint{0.048611in}{0.000000in}}%
\pgfusepath{stroke,fill}%
}%
\begin{pgfscope}%
\pgfsys@transformshift{2.182730in}{0.607955in}%
\pgfsys@useobject{currentmarker}{}%
\end{pgfscope}%
\end{pgfscope}%
\begin{pgfscope}%
\pgftext[x=2.279953in,y=0.560128in,left,base]{\rmfamily\fontsize{10.000000}{12.000000}\selectfont \(\displaystyle 2\)}%
\end{pgfscope}%
\begin{pgfscope}%
\pgfsetbuttcap%
\pgfsetroundjoin%
\definecolor{currentfill}{rgb}{0.000000,0.000000,0.000000}%
\pgfsetfillcolor{currentfill}%
\pgfsetlinewidth{0.803000pt}%
\definecolor{currentstroke}{rgb}{0.000000,0.000000,0.000000}%
\pgfsetstrokecolor{currentstroke}%
\pgfsetdash{}{0pt}%
\pgfsys@defobject{currentmarker}{\pgfqpoint{0.000000in}{0.000000in}}{\pgfqpoint{0.048611in}{0.000000in}}{%
\pgfpathmoveto{\pgfqpoint{0.000000in}{0.000000in}}%
\pgfpathlineto{\pgfqpoint{0.048611in}{0.000000in}}%
\pgfusepath{stroke,fill}%
}%
\begin{pgfscope}%
\pgfsys@transformshift{2.182730in}{0.896034in}%
\pgfsys@useobject{currentmarker}{}%
\end{pgfscope}%
\end{pgfscope}%
\begin{pgfscope}%
\pgftext[x=2.279953in,y=0.848206in,left,base]{\rmfamily\fontsize{10.000000}{12.000000}\selectfont \(\displaystyle 4\)}%
\end{pgfscope}%
\begin{pgfscope}%
\pgfsetbuttcap%
\pgfsetroundjoin%
\definecolor{currentfill}{rgb}{0.000000,0.000000,0.000000}%
\pgfsetfillcolor{currentfill}%
\pgfsetlinewidth{0.803000pt}%
\definecolor{currentstroke}{rgb}{0.000000,0.000000,0.000000}%
\pgfsetstrokecolor{currentstroke}%
\pgfsetdash{}{0pt}%
\pgfsys@defobject{currentmarker}{\pgfqpoint{0.000000in}{0.000000in}}{\pgfqpoint{0.048611in}{0.000000in}}{%
\pgfpathmoveto{\pgfqpoint{0.000000in}{0.000000in}}%
\pgfpathlineto{\pgfqpoint{0.048611in}{0.000000in}}%
\pgfusepath{stroke,fill}%
}%
\begin{pgfscope}%
\pgfsys@transformshift{2.182730in}{1.184112in}%
\pgfsys@useobject{currentmarker}{}%
\end{pgfscope}%
\end{pgfscope}%
\begin{pgfscope}%
\pgftext[x=2.279953in,y=1.136284in,left,base]{\rmfamily\fontsize{10.000000}{12.000000}\selectfont \(\displaystyle 6\)}%
\end{pgfscope}%
\begin{pgfscope}%
\pgfsetbuttcap%
\pgfsetroundjoin%
\definecolor{currentfill}{rgb}{0.000000,0.000000,0.000000}%
\pgfsetfillcolor{currentfill}%
\pgfsetlinewidth{0.803000pt}%
\definecolor{currentstroke}{rgb}{0.000000,0.000000,0.000000}%
\pgfsetstrokecolor{currentstroke}%
\pgfsetdash{}{0pt}%
\pgfsys@defobject{currentmarker}{\pgfqpoint{0.000000in}{0.000000in}}{\pgfqpoint{0.048611in}{0.000000in}}{%
\pgfpathmoveto{\pgfqpoint{0.000000in}{0.000000in}}%
\pgfpathlineto{\pgfqpoint{0.048611in}{0.000000in}}%
\pgfusepath{stroke,fill}%
}%
\begin{pgfscope}%
\pgfsys@transformshift{2.182730in}{1.472190in}%
\pgfsys@useobject{currentmarker}{}%
\end{pgfscope}%
\end{pgfscope}%
\begin{pgfscope}%
\pgftext[x=2.279953in,y=1.424362in,left,base]{\rmfamily\fontsize{10.000000}{12.000000}\selectfont \(\displaystyle 8\)}%
\end{pgfscope}%
\begin{pgfscope}%
\pgfsetbuttcap%
\pgfsetroundjoin%
\definecolor{currentfill}{rgb}{0.000000,0.000000,0.000000}%
\pgfsetfillcolor{currentfill}%
\pgfsetlinewidth{0.803000pt}%
\definecolor{currentstroke}{rgb}{0.000000,0.000000,0.000000}%
\pgfsetstrokecolor{currentstroke}%
\pgfsetdash{}{0pt}%
\pgfsys@defobject{currentmarker}{\pgfqpoint{0.000000in}{0.000000in}}{\pgfqpoint{0.048611in}{0.000000in}}{%
\pgfpathmoveto{\pgfqpoint{0.000000in}{0.000000in}}%
\pgfpathlineto{\pgfqpoint{0.048611in}{0.000000in}}%
\pgfusepath{stroke,fill}%
}%
\begin{pgfscope}%
\pgfsys@transformshift{2.182730in}{1.760268in}%
\pgfsys@useobject{currentmarker}{}%
\end{pgfscope}%
\end{pgfscope}%
\begin{pgfscope}%
\pgftext[x=2.279953in,y=1.712440in,left,base]{\rmfamily\fontsize{10.000000}{12.000000}\selectfont \(\displaystyle 10\)}%
\end{pgfscope}%
\begin{pgfscope}%
\pgfsetbuttcap%
\pgfsetroundjoin%
\definecolor{currentfill}{rgb}{0.000000,0.000000,0.000000}%
\pgfsetfillcolor{currentfill}%
\pgfsetlinewidth{0.803000pt}%
\definecolor{currentstroke}{rgb}{0.000000,0.000000,0.000000}%
\pgfsetstrokecolor{currentstroke}%
\pgfsetdash{}{0pt}%
\pgfsys@defobject{currentmarker}{\pgfqpoint{0.000000in}{0.000000in}}{\pgfqpoint{0.048611in}{0.000000in}}{%
\pgfpathmoveto{\pgfqpoint{0.000000in}{0.000000in}}%
\pgfpathlineto{\pgfqpoint{0.048611in}{0.000000in}}%
\pgfusepath{stroke,fill}%
}%
\begin{pgfscope}%
\pgfsys@transformshift{2.182730in}{2.048346in}%
\pgfsys@useobject{currentmarker}{}%
\end{pgfscope}%
\end{pgfscope}%
\begin{pgfscope}%
\pgftext[x=2.279953in,y=2.000518in,left,base]{\rmfamily\fontsize{10.000000}{12.000000}\selectfont \(\displaystyle 12\)}%
\end{pgfscope}%
\begin{pgfscope}%
\pgfsetbuttcap%
\pgfsetroundjoin%
\definecolor{currentfill}{rgb}{0.000000,0.000000,0.000000}%
\pgfsetfillcolor{currentfill}%
\pgfsetlinewidth{0.803000pt}%
\definecolor{currentstroke}{rgb}{0.000000,0.000000,0.000000}%
\pgfsetstrokecolor{currentstroke}%
\pgfsetdash{}{0pt}%
\pgfsys@defobject{currentmarker}{\pgfqpoint{0.000000in}{0.000000in}}{\pgfqpoint{0.048611in}{0.000000in}}{%
\pgfpathmoveto{\pgfqpoint{0.000000in}{0.000000in}}%
\pgfpathlineto{\pgfqpoint{0.048611in}{0.000000in}}%
\pgfusepath{stroke,fill}%
}%
\begin{pgfscope}%
\pgfsys@transformshift{2.182730in}{2.336424in}%
\pgfsys@useobject{currentmarker}{}%
\end{pgfscope}%
\end{pgfscope}%
\begin{pgfscope}%
\pgftext[x=2.279953in,y=2.288596in,left,base]{\rmfamily\fontsize{10.000000}{12.000000}\selectfont \(\displaystyle 14\)}%
\end{pgfscope}%
\begin{pgfscope}%
\pgfsetbuttcap%
\pgfsetroundjoin%
\definecolor{currentfill}{rgb}{0.000000,0.000000,0.000000}%
\pgfsetfillcolor{currentfill}%
\pgfsetlinewidth{0.803000pt}%
\definecolor{currentstroke}{rgb}{0.000000,0.000000,0.000000}%
\pgfsetstrokecolor{currentstroke}%
\pgfsetdash{}{0pt}%
\pgfsys@defobject{currentmarker}{\pgfqpoint{0.000000in}{0.000000in}}{\pgfqpoint{0.048611in}{0.000000in}}{%
\pgfpathmoveto{\pgfqpoint{0.000000in}{0.000000in}}%
\pgfpathlineto{\pgfqpoint{0.048611in}{0.000000in}}%
\pgfusepath{stroke,fill}%
}%
\begin{pgfscope}%
\pgfsys@transformshift{2.182730in}{2.624502in}%
\pgfsys@useobject{currentmarker}{}%
\end{pgfscope}%
\end{pgfscope}%
\begin{pgfscope}%
\pgftext[x=2.279953in,y=2.576674in,left,base]{\rmfamily\fontsize{10.000000}{12.000000}\selectfont \(\displaystyle 16\)}%
\end{pgfscope}%
\begin{pgfscope}%
\pgfsetbuttcap%
\pgfsetroundjoin%
\definecolor{currentfill}{rgb}{0.000000,0.000000,0.000000}%
\pgfsetfillcolor{currentfill}%
\pgfsetlinewidth{0.803000pt}%
\definecolor{currentstroke}{rgb}{0.000000,0.000000,0.000000}%
\pgfsetstrokecolor{currentstroke}%
\pgfsetdash{}{0pt}%
\pgfsys@defobject{currentmarker}{\pgfqpoint{0.000000in}{0.000000in}}{\pgfqpoint{0.048611in}{0.000000in}}{%
\pgfpathmoveto{\pgfqpoint{0.000000in}{0.000000in}}%
\pgfpathlineto{\pgfqpoint{0.048611in}{0.000000in}}%
\pgfusepath{stroke,fill}%
}%
\begin{pgfscope}%
\pgfsys@transformshift{2.182730in}{2.912580in}%
\pgfsys@useobject{currentmarker}{}%
\end{pgfscope}%
\end{pgfscope}%
\begin{pgfscope}%
\pgftext[x=2.279953in,y=2.864752in,left,base]{\rmfamily\fontsize{10.000000}{12.000000}\selectfont \(\displaystyle 18\)}%
\end{pgfscope}%
\begin{pgfscope}%
\pgfsetbuttcap%
\pgfsetmiterjoin%
\pgfsetlinewidth{0.803000pt}%
\definecolor{currentstroke}{rgb}{0.000000,0.000000,0.000000}%
\pgfsetstrokecolor{currentstroke}%
\pgfsetdash{}{0pt}%
\pgfpathmoveto{\pgfqpoint{2.053095in}{0.319877in}}%
\pgfpathlineto{\pgfqpoint{2.053095in}{0.330005in}}%
\pgfpathlineto{\pgfqpoint{2.053095in}{2.902452in}}%
\pgfpathlineto{\pgfqpoint{2.053095in}{2.912580in}}%
\pgfpathlineto{\pgfqpoint{2.182730in}{2.912580in}}%
\pgfpathlineto{\pgfqpoint{2.182730in}{2.902452in}}%
\pgfpathlineto{\pgfqpoint{2.182730in}{0.330005in}}%
\pgfpathlineto{\pgfqpoint{2.182730in}{0.319877in}}%
\pgfpathclose%
\pgfusepath{stroke}%
\end{pgfscope}%
\end{pgfpicture}%
\makeatother%
\endgroup%

    \vspace*{-0.4cm}
	\caption{1000 K. Bin size $0.020e$}
	\end{subfigure}
\caption{Change in nearest neighbours' dipoles of Ti vs change in Ti charge}
\label{on_site_PerrNN_vs_dq}
\end{figure}


\bibliographystyle{unsrt}
\bibliography{/home/vn713/Documents/Mendeley_Desktop/Bibtex_files/library}
%\bibliography{/home/vn713/Documents/Mendeley_Desktop/Bibtex_files/My_PhD_Related,/home/vn713/Documents/Mendeley_Desktop/Bibtex_files/My_MSc_Related}
\end{document}

