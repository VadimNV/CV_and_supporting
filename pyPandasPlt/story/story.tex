\documentclass[11pt, a4paper]{report}
\renewcommand{\baselinestretch}{1.5}

\usepackage[titletoc]{appendix}
\usepackage{amsmath, amssymb}
\usepackage[normalem]{ulem} % either use this (simple) or
\usepackage{xcolor}
\usepackage{arydshln}
\usepackage[makeroom]{cancel}
\usepackage{float}
\usepackage{lipsum}
\usepackage{graphicx}
\usepackage[english]{babel}
\usepackage[top=2cm, bottom=2.5cm]{geometry}
\usepackage{booktabs} %to make tables look nice
\renewcommand{\arraystretch}{1.2} % to make vertical spacing wider in the tables
\usepackage{multirow} %to make \multirow possible (otherwise only \multicolumn)
\usepackage{cancel}   %to strike out part of the equation setting it to 0 or 1, etc.
\usepackage{mathrsfs} %for curly letters use e.g. \mathscr{E}

%\DeclareUnicodeCharacter{2212}{-} % when bibtex has things like {$_{3\textendash\ x}$}

\listfiles %to see LaTeX version in  .log file

\usepackage{listings} %to incldue Mathematica code
%\usepackage{physics} %includes bracket notation
\lstloadlanguages{Mathematica}

\usepackage{geometry} %to use gnuplot figure as figures
\usepackage{color}    %to use gnuplot output as figures
\usepackage{anyfontsize} %to specify font of a single word (e.g. in gnuplot)
\usepackage{pgf} %allows Python->Tex figures look native. http://bkanuka.com/articles/native-latex-plots/
\newcommand\inputpgf[2]{{
\let\pgfimageWithoutPath\pgfimage
\renewcommand{\pgfimage}[2][]{\pgfimageWithoutPath[##1]{#1/##2}}
\input{#1/#2}
}} %wow, amazing solution to input pgf files using a path at https://tex.stackexchange.com/questions/127667/matplotlib-pgf-images-in-subdirectory

\usepackage{caption}
\usepackage{subcaption} %to have side-by-side subfigures
\usepackage{wrapfig} %to wrap text around figures
\newcommand{\pathnPlan}{/home/vadim/Documents/nplan_olympics/story/}
%%%

\newcommand{\pathFigs}{figs} %path to SizeEffect figures

\usepackage[sort&compress]{natbib} % to have [1-3] instead of [1,2,3]

\makeatletter
\newcommand{\rmnum}[1]{\romannumeral #1}
\newcommand{\Rmnum}[1]{\expandafter\@slowromancap\romannumeral #1@}
\makeatother %to have Roman numerals

\usepackage{lscape}

\begin{document}
\pagestyle{empty}

\begin{landscape}
\centerline{\LARGE\textbf{Competitiveness in the Olympics}}
\end{landscape}

\begin{landscape}
\begin{figure}[!htbp]
	\centering
	\inputpgf{\pathnPlan \pathFigs}{HW_scatter.pgf}
	\caption{Mean athletes' Height and Weight for each Olympic sports (only some outliers are marked for clarity).}
\end{figure}
\begin{itemize}
\item For some sports, having the right body type is crucial
\item Weightlifters are $\sim30$ kg heavier than Rhythmic Gymnasts while of the same height
\item At the same time, Volleyball players are as heavy as Weightlifters, but are $\sim20$ cm taller
\end{itemize}
\end{landscape}

\begin{landscape}
\begin{figure}[!htbp]
	\centering
	\inputpgf{\pathnPlan \pathFigs}{GymnasticsYearly.pgf}
	\caption{Mean (black) and gold medalist's (orange) weight and height in Gymnastics. Dashed line is a linear fit to mean values. Missing points are due to missing data.}
\end{figure}
\begin{itemize}
\item \underline{Example: Gymnastics}
\item Notice \textendash\ from $1960$ onward gold medalists are lighter and shorter than the average competitor
\item Participating athletes gradually became shorter and lighter over the decades, arguably due to the competition
\item Love gymnastics but haven't got the right body type?.. tough luck
\end{itemize}
\end{landscape}

\begin{landscape}
\begin{figure}[!htbp]
	\centering
	\inputpgf{\pathnPlan \pathFigs}{mapsSeason.pgf}
	\caption{Proportion of the total Gold medals ever won by any country, during the Summer (top) and Winter (bottom) Olympics. Proportion of a given country's victories is proportional to the bubble area, with a lower cap at $2\%$.}
\end{figure}
\begin{itemize}
\item Country's climate clearly plays a role in preparing the athletes, providing a competitive advantage or disadvantage
\item While most countries have won at least one Gold medal during the Summer Olympics, Winter Olympics are dominated by northern countries
\item Notice \textendash\ countries from the African and South American continents have never won a Winter Olympics Gold. No snow, no way to practice
\end{itemize}
\end{landscape}

\begin{landscape}
\centerline{\LARGE\textbf{The End}}
\end{landscape}

\end{document}\n